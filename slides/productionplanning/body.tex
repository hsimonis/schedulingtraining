\begin{frame}
\frametitle{Key Points}
\begin{itemize}
\item Case study from industry
\item Production planning and detailed scheduling
\item Based on project with medical devices company in Cork
\begin{itemize}
\item Real problem
\item Realistic data
\end{itemize}
\item Solved in two stages
\begin{itemize}
\item Production planning based on run-out days and safety stock levels
\item Scheduling using our generic scheduling tool
\end{itemize}
\end{itemize}
\end{frame}

\begin{frame}
\frametitle{Problem}
{\tiny
This is a case study from industry, combining production planning and detailed scheduling. It is based on a project that some of my colleagues were working on for a local medical devices company. 

The overall problem is to decide which products to make in which quantities over the planning horizon, so that we have enough stock to satisfy any customer demand, and make sure that we have some safety margin if the demand suddenly increases. At the same time we do not want to create inventory in products that we will not sell in the near future, as this increases our inventory carrying cost.

The company uses two main concepts for production planning:
The run-out days for each product state how long the current stock will last, given a projected customer demand profile. We try to achieve the same run-out days value for all products, this works well for fast and slow moving products.

The safety stock values says how much stock we should have for each product. This gives us more control over the stock levels, this works better if the demand cannot be predicted as accurately as we would like, but it is more difficult to compare the stock levels for  different products.

The production planning part of the application decides how much to produce for which product, but this is based on an estimate of the production capacity for the planning period. We use the detailed scheduling part of the application to validate the plan generated, and make sure that we can really produce the required capacities in the given planning period.

}
\end{frame}



\begin{frame}
\frametitle{Product List}
\includegraphics[width=\textwidth]{../productionplanning/images/products}
\end{frame}

\begin{frame}
\frametitle{Product List (Sorted by Daily Sales)}
\includegraphics[width=\textwidth]{../productionplanning/images/productssortedbysales}
\end{frame}

\begin{frame}
\frametitle{Product List (Sorted by Days Cover)}
\includegraphics[width=\textwidth]{../productionplanning/images/productssortedbydayscover}
\end{frame}

\begin{frame}
\frametitle{Product List (Sorted by Safety Alert)}
\includegraphics[width=\textwidth]{../productionplanning/images/productssortedbysafetyalert}
\end{frame}

\begin{frame}
\frametitle{Setup Matrix}
\includegraphics[width=\textwidth]{../productionplanning/images/setupmatrix}
\end{frame}

\begin{frame}
\frametitle{Running the Planning Solver}
\includegraphics[width=.4\textwidth]{../productionplanning/images/runningplanner}
\end{frame}

\begin{frame}
\frametitle{Planner Results}
\includegraphics[width=\textwidth]{../productionplanning/images/plannerresults}
\end{frame}

\begin{frame}
\frametitle{Product Level Chart for Product P3}
\includegraphics[width=\textwidth]{../productionplanning/images/productlevelchart}
\end{frame}

\begin{frame}
\frametitle{Product Level Chart for Product P35}
\includegraphics[width=\textwidth]{../productionplanning/images/productlevelp35}
\end{frame}

\begin{frame}
\frametitle{Scheduled Production Runs}
\includegraphics[width=.45\textwidth]{../productionplanning/images/productionruns}
\end{frame}

\begin{frame}
\frametitle{Production Level Chart for Product P3}
\includegraphics[width=\textwidth]{../productionplanning/images/productionlevelchart}
\end{frame}

\begin{frame}
\frametitle{Production Level Chart for Product P35}
\includegraphics[width=\textwidth]{../productionplanning/images/productionlevelp35}
\end{frame}

\begin{frame}
\frametitle{Detailed Schedule}
\includegraphics[width=\textwidth]{../productionplanning/images/detailedschedule}
\end{frame}

\begin{frame}
\frametitle{Showing Alternative Machines in Gantt Chart}
\includegraphics[width=\textwidth]{../productionplanning/images/alternativemachines}
\end{frame}



\begin{frame}
\frametitle{Summary}
\begin{itemize}
\item We demonstrated the use of our scheduling tool inside a production planning problem from industry
\item Production planning decides which products to make in which quantity
\begin{itemize}
\item Balance stock levels against projected demand
\item Allow for product specific safety stock levels
\end{itemize}
\item Uses estimate of production capacity over planning horizon
\item Use detailed scheduling to validate plan
\end{itemize}
\end{frame}

