\begin{frame}
\frametitle{Key Points}
\begin{itemize}
\item This section describes the scheduling tool
\begin{itemize}
\item This is a \emph{preview} of the current state, not released yet!
\end{itemize}
\item How to load/create data
\begin{itemize}
\item From files
\item By instance generator
\item From benchmark problems
\end{itemize}
\item How to run the solvers
\begin{itemize}
\item Which solvers are supported
\item What to expect in terms of performance
\end{itemize}
\item Experiments to try
\begin{itemize}
\item Limited time
\item Possible "test before invest" continuation
\end{itemize}

\end{itemize}
\end{frame}

\section{The Scheduling Tool}

\begin{frame}
\frametitle{The Scheduling Tool}
\begin{itemize}
\item We create the tool as basis for experiments
\item To test ideas and solvers
\item As a teaching tool
\item Slightly higher standard than usual academic prototypes
\begin{itemize}
\item This is a \emph{preview}, not released yet
\end{itemize}
\item Not a commercial tool
\begin{itemize}
\item But can use commercial solvers
\item Also open-source solvers
\end{itemize}
\item Written in Java, JavaFX
\item Can also be used as a back-end scheduling server
\item Uses our Java application framework generator
\item Will become available in early 2025
\end{itemize}
\end{frame}



\section{Under the hood}

\begin{frame}
\frametitle{Back-end solvers}
\begin{itemize}
\item Provide both open-source and commercial solver interfaces
\item Allow experimentation without having to buy commercial tools straightaway
\item Gives a level playing field to compare solvers and models
\item Provides out-of-the-box, generic performance
\end{itemize}
\end{frame}



\subsection{Google CPSat}

\begin{frame}
\frametitle{Google OR-Tools CPSat Solver}
\begin{columns}
\begin{column}{0.5\textwidth}
\begin{itemize}
\item Open-Source tool provided by Google
\item Available at \url{https://developers.google.com/optimization/cp/cp_solver}
\item Probably best open-source CP solver for scheduling
\item This solver is packaged with scheduler
\end{itemize}
\end{column}
\begin{column}{0.5\textwidth}
\includegraphics[width=\textwidth]{../04-experiments/images/cpsat-example}

{\tiny (from OR-Tools website)}
\end{column}
\end{columns}
\end{frame}

\subsection{IBM CPOptimizer}

\begin{frame}
\frametitle{CP Optimizer from IBM}
\begin{columns}
\begin{column}{0.5\textwidth}
\begin{itemize}
\item Commercial tool of IBM
\item \url{https://www.ibm.com/products/ilog-cplex-optimization-studio/cplex-cp-optimizer}
\item Part of optimization suite with Cplex, OPL
\item We do \textbf{not} provide this solver, we allow to interface with it
\item Academic licenses available
\item Well-known for capabilities for scheduling

\end{itemize}
\end{column}
\begin{column}{0.5\textwidth}
\includegraphics[width=.7\textwidth]{../04-experiments/images/cpoptimizer-example}

{\tiny (from CPOptimizer website)}
\end{column}
\end{columns}
\end{frame}

\subsection{MiniZinc}

\begin{frame}
\frametitle{MiniZinc from Monash University}
\begin{columns}
\begin{column}{0.5\textwidth}
\begin{itemize}
\item Modelling language and backend tools from Monash University in Melbourne, Australia
\item Available from \url{https://www.minizinc.org/}
\item Widely used for teaching
\item Allows different backend solver to run from same model
\item Generic CP tool, not optimized for scheduling
\item Requires separate installation, open-source
\end{itemize}
\end{column}
\begin{column}{0.5\textwidth}
\includegraphics[width=\textwidth]{../04-experiments/images/minizinc-example}

{\tiny (from MiniZinc Website)}
\end{column}
\end{columns}
\end{frame}

\subsection{Which solver is better?}

\begin{frame}
\frametitle{Which Solver is Better?}
\begin{itemize}
\item We present results on a few benchmark types
\item Fair comparison between solvers
\begin{itemize}
\item Same hardware, Windows 11 laptop
\item CPU i7-10875H @ 2.3GHz, 64GB, four cores
\item Same timeout (600 s)
\end{itemize}
\item Not a fair comparison to state-of-the-art
\begin{itemize}
\item Uses out-of-the-box model
\item Significant improvements possible
\item More specific models
\item Parameter tuning
\item Unlimited runtime
\end{itemize}

\end{itemize}
\end{frame}

\begin{frame}
\frametitle{Taillard Job-Shop Benchmarks}
\scalebox{0.6}{
\begin{tabular}{lr*{10}{r}} \toprule
& &\multicolumn{4}{c}{All Instances} & \multicolumn{2}{c}{Optimal Only} & \multicolumn{4}{c}{Non Optimal Only}\\ 
& &\multicolumn{4}{c}{Optimal (\% of All Instances)} & \multicolumn{2}{c}{Time (\% of VB)} & \multicolumn{2}{c}{Cost (\% of VB)} & \multicolumn{2}{c}{Bound (\% of VB)}\\ 
Group & Nr & Both & CPO & CPSat & None & CPO & CPSat & CPO & CPSat & CPO & CPSat\\ \midrule
15/15 & 10 & 90.00 &  0.00 &  0.00 & 10.00 & 105.19 & 141.18 & 100.00 & 100.00 & 97.17 & 100.00 \\ 
20/15 & 10 & 20.00 &  0.00 &  0.00 & 80.00 & 267.27 & 263.20 & 100.99 & 100.05 & 98.50 & 99.93 \\ 
20/20 & 10 &  0.00 &  0.00 &  0.00 & 100.00 & n/a & n/a & 100.74 & 100.06 & 97.96 & 100.00 \\ 
30/15 & 10 & 10.00 &  0.00 & 10.00 & 80.00 & 174.32 & 100.00 & 100.18 & 100.49 & 99.87 & 100.00 \\ 
30/20 & 10 &  0.00 &  0.00 &  0.00 & 100.00 & n/a & n/a & 100.30 & 101.30 & 99.40 & 100.00 \\ 
50/15 & 10 & 100.00 &  0.00 &  0.00 &  0.00 & 100.00 & 685.09 & n/a & n/a & n/a & n/a \\ 
50/20 & 10 & 10.00 & 60.00 &  0.00 & 30.00 & 100.00 & 381.38 & 100.00 & 101.60 & 100.00 & 100.00 \\ 
100/20 & 10 & 10.00 & 90.00 &  0.00 &  0.00 & 100.00 & 416.13 & 100.00 & 101.73 & 100.00 & 66.81 \\ 
\bottomrule
\end{tabular}
}

\begin{itemize}
\item Significant number of problems solved to optimality in 600s
\item In terms of quality, solvers are quite similar
\item CPO wins in terms of solution times for larger instances
\end{itemize}

\end{frame}

\begin{frame}
\frametitle{Results for Hybrid Flexible Flow-Shop}
\scalebox{0.6}{
\begin{tabular}{lr*{10}{r}} \toprule
& &\multicolumn{4}{c}{All Instances} & \multicolumn{2}{c}{Optimal Only} & \multicolumn{4}{c}{Non Optimal Only}\\ 
& &\multicolumn{4}{c}{Optimal (\% of All Instances)} & \multicolumn{2}{c}{Time (\% of VB)} & \multicolumn{2}{c}{Cost (\% of VB)} & \multicolumn{2}{c}{Bound (\% of VB)}\\ 
Group & Nr & Both & CPO & CPSat & None & CPO & CPSat & CPO & CPSat & CPO & CPSat\\ \midrule
20 & 25 & 76.00 &  0.00 & 20.00 &  4.00 & 100.00 & 580.71 & 100.00 & 100.00 & 96.52 & 100.00 \\ 
25 & 25 & 80.00 &  0.00 &  8.00 & 12.00 & 101.65 & 238.02 & 100.00 & 100.37 & 97.67 & 100.00 \\ 
30 & 25 & 60.00 &  0.00 &  4.00 & 36.00 & 100.35 & 264.69 & 100.18 & 101.05 & 100.00 & 100.00 \\ 
40 & 25 &  4.00 & 16.00 &  0.00 & 80.00 & 100.00 & 2554.03 & 100.00 & 104.68 & 100.00 & 100.00 \\ 
50 & 25 &  0.00 &  4.00 &  0.00 & 96.00 & n/a & n/a & 100.00 & 107.87 & 100.00 & 100.00 \\ 
100 & 25 &  0.00 &  0.00 &  0.00 & 100.00 & n/a & n/a & 100.00 & 120.43 & 100.00 & 100.00 \\ 
200 & 25 &  0.00 &  0.00 &  0.00 & 100.00 & n/a & n/a & 100.00 & 188.60 & 100.00 & 100.00 \\ 
300 & 24 &  0.00 &  0.00 &  0.00 & 100.00 & n/a & n/a & 100.00 & 263.22 & 100.00 & 100.00 \\ 
400 & 25 &  0.00 &  0.00 &  0.00 & 100.00 & n/a & n/a & 100.00 & 246.34 & 100.00 & 100.00 \\ 
\bottomrule
\end{tabular}
}
\begin{itemize}
\item Only smaller/medium instances solved to optimality
\item For those problems, both solvers perform well
\item CPO significantly better on large instances
\end{itemize}
\end{frame}

\begin{frame}
\frametitle{General Recommendations}
\begin{itemize}
\item If you already have access to CPO, use it!
\item For new problem types, do an evaluation with CPSat first
\item Out of the box, CPO performs more consistently
\item May be easier to extend CPSat with your own research
\item Use multiple cores and memory to your advantage
\end{itemize}
\end{frame}





\section{Input Data}

\begin{frame}
\frametitle{Input Data}
\begin{itemize}
\item We have defined a specific JSON data format to describe scheduling problems
\item This is different from the native/XML data format of the application (do not use)
\item Load with menu \texttt{File - Load DataFile...}
\item Save with menu \texttt{File - Save DataFile...}
\item The format is described in a document
\end{itemize}
\end{frame}

\begin{frame}
\frametitle{Base Data}
\begin{columns}
\begin{column}{0.5\textwidth}
\begin{itemize}
\item Description of 
\begin{itemize}
\item Product
\item Process
\item DisjunctiveResource
\item CumulativeNeed
\item ProcessStep
\item ResourceNeed
\item CumulativeProfile
\item Problem
\item CumulativeResource
\item ProcessSequence
\end{itemize}
\end{itemize}
\end{column}
\begin{column}{0.4\textwidth}
\includegraphics[width=\textwidth]{../04-experiments/images/basedata}
\end{column}
\end{columns}
\end{frame}

\begin{frame}
\frametitle{Schedule Input Data}
\begin{columns}
\begin{column}{0.5\textwidth}
\begin{itemize}
\item Description of
\begin{itemize}
\item Downtime
\item Task (x2)
\item Job
\item Order
\item WiP
\end{itemize}
\end{itemize}
\end{column}
\begin{column}{0.5\textwidth}
\includegraphics[width=\textwidth]{../04-experiments/images/schedule}
\end{column}
\end{columns}
\end{frame}



\section{Result Output}

\begin{frame}
\frametitle{Result Data}
\begin{itemize}
\item We use the same JSON format to describe the results of the schedule
\item Added field types for SolverRun, Solution, assigned Jobs and Tasks
\end{itemize}
\end{frame}

\begin{frame}
\frametitle{Sample Results}
\begin{columns}
\begin{column}{0.6\textwidth}
\begin{itemize}
\item Description of
\begin{itemize}
\item Solution
\item SolverRun
\item Job Assignment
\item Task Assignment
\end{itemize}

\end{itemize}
\end{column}
\begin{column}{0.4\textwidth}
\includegraphics[width=.7\textwidth]{../04-experiments/images/samplerresult}
\end{column}
\end{columns}
\end{frame}



\section{Instance Generator}

\begin{frame}
\frametitle{Instance Generator}
\begin{itemize}
\item Application allows to generate different types of test problems
\item Different types of resource models
\item Different numbers of orders, resources, WiP, downtime
\item Useful to generate more life-like examples combining different constraint types
\end{itemize}
\end{frame}


\begin{frame}
\frametitle{Instance Generator Dialog}
\begin{columns}
\begin{column}{0.5\textwidth}
\begin{itemize}
\item Resource Model
\begin{itemize}
\item Select a resource model defining the overall structure of problem
\end{itemize}
\item Nr Disjunctive Resources
\begin{itemize}
\item Describe how many disjunctive resources are generated
\end{itemize}
\item Resource Probability
\begin{itemize}
\item The probability that a resource is compatible with a task
\item Only for some resource models
\end{itemize}
\end{itemize}
\end{column}
\begin{column}{0.5\textwidth}
\includegraphics[width=\textwidth]{../04-experiments/images/instance-generator}
\end{column}
\end{columns}
\end{frame}

\begin{frame}
\frametitle{Resource Models}
\begin{itemize}
\item Flow-Shop
\begin{itemize}
\item Multiple stages, all jobs use machines in same order
\end{itemize}

\item Job-Shop
\begin{itemize}
\item Multiple stages, jobs use machines in different order
\end{itemize}

\item Open-Shop
\begin{itemize}
\item Multiple stages, no predefined order of machines
\end{itemize}

\item Hybrid Flow-Shop (default)
\item Hybrid Job-Shop
\item Hybrid Open-Shop
\begin{itemize}
\item Like x-shop, but with multiple machines per stage
\end{itemize}

\item Random
\begin{itemize}
\item Multiple stages, each stage using a random subset of machines
\end{itemize}

\item All
\begin{itemize}
\item Multiple stages, each stage allowing all machines
\end{itemize}

\end{itemize}
\end{frame}



\begin{frame}
\frametitle{Instance Generator - Products}
\begin{columns}
\begin{column}{0.5\textwidth}
\begin{itemize}
\item Nr Products
\begin{itemize}
\item Number of products to be generated
\item Products may be reused by multiple orders
\end{itemize}
\item Stages Range
\begin{itemize}
\item Range slider, sets lower and upper bound on number of stages
\end{itemize}

\end{itemize}
\end{column}
\begin{column}{0.5\textwidth}
\includegraphics[width=\textwidth]{../04-experiments/images/instance-generator-products}
\end{column}
\end{columns}
\end{frame}

\begin{frame}
\frametitle{Instance Generator - Duration}
\begin{columns}
\begin{column}{0.5\textwidth}
\begin{itemize}
\item Duration Model
\begin{itemize}
\item Different ways to link duration of processSteps
\end{itemize}
\item Duration Range
\begin{itemize}
\item Range slider to set lower and upper bounds on perUnit duration
\end{itemize}

\item Duration Fixed Factor
\begin{itemize}
\item How fixed and perUnit duration values are linked
\end{itemize}

\end{itemize}
\end{column}
\begin{column}{0.5\textwidth}
\includegraphics[width=\textwidth]{../04-experiments/images/instance-generator-duration}
\end{column}
\end{columns}
\end{frame}

\begin{frame}
\frametitle{Instance Generator - Cumulative}
\begin{columns}
\begin{column}{0.5\textwidth}
\begin{itemize}
\item Nr Cumulative Resources
\begin{itemize}
\item Number of cumulative resources generated
\end{itemize}

\item Cumul Demand Range
\begin{itemize}
\item Range slider to select lower and upper bound on cumulativeResourceNeed demands
\end{itemize}

\item Profile Pieces
\begin{itemize}
\item Number of segments of CumulativeProfile generated for each resource
\end{itemize}

\item Cumul Capacity Range
\begin{itemize}
\item Range slider to select lower and upper bounds on cumulative profile capacity values
\end{itemize}

\end{itemize}
\end{column}
\begin{column}{0.5\textwidth}
\includegraphics[width=\textwidth]{../04-experiments/images/instance-generator-cumulative}
\end{column}
\end{columns}
\end{frame}

\begin{frame}
\frametitle{Instance Generator - Orders}
\begin{columns}
\begin{column}{0.5\textwidth}
\begin{itemize}
\item Nr Orders
\begin{itemize}
\item Number of orders generated, each order is assigned a random product/process
\end{itemize}

\item Qty Range
\begin{itemize}
\item Range slider to select lower and upper bounds on quantity for each order
\end{itemize}

\end{itemize}
\end{column}
\begin{column}{0.5\textwidth}
\includegraphics[width=\textwidth]{../04-experiments/images/instance-generator-orders}
\end{column}
\end{columns}
\end{frame}

\begin{frame}
\frametitle{Instance Generator - WiP (Work in Progress)}
\begin{columns}
\begin{column}{0.5\textwidth}
\begin{itemize}
\item WiP Probability
\begin{itemize}
\item Probability of generating a WiP for a disjunctive resource
\end{itemize}

\item WiP Range
\begin{itemize}
\item Range slider to set lower and upper bound on WiP duration
\end{itemize}

\end{itemize}
\end{column}
\begin{column}{0.5\textwidth}
\includegraphics[width=\textwidth]{../04-experiments/images/instance-generator-wip}
\end{column}
\end{columns}
\end{frame}

\begin{frame}
\frametitle{Instance Generator - Downtime}
\begin{columns}
\begin{column}{0.5\textwidth}
\begin{itemize}
\item Downtime Probability
\begin{itemize}
\item Probability of generating a downtime for a disjunctive resource
\end{itemize}
\item Downtime Range
\begin{itemize}
\item Range slider to select lower and upper bounds on downtime duration
\end{itemize}

\end{itemize}
\end{column}
\begin{column}{0.5\textwidth}
\includegraphics[width=\textwidth]{../04-experiments/images/instance-generator-downtime}
\end{column}
\end{columns}
\end{frame}

\begin{frame}
\frametitle{Instance Generator - Other Parameters}
\begin{columns}
\begin{column}{0.5\textwidth}
\begin{itemize}
\item Earliest Due
\begin{itemize}
\item Smallest allowed value for a due date
\end{itemize}

\item Horizon Days
\begin{itemize}
\item What planning horizon to consider (in days)
\end{itemize}

\item Time Resolution
\begin{itemize}
\item In minutes, links internal and external time presentation
\end{itemize}

\item Random Seed
\begin{itemize}
\item Random seed to make reproducible random choices
\end{itemize}

\end{itemize}
\end{column}
\begin{column}{0.5\textwidth}
\includegraphics[width=\textwidth]{../04-experiments/images/instance-generator-parameters}
\end{column}
\end{columns}
\end{frame}




\section{Predefined Problem Sets}
\subsection{Taillard}

\begin{frame}
\frametitle{Taillard Scheduling Benchmarks}
\begin{itemize}
\item Three datasets of different sizes
\begin{itemize}
\item Job-shop
\item Flow-shop
\item Open-shop
\end{itemize}
\item Load with menu \texttt{File - Load DataFile... - Taillard - }
\item Larger instances need more solver time to reach good solutions (600 s)
\end{itemize}
\end{frame}


\subsection{SALBP}

\begin{frame}
\frametitle{Simple Assembly Line Balancing Problem (SALBP)}
\begin{itemize}
\item Will be discussed on more details as case study
\item Design an assembly line setup by solving a scheduling problem
\item Balance a set of operations across a number of stations of an assembly line
\item Precedence graph is not a chain, can be very complex
\item Specialized problem normally solved with specialized tools
\item Load with menu \texttt{File - Load SALBP Problem...}
\end{itemize}
\end{frame}


\subsection{Test Scheduling}

\begin{frame}
\frametitle{Test Scheduling Benchmark set from ABB}
\begin{itemize}
\item Will be discussed in more details as case study
\item Schedule a set of tests on a number of machines, minimizing total duration
\item Single stage tests, possibly large number of resources
\item Closely related to bin-packing
\item Load with menu \texttt{File - Load Test Scheduling Problem...}

\end{itemize}
\end{frame}

\section{Some Suggested Experiments}

\begin{frame}
\frametitle{Experiment 1}
\begin{itemize}
\item Start the application
\begin{itemize}
\item Our running example will be automatically generated
\end{itemize}
\item Look at the process diagram \texttt{Window-Product-Process Diagram}
\item Run the solver \texttt{Scenario - Run ScheduleJobs Solver}
\item Observe the results in Gantt Chart
\item Customize display
\item Look at Cumulative Resource Chart \texttt{Window-Solution-Cumulative Resource Chart}

\end{itemize}
\end{frame}

\begin{frame}
\frametitle{Experiment 2}
\begin{itemize}
\item Re-run solver disabling cumulative constraint
\item Observe result in Gantt chart
\item See impact on Cumulative Resource chart
\item Switch between solutions in charts
\end{itemize}
\end{frame}

\begin{frame}
\frametitle{Experiment 3}
\begin{itemize}
\item Check Gantt chart display for delayed tasks, enabling lateness display
\item Re-run solver, enforcing due-date constraints
\item What impact does this have on objective
\end{itemize}
\end{frame}

\begin{frame}
\frametitle{Experiment 4}
\begin{itemize}
\item Change objective to on-time delivery
\item Results are very different, why?
\item More explanations on this tomorrow
\end{itemize}
\end{frame}

\begin{frame}
\frametitle{Experiment 5}
\begin{itemize}
\item Load one of the other example types
\item For example, Taillard Job-shop 15x15
\item Understand process diagram
\item Run solver
\item Look at intermediate solutions found
\end{itemize}
\end{frame}










\section{Summary}

\begin{frame}
\frametitle{Summary}
\begin{itemize}
\item We presented an overview of our generic scheduling tool
\item Discussed available solvers, both commercial and open-source
\item Described the JSON data format for input and output
\item Gave an overview of the instance generator provided
\item Shows example problems included with tool
\item Suggested some experiments to run
\end{itemize}
\end{frame}

