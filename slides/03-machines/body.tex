\begin{frame}
\frametitle{Key Points}
\begin{itemize}
\item Introduce Different Types of Resources
\item Disjunctive Resources - One Task at a Time
\item Cumulative Resources - Demands and Capacity
\item Machine Choice - Use one of multiple machines
\item Calendars - Not working all the time
\end{itemize}
\end{frame}

\section{Disjunctive Resources}

\begin{frame}
\frametitle{Disjunctive Resource}
\begin{itemize}
\item A \emph{disjunctive resource} works on one task at a time
\item Each task runs uninterrupted from start to end
\item The machine may be \emph{idle} between tasks
\item The machine may be unused at start and end of schedule
\begin{itemize}
\item Some of this may be unreachable, there is not work that can be done in these periods
\item Problem of cold start, especially for flow-shop type problems
\end{itemize}
\item Active time is time between first and last use
\item Resource utilization compares productive time to active or available time
\end{itemize}
\end{frame}

\begin{frame}
\frametitle{Disjunctive Machines Examples}
\includegraphics[width=\textwidth]{../03-machines/images/disjunctive-flow-shop}

Flow-Shop example, some unreachable time on later resources in process, some idle time

\includegraphics[width=\textwidth]{../03-machines/images/disjunctive-job-shop}

Job-Shop example, a lot of idle time
\end{frame}





\subsection{Preemption}

\begin{frame}
\frametitle{Preemption}
\begin{itemize}
\item Normal constraint for disjunctive constraints is one task at a time
\item Once a task is started, it runs until it is finished
\item \emph{Preemption} allows to stop a task, run a different task, then resume the previous task to the end
\item Example: This is how Operating Systems run tasks inside a computer
\begin{itemize}
\item This works since cost of suspending a task is relatively low
\item Also needed as tasks continuously produce output which is expected
\end{itemize}
\item In manufacturing, preemption often is an exception in an emergency
\end{itemize}
\end{frame}

\begin{frame}
\frametitle{How to Deal with Preemption in Scheduling}
\begin{enumerate}
\item Handle this as manual intervention for critical situations
\item Dedicated preemptive scheduling constraints
\item Allow limited number of interruptions
\begin{itemize}
\item Split each task into multiple pieces of unknown length
\item Normally, schedule all parts together for total duration
\item For preemption, schedule other task after first/second part
\item All parts of task must add up to total duration 
\end{itemize}
\end{enumerate}
\end{frame}



\section{Cumulative Resources}

\begin{frame}
\frametitle{Cumulative Resources}
\begin{itemize}
\item A cumulative resource provides capacity over time, the sum of the demands at each timepoint cannot exceed the available capacity at that time
\item Resource demand by one task is considered constant from start to end
\begin{itemize}
\item Need to break task into smaller segments to model time variable demand
\end{itemize}
\item In itself a hard problem, so full propagation not possible
\begin{itemize}
\item Active research area since 1993, when the constraint was introduced in CHIP
\end{itemize}
\end{itemize}
\end{frame}



\subsection{Demand and Capacity}


\subsection{Variants}

\begin{frame}
\frametitle{Time Variable Resource Cost}
\begin{itemize}
\item
\end{itemize}
\end{frame}

\begin{frame}
\frametitle{Soft/Hard Limit, Overtime Cost}
\begin{itemize}
\item
\end{itemize}
\end{frame}

\begin{frame}
\frametitle{Lower Utilization Limit}
\begin{itemize}
\item
\end{itemize}
\end{frame}



\subsection{Manpower}

\subsection{Nested Skill Levels}

\subsection{Assigned Operators}

\subsection{Fractional Manpower Needs}

\section{Machine Choice}

\begin{frame}
\frametitle{Choosing which machine to use}
\begin{itemize}
\item Problem with Job-shop/Flow-shop: There is only one machine
\begin{itemize}
\item What happens if any of those machines stops working?
\item Do we stop production completely
\end{itemize}
\item Most plants have multiple machine for the same task
\item Three fundamental alternatives
\begin{itemize}
\item Multiple, identical machines
\item Multiple machines with different speeds
\item Preferences for specific machines, but viable alternatives exist
\end{itemize}
\item On the other hand, sometimes identical machines are treated as different
\begin{itemize}
\item Dedicated lines for major products, avoiding setup/cleaning times
\end{itemize}
\end{itemize}
\end{frame}


\subsection{Identical Machines}
\subsection{Machine Dependent Speed}
\subsection{Machine Preferences}

\section{Work in Progress and Planned Downtimes}

\section{Calendars}


\subsection{Factory Wide Calendars}
\subsection{Machine Specific Calendars}
\subsection{Changing Work Pattern}
\subsection{Varying Machine Speed}
\subsection{Task/Break Interaction}

\section{Summary}

\begin{frame}
\frametitle{Summary}
\begin{itemize}
\item Introduced different resource types
\item Identifying resources is a key element of defining scheduling problem
\item As simple as possible - as complex as required
\end{itemize}
\end{frame}

