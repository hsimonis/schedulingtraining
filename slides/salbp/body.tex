\begin{frame}
\frametitle{Key Points}
\begin{itemize}
\item There can be more than one formulation of a problem
\item Typically there is a direct model where all constraints are taken from problem description
\item There may be other representations of problem using other variables, domains, constraints
\item Needs mapping to original problem
\item Performance may vary a lot
\end{itemize}
\end{frame}

\begin{frame}
\frametitle{Problem Description}
{\scriptsize 
A car is made on an assembly line, where the $n$ pieces of the car are added at different
stations. The car moves along the assembly line, running through the stations
in sequence, where it stays for a fixed amount of time (the \emph{cycle time} or \emph{Takt} $t$) at each
station. At each station different pieces can be added, each item $i$ requiring
a certain amount of time $d_i$. The total amount of time spent in each station
cannot be more than the cycle time. A precedence graph states which items must be installed before another item can be added. All items must be placed on the car in
some station, the precedence constraints state that all preceding items must be placed on the car at an earlier or the same station. The problem
is to design the assembly line, and minimize $k$, the number of stations needed.
The stations should be numbered consecutively from 1 to $k$.
}
\end{frame}

\begin{frame}
\frametitle{Feature Overview}
\begin{itemize}
\item Direct model
\begin{itemize}
\item One task for each item to be placed, with a variable indicating the station it is added
\item Domain: possible stations, all tasks have duration one
\item Tasks with StartBeforeStart time constraints
\item Can be strengthened to EndBeforeStart for certain task pairs
\item One cumulative constraint with overall Takt resource capacity $t$
\item CumulativeNeed of each Task is its duration $d_{i}$ in seconds
\end{itemize}
\item Other models possible

\end{itemize}

\end{frame}


\begin{frame}[fragile]
\frametitle{Large Scale Example}
\begin{itemize}
\item Problem \verb|instance_n=1000\_511.alb|
\item 1000 tasks, complex precedence graph
\item Cumulative lower bound 230
\begin{itemize}
\item Sum of CumulativeNeed 229,447
\item Takt 1,000
\end{itemize}

\end{itemize}
\end{frame}



\begin{frame}
\frametitle{Process Diagram}
\includegraphics[width=\textwidth]{../salbp/images/process-diagram}
\begin{itemize}
\item Only part of overall process shown
\item StartBeforeStart links shown in green
\item Derived EndBeforeStart links in black
\end{itemize}

\end{frame}

\begin{frame}
\frametitle{Solutions}
\begin{itemize}
\item CPO
\begin{itemize}
\item 300s timeout, 4 threads
\item Solver lower bound 230
\item Cost 237 (Gap 2.95\%) after 30s
\end{itemize}

\item CPSat
\begin{itemize}
\item 300s timeout, 8 threads
\item Solver lower bound 224
\item Cost 237 (Gap 2.95\%) after 22s
\end{itemize}

\end{itemize}
\end{frame}



\begin{frame}
\frametitle{PERT Diagram of Solution}
\includegraphics[width=\textwidth]{../salbp/images/pert}
\begin{itemize}
\item Only part of diagram shown

\end{itemize}

\end{frame}

\begin{frame}
\frametitle{Cumulative Resource Chart}
\includegraphics[width=\textwidth]{../salbp/images/cumulative-level}
\begin{itemize}
\item Overall resource use quite balanced
\item Some stations have spare capacity
\end{itemize}

\end{frame}

\begin{frame}
\frametitle{Placement of Cumulative Needs}
\includegraphics[width=\textwidth]{../salbp/images/placement}
\begin{itemize}
\item Not the same solution
\item Task height is high compared to station capacity
\end{itemize}

\end{frame}


\begin{frame}
\frametitle{Overview of Benchmark Results}
\includegraphics[width=\textwidth]{../salbp/images/overview-results}
\begin{itemize}
\item Benchmark set from Otto et al, EJOR, 2013
\item 30s timeout, 4 threads for CPO, 8 threads for CPSat
\item Small instances solved to optimality
\item Larger instances have good, but not optimal solutions for both CPO and CPSat
\item Weaker lower bound for CPSat on large instances
\end{itemize}

\end{frame}

\begin{frame}
\frametitle{Alternative Model}
\begin{itemize}
\item One task per item to be placed, with a variable indicating the start time within the station
\item Domain: k*(t+1), tasks have duration $d_{i}$
\item Place \emph{Exclusion markers} in timeline to avoid tasks spanning multiple stations
\begin{itemize}
\item Place fixed Downtime at end of each station period
\item Start $t+i*(t+1)$
\item Duration one
\item Example for t=1000 and start=0: 1000, 2001, 3002, 4003 ...
\end{itemize}
\item EndBeforeStart precedence constraints between tasks
\begin{itemize}
\item Preceding tasks are either in the same station (ordered in time), or in an earlier station 
\end{itemize}

\item One disjunctive resource: maximum time available in each station is $t$ 
\end{itemize}
\end{frame}

\begin{frame}
\frametitle{Alternative Model (II)}
\begin{itemize}
\item Objective
\begin{itemize}
\item Minimize makespan (latest end) of tasks
\item Ignore downtime objects in objective
\end{itemize}
\item Needs a mapping to find station of each item
\begin{itemize}
\item $station = \lceil start/(t+1)\rceil$
\end{itemize}
\item Needs an upper bound on the number of stations to create the correct number of downtime objects
\begin{itemize}
\item Trivial: $n$
\end{itemize}


\end{itemize}
\end{frame}









\begin{frame}
\frametitle{Summary}
\begin{itemize}
\item Scheduling can be used to design factories as well
\item There can be more than one model for the same problem
\item Choosing the best model is quite difficult
\begin{itemize}
\item Experiments with realistic data are key
\end{itemize}
\item Good results compared to specialized methods
\end{itemize}
\end{frame}

