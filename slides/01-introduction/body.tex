\begin{frame}
\frametitle{Key Points}
\begin{itemize}
\item Introducing a running example
\item AI is more than LLM
\item Stochastic vs. deductive AI methods
\item Constraint Based Scheduling and its alternatives
\item Key advantages
\begin{itemize}
\item Compositional
\item Reusable
\item Explainable
\end{itemize}
\item Course structure
\end{itemize}
\end{frame}

\section{A Running Example}

\begin{frame}
\frametitle{Developing a Generic Scheduling Tool}
\begin{itemize}
\item No programming, configured by JSON input data
\item Compositional use of different constraint types
\item Different commercial or open-source back-end solvers
\item Developed in Java
\item Interactive JavaFX front-end
\item Can by used as back-end scheduling tool/server
\item Instance generator included
\item Readers for multiple benchmark types included
\item Release planned early 2025
\item Preview during the course, hands-on experience this afternoon
\end{itemize}
\end{frame}

\begin{frame}
\frametitle{Introducing a Simple Scheduling Problem}
\begin{itemize}
\item Will be used throughout the program
\item Generated by instance generator
\item 50 orders for different products, release and due dates
\item 4 stages, always performed in the same sequence
\item Two identical machines available for each stage
\item Cumulative manpower constraint
\item Complete description as JSON document
\end{itemize}
\end{frame}

\begin{frame}[fragile]
\frametitle{Excerpt of JSON Description}
\begin{lstlisting}[language=json]
  "order": [
    {
      "product": "Prod0",
      "process": "Process 0",
      "due": 5449,
      "releaseDate": "1/10/2024 00:00",
      "release": 0,
      "qty": 7,
      "dueDate": "19/10/2024 22:05",
      "name": "Order0",
      "earlinessWeight": 1,
      "latenessWeight": 1
    },
\end{lstlisting}
\end{frame}

\begin{frame}
\frametitle{Orders Loaded}
\includegraphics[width=.6\textwidth]{../01-introduction/images/orders}
\end{frame}

\begin{frame}
\frametitle{Process Diagram}
\begin{columns}
\begin{column}{0.6\textwidth}
\begin{itemize}
\item Processes describe how products are made
\item Multiple process steps
\item Not always in a straight sequence
\item Duration formula based on quantity made
\item Temporal constraints between steps
\item Possible machines to run on
\item Resource requirements (manpower, electricity,...)
\end{itemize}
\end{column}
\begin{column}{0.4\textwidth}
\includegraphics[width=\textwidth]{../01-introduction/images/process-diagram}
\end{column}
\end{columns}
\end{frame}


\begin{frame}
\frametitle{Selecting Solver Options}
\begin{columns}
\begin{column}{0.65\textwidth}
\begin{itemize}
\item Which constraints to enforce
\begin{itemize}
\item Here: do not enforce due dates
\end{itemize}
\item Additional constraints to try
\item Why solver to run
\begin{itemize}
\item Here: Use open-source CPSat solver
\end{itemize}
\item Which objective to use
\begin{itemize}
\item Here: Makespan, overall project end
\end{itemize}
\item What resources to use
\begin{itemize}
\item Allow 30 seconds
\item Use 8 parallel threads
\end{itemize}
\end{itemize}
\end{column}
\begin{column}{0.35\textwidth}
\includegraphics[width=.8\textwidth]{../01-introduction/images/options}
\end{column}
\end{columns}
\end{frame}

\begin{frame}
\frametitle{Schedule - Initial Gantt Chart}
\includegraphics[width=\textwidth]{../01-introduction/images/gantt-initial}
\end{frame}

\begin{frame}
\frametitle{Adapted Gantt Chart}
\includegraphics[width=\textwidth]{../01-introduction/images/gantt-adapted}
\end{frame}

\begin{frame}
\frametitle{Cumulative Resource Chart}
\includegraphics[width=\textwidth]{../01-introduction/images/cumulative-chart}
\end{frame}

\begin{frame}
\frametitle{Intermediate Solutions Found}
\includegraphics[width=\textwidth]{../01-introduction/images/intermediate-solutions}

\begin{itemize}
\item Ongoing search for improved solutions
\item Depends on time and resources, solver used
\end{itemize}

\end{frame}






\section{Artificial Intelligence}

\begin{frame}
\frametitle{What is Artificial Intelligence?}
\begin{quote}
Artificial intelligence, or AI, is the field that studies the synthesis and analysis of computational agents that act intelligently.
\end{quote}
David Poole, Alan Mackworth. Artificial Intelligence, Cambridge University Press, 3rd Edition, 2023.

\begin{itemize}
\item This definition leaves a lot of questions.

\end{itemize}
\end{frame}

\begin{frame}
\frametitle{The Great Divide}
\begin{itemize}
\item Two fundamentally different approaches to AI
\begin{itemize}
\item Reasoning based
\item Stochastic
\end{itemize}
\item Currently, the stochastic methods get all the attention
\item But they have their problems
\begin{itemize}
\item Impossible to understand what is happening inside
\item Hallucinations, making up convincing false statements
\item Enormous resource requirements
\item Privacy/IP of training data
\item Really limited to a few multi-nationals
\end{itemize}

\end{itemize}
\end{frame}




\begin{frame}
\frametitle{Topics in AI}
\begin{columns}
\begin{column}{0.5\textwidth}
\begin{itemize}
\item Chapter Structure of AI Book
\item Shows importance of deductive/search based approaches
\end{itemize}
\end{column}
\begin{column}{0.5\textwidth}
\includegraphics[width=\textwidth]{../01-introduction/images/poole-mackworth-chapters-arrow}
\end{column}
\end{columns}
\end{frame}



\begin{frame}
\frametitle{What is Constraint Programming?}
\begin{quote}
Constraint programming technology is used to find solutions to scheduling and combinatorial optimization problems. It is based primarily on computer science fundamentals, such as logic programming and graph theory, in contrast to mathematical programming, which is based on numerical linear algebra.

Constraint programming is invaluable when dealing with the complexity of many real-world sequencing and scheduling problems.
\end{quote}
IBM {\tiny (\url{https://ibmdecisionoptimization.github.io/docplex-doc/cp.html})}
\end{frame}



\section{Scheduling}

\begin{frame}
\frametitle{What is Scheduling?}
\begin{quote}
Scheduling is a decision-making process that is used on a regular basis in
many manufacturing and services industries. It deals with the allocation of
resources to tasks over given time periods and its goal is to optimize one or
more objectives.
\end{quote}
Michael Pinedo. Scheduling. Springer, 5th edition, 2016. 
\end{frame}

\begin{frame}
\frametitle{Information Flow Diagram in a Manufacturing System}
\begin{columns}
\begin{column}{0.5\textwidth}
\begin{itemize}
\item According to Pinedo, page 5.
\item We focus on what is shown as \emph{detailed scheduling}
\end{itemize}
\end{column}
\begin{column}{0.5\textwidth}
\includegraphics[width=\textwidth]{../01-introduction/images/information-flow}
\end{column}
\end{columns}
\end{frame}




\subsection{Constraint-Based Scheduling}

\begin{frame}
\frametitle{Constraint Programming - in a nutshell}
\begin{itemize}
\item Declarative description of problems with
\begin{itemize}
\item {\em Variables} which range over (finite) sets of values
\item {\em Constraints} over subsets of variables which restrict possible value combinations
\item A {\em solution} is a value assignment which satisfies all constraints
\end{itemize}

\item Constraint propagation/reasoning
\begin{itemize}
\item Removing inconsistent values for variables
\item Detect failure if constraint can not be satisfied
\item Interaction of constraints via shared variables
\item Incomplete
\end{itemize}

\item Search
\begin{itemize}
\item User controlled assignment of values to variables
\item Each step triggers constraint propagation 
\end{itemize}
\item Different domains require/allow different methods
\end{itemize}
\end{frame}

\begin{frame}
  \frametitle{Constraint Programming is Different}
  \begin{itemize}
  \item Declarative Programming
    \begin{itemize}
    \item Concentrate on what you want
      \item Not how to get there
      \item Program != Algorithm
      \item Program = Model
    \end{itemize}
    \item Applied to Combinatorial Problems
      \begin{itemize}
        \item No complete polynomial algorithms known (exist?)
        \item CP less ad-hoc than heuristics
        \item Models can evolve
  \end{itemize}
  \end{itemize}
  \end{frame}
    
\begin{frame}
  \frametitle{A Subtractive Process}
  \begin{textblock}{4}(8,-3)
    \includegraphics[width=4cm]{../01-introduction/images/stages}
  \end{textblock}
  \vfill
  \begin{quote}
    ``Oh, bosh, as Mr. Ruskin says. Sculpture, per se, is the simplest thing in the world. All you have to do is to take a big chunk of marble and a hammer and chisel, make up your mind what you are about to create and chip off all the marble you don't want.''-Paris Gaulois.
  \end{quote}
  
  {\tiny Source: \url{https://quoteinvestigator.com/2014/06/22/chip-away/}}
\end{frame}

\subsection{Other Scheduling Solution Approaches}

\begin{frame}
\frametitle{Other Technologies}
\begin{itemize}
\item Heuristics
\item Integer Programming
\item Local search
\item Deep neural networks
\end{itemize}
\end{frame}

\begin{frame}
\frametitle{Heuristics}
\begin{columns}
\begin{column}{0.5\textwidth}
\begin{itemize}
\item Do not try to explore the search space
\item Find a good enough solution by making greedy choices
\item More general meta-heuristics schemes
\item Very good heuristics exist for specific problem types
\item Not compositional, added constraints may destroy existing approach
\item Often not reusable code base
\end{itemize}
\end{column}
\begin{column}{0.5\textwidth}
\includegraphics[width=\textwidth]{../01-introduction/images/schedulingschemes}
{\tiny From: Singh, Kumar, and Singh: An empirical investigation of task scheduling and VM consolidation schemes in cloud environment, Computer Science review, 2023, \url{https://www.sciencedirect.com/science/article/pii/S1574013723000503}}

\end{column}
\end{columns}
\end{frame}

\begin{frame}
\frametitle{Integer Programming}
\begin{columns}
\begin{column}{0.5\textwidth}
\begin{itemize}
\item Restrict yourself to linear constraints
\item Powerful reasoning on the complete set of constraints
\begin{itemize}
\item Linear Programming
\item Cut generation
\end{itemize}
\item Expressing scheduling constraints can be difficult
\item Scalability issues for detailed scheduling
\end{itemize}
\end{column}
\begin{column}{0.5\textwidth}
\includegraphics[width=.7\textwidth]{../01-introduction/images/production-planning}
{\tiny \url{https://link.springer.com/book/10.1007/0-387-33477-7}}
\end{column}
\end{columns}
\end{frame}

\begin{frame}
\frametitle{Local Search}
\begin{columns}
\begin{column}{0.5\textwidth}
\begin{itemize}
\item Start with an initial solution
\item Try out changes that maintain feasibility
\item Gradual improvement over time
\item Not compositional
\item No guarantee of solution quality
\item Unifying approach: Constraint-Based Local Search
\end{itemize}
\end{column}
\begin{column}{0.5\textwidth}
\includegraphics[width=.8\textwidth]{../01-introduction/images/constraint-based-local-search}
{\tiny \url{https://mitpress.mit.edu/9780262220774/constraint-based-local-search/}}
\end{column}
\end{columns}
\end{frame}

% \begin{frame}
% \frametitle{Neural Networks}
% \begin{columns}
% \begin{column}{0.5\textwidth}
% \begin{itemize}
% \item 
% \end{itemize}
% \end{column}
% \begin{column}{0.5\textwidth}
% \includegraphics[width=\textwidth]{../01-introduction/images/}
% \end{column}
% \end{columns}
% \end{frame}



\section{Course Structure}

\subsection{Timetable}

\begin{frame}
\frametitle{Course Structure}
\begin{tabular}{lp{4cm}p{4cm}}\toprule
Time & Day 1 & Day 2\\\midrule
09:00-09:30 & Registration & - \\
09:30-10:30 & Introduction \& Motivation & Costs \& Objective Functions\\
10:30-11:00 & Coffee & Coffee \\
11:00-12:30 & Scheduling Concepts& Advanced Concepts\\
12:30-13:30 & Lunch & Lunch \\
13:30-15:00 & Machine Constraints & Case Studies\\
15:00-15:30 & Coffee & Coffee \& Close\\
15:30-16:30 & Experiments & -\\
\bottomrule
\end{tabular}
\end{frame}


\subsection{What is not covered?}

\begin{frame}
\frametitle{What is not covered?}
\begin{itemize}
\item How does it all work?
\item How to integrate into an existing IT environment
\item How to define and solve new constraints
\item Interactive solving techniques 
\end{itemize}
\end{frame}



\begin{frame}
\frametitle{How does it all work?}
\begin{itemize}
\item You don't really need to know this to use Constraint Programming
\item Advantage of declarative, compositional formulation
\item I teach an introductory course on Constraint Programming for CRT-AI
\begin{itemize}
\item Ask for details if interested
\end{itemize}

\item Overview of courses, books and materials at \url{https://arxiv.org/abs/2403.12717}
\end{itemize}
\end{frame}

%\subsection{A Short History}

\section{Summary}

\begin{frame}
\frametitle{Summary}
\begin{itemize}
\item Why use Constraint Based Scheduling?
\item Compared to other AI methods
\item Compared to other solution approaches
\end{itemize}
\end{frame}

