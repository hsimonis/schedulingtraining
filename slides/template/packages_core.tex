
%\usepackage[medium]{ubuntu}  % May need to be installed separately from https://github.com/tzwenn/ubuntu-latex-fonts
\usepackage{url}
\usepackage[normalem]{ulem}
\usepackage{listings}
\usepackage{bera}% optional: just to have a nice mono-spaced font
\usepackage{xcolor}

\definecolor{eclipseStrings}{RGB}{42,0.0,255}
\definecolor{eclipseKeywords}{RGB}{127,0,85}
\colorlet{numb}{magenta!60!black}

\lstdefinelanguage{json}{
    basicstyle=\normalfont\ttfamily,
    commentstyle=\color{eclipseStrings}, % style of comment
    stringstyle=\color{eclipseKeywords}, % style of strings
    numbers=left,
    numberstyle=\scriptsize,
    stepnumber=1,
    numbersep=8pt,
    showstringspaces=false,
    breaklines=true,
    frame=lines,
%    backgroundcolor=\color{gray}, %only if you like
    string=[s]{"}{"},
    comment=[l]{:\ "},
    morecomment=[l]{:"},
    literate=
        *{0}{{{\color{numb}0}}}{1}
         {1}{{{\color{numb}1}}}{1}
         {2}{{{\color{numb}2}}}{1}
         {3}{{{\color{numb}3}}}{1}
         {4}{{{\color{numb}4}}}{1}
         {5}{{{\color{numb}5}}}{1}
         {6}{{{\color{numb}6}}}{1}
         {7}{{{\color{numb}7}}}{1}
         {8}{{{\color{numb}8}}}{1}
         {9}{{{\color{numb}9}}}{1}
}
\usepackage{graphics}
\usepackage{tikz}
\usetikzlibrary{shapes,calc,through,backgrounds,arrows,automata,positioning}

\usepackage{ragged2e}
\usepackage{multicol}
\usepackage{subfigure}
\usepackage{booktabs}
\usepackage{amsmath}
\usepackage{hyperref}
\input{../template/entire-colors}
\usepackage{xspace}
\usepackage{mhchem}
\usepackage{MnSymbol,wasysym}

\usepackage{qrcode}

 
\usepackage{amssymb,marvosym}
\usepackage{colortbl}
\usepackage{array}
\usepackage{textpos}
\usepackage{textcomp,pifont}
\newcommand{\pred}[1]{\texttt{#1}}
\mode<all>{
}
\newcommand{\avail}{\textcolor{green}{\ding{51}}}%
\newcommand{\soon}{\textcolor{green}{(\ding{51})}}%
\newcommand{\navail}{\textcolor{orange}{\ding{55}}}%

\usepackage{pdfpages}

\usepackage{pgfplots}
\pgfplotsset{width=7cm,compat=1.15}
%\usepackage[absolute,overlay]{textpos}

\usepackage{amssymb}
\usepackage{pifont}
\newcommand{\greentick}{{\color{green}\checkmark}}
\newcommand{\orangetick}{{\color{orange}(\checkmark)}}
\newcommand{\redx}{{\color{red}\ding{55}}}

\usepackage{adjustbox}

\newcolumntype{R}[2]{%
    >{\adjustbox{angle=#1,lap=\width-(#2)}\bgroup}%
    l%
    <{\egroup}%
}
\newcommand*\rot{\multicolumn{1}{R{45}{1em}}}% no optional argument here, please!

\input{../template/mznlisting}
\lstset{language=Mzn}

\usepackage[english]{babel}
% or whatever

\usepackage[latin1]{inputenc}
% or whatever

\usepackage{times}
\usepackage[T1]{fontenc}
% Or whatever. Note that the encoding and the font should match. If T1
% does not look nice, try deleting the line with the fontenc.

\mode<article>{
\usepackage{makeidx}
\makeindex
\usepackage{fullpage}
\usepackage{hyperref}
}
\hypersetup{pdfauthor={Helmut Simonis},pdfsubject={Constraint Based Production Scheduling},pdfkeywords={Constraint Based;Production Scheduling;Constraint Programming;Scheduling;Manufacturing;Cumulative Constraint;Disjunctive Constraint}}

\mode<presentation>{
\author{Helmut Simonis}
%\email{helmut.simonis@insight-centre.org}
}
\mode<article>{
\author{Helmut Simonis\\%
\\%
email: \url{helmut.simonis@insight-centre.org}\\%
homepage: \url{http://http://insight-centre.org/}\\%
\\%
ENTIRE EDIH\\%
Insight SFI Centre for Data Analytics\\%
School of Computer Science and Information Technology\\%
University College Cork\\
Ireland}
}
\institute[ENTIRE] % (optional, but mostly needed)
{ENTIRE EDIH\\School of Computer Science and Information Technology\\University College Cork\\Ireland}

\date[Production Scheduling] % (optional)
{Constraint Based Production Scheduling}



% Delete this, if you do not want the table of contents to pop up at
% the beginning of each subsection:
\AtBeginLecture{
\begin{frame}
\Large Lecture: \insertlecture
\end{frame}
}
\AtBeginPart{
\begin{frame}
\partpage
\end{frame}
}

\AtBeginSection[]
{
  \begin{frame}<beamer>
    \frametitle{Outline}
    \tableofcontents[sectionstyle=show/shaded,subsectionstyle=show/show/hide]
  \end{frame}
}


% If you wish to uncover everything in a step-wise fashion, uncomment
% the following command: 

%\beamerdefaultoverlayspecification{<+->}

