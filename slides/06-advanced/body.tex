\begin{frame}
\frametitle{Key Points}
\begin{itemize}
\item We present some more advanced concepts in scheduling
\item These occur in more specialized problem areas
\item Typically require more work on modelling
\item Solver support may be limited
\end{itemize}
\end{frame}

\section{Sequence Dependent Setup-Time}

\begin{frame}
\frametitle{Sequence Dependent Setup-Time \avail}
\begin{itemize}
\item Our usual disjunctive resource model assumes we can change easily from one task to the next
\item There might be a cleaning/setup time required
\begin{itemize}
\item This is part of the fixed duration part of a processStep description
\end{itemize}
\item In some cases it is more complex
\begin{itemize}
\item On some machines there is a setup-time required which depends on both the previous and the next product
\item This time varies significantly between product combinations
\item Typically, the time depends on some properties of the products
\end{itemize}
\item The setup time is non-productive, and should be avoided when possible
\end{itemize}
\end{frame}

\begin{frame}
\frametitle{Computed Setup-Time Matrix}
\includegraphics[width=\textwidth]{../06-advanced/images/setupmatrix}
\begin{itemize}
\item This needs to be computed from first principles, not maintained by hand!
\item Available as input data in JSON format
\end{itemize}

\end{frame}

\begin{frame}
\frametitle{Relation to TSP}
\begin{itemize}
\item Computing the optimal sequence of setup times is a variant of the \emph{Travelling Salesman Problem (TSP)}
\item Another of the classical hard combinatorial problems
\item Due to the structure of the data, setup-time problem often are simpler to solve
\begin{itemize}
\item Changing between very similar products needs no setup-time
\item Using a simple rule about product compatibility produces best results
\item Example: dark-chocolate $\rightarrow$ milk chocolate $\rightarrow$ white chocolate $\rightarrow$ milk chocolate $\rightarrow$ draw chocolate
\end{itemize}
\item Problems get more difficult when release/due dates need to be respected
\item This is the equivalent to the \emph{VRPTW (Vehicle Routing Problem with Time Windows)}
\end{itemize}
\end{frame}

\begin{frame}
\frametitle{Xmas Shopping Hint}
\begin{columns}
\begin{column}{0.5\textwidth}
\begin{itemize}
\item W. Cook. In Pursuit of the Travelling Salesman. Princeton University Press, 2011
\item Entertaining general science presentation of the TSP and related issues
\end{itemize}
\end{column}
\begin{column}{0.4\textwidth}
\includegraphics[width=.8\textwidth]{../06-advanced/images/In_Pursuit_of_the_Traveling_Salesman.jpg}
\end{column}
\end{columns}
\end{frame}



\begin{frame}
\frametitle{Setup Times Constraints can be Included in Model}
\includegraphics[width=\textwidth]{../06-advanced/images/schedule-with-setup}
\begin{itemize}
\item Shown in Machine Gantt chart, enable display in \texttt{Layout} tab
\item So far, only in CPO, not in CPSat model
\end{itemize}

\end{frame}


\begin{frame}
\frametitle{Related Problem: Forbidden Transitions \navail}
\begin{itemize}
\item For safety reasons, it may be forbidden to change from some product to some specific other products
\item Contamination risk is considered too high
\item Examples
\begin{itemize}
\item In food production: Is this product peanut free?
\item In food production: Directly changing from dark to white chocolate is not allowed
\item In chemical plants: Contamination may lead to explosions
\end{itemize}
\item These transitions are called \emph{forbidden}, and must be avoided
\item Careful, it is easy to paint yourself into a corner! 
\end{itemize}
\end{frame}

\section{Transportation Time}

\begin{frame}
\frametitle{Dealing with Transportation Times}
\begin{itemize}
\item Really two different problems 
\begin{itemize}
\item In one, the resources are in fixed locations, and we transport the jobs between the locations
\item In the other, the tasks are in fixed locations, and we transport the resources between them
\end{itemize}

\end{itemize}
\end{frame}

\subsection{Transportation of Materials}

\begin{frame}
\frametitle{Transportation of Jobs}
\begin{columns}
\begin{column}{0.5\textwidth}
\begin{itemize}
\item Example from a project with J\&J in Limerick
\item Considering a \emph{factory of the future} based on agile machines
\item Robots that can be configured to perform many different tasks
\item These robots may be inside one or more factories
\item How to arrange them to minimize impact of transport on production
\end{itemize}
\end{column}
\begin{column}{0.5\textwidth}
\includegraphics[width=\textwidth]{../06-advanced/images/agilemachine}

{\tiny from J\&J}
\end{column}
\end{columns}
\end{frame}

\begin{frame}
\frametitle{Layout of Factor in Matrix Form}
\includegraphics[width=\textwidth]{../06-advanced/images/factorymodel}
\begin{itemize}
\item Materials are transported between stations by moving robots
\item Layout of factory determines delay caused by transport
\end{itemize}

\end{frame}

\begin{frame}
\frametitle{Inclusion in Model \soon}
\includegraphics[width=\textwidth]{../06-advanced/images/transport-matrix}
\begin{itemize}
\item Add location attribute to each resource
\item Include transport time as element in temporal constraints 
\end{itemize}
\end{frame}

\begin{frame}
\frametitle{More Complex Variant}
\begin{itemize}
\item Schedule the moving robots as well
\item Assume that an empty robot travels much faster than a loaded one
\item We can treat the robots as a machine choice resource for the transportation tasks
\end{itemize}
\end{frame}

\begin{frame}
\frametitle{Even More Complex Variant}
\begin{itemize}
\item Schedule the moving robots as well
\item They move at the same speed empty and loaded
\item We can to bring them from the end of one transport task to the start of the next one
\item This is a vehicle routine problem
\item In some industries, this is the harder problem compared to scheduling the plant itself
\begin{itemize}
\item Torpedo scheduling in steel plant: rail cars holding molten steel, quantities limited  
\end{itemize}

\end{itemize}
\end{frame}

\begin{frame}
\frametitle{Torpedo Scheduling (CP 2016 Challenge)}
\includegraphics[width=\textwidth]{../06-advanced/images/torpedo}
{\tiny (from ACP Website \url{http://cp2016.a4cp.org/program/acp-challenge/})} 
\end{frame}




\subsection{Transportation of Resources/Personnel}

\begin{frame}
\frametitle{Scheduling Service Visits}
\begin{itemize}
\item Based on a project with UTRC-I, UTRC, OTIS
\item Schedule visits to maintain equipment installed in customer premises
\item Resources are the service engineers
\item They have to travel between locations and perform work there
\item The tasks are the maintenance operations required to keep equipment working
\item Also called \emph{Traveling Repair-person Problem}

\end{itemize}
\end{frame}

\begin{frame}
\frametitle{Planning Maintenance Visits for Service Personnel}
\includegraphics[width=.85\textwidth]{../06-advanced/images/schedule}
\begin{itemize}
\item Include single day trips, multi-day tours
\item Most of the time spent at customer locations
\end{itemize}
\end{frame}

\begin{frame}
\frametitle{Re-scheduling Problem}
\includegraphics[width=.85\textwidth]{../06-advanced/images/callbacks}
\begin{itemize}
\item How to react when a customer is trapped in an elevator
\item All you engineers are on service calls
\item Who you gonna call?
\end{itemize}

\end{frame}



\begin{frame}
\frametitle{Advertisement}
\begin{itemize}
\item This will be described in more detail in a new course
\item AI Fundamentals: Skill Development Program on Transportation Optimization
\item Arriving in 2025 at this location
\end{itemize}
\end{frame}


%\section{Raw Material Availability}

%\section{Producer/Consumer Constraints}

%\section{Explainability}

\section{Summary}

\begin{frame}
\frametitle{Summary}
\begin{itemize}
\item We presented some more advanced topics
\begin{itemize}
\item Sequence dependent setup
\item Transportation time
%\item Raw material availability
%\item Producer/Consumer constraints
\end{itemize}
\item Not available in every solver
\item Useful concepts when dealing with specific scheduling problems
\item Leading to another \emph{Skills Development Program}
\end{itemize}
\end{frame}

