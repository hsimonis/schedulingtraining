\begin{frame}
\frametitle{Key Points}
\begin{itemize}
\item Discusses two topics:
\begin{itemize}
\item Solve a very specific industrial scheduling problem from the ASSISTANT EU project
\item Discuss the general issue of short-term scheduling vs. long-term objectives
\end{itemize}
\end{itemize}
\end{frame}

%\includepdf[pages={2},frame=true,pagecommand={\thispagestyle{empty}}]{InsightOneSlide}
% \begin{frame}
% \frametitle{Insight is one of the largest data research and innovation centres in Europe...}
% \includegraphics[width=.95\textwidth]{../ovenscheduling/images/onepage}
% \end{frame}

\begin{frame}
\frametitle{Research Challenge}
\begin{itemize}
\item Often the long-term business objectives are not visible in the operational decision problem
\item We optimize a short-term objective without understanding the impact in the long term
\item What choices should we make in short-term to improve overall result?
\item Especially important when future data not yet visible
\item Surprisingly, this problem is rarely discussed in literature
\end{itemize}
\end{frame}

\begin{frame}
\frametitle{Examples}
\begin{columns}[t]
\begin{column}{0.5\textwidth}
\begin{itemize}
\item Production Scheduling
\item Nearly all scheduling benchmarks use $c_{max}$ (makespan) as objective
\item Why?
\item Do we want to close factory as rapidly as possible?
\end{itemize}
\includegraphics[width=6cm]{../ovenscheduling/images/scheduling}
\end{column}
\begin{column}{0.5\textwidth}
\begin{itemize}
\item Car Sequencing
\item The best heuristics push difficult cars to the edge of schedule
\item Because they are easier to schedule this way
\item But: It makes it hard to schedule next day
\end{itemize}
\includegraphics[width=6cm]{../ovenscheduling/images/carseq}
\end{column}
\end{columns}
\end{frame}

\begin{frame}
\frametitle{Examples}
\begin{columns}[t]
\begin{column}{0.5\textwidth}
\begin{itemize}
\item Personnel Rostering
\item Satisfy working rules and demands for period
\item But: rules apply on a rolling horizon
\item Easy to over-constrain problem for next period
\end{itemize}
\includegraphics[width=4cm]{../ovenscheduling/images/rfo}
\end{column}
\begin{column}{0.5\textwidth}
\begin{itemize}
\item Transportation Planning
\item Build daily delivery tours, optimizing cost
\item Where are your trucks at 10PM?
\item Also, avoid cherry-picking at start of week
\end{itemize}
\includegraphics[width=5.5cm,angle=180]{../ovenscheduling/images/metagointago}
\end{column}
\end{columns}
\end{frame}

\begin{frame}
\frametitle{Problem Studied Here}
\begin{itemize}
\item Example from the ASSISTANT EU project (ended last year)
\item Oven schedule for one of the industrial partners
\item Schedule tasks on a set of ovens
\item Tasks can share oven only if they are compatible
\item Conflicting objectives
\begin{itemize}
\item Energy use of ovens very significant, reduce when ovens are used
\item Waiting for an oven affects quality of product
\end{itemize}
\item Jobs only visible when previous process step starts
\item Currently scheduled by hand, industry partner expressed strong need for change
\end{itemize}
\end{frame}


\begin{frame}
\frametitle{What does this look like in the real world?}
\begin{tabular}{cc}
\includegraphics[width=4cm]{../ovenscheduling/images/oven}
&
\includegraphics[width=6cm]{../ovenscheduling/images/compressor}
\\
Industrial Oven & Rotors in Compressor
\end{tabular}
\end{frame}

\begin{frame}
\frametitle{Solution Approach: Constraint Programming}
\begin{columns}[t]
\begin{column}{0.5\textwidth}
\begin{itemize}
\item Declarative modelling approach for combinatorial problems
\begin{itemize}
\item Problem expressed in terms of variables and constraints
\end{itemize}
\item Global constraints
\begin{itemize}
\item Combines expressive modelling abstractions and powerful reasoning
\item Examples: disjunctive, cumulative, global\_cardinality
\end{itemize}
\item Compositional: Add constraints as required
\end{itemize}
\end{column}
\begin{column}{0.5\textwidth}
\begin{itemize}
\item Main application areas
\begin{itemize}
\item Scheduling, rostering, transportation
\item Also: test generation, verification, configuration
\end{itemize}
\end{itemize}

\includegraphics[width=.9\textwidth]{../ovenscheduling/images/wordcloud}
\end{column}
\end{columns}
\end{frame}

\begin{frame}
\frametitle{Overall Decomposition (Standard)}
\begin{itemize}
\item We can only see that far into future
\item We do not want to take decisions now that\\ we might regret later
\item We have to make some decisions now otherwise we never do anything
\item \emph{Rolling horizon} decomposition
\begin{itemize}
\item We schedule up to \emph{lookahead horizon} units into the future
\item We commit to implement resulting schedule only to up \emph{commitHorizon}
\item We reschedule when we receive new information, or we reach the end of commitment 
\item We solve each short-term sub problem based on short-term objectives
\end{itemize}
\end{itemize}
\scalebox{0.8}{
\begin{tikzpicture}
\draw[left color=insight-blue] (0,0)  rectangle (10,1);
\node[left] at (0,0.5) {Certain};
\node[right] at (10,0.5) {Uncertain};
\draw[thick,<->] (0,-0.2) -- node[below] {Lookahead horizon} (4,-0.2);
\draw[thick,<->] (0,1.2) -- node[above] {Commit} (2,1.2);
\draw[thick,<->,draw=black!20] (2,1.2) -- node[above] {Ignore} (4,1.2);
\end{tikzpicture}
}


\end{frame}



\begin{frame}
\frametitle{Short-Term Schedule Modelling}
\begin{itemize}
\item Challenge: There is no global constraint to express the oven resource constraint
\item We are not able to invest a lot of time/resources to develop such a constraint
\item Two choices:
\begin{itemize}
\item Two traditional models with variables linking them (Lackner et al, Constraints 2023)
\item Direct model expressing conditions as disjunctions of basic constraints
\end{itemize}
\end{itemize}
\end{frame}

\begin{frame}
\frametitle{The Standard Pieces}
\begin{itemize}
\item Jobs $N$ consisting of multiple stages $Q$, tasks for each stage of each job, running on machines $M$
\item Release dates $r_{i}$ of jobs given by up-stream schedule
\item WiP $w_{k}$ on certain machines resulting from earlier schedule
\item Machine $m_{ij}$ and start variables $s_{ij}$ for each task
\item Precedence constraints between tasks of each jobs, with total waiting time $c_{i}$ when waiting for resource
\item Total number of ovens used in schedule \emph{nrOvens} by \emph{nvalue} constraint
\end{itemize}
\begin{equation*}
\textrm{nvalue}(\textrm{nrOvens},[m_{ij} | i \in N, j \in Q] \textrm{++} [k | k \in M \textrm{ s.t. } w_{k} > 0])
\end{equation*}
\end{frame}

\begin{frame}
\frametitle{Resource Constraints}
We start from the basic decomposition of the\\disjunctive machine choice constraint
\begin{align*}
\forall_{i_{1},i_{2} \in N}\forall_{j_{1},j_{2} \in Q \textrm{ s.t. } <i_{1},j_{1}> \neq <i_{2},j_{2}>}:\quad m_{i_{1}j_{1}} \neq m_{i_{2}j_{2}} \lor&\\
 s_{i_{1}j_{1}} \geq s_{i_{2}j_{2}} + d_{i_{2}j_{2}} \lor& \\
 s_{i_{2}j_{2}} \geq s_{i_{1}j_{1}} + d_{i_{1}j_{1}}
\end{align*}
Express case where tasks share an oven (only when types and stages are the same)
\begin{align*}
\forall_{i_{1},i_{2} \in N \textrm{ s.t. } i_{1} \neq i_{2}}\forall_{j \in Q}:\quad m_{i_{1}j} \neq m_{i_{2}j} \lor & \\
s_{i_{1}j} \geq s_{i_{2}j} + d_{i_{2}j} \lor& \\
 s_{i_{2}j} \geq s_{i_{1}j} + d_{i_{1}j} \lor& \\
 (t_{i_{1}j_{1}} = t_{i_{2}j_{2}} \land m_{i_{1}j} = m_{i_{2}j} \land s_{i_{1}j} = s_{i_{2}j})
\end{align*}
\end{frame}

\begin{frame}
\frametitle{Limit stacking}
Need binary variables $b_{i_1i_2j}$ to state that two jobs\\$i_1$ and $i_2$ share oven in stage $j$
\begin{align*}
\forall_{i_{1},i_{2} \in N \textrm{ s.t. } i_{1} < i_{2}}\forall_{j \in Q}:\quad (b_{i_{1}i_{2}j} = 0 \land (m_{i_{1}j} \neq m_{i_{2}j} \lor & \\
s_{i_{1}j} \geq s_{i_{2}j} + d_{i_{2}j} \lor& \\
 s_{i_{2}j} \geq s_{i_{1}j} + d_{i_{1}j}) \lor& \\
 (b_{i_{1}i_{2}j} = 1 \land t_{i_{1}j_{1}} = t_{i_{2}j_{2}} \land m_{i_{1}j} = m_{i_{2}j} \land s_{i_{1}j} = s_{i_{2}j})
\end{align*}
Count how many jobs share stage $j$ with job $i$
\begin{equation*}
\forall_{i \in N}\forall_{j \in Q}:\quad z_{ij} = \sum_{i_{1} = 1}^{i-1} b_{i_{1}ij} + \sum_{i2 = i+1}^{n} b_{ii_{i2}j}
\end{equation*}
Limit how many tasks can be stacked together
\begin{equation*}
\forall_{i \in N}\forall_{j \in Q}:\quad z_{ij} < \textrm{maxStacked}
\end{equation*}

\end{frame}

\begin{frame}
\frametitle{This should not work!}
\begin{itemize}
\item Weakness of basic decomposition model was the reason to develop the scheduling constraints in the first place
\item Does not scale well to thousands of tasks
\item But model is well suited to some solvers
\begin{itemize}
\item SAT based solvers, Chuffed, CP-SAT (OR-Tools)
\item MIP solvers
\end{itemize} 
\item This works (only) as long as problem size stays manageable
\end{itemize}
\end{frame}

\begin{frame}
\frametitle{Compound Objective}
\begin{equation*}
\min \alpha_{1}\sum_{i \in N} c_{i} + \alpha_{2} \textrm{nrOvens} +\alpha_{3}\sum_{i \in N, j \in Q} z_{ij} \label{obj}
\end{equation*}
\begin{itemize}
\item Three conflicting elements
\begin{itemize}
\item Total waiting time for jobs
\item Number of ovens used
\item Number of tasks stacked (negative coefficient)
\end{itemize}
\item Reducing waiting time requires using more ovens
\item Improved stacking will require for one job to wait until second is ready
\end{itemize}
\end{frame}


\begin{frame}
\frametitle{Short-Term Schedule: Job View}
\includegraphics[width=.8\textwidth]{../ovenscheduling/images/jobgantt}
\end{frame}

\begin{frame}
\frametitle{Short Term Schedule: Resource View}
\includegraphics[width=.8\textwidth]{../ovenscheduling/images/ovengantt}
\end{frame}

\begin{frame}
\frametitle{Are the short-term solutions good?}
\begin{itemize}
\item We solve many problems to optimality, depending on solver
\item Optimality gap is small, increasing search time helps a bit
\item But are we optimizing the best possible objective?
\end{itemize}
\end{frame}

\begin{frame}
\frametitle{Long Term Schedule: Detailed Schedule}
\includegraphics[width=.8\textwidth]{../ovenscheduling/images/ovenschedule}
\end{frame}

\begin{frame}
\frametitle{Long Term Schedule: Abstracted Oven Runs}
\includegraphics[width=.8\textwidth]{../ovenscheduling/images/ovenruns}
\end{frame}

\begin{frame}
\frametitle{Is that a good global schedule? KPIs}
\scalebox{0.6}{
\begin{tabular}{llp{6cm}}\toprule
Name & Unit & Explanation \\\midrule
Global Time & Seconds & Total time for solving all sub problems\\
Nr Jobs & -& Total number of jobs scheduled\\
Nr Tasks & -& Total number of tasks scheduled\\
Percent Optimal & Percentage (0-100) & How many sub problems were solved to optimality\\
Percent Stacked Tasks & Percentage (0-100) & Percentage of all tasks scheduled that were stacked\\
Percent Jobs No Wait & Percentage (0-100) & Percentage of jobs that were scheduled without any waiting time\\
Job Average Wait & Minutes & Average wait time over all jobs \\
Job Maximal Wait & Minutes & Largest waiting time for any job scheduled\\
Ovens Used & - & Total number of ovens used during period\\
Avg Task Duration & Minutes & Average tasks duration (influenced by stacking)\\
Oven Runs & - & Number of oven runs over total horizon\\
Run Overhead Percent & Percentage (0-100) & Overhead during oven runs when machine is idle\\
Avg Runs per Oven Used & - & Average number of oven runs per oven used\\ 
\bottomrule
\end{tabular}
}
\end{frame}



\begin{frame}
\frametitle{Impact of Lookahead Parameter}
\includegraphics[width=.8\textwidth]{../ovenscheduling/images/lookahead}
\end{frame}

\begin{frame}
\frametitle{Impact of CommitHorizon Parameter}
\includegraphics[width=.8\textwidth]{../ovenscheduling/images/commithorizon}
\end{frame}

\begin{frame}
\frametitle{Comparing Different Solvers}
\includegraphics[width=.8\textwidth]{../ovenscheduling/images/solvertype}
\end{frame}

\begin{frame}
\frametitle{Is the global solution really good?}
\begin{itemize}
\item We schedule with limited information
\item Hindsight is 20/20, we cannot expect best possible solution from partial information
\item Process Challenge: Can we improve data visibility?
\item Demand is variable over time, no steady-state solution
\item Modelling Challenge: Can we define a short-term objective that produces better long-term solutions?
\item Algorithm Challenge: Can we solve the global problem to optimality?
\begin{itemize}
\item Assumes "a priori" visibility of data
\item This would provide a lower bound 
\item But we need optimality to use as bound
\end{itemize}
\end{itemize}
\end{frame}




\begin{frame}
\frametitle{Summary}
\begin{itemize}
\item Discussed a non-standard oven scheduling problem from industry
\item Models with decomposition of resource constraints
\item Good/very good short-term solutions
\item But is the overall schedule close to the global optimum?
\item In any case, industry partner was happy with solution and analysis
\end{itemize}
\end{frame}
