\documentclass[a4paper]{report}
\usepackage{booktabs}
\usepackage{longtable}
\usepackage{colortbl}
\usepackage{xcolor}
\usepackage{multirow}
\usepackage{hyperref}
\title{Results for Scheduling Benchmark Classes}
\author{Luis Quesada and Helmut Simonis}
\begin{document}
\maketitle
\begin{abstract}
this reposts lists results of the \emph{tbischeduling} tool for a number of existing benchmarks on scheduling related problems. THe results indicate that depending on the problem type, only a fraction of the benchmarks are solved to optimality, while good or reasonable results are obtained by CPOptimizer of IBM.
\end{abstract}

\tableofcontents

\chapter{Introduction}

The results are obtained by running the TestAll main routine for the different benchmark problems, selecting the necessary parameters and limits for each benchmark type.

The detailed execution time depends on many parameters that are not well controlled in the test environment, so the results should be considered with caution. Tests were run on a Windows 11 laptop using CPOptimizer 22.1.0, and CPSat 9.11, both using their Java API. 

\chapter{Overview}

The following tables compare the results of CPOptimizer and CPSat on a number of well-known benchmark problems. Note that the programs used are the generic solutions of the problems, there was no attempt to improve lower bounds, or add redundant constraints that would improve pruning for a specific problem class. As such, the results indicate "out-of-the-box"  performance of the solvers.

The comparison does not attempt to compare the solutions found to the best known results, it is only intended to compare the results of the solvers considered based on the same underlying data model, running on the same hardware, with the same time limit.


Table~\ref{tab:compareoss} shows the results for the Taillard open shop problem set. All problems are solved to optimality by both solvers, given 600 seconds timeout and 4 resp. 8 threads for the solver. The results are grouped by instances classes where $x/y$ indicate $x$ jobs on $y$ machines. We compare the time taken to find and prove the optimal solution based on the virtual best solution of the faster time of the solvers. For smaller instances, CPSat seems to be faster, but for larger instances CPOptimizer finds the solution more rapidly. Given that all results are obtained in atmost a few seconds on either solver, this does not seem to be a significant difference.

\begin{table}[htbp]
\caption{\label{tab:compareoss}Comparison of CPO and CPSat for Result Groups of Taillard Open Shop Problems}
\begin{tabular}{l*{10}{r}}\toprule
& \multicolumn{4}{c}{All Instances} & \multicolumn{2}{c}{Optimal Only} & \multicolumn{4}{c}{Non-Optimal Only}\\
 & \multicolumn{4}{c}{Optimal (\% of Instances)} & \multicolumn{2}{c}{Time (\% of VB)} & \multicolumn{2}{c}{Cost (\% of VB)} & \multicolumn{2}{c}{Bound (\% of VB)}\\
Group & Both & CPO & CPSat & Neither & CPO & CPSat & CPO &CPSat & CPO & CPSat\\ \midrule
4/4 & 100.00&  0.00&  0.00&  0.00& 639.09& 100.00& n/a& n/a& n/a& n/a\\
5/5 & 100.00&  0.00&  0.00&  0.00& 217.83& 100.00& n/a& n/a& n/a& n/a\\
7/7 & 100.00&  0.00&  0.00&  0.00& 206.25& 106.99& n/a& n/a& n/a& n/a\\
10/10 & 100.00&  0.00&  0.00&  0.00& 119.36& 150.15& n/a& n/a& n/a& n/a\\
15/15 & 100.00&  0.00&  0.00&  0.00& 100.00& 288.65& n/a& n/a& n/a& n/a\\
20/20 & 100.00&  0.00&  0.00&  0.00& 114.30& 207.41& n/a& n/a& n/a& n/a\\
\bottomrule
\end{tabular}
\end{table}


\begin{table}[htbp]
\caption{\label{tab:comparejss}Comparison of CPO and CPSat for Result Groups of Taillard JobShop Problems}
{\scriptsize
\begin{tabular}{lr*{10}{r}} \toprule
& &\multicolumn{4}{c}{All Instances} & \multicolumn{2}{c}{Optimal Only} & \multicolumn{4}{c}{Non Optimal Only}\\ 
& &\multicolumn{4}{c}{Optimal (\% of All Instances)} & \multicolumn{2}{c}{Time (\% of VB)} & \multicolumn{2}{c}{Cost (\% of VB)} & \multicolumn{2}{c}{Bound (\% of VB)}\\ 
Group & Nr & Both & CPO & CPSat & None & CPO & CPSat & CPO & CPSat & CPO & CPSat\\ \midrule
15/15 & 10 & 90.00 &  0.00 &  0.00 & 10.00 & 105.19 & 141.18 & 100.00 & 100.00 & 97.17 & 100.00 \\ 
20/15 & 10 & 20.00 &  0.00 &  0.00 & 80.00 & 267.27 & 263.20 & 100.99 & 100.05 & 98.50 & 99.93 \\ 
20/20 & 10 &  0.00 &  0.00 &  0.00 & 100.00 & n/a & n/a & 100.74 & 100.06 & 97.96 & 100.00 \\ 
30/15 & 10 & 10.00 &  0.00 & 10.00 & 80.00 & 174.32 & 100.00 & 100.18 & 100.49 & 99.87 & 100.00 \\ 
30/20 & 10 &  0.00 &  0.00 &  0.00 & 100.00 & n/a & n/a & 100.30 & 101.30 & 99.40 & 100.00 \\ 
50/15 & 10 & 100.00 &  0.00 &  0.00 &  0.00 & 100.00 & 685.09 & n/a & n/a & n/a & n/a \\ 
50/20 & 10 & 10.00 & 60.00 &  0.00 & 30.00 & 100.00 & 381.38 & 100.00 & 101.60 & 100.00 & 100.00 \\ 
100/20 & 10 & 10.00 & 90.00 &  0.00 &  0.00 & 100.00 & 416.13 & 100.00 & 101.73 & 100.00 & 66.81 \\ 
\bottomrule
\end{tabular}
}
\end{table}



Table~\ref{tab:comparejss} compares the results for the Taillard job shop problems. Only some of the problem groups are solved to optimality. For the 10/20 set, CPOptimizer proves optimality, while CPSat finds solutions which are very close to the optimal results. The bound results for 100/20 with CPSat are incorrect, and need to be recomputed.

\begin{table}[htbp]
\caption{\label{tab:comparefss}Comparison of CPO and CPSat for Result Groups of Taillard Flow Shop Problems}
\begin{tabular}{l*{10}{r}}\toprule
& \multicolumn{4}{c}{All Instances} & \multicolumn{2}{c}{Optimal Only} & \multicolumn{4}{c}{Non-Optimal Only}\\
 & \multicolumn{4}{c}{Optimal (\% of Instances)} & \multicolumn{2}{c}{Time (\% of VB)} & \multicolumn{2}{c}{Cost (\% of VB)} & \multicolumn{2}{c}{Bound (\% of VB)}\\
Group & Both & CPO & CPSat & Neither & CPO & CPSat & CPO &CPSat & CPO & CPSat\\ \midrule
20/5 & 100.00&  0.00&  0.00&  0.00& 101.36& 203.99& n/a& n/a& n/a& n/a\\
20/10 & 10.00& 10.00&  0.00& 80.00& 100.00& 415.20& 100.19& 100.54& 98.52& 100.00\\
20/20 &  0.00&  0.00&  0.00& 100.00& n/a& n/a& 100.19& 102.07& 96.83& 100.00\\
50/5 & 100.00&  0.00&  0.00&  0.00& 101.91& 317.33& n/a& n/a& n/a& n/a\\
50/10 &  0.00&  0.00&  0.00& 100.00& n/a& n/a& 100.00& 103.31& 99.49& 100.00\\
50/20 &  0.00&  0.00&  0.00& 100.00& n/a& n/a& 100.00& 105.18& 99.27& 100.00\\
100/5 &  0.00& 70.00&  0.00& 30.00& n/a& n/a& 100.00& 100.49& 99.98& 99.98\\
100/10 &  0.00&  0.00&  0.00& 100.00& n/a& n/a& 100.00& 107.11& 99.91& 100.00\\
100/20 &  0.00&  0.00&  0.00& 100.00& n/a& n/a& 100.00& 107.94& 99.76& 100.00\\
200/10 &  0.00&  0.00&  0.00& 100.00& n/a& n/a& 100.00& 108.32& 99.97& 100.00\\
200/20 &  0.00&  0.00&  0.00& 100.00& n/a& n/a& 100.00& 105.86& 99.98& 99.98\\
500/20 &  0.00&  0.00&  0.00& 100.00& n/a& n/a& 100.00& 106.46& 100.00& 52.45\\
\bottomrule
\end{tabular}
\end{table}



Table~\ref{tab:comparefss} shows the results for the Taillard flow shop problems. Instance sets 20/5 and 50/5 are solved to optimality by both solvers, CPOptimizer finds many optimal solutions for the 100/5 sets. There are a few optimal solutions for the 20/10 set as well. Execution times for the optimal solutions seems to be significantly higher for CPSat. Comparing the non-optimal solutions, CPOptimizer consistently finds better solutions, but on average, CPSat is within 10\% of the CPOptimizer result. For the bounds, CPSat often provide slightly better lower bounds, but CPOptimizer results are pretty close. Note that for the bounds, a higher value is better, so achieving 100\% of the virtual best bound is better than achieving 98\%.

Table~\ref{tab:comparepfss} compares the results for CPOptimizer for the regular flow shop and the permutation flow shop version on the same data. The model for the permutation flow shop is not available with CPSat. A number of instances are solved to optimality with the permutation flow shop variant, but these optimal solutions typically are not optimal for the unrestricted version. In general, it is much faster to find the optimal solution for the permutation flowshop version, and perhaps surprisingly, the results for the non-optimal instances are often superior for the permutation flowshop. The bounds for the permutation flowshop are stronger, but they are not valid bounds for the unrestricted version. 

\begin{table}[htbp]
\caption{\label{tab:comparepfss}Comparison of CPO for Result Groups of Taillard Flow Shop and Permutation FlowShop Problems}
\begin{tabular}{l*{10}{r}}\toprule
& \multicolumn{4}{c}{All Instances} & \multicolumn{2}{c}{Optimal Only} & \multicolumn{4}{c}{Non-Optimal Only}\\
 & \multicolumn{4}{c}{Optimal (\% of Instances)} & \multicolumn{2}{c}{Time (\% of VB)} & \multicolumn{2}{c}{Cost (\% of VB)} & \multicolumn{2}{c}{Bound (\% of VB)}\\
Group & Both & FSS & PFSS & Neither & FSS & PFSS & FSS &PFSS & FSS & PFSS\\ \midrule
20/5 & 100.00&  0.00&  0.00&  0.00& 188.74& 100.00& n/a& n/a& n/a& n/a\\
20/10 & 20.00&  0.00& 50.00& 30.00& 212.55& 100.00& 100.13& 101.25& 96.28& 99.90\\
20/20 &  0.00&  0.00&  0.00& 100.00& n/a& n/a& 100.00& 101.33& 98.44& 99.87\\
50/5 & 100.00&  0.00&  0.00&  0.00& 494.74& 107.07& n/a& n/a& n/a& n/a\\
50/10 &  0.00&  0.00& 10.00& 90.00& n/a& n/a& 100.97& 100.02& 99.44& 100.00\\
50/20 &  0.00&  0.00&  0.00& 100.00& n/a& n/a& 102.96& 100.00& 99.27& 100.00\\
100/5 & 70.00&  0.00& 30.00&  0.00& 910.13& 100.00& 100.22& 100.00& 99.83& 100.00\\
100/10 &  0.00&  0.00&  0.00& 100.00& n/a& n/a& 101.47& 100.00& 99.75& 100.00\\
100/20 &  0.00&  0.00&  0.00& 100.00& n/a& n/a& 104.64& 100.00& 99.53& 100.00\\
200/10 &  0.00&  0.00&  0.00& 100.00& n/a& n/a& 102.82& 100.00& 99.89& 100.00\\
200/20 &  0.00&  0.00&  0.00& 100.00& n/a& n/a& 105.11& 100.00& 99.62& 100.00\\
500/20 &  0.00&  0.00&  0.00& 100.00& n/a& n/a& 100.03& 100.70& 99.88& 100.00\\
\bottomrule
\end{tabular}
\end{table}



\begin{table}[htbp]
\caption{\label{tab:comparesalbp}Comparison of CPO and CPSat for Result Groups of SALBP-1 Problems}
\begin{tabular}{l*{10}{r}}\toprule
& \multicolumn{4}{c}{All Instances} & \multicolumn{2}{c}{Optimal Only} & \multicolumn{4}{c}{Non-Optimal Only}\\
 & \multicolumn{4}{c}{Optimal (\% of Instances)} & \multicolumn{2}{c}{Time (\% of VB)} & \multicolumn{2}{c}{Cost (\% of VB)} & \multicolumn{2}{c}{Bound (\% of VB)}\\
Group & Both & CPO & CPSat & Neither & CPO & CPSat & CPO &CPSat & CPO & CPSat\\ \midrule
20 & 99.62&  0.00&  0.38&  0.00& 1079.77& 134.05& 100.00& 100.00& 82.14& 100.00\\
50 & 43.81& 28.95& 15.81& 11.43& 327.51& 938.89& 100.16& 100.04& 95.96& 71.28\\
100 & n/a& n/a& n/a& n/a& n/a& n/a& n/a& n/a& n/a& n/a\\
1000 & n/a& n/a& n/a& n/a& n/a& n/a& n/a& n/a& n/a& n/a\\
\bottomrule
\end{tabular}
\end{table}


\begin{table}[htbp]
\caption{\label{tab:comparesalbpalternative}Comparison of CPO and CPSat for Result Groups of SALBP-1 Problems Alternative }
{\scriptsize
\begin{tabular}{lr*{10}{r}} \toprule
& &\multicolumn{4}{c}{All Instances} & \multicolumn{2}{c}{Optimal Only} & \multicolumn{4}{c}{Non Optimal Only}\\ 
& &\multicolumn{4}{c}{Optimal (\% of All Instances)} & \multicolumn{2}{c}{Time (\% of VB)} & \multicolumn{2}{c}{Cost (\% of VB)} & \multicolumn{2}{c}{Bound (\% of VB)}\\ 
Group & Nr & Both & CPO & CPSat & None & CPO & CPSat & CPO & CPSat & CPO & CPSat\\ \midrule
20 & 525 & 88.19 &  9.71 &  0.00 &  2.10 & 100.26 & 6264.35 & 100.00 & 100.00 & 99.87 & 92.41 \\ 
50 & 525 & 25.90 & 35.62 &  0.00 & 38.48 & 100.17 & 2231.63 & 100.00 & 100.75 & 99.73 & 99.62 \\ 
100 & 525 &  2.10 &  8.00 &  0.00 & 89.90 & 100.00 & 550.02 & 100.01 & 101.74 & 99.95 & 99.80 \\ 
1000 & 386 &  0.00 &  0.00 &  0.00 & 100.00 & n/a & n/a & 100.00 & 100.98 & 100.00 & 99.89 \\ 
\bottomrule
\end{tabular}
}
\end{table}



The results for SALBP-1 in Table~\ref{tab:comparesalbp} were obtained with a 30 second timeout. All of the 20 task instances were solved to optimality (some only with CPSat), while 88\% of the 50 task instances were solved to optimality, but only 43\% were solved by both. The time taken to find the common optimal solutions varies significantly, CPSat seems on average faster on the small instances, while CPOptimizer is faster on a larger instances, but this does not hold for all instances. For the non-optimal solutions, solution quality seems very evenly balanced.

The results for the test scheduling case study are shown in Table~\ref{tab:compareTest}.
Very likely the 30 second timeout is too small for the large problem instances. Optimal solutions are found by both solvers for the smaller instance sizes, CPO does find some optimal solutions even for large problem sizes. Solution quality for up to 100 tasks is very close, while results for CPSat on the 500 task problems is disappointing compared to the CPO results. Already for 100 tasks, the time needed to find the optimal solutions is much higher for CPSat, the results may improve if more time is given for both solvers.

\begin{table}[htbp]
\caption{\label{tab:compareTest}Comparison of CPO and CPSat for Result Groups of Test Scheduling Problems}
{\scriptsize
\begin{tabular}{lr*{10}{r}} \toprule
& &\multicolumn{4}{c}{All Instances} & \multicolumn{2}{c}{Optimal Only} & \multicolumn{4}{c}{Non Optimal Only}\\ 
& &\multicolumn{4}{c}{Optimal (\% of All Instances)} & \multicolumn{2}{c}{Time (\% of VB)} & \multicolumn{2}{c}{Cost (\% of VB)} & \multicolumn{2}{c}{Bound (\% of VB)}\\ 
Group & Nr & Both & CPO & CPSat & None & CPO & CPSat & CPO & CPSat & CPO & CPSat\\ \midrule
20/10/3 & 20 & 90.00 &  0.00 &  5.00 &  5.00 & 393.93 & 133.28 & 100.00 & 100.00 & 92.00 & 100.00 \\ 
20/10/5 & 20 & 100.00 &  0.00 &  0.00 &  0.00 & 294.88 & 117.68 & n/a & n/a & n/a & n/a \\ 
20/10/10 & 20 & 95.00 &  0.00 &  0.00 &  5.00 & 501.45 & 107.85 & 100.00 & 100.00 & 100.00 & 100.00 \\ 
30/10/3 & 20 & 95.00 &  0.00 &  0.00 &  5.00 & 100.48 & 183.16 & 100.08 & 100.00 & 100.00 & 100.00 \\ 
30/10/5 & 20 & 90.00 &  0.00 &  5.00 &  5.00 & 100.00 & 205.76 & 100.00 & 100.00 & 89.37 & 100.00 \\ 
30/10/10 & 20 & 70.00 &  0.00 & 15.00 & 15.00 & 382.57 & 104.61 & 100.01 & 100.00 & 93.29 & 100.00 \\ 
30/20/3 & 20 & 95.00 &  0.00 &  5.00 &  0.00 & 205.53 & 144.45 & 100.00 & 100.00 & 97.45 & 100.00 \\ 
30/20/5 & 20 & 90.00 &  0.00 & 10.00 &  0.00 & 229.19 & 132.32 & 100.00 & 100.00 & 88.36 & 100.00 \\ 
30/20/10 & 20 & 60.00 &  0.00 & 30.00 & 10.00 & 439.18 & 104.94 & 100.00 & 100.00 & 89.44 & 100.00 \\ 
40/10/3 & 20 & 85.00 &  0.00 &  0.00 & 15.00 & 104.79 & 280.75 & 100.02 & 100.02 & 100.00 & 100.00 \\ 
40/10/5 & 20 & 90.00 &  0.00 &  5.00 &  5.00 & 194.05 & 150.11 & 100.00 & 100.00 & 98.22 & 100.00 \\ 
40/10/10 & 20 & 70.00 &  0.00 & 20.00 & 10.00 & 231.87 & 109.84 & 100.00 & 100.00 & 93.80 & 100.00 \\ 
40/20/3 & 20 & 100.00 &  0.00 &  0.00 &  0.00 & 100.19 & 172.54 & n/a & n/a & n/a & n/a \\ 
40/20/5 & 20 & 75.00 &  0.00 & 20.00 &  5.00 & 154.81 & 109.98 & 100.00 & 100.00 & 95.30 & 100.00 \\ 
40/20/10 & 20 & 50.00 &  0.00 & 35.00 & 15.00 & 285.15 & 106.26 & 100.00 & 100.00 & 92.51 & 100.00 \\ 
50/10/3 & 20 & 85.00 &  0.00 &  0.00 & 15.00 & 466.49 & 100.00 & 100.00 & 100.00 & 100.00 & 100.00 \\ 
50/10/5 & 20 & 45.00 &  0.00 & 10.00 & 45.00 & 682.15 & 100.00 & 100.00 & 100.00 & 99.31 & 100.00 \\ 
50/10/10 & 20 & 15.00 &  0.00 & 30.00 & 55.00 & 1381.55 & 100.00 & 100.00 & 100.00 & 98.26 & 100.00 \\ 
50/20/3 & 20 & 90.00 &  0.00 & 10.00 &  0.00 & 726.25 & 100.00 & 100.00 & 100.00 & 94.42 & 100.00 \\ 
50/20/5 & 20 & 60.00 &  0.00 &  0.00 & 40.00 & 573.69 & 100.00 & 100.00 & 100.00 & 100.00 & 100.00 \\ 
50/20/10 & 20 & 35.00 &  0.00 & 10.00 & 55.00 & 671.58 & 100.00 & 100.00 & 100.00 & 94.08 & 100.00 \\ 
100/10/3 & 20 & 90.00 &  0.00 &  0.00 & 10.00 & 100.00 & 1998.93 & 100.00 & 100.00 & 100.00 & 100.00 \\ 
100/10/5 & 20 & 45.00 &  5.00 &  0.00 & 50.00 & 100.00 & 1105.69 & 100.00 & 100.14 & 100.00 & 100.00 \\ 
100/10/10 & 20 &  0.00 &  0.00 &  0.00 & 100.00 & n/a & n/a & 100.00 & 100.23 & 100.00 & 100.00 \\ 
100/20/3 & 20 & 85.00 &  0.00 &  0.00 & 15.00 & 100.00 & 1552.40 & 100.00 & 100.12 & 100.00 & 100.00 \\ 
100/20/5 & 20 & 35.00 & 15.00 &  0.00 & 50.00 & 100.00 & 2147.25 & 100.00 & 100.75 & 100.00 & 100.00 \\ 
100/20/10 & 20 &  5.00 &  0.00 &  0.00 & 95.00 & 100.00 & 903.53 & 100.00 & 100.83 & 100.00 & 100.00 \\ 
100/50/3 & 20 & 80.00 & 10.00 &  0.00 & 10.00 & 100.00 & 1410.06 & 100.00 & 100.78 & 100.00 & 100.00 \\ 
100/50/5 & 20 & 45.00 & 10.00 &  0.00 & 45.00 & 100.00 & 805.32 & 100.00 & 100.11 & 100.00 & 100.00 \\ 
100/50/10 & 20 & 10.00 &  5.00 &  0.00 & 85.00 & 100.00 & 1260.03 & 100.00 & 100.63 & 100.00 & 100.00 \\ 
500/10/3 & 20 &  0.00 &  5.00 &  0.00 & 95.00 & n/a & n/a & 100.00 & 227.86 & 16.72 & 100.00 \\ 
500/10/5 & 20 &  0.00 &  0.00 &  0.00 & 100.00 & n/a & n/a & 100.00 & 226.47 &  2.06 & 100.00 \\ 
500/10/10 & 20 &  0.00 &  0.00 &  0.00 & 100.00 & n/a & n/a & 100.00 & 226.89 & 27.57 & 100.00 \\ 
500/20/3 & 20 &  0.00 & 20.00 &  0.00 & 80.00 & n/a & n/a & 100.00 & 224.93 & 22.74 & 100.00 \\ 
500/20/5 & 20 &  0.00 &  0.00 &  0.00 & 100.00 & n/a & n/a & 100.00 & 230.90 &  2.03 & 100.00 \\ 
500/20/10 & 20 &  0.00 &  0.00 &  0.00 & 100.00 & n/a & n/a & 100.00 & 235.16 &  2.25 & 100.00 \\ 
500/50/3 & 20 &  0.00 &  5.00 &  0.00 & 95.00 & n/a & n/a & 100.00 & 234.08 &  7.58 & 100.00 \\ 
500/50/5 & 20 &  0.00 &  0.00 &  0.00 & 100.00 & n/a & n/a & 100.00 & 240.48 &  2.09 & 100.00 \\ 
500/50/10 & 20 &  0.00 &  0.00 &  0.00 & 100.00 & n/a & n/a & 100.00 & 241.28 &  2.31 & 100.00 \\ 
500/100/3 & 20 &  0.00 & 10.00 &  0.00 & 90.00 & n/a & n/a & 100.00 & 247.90 & 12.28 & 100.00 \\ 
500/100/5 & 20 &  0.00 &  0.00 &  0.00 & 100.00 & n/a & n/a & 100.00 & 251.03 &  2.11 & 100.00 \\ 
500/100/10 & 20 &  0.00 &  0.00 &  0.00 & 100.00 & n/a & n/a & 100.00 & 239.37 &  2.24 & 100.00 \\ 
\bottomrule
\end{tabular}
}
\end{table}



Table~\ref{tab:compareTrans} compares CPO and CPSat for the Hybrid flexible flow shop problem of the factory design case study. For the smaller problem sizes, up to 40 jobs, both system offer comparable solution quality, with significantly high run times for CPSat. Starting from the 50 job problem instances, but especially for the large 300 and 400 job problems, the solution quality of CPO is much better. Optimal solutions are found by both system up to size 30, CPSat finds more optimal solutions for the smaller problems, and CPO finds more optimal solution for larger instances (up to size 50). 

While the results for CPO are clearly superior to the ones for CPSat for the 500 task problem instances, the given lower bound is very poor for CPO. 

\begin{table}[htbp]
\caption{\label{tab:compareTrans}Comparison of CPO and CPSat for Result Groups of Factory Design Problems}
{\scriptsize
\begin{tabular}{lr*{10}{r}} \toprule
& &\multicolumn{4}{c}{All Instances} & \multicolumn{2}{c}{Optimal Only} & \multicolumn{4}{c}{Non Optimal Only}\\ 
& &\multicolumn{4}{c}{Optimal (\% of All Instances)} & \multicolumn{2}{c}{Time (\% of VB)} & \multicolumn{2}{c}{Cost (\% of VB)} & \multicolumn{2}{c}{Bound (\% of VB)}\\ 
Group & Nr & Both & CPO & CPSat & None & CPO & CPSat & CPO & CPSat & CPO & CPSat\\ \midrule
20 & 25 & 76.00 &  0.00 & 20.00 &  4.00 & 100.00 & 580.71 & 100.00 & 100.00 & 96.52 & 100.00 \\ 
25 & 25 & 80.00 &  0.00 &  8.00 & 12.00 & 101.65 & 238.02 & 100.00 & 100.37 & 97.67 & 100.00 \\ 
30 & 25 & 60.00 &  0.00 &  4.00 & 36.00 & 100.35 & 264.69 & 100.18 & 101.05 & 100.00 & 100.00 \\ 
40 & 25 &  4.00 & 16.00 &  0.00 & 80.00 & 100.00 & 2554.03 & 100.00 & 104.68 & 100.00 & 100.00 \\ 
50 & 25 &  0.00 &  4.00 &  0.00 & 96.00 & n/a & n/a & 100.00 & 107.87 & 100.00 & 100.00 \\ 
100 & 25 &  0.00 &  0.00 &  0.00 & 100.00 & n/a & n/a & 100.00 & 120.43 & 100.00 & 100.00 \\ 
200 & 25 &  0.00 &  0.00 &  0.00 & 100.00 & n/a & n/a & 100.00 & 188.60 & 100.00 & 100.00 \\ 
300 & 24 &  0.00 &  0.00 &  0.00 & 100.00 & n/a & n/a & 100.00 & 263.22 & 100.00 & 100.00 \\ 
400 & 25 &  0.00 &  0.00 &  0.00 & 100.00 & n/a & n/a & 100.00 & 246.34 & 100.00 & 100.00 \\ 
\bottomrule
\end{tabular}
}
\end{table}



Note that there is a single 300 job instance 300\_2 for which CPSat did not find any solution within the timeout, so only 24 instances are compared.

The results for the single mode RCPSP problems do not use consistent settings, the smaller (j30 and j60) instances are run with a 600 seconds timeout, the j90 instances with a 30 seconds, and the j120 instances with a 60 second timeout. These larger instances should be rerun with a 600 seconds timeout, but that will require several days of CPU time.  

%\input{comparercpspj30} 
%\begin{table}[htbp]
\caption{\label{tab:comparercpspj60}Comparison of CPO and CPSat for Results of RCPSP}
{\scriptsize
\begin{tabular}{lr*{10}{r}} \toprule
& &\multicolumn{4}{c}{All Instances} & \multicolumn{2}{c}{Optimal Only} & \multicolumn{4}{c}{Non Optimal Only}\\ 
& &\multicolumn{4}{c}{Optimal (\% of All Instances)} & \multicolumn{2}{c}{Time (\% of VB)} & \multicolumn{2}{c}{Cost (\% of VB)} & \multicolumn{2}{c}{Bound (\% of VB)}\\ 
Group & Nr & Both & CPO & CPSat & None & CPO & CPSat & CPO & CPSat & CPO & CPSat\\ \midrule
30 & 480 & 100.00 &  0.00 &  0.00 &  0.00 & 200.13 & 166.70 & n/a & n/a & n/a & n/a \\ 
60 & 348 & 90.80 &  0.00 &  0.86 &  8.33 & 109.73 & 410.01 & 100.17 & 100.46 & 98.68 & 98.87 \\ 
\bottomrule
\end{tabular}
}
\end{table}

 
%\begin{table}[htbp]
\caption{\label{tab:comparercpspj90}Comparison of CPO and CPSat for Results of RCPSP}
{\scriptsize
\begin{tabular}{lr*{10}{r}} \toprule
& &\multicolumn{4}{c}{All Instances} & \multicolumn{2}{c}{Optimal Only} & \multicolumn{4}{c}{Non Optimal Only}\\ 
& &\multicolumn{4}{c}{Optimal (\% of All Instances)} & \multicolumn{2}{c}{Time (\% of VB)} & \multicolumn{2}{c}{Cost (\% of VB)} & \multicolumn{2}{c}{Bound (\% of VB)}\\ 
Group & Nr & Both & CPO & CPSat & None & CPO & CPSat & CPO & CPSat & CPO & CPSat\\ \midrule
30 & 480 & 100.00 &  0.00 &  0.00 &  0.00 & 200.13 & 166.70 & n/a & n/a & n/a & n/a \\ 
60 & 480 & 89.58 &  0.00 &  1.04 &  9.38 & 109.84 & 358.74 & 100.20 & 100.45 & 98.89 & 98.66 \\ 
90 & 480 & 80.21 &  0.00 &  0.63 & 19.17 & 108.11 & 332.72 & 100.34 & 101.13 & 99.76 & 99.19 \\ 
\bottomrule
\end{tabular}
}
\end{table}

 
\input{comparercpspj120} 

Overall, the results are similar to other problem sets. For all instance sets, both CPO and SPSat find and prove optimal solutions, 100\% for J30 instances, 90\% for j60 instances, 80\% for j90 instances, but only 45\% for j120 instances. While the times for the j30 instances are comparable, the overall time for the larger instances is significantly higher for CPSat. Detailed results shows that many instances are solved by both solvers in less than a second, the differences arise from relatively few instances that take much longer in CPSat, while there are only few instances where CPO takes longer than CPSat.

For all non-optimal solutions, the solution quality obtained by both solvers within the timeout is nearly identical, with a very slight advantage for CPO for both solutions found and lower bound calculated.

\clearpage
\chapter{Taillard Open Shop Problems}
All problems are solved to optimality, possibly due to their small to moderate size.

\section{Results for CPOptimizer}

\begin{longtable}{lrrlrrrr}
\caption{Results for Taillard OpenShop (CPOptimizer) (60 Instances)}\\\toprule
Name & \shortstack{Nr\\Jobs} & \shortstack{Nr\\Machines} & Status & Time & Makespan & Bound & \shortstack{Gap\\Percent}\\ \midrule
\endhead
\bottomrule
\endfoot
tai10 10 0.json & 10 & 10 & Optimal &  0.45 & 637 & 637.00 &  0.00\\
tai10 10 1.json & 10 & 10 & Optimal &  0.06 & 588 & 588.00 &  0.00\\
tai10 10 2.json & 10 & 10 & Optimal &  0.27 & 598 & 598.00 &  0.00\\
tai10 10 3.json & 10 & 10 & Optimal &  0.05 & 577 & 577.00 &  0.00\\
tai10 10 4.json & 10 & 10 & Optimal &  0.05 & 640 & 640.00 &  0.00\\
tai10 10 5.json & 10 & 10 & Optimal &  0.04 & 538 & 538.00 &  0.00\\
tai10 10 6.json & 10 & 10 & Optimal &  0.06 & 616 & 616.00 &  0.00\\
tai10 10 7.json & 10 & 10 & Optimal &  0.11 & 595 & 595.00 &  0.00\\
tai10 10 8.json & 10 & 10 & Optimal &  0.05 & 595 & 595.00 &  0.00\\
tai10 10 9.json & 10 & 10 & Optimal &  0.08 & 596 & 596.00 &  0.00\\
tai15 15 0.json & 15 & 15 & Optimal &  0.11 & 937 & 937.00 &  0.00\\
tai15 15 1.json & 15 & 15 & Optimal &  0.11 & 918 & 918.00 &  0.00\\
tai15 15 2.json & 15 & 15 & Optimal &  0.08 & 871 & 871.00 &  0.00\\
tai15 15 3.json & 15 & 15 & Optimal &  0.13 & 934 & 934.00 &  0.00\\
tai15 15 4.json & 15 & 15 & Optimal &  0.09 & 946 & 946.00 &  0.00\\
tai15 15 5.json & 15 & 15 & Optimal &  0.08 & 933 & 933.00 &  0.00\\
tai15 15 6.json & 15 & 15 & Optimal &  0.16 & 891 & 891.00 &  0.00\\
tai15 15 7.json & 15 & 15 & Optimal &  0.13 & 893 & 893.00 &  0.00\\
tai15 15 8.json & 15 & 15 & Optimal &  0.28 & 899 & 899.00 &  0.00\\
tai15 15 9.json & 15 & 15 & Optimal &  0.17 & 902 & 902.00 &  0.00\\
tai20 20 0.json & 20 & 20 & Optimal &  0.35 & 1155 & 1155.00 &  0.00\\
tai20 20 1.json & 20 & 20 & Optimal &  1.00 & 1241 & 1241.00 &  0.00\\
tai20 20 2.json & 20 & 20 & Optimal &  0.56 & 1257 & 1257.00 &  0.00\\
tai20 20 3.json & 20 & 20 & Optimal &  0.25 & 1248 & 1248.00 &  0.00\\
tai20 20 4.json & 20 & 20 & Optimal &  0.19 & 1256 & 1256.00 &  0.00\\
tai20 20 5.json & 20 & 20 & Optimal &  0.16 & 1204 & 1204.00 &  0.00\\
tai20 20 6.json & 20 & 20 & Optimal &  0.66 & 1294 & 1294.00 &  0.00\\
tai20 20 7.json & 20 & 20 & Optimal &  1.18 & 1169 & 1169.00 &  0.00\\
tai20 20 8.json & 20 & 20 & Optimal &  0.17 & 1289 & 1289.00 &  0.00\\
tai20 20 9.json & 20 & 20 & Optimal &  0.17 & 1241 & 1241.00 &  0.00\\
tai4 4 0.json & 4 & 4 & Optimal &  0.13 & 193 & 193.00 &  0.00\\
tai4 4 1.json & 4 & 4 & Optimal &  0.11 & 236 & 236.00 &  0.00\\
tai4 4 2.json & 4 & 4 & Optimal &  0.08 & 271 & 271.00 &  0.00\\
tai4 4 3.json & 4 & 4 & Optimal &  0.15 & 250 & 250.00 &  0.00\\
tai4 4 4.json & 4 & 4 & Optimal &  0.17 & 295 & 295.00 &  0.00\\
tai4 4 5.json & 4 & 4 & Optimal &  0.05 & 189 & 189.00 &  0.00\\
tai4 4 6.json & 4 & 4 & Optimal &  0.10 & 201 & 201.00 &  0.00\\
tai4 4 7.json & 4 & 4 & Optimal &  0.05 & 217 & 217.00 &  0.00\\
tai4 4 8.json & 4 & 4 & Optimal &  0.13 & 261 & 261.00 &  0.00\\
tai4 4 9.json & 4 & 4 & Optimal &  0.12 & 217 & 217.00 &  0.00\\
tai5 5 0.json & 5 & 5 & Optimal &  0.18 & 300 & 300.00 &  0.00\\
tai5 5 1.json & 5 & 5 & Optimal &  0.16 & 262 & 262.00 &  0.00\\
tai5 5 2.json & 5 & 5 & Optimal &  0.20 & 323 & 323.00 &  0.00\\
tai5 5 3.json & 5 & 5 & Optimal &  0.17 & 310 & 310.00 &  0.00\\
tai5 5 4.json & 5 & 5 & Optimal &  0.27 & 326 & 326.00 &  0.00\\
tai5 5 5.json & 5 & 5 & Optimal &  0.16 & 312 & 312.00 &  0.00\\
tai5 5 6.json & 5 & 5 & Optimal &  0.21 & 303 & 303.00 &  0.00\\
tai5 5 7.json & 5 & 5 & Optimal &  0.25 & 300 & 300.00 &  0.00\\
tai5 5 8.json & 5 & 5 & Optimal &  0.17 & 353 & 353.00 &  0.00\\
tai5 5 9.json & 5 & 5 & Optimal &  0.25 & 326 & 326.00 &  0.00\\
tai7 7 0.json & 7 & 7 & Optimal &  0.03 & 435 & 435.00 &  0.00\\
tai7 7 1.json & 7 & 7 & Optimal &  0.12 & 443 & 443.00 &  0.00\\
tai7 7 2.json & 7 & 7 & Optimal &  0.31 & 468 & 468.00 &  0.00\\
tai7 7 3.json & 7 & 7 & Optimal &  0.03 & 463 & 463.00 &  0.00\\
tai7 7 4.json & 7 & 7 & Optimal &  0.03 & 416 & 416.00 &  0.00\\
tai7 7 5.json & 7 & 7 & Optimal &  0.80 & 451 & 451.00 &  0.00\\
tai7 7 6.json & 7 & 7 & Optimal &  1.10 & 422 & 422.00 &  0.00\\
tai7 7 7.json & 7 & 7 & Optimal &  0.05 & 424 & 424.00 &  0.00\\
tai7 7 8.json & 7 & 7 & Optimal &  0.09 & 458 & 458.00 &  0.00\\
tai7 7 9.json & 7 & 7 & Optimal &  0.06 & 398 & 398.00 &  0.00\\
\end{longtable}



\section{Results for CPSat}

\begin{longtable}{lrrlrrrr}
\caption{Results for Taillard OpenShop (CPSat) (60 Instances)}\\\toprule
Name & \shortstack{Nr\\Jobs} & \shortstack{Nr\\Machines} & Status & Time & Makespan & Bound & \shortstack{Gap\\Percent}\\ \midrule
\endhead
\bottomrule
\endfoot
tai10 10 0.json & 10 & 10 & Optimal &  0.37 & 637 &  0.00 &  0.00\\
tai10 10 1.json & 10 & 10 & Optimal &  0.07 & 588 &  0.00 &  0.00\\
tai10 10 2.json & 10 & 10 & Optimal &  0.16 & 598 &  0.00 &  0.00\\
tai10 10 3.json & 10 & 10 & Optimal &  0.09 & 577 &  0.00 &  0.00\\
tai10 10 4.json & 10 & 10 & Optimal &  0.20 & 640 &  0.00 &  0.00\\
tai10 10 5.json & 10 & 10 & Optimal &  0.13 & 538 &  0.00 &  0.00\\
tai10 10 6.json & 10 & 10 & Optimal &  0.10 & 616 &  0.00 &  0.00\\
tai10 10 7.json & 10 & 10 & Optimal &  0.17 & 595 &  0.00 &  0.00\\
tai10 10 8.json & 10 & 10 & Optimal &  0.11 & 595 &  0.00 &  0.00\\
tai10 10 9.json & 10 & 10 & Optimal &  0.14 & 596 &  0.00 &  0.00\\
tai15 15 0.json & 15 & 15 & Optimal &  0.31 & 937 &  0.00 &  0.00\\
tai15 15 1.json & 15 & 15 & Optimal &  0.45 & 918 &  0.00 &  0.00\\
tai15 15 2.json & 15 & 15 & Optimal &  0.17 & 871 &  0.00 &  0.00\\
tai15 15 3.json & 15 & 15 & Optimal &  0.17 & 934 &  0.00 &  0.00\\
tai15 15 4.json & 15 & 15 & Optimal &  0.27 & 946 &  0.00 &  0.00\\
tai15 15 5.json & 15 & 15 & Optimal &  0.25 & 933 &  0.00 &  0.00\\
tai15 15 6.json & 15 & 15 & Optimal &  0.25 & 891 &  0.00 &  0.00\\
tai15 15 7.json & 15 & 15 & Optimal &  0.32 & 893 &  0.00 &  0.00\\
tai15 15 8.json & 15 & 15 & Optimal &  1.27 & 899 &  0.00 &  0.00\\
tai15 15 9.json & 15 & 15 & Optimal &  0.38 & 902 &  0.00 &  0.00\\
tai20 20 0.json & 20 & 20 & Optimal &  1.01 & 1155 &  0.00 &  0.00\\
tai20 20 1.json & 20 & 20 & Optimal &  2.44 & 1241 &  0.00 &  0.00\\
tai20 20 2.json & 20 & 20 & Optimal &  0.12 & 1257 &  0.00 &  0.00\\
tai20 20 3.json & 20 & 20 & Optimal &  0.35 & 1248 &  0.00 &  0.00\\
tai20 20 4.json & 20 & 20 & Optimal &  0.40 & 1256 &  0.00 &  0.00\\
tai20 20 5.json & 20 & 20 & Optimal &  0.62 & 1204 &  0.00 &  0.00\\
tai20 20 6.json & 20 & 20 & Optimal &  0.52 & 1294 &  0.00 &  0.00\\
tai20 20 7.json & 20 & 20 & Optimal &  2.13 & 1169 &  0.00 &  0.00\\
tai20 20 8.json & 20 & 20 & Optimal &  0.26 & 1289 &  0.00 &  0.00\\
tai20 20 9.json & 20 & 20 & Optimal &  0.65 & 1241 &  0.00 &  0.00\\
tai4 4 0.json & 4 & 4 & Optimal &  0.02 & 193 &  0.00 &  0.00\\
tai4 4 1.json & 4 & 4 & Optimal &  0.03 & 236 &  0.00 &  0.00\\
tai4 4 2.json & 4 & 4 & Optimal &  0.01 & 271 &  0.00 &  0.00\\
tai4 4 3.json & 4 & 4 & Optimal &  0.01 & 250 &  0.00 &  0.00\\
tai4 4 4.json & 4 & 4 & Optimal &  0.03 & 295 &  0.00 &  0.00\\
tai4 4 5.json & 4 & 4 & Optimal &  0.01 & 189 &  0.00 &  0.00\\
tai4 4 6.json & 4 & 4 & Optimal &  0.01 & 201 &  0.00 &  0.00\\
tai4 4 7.json & 4 & 4 & Optimal &  0.01 & 217 &  0.00 &  0.00\\
tai4 4 8.json & 4 & 4 & Optimal &  0.01 & 261 &  0.00 &  0.00\\
tai4 4 9.json & 4 & 4 & Optimal &  0.01 & 217 &  0.00 &  0.00\\
tai5 5 0.json & 5 & 5 & Optimal &  0.06 & 300 &  0.00 &  0.00\\
tai5 5 1.json & 5 & 5 & Optimal &  0.04 & 262 &  0.00 &  0.00\\
tai5 5 2.json & 5 & 5 & Optimal &  0.12 & 323 &  0.00 &  0.00\\
tai5 5 3.json & 5 & 5 & Optimal &  0.07 & 310 &  0.00 &  0.00\\
tai5 5 4.json & 5 & 5 & Optimal &  0.16 & 326 &  0.00 &  0.00\\
tai5 5 5.json & 5 & 5 & Optimal &  0.07 & 312 &  0.00 &  0.00\\
tai5 5 6.json & 5 & 5 & Optimal &  0.09 & 303 &  0.00 &  0.00\\
tai5 5 7.json & 5 & 5 & Optimal &  0.11 & 300 &  0.00 &  0.00\\
tai5 5 8.json & 5 & 5 & Optimal &  0.11 & 353 &  0.00 &  0.00\\
tai5 5 9.json & 5 & 5 & Optimal &  0.11 & 326 &  0.00 &  0.00\\
tai7 7 0.json & 7 & 7 & Optimal &  0.06 & 435 &  0.00 &  0.00\\
tai7 7 1.json & 7 & 7 & Optimal &  0.11 & 443 &  0.00 &  0.00\\
tai7 7 2.json & 7 & 7 & Optimal &  0.15 & 468 &  0.00 &  0.00\\
tai7 7 3.json & 7 & 7 & Optimal &  0.06 & 463 &  0.00 &  0.00\\
tai7 7 4.json & 7 & 7 & Optimal &  0.05 & 416 &  0.00 &  0.00\\
tai7 7 5.json & 7 & 7 & Optimal &  0.48 & 451 &  0.00 &  0.00\\
tai7 7 6.json & 7 & 7 & Optimal &  0.29 & 422 &  0.00 &  0.00\\
tai7 7 7.json & 7 & 7 & Optimal &  0.04 & 424 &  0.00 &  0.00\\
tai7 7 8.json & 7 & 7 & Optimal &  0.05 & 458 &  0.00 &  0.00\\
tai7 7 9.json & 7 & 7 & Optimal &  0.08 & 398 &  0.00 &  0.00\\
\end{longtable}



\clearpage
\chapter{Taillard Job Shop Problems}

The results are rather confusing, as some smaller problems cannot be solved to optimality, while complete groups of larger instances can. The number of jobs clearly is not the only indicator of difficulty of these problems.

\section{Results for CPOptimizer}

\begin{longtable}{lrrlrrrr}
\caption{Results for Taillard JobShop (80 Instances)}\\\toprule
Name & \shortstack{Nr\\Jobs} & \shortstack{Nr\\Machines} & Status & Time & Makespan & Bound & \shortstack{Gap\\Percent}\\ \midrule
\endhead
\bottomrule
\endfoot
tai100 20 0.json & 100 & 20 & Optimal & 266.17 & 5464 & 5464.00 &  0.00\\
tai100 20 1.json & 100 & 20 & Optimal & 31.20 & 5181 & 5181.00 &  0.00\\
tai100 20 2.json & 100 & 20 & Optimal & 38.67 & 5568 & 5568.00 &  0.00\\
tai100 20 3.json & 100 & 20 & Optimal & 177.17 & 5339 & 5339.00 &  0.00\\
tai100 20 4.json & 100 & 20 & Solution & 300.04 & 5412 & 5392.00 &  0.37\\
tai100 20 5.json & 100 & 20 & Optimal & 172.70 & 5342 & 5342.00 &  0.00\\
tai100 20 6.json & 100 & 20 & Optimal & 298.02 & 5436 & 5436.00 &  0.00\\
tai100 20 7.json & 100 & 20 & Optimal & 111.51 & 5394 & 5394.00 &  0.00\\
tai100 20 8.json & 100 & 20 & Optimal & 86.84 & 5358 & 5358.00 &  0.00\\
tai100 20 9.json & 100 & 20 & Optimal & 188.94 & 5183 & 5183.00 &  0.00\\
tai15 15 0.json & 15 & 15 & Optimal & 12.44 & 1231 & 1231.00 &  0.00\\
tai15 15 1.json & 15 & 15 & Optimal & 35.14 & 1244 & 1244.00 &  0.00\\
tai15 15 2.json & 15 & 15 & Optimal & 17.26 & 1218 & 1218.00 &  0.00\\
tai15 15 3.json & 15 & 15 & Optimal & 30.94 & 1175 & 1175.00 &  0.00\\
tai15 15 4.json & 15 & 15 & Optimal & 134.85 & 1224 & 1224.00 &  0.00\\
tai15 15 5.json & 15 & 15 & Solution & 300.02 & 1243 & 1183.00 &  4.83\\
tai15 15 6.json & 15 & 15 & Optimal & 104.11 & 1227 & 1227.00 &  0.00\\
tai15 15 7.json & 15 & 15 & Optimal & 102.79 & 1217 & 1217.00 &  0.00\\
tai15 15 8.json & 15 & 15 & Optimal & 133.01 & 1274 & 1274.00 &  0.00\\
tai15 15 9.json & 15 & 15 & Optimal & 32.78 & 1241 & 1241.00 &  0.00\\
tai20 15 0.json & 20 & 15 & Solution & 300.02 & 1424 & 1274.00 & 10.53\\
tai20 15 1.json & 20 & 15 & Solution & 300.01 & 1378 & 1328.00 &  3.63\\
tai20 15 2.json & 20 & 15 & Solution & 300.02 & 1398 & 1243.00 & 11.09\\
tai20 15 3.json & 20 & 15 & Optimal & 17.73 & 1345 & 1345.00 &  0.00\\
tai20 15 4.json & 20 & 15 & Solution & 300.02 & 1374 & 1270.00 &  7.57\\
tai20 15 5.json & 20 & 15 & Solution & 300.02 & 1389 & 1268.00 &  8.71\\
tai20 15 6.json & 20 & 15 & Optimal & 76.19 & 1462 & 1462.00 &  0.00\\
tai20 15 7.json & 20 & 15 & Solution & 300.02 & 1427 & 1358.00 &  4.84\\
tai20 15 8.json & 20 & 15 & Solution & 300.02 & 1369 & 1258.00 &  8.11\\
tai20 15 9.json & 20 & 15 & Solution & 300.02 & 1406 & 1289.00 &  8.32\\
tai20 20 0.json & 20 & 20 & Solution & 300.02 & 1688 & 1514.00 & 10.31\\
tai20 20 1.json & 20 & 20 & Solution & 300.02 & 1640 & 1454.00 & 11.34\\
tai20 20 2.json & 20 & 20 & Solution & 300.02 & 1585 & 1456.00 &  8.14\\
tai20 20 3.json & 20 & 20 & Solution & 300.01 & 1656 & 1583.00 &  4.41\\
tai20 20 4.json & 20 & 20 & Solution & 300.02 & 1642 & 1474.00 & 10.23\\
tai20 20 5.json & 20 & 20 & Solution & 300.02 & 1663 & 1490.00 & 10.40\\
tai20 20 6.json & 20 & 20 & Solution & 300.02 & 1724 & 1605.00 &  6.90\\
tai20 20 7.json & 20 & 20 & Solution & 300.02 & 1629 & 1564.00 &  3.99\\
tai20 20 8.json & 20 & 20 & Solution & 300.02 & 1675 & 1466.00 & 12.48\\
tai20 20 9.json & 20 & 20 & Solution & 300.01 & 1627 & 1424.00 & 12.48\\
tai30 15 0.json & 30 & 15 & Solution & 300.03 & 1766 & 1764.00 &  0.11\\
tai30 15 1.json & 30 & 15 & Solution & 300.03 & 1860 & 1774.00 &  4.62\\
tai30 15 2.json & 30 & 15 & Solution & 300.03 & 1828 & 1778.00 &  2.74\\
tai30 15 3.json & 30 & 15 & Solution & 300.03 & 1885 & 1828.00 &  3.02\\
tai30 15 4.json & 30 & 15 & Optimal & 15.87 & 2007 & 2007.00 &  0.00\\
tai30 15 5.json & 30 & 15 & Solution & 300.02 & 1852 & 1819.00 &  1.78\\
tai30 15 6.json & 30 & 15 & Solution & 300.03 & 1804 & 1771.00 &  1.83\\
tai30 15 7.json & 30 & 15 & Solution & 300.03 & 1701 & 1673.00 &  1.65\\
tai30 15 8.json & 30 & 15 & Solution & 300.02 & 1821 & 1795.00 &  1.43\\
tai30 15 9.json & 30 & 15 & Solution & 300.02 & 1706 & 1631.00 &  4.40\\
tai30 20 0.json & 30 & 20 & Solution & 300.03 & 2071 & 1857.00 & 10.33\\
tai30 20 1.json & 30 & 20 & Solution & 300.03 & 2006 & 1867.00 &  6.93\\
tai30 20 2.json & 30 & 20 & Solution & 300.04 & 1912 & 1809.00 &  5.39\\
tai30 20 3.json & 30 & 20 & Solution & 300.04 & 2086 & 1923.00 &  7.81\\
tai30 20 4.json & 30 & 20 & Solution & 300.03 & 2014 & 1997.00 &  0.84\\
tai30 20 5.json & 30 & 20 & Solution & 300.03 & 2070 & 1940.00 &  6.28\\
tai30 20 6.json & 30 & 20 & Solution & 300.02 & 1942 & 1783.00 &  8.19\\
tai30 20 7.json & 30 & 20 & Solution & 300.03 & 2027 & 1905.00 &  6.02\\
tai30 20 8.json & 30 & 20 & Solution & 300.03 & 2026 & 1903.00 &  6.07\\
tai30 20 9.json & 30 & 20 & Solution & 300.03 & 2006 & 1806.00 &  9.97\\
tai50 15 0.json & 50 & 15 & Optimal & 90.61 & 2760 & 2760.00 &  0.00\\
tai50 15 1.json & 50 & 15 & Optimal & 52.92 & 2756 & 2756.00 &  0.00\\
tai50 15 2.json & 50 & 15 & Optimal & 16.80 & 2717 & 2717.00 &  0.00\\
tai50 15 3.json & 50 & 15 & Optimal & 10.28 & 2839 & 2839.00 &  0.00\\
tai50 15 4.json & 50 & 15 & Optimal & 35.27 & 2679 & 2679.00 &  0.00\\
tai50 15 5.json & 50 & 15 & Optimal & 66.12 & 2781 & 2781.00 &  0.00\\
tai50 15 6.json & 50 & 15 & Optimal & 15.50 & 2943 & 2943.00 &  0.00\\
tai50 15 7.json & 50 & 15 & Optimal & 30.88 & 2885 & 2885.00 &  0.00\\
tai50 15 8.json & 50 & 15 & Optimal & 35.67 & 2655 & 2655.00 &  0.00\\
tai50 15 9.json & 50 & 15 & Optimal & 37.75 & 2723 & 2723.00 &  0.00\\
tai50 20 0.json & 50 & 20 & Optimal & 165.80 & 2868 & 2868.00 &  0.00\\
tai50 20 1.json & 50 & 20 & Solution & 300.11 & 2907 & 2869.00 &  1.31\\
tai50 20 2.json & 50 & 20 & Solution & 300.10 & 2784 & 2755.00 &  1.04\\
tai50 20 3.json & 50 & 20 & Solution & 300.10 & 2708 & 2702.00 &  0.22\\
tai50 20 4.json & 50 & 20 & Solution & 300.11 & 2738 & 2725.00 &  0.47\\
tai50 20 5.json & 50 & 20 & Optimal & 199.89 & 2845 & 2845.00 &  0.00\\
tai50 20 6.json & 50 & 20 & Solution & 300.12 & 2826 & 2825.00 &  0.04\\
tai50 20 7.json & 50 & 20 & Optimal & 144.63 & 2784 & 2784.00 &  0.00\\
tai50 20 8.json & 50 & 20 & Optimal & 87.12 & 3071 & 3071.00 &  0.00\\
tai50 20 9.json & 50 & 20 & Solution & 300.12 & 3036 & 2995.00 &  1.35\\
\end{longtable}



\section{Results for CPSat}

\begin{longtable}{lrrlrrrr}
\caption{Results for Taillard JobShop (CPSat) (30 Instances)}\\\toprule
Name & \shortstack{Nr\\Jobs} & \shortstack{Nr\\Machines} & Status & Time & Makespan & Bound & \shortstack{Gap\\Percent}\\ \midrule
\endhead
\bottomrule
\endfoot
tai100 20 0.json & 100 & 20 & Solution & 600.40 & 5620 &  0.00 &  0.00\\
tai100 20 1.json & 100 & 20 & Solution & 600.30 & 5280 &  0.00 &  0.00\\
tai100 20 2.json & 100 & 20 & Solution & 601.15 & 5638 &  0.00 &  0.00\\
tai100 20 3.json & 100 & 20 & Solution & 600.36 & 5355 & 5339.00 &  0.00\\
tai100 20 4.json & 100 & 20 & Solution & 600.16 & 5664 & 5392.00 &  0.00\\
tai100 20 5.json & 100 & 20 & Solution & 600.14 & 5433 & 5342.00 &  0.00\\
tai100 20 6.json & 100 & 20 & Solution & 600.47 & 5457 & 5436.00 &  0.00\\
tai100 20 7.json & 100 & 20 & Solution & 600.46 & 5435 & 5394.00 &  0.00\\
tai100 20 8.json & 100 & 20 & Solution & 600.34 & 5397 & 5358.00 &  0.00\\
tai100 20 9.json & 100 & 20 & Solution & 600.43 & 5267 & 5183.00 &  0.00\\
tai15 15 0.json & 15 & 15 & Optimal &  5.32 & 1231 & 1231.00 &  0.00\\
tai15 15 1.json & 15 & 15 & Optimal & 44.73 & 1244 & 1244.00 &  0.00\\
tai15 15 2.json & 15 & 15 & Optimal & 18.76 & 1218 & 1218.00 &  0.00\\
tai15 15 3.json & 15 & 15 & Optimal & 19.12 & 1175 & 1175.00 &  0.00\\
tai15 15 4.json & 15 & 15 & Optimal & 216.74 & 1224 & 1224.00 &  0.00\\
tai15 15 5.json & 15 & 15 & Solution & 600.10 & 1238 & 1202.00 &  2.91\\
tai15 15 6.json & 15 & 15 & Optimal & 246.22 & 1227 & 1227.00 &  0.00\\
tai15 15 7.json & 15 & 15 & Optimal & 186.11 & 1217 & 1217.00 &  0.00\\
tai15 15 8.json & 15 & 15 & Optimal & 134.40 & 1274 & 1274.00 &  0.00\\
tai15 15 9.json & 15 & 15 & Optimal & 20.74 & 1241 & 1241.00 &  0.00\\
tai50 15 0.json & 50 & 15 & Optimal & 186.33 & 2760 & 2760.00 &  0.00\\
tai50 15 1.json & 50 & 15 & Optimal & 155.63 & 2756 & 2756.00 &  0.00\\
tai50 15 2.json & 50 & 15 & Optimal & 68.58 & 2717 & 2717.00 &  0.00\\
tai50 15 3.json & 50 & 15 & Optimal & 26.60 & 2839 & 2839.00 &  0.00\\
tai50 15 4.json & 50 & 15 & Optimal & 362.73 & 2679 & 2679.00 &  0.00\\
tai50 15 5.json & 50 & 15 & Optimal & 249.56 & 2781 & 2781.00 &  0.00\\
tai50 15 6.json & 50 & 15 & Optimal & 120.38 & 2943 & 2943.00 &  0.00\\
tai50 15 7.json & 50 & 15 & Optimal & 216.50 & 2885 & 2885.00 &  0.00\\
tai50 15 8.json & 50 & 15 & Optimal & 435.42 & 2655 & 2655.00 &  0.00\\
tai50 15 9.json & 50 & 15 & Optimal & 217.29 & 2723 & 2723.00 &  0.00\\
\end{longtable}



\section{Sample Results on Mac (CPOptimizer)}
For a selected subset of the tests, we also tried running on a mac laptop, results show some good improvement of the m2 based laptop over the Intel based Windows machine, but the improvements are not consistent.
\begin{longtable}{lrrlrrrr}
\caption{Results for Taillard Jobshop (Selected Instances on Mac)}\\\toprule
Name & \shortstack{Nr\\Jobs} & \shortstack{Nr\\Machines} & Status & Time & Makespan & Bound & \shortstack{Gap\\Percent}\\ \midrule
\endhead
\bottomrule
\endfoot
tai100 20 0.json & 100 & 20 & Optimal & 143.93 & 5464 & 5464.00 &  0.00\\
tai100 20 1.json & 100 & 20 & Optimal & 86.52 & 5181 & 5181.00 &  0.00\\
tai100 20 2.json & 100 & 20 & Optimal & 63.63 & 5568 & 5568.00 &  0.00\\
tai100 20 3.json & 100 & 20 & Optimal & 19.51 & 5339 & 5339.00 &  0.00\\
tai100 20 4.json & 100 & 20 & Optimal & 174.11 & 5392 & 5392.00 &  0.00\\
tai100 20 5.json & 100 & 20 & Optimal & 80.95 & 5342 & 5342.00 &  0.00\\
tai100 20 6.json & 100 & 20 & Optimal & 139.30 & 5436 & 5436.00 &  0.00\\
tai100 20 7.json & 100 & 20 & Optimal & 48.86 & 5394 & 5394.00 &  0.00\\
tai100 20 8.json & 100 & 20 & Optimal & 82.22 & 5358 & 5358.00 &  0.00\\
tai100 20 9.json & 100 & 20 & Optimal & 143.55 & 5183 & 5183.00 &  0.00\\
\end{longtable}



\clearpage
\chapter{Taillard Flow Shop Problems}

These problems seem to be more difficult to solve to optimality. The number of stages seems to make a huge difference, we can solve the problems with five stages (machines) much more easily than the problems with 10 or twenty stages.

\section{Results for CPOptimizer}

\begin{longtable}{lrrlrrrr}
\caption{Results for Taillard Flowshop (120 Instances)}\\\toprule
Name & \shortstack{Nr\\Jobs} & \shortstack{Nr\\Machines} & Status & Time & Makespan & Bound & \shortstack{Gap\\Percent}\\ \midrule
\endhead
\bottomrule
\endfoot
tai100 10 0.json & 100 & 10 & Solution & 600.16 & 5979 & 5759.00 &  3.68\\
tai100 10 1.json & 100 & 10 & Solution & 600.06 & 5418 & 5345.00 &  1.35\\
tai100 10 2.json & 100 & 10 & Solution & 600.06 & 5798 & 5646.00 &  2.62\\
tai100 10 3.json & 100 & 10 & Solution & 600.03 & 6040 & 5737.00 &  5.02\\
tai100 10 4.json & 100 & 10 & Solution & 600.02 & 5663 & 5431.00 &  4.10\\
tai100 10 5.json & 100 & 10 & Solution & 600.05 & 5378 & 5274.00 &  1.93\\
tai100 10 6.json & 100 & 10 & Solution & 600.04 & 5697 & 5553.00 &  2.53\\
tai100 10 7.json & 100 & 10 & Solution & 600.03 & 5813 & 5575.00 &  4.09\\
tai100 10 8.json & 100 & 10 & Solution & 600.04 & 5983 & 5838.00 &  2.42\\
tai100 10 9.json & 100 & 10 & Solution & 600.02 & 5903 & 5835.00 &  1.15\\
tai100 20 0.json & 100 & 20 & Solution & 600.07 & 6731 & 5914.00 & 12.14\\
tai100 20 1.json & 100 & 20 & Solution & 600.05 & 6840 & 6115.00 & 10.60\\
tai100 20 2.json & 100 & 20 & Solution & 600.06 & 6778 & 6139.00 &  9.43\\
tai100 20 3.json & 100 & 20 & Solution & 600.06 & 6720 & 6117.00 &  8.97\\
tai100 20 4.json & 100 & 20 & Solution & 600.05 & 6853 & 6148.00 & 10.29\\
tai100 20 5.json & 100 & 20 & Solution & 600.07 & 6989 & 6192.00 & 11.40\\
tai100 20 6.json & 100 & 20 & Solution & 600.04 & 6772 & 6045.00 & 10.74\\
tai100 20 7.json & 100 & 20 & Solution & 600.05 & 6940 & 6113.00 & 11.92\\
tai100 20 8.json & 100 & 20 & Solution & 600.04 & 7092 & 6014.00 & 15.20\\
tai100 20 9.json & 100 & 20 & Solution & 600.06 & 6871 & 6359.00 &  7.45\\
tai100 5 0.json & 100 & 5 & Optimal & 73.90 & 5493 & 5493.00 &  0.00\\
tai100 5 1.json & 100 & 5 & Solution & 600.13 & 5276 & 5232.00 &  0.83\\
tai100 5 2.json & 100 & 5 & Solution & 600.11 & 5178 & 5170.00 &  0.15\\
tai100 5 3.json & 100 & 5 & Solution & 600.12 & 4996 & 4993.00 &  0.06\\
tai100 5 4.json & 100 & 5 & Optimal & 79.87 & 5247 & 5247.00 &  0.00\\
tai100 5 5.json & 100 & 5 & Optimal & 231.03 & 5135 & 5135.00 &  0.00\\
tai100 5 6.json & 100 & 5 & Optimal & 95.07 & 5232 & 5232.00 &  0.00\\
tai100 5 7.json & 100 & 5 & Optimal & 293.37 & 5083 & 5083.00 &  0.00\\
tai100 5 8.json & 100 & 5 & Solution & 600.12 & 5464 & 5438.00 &  0.48\\
tai100 5 9.json & 100 & 5 & Optimal & 70.58 & 5318 & 5318.00 &  0.00\\
tai200 10 0.json & 200 & 10 & Solution & 600.06 & 11136 & 10842.00 &  2.64\\
tai200 10 1.json & 200 & 10 & Solution & 600.04 & 10981 & 10429.00 &  5.03\\
tai200 10 2.json & 200 & 10 & Solution & 600.06 & 11276 & 10915.00 &  3.20\\
tai200 10 3.json & 200 & 10 & Solution & 600.06 & 11217 & 10826.00 &  3.49\\
tai200 10 4.json & 200 & 10 & Solution & 600.06 & 11139 & 10474.00 &  5.97\\
tai200 10 5.json & 200 & 10 & Solution & 600.06 & 10828 & 10311.00 &  4.77\\
tai200 10 6.json & 200 & 10 & Solution & 600.07 & 11202 & 10825.00 &  3.37\\
tai200 10 7.json & 200 & 10 & Solution & 600.07 & 11287 & 10709.00 &  5.12\\
tai200 10 8.json & 200 & 10 & Solution & 600.07 & 11004 & 10419.00 &  5.32\\
tai200 10 9.json & 200 & 10 & Solution & 600.05 & 11178 & 10664.00 &  4.60\\
tai200 20 0.json & 200 & 20 & Solution & 600.11 & 12439 & 11010.00 & 11.49\\
tai200 20 1.json & 200 & 20 & Solution & 600.12 & 12878 & 10976.00 & 14.77\\
tai200 20 2.json & 200 & 20 & Solution & 600.12 & 12506 & 11168.00 & 10.70\\
tai200 20 3.json & 200 & 20 & Solution & 600.09 & 12618 & 11131.00 & 11.78\\
tai200 20 4.json & 200 & 20 & Solution & 600.06 & 12649 & 11160.00 & 11.77\\
tai200 20 5.json & 200 & 20 & Solution & 600.13 & 12740 & 11114.00 & 12.76\\
tai200 20 6.json & 200 & 20 & Solution & 600.10 & 12985 & 11249.00 & 13.37\\
tai200 20 7.json & 200 & 20 & Solution & 600.08 & 12724 & 11149.00 & 12.38\\
tai200 20 8.json & 200 & 20 & Solution & 600.07 & 12693 & 11013.00 & 13.24\\
tai200 20 9.json & 200 & 20 & Solution & 600.07 & 12869 & 11167.00 & 13.23\\
tai20 10 0.json & 20 & 10 & Solution & 600.01 & 1575 & 1494.00 &  5.14\\
tai20 10 1.json & 20 & 10 & Solution & 600.02 & 1648 & 1555.00 &  5.64\\
tai20 10 2.json & 20 & 10 & Solution & 600.04 & 1477 & 1430.00 &  3.18\\
tai20 10 3.json & 20 & 10 & Optimal & 170.76 & 1356 & 1356.00 &  0.00\\
tai20 10 4.json & 20 & 10 & Solution & 600.04 & 1403 & 1353.00 &  3.56\\
tai20 10 5.json & 20 & 10 & Solution & 600.03 & 1374 & 1352.00 &  1.60\\
tai20 10 6.json & 20 & 10 & Solution & 600.03 & 1446 & 1388.00 &  4.01\\
tai20 10 7.json & 20 & 10 & Solution & 600.02 & 1579 & 1407.00 & 10.89\\
tai20 10 8.json & 20 & 10 & Optimal & 74.54 & 1586 & 1586.00 &  0.00\\
tai20 10 9.json & 20 & 10 & Solution & 600.02 & 1600 & 1473.00 &  7.94\\
tai20 20 0.json & 20 & 20 & Solution & 600.04 & 2284 & 1970.00 & 13.75\\
tai20 20 1.json & 20 & 20 & Solution & 600.05 & 2109 & 1784.00 & 15.41\\
tai20 20 2.json & 20 & 20 & Solution & 600.05 & 2341 & 1924.00 & 17.81\\
tai20 20 3.json & 20 & 20 & Solution & 600.04 & 2266 & 1892.00 & 16.50\\
tai20 20 4.json & 20 & 20 & Solution & 600.04 & 2311 & 1994.00 & 13.72\\
tai20 20 5.json & 20 & 20 & Solution & 600.03 & 2215 & 1899.00 & 14.27\\
tai20 20 6.json & 20 & 20 & Solution & 600.03 & 2265 & 1952.00 & 13.82\\
tai20 20 7.json & 20 & 20 & Solution & 600.04 & 2206 & 1931.00 & 12.47\\
tai20 20 8.json & 20 & 20 & Solution & 600.03 & 2233 & 1900.00 & 14.91\\
tai20 20 9.json & 20 & 20 & Solution & 600.03 & 2181 & 1939.00 & 11.10\\
tai20 5 0.json & 20 & 5 & Optimal &  3.03 & 1278 & 1278.00 &  0.00\\
tai20 5 1.json & 20 & 5 & Optimal &  2.20 & 1358 & 1358.00 &  0.00\\
tai20 5 2.json & 20 & 5 & Optimal &  2.72 & 1073 & 1073.00 &  0.00\\
tai20 5 3.json & 20 & 5 & Optimal &  3.28 & 1292 & 1292.00 &  0.00\\
tai20 5 4.json & 20 & 5 & Optimal &  5.46 & 1231 & 1231.00 &  0.00\\
tai20 5 5.json & 20 & 5 & Optimal &  2.45 & 1193 & 1193.00 &  0.00\\
tai20 5 6.json & 20 & 5 & Optimal &  1.89 & 1234 & 1234.00 &  0.00\\
tai20 5 7.json & 20 & 5 & Optimal &  3.99 & 1199 & 1199.00 &  0.00\\
tai20 5 8.json & 20 & 5 & Optimal &  1.72 & 1210 & 1210.00 &  0.00\\
tai20 5 9.json & 20 & 5 & Optimal &  1.76 & 1103 & 1103.00 &  0.00\\
tai500 20 0.json & 500 & 20 & Solution & 600.23 & 28669 & 25931.00 &  9.55\\
tai500 20 1.json & 500 & 20 & Solution & 600.18 & 29047 & 26390.00 &  9.15\\
tai500 20 2.json & 500 & 20 & Solution & 600.22 & 28796 & 26330.00 &  8.56\\
tai500 20 3.json & 500 & 20 & Solution & 600.18 & 28907 & 26456.00 &  8.48\\
tai500 20 4.json & 500 & 20 & Solution & 600.21 & 28841 & 26205.00 &  9.14\\
tai500 20 5.json & 500 & 20 & Solution & 600.19 & 28954 & 26436.00 &  8.70\\
tai500 20 6.json & 500 & 20 & Solution & 600.17 & 28750 & 26329.00 &  8.42\\
tai500 20 7.json & 500 & 20 & Solution & 600.17 & 28924 & 26451.00 &  8.55\\
tai500 20 8.json & 500 & 20 & Solution & 600.19 & 28038 & 25929.00 &  7.52\\
tai500 20 9.json & 500 & 20 & Solution & 600.22 & 28726 & 26355.00 &  8.25\\
tai50 10 0.json & 50 & 10 & Solution & 600.07 & 3055 & 2962.00 &  3.04\\
tai50 10 1.json & 50 & 10 & Solution & 600.10 & 2929 & 2829.00 &  3.41\\
tai50 10 2.json & 50 & 10 & Solution & 600.11 & 2908 & 2825.00 &  2.85\\
tai50 10 3.json & 50 & 10 & Solution & 600.06 & 3125 & 3038.00 &  2.78\\
tai50 10 4.json & 50 & 10 & Solution & 600.06 & 3100 & 2923.00 &  5.71\\
tai50 10 5.json & 50 & 10 & Solution & 600.11 & 3065 & 2966.00 &  3.23\\
tai50 10 6.json & 50 & 10 & Solution & 600.06 & 3133 & 3063.00 &  2.23\\
tai50 10 7.json & 50 & 10 & Solution & 600.11 & 3096 & 3000.00 &  3.10\\
tai50 10 8.json & 50 & 10 & Solution & 600.09 & 2959 & 2832.00 &  4.29\\
tai50 10 9.json & 50 & 10 & Solution & 600.08 & 3111 & 3046.00 &  2.09\\
tai50 20 0.json & 50 & 20 & Solution & 600.17 & 4003 & 3562.00 & 11.02\\
tai50 20 1.json & 50 & 20 & Solution & 600.20 & 4041 & 3533.00 & 12.57\\
tai50 20 2.json & 50 & 20 & Solution & 600.18 & 3972 & 3412.00 & 14.10\\
tai50 20 3.json & 50 & 20 & Solution & 600.16 & 3922 & 3383.00 & 13.74\\
tai50 20 4.json & 50 & 20 & Solution & 600.15 & 3830 & 3387.00 & 11.57\\
tai50 20 5.json & 50 & 20 & Solution & 600.19 & 3861 & 3499.00 &  9.38\\
tai50 20 6.json & 50 & 20 & Solution & 600.21 & 3930 & 3461.00 & 11.93\\
tai50 20 7.json & 50 & 20 & Solution & 60.25 & 4056 & 3411.00 & 15.90\\
tai50 20 8.json & 50 & 20 & Solution & 60.17 & 4115 & 3468.00 & 15.72\\
tai50 20 9.json & 50 & 20 & Solution & 60.19 & 3986 & 3474.00 & 12.84\\
tai50 5 0.json & 50 & 5 & Optimal & 17.70 & 2724 & 2724.00 &  0.00\\
tai50 5 1.json & 50 & 5 & Optimal & 57.29 & 2834 & 2834.00 &  0.00\\
tai50 5 2.json & 50 & 5 & Optimal & 47.53 & 2612 & 2612.00 &  0.00\\
tai50 5 3.json & 50 & 5 & Optimal & 11.58 & 2751 & 2751.00 &  0.00\\
tai50 5 4.json & 50 & 5 & Optimal & 23.50 & 2853 & 2853.00 &  0.00\\
tai50 5 5.json & 50 & 5 & Optimal & 28.67 & 2825 & 2825.00 &  0.00\\
tai50 5 6.json & 50 & 5 & Optimal & 42.14 & 2716 & 2716.00 &  0.00\\
tai50 5 7.json & 50 & 5 & Optimal & 19.26 & 2683 & 2683.00 &  0.00\\
tai50 5 8.json & 50 & 5 & Optimal & 56.32 & 2545 & 2545.00 &  0.00\\
tai50 5 9.json & 50 & 5 & Optimal &  5.82 & 2776 & 2776.00 &  0.00\\
\end{longtable}



\section{Results for CPSat}

\begin{longtable}{lrrlrrrr}
\caption{Results for Taillard Flowshop (CPSat) (120 Instances)}\\\toprule
Name & \shortstack{Nr\\Jobs} & \shortstack{Nr\\Machines} & Status & Time & Makespan & Bound & \shortstack{Gap\\Percent}\\ \midrule
\endhead
\bottomrule
\endfoot
tai100 10 0.json & 100 & 10 & Solution & 600.16 & 6170 & 5759.00 &  6.66\\
tai100 10 1.json & 100 & 10 & Solution & 600.16 & 5813 & 4707.00 & 19.03\\
tai100 10 2.json & 100 & 10 & Solution & 600.31 & 6133 & 5543.00 &  9.62\\
tai100 10 3.json & 100 & 10 & Solution & 600.14 & 6464 & 5716.00 & 11.57\\
tai100 10 4.json & 100 & 10 & Solution & 600.19 & 6143 & 5153.00 & 16.12\\
tai100 10 5.json & 100 & 10 & Solution & 600.21 & 5844 & 4823.00 & 17.47\\
tai100 10 6.json & 100 & 10 & Solution & 600.21 & 5949 & 5179.00 & 12.94\\
tai100 10 7.json & 100 & 10 & Solution & 600.17 & 6180 & 5237.00 & 15.26\\
tai100 10 8.json & 100 & 10 & Solution & 600.18 & 6341 & 5844.00 &  7.84\\
tai100 10 9.json & 100 & 10 & Solution & 600.35 & 6317 & 5254.00 & 16.83\\
tai100 20 0.json & 100 & 20 & Solution & 600.28 & 7389 & 4511.00 & 38.95\\
tai100 20 1.json & 100 & 20 & Solution & 600.23 & 7196 & 4343.00 & 39.65\\
tai100 20 2.json & 100 & 20 & Solution & 600.21 & 7408 & 4853.00 & 34.49\\
tai100 20 3.json & 100 & 20 & Solution & 600.22 & 7050 & 4667.00 & 33.80\\
tai100 20 4.json & 100 & 20 & Solution & 600.21 & 7220 & 4908.00 & 32.02\\
tai100 20 5.json & 100 & 20 & Solution & 600.28 & 7646 & 4680.00 & 38.79\\
tai100 20 6.json & 100 & 20 & Solution & 600.24 & 7164 & 4457.00 & 37.79\\
tai100 20 7.json & 100 & 20 & Solution & 600.22 & 7691 & 4792.00 & 37.69\\
tai100 20 8.json & 100 & 20 & Solution & 600.24 & 7445 & 4852.00 & 34.83\\
tai100 20 9.json & 100 & 20 & Solution & 600.22 & 7429 & 5405.00 & 27.24\\
tai100 5 0.json & 100 & 5 & Optimal & 600.07 & 5493 & 5493.00 &  0.00\\
tai100 5 1.json & 100 & 5 & Solution & 600.36 & 5294 & 5240.00 &  1.02\\
tai100 5 2.json & 100 & 5 & Solution & 600.16 & 5216 & 5173.00 &  0.82\\
tai100 5 3.json & 100 & 5 & Solution & 600.13 & 5001 & 4993.00 &  0.16\\
tai100 5 4.json & 100 & 5 & Solution & 600.36 & 5279 & 5247.00 &  0.61\\
tai100 5 5.json & 100 & 5 & Optimal & 600.07 & 5135 & 5135.00 &  0.00\\
tai100 5 6.json & 100 & 5 & Solution & 600.20 & 5281 & 5228.00 &  1.00\\
tai100 5 7.json & 100 & 5 & Solution & 600.22 & 5137 & 5083.00 &  1.05\\
tai100 5 8.json & 100 & 5 & Solution & 600.16 & 5481 & 5442.00 &  0.71\\
tai100 5 9.json & 100 & 5 & Solution & 600.17 & 5346 & 5305.00 &  0.77\\
tai200 10 0.json & 200 & 10 & Solution & 600.35 & 11993 & 6397.00 & 46.66\\
tai200 10 1.json & 200 & 10 & Solution & 600.27 & 12255 & 6257.00 & 48.94\\
tai200 10 2.json & 200 & 10 & Solution & 600.37 & 11987 & 6508.00 & 45.71\\
tai200 10 3.json & 200 & 10 & Solution & 600.35 & 11934 & 6160.00 & 48.38\\
tai200 10 4.json & 200 & 10 & Solution & 600.32 & 11982 & 6309.00 & 47.35\\
tai200 10 5.json & 200 & 10 & Solution & 600.35 & 11666 & 6156.00 & 47.23\\
tai200 10 6.json & 200 & 10 & Solution & 600.38 & 12236 & 6001.00 & 50.96\\
tai200 10 7.json & 200 & 10 & Solution & 600.32 & 12186 & 6214.00 & 49.01\\
tai200 10 8.json & 200 & 10 & Solution & 600.35 & 11818 & 6108.00 & 48.32\\
tai200 10 9.json & 200 & 10 & Solution & 600.29 & 11764 & 6496.00 & 44.78\\
tai200 20 0.json & 200 & 20 & Solution & 600.37 & 13270 & 11010.00 & 17.03\\
tai200 20 1.json & 200 & 20 & Solution & 600.36 & 13276 & 6170.00 & 53.53\\
tai200 20 2.json & 200 & 20 & Solution & 600.38 & 13303 & 6157.00 & 53.72\\
tai200 20 3.json & 200 & 20 & Solution & 600.37 & 13647 & 6224.00 & 54.39\\
tai200 20 4.json & 200 & 20 & Solution & 600.35 & 13271 & 6162.00 & 53.57\\
tai200 20 5.json & 200 & 20 & Solution & 602.34 & 13608 & 6257.00 & 54.02\\
tai200 20 6.json & 200 & 20 & Solution & 601.71 & 13515 & 6352.00 & 53.00\\
tai200 20 7.json & 200 & 20 & Solution & 600.40 & 13193 & 6242.00 & 52.69\\
tai200 20 8.json & 200 & 20 & Solution & 600.37 & 13298 & 6210.00 & 53.30\\
tai200 20 9.json & 200 & 20 & Solution & 600.39 & 13355 & 6194.00 & 53.62\\
tai20 10 0.json & 20 & 10 & Solution & 600.16 & 1579 & 1547.00 &  2.03\\
tai20 10 1.json & 20 & 10 & Solution & 600.16 & 1686 & 1587.00 &  5.87\\
tai20 10 2.json & 20 & 10 & Solution & 600.12 & 1481 & 1438.00 &  2.90\\
tai20 10 3.json & 20 & 10 & Solution & 600.16 & 1400 & 1356.00 &  3.14\\
tai20 10 4.json & 20 & 10 & Solution & 600.29 & 1411 & 1360.00 &  3.61\\
tai20 10 5.json & 20 & 10 & Solution & 600.45 & 1374 & 1356.00 &  1.31\\
tai20 10 6.json & 20 & 10 & Solution & 600.14 & 1446 & 1398.00 &  3.32\\
tai20 10 7.json & 20 & 10 & Solution & 600.15 & 1548 & 1448.00 &  6.46\\
tai20 10 8.json & 20 & 10 & Optimal & 600.04 & 1586 & 1586.00 &  0.00\\
tai20 10 9.json & 20 & 10 & Solution & 600.11 & 1590 & 1529.00 &  3.84\\
tai20 20 0.json & 20 & 20 & Solution & 600.11 & 2265 & 2047.00 &  9.62\\
tai20 20 1.json & 20 & 20 & Solution & 600.28 & 2125 & 1844.00 & 13.22\\
tai20 20 2.json & 20 & 20 & Solution & 600.11 & 2349 & 1993.00 & 15.16\\
tai20 20 3.json & 20 & 20 & Solution & 600.14 & 2281 & 1957.00 & 14.20\\
tai20 20 4.json & 20 & 20 & Solution & 600.12 & 2376 & 2058.00 & 13.38\\
tai20 20 5.json & 20 & 20 & Solution & 600.11 & 2176 & 1974.00 &  9.28\\
tai20 20 6.json & 20 & 20 & Solution & 600.12 & 2320 & 2001.00 & 13.75\\
tai20 20 7.json & 20 & 20 & Solution & 600.12 & 2247 & 1982.00 & 11.79\\
tai20 20 8.json & 20 & 20 & Solution & 600.11 & 2267 & 1960.00 & 13.54\\
tai20 20 9.json & 20 & 20 & Solution & 600.12 & 2176 & 1971.00 &  9.42\\
tai20 5 0.json & 20 & 5 & Optimal & 409.17 & 1278 & 1278.00 &  0.00\\
tai20 5 1.json & 20 & 5 & Optimal & 12.58 & 1358 & 1358.00 &  0.00\\
tai20 5 2.json & 20 & 5 & Optimal &  2.90 & 1073 & 1073.00 &  0.00\\
tai20 5 3.json & 20 & 5 & Optimal &  8.44 & 1292 & 1292.00 &  0.00\\
tai20 5 4.json & 20 & 5 & Optimal & 56.24 & 1231 & 1231.00 &  0.00\\
tai20 5 5.json & 20 & 5 & Optimal & 600.01 & 1193 & 1193.00 &  0.00\\
tai20 5 6.json & 20 & 5 & Optimal &  3.20 & 1234 & 1234.00 &  0.00\\
tai20 5 7.json & 20 & 5 & Optimal & 13.83 & 1199 & 1199.00 &  0.00\\
tai20 5 8.json & 20 & 5 & Optimal &  8.57 & 1210 & 1210.00 &  0.00\\
tai20 5 9.json & 20 & 5 & Optimal &  3.37 & 1103 & 1103.00 &  0.00\\
tai500 20 0.json & 500 & 20 & Solution & 601.11 & 30220 & 13561.00 & 55.13\\
tai500 20 1.json & 500 & 20 & Solution & 602.31 & 30765 & 13909.00 & 54.79\\
tai500 20 2.json & 500 & 20 & Solution & 609.69 & 30517 & 13847.00 & 54.63\\
tai500 20 3.json & 500 & 20 & Solution & 603.09 & 30572 & 13410.00 & 56.14\\
tai500 20 4.json & 500 & 20 & Solution & 601.39 & 30483 & 13439.00 & 55.91\\
tai500 20 5.json & 500 & 20 & Solution & 601.50 & 30843 & 13725.00 & 55.50\\
tai500 20 6.json & 500 & 20 & Solution & 603.24 & 30714 & 13837.00 & 54.95\\
tai500 20 7.json & 500 & 20 & Solution & 601.54 & 30625 & 13932.00 & 54.51\\
tai500 20 8.json & 500 & 20 & Solution & 609.39 & 30367 & 13646.00 & 55.06\\
tai500 20 9.json & 500 & 20 & Solution & 601.57 & 30643 & 13782.00 & 55.02\\
tai50 10 0.json & 50 & 10 & Solution & 600.22 & 3162 & 2976.00 &  5.88\\
tai50 10 1.json & 50 & 10 & Solution & 600.32 & 3048 & 2829.00 &  7.19\\
tai50 10 2.json & 50 & 10 & Solution & 600.19 & 2962 & 2830.00 &  4.46\\
tai50 10 3.json & 50 & 10 & Solution & 600.17 & 3166 & 3059.00 &  3.38\\
tai50 10 4.json & 50 & 10 & Solution & 600.20 & 3093 & 2933.00 &  5.17\\
tai50 10 5.json & 50 & 10 & Solution & 600.16 & 3155 & 2986.00 &  5.36\\
tai50 10 6.json & 50 & 10 & Solution & 600.44 & 3201 & 3093.00 &  3.37\\
tai50 10 7.json & 50 & 10 & Solution & 600.32 & 3184 & 3003.00 &  5.68\\
tai50 10 8.json & 50 & 10 & Solution & 600.20 & 3004 & 2864.00 &  4.66\\
tai50 10 9.json & 50 & 10 & Solution & 600.21 & 3192 & 3046.00 &  4.57\\
tai50 20 0.json & 50 & 20 & Solution & 600.19 & 4301 & 3591.00 & 16.51\\
tai50 20 1.json & 50 & 20 & Solution & 600.21 & 4085 & 3554.00 & 13.00\\
tai50 20 2.json & 50 & 20 & Solution & 600.24 & 4227 & 3431.00 & 18.83\\
tai50 20 3.json & 50 & 20 & Solution & 600.21 & 4203 & 3419.00 & 18.65\\
tai50 20 4.json & 50 & 20 & Solution & 600.18 & 4100 & 3415.00 & 16.71\\
tai50 20 5.json & 50 & 20 & Solution & 600.17 & 4109 & 3516.00 & 14.43\\
tai50 20 6.json & 50 & 20 & Solution & 600.22 & 4079 & 3494.00 & 14.34\\
tai50 20 7.json & 50 & 20 & Solution & 600.38 & 4129 & 3456.00 & 16.30\\
tai50 20 8.json & 50 & 20 & Solution & 600.43 & 4143 & 3489.00 & 15.79\\
tai50 20 9.json & 50 & 20 & Solution & 600.24 & 4300 & 3520.00 & 18.14\\
tai50 5 0.json & 50 & 5 & Optimal & 600.05 & 2724 & 2724.00 &  0.00\\
tai50 5 1.json & 50 & 5 & Optimal & 600.03 & 2834 & 2834.00 &  0.00\\
tai50 5 2.json & 50 & 5 & Optimal & 166.99 & 2612 & 2612.00 &  0.00\\
tai50 5 3.json & 50 & 5 & Optimal & 193.35 & 2751 & 2751.00 &  0.00\\
tai50 5 4.json & 50 & 5 & Optimal & 600.03 & 2853 & 2853.00 &  0.00\\
tai50 5 5.json & 50 & 5 & Optimal & 600.03 & 2825 & 2825.00 &  0.00\\
tai50 5 6.json & 50 & 5 & Optimal & 600.06 & 2716 & 2716.00 &  0.00\\
tai50 5 7.json & 50 & 5 & Optimal & 600.03 & 2683 & 2683.00 &  0.00\\
tai50 5 8.json & 50 & 5 & Solution & 600.20 & 2549 & 2545.00 &  0.16\\
tai50 5 9.json & 50 & 5 & Optimal & 600.03 & 2776 & 2776.00 &  0.00\\
\end{longtable}



\section{Permutation Flowshop Results for CPOptimizer}

We can run the flowshop benchmarks with an additional constraint to be solved as a permutation flowshop, which dramatically reduces the sets of feasible solutions, and the search tree to be searched. This might results in improved solutions found as a larger part of that search space can be explored, but solutions can be worse than for the original problem. In particular the optimal solution for the permutation flowshop can be worse than a good feasible solution for the unrestricted flowshop.

\begin{longtable}{lrrlrrrr}
\caption{Results for Taillard Permutation Flowshop (120 Instances)}\\\toprule
Name & \shortstack{Nr\\Jobs} & \shortstack{Nr\\Machines} & Status & Time & Makespan & Bound & \shortstack{Gap\\Percent}\\ \midrule
\endhead
\bottomrule
\endfoot
tai100 10 0.json & 100 & 10 & Solution & 600.34 & 5789 & 5766.00 &  0.40\\
tai100 10 1.json & 100 & 10 & Solution & 600.07 & 5391 & 5347.00 &  0.82\\
tai100 10 2.json & 100 & 10 & Solution & 600.08 & 5691 & 5659.00 &  0.56\\
tai100 10 3.json & 100 & 10 & Solution & 600.06 & 5860 & 5776.00 &  1.43\\
tai100 10 4.json & 100 & 10 & Solution & 600.05 & 5513 & 5450.00 &  1.14\\
tai100 10 5.json & 100 & 10 & Solution & 600.03 & 5308 & 5290.00 &  0.34\\
tai100 10 6.json & 100 & 10 & Solution & 600.03 & 5647 & 5556.00 &  1.61\\
tai100 10 7.json & 100 & 10 & Solution & 600.03 & 5689 & 5586.00 &  1.81\\
tai100 10 8.json & 100 & 10 & Solution & 600.05 & 5903 & 5865.00 &  0.64\\
tai100 10 9.json & 100 & 10 & Solution & 600.04 & 5860 & 5837.00 &  0.39\\
tai100 20 0.json & 100 & 20 & Solution & 600.05 & 6526 & 5936.00 &  9.04\\
tai100 20 1.json & 100 & 20 & Solution & 600.07 & 6390 & 6122.00 &  4.19\\
tai100 20 2.json & 100 & 20 & Solution & 600.07 & 6481 & 6162.00 &  4.92\\
tai100 20 3.json & 100 & 20 & Solution & 600.08 & 6463 & 6163.00 &  4.64\\
tai100 20 4.json & 100 & 20 & Solution & 600.05 & 6497 & 6161.00 &  5.17\\
tai100 20 5.json & 100 & 20 & Solution & 600.05 & 6554 & 6203.00 &  5.36\\
tai100 20 6.json & 100 & 20 & Solution & 600.07 & 6483 & 6061.00 &  6.51\\
tai100 20 7.json & 100 & 20 & Solution & 600.08 & 6670 & 6190.00 &  7.20\\
tai100 20 8.json & 100 & 20 & Solution & 600.05 & 6577 & 6063.00 &  7.82\\
tai100 20 9.json & 100 & 20 & Solution & 600.06 & 6684 & 6382.00 &  4.52\\
tai100 5 0.json & 100 & 5 & Optimal &  4.06 & 5493 & 5493.00 &  0.00\\
tai100 5 1.json & 100 & 5 & Optimal & 67.53 & 5268 & 5268.00 &  0.00\\
tai100 5 2.json & 100 & 5 & Optimal &  7.66 & 5175 & 5175.00 &  0.00\\
tai100 5 3.json & 100 & 5 & Optimal & 60.38 & 5014 & 5014.00 &  0.00\\
tai100 5 4.json & 100 & 5 & Optimal & 62.17 & 5250 & 5250.00 &  0.00\\
tai100 5 5.json & 100 & 5 & Optimal &  6.22 & 5135 & 5135.00 &  0.00\\
tai100 5 6.json & 100 & 5 & Optimal &  9.45 & 5246 & 5246.00 &  0.00\\
tai100 5 7.json & 100 & 5 & Optimal &  9.90 & 5094 & 5094.00 &  0.00\\
tai100 5 8.json & 100 & 5 & Optimal & 65.13 & 5448 & 5448.00 &  0.00\\
tai100 5 9.json & 100 & 5 & Optimal & 67.74 & 5322 & 5322.00 &  0.00\\
tai200 10 0.json & 200 & 10 & Solution & 600.05 & 10918 & 10861.00 &  0.52\\
tai200 10 1.json & 200 & 10 & Solution & 600.07 & 10718 & 10447.00 &  2.53\\
tai200 10 2.json & 200 & 10 & Solution & 600.05 & 11060 & 10920.00 &  1.27\\
tai200 10 3.json & 200 & 10 & Solution & 600.07 & 10934 & 10846.00 &  0.80\\
tai200 10 4.json & 200 & 10 & Solution & 600.08 & 10626 & 10494.00 &  1.24\\
tai200 10 5.json & 200 & 10 & Solution & 600.07 & 10453 & 10312.00 &  1.35\\
tai200 10 6.json & 200 & 10 & Solution & 600.07 & 10979 & 10853.00 &  1.15\\
tai200 10 7.json & 200 & 10 & Solution & 600.07 & 10856 & 10715.00 &  1.30\\
tai200 10 8.json & 200 & 10 & Solution & 600.06 & 10558 & 10422.00 &  1.29\\
tai200 10 9.json & 200 & 10 & Solution & 600.05 & 10761 & 10666.00 &  0.88\\
tai200 20 0.json & 200 & 20 & Solution & 600.13 & 11928 & 11048.00 &  7.38\\
tai200 20 1.json & 200 & 20 & Solution & 600.09 & 11991 & 11009.00 &  8.19\\
tai200 20 2.json & 200 & 20 & Solution & 600.09 & 12248 & 11217.00 &  8.42\\
tai200 20 3.json & 200 & 20 & Solution & 600.12 & 11967 & 11179.00 &  6.58\\
tai200 20 4.json & 200 & 20 & Solution & 600.13 & 11915 & 11168.00 &  6.27\\
tai200 20 5.json & 200 & 20 & Solution & 600.08 & 11923 & 11159.00 &  6.41\\
tai200 20 6.json & 200 & 20 & Solution & 600.10 & 12205 & 11269.00 &  7.67\\
tai200 20 7.json & 200 & 20 & Solution & 600.10 & 12221 & 11216.00 &  8.22\\
tai200 20 8.json & 200 & 20 & Solution & 600.12 & 11991 & 11054.00 &  7.81\\
tai200 20 9.json & 200 & 20 & Solution & 600.11 & 12022 & 11242.00 &  6.49\\
tai20 10 0.json & 20 & 10 & Optimal & 292.19 & 1582 & 1582.00 &  0.00\\
tai20 10 1.json & 20 & 10 & Solution & 600.02 & 1659 & 1580.00 &  4.76\\
tai20 10 2.json & 20 & 10 & Optimal & 587.59 & 1496 & 1496.00 &  0.00\\
tai20 10 3.json & 20 & 10 & Optimal & 62.06 & 1377 & 1377.00 &  0.00\\
tai20 10 4.json & 20 & 10 & Optimal & 101.03 & 1419 & 1419.00 &  0.00\\
tai20 10 5.json & 20 & 10 & Optimal & 119.12 & 1397 & 1397.00 &  0.00\\
tai20 10 6.json & 20 & 10 & Solution & 600.02 & 1484 & 1399.00 &  5.73\\
tai20 10 7.json & 20 & 10 & Optimal & 357.94 & 1538 & 1538.00 &  0.00\\
tai20 10 8.json & 20 & 10 & Optimal & 31.26 & 1593 & 1593.00 &  0.00\\
tai20 10 9.json & 20 & 10 & Solution & 600.04 & 1603 & 1492.00 &  6.92\\
tai20 20 0.json & 20 & 20 & Solution & 600.04 & 2340 & 2010.00 & 14.10\\
tai20 20 1.json & 20 & 20 & Solution & 600.03 & 2130 & 1823.00 & 14.41\\
tai20 20 2.json & 20 & 20 & Solution & 600.04 & 2329 & 1945.00 & 16.49\\
tai20 20 3.json & 20 & 20 & Solution & 600.04 & 2229 & 1933.00 & 13.28\\
tai20 20 4.json & 20 & 20 & Solution & 600.02 & 2324 & 2034.00 & 12.48\\
tai20 20 5.json & 20 & 20 & Solution & 600.04 & 2235 & 1967.00 & 11.99\\
tai20 20 6.json & 20 & 20 & Solution & 600.05 & 2291 & 1976.00 & 13.75\\
tai20 20 7.json & 20 & 20 & Solution & 600.04 & 2222 & 1936.00 & 12.87\\
tai20 20 8.json & 20 & 20 & Solution & 600.04 & 2250 & 1909.00 & 15.16\\
tai20 20 9.json & 20 & 20 & Solution & 600.02 & 2189 & 1954.00 & 10.74\\
tai20 5 0.json & 20 & 5 & Optimal &  0.79 & 1278 & 1278.00 &  0.00\\
tai20 5 1.json & 20 & 5 & Optimal &  0.39 & 1359 & 1359.00 &  0.00\\
tai20 5 2.json & 20 & 5 & Optimal &  0.76 & 1081 & 1081.00 &  0.00\\
tai20 5 3.json & 20 & 5 & Optimal &  1.38 & 1293 & 1293.00 &  0.00\\
tai20 5 4.json & 20 & 5 & Optimal &  4.98 & 1235 & 1235.00 &  0.00\\
tai20 5 5.json & 20 & 5 & Optimal &  0.45 & 1195 & 1195.00 &  0.00\\
tai20 5 6.json & 20 & 5 & Optimal &  0.37 & 1234 & 1234.00 &  0.00\\
tai20 5 7.json & 20 & 5 & Optimal &  1.22 & 1206 & 1206.00 &  0.00\\
tai20 5 8.json & 20 & 5 & Optimal &  0.65 & 1230 & 1230.00 &  0.00\\
tai20 5 9.json & 20 & 5 & Optimal &  0.58 & 1108 & 1108.00 &  0.00\\
tai500 20 0.json & 500 & 20 & Solution & 600.40 & 28935 & 25955.00 & 10.30\\
tai500 20 1.json & 500 & 20 & Solution & 600.21 & 29270 & 26432.00 &  9.70\\
tai500 20 2.json & 500 & 20 & Solution & 600.25 & 28956 & 26330.00 &  9.07\\
tai500 20 3.json & 500 & 20 & Solution & 600.21 & 28977 & 26456.00 &  8.70\\
tai500 20 4.json & 500 & 20 & Solution & 600.23 & 28999 & 26263.00 &  9.43\\
tai500 20 5.json & 500 & 20 & Solution & 600.28 & 28939 & 26440.00 &  8.64\\
tai500 20 6.json & 500 & 20 & Solution & 600.27 & 28709 & 26362.00 &  8.18\\
tai500 20 7.json & 500 & 20 & Solution & 600.29 & 29115 & 26514.00 &  8.93\\
tai500 20 8.json & 500 & 20 & Solution & 600.22 & 28659 & 25952.00 &  9.45\\
tai500 20 9.json & 500 & 20 & Solution & 600.25 & 28948 & 26429.00 &  8.70\\
tai50 10 0.json & 50 & 10 & Solution & 600.09 & 3039 & 2967.00 &  2.37\\
tai50 10 1.json & 50 & 10 & Solution & 600.09 & 2933 & 2829.00 &  3.55\\
tai50 10 2.json & 50 & 10 & Solution & 600.11 & 2921 & 2828.00 &  3.18\\
tai50 10 3.json & 50 & 10 & Optimal & 535.73 & 3063 & 3063.00 &  0.00\\
tai50 10 4.json & 50 & 10 & Solution & 600.10 & 3021 & 2928.00 &  3.08\\
tai50 10 5.json & 50 & 10 & Solution & 600.12 & 3050 & 2987.00 &  2.07\\
tai50 10 6.json & 50 & 10 & Solution & 600.10 & 3124 & 3065.00 &  1.89\\
tai50 10 7.json & 50 & 10 & Solution & 600.05 & 3040 & 3037.00 &  0.10\\
tai50 10 8.json & 50 & 10 & Solution & 600.12 & 2902 & 2883.00 &  0.65\\
tai50 10 9.json & 50 & 10 & Solution & 600.06 & 3121 & 3046.00 &  2.40\\
tai50 20 0.json & 50 & 20 & Solution & 600.21 & 3931 & 3591.00 &  8.65\\
tai50 20 1.json & 50 & 20 & Solution & 600.24 & 3812 & 3534.00 &  7.29\\
tai50 20 2.json & 50 & 20 & Solution & 600.24 & 3756 & 3428.00 &  8.73\\
tai50 20 3.json & 50 & 20 & Solution & 600.24 & 3817 & 3453.00 &  9.54\\
tai50 20 4.json & 50 & 20 & Solution & 600.20 & 3736 & 3389.00 &  9.29\\
tai50 20 5.json & 50 & 20 & Solution & 600.17 & 3784 & 3535.00 &  6.58\\
tai50 20 6.json & 50 & 20 & Solution & 600.18 & 3799 & 3495.00 &  8.00\\
tai50 20 7.json & 50 & 20 & Solution & 600.18 & 3836 & 3443.00 & 10.25\\
tai50 20 8.json & 50 & 20 & Solution & 600.22 & 3908 & 3482.00 & 10.90\\
tai50 20 9.json & 50 & 20 & Solution & 600.16 & 3857 & 3538.00 &  8.27\\
tai50 5 0.json & 50 & 5 & Optimal &  1.24 & 2724 & 2724.00 &  0.00\\
tai50 5 1.json & 50 & 5 & Optimal &  2.71 & 2834 & 2834.00 &  0.00\\
tai50 5 2.json & 50 & 5 & Optimal & 32.80 & 2621 & 2621.00 &  0.00\\
tai50 5 3.json & 50 & 5 & Optimal &  1.66 & 2751 & 2751.00 &  0.00\\
tai50 5 4.json & 50 & 5 & Optimal &  2.22 & 2863 & 2863.00 &  0.00\\
tai50 5 5.json & 50 & 5 & Optimal &  3.09 & 2829 & 2829.00 &  0.00\\
tai50 5 6.json & 50 & 5 & Optimal & 14.28 & 2725 & 2725.00 &  0.00\\
tai50 5 7.json & 50 & 5 & Optimal &  2.61 & 2683 & 2683.00 &  0.00\\
tai50 5 8.json & 50 & 5 & Optimal &  3.82 & 2552 & 2552.00 &  0.00\\
tai50 5 9.json & 50 & 5 & Optimal &  2.03 & 2782 & 2782.00 &  0.00\\
\end{longtable}



\clearpage
\chapter{SALBP-1 Assembly Line Balancing Problems}

The assembly line balancing problems have a single cumulative and no disjunctive constraints, so the indicated number of (disjunctive) machines is zero. 

The larger problem instances are still missing. For the small instances (20 tasks), only a few are not solved to optimality, for the medium sizes the number of optimal solutions found is reduced, and for larger instances, optimal solutions are rare.

\section{Results for CPOptimizer}

\begin{longtable}{lrrlrrrr}
\caption{Results for SALBP-1 Problems (CPO) (2100 Instances)}\\\toprule
Name & \shortstack{Nr\\Jobs} & \shortstack{Nr\\Machines} & Status & Time & Makespan & Bound & \shortstack{Gap\\Percent}\\ \midrule
\endhead
\bottomrule
\endfoot
instance n=1000 1.alb & 1 & 0 & Solution & 120.19 & 136 & 135.00 &  0.74\\
instance n=1000 10.alb & 1 & 0 & Solution & 120.07 & 141 & 140.00 &  0.71\\
instance n=1000 100.alb & 1 & 0 & Solution & 120.10 & 139 & 137.00 &  1.44\\
instance n=1000 101.alb & 1 & 0 & Solution & 120.18 & 558 & 505.00 &  9.50\\
instance n=1000 102.alb & 1 & 0 & Solution & 120.20 & 556 & 503.00 &  9.53\\
instance n=1000 103.alb & 1 & 0 & Solution & 120.25 & 562 & 503.00 & 10.50\\
instance n=1000 104.alb & 1 & 0 & Solution & 120.20 & 553 & 504.00 &  8.86\\
instance n=1000 105.alb & 1 & 0 & Solution & 120.19 & 548 & 499.00 &  8.94\\
instance n=1000 106.alb & 1 & 0 & Solution & 120.22 & 556 & 499.00 & 10.25\\
instance n=1000 107.alb & 1 & 0 & Solution & 120.20 & 540 & 496.00 &  8.15\\
instance n=1000 108.alb & 1 & 0 & Solution & 120.18 & 545 & 498.00 &  8.62\\
instance n=1000 109.alb & 1 & 0 & Solution & 120.21 & 549 & 500.00 &  8.93\\
instance n=1000 11.alb & 1 & 0 & Solution & 120.07 & 135 & 134.00 &  0.74\\
instance n=1000 110.alb & 1 & 0 & Solution & 120.21 & 555 & 501.00 &  9.73\\
instance n=1000 111.alb & 1 & 0 & Solution & 120.25 & 546 & 500.00 &  8.42\\
instance n=1000 112.alb & 1 & 0 & Solution & 120.22 & 548 & 499.00 &  8.94\\
instance n=1000 113.alb & 1 & 0 & Solution & 120.17 & 540 & 495.00 &  8.33\\
instance n=1000 114.alb & 1 & 0 & Solution & 120.20 & 550 & 502.00 &  8.73\\
instance n=1000 115.alb & 1 & 0 & Solution & 120.19 & 539 & 498.00 &  7.61\\
instance n=1000 116.alb & 1 & 0 & Solution & 120.19 & 545 & 496.00 &  8.99\\
instance n=1000 117.alb & 1 & 0 & Solution & 120.20 & 552 & 500.00 &  9.42\\
instance n=1000 118.alb & 1 & 0 & Solution & 120.30 & 563 & 509.00 &  9.59\\
instance n=1000 119.alb & 1 & 0 & Solution & 120.25 & 529 & 496.00 &  6.24\\
instance n=1000 12.alb & 1 & 0 & Solution & 120.05 & 135 & 134.00 &  0.74\\
instance n=1000 120.alb & 1 & 0 & Solution & 120.19 & 549 & 502.00 &  8.56\\
instance n=1000 121.alb & 1 & 0 & Solution & 120.24 & 541 & 496.00 &  8.32\\
instance n=1000 122.alb & 1 & 0 & Solution & 120.22 & 535 & 493.00 &  7.85\\
instance n=1000 123.alb & 1 & 0 & Solution & 120.19 & 555 & 504.00 &  9.19\\
instance n=1000 124.alb & 1 & 0 & Solution & 120.25 & 543 & 498.00 &  8.29\\
instance n=1000 125.alb & 1 & 0 & Solution & 120.21 & 545 & 499.00 &  8.44\\
instance n=1000 126.alb & 1 & 0 & Solution & 120.12 & 232 & 228.00 &  1.72\\
instance n=1000 127.alb & 1 & 0 & Solution & 120.12 & 224 & 221.00 &  1.34\\
instance n=1000 128.alb & 1 & 0 & Solution & 120.15 & 225 & 222.00 &  1.33\\
instance n=1000 129.alb & 1 & 0 & Solution & 120.10 & 226 & 223.00 &  1.33\\
instance n=1000 13.alb & 1 & 0 & Solution & 120.09 & 132 & 131.00 &  0.76\\
instance n=1000 130.alb & 1 & 0 & Solution & 120.18 & 225 & 221.00 &  1.78\\
instance n=1000 131.alb & 1 & 0 & Solution & 120.09 & 223 & 220.00 &  1.35\\
instance n=1000 132.alb & 1 & 0 & Solution & 120.10 & 218 & 214.00 &  1.83\\
instance n=1000 133.alb & 1 & 0 & Solution & 120.14 & 229 & 226.00 &  1.31\\
instance n=1000 134.alb & 1 & 0 & Solution & 120.14 & 219 & 215.00 &  1.83\\
instance n=1000 135.alb & 1 & 0 & Solution & 120.12 & 229 & 225.00 &  1.75\\
instance n=1000 136.alb & 1 & 0 & Solution & 120.25 & 232 & 228.00 &  1.72\\
instance n=1000 137.alb & 1 & 0 & Solution & 120.10 & 216 & 213.00 &  1.39\\
instance n=1000 138.alb & 1 & 0 & Solution & 120.11 & 225 & 221.00 &  1.78\\
instance n=1000 139.alb & 1 & 0 & Solution & 120.13 & 227 & 224.00 &  1.32\\
instance n=1000 14.alb & 1 & 0 & Solution & 120.05 & 138 & 136.00 &  1.45\\
instance n=1000 140.alb & 1 & 0 & Solution & 120.13 & 230 & 226.00 &  1.74\\
instance n=1000 141.alb & 1 & 0 & Solution & 120.16 & 218 & 215.00 &  1.38\\
instance n=1000 142.alb & 1 & 0 & Solution & 120.11 & 223 & 220.00 &  1.35\\
instance n=1000 143.alb & 1 & 0 & Solution & 120.11 & 216 & 213.00 &  1.39\\
instance n=1000 144.alb & 1 & 0 & Solution & 120.09 & 221 & 217.00 &  1.81\\
instance n=1000 145.alb & 1 & 0 & Solution & 120.15 & 223 & 220.00 &  1.35\\
instance n=1000 146.alb & 1 & 0 & Solution & 120.13 & 223 & 219.00 &  1.79\\
instance n=1000 147.alb & 1 & 0 & Solution & 120.13 & 234 & 229.00 &  2.14\\
instance n=1000 148.alb & 1 & 0 & Solution & 120.14 & 223 & 219.00 &  1.79\\
instance n=1000 149.alb & 1 & 0 & Solution & 120.11 & 241 & 237.00 &  1.66\\
instance n=1000 15.alb & 1 & 0 & Solution & 120.08 & 137 & 136.00 &  0.73\\
instance n=1000 150.alb & 1 & 0 & Solution & 120.10 & 225 & 222.00 &  1.33\\
instance n=1000 151.alb & 1 & 0 & Solution & 120.16 & 140 & 138.00 &  1.43\\
instance n=1000 152.alb & 1 & 0 & Solution & 120.13 & 138 & 136.00 &  1.45\\
instance n=1000 153.alb & 1 & 0 & Solution & 120.06 & 139 & 137.00 &  1.44\\
instance n=1000 154.alb & 1 & 0 & Solution & 120.22 & 142 & 140.00 &  1.41\\
instance n=1000 155.alb & 1 & 0 & Solution & 120.11 & 141 & 139.00 &  1.42\\
instance n=1000 156.alb & 1 & 0 & Solution & 120.22 & 143 & 141.00 &  1.40\\
instance n=1000 157.alb & 1 & 0 & Solution & 120.15 & 141 & 140.00 &  0.71\\
instance n=1000 158.alb & 1 & 0 & Solution & 120.12 & 137 & 136.00 &  0.73\\
instance n=1000 159.alb & 1 & 0 & Solution & 120.09 & 140 & 138.00 &  1.43\\
instance n=1000 16.alb & 1 & 0 & Solution & 120.06 & 138 & 137.00 &  0.72\\
instance n=1000 160.alb & 1 & 0 & Solution & 120.12 & 140 & 138.00 &  1.43\\
instance n=1000 161.alb & 1 & 0 & Solution & 120.10 & 134 & 133.00 &  0.75\\
instance n=1000 162.alb & 1 & 0 & Solution & 120.06 & 137 & 136.00 &  0.73\\
instance n=1000 163.alb & 1 & 0 & Solution & 120.16 & 141 & 139.00 &  1.42\\
instance n=1000 164.alb & 1 & 0 & Solution & 120.08 & 143 & 141.00 &  1.40\\
instance n=1000 165.alb & 1 & 0 & Solution & 120.15 & 137 & 135.00 &  1.46\\
instance n=1000 166.alb & 1 & 0 & Solution & 120.08 & 141 & 139.00 &  1.42\\
instance n=1000 167.alb & 1 & 0 & Solution & 120.06 & 141 & 139.00 &  1.42\\
instance n=1000 168.alb & 1 & 0 & Solution & 120.05 & 140 & 138.00 &  1.43\\
instance n=1000 169.alb & 1 & 0 & Solution & 120.13 & 136 & 134.00 &  1.47\\
instance n=1000 17.alb & 1 & 0 & Solution & 120.06 & 136 & 135.00 &  0.74\\
instance n=1000 170.alb & 1 & 0 & Solution & 120.11 & 136 & 134.00 &  1.47\\
instance n=1000 171.alb & 1 & 0 & Solution & 120.06 & 139 & 137.00 &  1.44\\
instance n=1000 172.alb & 1 & 0 & Solution & 120.13 & 136 & 135.00 &  0.74\\
instance n=1000 173.alb & 1 & 0 & Solution & 120.15 & 137 & 135.00 &  1.46\\
instance n=1000 174.alb & 1 & 0 & Solution & 120.17 & 138 & 136.00 &  1.45\\
instance n=1000 175.alb & 1 & 0 & Solution & 120.13 & 140 & 138.00 &  1.43\\
instance n=1000 176.alb & 1 & 0 & Solution & 120.22 & 557 & 500.00 & 10.23\\
instance n=1000 177.alb & 1 & 0 & Solution & 120.19 & 552 & 499.00 &  9.60\\
instance n=1000 178.alb & 1 & 0 & Solution & 120.22 & 566 & 506.00 & 10.60\\
instance n=1000 179.alb & 1 & 0 & Solution & 120.24 & 564 & 505.00 & 10.46\\
instance n=1000 18.alb & 1 & 0 & Solution & 120.07 & 135 & 134.00 &  0.74\\
instance n=1000 180.alb & 1 & 0 & Solution & 120.20 & 559 & 503.00 & 10.02\\
instance n=1000 181.alb & 1 & 0 & Solution & 120.24 & 561 & 505.00 &  9.98\\
instance n=1000 182.alb & 1 & 0 & Solution & 120.24 & 557 & 502.00 &  9.87\\
instance n=1000 183.alb & 1 & 0 & Solution & 120.21 & 552 & 500.00 &  9.42\\
instance n=1000 184.alb & 1 & 0 & Solution & 120.22 & 559 & 502.00 & 10.20\\
instance n=1000 185.alb & 1 & 0 & Solution & 120.24 & 560 & 503.00 & 10.18\\
instance n=1000 186.alb & 1 & 0 & Solution & 120.22 & 552 & 500.00 &  9.42\\
instance n=1000 187.alb & 1 & 0 & Solution & 120.21 & 565 & 505.00 & 10.62\\
instance n=1000 188.alb & 1 & 0 & Solution & 120.24 & 552 & 498.00 &  9.78\\
instance n=1000 189.alb & 1 & 0 & Solution & 120.22 & 552 & 498.00 &  9.78\\
instance n=1000 19.alb & 1 & 0 & Solution & 120.06 & 138 & 137.00 &  0.72\\
instance n=1000 190.alb & 1 & 0 & Solution & 120.23 & 556 & 501.00 &  9.89\\
instance n=1000 191.alb & 1 & 0 & Solution & 120.21 & 553 & 501.00 &  9.40\\
instance n=1000 192.alb & 1 & 0 & Solution & 120.19 & 556 & 501.00 &  9.89\\
instance n=1000 193.alb & 1 & 0 & Solution & 120.20 & 559 & 503.00 & 10.02\\
instance n=1000 194.alb & 1 & 0 & Solution & 120.23 & 560 & 502.00 & 10.36\\
instance n=1000 195.alb & 1 & 0 & Solution & 120.20 & 562 & 502.00 & 10.68\\
instance n=1000 196.alb & 1 & 0 & Solution & 120.21 & 559 & 500.00 & 10.55\\
instance n=1000 197.alb & 1 & 0 & Solution & 120.19 & 546 & 496.00 &  9.16\\
instance n=1000 198.alb & 1 & 0 & Solution & 120.19 & 562 & 503.00 & 10.50\\
instance n=1000 199.alb & 1 & 0 & Solution & 120.24 & 541 & 495.00 &  8.50\\
instance n=1000 2.alb & 1 & 0 & Solution & 120.07 & 138 & 137.00 &  0.72\\
instance n=1000 20.alb & 1 & 0 & Solution & 120.05 & 139 & 138.00 &  0.72\\
instance n=1000 200.alb & 1 & 0 & Solution & 120.23 & 550 & 498.00 &  9.45\\
instance n=1000 201.alb & 1 & 0 & Solution & 120.20 & 233 & 229.00 &  1.72\\
instance n=1000 202.alb & 1 & 0 & Solution & 120.13 & 230 & 225.00 &  2.17\\
instance n=1000 203.alb & 1 & 0 & Solution & 120.10 & 234 & 229.00 &  2.14\\
instance n=1000 204.alb & 1 & 0 & Solution & 120.15 & 233 & 228.00 &  2.15\\
instance n=1000 205.alb & 1 & 0 & Solution & 120.25 & 234 & 229.00 &  2.14\\
instance n=1000 206.alb & 1 & 0 & Solution & 120.11 & 233 & 229.00 &  1.72\\
instance n=1000 207.alb & 1 & 0 & Solution & 120.13 & 234 & 230.00 &  1.71\\
instance n=1000 208.alb & 1 & 0 & Solution & 120.24 & 234 & 229.00 &  2.14\\
instance n=1000 209.alb & 1 & 0 & Solution & 120.13 & 233 & 228.00 &  2.15\\
instance n=1000 21.alb & 1 & 0 & Solution & 120.07 & 139 & 138.00 &  0.72\\
instance n=1000 210.alb & 1 & 0 & Solution & 120.13 & 229 & 224.00 &  2.18\\
instance n=1000 211.alb & 1 & 0 & Solution & 120.14 & 223 & 219.00 &  1.79\\
instance n=1000 212.alb & 1 & 0 & Solution & 120.12 & 221 & 217.00 &  1.81\\
instance n=1000 213.alb & 1 & 0 & Solution & 120.17 & 238 & 233.00 &  2.10\\
instance n=1000 214.alb & 1 & 0 & Solution & 120.16 & 230 & 225.00 &  2.17\\
instance n=1000 215.alb & 1 & 0 & Solution & 120.14 & 227 & 223.00 &  1.76\\
instance n=1000 216.alb & 1 & 0 & Solution & 120.10 & 225 & 221.00 &  1.78\\
instance n=1000 217.alb & 1 & 0 & Solution & 120.31 & 229 & 225.00 &  1.75\\
instance n=1000 218.alb & 1 & 0 & Solution & 120.14 & 223 & 219.00 &  1.79\\
instance n=1000 219.alb & 1 & 0 & Solution & 120.09 & 236 & 232.00 &  1.69\\
instance n=1000 22.alb & 1 & 0 & Solution & 120.05 & 139 & 137.00 &  1.44\\
instance n=1000 220.alb & 1 & 0 & Solution & 120.11 & 229 & 225.00 &  1.75\\
instance n=1000 221.alb & 1 & 0 & Solution & 120.11 & 236 & 231.00 &  2.12\\
instance n=1000 222.alb & 1 & 0 & Solution & 120.11 & 226 & 221.00 &  2.21\\
instance n=1000 223.alb & 1 & 0 & Solution & 120.17 & 226 & 221.00 &  2.21\\
instance n=1000 224.alb & 1 & 0 & Solution & 120.11 & 231 & 226.00 &  2.16\\
instance n=1000 225.alb & 1 & 0 & Solution & 120.24 & 234 & 229.00 &  2.14\\
instance n=1000 226.alb & 1 & 0 & Solution & 120.15 & 138 & 136.00 &  1.45\\
instance n=1000 227.alb & 1 & 0 & Solution & 120.11 & 140 & 138.00 &  1.43\\
instance n=1000 228.alb & 1 & 0 & Solution & 120.16 & 135 & 133.00 &  1.48\\
instance n=1000 229.alb & 1 & 0 & Solution & 120.17 & 136 & 134.00 &  1.47\\
instance n=1000 23.alb & 1 & 0 & Solution & 120.06 & 137 & 136.00 &  0.73\\
instance n=1000 230.alb & 1 & 0 & Solution & 120.15 & 134 & 131.00 &  2.24\\
instance n=1000 231.alb & 1 & 0 & Solution & 120.17 & 141 & 138.00 &  2.13\\
instance n=1000 232.alb & 1 & 0 & Solution & 120.11 & 135 & 133.00 &  1.48\\
instance n=1000 233.alb & 1 & 0 & Solution & 120.30 & 138 & 135.00 &  2.17\\
instance n=1000 234.alb & 1 & 0 & Solution & 120.07 & 139 & 137.00 &  1.44\\
instance n=1000 235.alb & 1 & 0 & Solution & 120.29 & 134 & 133.00 &  0.75\\
instance n=1000 236.alb & 1 & 0 & Solution & 120.16 & 138 & 136.00 &  1.45\\
instance n=1000 237.alb & 1 & 0 & Solution & 120.18 & 141 & 138.00 &  2.13\\
instance n=1000 238.alb & 1 & 0 & Solution & 120.13 & 140 & 138.00 &  1.43\\
instance n=1000 239.alb & 1 & 0 & Solution & 120.11 & 137 & 135.00 &  1.46\\
instance n=1000 24.alb & 1 & 0 & Solution & 120.06 & 141 & 140.00 &  0.71\\
instance n=1000 240.alb & 1 & 0 & Solution & 120.19 & 137 & 135.00 &  1.46\\
instance n=1000 241.alb & 1 & 0 & Solution & 120.15 & 140 & 138.00 &  1.43\\
instance n=1000 242.alb & 1 & 0 & Solution & 120.11 & 137 & 135.00 &  1.46\\
instance n=1000 243.alb & 1 & 0 & Solution & 120.09 & 139 & 137.00 &  1.44\\
instance n=1000 244.alb & 1 & 0 & Solution & 120.20 & 139 & 137.00 &  1.44\\
instance n=1000 245.alb & 1 & 0 & Solution & 120.10 & 137 & 135.00 &  1.46\\
instance n=1000 246.alb & 1 & 0 & Solution & 120.12 & 137 & 135.00 &  1.46\\
instance n=1000 247.alb & 1 & 0 & Solution & 120.22 & 141 & 138.00 &  2.13\\
instance n=1000 248.alb & 1 & 0 & Solution & 120.07 & 141 & 138.00 &  2.13\\
instance n=1000 249.alb & 1 & 0 & Solution & 120.20 & 141 & 138.00 &  2.13\\
instance n=1000 25.alb & 1 & 0 & Solution & 120.06 & 137 & 136.00 &  0.73\\
instance n=1000 250.alb & 1 & 0 & Solution & 120.35 & 142 & 140.00 &  1.41\\
instance n=1000 251.alb & 1 & 0 & Solution & 120.30 & 568 & 502.00 & 11.62\\
instance n=1000 252.alb & 1 & 0 & Solution & 120.24 & 567 & 501.00 & 11.64\\
instance n=1000 253.alb & 1 & 0 & Solution & 120.25 & 560 & 502.00 & 10.36\\
instance n=1000 254.alb & 1 & 0 & Solution & 120.20 & 563 & 501.00 & 11.01\\
instance n=1000 255.alb & 1 & 0 & Solution & 120.33 & 551 & 498.00 &  9.62\\
instance n=1000 256.alb & 1 & 0 & Solution & 120.24 & 558 & 495.00 & 11.29\\
instance n=1000 257.alb & 1 & 0 & Solution & 120.24 & 566 & 502.00 & 11.31\\
instance n=1000 258.alb & 1 & 0 & Solution & 120.35 & 557 & 497.00 & 10.77\\
instance n=1000 259.alb & 1 & 0 & Solution & 120.33 & 557 & 496.00 & 10.95\\
instance n=1000 26.alb & 1 & 0 & Solution & 120.19 & 555 & 502.00 &  9.55\\
instance n=1000 260.alb & 1 & 0 & Solution & 120.20 & 556 & 495.00 & 10.97\\
instance n=1000 261.alb & 1 & 0 & Solution & 120.26 & 564 & 501.00 & 11.17\\
instance n=1000 262.alb & 1 & 0 & Solution & 120.23 & 544 & 495.00 &  9.01\\
instance n=1000 263.alb & 1 & 0 & Solution & 120.23 & 561 & 499.00 & 11.05\\
instance n=1000 264.alb & 1 & 0 & Solution & 120.31 & 557 & 499.00 & 10.41\\
instance n=1000 265.alb & 1 & 0 & Solution & 120.24 & 579 & 506.00 & 12.61\\
instance n=1000 266.alb & 1 & 0 & Solution & 120.23 & 562 & 500.00 & 11.03\\
instance n=1000 267.alb & 1 & 0 & Solution & 120.25 & 571 & 506.00 & 11.38\\
instance n=1000 268.alb & 1 & 0 & Solution & 120.24 & 554 & 497.00 & 10.29\\
instance n=1000 269.alb & 1 & 0 & Solution & 120.40 & 558 & 500.00 & 10.39\\
instance n=1000 27.alb & 1 & 0 & Solution & 120.19 & 551 & 502.00 &  8.89\\
instance n=1000 270.alb & 1 & 0 & Solution & 120.21 & 581 & 508.00 & 12.56\\
instance n=1000 271.alb & 1 & 0 & Solution & 120.38 & 553 & 497.00 & 10.13\\
instance n=1000 272.alb & 1 & 0 & Solution & 120.24 & 567 & 502.00 & 11.46\\
instance n=1000 273.alb & 1 & 0 & Solution & 120.19 & 563 & 500.00 & 11.19\\
instance n=1000 274.alb & 1 & 0 & Solution & 120.22 & 559 & 496.00 & 11.27\\
instance n=1000 275.alb & 1 & 0 & Solution & 120.21 & 565 & 504.00 & 10.80\\
instance n=1000 276.alb & 1 & 0 & Solution & 120.09 & 223 & 217.00 &  2.69\\
instance n=1000 277.alb & 1 & 0 & Solution & 120.38 & 230 & 225.00 &  2.17\\
instance n=1000 278.alb & 1 & 0 & Solution & 120.22 & 226 & 220.00 &  2.65\\
instance n=1000 279.alb & 1 & 0 & Solution & 120.21 & 220 & 215.00 &  2.27\\
instance n=1000 28.alb & 1 & 0 & Solution & 120.18 & 538 & 497.00 &  7.62\\
instance n=1000 280.alb & 1 & 0 & Solution & 120.10 & 231 & 226.00 &  2.16\\
instance n=1000 281.alb & 1 & 0 & Solution & 120.08 & 225 & 219.00 &  2.67\\
instance n=1000 282.alb & 1 & 0 & Solution & 120.10 & 220 & 214.00 &  2.73\\
instance n=1000 283.alb & 1 & 0 & Solution & 120.25 & 230 & 224.00 &  2.61\\
instance n=1000 284.alb & 1 & 0 & Solution & 120.11 & 222 & 217.00 &  2.25\\
instance n=1000 285.alb & 1 & 0 & Solution & 120.14 & 227 & 221.00 &  2.64\\
instance n=1000 286.alb & 1 & 0 & Solution & 120.14 & 227 & 221.00 &  2.64\\
instance n=1000 287.alb & 1 & 0 & Solution & 120.10 & 230 & 224.00 &  2.61\\
instance n=1000 288.alb & 1 & 0 & Solution & 120.16 & 225 & 219.00 &  2.67\\
instance n=1000 289.alb & 1 & 0 & Solution & 120.13 & 225 & 220.00 &  2.22\\
instance n=1000 29.alb & 1 & 0 & Solution & 120.19 & 542 & 498.00 &  8.12\\
instance n=1000 290.alb & 1 & 0 & Solution & 120.24 & 228 & 222.00 &  2.63\\
instance n=1000 291.alb & 1 & 0 & Solution & 120.16 & 231 & 225.00 &  2.60\\
instance n=1000 292.alb & 1 & 0 & Solution & 120.11 & 232 & 226.00 &  2.59\\
instance n=1000 293.alb & 1 & 0 & Solution & 120.14 & 231 & 225.00 &  2.60\\
instance n=1000 294.alb & 1 & 0 & Solution & 120.19 & 236 & 230.00 &  2.54\\
instance n=1000 295.alb & 1 & 0 & Solution & 120.16 & 233 & 227.00 &  2.58\\
instance n=1000 296.alb & 1 & 0 & Solution & 120.21 & 213 & 208.00 &  2.35\\
instance n=1000 297.alb & 1 & 0 & Solution & 120.13 & 222 & 217.00 &  2.25\\
instance n=1000 298.alb & 1 & 0 & Solution & 120.14 & 219 & 214.00 &  2.28\\
instance n=1000 299.alb & 1 & 0 & Solution & 120.25 & 232 & 226.00 &  2.59\\
instance n=1000 3.alb & 1 & 0 & Solution & 120.10 & 138 & 136.00 &  1.45\\
instance n=1000 30.alb & 1 & 0 & Solution & 120.22 & 559 & 506.00 &  9.48\\
instance n=1000 300.alb & 1 & 0 & Solution & 120.21 & 234 & 228.00 &  2.56\\
instance n=1000 301.alb & 1 & 0 & Solution & 120.10 & 138 & 137.00 &  0.72\\
instance n=1000 302.alb & 1 & 0 & Solution & 120.15 & 140 & 139.00 &  0.71\\
instance n=1000 303.alb & 1 & 0 & Solution & 120.14 & 140 & 138.00 &  1.43\\
instance n=1000 304.alb & 1 & 0 & Solution & 120.11 & 138 & 136.00 &  1.45\\
instance n=1000 305.alb & 1 & 0 & Solution & 120.20 & 141 & 140.00 &  0.71\\
instance n=1000 306.alb & 1 & 0 & Solution & 120.22 & 136 & 135.00 &  0.74\\
instance n=1000 307.alb & 1 & 0 & Solution & 120.23 & 137 & 136.00 &  0.73\\
instance n=1000 308.alb & 1 & 0 & Solution & 120.13 & 138 & 137.00 &  0.72\\
instance n=1000 309.alb & 1 & 0 & Solution & 120.24 & 136 & 135.00 &  0.74\\
instance n=1000 31.alb & 1 & 0 & Solution & 120.19 & 555 & 506.00 &  8.83\\
instance n=1000 310.alb & 1 & 0 & Solution & 120.16 & 143 & 141.00 &  1.40\\
instance n=1000 311.alb & 1 & 0 & Solution & 120.24 & 141 & 139.00 &  1.42\\
instance n=1000 312.alb & 1 & 0 & Solution & 120.13 & 136 & 135.00 &  0.74\\
instance n=1000 313.alb & 1 & 0 & Solution & 120.15 & 139 & 138.00 &  0.72\\
instance n=1000 314.alb & 1 & 0 & Solution & 120.20 & 143 & 142.00 &  0.70\\
instance n=1000 315.alb & 1 & 0 & Solution & 120.36 & 138 & 136.00 &  1.45\\
instance n=1000 316.alb & 1 & 0 & Solution & 120.23 & 139 & 137.00 &  1.44\\
instance n=1000 317.alb & 1 & 0 & Solution & 120.23 & 137 & 136.00 &  0.73\\
instance n=1000 318.alb & 1 & 0 & Solution & 120.09 & 139 & 138.00 &  0.72\\
instance n=1000 319.alb & 1 & 0 & Solution & 120.09 & 142 & 140.00 &  1.41\\
instance n=1000 32.alb & 1 & 0 & Solution & 120.20 & 542 & 502.00 &  7.38\\
instance n=1000 320.alb & 1 & 0 & Solution & 120.09 & 142 & 141.00 &  0.70\\
instance n=1000 321.alb & 1 & 0 & Solution & 120.19 & 141 & 140.00 &  0.71\\
instance n=1000 322.alb & 1 & 0 & Solution & 120.20 & 140 & 139.00 &  0.71\\
instance n=1000 323.alb & 1 & 0 & Solution & 120.10 & 140 & 138.00 &  1.43\\
instance n=1000 324.alb & 1 & 0 & Solution & 120.11 & 141 & 140.00 &  0.71\\
instance n=1000 325.alb & 1 & 0 & Solution & 120.11 & 140 & 138.00 &  1.43\\
instance n=1000 326.alb & 1 & 0 & Solution & 120.35 & 541 & 496.00 &  8.32\\
instance n=1000 327.alb & 1 & 0 & Solution & 120.22 & 552 & 503.00 &  8.88\\
instance n=1000 328.alb & 1 & 0 & Solution & 120.18 & 545 & 500.00 &  8.26\\
instance n=1000 329.alb & 1 & 0 & Solution & 120.27 & 554 & 502.00 &  9.39\\
instance n=1000 33.alb & 1 & 0 & Solution & 120.18 & 548 & 501.00 &  8.58\\
instance n=1000 330.alb & 1 & 0 & Solution & 120.23 & 538 & 498.00 &  7.43\\
instance n=1000 331.alb & 1 & 0 & Solution & 120.22 & 547 & 498.00 &  8.96\\
instance n=1000 332.alb & 1 & 0 & Solution & 120.25 & 535 & 495.00 &  7.48\\
instance n=1000 333.alb & 1 & 0 & Solution & 120.22 & 553 & 499.00 &  9.76\\
instance n=1000 334.alb & 1 & 0 & Solution & 120.27 & 540 & 498.00 &  7.78\\
instance n=1000 335.alb & 1 & 0 & Solution & 120.21 & 544 & 496.00 &  8.82\\
instance n=1000 336.alb & 1 & 0 & Solution & 120.47 & 534 & 497.00 &  6.93\\
instance n=1000 337.alb & 1 & 0 & Solution & 120.22 & 551 & 501.00 &  9.07\\
instance n=1000 338.alb & 1 & 0 & Solution & 120.23 & 553 & 502.00 &  9.22\\
instance n=1000 339.alb & 1 & 0 & Solution & 120.35 & 555 & 500.00 &  9.91\\
instance n=1000 34.alb & 1 & 0 & Solution & 120.19 & 563 & 507.00 &  9.95\\
instance n=1000 340.alb & 1 & 0 & Solution & 120.46 & 563 & 505.00 & 10.30\\
instance n=1000 341.alb & 1 & 0 & Solution & 120.25 & 552 & 503.00 &  8.88\\
instance n=1000 342.alb & 1 & 0 & Solution & 120.20 & 549 & 500.00 &  8.93\\
instance n=1000 343.alb & 1 & 0 & Solution & 120.29 & 554 & 500.00 &  9.75\\
instance n=1000 344.alb & 1 & 0 & Solution & 120.26 & 545 & 500.00 &  8.26\\
instance n=1000 345.alb & 1 & 0 & Solution & 120.27 & 552 & 502.00 &  9.06\\
instance n=1000 346.alb & 1 & 0 & Solution & 120.25 & 551 & 501.00 &  9.07\\
instance n=1000 347.alb & 1 & 0 & Solution & 120.37 & 547 & 498.00 &  8.96\\
instance n=1000 348.alb & 1 & 0 & Solution & 120.33 & 566 & 506.00 & 10.60\\
instance n=1000 349.alb & 1 & 0 & Solution & 120.42 & 558 & 503.00 &  9.86\\
instance n=1000 35.alb & 1 & 0 & Solution & 120.19 & 544 & 501.00 &  7.90\\
instance n=1000 350.alb & 1 & 0 & Solution & 120.18 & 534 & 496.00 &  7.12\\
instance n=1000 351.alb & 1 & 0 & Solution & 120.19 & 231 & 227.00 &  1.73\\
instance n=1000 352.alb & 1 & 0 & Solution & 120.13 & 231 & 227.00 &  1.73\\
instance n=1000 353.alb & 1 & 0 & Solution & 120.12 & 221 & 217.00 &  1.81\\
instance n=1000 354.alb & 1 & 0 & Solution & 120.24 & 226 & 222.00 &  1.77\\
instance n=1000 355.alb & 1 & 0 & Solution & 120.29 & 224 & 220.00 &  1.79\\
instance n=1000 356.alb & 1 & 0 & Solution & 120.13 & 230 & 226.00 &  1.74\\
instance n=1000 357.alb & 1 & 0 & Solution & 120.26 & 217 & 213.00 &  1.84\\
instance n=1000 358.alb & 1 & 0 & Solution & 120.12 & 223 & 219.00 &  1.79\\
instance n=1000 359.alb & 1 & 0 & Solution & 120.11 & 226 & 222.00 &  1.77\\
instance n=1000 36.alb & 1 & 0 & Solution & 120.19 & 538 & 497.00 &  7.62\\
instance n=1000 360.alb & 1 & 0 & Solution & 120.35 & 233 & 229.00 &  1.72\\
instance n=1000 361.alb & 1 & 0 & Solution & 120.11 & 219 & 215.00 &  1.83\\
instance n=1000 362.alb & 1 & 0 & Solution & 120.13 & 226 & 223.00 &  1.33\\
instance n=1000 363.alb & 1 & 0 & Solution & 120.10 & 218 & 215.00 &  1.38\\
instance n=1000 364.alb & 1 & 0 & Solution & 120.29 & 225 & 221.00 &  1.78\\
instance n=1000 365.alb & 1 & 0 & Solution & 120.30 & 231 & 227.00 &  1.73\\
instance n=1000 366.alb & 1 & 0 & Solution & 120.48 & 232 & 228.00 &  1.72\\
instance n=1000 367.alb & 1 & 0 & Solution & 120.11 & 231 & 227.00 &  1.73\\
instance n=1000 368.alb & 1 & 0 & Solution & 120.21 & 230 & 226.00 &  1.74\\
instance n=1000 369.alb & 1 & 0 & Solution & 120.29 & 224 & 220.00 &  1.79\\
instance n=1000 37.alb & 1 & 0 & Solution & 120.18 & 559 & 506.00 &  9.48\\
instance n=1000 370.alb & 1 & 0 & Solution & 120.20 & 227 & 223.00 &  1.76\\
instance n=1000 371.alb & 1 & 0 & Solution & 120.31 & 223 & 220.00 &  1.35\\
instance n=1000 372.alb & 1 & 0 & Solution & 120.35 & 234 & 230.00 &  1.71\\
instance n=1000 373.alb & 1 & 0 & Solution & 120.24 & 223 & 219.00 &  1.79\\
instance n=1000 374.alb & 1 & 0 & Solution & 120.19 & 222 & 219.00 &  1.35\\
instance n=1000 375.alb & 1 & 0 & Solution & 120.30 & 231 & 227.00 &  1.73\\
instance n=1000 376.alb & 1 & 0 & Solution & 120.19 & 134 & 132.00 &  1.49\\
instance n=1000 377.alb & 1 & 0 & Solution & 120.22 & 138 & 137.00 &  0.72\\
instance n=1000 378.alb & 1 & 0 & Solution & 120.24 & 136 & 134.00 &  1.47\\
instance n=1000 379.alb & 1 & 0 & Solution & 120.09 & 139 & 137.00 &  1.44\\
instance n=1000 38.alb & 1 & 0 & Solution & 120.20 & 557 & 504.00 &  9.52\\
instance n=1000 380.alb & 1 & 0 & Solution & 120.17 & 136 & 134.00 &  1.47\\
instance n=1000 381.alb & 1 & 0 & Solution & 120.21 & 140 & 138.00 &  1.43\\
instance n=1000 382.alb & 1 & 0 & Solution & 120.28 & 133 & 131.00 &  1.50\\
instance n=1000 383.alb & 1 & 0 & Solution & 120.19 & 141 & 138.00 &  2.13\\
instance n=1000 384.alb & 1 & 0 & Solution & 120.48 & 141 & 139.00 &  1.42\\
instance n=1000 385.alb & 1 & 0 & Solution & 120.15 & 137 & 135.00 &  1.46\\
instance n=1000 386.alb & 1 & 0 & Solution & 120.08 & 141 & 139.00 &  1.42\\
instance n=1000 387.alb & 1 & 0 & Solution & 120.07 & 139 & 137.00 &  1.44\\
instance n=1000 388.alb & 1 & 0 & Solution & 120.06 & 138 & 137.00 &  0.72\\
instance n=1000 389.alb & 1 & 0 & Solution & 120.34 & 138 & 136.00 &  1.45\\
instance n=1000 39.alb & 1 & 0 & Solution & 120.21 & 560 & 507.00 &  9.46\\
instance n=1000 390.alb & 1 & 0 & Solution & 120.27 & 138 & 136.00 &  1.45\\
instance n=1000 391.alb & 1 & 0 & Solution & 120.27 & 137 & 135.00 &  1.46\\
instance n=1000 392.alb & 1 & 0 & Solution & 120.15 & 137 & 136.00 &  0.73\\
instance n=1000 393.alb & 1 & 0 & Solution & 120.17 & 138 & 136.00 &  1.45\\
instance n=1000 394.alb & 1 & 0 & Solution & 120.52 & 140 & 138.00 &  1.43\\
instance n=1000 395.alb & 1 & 0 & Solution & 120.25 & 141 & 139.00 &  1.42\\
instance n=1000 396.alb & 1 & 0 & Solution & 120.21 & 138 & 136.00 &  1.45\\
instance n=1000 397.alb & 1 & 0 & Solution & 120.29 & 142 & 140.00 &  1.41\\
instance n=1000 398.alb & 1 & 0 & Solution & 120.19 & 136 & 134.00 &  1.47\\
instance n=1000 399.alb & 1 & 0 & Solution & 120.30 & 140 & 139.00 &  0.71\\
instance n=1000 4.alb & 1 & 0 & Solution & 120.09 & 139 & 138.00 &  0.72\\
instance n=1000 40.alb & 1 & 0 & Solution & 120.17 & 531 & 496.00 &  6.59\\
instance n=1000 400.alb & 1 & 0 & Solution & 120.45 & 142 & 140.00 &  1.41\\
instance n=1000 401.alb & 1 & 0 & Solution & 120.26 & 554 & 497.00 & 10.29\\
instance n=1000 402.alb & 1 & 0 & Solution & 120.35 & 559 & 500.00 & 10.55\\
instance n=1000 403.alb & 1 & 0 & Solution & 120.23 & 555 & 500.00 &  9.91\\
instance n=1000 404.alb & 1 & 0 & Solution & 120.25 & 555 & 500.00 &  9.91\\
instance n=1000 405.alb & 1 & 0 & Solution & 120.29 & 564 & 501.00 & 11.17\\
instance n=1000 406.alb & 1 & 0 & Solution & 120.40 & 547 & 495.00 &  9.51\\
instance n=1000 407.alb & 1 & 0 & Solution & 120.39 & 559 & 498.00 & 10.91\\
instance n=1000 408.alb & 1 & 0 & Solution & 120.32 & 564 & 501.00 & 11.17\\
instance n=1000 409.alb & 1 & 0 & Solution & 120.32 & 565 & 504.00 & 10.80\\
instance n=1000 41.alb & 1 & 0 & Solution & 120.20 & 543 & 500.00 &  7.92\\
instance n=1000 410.alb & 1 & 0 & Solution & 120.46 & 575 & 505.00 & 12.17\\
instance n=1000 411.alb & 1 & 0 & Solution & 120.31 & 559 & 498.00 & 10.91\\
instance n=1000 412.alb & 1 & 0 & Solution & 120.24 & 558 & 499.00 & 10.57\\
instance n=1000 413.alb & 1 & 0 & Solution & 120.25 & 564 & 503.00 & 10.82\\
instance n=1000 414.alb & 1 & 0 & Solution & 120.29 & 558 & 501.00 & 10.22\\
instance n=1000 415.alb & 1 & 0 & Solution & 120.23 & 559 & 501.00 & 10.38\\
instance n=1000 416.alb & 1 & 0 & Solution & 120.24 & 564 & 502.00 & 10.99\\
instance n=1000 417.alb & 1 & 0 & Solution & 120.33 & 585 & 512.00 & 12.48\\
instance n=1000 418.alb & 1 & 0 & Solution & 120.24 & 558 & 501.00 & 10.22\\
instance n=1000 419.alb & 1 & 0 & Solution & 120.27 & 579 & 510.00 & 11.92\\
instance n=1000 42.alb & 1 & 0 & Solution & 120.19 & 533 & 497.00 &  6.75\\
instance n=1000 420.alb & 1 & 0 & Solution & 120.18 & 561 & 501.00 & 10.70\\
instance n=1000 421.alb & 1 & 0 & Solution & 120.21 & 556 & 499.00 & 10.25\\
instance n=1000 422.alb & 1 & 0 & Solution & 120.25 & 552 & 495.00 & 10.33\\
instance n=1000 423.alb & 1 & 0 & Solution & 120.36 & 562 & 500.00 & 11.03\\
instance n=1000 424.alb & 1 & 0 & Solution & 120.34 & 550 & 495.00 & 10.00\\
instance n=1000 425.alb & 1 & 0 & Solution & 120.26 & 565 & 504.00 & 10.80\\
instance n=1000 426.alb & 1 & 0 & Solution & 120.10 & 229 & 224.00 &  2.18\\
instance n=1000 427.alb & 1 & 0 & Solution & 120.15 & 235 & 229.00 &  2.55\\
instance n=1000 428.alb & 1 & 0 & Solution & 120.14 & 228 & 224.00 &  1.75\\
instance n=1000 429.alb & 1 & 0 & Solution & 120.12 & 240 & 235.00 &  2.08\\
instance n=1000 43.alb & 1 & 0 & Solution & 120.17 & 534 & 496.00 &  7.12\\
instance n=1000 430.alb & 1 & 0 & Solution & 120.34 & 224 & 220.00 &  1.79\\
instance n=1000 431.alb & 1 & 0 & Solution & 120.21 & 234 & 230.00 &  1.71\\
instance n=1000 432.alb & 1 & 0 & Solution & 120.15 & 232 & 227.00 &  2.16\\
instance n=1000 433.alb & 1 & 0 & Solution & 120.16 & 234 & 229.00 &  2.14\\
instance n=1000 434.alb & 1 & 0 & Solution & 120.40 & 215 & 212.00 &  1.40\\
instance n=1000 435.alb & 1 & 0 & Solution & 120.40 & 232 & 227.00 &  2.16\\
instance n=1000 436.alb & 1 & 0 & Solution & 120.13 & 231 & 226.00 &  2.16\\
instance n=1000 437.alb & 1 & 0 & Solution & 120.44 & 226 & 222.00 &  1.77\\
instance n=1000 438.alb & 1 & 0 & Solution & 120.13 & 226 & 221.00 &  2.21\\
instance n=1000 439.alb & 1 & 0 & Solution & 120.20 & 229 & 225.00 &  1.75\\
instance n=1000 44.alb & 1 & 0 & Solution & 120.16 & 550 & 502.00 &  8.73\\
instance n=1000 440.alb & 1 & 0 & Solution & 120.56 & 230 & 225.00 &  2.17\\
instance n=1000 441.alb & 1 & 0 & Solution & 120.20 & 226 & 221.00 &  2.21\\
instance n=1000 442.alb & 1 & 0 & Solution & 120.12 & 235 & 230.00 &  2.13\\
instance n=1000 443.alb & 1 & 0 & Solution & 120.34 & 222 & 217.00 &  2.25\\
instance n=1000 444.alb & 1 & 0 & Solution & 120.23 & 227 & 222.00 &  2.20\\
instance n=1000 445.alb & 1 & 0 & Solution & 120.28 & 235 & 229.00 &  2.55\\
instance n=1000 446.alb & 1 & 0 & Solution & 120.35 & 232 & 228.00 &  1.72\\
instance n=1000 447.alb & 1 & 0 & Solution & 120.14 & 227 & 221.00 &  2.64\\
instance n=1000 448.alb & 1 & 0 & Solution & 120.22 & 226 & 222.00 &  1.77\\
instance n=1000 449.alb & 1 & 0 & Solution & 120.25 & 238 & 232.00 &  2.52\\
instance n=1000 45.alb & 1 & 0 & Solution & 120.18 & 524 & 492.00 &  6.11\\
instance n=1000 450.alb & 1 & 0 & Solution & 120.33 & 225 & 220.00 &  2.22\\
instance n=1000 451.alb & 1 & 0 & Solution & 120.23 & 140 & 136.00 &  2.86\\
instance n=1000 452.alb & 1 & 0 & Solution & 120.10 & 134 & 132.00 &  1.49\\
instance n=1000 453.alb & 1 & 0 & Solution & 120.41 & 141 & 138.00 &  2.13\\
instance n=1000 454.alb & 1 & 0 & Solution & 120.19 & 142 & 139.00 &  2.11\\
instance n=1000 455.alb & 1 & 0 & Solution & 120.68 & 139 & 136.00 &  2.16\\
instance n=1000 456.alb & 1 & 0 & Solution & 120.14 & 138 & 135.00 &  2.17\\
instance n=1000 457.alb & 1 & 0 & Solution & 120.11 & 140 & 137.00 &  2.14\\
instance n=1000 458.alb & 1 & 0 & Solution & 120.26 & 137 & 135.00 &  1.46\\
instance n=1000 459.alb & 1 & 0 & Solution & 120.33 & 140 & 137.00 &  2.14\\
instance n=1000 46.alb & 1 & 0 & Solution & 120.19 & 538 & 498.00 &  7.43\\
instance n=1000 460.alb & 1 & 0 & Solution & 120.13 & 141 & 138.00 &  2.13\\
instance n=1000 461.alb & 1 & 0 & Solution & 120.34 & 140 & 137.00 &  2.14\\
instance n=1000 462.alb & 1 & 0 & Solution & 120.59 & 139 & 136.00 &  2.16\\
instance n=1000 463.alb & 1 & 0 & Solution & 120.10 & 138 & 136.00 &  1.45\\
instance n=1000 464.alb & 1 & 0 & Solution & 120.30 & 141 & 138.00 &  2.13\\
instance n=1000 465.alb & 1 & 0 & Solution & 120.42 & 141 & 138.00 &  2.13\\
instance n=1000 466.alb & 1 & 0 & Solution & 120.39 & 137 & 133.00 &  2.92\\
instance n=1000 467.alb & 1 & 0 & Solution & 120.20 & 140 & 138.00 &  1.43\\
instance n=1000 468.alb & 1 & 0 & Solution & 120.27 & 139 & 137.00 &  1.44\\
instance n=1000 469.alb & 1 & 0 & Solution & 120.55 & 140 & 137.00 &  2.14\\
instance n=1000 47.alb & 1 & 0 & Solution & 120.19 & 542 & 499.00 &  7.93\\
instance n=1000 470.alb & 1 & 0 & Solution & 120.17 & 138 & 135.00 &  2.17\\
instance n=1000 471.alb & 1 & 0 & Solution & 120.36 & 138 & 135.00 &  2.17\\
instance n=1000 472.alb & 1 & 0 & Solution & 120.19 & 142 & 140.00 &  1.41\\
instance n=1000 473.alb & 1 & 0 & Solution & 120.14 & 138 & 135.00 &  2.17\\
instance n=1000 474.alb & 1 & 0 & Solution & 120.10 & 139 & 136.00 &  2.16\\
instance n=1000 475.alb & 1 & 0 & Solution & 120.41 & 139 & 136.00 &  2.16\\
instance n=1000 476.alb & 1 & 0 & Solution & 120.22 & 574 & 503.00 & 12.37\\
instance n=1000 477.alb & 1 & 0 & Solution & 120.34 & 586 & 507.00 & 13.48\\
instance n=1000 478.alb & 1 & 0 & Solution & 120.29 & 596 & 510.00 & 14.43\\
instance n=1000 479.alb & 1 & 0 & Solution & 120.20 & 579 & 503.00 & 13.13\\
instance n=1000 48.alb & 1 & 0 & Solution & 120.20 & 565 & 508.00 & 10.09\\
instance n=1000 480.alb & 1 & 0 & Solution & 120.19 & 566 & 498.00 & 12.01\\
instance n=1000 481.alb & 1 & 0 & Solution & 120.30 & 581 & 504.00 & 13.25\\
instance n=1000 482.alb & 1 & 0 & Solution & 120.21 & 596 & 505.00 & 15.27\\
instance n=1000 483.alb & 1 & 0 & Solution & 120.33 & 569 & 499.00 & 12.30\\
instance n=1000 484.alb & 1 & 0 & Solution & 120.21 & 589 & 508.00 & 13.75\\
instance n=1000 485.alb & 1 & 0 & Solution & 120.21 & 586 & 505.00 & 13.82\\
instance n=1000 486.alb & 1 & 0 & Solution & 120.21 & 571 & 500.00 & 12.43\\
instance n=1000 487.alb & 1 & 0 & Solution & 120.22 & 584 & 502.00 & 14.04\\
instance n=1000 488.alb & 1 & 0 & Solution & 120.28 & 572 & 502.00 & 12.24\\
instance n=1000 489.alb & 1 & 0 & Solution & 120.43 & 568 & 498.00 & 12.32\\
instance n=1000 49.alb & 1 & 0 & Solution & 120.19 & 544 & 500.00 &  8.09\\
instance n=1000 490.alb & 1 & 0 & Solution & 120.52 & 573 & 501.00 & 12.57\\
instance n=1000 491.alb & 1 & 0 & Solution & 120.36 & 575 & 500.00 & 13.04\\
instance n=1000 492.alb & 1 & 0 & Solution & 120.40 & 585 & 509.00 & 12.99\\
instance n=1000 493.alb & 1 & 0 & Solution & 120.27 & 561 & 495.00 & 11.76\\
instance n=1000 494.alb & 1 & 0 & Solution & 120.31 & 571 & 500.00 & 12.43\\
instance n=1000 495.alb & 1 & 0 & Solution & 120.23 & 587 & 507.00 & 13.63\\
instance n=1000 496.alb & 1 & 0 & Solution & 120.20 & 553 & 495.00 & 10.49\\
instance n=1000 497.alb & 1 & 0 & Solution & 120.25 & 566 & 499.00 & 11.84\\
instance n=1000 498.alb & 1 & 0 & Solution & 120.23 & 588 & 506.00 & 13.95\\
instance n=1000 499.alb & 1 & 0 & Solution & 120.27 & 566 & 499.00 & 11.84\\
instance n=1000 5.alb & 1 & 0 & Solution & 120.07 & 136 & 135.00 &  0.74\\
instance n=1000 50.alb & 1 & 0 & Solution & 120.15 & 526 & 493.00 &  6.27\\
instance n=1000 500.alb & 1 & 0 & Solution & 120.23 & 571 & 503.00 & 11.91\\
instance n=1000 501.alb & 1 & 0 & Solution & 120.23 & 234 & 227.00 &  2.99\\
instance n=1000 502.alb & 1 & 0 & Solution & 120.29 & 232 & 224.00 &  3.45\\
instance n=1000 503.alb & 1 & 0 & Solution & 120.69 & 233 & 224.00 &  3.86\\
instance n=1000 504.alb & 1 & 0 & Solution & 120.44 & 236 & 227.00 &  3.81\\
instance n=1000 505.alb & 1 & 0 & Solution & 120.24 & 219 & 213.00 &  2.74\\
instance n=1000 506.alb & 1 & 0 & Solution & 120.26 & 230 & 223.00 &  3.04\\
instance n=1000 507.alb & 1 & 0 & Solution & 120.18 & 228 & 220.00 &  3.51\\
instance n=1000 508.alb & 1 & 0 & Solution & 120.36 & 226 & 219.00 &  3.10\\
instance n=1000 509.alb & 1 & 0 & Solution & 120.13 & 232 & 225.00 &  3.02\\
instance n=1000 51.alb & 1 & 0 & Solution & 120.13 & 229 & 226.00 &  1.31\\
instance n=1000 510.alb & 1 & 0 & Solution & 120.43 & 235 & 226.00 &  3.83\\
instance n=1000 511.alb & 1 & 0 & Solution & 120.13 & 237 & 230.00 &  2.95\\
instance n=1000 512.alb & 1 & 0 & Solution & 120.27 & 226 & 219.00 &  3.10\\
instance n=1000 513.alb & 1 & 0 & Solution & 120.15 & 227 & 219.00 &  3.52\\
instance n=1000 514.alb & 1 & 0 & Solution & 120.14 & 233 & 226.00 &  3.00\\
instance n=1000 515.alb & 1 & 0 & Solution & 120.18 & 228 & 221.00 &  3.07\\
instance n=1000 516.alb & 1 & 0 & Solution & 120.47 & 237 & 229.00 &  3.38\\
instance n=1000 517.alb & 1 & 0 & Solution & 120.17 & 229 & 221.00 &  3.49\\
instance n=1000 518.alb & 1 & 0 & Solution & 120.48 & 226 & 220.00 &  2.65\\
instance n=1000 519.alb & 1 & 0 & Solution & 120.34 & 229 & 221.00 &  3.49\\
instance n=1000 52.alb & 1 & 0 & Solution & 120.09 & 231 & 228.00 &  1.30\\
instance n=1000 520.alb & 1 & 0 & Solution & 120.33 & 234 & 226.00 &  3.42\\
instance n=1000 521.alb & 1 & 0 & Solution & 120.25 & 236 & 229.00 &  2.97\\
instance n=1000 522.alb & 1 & 0 & Solution & 120.21 & 221 & 215.00 &  2.71\\
instance n=1000 523.alb & 1 & 0 & Solution & 120.23 & 228 & 220.00 &  3.51\\
instance n=1000 524.alb & 1 & 0 & Solution & 120.26 & 232 & 226.00 &  2.59\\
instance n=1000 525.alb & 1 & 0 & Solution & 120.36 & 229 & 221.00 &  3.49\\
instance n=1000 53.alb & 1 & 0 & Solution & 120.09 & 230 & 227.00 &  1.30\\
instance n=1000 54.alb & 1 & 0 & Solution & 120.10 & 223 & 219.00 &  1.79\\
instance n=1000 55.alb & 1 & 0 & Solution & 120.10 & 220 & 217.00 &  1.36\\
instance n=1000 56.alb & 1 & 0 & Solution & 120.08 & 232 & 228.00 &  1.72\\
instance n=1000 57.alb & 1 & 0 & Solution & 120.08 & 227 & 224.00 &  1.32\\
instance n=1000 58.alb & 1 & 0 & Solution & 120.11 & 227 & 224.00 &  1.32\\
instance n=1000 59.alb & 1 & 0 & Solution & 120.11 & 226 & 223.00 &  1.33\\
instance n=1000 6.alb & 1 & 0 & Solution & 120.08 & 143 & 141.00 &  1.40\\
instance n=1000 60.alb & 1 & 0 & Solution & 120.08 & 233 & 230.00 &  1.29\\
instance n=1000 61.alb & 1 & 0 & Solution & 120.11 & 233 & 229.00 &  1.72\\
instance n=1000 62.alb & 1 & 0 & Solution & 120.08 & 226 & 223.00 &  1.33\\
instance n=1000 63.alb & 1 & 0 & Solution & 120.10 & 230 & 227.00 &  1.30\\
instance n=1000 64.alb & 1 & 0 & Solution & 120.11 & 233 & 229.00 &  1.72\\
instance n=1000 65.alb & 1 & 0 & Solution & 120.11 & 227 & 225.00 &  0.88\\
instance n=1000 66.alb & 1 & 0 & Solution & 120.08 & 230 & 227.00 &  1.30\\
instance n=1000 67.alb & 1 & 0 & Solution & 120.08 & 226 & 223.00 &  1.33\\
instance n=1000 68.alb & 1 & 0 & Solution & 120.11 & 230 & 226.00 &  1.74\\
instance n=1000 69.alb & 1 & 0 & Solution & 120.11 & 227 & 224.00 &  1.32\\
instance n=1000 7.alb & 1 & 0 & Solution & 120.07 & 138 & 136.00 &  1.45\\
instance n=1000 70.alb & 1 & 0 & Solution & 120.11 & 231 & 228.00 &  1.30\\
instance n=1000 71.alb & 1 & 0 & Solution & 120.09 & 233 & 230.00 &  1.29\\
instance n=1000 72.alb & 1 & 0 & Solution & 120.08 & 225 & 222.00 &  1.33\\
instance n=1000 73.alb & 1 & 0 & Solution & 120.12 & 224 & 221.00 &  1.34\\
instance n=1000 74.alb & 1 & 0 & Solution & 120.14 & 231 & 227.00 &  1.73\\
instance n=1000 75.alb & 1 & 0 & Solution & 120.10 & 230 & 227.00 &  1.30\\
instance n=1000 76.alb & 1 & 0 & Solution & 120.10 & 137 & 136.00 &  0.73\\
instance n=1000 77.alb & 1 & 0 & Solution & 120.06 & 137 & 136.00 &  0.73\\
instance n=1000 78.alb & 1 & 0 & Solution & 120.11 & 140 & 138.00 &  1.43\\
instance n=1000 79.alb & 1 & 0 & Solution & 120.13 & 143 & 142.00 &  0.70\\
instance n=1000 8.alb & 1 & 0 & Solution & 120.08 & 140 & 138.00 &  1.43\\
instance n=1000 80.alb & 1 & 0 & Solution & 120.06 & 141 & 140.00 &  0.71\\
instance n=1000 81.alb & 1 & 0 & Solution & 120.14 & 138 & 136.00 &  1.45\\
instance n=1000 82.alb & 1 & 0 & Solution & 120.13 & 137 & 136.00 &  0.73\\
instance n=1000 83.alb & 1 & 0 & Solution & 120.13 & 141 & 140.00 &  0.71\\
instance n=1000 84.alb & 1 & 0 & Solution & 120.07 & 136 & 135.00 &  0.74\\
instance n=1000 85.alb & 1 & 0 & Solution & 120.08 & 137 & 136.00 &  0.73\\
instance n=1000 86.alb & 1 & 0 & Solution & 120.13 & 139 & 138.00 &  0.72\\
instance n=1000 87.alb & 1 & 0 & Solution & 120.10 & 142 & 140.00 &  1.41\\
instance n=1000 88.alb & 1 & 0 & Solution & 120.09 & 142 & 140.00 &  1.41\\
instance n=1000 89.alb & 1 & 0 & Solution & 120.07 & 141 & 140.00 &  0.71\\
instance n=1000 9.alb & 1 & 0 & Solution & 120.06 & 136 & 134.00 &  1.47\\
instance n=1000 90.alb & 1 & 0 & Solution & 120.11 & 139 & 138.00 &  0.72\\
instance n=1000 91.alb & 1 & 0 & Solution & 120.05 & 142 & 141.00 &  0.70\\
instance n=1000 92.alb & 1 & 0 & Solution & 120.06 & 137 & 136.00 &  0.73\\
instance n=1000 93.alb & 1 & 0 & Solution & 120.11 & 138 & 137.00 &  0.72\\
instance n=1000 94.alb & 1 & 0 & Solution & 120.10 & 139 & 137.00 &  1.44\\
instance n=1000 95.alb & 1 & 0 & Solution & 120.14 & 137 & 136.00 &  0.73\\
instance n=1000 96.alb & 1 & 0 & Solution & 120.09 & 139 & 137.00 &  1.44\\
instance n=1000 97.alb & 1 & 0 & Solution & 120.12 & 140 & 138.00 &  1.43\\
instance n=1000 98.alb & 1 & 0 & Solution & 120.08 & 137 & 136.00 &  0.73\\
instance n=1000 99.alb & 1 & 0 & Solution & 120.08 & 137 & 136.00 &  0.73\\
instance n=100 1.alb & 1 & 0 & Optimal &  5.53 & 23 & 23.00 &  0.00\\
instance n=100 10.alb & 1 & 0 & Optimal &  0.05 & 22 & 22.00 &  0.00\\
instance n=100 100.alb & 1 & 0 & Optimal &  2.72 & 25 & 25.00 &  0.00\\
instance n=100 101.alb & 1 & 0 & Optimal &  0.73 & 15 & 15.00 &  0.00\\
instance n=100 102.alb & 1 & 0 & Optimal &  0.15 & 14 & 14.00 &  0.00\\
instance n=100 103.alb & 1 & 0 & Optimal &  0.13 & 14 & 14.00 &  0.00\\
instance n=100 104.alb & 1 & 0 & Optimal &  0.13 & 14 & 14.00 &  0.00\\
instance n=100 105.alb & 1 & 0 & Optimal &  0.13 & 13 & 13.00 &  0.00\\
instance n=100 106.alb & 1 & 0 & Optimal &  0.16 & 14 & 14.00 &  0.00\\
instance n=100 107.alb & 1 & 0 & Optimal &  0.20 & 14 & 14.00 &  0.00\\
instance n=100 108.alb & 1 & 0 & Optimal &  3.54 & 14 & 14.00 &  0.00\\
instance n=100 109.alb & 1 & 0 & Optimal &  0.19 & 15 & 15.00 &  0.00\\
instance n=100 11.alb & 1 & 0 & Optimal &  0.06 & 24 & 24.00 &  0.00\\
instance n=100 110.alb & 1 & 0 & Optimal &  0.22 & 13 & 13.00 &  0.00\\
instance n=100 111.alb & 1 & 0 & Optimal &  0.19 & 16 & 16.00 &  0.00\\
instance n=100 112.alb & 1 & 0 & Optimal &  3.03 & 13 & 13.00 &  0.00\\
instance n=100 113.alb & 1 & 0 & Optimal &  0.12 & 14 & 14.00 &  0.00\\
instance n=100 114.alb & 1 & 0 & Optimal &  0.29 & 13 & 13.00 &  0.00\\
instance n=100 115.alb & 1 & 0 & Optimal &  0.22 & 14 & 14.00 &  0.00\\
instance n=100 116.alb & 1 & 0 & Optimal &  0.16 & 16 & 16.00 &  0.00\\
instance n=100 117.alb & 1 & 0 & Optimal &  4.30 & 15 & 15.00 &  0.00\\
instance n=100 118.alb & 1 & 0 & Optimal &  0.11 & 15 & 15.00 &  0.00\\
instance n=100 119.alb & 1 & 0 & Optimal &  0.18 & 14 & 14.00 &  0.00\\
instance n=100 12.alb & 1 & 0 & Optimal &  2.85 & 25 & 25.00 &  0.00\\
instance n=100 120.alb & 1 & 0 & Optimal &  0.22 & 14 & 14.00 &  0.00\\
instance n=100 121.alb & 1 & 0 & Optimal &  0.15 & 15 & 15.00 &  0.00\\
instance n=100 122.alb & 1 & 0 & Optimal &  0.23 & 13 & 13.00 &  0.00\\
instance n=100 123.alb & 1 & 0 & Optimal &  0.19 & 15 & 15.00 &  0.00\\
instance n=100 124.alb & 1 & 0 & Optimal &  3.34 & 15 & 15.00 &  0.00\\
instance n=100 125.alb & 1 & 0 & Optimal &  0.12 & 14 & 14.00 &  0.00\\
instance n=100 126.alb & 1 & 0 & Solution & 120.01 & 51 & 49.00 &  3.92\\
instance n=100 127.alb & 1 & 0 & Solution & 120.01 & 53 & 49.00 &  7.55\\
instance n=100 128.alb & 1 & 0 & Solution & 120.03 & 57 & 52.00 &  8.77\\
instance n=100 129.alb & 1 & 0 & Solution & 120.02 & 55 & 50.00 &  9.09\\
instance n=100 13.alb & 1 & 0 & Optimal &  0.48 & 24 & 24.00 &  0.00\\
instance n=100 130.alb & 1 & 0 & Solution & 120.02 & 55 & 51.00 &  7.27\\
instance n=100 131.alb & 1 & 0 & Solution & 120.03 & 53 & 50.00 &  5.66\\
instance n=100 132.alb & 1 & 0 & Solution & 120.02 & 57 & 53.00 &  7.02\\
instance n=100 133.alb & 1 & 0 & Solution & 120.03 & 55 & 51.00 &  7.27\\
instance n=100 134.alb & 1 & 0 & Solution & 120.03 & 54 & 51.00 &  5.56\\
instance n=100 135.alb & 1 & 0 & Solution & 120.03 & 56 & 51.00 &  8.93\\
instance n=100 136.alb & 1 & 0 & Solution & 120.05 & 52 & 49.00 &  5.77\\
instance n=100 137.alb & 1 & 0 & Solution & 120.03 & 54 & 50.00 &  7.41\\
instance n=100 138.alb & 1 & 0 & Solution & 120.02 & 56 & 52.00 &  7.14\\
instance n=100 139.alb & 1 & 0 & Solution & 120.03 & 52 & 49.00 &  5.77\\
instance n=100 14.alb & 1 & 0 & Optimal &  1.46 & 20 & 20.00 &  0.00\\
instance n=100 140.alb & 1 & 0 & Solution & 120.04 & 55 & 51.00 &  7.27\\
instance n=100 141.alb & 1 & 0 & Solution & 120.03 & 51 & 49.00 &  3.92\\
instance n=100 142.alb & 1 & 0 & Solution & 120.03 & 55 & 50.00 &  9.09\\
instance n=100 143.alb & 1 & 0 & Solution & 120.02 & 53 & 51.00 &  3.77\\
instance n=100 144.alb & 1 & 0 & Solution & 120.03 & 49 & 47.00 &  4.08\\
instance n=100 145.alb & 1 & 0 & Solution & 120.04 & 56 & 51.00 &  8.93\\
instance n=100 146.alb & 1 & 0 & Solution & 120.03 & 53 & 50.00 &  5.66\\
instance n=100 147.alb & 1 & 0 & Solution & 120.03 & 59 & 52.00 & 11.86\\
instance n=100 148.alb & 1 & 0 & Solution & 120.04 & 53 & 50.00 &  5.66\\
instance n=100 149.alb & 1 & 0 & Solution & 120.03 & 55 & 51.00 &  7.27\\
instance n=100 15.alb & 1 & 0 & Optimal &  0.06 & 24 & 24.00 &  0.00\\
instance n=100 150.alb & 1 & 0 & Solution & 120.03 & 58 & 51.00 & 12.07\\
instance n=100 151.alb & 1 & 0 & Solution & 120.04 & 22 & 21.00 &  4.55\\
instance n=100 152.alb & 1 & 0 & Optimal &  0.31 & 22 & 22.00 &  0.00\\
instance n=100 153.alb & 1 & 0 & Optimal &  0.17 & 21 & 21.00 &  0.00\\
instance n=100 154.alb & 1 & 0 & Optimal &  0.25 & 25 & 25.00 &  0.00\\
instance n=100 155.alb & 1 & 0 & Optimal &  0.23 & 22 & 22.00 &  0.00\\
instance n=100 156.alb & 1 & 0 & Optimal &  0.28 & 23 & 23.00 &  0.00\\
instance n=100 157.alb & 1 & 0 & Optimal &  1.29 & 26 & 26.00 &  0.00\\
instance n=100 158.alb & 1 & 0 & Optimal &  0.29 & 23 & 23.00 &  0.00\\
instance n=100 159.alb & 1 & 0 & Optimal &  0.14 & 19 & 19.00 &  0.00\\
instance n=100 16.alb & 1 & 0 & Optimal &  0.04 & 23 & 23.00 &  0.00\\
instance n=100 160.alb & 1 & 0 & Optimal &  0.30 & 22 & 22.00 &  0.00\\
instance n=100 161.alb & 1 & 0 & Optimal & 117.95 & 22 & 22.00 &  0.00\\
instance n=100 162.alb & 1 & 0 & Solution & 120.04 & 23 & 22.00 &  4.35\\
instance n=100 163.alb & 1 & 0 & Optimal &  0.20 & 25 & 25.00 &  0.00\\
instance n=100 164.alb & 1 & 0 & Optimal &  0.20 & 23 & 23.00 &  0.00\\
instance n=100 165.alb & 1 & 0 & Solution & 120.02 & 25 & 24.00 &  4.00\\
instance n=100 166.alb & 1 & 0 & Optimal &  2.08 & 24 & 24.00 &  0.00\\
instance n=100 167.alb & 1 & 0 & Optimal &  0.25 & 22 & 22.00 &  0.00\\
instance n=100 168.alb & 1 & 0 & Solution & 120.05 & 22 & 21.00 &  4.55\\
instance n=100 169.alb & 1 & 0 & Optimal &  0.38 & 21 & 21.00 &  0.00\\
instance n=100 17.alb & 1 & 0 & Solution & 120.02 & 22 & 21.00 &  4.55\\
instance n=100 170.alb & 1 & 0 & Optimal &  4.00 & 24 & 24.00 &  0.00\\
instance n=100 171.alb & 1 & 0 & Solution & 120.03 & 25 & 24.00 &  4.00\\
instance n=100 172.alb & 1 & 0 & Optimal &  0.25 & 24 & 24.00 &  0.00\\
instance n=100 173.alb & 1 & 0 & Solution & 120.03 & 25 & 24.00 &  4.00\\
instance n=100 174.alb & 1 & 0 & Optimal &  4.22 & 22 & 22.00 &  0.00\\
instance n=100 175.alb & 1 & 0 & Solution & 120.03 & 27 & 26.00 &  3.70\\
instance n=100 176.alb & 1 & 0 & Optimal &  0.20 & 13 & 13.00 &  0.00\\
instance n=100 177.alb & 1 & 0 & Optimal &  0.18 & 14 & 14.00 &  0.00\\
instance n=100 178.alb & 1 & 0 & Optimal &  0.23 & 15 & 15.00 &  0.00\\
instance n=100 179.alb & 1 & 0 & Optimal &  0.14 & 15 & 15.00 &  0.00\\
instance n=100 18.alb & 1 & 0 & Solution & 120.01 & 20 & 19.00 &  5.00\\
instance n=100 180.alb & 1 & 0 & Optimal &  0.22 & 15 & 15.00 &  0.00\\
instance n=100 181.alb & 1 & 0 & Optimal &  0.22 & 13 & 13.00 &  0.00\\
instance n=100 182.alb & 1 & 0 & Optimal &  0.23 & 15 & 15.00 &  0.00\\
instance n=100 183.alb & 1 & 0 & Optimal &  0.20 & 14 & 14.00 &  0.00\\
instance n=100 184.alb & 1 & 0 & Optimal &  0.28 & 14 & 14.00 &  0.00\\
instance n=100 185.alb & 1 & 0 & Optimal &  0.26 & 15 & 15.00 &  0.00\\
instance n=100 186.alb & 1 & 0 & Optimal &  2.05 & 14 & 14.00 &  0.00\\
instance n=100 187.alb & 1 & 0 & Optimal & 10.45 & 13 & 13.00 &  0.00\\
instance n=100 188.alb & 1 & 0 & Optimal &  0.26 & 16 & 16.00 &  0.00\\
instance n=100 189.alb & 1 & 0 & Optimal &  0.23 & 14 & 14.00 &  0.00\\
instance n=100 19.alb & 1 & 0 & Optimal &  0.43 & 23 & 23.00 &  0.00\\
instance n=100 190.alb & 1 & 0 & Optimal &  0.26 & 13 & 13.00 &  0.00\\
instance n=100 191.alb & 1 & 0 & Optimal &  0.23 & 14 & 14.00 &  0.00\\
instance n=100 192.alb & 1 & 0 & Optimal &  2.87 & 13 & 13.00 &  0.00\\
instance n=100 193.alb & 1 & 0 & Optimal &  0.38 & 15 & 15.00 &  0.00\\
instance n=100 194.alb & 1 & 0 & Optimal &  0.30 & 15 & 15.00 &  0.00\\
instance n=100 195.alb & 1 & 0 & Optimal &  0.23 & 15 & 15.00 &  0.00\\
instance n=100 196.alb & 1 & 0 & Optimal &  0.28 & 15 & 15.00 &  0.00\\
instance n=100 197.alb & 1 & 0 & Optimal &  0.17 & 15 & 15.00 &  0.00\\
instance n=100 198.alb & 1 & 0 & Optimal &  2.85 & 13 & 13.00 &  0.00\\
instance n=100 199.alb & 1 & 0 & Optimal &  0.25 & 14 & 14.00 &  0.00\\
instance n=100 2.alb & 1 & 0 & Optimal &  0.05 & 21 & 21.00 &  0.00\\
instance n=100 20.alb & 1 & 0 & Optimal &  0.04 & 21 & 21.00 &  0.00\\
instance n=100 200.alb & 1 & 0 & Optimal &  0.22 & 15 & 15.00 &  0.00\\
instance n=100 201.alb & 1 & 0 & Solution & 120.06 & 53 & 51.00 &  3.77\\
instance n=100 202.alb & 1 & 0 & Solution & 120.05 & 61 & 52.00 & 14.75\\
instance n=100 203.alb & 1 & 0 & Solution & 120.03 & 53 & 49.00 &  7.55\\
instance n=100 204.alb & 1 & 0 & Solution & 120.05 & 51 & 48.00 &  5.88\\
instance n=100 205.alb & 1 & 0 & Solution & 120.03 & 57 & 51.00 & 10.53\\
instance n=100 206.alb & 1 & 0 & Solution & 120.03 & 52 & 49.00 &  5.77\\
instance n=100 207.alb & 1 & 0 & Solution & 120.01 & 52 & 49.00 &  5.77\\
instance n=100 208.alb & 1 & 0 & Solution & 120.03 & 57 & 51.00 & 10.53\\
instance n=100 209.alb & 1 & 0 & Solution & 120.07 & 55 & 51.00 &  7.27\\
instance n=100 21.alb & 1 & 0 & Optimal &  0.45 & 21 & 21.00 &  0.00\\
instance n=100 210.alb & 1 & 0 & Solution & 120.04 & 53 & 49.00 &  7.55\\
instance n=100 211.alb & 1 & 0 & Solution & 120.04 & 52 & 49.00 &  5.77\\
instance n=100 212.alb & 1 & 0 & Solution & 120.04 & 53 & 50.00 &  5.66\\
instance n=100 213.alb & 1 & 0 & Solution & 120.02 & 53 & 50.00 &  5.66\\
instance n=100 214.alb & 1 & 0 & Solution & 120.04 & 55 & 50.00 &  9.09\\
instance n=100 215.alb & 1 & 0 & Solution & 120.05 & 49 & 47.00 &  4.08\\
instance n=100 216.alb & 1 & 0 & Solution & 120.01 & 53 & 50.00 &  5.66\\
instance n=100 217.alb & 1 & 0 & Solution & 120.06 & 52 & 49.00 &  5.77\\
instance n=100 218.alb & 1 & 0 & Solution & 120.03 & 54 & 50.00 &  7.41\\
instance n=100 219.alb & 1 & 0 & Solution & 120.05 & 52 & 49.00 &  5.77\\
instance n=100 22.alb & 1 & 0 & Solution & 120.01 & 25 & 24.00 &  4.00\\
instance n=100 220.alb & 1 & 0 & Solution & 120.04 & 54 & 50.00 &  7.41\\
instance n=100 221.alb & 1 & 0 & Solution & 120.03 & 57 & 51.00 & 10.53\\
instance n=100 222.alb & 1 & 0 & Solution & 120.05 & 53 & 50.00 &  5.66\\
instance n=100 223.alb & 1 & 0 & Solution & 120.04 & 51 & 49.00 &  3.92\\
instance n=100 224.alb & 1 & 0 & Solution & 120.05 & 56 & 51.00 &  8.93\\
instance n=100 225.alb & 1 & 0 & Solution & 120.03 & 53 & 51.00 &  3.77\\
instance n=100 226.alb & 1 & 0 & Solution & 120.05 & 25 & 24.00 &  4.00\\
instance n=100 227.alb & 1 & 0 & Optimal & 82.63 & 26 & 26.00 &  0.00\\
instance n=100 228.alb & 1 & 0 & Optimal &  8.12 & 22 & 22.00 &  0.00\\
instance n=100 229.alb & 1 & 0 & Optimal &  0.40 & 24 & 24.00 &  0.00\\
instance n=100 23.alb & 1 & 0 & Optimal &  0.07 & 24 & 24.00 &  0.00\\
instance n=100 230.alb & 1 & 0 & Optimal & 19.43 & 23 & 23.00 &  0.00\\
instance n=100 231.alb & 1 & 0 & Optimal & 28.64 & 22 & 22.00 &  0.00\\
instance n=100 232.alb & 1 & 0 & Optimal &  0.52 & 22 & 22.00 &  0.00\\
instance n=100 233.alb & 1 & 0 & Solution & 120.03 & 23 & 22.00 &  4.35\\
instance n=100 234.alb & 1 & 0 & Optimal &  0.36 & 23 & 23.00 &  0.00\\
instance n=100 235.alb & 1 & 0 & Optimal &  2.66 & 26 & 26.00 &  0.00\\
instance n=100 236.alb & 1 & 0 & Solution & 120.03 & 23 & 22.00 &  4.35\\
instance n=100 237.alb & 1 & 0 & Optimal &  5.82 & 23 & 23.00 &  0.00\\
instance n=100 238.alb & 1 & 0 & Optimal &  4.46 & 23 & 23.00 &  0.00\\
instance n=100 239.alb & 1 & 0 & Optimal &  0.31 & 21 & 21.00 &  0.00\\
instance n=100 24.alb & 1 & 0 & Optimal &  0.07 & 24 & 24.00 &  0.00\\
instance n=100 240.alb & 1 & 0 & Optimal &  3.22 & 22 & 22.00 &  0.00\\
instance n=100 241.alb & 1 & 0 & Optimal &  4.17 & 22 & 22.00 &  0.00\\
instance n=100 242.alb & 1 & 0 & Optimal &  4.05 & 23 & 23.00 &  0.00\\
instance n=100 243.alb & 1 & 0 & Optimal & 112.03 & 23 & 23.00 &  0.00\\
instance n=100 244.alb & 1 & 0 & Optimal &  0.38 & 21 & 21.00 &  0.00\\
instance n=100 245.alb & 1 & 0 & Solution & 120.05 & 24 & 23.00 &  4.17\\
instance n=100 246.alb & 1 & 0 & Optimal &  6.42 & 26 & 26.00 &  0.00\\
instance n=100 247.alb & 1 & 0 & Optimal &  4.47 & 22 & 22.00 &  0.00\\
instance n=100 248.alb & 1 & 0 & Optimal &  3.60 & 19 & 19.00 &  0.00\\
instance n=100 249.alb & 1 & 0 & Optimal &  3.14 & 21 & 21.00 &  0.00\\
instance n=100 25.alb & 1 & 0 & Optimal &  0.52 & 22 & 22.00 &  0.00\\
instance n=100 250.alb & 1 & 0 & Optimal &  2.48 & 24 & 24.00 &  0.00\\
instance n=100 251.alb & 1 & 0 & Optimal &  0.19 & 15 & 15.00 &  0.00\\
instance n=100 252.alb & 1 & 0 & Optimal &  0.48 & 14 & 14.00 &  0.00\\
instance n=100 253.alb & 1 & 0 & Optimal &  0.20 & 14 & 14.00 &  0.00\\
instance n=100 254.alb & 1 & 0 & Optimal &  0.23 & 14 & 14.00 &  0.00\\
instance n=100 255.alb & 1 & 0 & Optimal &  0.20 & 14 & 14.00 &  0.00\\
instance n=100 256.alb & 1 & 0 & Optimal &  0.30 & 15 & 15.00 &  0.00\\
instance n=100 257.alb & 1 & 0 & Optimal &  3.57 & 12 & 12.00 &  0.00\\
instance n=100 258.alb & 1 & 0 & Optimal &  3.55 & 14 & 14.00 &  0.00\\
instance n=100 259.alb & 1 & 0 & Optimal &  1.84 & 15 & 15.00 &  0.00\\
instance n=100 26.alb & 1 & 0 & Optimal &  0.55 & 14 & 14.00 &  0.00\\
instance n=100 260.alb & 1 & 0 & Optimal &  0.31 & 15 & 15.00 &  0.00\\
instance n=100 261.alb & 1 & 0 & Optimal &  0.35 & 14 & 14.00 &  0.00\\
instance n=100 262.alb & 1 & 0 & Optimal &  0.28 & 14 & 14.00 &  0.00\\
instance n=100 263.alb & 1 & 0 & Optimal &  0.39 & 14 & 14.00 &  0.00\\
instance n=100 264.alb & 1 & 0 & Optimal &  0.28 & 15 & 15.00 &  0.00\\
instance n=100 265.alb & 1 & 0 & Optimal &  0.39 & 14 & 14.00 &  0.00\\
instance n=100 266.alb & 1 & 0 & Optimal &  3.19 & 13 & 13.00 &  0.00\\
instance n=100 267.alb & 1 & 0 & Optimal &  0.35 & 13 & 13.00 &  0.00\\
instance n=100 268.alb & 1 & 0 & Optimal &  0.28 & 15 & 15.00 &  0.00\\
instance n=100 269.alb & 1 & 0 & Optimal &  0.33 & 15 & 15.00 &  0.00\\
instance n=100 27.alb & 1 & 0 & Optimal &  0.32 & 13 & 13.00 &  0.00\\
instance n=100 270.alb & 1 & 0 & Optimal &  0.46 & 13 & 13.00 &  0.00\\
instance n=100 271.alb & 1 & 0 & Optimal & 15.96 & 13 & 13.00 &  0.00\\
instance n=100 272.alb & 1 & 0 & Optimal &  0.26 & 14 & 14.00 &  0.00\\
instance n=100 273.alb & 1 & 0 & Optimal &  7.35 & 13 & 13.00 &  0.00\\
instance n=100 274.alb & 1 & 0 & Optimal &  4.30 & 13 & 13.00 &  0.00\\
instance n=100 275.alb & 1 & 0 & Optimal &  0.43 & 13 & 13.00 &  0.00\\
instance n=100 276.alb & 1 & 0 & Solution & 120.05 & 60 & 52.00 & 13.33\\
instance n=100 277.alb & 1 & 0 & Solution & 120.04 & 57 & 52.00 &  8.77\\
instance n=100 278.alb & 1 & 0 & Solution & 120.05 & 58 & 52.00 & 10.34\\
instance n=100 279.alb & 1 & 0 & Solution & 120.06 & 54 & 51.00 &  5.56\\
instance n=100 28.alb & 1 & 0 & Optimal &  0.40 & 14 & 14.00 &  0.00\\
instance n=100 280.alb & 1 & 0 & Solution & 120.06 & 56 & 51.00 &  8.93\\
instance n=100 281.alb & 1 & 0 & Solution & 120.05 & 62 & 52.00 & 16.13\\
instance n=100 282.alb & 1 & 0 & Solution & 120.04 & 60 & 53.00 & 11.67\\
instance n=100 283.alb & 1 & 0 & Solution & 120.07 & 55 & 51.00 &  7.27\\
instance n=100 284.alb & 1 & 0 & Solution & 120.08 & 55 & 51.00 &  7.27\\
instance n=100 285.alb & 1 & 0 & Solution & 120.05 & 55 & 51.00 &  7.27\\
instance n=100 286.alb & 1 & 0 & Solution & 120.04 & 57 & 51.00 & 10.53\\
instance n=100 287.alb & 1 & 0 & Solution & 120.04 & 54 & 50.00 &  7.41\\
instance n=100 288.alb & 1 & 0 & Solution & 120.07 & 56 & 51.00 &  8.93\\
instance n=100 289.alb & 1 & 0 & Solution & 120.05 & 62 & 52.00 & 16.13\\
instance n=100 29.alb & 1 & 0 & Optimal &  0.35 & 14 & 14.00 &  0.00\\
instance n=100 290.alb & 1 & 0 & Solution & 120.04 & 55 & 51.00 &  7.27\\
instance n=100 291.alb & 1 & 0 & Solution & 120.04 & 53 & 49.00 &  7.55\\
instance n=100 292.alb & 1 & 0 & Solution & 120.05 & 58 & 51.00 & 12.07\\
instance n=100 293.alb & 1 & 0 & Solution & 120.05 & 53 & 50.00 &  5.66\\
instance n=100 294.alb & 1 & 0 & Solution & 120.05 & 58 & 52.00 & 10.34\\
instance n=100 295.alb & 1 & 0 & Solution & 120.05 & 57 & 51.00 & 10.53\\
instance n=100 296.alb & 1 & 0 & Solution & 120.06 & 55 & 51.00 &  7.27\\
instance n=100 297.alb & 1 & 0 & Solution & 120.04 & 59 & 51.00 & 13.56\\
instance n=100 298.alb & 1 & 0 & Solution & 120.03 & 59 & 52.00 & 11.86\\
instance n=100 299.alb & 1 & 0 & Solution & 120.09 & 55 & 50.00 &  9.09\\
instance n=100 3.alb & 1 & 0 & Optimal &  0.05 & 20 & 20.00 &  0.00\\
instance n=100 30.alb & 1 & 0 & Optimal &  0.04 & 15 & 15.00 &  0.00\\
instance n=100 300.alb & 1 & 0 & Solution & 120.04 & 54 & 50.00 &  7.41\\
instance n=100 301.alb & 1 & 0 & Optimal &  0.49 & 23 & 23.00 &  0.00\\
instance n=100 302.alb & 1 & 0 & Optimal &  0.39 & 24 & 24.00 &  0.00\\
instance n=100 303.alb & 1 & 0 & Optimal &  5.07 & 24 & 24.00 &  0.00\\
instance n=100 304.alb & 1 & 0 & Optimal &  1.82 & 21 & 21.00 &  0.00\\
instance n=100 305.alb & 1 & 0 & Optimal &  0.31 & 22 & 22.00 &  0.00\\
instance n=100 306.alb & 1 & 0 & Optimal &  0.50 & 24 & 24.00 &  0.00\\
instance n=100 307.alb & 1 & 0 & Solution & 120.06 & 24 & 23.00 &  4.17\\
instance n=100 308.alb & 1 & 0 & Solution & 120.06 & 21 & 20.00 &  4.76\\
instance n=100 309.alb & 1 & 0 & Solution & 120.05 & 22 & 21.00 &  4.55\\
instance n=100 31.alb & 1 & 0 & Optimal &  0.05 & 14 & 14.00 &  0.00\\
instance n=100 310.alb & 1 & 0 & Optimal &  2.46 & 23 & 23.00 &  0.00\\
instance n=100 311.alb & 1 & 0 & Optimal &  0.41 & 21 & 21.00 &  0.00\\
instance n=100 312.alb & 1 & 0 & Optimal &  0.41 & 22 & 22.00 &  0.00\\
instance n=100 313.alb & 1 & 0 & Optimal &  0.61 & 23 & 23.00 &  0.00\\
instance n=100 314.alb & 1 & 0 & Optimal &  0.60 & 19 & 19.00 &  0.00\\
instance n=100 315.alb & 1 & 0 & Solution & 120.07 & 23 & 22.00 &  4.35\\
instance n=100 316.alb & 1 & 0 & Optimal &  0.38 & 24 & 24.00 &  0.00\\
instance n=100 317.alb & 1 & 0 & Optimal &  0.43 & 26 & 26.00 &  0.00\\
instance n=100 318.alb & 1 & 0 & Optimal &  0.41 & 21 & 21.00 &  0.00\\
instance n=100 319.alb & 1 & 0 & Optimal &  2.56 & 23 & 23.00 &  0.00\\
instance n=100 32.alb & 1 & 0 & Optimal &  0.04 & 14 & 14.00 &  0.00\\
instance n=100 320.alb & 1 & 0 & Optimal &  0.30 & 22 & 22.00 &  0.00\\
instance n=100 321.alb & 1 & 0 & Optimal &  0.42 & 26 & 26.00 &  0.00\\
instance n=100 322.alb & 1 & 0 & Solution & 120.07 & 24 & 23.00 &  4.17\\
instance n=100 323.alb & 1 & 0 & Optimal &  0.33 & 24 & 24.00 &  0.00\\
instance n=100 324.alb & 1 & 0 & Optimal &  0.50 & 23 & 23.00 &  0.00\\
instance n=100 325.alb & 1 & 0 & Solution & 120.09 & 26 & 25.00 &  3.85\\
instance n=100 326.alb & 1 & 0 & Optimal &  0.27 & 13 & 13.00 &  0.00\\
instance n=100 327.alb & 1 & 0 & Optimal &  0.57 & 14 & 14.00 &  0.00\\
instance n=100 328.alb & 1 & 0 & Optimal & 17.30 & 14 & 14.00 &  0.00\\
instance n=100 329.alb & 1 & 0 & Optimal &  0.45 & 14 & 14.00 &  0.00\\
instance n=100 33.alb & 1 & 0 & Optimal &  0.05 & 15 & 15.00 &  0.00\\
instance n=100 330.alb & 1 & 0 & Optimal &  9.76 & 14 & 14.00 &  0.00\\
instance n=100 331.alb & 1 & 0 & Optimal &  0.27 & 14 & 14.00 &  0.00\\
instance n=100 332.alb & 1 & 0 & Optimal &  0.35 & 14 & 14.00 &  0.00\\
instance n=100 333.alb & 1 & 0 & Optimal &  0.38 & 15 & 15.00 &  0.00\\
instance n=100 334.alb & 1 & 0 & Optimal &  3.56 & 14 & 14.00 &  0.00\\
instance n=100 335.alb & 1 & 0 & Optimal &  0.38 & 13 & 13.00 &  0.00\\
instance n=100 336.alb & 1 & 0 & Optimal &  0.43 & 15 & 15.00 &  0.00\\
instance n=100 337.alb & 1 & 0 & Optimal &  0.78 & 13 & 13.00 &  0.00\\
instance n=100 338.alb & 1 & 0 & Solution & 120.06 & 15 & 14.00 &  6.67\\
instance n=100 339.alb & 1 & 0 & Optimal &  0.31 & 14 & 14.00 &  0.00\\
instance n=100 34.alb & 1 & 0 & Optimal &  0.05 & 15 & 15.00 &  0.00\\
instance n=100 340.alb & 1 & 0 & Optimal &  0.41 & 14 & 14.00 &  0.00\\
instance n=100 341.alb & 1 & 0 & Optimal &  0.49 & 16 & 16.00 &  0.00\\
instance n=100 342.alb & 1 & 0 & Optimal &  2.14 & 14 & 14.00 &  0.00\\
instance n=100 343.alb & 1 & 0 & Optimal &  0.57 & 16 & 16.00 &  0.00\\
instance n=100 344.alb & 1 & 0 & Optimal &  0.45 & 15 & 15.00 &  0.00\\
instance n=100 345.alb & 1 & 0 & Optimal &  0.32 & 14 & 14.00 &  0.00\\
instance n=100 346.alb & 1 & 0 & Optimal &  0.37 & 14 & 14.00 &  0.00\\
instance n=100 347.alb & 1 & 0 & Optimal &  0.40 & 14 & 14.00 &  0.00\\
instance n=100 348.alb & 1 & 0 & Optimal &  0.36 & 14 & 14.00 &  0.00\\
instance n=100 349.alb & 1 & 0 & Optimal &  0.36 & 13 & 13.00 &  0.00\\
instance n=100 35.alb & 1 & 0 & Optimal &  0.05 & 15 & 15.00 &  0.00\\
instance n=100 350.alb & 1 & 0 & Optimal &  0.43 & 14 & 14.00 &  0.00\\
instance n=100 351.alb & 1 & 0 & Solution & 120.08 & 59 & 52.00 & 11.86\\
instance n=100 352.alb & 1 & 0 & Solution & 120.07 & 63 & 52.00 & 17.46\\
instance n=100 353.alb & 1 & 0 & Solution & 120.06 & 51 & 49.00 &  3.92\\
instance n=100 354.alb & 1 & 0 & Solution & 120.05 & 53 & 49.00 &  7.55\\
instance n=100 355.alb & 1 & 0 & Solution & 120.04 & 55 & 51.00 &  7.27\\
instance n=100 356.alb & 1 & 0 & Solution & 120.09 & 61 & 53.00 & 13.11\\
instance n=100 357.alb & 1 & 0 & Solution & 120.08 & 54 & 50.00 &  7.41\\
instance n=100 358.alb & 1 & 0 & Solution & 120.02 & 53 & 50.00 &  5.66\\
instance n=100 359.alb & 1 & 0 & Solution & 120.07 & 54 & 50.00 &  7.41\\
instance n=100 36.alb & 1 & 0 & Optimal &  1.95 & 14 & 14.00 &  0.00\\
instance n=100 360.alb & 1 & 0 & Solution & 120.08 & 55 & 51.00 &  7.27\\
instance n=100 361.alb & 1 & 0 & Solution & 120.05 & 52 & 49.00 &  5.77\\
instance n=100 362.alb & 1 & 0 & Solution & 120.07 & 57 & 51.00 & 10.53\\
instance n=100 363.alb & 1 & 0 & Solution & 120.09 & 53 & 50.00 &  5.66\\
instance n=100 364.alb & 1 & 0 & Solution & 120.06 & 53 & 50.00 &  5.66\\
instance n=100 365.alb & 1 & 0 & Solution & 120.06 & 53 & 50.00 &  5.66\\
instance n=100 366.alb & 1 & 0 & Solution & 120.08 & 61 & 53.00 & 13.11\\
instance n=100 367.alb & 1 & 0 & Solution & 120.05 & 56 & 51.00 &  8.93\\
instance n=100 368.alb & 1 & 0 & Solution & 120.04 & 59 & 52.00 & 11.86\\
instance n=100 369.alb & 1 & 0 & Solution & 120.05 & 51 & 49.00 &  3.92\\
instance n=100 37.alb & 1 & 0 & Optimal &  0.04 & 14 & 14.00 &  0.00\\
instance n=100 370.alb & 1 & 0 & Solution & 120.05 & 57 & 52.00 &  8.77\\
instance n=100 371.alb & 1 & 0 & Solution & 120.08 & 53 & 50.00 &  5.66\\
instance n=100 372.alb & 1 & 0 & Solution & 120.04 & 49 & 47.00 &  4.08\\
instance n=100 373.alb & 1 & 0 & Solution & 120.09 & 51 & 49.00 &  3.92\\
instance n=100 374.alb & 1 & 0 & Solution & 120.06 & 53 & 50.00 &  5.66\\
instance n=100 375.alb & 1 & 0 & Solution & 120.07 & 58 & 52.00 & 10.34\\
instance n=100 376.alb & 1 & 0 & Optimal &  0.59 & 23 & 23.00 &  0.00\\
instance n=100 377.alb & 1 & 0 & Solution & 120.10 & 21 & 20.00 &  4.76\\
instance n=100 378.alb & 1 & 0 & Optimal &  5.15 & 22 & 22.00 &  0.00\\
instance n=100 379.alb & 1 & 0 & Optimal & 50.33 & 23 & 23.00 &  0.00\\
instance n=100 38.alb & 1 & 0 & Optimal &  0.05 & 14 & 14.00 &  0.00\\
instance n=100 380.alb & 1 & 0 & Solution & 120.07 & 23 & 22.00 &  4.35\\
instance n=100 381.alb & 1 & 0 & Optimal &  2.24 & 24 & 24.00 &  0.00\\
instance n=100 382.alb & 1 & 0 & Optimal &  6.90 & 25 & 25.00 &  0.00\\
instance n=100 383.alb & 1 & 0 & Optimal &  0.50 & 25 & 25.00 &  0.00\\
instance n=100 384.alb & 1 & 0 & Optimal &  1.24 & 25 & 25.00 &  0.00\\
instance n=100 385.alb & 1 & 0 & Optimal &  0.44 & 22 & 22.00 &  0.00\\
instance n=100 386.alb & 1 & 0 & Optimal & 67.35 & 23 & 23.00 &  0.00\\
instance n=100 387.alb & 1 & 0 & Optimal &  0.91 & 22 & 22.00 &  0.00\\
instance n=100 388.alb & 1 & 0 & Solution & 120.06 & 26 & 25.00 &  3.85\\
instance n=100 389.alb & 1 & 0 & Optimal &  0.35 & 23 & 23.00 &  0.00\\
instance n=100 39.alb & 1 & 0 & Optimal &  0.03 & 14 & 14.00 &  0.00\\
instance n=100 390.alb & 1 & 0 & Solution & 120.06 & 23 & 22.00 &  4.35\\
instance n=100 391.alb & 1 & 0 & Optimal &  0.44 & 20 & 20.00 &  0.00\\
instance n=100 392.alb & 1 & 0 & Optimal &  0.46 & 22 & 22.00 &  0.00\\
instance n=100 393.alb & 1 & 0 & Solution & 120.06 & 24 & 23.00 &  4.17\\
instance n=100 394.alb & 1 & 0 & Optimal &  0.64 & 22 & 22.00 &  0.00\\
instance n=100 395.alb & 1 & 0 & Optimal &  9.99 & 24 & 24.00 &  0.00\\
instance n=100 396.alb & 1 & 0 & Optimal & 11.36 & 20 & 20.00 &  0.00\\
instance n=100 397.alb & 1 & 0 & Solution & 120.06 & 26 & 25.00 &  3.85\\
instance n=100 398.alb & 1 & 0 & Solution & 120.07 & 25 & 24.00 &  4.00\\
instance n=100 399.alb & 1 & 0 & Optimal &  1.01 & 23 & 23.00 &  0.00\\
instance n=100 4.alb & 1 & 0 & Optimal &  0.07 & 24 & 24.00 &  0.00\\
instance n=100 40.alb & 1 & 0 & Optimal &  0.09 & 14 & 14.00 &  0.00\\
instance n=100 400.alb & 1 & 0 & Optimal &  4.68 & 24 & 24.00 &  0.00\\
instance n=100 401.alb & 1 & 0 & Optimal &  0.46 & 15 & 15.00 &  0.00\\
instance n=100 402.alb & 1 & 0 & Optimal &  0.46 & 15 & 15.00 &  0.00\\
instance n=100 403.alb & 1 & 0 & Optimal &  0.60 & 14 & 14.00 &  0.00\\
instance n=100 404.alb & 1 & 0 & Optimal &  0.57 & 15 & 15.00 &  0.00\\
instance n=100 405.alb & 1 & 0 & Optimal &  0.45 & 13 & 13.00 &  0.00\\
instance n=100 406.alb & 1 & 0 & Optimal &  0.49 & 14 & 14.00 &  0.00\\
instance n=100 407.alb & 1 & 0 & Optimal &  0.69 & 15 & 15.00 &  0.00\\
instance n=100 408.alb & 1 & 0 & Optimal &  0.71 & 14 & 14.00 &  0.00\\
instance n=100 409.alb & 1 & 0 & Optimal &  0.36 & 15 & 15.00 &  0.00\\
instance n=100 41.alb & 1 & 0 & Optimal &  0.08 & 13 & 13.00 &  0.00\\
instance n=100 410.alb & 1 & 0 & Optimal &  0.33 & 14 & 14.00 &  0.00\\
instance n=100 411.alb & 1 & 0 & Optimal &  4.24 & 14 & 14.00 &  0.00\\
instance n=100 412.alb & 1 & 0 & Optimal &  0.55 & 14 & 14.00 &  0.00\\
instance n=100 413.alb & 1 & 0 & Optimal &  0.61 & 14 & 14.00 &  0.00\\
instance n=100 414.alb & 1 & 0 & Optimal & 34.16 & 14 & 14.00 &  0.00\\
instance n=100 415.alb & 1 & 0 & Optimal &  4.46 & 13 & 13.00 &  0.00\\
instance n=100 416.alb & 1 & 0 & Optimal &  0.53 & 14 & 14.00 &  0.00\\
instance n=100 417.alb & 1 & 0 & Optimal &  0.52 & 15 & 15.00 &  0.00\\
instance n=100 418.alb & 1 & 0 & Optimal &  0.60 & 16 & 16.00 &  0.00\\
instance n=100 419.alb & 1 & 0 & Optimal &  4.26 & 14 & 14.00 &  0.00\\
instance n=100 42.alb & 1 & 0 & Optimal &  0.05 & 14 & 14.00 &  0.00\\
instance n=100 420.alb & 1 & 0 & Optimal &  0.35 & 14 & 14.00 &  0.00\\
instance n=100 421.alb & 1 & 0 & Optimal &  0.35 & 14 & 14.00 &  0.00\\
instance n=100 422.alb & 1 & 0 & Optimal &  0.46 & 15 & 15.00 &  0.00\\
instance n=100 423.alb & 1 & 0 & Optimal &  3.71 & 14 & 14.00 &  0.00\\
instance n=100 424.alb & 1 & 0 & Optimal &  0.41 & 14 & 14.00 &  0.00\\
instance n=100 425.alb & 1 & 0 & Optimal &  0.58 & 15 & 15.00 &  0.00\\
instance n=100 426.alb & 1 & 0 & Solution & 120.09 & 60 & 53.00 & 11.67\\
instance n=100 427.alb & 1 & 0 & Solution & 120.06 & 56 & 50.00 & 10.71\\
instance n=100 428.alb & 1 & 0 & Solution & 120.07 & 55 & 51.00 &  7.27\\
instance n=100 429.alb & 1 & 0 & Solution & 120.06 & 59 & 52.00 & 11.86\\
instance n=100 43.alb & 1 & 0 & Optimal &  0.83 & 14 & 14.00 &  0.00\\
instance n=100 430.alb & 1 & 0 & Solution & 120.07 & 54 & 50.00 &  7.41\\
instance n=100 431.alb & 1 & 0 & Solution & 120.05 & 54 & 50.00 &  7.41\\
instance n=100 432.alb & 1 & 0 & Solution & 120.03 & 56 & 51.00 &  8.93\\
instance n=100 433.alb & 1 & 0 & Solution & 120.04 & 53 & 49.00 &  7.55\\
instance n=100 434.alb & 1 & 0 & Solution & 120.08 & 57 & 51.00 & 10.53\\
instance n=100 435.alb & 1 & 0 & Solution & 120.08 & 56 & 50.00 & 10.71\\
instance n=100 436.alb & 1 & 0 & Solution & 120.09 & 52 & 48.00 &  7.69\\
instance n=100 437.alb & 1 & 0 & Solution & 120.03 & 53 & 50.00 &  5.66\\
instance n=100 438.alb & 1 & 0 & Solution & 120.05 & 55 & 51.00 &  7.27\\
instance n=100 439.alb & 1 & 0 & Solution & 120.06 & 56 & 51.00 &  8.93\\
instance n=100 44.alb & 1 & 0 & Optimal &  0.05 & 14 & 14.00 &  0.00\\
instance n=100 440.alb & 1 & 0 & Solution & 120.08 & 53 & 49.00 &  7.55\\
instance n=100 441.alb & 1 & 0 & Solution & 120.06 & 53 & 50.00 &  5.66\\
instance n=100 442.alb & 1 & 0 & Solution & 120.03 & 53 & 48.00 &  9.43\\
instance n=100 443.alb & 1 & 0 & Solution & 120.11 & 56 & 50.00 & 10.71\\
instance n=100 444.alb & 1 & 0 & Solution & 120.07 & 54 & 50.00 &  7.41\\
instance n=100 445.alb & 1 & 0 & Solution & 120.04 & 55 & 51.00 &  7.27\\
instance n=100 446.alb & 1 & 0 & Solution & 120.02 & 57 & 52.00 &  8.77\\
instance n=100 447.alb & 1 & 0 & Solution & 120.06 & 54 & 50.00 &  7.41\\
instance n=100 448.alb & 1 & 0 & Solution & 120.09 & 56 & 51.00 &  8.93\\
instance n=100 449.alb & 1 & 0 & Solution & 120.05 & 56 & 50.00 & 10.71\\
instance n=100 45.alb & 1 & 0 & Optimal &  0.04 & 14 & 14.00 &  0.00\\
instance n=100 450.alb & 1 & 0 & Solution & 120.10 & 54 & 51.00 &  5.56\\
instance n=100 451.alb & 1 & 0 & Optimal &  3.90 & 26 & 26.00 &  0.00\\
instance n=100 452.alb & 1 & 0 & Optimal &  3.31 & 22 & 22.00 &  0.00\\
instance n=100 453.alb & 1 & 0 & Optimal &  2.72 & 24 & 24.00 &  0.00\\
instance n=100 454.alb & 1 & 0 & Optimal &  1.96 & 23 & 23.00 &  0.00\\
instance n=100 455.alb & 1 & 0 & Optimal &  3.85 & 23 & 23.00 &  0.00\\
instance n=100 456.alb & 1 & 0 & Optimal &  4.02 & 26 & 26.00 &  0.00\\
instance n=100 457.alb & 1 & 0 & Optimal &  2.42 & 23 & 23.00 &  0.00\\
instance n=100 458.alb & 1 & 0 & Optimal &  2.61 & 24 & 24.00 &  0.00\\
instance n=100 459.alb & 1 & 0 & Optimal &  3.47 & 23 & 23.00 &  0.00\\
instance n=100 46.alb & 1 & 0 & Optimal &  0.05 & 14 & 14.00 &  0.00\\
instance n=100 460.alb & 1 & 0 & Optimal &  5.71 & 23 & 23.00 &  0.00\\
instance n=100 461.alb & 1 & 0 & Optimal &  2.91 & 23 & 23.00 &  0.00\\
instance n=100 462.alb & 1 & 0 & Optimal &  4.90 & 23 & 23.00 &  0.00\\
instance n=100 463.alb & 1 & 0 & Optimal &  1.62 & 26 & 26.00 &  0.00\\
instance n=100 464.alb & 1 & 0 & Optimal &  4.91 & 25 & 25.00 &  0.00\\
instance n=100 465.alb & 1 & 0 & Optimal &  4.37 & 22 & 22.00 &  0.00\\
instance n=100 466.alb & 1 & 0 & Optimal &  3.47 & 26 & 25.00 &  3.85\\
instance n=100 467.alb & 1 & 0 & Optimal &  6.89 & 21 & 21.00 &  0.00\\
instance n=100 468.alb & 1 & 0 & Optimal &  8.72 & 25 & 25.00 &  0.00\\
instance n=100 469.alb & 1 & 0 & Optimal &  1.76 & 22 & 22.00 &  0.00\\
instance n=100 47.alb & 1 & 0 & Optimal &  0.08 & 14 & 14.00 &  0.00\\
instance n=100 470.alb & 1 & 0 & Optimal & 34.68 & 26 & 26.00 &  0.00\\
instance n=100 471.alb & 1 & 0 & Optimal &  7.15 & 26 & 26.00 &  0.00\\
instance n=100 472.alb & 1 & 0 & Optimal &  0.85 & 23 & 23.00 &  0.00\\
instance n=100 473.alb & 1 & 0 & Optimal &  2.90 & 28 & 28.00 &  0.00\\
instance n=100 474.alb & 1 & 0 & Optimal &  1.88 & 23 & 23.00 &  0.00\\
instance n=100 475.alb & 1 & 0 & Optimal & 33.97 & 24 & 24.00 &  0.00\\
instance n=100 476.alb & 1 & 0 & Optimal &  0.49 & 14 & 14.00 &  0.00\\
instance n=100 477.alb & 1 & 0 & Optimal &  0.52 & 14 & 14.00 &  0.00\\
instance n=100 478.alb & 1 & 0 & Optimal &  0.71 & 14 & 14.00 &  0.00\\
instance n=100 479.alb & 1 & 0 & Optimal &  0.97 & 16 & 16.00 &  0.00\\
instance n=100 48.alb & 1 & 0 & Optimal &  0.07 & 15 & 15.00 &  0.00\\
instance n=100 480.alb & 1 & 0 & Optimal &  1.16 & 15 & 15.00 &  0.00\\
instance n=100 481.alb & 1 & 0 & Optimal &  1.82 & 15 & 15.00 &  0.00\\
instance n=100 482.alb & 1 & 0 & Optimal &  2.27 & 15 & 15.00 &  0.00\\
instance n=100 483.alb & 1 & 0 & Optimal &  1.41 & 14 & 14.00 &  0.00\\
instance n=100 484.alb & 1 & 0 & Optimal &  0.54 & 14 & 14.00 &  0.00\\
instance n=100 485.alb & 1 & 0 & Optimal &  1.91 & 16 & 16.00 &  0.00\\
instance n=100 486.alb & 1 & 0 & Optimal &  0.85 & 15 & 15.00 &  0.00\\
instance n=100 487.alb & 1 & 0 & Optimal &  1.90 & 15 & 15.00 &  0.00\\
instance n=100 488.alb & 1 & 0 & Optimal &  1.12 & 16 & 16.00 &  0.00\\
instance n=100 489.alb & 1 & 0 & Optimal &  3.57 & 13 & 13.00 &  0.00\\
instance n=100 49.alb & 1 & 0 & Optimal &  0.08 & 14 & 14.00 &  0.00\\
instance n=100 490.alb & 1 & 0 & Optimal &  1.13 & 15 & 15.00 &  0.00\\
instance n=100 491.alb & 1 & 0 & Optimal &  1.38 & 16 & 16.00 &  0.00\\
instance n=100 492.alb & 1 & 0 & Optimal &  2.66 & 14 & 14.00 &  0.00\\
instance n=100 493.alb & 1 & 0 & Optimal &  1.67 & 14 & 14.00 &  0.00\\
instance n=100 494.alb & 1 & 0 & Optimal &  0.72 & 14 & 14.00 &  0.00\\
instance n=100 495.alb & 1 & 0 & Optimal &  1.54 & 15 & 15.00 &  0.00\\
instance n=100 496.alb & 1 & 0 & Optimal &  1.23 & 14 & 14.00 &  0.00\\
instance n=100 497.alb & 1 & 0 & Optimal &  0.52 & 13 & 13.00 &  0.00\\
instance n=100 498.alb & 1 & 0 & Optimal &  1.04 & 14 & 14.00 &  0.00\\
instance n=100 499.alb & 1 & 0 & Optimal &  1.26 & 14 & 14.00 &  0.00\\
instance n=100 5.alb & 1 & 0 & Optimal &  0.08 & 22 & 22.00 &  0.00\\
instance n=100 50.alb & 1 & 0 & Optimal &  0.06 & 14 & 14.00 &  0.00\\
instance n=100 500.alb & 1 & 0 & Optimal &  0.70 & 14 & 14.00 &  0.00\\
instance n=100 501.alb & 1 & 0 & Solution & 120.09 & 63 & 58.00 &  7.94\\
instance n=100 502.alb & 1 & 0 & Solution & 120.08 & 64 & 61.00 &  4.69\\
instance n=100 503.alb & 1 & 0 & Solution & 120.06 & 60 & 55.00 &  8.33\\
instance n=100 504.alb & 1 & 0 & Solution & 120.06 & 60 & 58.00 &  3.33\\
instance n=100 505.alb & 1 & 0 & Solution & 120.07 & 61 & 55.00 &  9.84\\
instance n=100 506.alb & 1 & 0 & Solution & 120.08 & 58 & 53.00 &  8.62\\
instance n=100 507.alb & 1 & 0 & Solution & 120.05 & 59 & 55.00 &  6.78\\
instance n=100 508.alb & 1 & 0 & Optimal & 118.93 & 56 & 56.00 &  0.00\\
instance n=100 509.alb & 1 & 0 & Solution & 120.11 & 57 & 54.00 &  5.26\\
instance n=100 51.alb & 1 & 0 & Solution & 120.03 & 51 & 48.00 &  5.88\\
instance n=100 510.alb & 1 & 0 & Solution & 120.05 & 58 & 55.00 &  5.17\\
instance n=100 511.alb & 1 & 0 & Solution & 120.08 & 60 & 57.00 &  5.00\\
instance n=100 512.alb & 1 & 0 & Solution & 120.09 & 60 & 58.00 &  3.33\\
instance n=100 513.alb & 1 & 0 & Solution & 120.10 & 62 & 56.00 &  9.68\\
instance n=100 514.alb & 1 & 0 & Solution & 120.04 & 58 & 55.00 &  5.17\\
instance n=100 515.alb & 1 & 0 & Solution & 120.10 & 61 & 56.00 &  8.20\\
instance n=100 516.alb & 1 & 0 & Solution & 120.09 & 70 & 60.00 & 14.29\\
instance n=100 517.alb & 1 & 0 & Solution & 120.10 & 62 & 57.00 &  8.06\\
instance n=100 518.alb & 1 & 0 & Solution & 120.05 & 57 & 53.00 &  7.02\\
instance n=100 519.alb & 1 & 0 & Solution & 120.07 & 61 & 58.00 &  4.92\\
instance n=100 52.alb & 1 & 0 & Solution & 120.01 & 53 & 50.00 &  5.66\\
instance n=100 520.alb & 1 & 0 & Solution & 120.03 & 60 & 56.00 &  6.67\\
instance n=100 521.alb & 1 & 0 & Solution & 120.06 & 70 & 61.00 & 12.86\\
instance n=100 522.alb & 1 & 0 & Solution & 120.11 & 59 & 55.00 &  6.78\\
instance n=100 523.alb & 1 & 0 & Solution & 120.10 & 55 & 53.00 &  3.64\\
instance n=100 524.alb & 1 & 0 & Solution & 120.10 & 59 & 55.00 &  6.78\\
instance n=100 525.alb & 1 & 0 & Solution & 120.08 & 62 & 56.00 &  9.68\\
instance n=100 53.alb & 1 & 0 & Solution & 120.00 & 52 & 50.00 &  3.85\\
instance n=100 54.alb & 1 & 0 & Solution & 120.01 & 51 & 49.00 &  3.92\\
instance n=100 55.alb & 1 & 0 & Solution & 120.02 & 53 & 50.00 &  5.66\\
instance n=100 56.alb & 1 & 0 & Solution & 120.01 & 52 & 50.00 &  3.85\\
instance n=100 57.alb & 1 & 0 & Solution & 120.01 & 55 & 51.00 &  7.27\\
instance n=100 58.alb & 1 & 0 & Solution & 120.01 & 57 & 52.00 &  8.77\\
instance n=100 59.alb & 1 & 0 & Solution & 120.02 & 57 & 51.00 & 10.53\\
instance n=100 6.alb & 1 & 0 & Optimal &  0.36 & 22 & 22.00 &  0.00\\
instance n=100 60.alb & 1 & 0 & Solution & 120.01 & 54 & 51.00 &  5.56\\
instance n=100 61.alb & 1 & 0 & Solution & 120.02 & 54 & 51.00 &  5.56\\
instance n=100 62.alb & 1 & 0 & Solution & 120.01 & 52 & 49.00 &  5.77\\
instance n=100 63.alb & 1 & 0 & Solution & 120.01 & 61 & 52.00 & 14.75\\
instance n=100 64.alb & 1 & 0 & Solution & 120.00 & 57 & 51.00 & 10.53\\
instance n=100 65.alb & 1 & 0 & Solution & 120.03 & 62 & 53.00 & 14.52\\
instance n=100 66.alb & 1 & 0 & Solution & 120.01 & 52 & 49.00 &  5.77\\
instance n=100 67.alb & 1 & 0 & Solution & 120.02 & 55 & 51.00 &  7.27\\
instance n=100 68.alb & 1 & 0 & Solution & 120.03 & 57 & 49.00 & 14.04\\
instance n=100 69.alb & 1 & 0 & Solution & 120.02 & 53 & 51.00 &  3.77\\
instance n=100 7.alb & 1 & 0 & Optimal &  0.05 & 26 & 26.00 &  0.00\\
instance n=100 70.alb & 1 & 0 & Solution & 120.02 & 53 & 50.00 &  5.66\\
instance n=100 71.alb & 1 & 0 & Solution & 120.02 & 53 & 50.00 &  5.66\\
instance n=100 72.alb & 1 & 0 & Solution & 120.03 & 54 & 50.00 &  7.41\\
instance n=100 73.alb & 1 & 0 & Solution & 120.03 & 56 & 52.00 &  7.14\\
instance n=100 74.alb & 1 & 0 & Solution & 120.03 & 52 & 49.00 &  5.77\\
instance n=100 75.alb & 1 & 0 & Solution & 120.03 & 55 & 51.00 &  7.27\\
instance n=100 76.alb & 1 & 0 & Optimal &  0.15 & 23 & 23.00 &  0.00\\
instance n=100 77.alb & 1 & 0 & Optimal &  0.13 & 20 & 20.00 &  0.00\\
instance n=100 78.alb & 1 & 0 & Optimal &  3.36 & 21 & 21.00 &  0.00\\
instance n=100 79.alb & 1 & 0 & Optimal &  0.14 & 21 & 21.00 &  0.00\\
instance n=100 8.alb & 1 & 0 & Optimal &  0.09 & 24 & 24.00 &  0.00\\
instance n=100 80.alb & 1 & 0 & Optimal &  2.20 & 22 & 22.00 &  0.00\\
instance n=100 81.alb & 1 & 0 & Optimal &  2.78 & 20 & 20.00 &  0.00\\
instance n=100 82.alb & 1 & 0 & Optimal &  0.87 & 21 & 21.00 &  0.00\\
instance n=100 83.alb & 1 & 0 & Optimal &  0.12 & 22 & 22.00 &  0.00\\
instance n=100 84.alb & 1 & 0 & Solution & 120.03 & 27 & 26.00 &  3.70\\
instance n=100 85.alb & 1 & 0 & Solution & 120.01 & 25 & 24.00 &  4.00\\
instance n=100 86.alb & 1 & 0 & Optimal &  0.34 & 23 & 23.00 &  0.00\\
instance n=100 87.alb & 1 & 0 & Optimal &  0.13 & 22 & 22.00 &  0.00\\
instance n=100 88.alb & 1 & 0 & Solution & 120.01 & 24 & 23.00 &  4.17\\
instance n=100 89.alb & 1 & 0 & Optimal &  1.33 & 24 & 24.00 &  0.00\\
instance n=100 9.alb & 1 & 0 & Optimal & 24.45 & 23 & 23.00 &  0.00\\
instance n=100 90.alb & 1 & 0 & Solution & 120.03 & 21 & 20.00 &  4.76\\
instance n=100 91.alb & 1 & 0 & Optimal &  0.17 & 25 & 25.00 &  0.00\\
instance n=100 92.alb & 1 & 0 & Optimal &  0.12 & 24 & 24.00 &  0.00\\
instance n=100 93.alb & 1 & 0 & Optimal &  5.69 & 27 & 27.00 &  0.00\\
instance n=100 94.alb & 1 & 0 & Optimal &  4.16 & 22 & 22.00 &  0.00\\
instance n=100 95.alb & 1 & 0 & Optimal &  1.53 & 21 & 21.00 &  0.00\\
instance n=100 96.alb & 1 & 0 & Optimal &  1.63 & 21 & 21.00 &  0.00\\
instance n=100 97.alb & 1 & 0 & Optimal &  0.99 & 22 & 22.00 &  0.00\\
instance n=100 98.alb & 1 & 0 & Optimal &  0.29 & 22 & 22.00 &  0.00\\
instance n=100 99.alb & 1 & 0 & Optimal &  0.15 & 22 & 22.00 &  0.00\\
instance n=20 1.alb & 1 & 0 & Optimal &  0.03 & 3 &  3.00 &  0.00\\
instance n=20 10.alb & 1 & 0 & Optimal &  0.03 & 3 &  3.00 &  0.00\\
instance n=20 100.alb & 1 & 0 & Optimal &  0.38 & 11 & 11.00 &  0.00\\
instance n=20 101.alb & 1 & 0 & Optimal &  4.28 & 13 & 13.00 &  0.00\\
instance n=20 102.alb & 1 & 0 & Optimal &  0.82 & 13 & 13.00 &  0.00\\
instance n=20 103.alb & 1 & 0 & Optimal &  0.25 & 12 & 12.00 &  0.00\\
instance n=20 104.alb & 1 & 0 & Optimal &  0.25 & 11 & 11.00 &  0.00\\
instance n=20 105.alb & 1 & 0 & Optimal &  0.24 & 12 & 12.00 &  0.00\\
instance n=20 106.alb & 1 & 0 & Optimal &  0.08 & 10 & 10.00 &  0.00\\
instance n=20 107.alb & 1 & 0 & Optimal &  2.06 & 14 & 14.00 &  0.00\\
instance n=20 108.alb & 1 & 0 & Optimal &  3.49 & 15 & 15.00 &  0.00\\
instance n=20 109.alb & 1 & 0 & Optimal &  0.56 & 12 & 12.00 &  0.00\\
instance n=20 11.alb & 1 & 0 & Optimal &  0.02 & 3 &  3.00 &  0.00\\
instance n=20 110.alb & 1 & 0 & Optimal &  0.24 & 11 & 11.00 &  0.00\\
instance n=20 111.alb & 1 & 0 & Optimal &  0.58 & 13 & 13.00 &  0.00\\
instance n=20 112.alb & 1 & 0 & Optimal &  0.27 & 11 & 11.00 &  0.00\\
instance n=20 113.alb & 1 & 0 & Optimal &  0.64 & 12 & 12.00 &  0.00\\
instance n=20 114.alb & 1 & 0 & Optimal &  0.88 & 13 & 13.00 &  0.00\\
instance n=20 115.alb & 1 & 0 & Optimal &  0.16 & 11 & 11.00 &  0.00\\
instance n=20 116.alb & 1 & 0 & Optimal &  0.09 & 5 &  5.00 &  0.00\\
instance n=20 117.alb & 1 & 0 & Optimal &  0.09 & 5 &  5.00 &  0.00\\
instance n=20 118.alb & 1 & 0 & Optimal &  0.06 & 5 &  5.00 &  0.00\\
instance n=20 119.alb & 1 & 0 & Optimal &  0.12 & 6 &  6.00 &  0.00\\
instance n=20 12.alb & 1 & 0 & Optimal &  0.02 & 3 &  3.00 &  0.00\\
instance n=20 120.alb & 1 & 0 & Optimal &  0.08 & 6 &  6.00 &  0.00\\
instance n=20 121.alb & 1 & 0 & Optimal &  0.11 & 5 &  5.00 &  0.00\\
instance n=20 122.alb & 1 & 0 & Optimal &  0.09 & 6 &  6.00 &  0.00\\
instance n=20 123.alb & 1 & 0 & Optimal &  0.10 & 5 &  5.00 &  0.00\\
instance n=20 124.alb & 1 & 0 & Optimal &  0.06 & 5 &  5.00 &  0.00\\
instance n=20 125.alb & 1 & 0 & Optimal &  0.08 & 5 &  5.00 &  0.00\\
instance n=20 126.alb & 1 & 0 & Optimal &  0.08 & 5 &  5.00 &  0.00\\
instance n=20 127.alb & 1 & 0 & Optimal &  0.09 & 4 &  4.00 &  0.00\\
instance n=20 128.alb & 1 & 0 & Optimal &  0.08 & 5 &  5.00 &  0.00\\
instance n=20 129.alb & 1 & 0 & Optimal &  0.09 & 5 &  5.00 &  0.00\\
instance n=20 13.alb & 1 & 0 & Optimal &  0.03 & 3 &  3.00 &  0.00\\
instance n=20 130.alb & 1 & 0 & Optimal &  0.09 & 6 &  6.00 &  0.00\\
instance n=20 131.alb & 1 & 0 & Optimal &  0.09 & 7 &  7.00 &  0.00\\
instance n=20 132.alb & 1 & 0 & Optimal &  0.06 & 4 &  4.00 &  0.00\\
instance n=20 133.alb & 1 & 0 & Optimal &  0.08 & 5 &  5.00 &  0.00\\
instance n=20 134.alb & 1 & 0 & Optimal &  0.11 & 6 &  6.00 &  0.00\\
instance n=20 135.alb & 1 & 0 & Optimal &  0.08 & 6 &  6.00 &  0.00\\
instance n=20 136.alb & 1 & 0 & Optimal &  0.39 & 6 &  6.00 &  0.00\\
instance n=20 137.alb & 1 & 0 & Optimal &  0.08 & 5 &  5.00 &  0.00\\
instance n=20 138.alb & 1 & 0 & Optimal &  0.16 & 5 &  5.00 &  0.00\\
instance n=20 139.alb & 1 & 0 & Optimal &  0.11 & 5 &  5.00 &  0.00\\
instance n=20 14.alb & 1 & 0 & Optimal &  0.03 & 3 &  3.00 &  0.00\\
instance n=20 140.alb & 1 & 0 & Optimal &  0.09 & 5 &  5.00 &  0.00\\
instance n=20 141.alb & 1 & 0 & Optimal &  0.07 & 3 &  3.00 &  0.00\\
instance n=20 142.alb & 1 & 0 & Optimal &  0.09 & 3 &  3.00 &  0.00\\
instance n=20 143.alb & 1 & 0 & Optimal &  0.11 & 3 &  3.00 &  0.00\\
instance n=20 144.alb & 1 & 0 & Optimal &  0.09 & 4 &  4.00 &  0.00\\
instance n=20 145.alb & 1 & 0 & Optimal &  0.12 & 3 &  3.00 &  0.00\\
instance n=20 146.alb & 1 & 0 & Optimal &  0.08 & 3 &  3.00 &  0.00\\
instance n=20 147.alb & 1 & 0 & Optimal &  0.10 & 3 &  3.00 &  0.00\\
instance n=20 148.alb & 1 & 0 & Optimal &  0.08 & 3 &  3.00 &  0.00\\
instance n=20 149.alb & 1 & 0 & Optimal &  0.11 & 3 &  3.00 &  0.00\\
instance n=20 15.alb & 1 & 0 & Optimal &  0.03 & 3 &  3.00 &  0.00\\
instance n=20 150.alb & 1 & 0 & Optimal &  0.08 & 3 &  3.00 &  0.00\\
instance n=20 151.alb & 1 & 0 & Optimal &  0.08 & 3 &  3.00 &  0.00\\
instance n=20 152.alb & 1 & 0 & Optimal &  0.10 & 3 &  3.00 &  0.00\\
instance n=20 153.alb & 1 & 0 & Optimal &  0.11 & 3 &  3.00 &  0.00\\
instance n=20 154.alb & 1 & 0 & Optimal &  0.09 & 3 &  3.00 &  0.00\\
instance n=20 155.alb & 1 & 0 & Optimal &  0.08 & 3 &  3.00 &  0.00\\
instance n=20 156.alb & 1 & 0 & Optimal &  0.10 & 3 &  3.00 &  0.00\\
instance n=20 157.alb & 1 & 0 & Optimal &  0.10 & 3 &  3.00 &  0.00\\
instance n=20 158.alb & 1 & 0 & Optimal &  0.08 & 3 &  3.00 &  0.00\\
instance n=20 159.alb & 1 & 0 & Optimal &  0.09 & 3 &  3.00 &  0.00\\
instance n=20 16.alb & 1 & 0 & Optimal &  0.36 & 12 & 12.00 &  0.00\\
instance n=20 160.alb & 1 & 0 & Optimal &  0.09 & 3 &  3.00 &  0.00\\
instance n=20 161.alb & 1 & 0 & Optimal &  0.07 & 3 &  3.00 &  0.00\\
instance n=20 162.alb & 1 & 0 & Optimal &  0.10 & 3 &  3.00 &  0.00\\
instance n=20 163.alb & 1 & 0 & Optimal &  0.09 & 3 &  3.00 &  0.00\\
instance n=20 164.alb & 1 & 0 & Optimal &  0.08 & 4 &  4.00 &  0.00\\
instance n=20 165.alb & 1 & 0 & Optimal &  0.09 & 3 &  3.00 &  0.00\\
instance n=20 166.alb & 1 & 0 & Optimal &  5.90 & 12 & 12.00 &  0.00\\
instance n=20 167.alb & 1 & 0 & Optimal &  2.21 & 11 & 11.00 &  0.00\\
instance n=20 168.alb & 1 & 0 & Optimal &  0.36 & 10 & 10.00 &  0.00\\
instance n=20 169.alb & 1 & 0 & Optimal &  0.91 & 11 & 11.00 &  0.00\\
instance n=20 17.alb & 1 & 0 & Optimal &  0.04 & 10 & 10.00 &  0.00\\
instance n=20 170.alb & 1 & 0 & Optimal &  0.26 & 11 & 11.00 &  0.00\\
instance n=20 171.alb & 1 & 0 & Optimal & 71.31 & 13 & 13.00 &  0.00\\
instance n=20 172.alb & 1 & 0 & Optimal &  0.35 & 11 & 11.00 &  0.00\\
instance n=20 173.alb & 1 & 0 & Optimal &  0.10 & 11 & 11.00 &  0.00\\
instance n=20 174.alb & 1 & 0 & Optimal &  1.84 & 12 & 12.00 &  0.00\\
instance n=20 175.alb & 1 & 0 & Optimal &  0.10 & 10 & 10.00 &  0.00\\
instance n=20 176.alb & 1 & 0 & Optimal &  1.66 & 11 & 11.00 &  0.00\\
instance n=20 177.alb & 1 & 0 & Optimal &  2.55 & 10 & 10.00 &  0.00\\
instance n=20 178.alb & 1 & 0 & Optimal &  0.30 & 11 & 11.00 &  0.00\\
instance n=20 179.alb & 1 & 0 & Optimal &  0.17 & 11 & 11.00 &  0.00\\
instance n=20 18.alb & 1 & 0 & Optimal &  0.35 & 11 & 11.00 &  0.00\\
instance n=20 180.alb & 1 & 0 & Optimal & 20.93 & 13 & 13.00 &  0.00\\
instance n=20 181.alb & 1 & 0 & Optimal &  0.31 & 11 & 11.00 &  0.00\\
instance n=20 182.alb & 1 & 0 & Optimal &  3.31 & 11 & 11.00 &  0.00\\
instance n=20 183.alb & 1 & 0 & Optimal & 15.45 & 13 & 13.00 &  0.00\\
instance n=20 184.alb & 1 & 0 & Optimal &  2.65 & 12 & 12.00 &  0.00\\
instance n=20 185.alb & 1 & 0 & Optimal & 25.00 & 15 & 15.00 &  0.00\\
instance n=20 186.alb & 1 & 0 & Optimal & 17.88 & 14 & 14.00 &  0.00\\
instance n=20 187.alb & 1 & 0 & Optimal &  0.12 & 10 & 10.00 &  0.00\\
instance n=20 188.alb & 1 & 0 & Optimal &  0.73 & 11 & 11.00 &  0.00\\
instance n=20 189.alb & 1 & 0 & Optimal &  3.72 & 13 & 13.00 &  0.00\\
instance n=20 19.alb & 1 & 0 & Optimal &  3.65 & 14 & 14.00 &  0.00\\
instance n=20 190.alb & 1 & 0 & Optimal & 69.35 & 15 & 15.00 &  0.00\\
instance n=20 191.alb & 1 & 0 & Optimal &  0.14 & 4 &  4.00 &  0.00\\
instance n=20 192.alb & 1 & 0 & Optimal &  0.13 & 5 &  5.00 &  0.00\\
instance n=20 193.alb & 1 & 0 & Optimal &  0.08 & 5 &  5.00 &  0.00\\
instance n=20 194.alb & 1 & 0 & Optimal &  0.09 & 6 &  6.00 &  0.00\\
instance n=20 195.alb & 1 & 0 & Optimal &  0.12 & 6 &  6.00 &  0.00\\
instance n=20 196.alb & 1 & 0 & Optimal &  0.14 & 5 &  5.00 &  0.00\\
instance n=20 197.alb & 1 & 0 & Optimal &  0.16 & 4 &  4.00 &  0.00\\
instance n=20 198.alb & 1 & 0 & Optimal &  0.15 & 6 &  6.00 &  0.00\\
instance n=20 199.alb & 1 & 0 & Optimal &  0.15 & 5 &  5.00 &  0.00\\
instance n=20 2.alb & 1 & 0 & Optimal &  0.02 & 3 &  3.00 &  0.00\\
instance n=20 20.alb & 1 & 0 & Optimal &  0.25 & 11 & 11.00 &  0.00\\
instance n=20 200.alb & 1 & 0 & Optimal &  0.13 & 6 &  6.00 &  0.00\\
instance n=20 201.alb & 1 & 0 & Optimal &  0.11 & 6 &  6.00 &  0.00\\
instance n=20 202.alb & 1 & 0 & Optimal &  0.58 & 4 &  4.00 &  0.00\\
instance n=20 203.alb & 1 & 0 & Optimal &  0.14 & 4 &  4.00 &  0.00\\
instance n=20 204.alb & 1 & 0 & Optimal &  0.11 & 5 &  5.00 &  0.00\\
instance n=20 205.alb & 1 & 0 & Optimal &  0.11 & 6 &  6.00 &  0.00\\
instance n=20 206.alb & 1 & 0 & Optimal &  0.11 & 5 &  5.00 &  0.00\\
instance n=20 207.alb & 1 & 0 & Optimal &  0.16 & 6 &  6.00 &  0.00\\
instance n=20 208.alb & 1 & 0 & Optimal &  0.16 & 5 &  5.00 &  0.00\\
instance n=20 209.alb & 1 & 0 & Optimal &  0.13 & 4 &  4.00 &  0.00\\
instance n=20 21.alb & 1 & 0 & Optimal &  1.87 & 14 & 14.00 &  0.00\\
instance n=20 210.alb & 1 & 0 & Optimal &  0.13 & 5 &  5.00 &  0.00\\
instance n=20 211.alb & 1 & 0 & Optimal &  0.13 & 5 &  5.00 &  0.00\\
instance n=20 212.alb & 1 & 0 & Optimal &  0.13 & 5 &  5.00 &  0.00\\
instance n=20 213.alb & 1 & 0 & Optimal &  0.11 & 5 &  5.00 &  0.00\\
instance n=20 214.alb & 1 & 0 & Optimal &  0.10 & 5 &  5.00 &  0.00\\
instance n=20 215.alb & 1 & 0 & Optimal &  0.10 & 5 &  5.00 &  0.00\\
instance n=20 216.alb & 1 & 0 & Optimal &  0.10 & 3 &  3.00 &  0.00\\
instance n=20 217.alb & 1 & 0 & Optimal &  0.20 & 4 &  4.00 &  0.00\\
instance n=20 218.alb & 1 & 0 & Optimal &  0.10 & 3 &  3.00 &  0.00\\
instance n=20 219.alb & 1 & 0 & Optimal &  0.09 & 3 &  3.00 &  0.00\\
instance n=20 22.alb & 1 & 0 & Optimal &  0.53 & 12 & 12.00 &  0.00\\
instance n=20 220.alb & 1 & 0 & Optimal &  0.14 & 3 &  3.00 &  0.00\\
instance n=20 221.alb & 1 & 0 & Optimal &  0.10 & 3 &  3.00 &  0.00\\
instance n=20 222.alb & 1 & 0 & Optimal &  0.08 & 3 &  3.00 &  0.00\\
instance n=20 223.alb & 1 & 0 & Optimal &  0.08 & 3 &  3.00 &  0.00\\
instance n=20 224.alb & 1 & 0 & Optimal &  0.10 & 3 &  3.00 &  0.00\\
instance n=20 225.alb & 1 & 0 & Optimal &  0.10 & 3 &  3.00 &  0.00\\
instance n=20 226.alb & 1 & 0 & Optimal &  0.08 & 3 &  3.00 &  0.00\\
instance n=20 227.alb & 1 & 0 & Optimal &  0.11 & 3 &  3.00 &  0.00\\
instance n=20 228.alb & 1 & 0 & Optimal &  0.06 & 2 &  2.00 &  0.00\\
instance n=20 229.alb & 1 & 0 & Optimal &  0.12 & 3 &  3.00 &  0.00\\
instance n=20 23.alb & 1 & 0 & Optimal & 12.65 & 13 & 13.00 &  0.00\\
instance n=20 230.alb & 1 & 0 & Optimal &  0.10 & 3 &  3.00 &  0.00\\
instance n=20 231.alb & 1 & 0 & Optimal &  0.08 & 3 &  3.00 &  0.00\\
instance n=20 232.alb & 1 & 0 & Optimal &  0.12 & 3 &  3.00 &  0.00\\
instance n=20 233.alb & 1 & 0 & Optimal &  0.10 & 3 &  3.00 &  0.00\\
instance n=20 234.alb & 1 & 0 & Optimal &  0.10 & 3 &  3.00 &  0.00\\
instance n=20 235.alb & 1 & 0 & Optimal &  0.08 & 3 &  3.00 &  0.00\\
instance n=20 236.alb & 1 & 0 & Optimal &  0.08 & 3 &  3.00 &  0.00\\
instance n=20 237.alb & 1 & 0 & Optimal &  0.10 & 3 &  3.00 &  0.00\\
instance n=20 238.alb & 1 & 0 & Optimal &  0.10 & 3 &  3.00 &  0.00\\
instance n=20 239.alb & 1 & 0 & Optimal &  0.12 & 3 &  3.00 &  0.00\\
instance n=20 24.alb & 1 & 0 & Optimal &  0.10 & 11 & 11.00 &  0.00\\
instance n=20 240.alb & 1 & 0 & Optimal &  0.10 & 3 &  3.00 &  0.00\\
instance n=20 241.alb & 1 & 0 & Optimal &  0.86 & 13 & 13.00 &  0.00\\
instance n=20 242.alb & 1 & 0 & Optimal &  0.52 & 12 & 12.00 &  0.00\\
instance n=20 243.alb & 1 & 0 & Optimal &  0.53 & 10 & 10.00 &  0.00\\
instance n=20 244.alb & 1 & 0 & Optimal &  0.47 & 11 & 11.00 &  0.00\\
instance n=20 245.alb & 1 & 0 & Optimal &  0.47 & 13 & 13.00 &  0.00\\
instance n=20 246.alb & 1 & 0 & Optimal &  1.38 & 13 & 13.00 &  0.00\\
instance n=20 247.alb & 1 & 0 & Optimal &  0.33 & 11 & 11.00 &  0.00\\
instance n=20 248.alb & 1 & 0 & Optimal &  0.47 & 11 & 11.00 &  0.00\\
instance n=20 249.alb & 1 & 0 & Optimal &  1.32 & 13 & 13.00 &  0.00\\
instance n=20 25.alb & 1 & 0 & Optimal &  0.19 & 11 & 11.00 &  0.00\\
instance n=20 250.alb & 1 & 0 & Optimal &  0.15 & 10 & 10.00 &  0.00\\
instance n=20 251.alb & 1 & 0 & Optimal &  0.49 & 12 & 12.00 &  0.00\\
instance n=20 252.alb & 1 & 0 & Optimal &  1.19 & 11 & 11.00 &  0.00\\
instance n=20 253.alb & 1 & 0 & Optimal &  1.37 & 13 & 13.00 &  0.00\\
instance n=20 254.alb & 1 & 0 & Optimal &  0.50 & 12 & 12.00 &  0.00\\
instance n=20 255.alb & 1 & 0 & Optimal &  2.09 & 13 & 13.00 &  0.00\\
instance n=20 256.alb & 1 & 0 & Optimal &  0.97 & 14 & 14.00 &  0.00\\
instance n=20 257.alb & 1 & 0 & Optimal &  0.11 & 10 & 10.00 &  0.00\\
instance n=20 258.alb & 1 & 0 & Optimal &  0.88 & 13 & 13.00 &  0.00\\
instance n=20 259.alb & 1 & 0 & Optimal &  0.50 & 13 & 13.00 &  0.00\\
instance n=20 26.alb & 1 & 0 & Optimal &  0.99 & 12 & 12.00 &  0.00\\
instance n=20 260.alb & 1 & 0 & Optimal &  1.71 & 12 & 12.00 &  0.00\\
instance n=20 261.alb & 1 & 0 & Optimal &  0.99 & 12 & 12.00 &  0.00\\
instance n=20 262.alb & 1 & 0 & Optimal &  0.57 & 11 & 11.00 &  0.00\\
instance n=20 263.alb & 1 & 0 & Optimal &  0.93 & 12 & 12.00 &  0.00\\
instance n=20 264.alb & 1 & 0 & Optimal &  0.88 & 12 & 12.00 &  0.00\\
instance n=20 265.alb & 1 & 0 & Optimal &  0.52 & 12 & 12.00 &  0.00\\
instance n=20 266.alb & 1 & 0 & Optimal &  0.66 & 5 &  5.00 &  0.00\\
instance n=20 267.alb & 1 & 0 & Optimal &  0.11 & 6 &  6.00 &  0.00\\
instance n=20 268.alb & 1 & 0 & Optimal &  0.16 & 6 &  6.00 &  0.00\\
instance n=20 269.alb & 1 & 0 & Optimal &  0.71 & 7 &  7.00 &  0.00\\
instance n=20 27.alb & 1 & 0 & Optimal &  3.25 & 13 & 13.00 &  0.00\\
instance n=20 270.alb & 1 & 0 & Optimal &  0.69 & 7 &  7.00 &  0.00\\
instance n=20 271.alb & 1 & 0 & Optimal &  0.66 & 6 &  6.00 &  0.00\\
instance n=20 272.alb & 1 & 0 & Optimal &  0.15 & 5 &  5.00 &  0.00\\
instance n=20 273.alb & 1 & 0 & Optimal &  0.11 & 5 &  5.00 &  0.00\\
instance n=20 274.alb & 1 & 0 & Optimal &  0.69 & 6 &  6.00 &  0.00\\
instance n=20 275.alb & 1 & 0 & Optimal &  0.14 & 5 &  5.00 &  0.00\\
instance n=20 276.alb & 1 & 0 & Optimal &  0.17 & 4 &  4.00 &  0.00\\
instance n=20 277.alb & 1 & 0 & Optimal &  0.13 & 4 &  4.00 &  0.00\\
instance n=20 278.alb & 1 & 0 & Optimal &  0.76 & 6 &  6.00 &  0.00\\
instance n=20 279.alb & 1 & 0 & Optimal &  0.17 & 6 &  6.00 &  0.00\\
instance n=20 28.alb & 1 & 0 & Optimal &  2.05 & 12 & 12.00 &  0.00\\
instance n=20 280.alb & 1 & 0 & Optimal &  0.17 & 5 &  5.00 &  0.00\\
instance n=20 281.alb & 1 & 0 & Optimal &  0.13 & 4 &  4.00 &  0.00\\
instance n=20 282.alb & 1 & 0 & Optimal &  0.21 & 4 &  4.00 &  0.00\\
instance n=20 283.alb & 1 & 0 & Optimal &  0.17 & 5 &  5.00 &  0.00\\
instance n=20 284.alb & 1 & 0 & Optimal &  0.11 & 5 &  5.00 &  0.00\\
instance n=20 285.alb & 1 & 0 & Optimal &  0.14 & 5 &  5.00 &  0.00\\
instance n=20 286.alb & 1 & 0 & Optimal &  0.14 & 5 &  5.00 &  0.00\\
instance n=20 287.alb & 1 & 0 & Optimal &  0.16 & 5 &  5.00 &  0.00\\
instance n=20 288.alb & 1 & 0 & Optimal &  0.16 & 6 &  6.00 &  0.00\\
instance n=20 289.alb & 1 & 0 & Optimal &  0.14 & 5 &  5.00 &  0.00\\
instance n=20 29.alb & 1 & 0 & Optimal &  0.03 & 10 & 10.00 &  0.00\\
instance n=20 290.alb & 1 & 0 & Optimal &  0.17 & 5 &  5.00 &  0.00\\
instance n=20 291.alb & 1 & 0 & Optimal &  0.24 & 3 &  3.00 &  0.00\\
instance n=20 292.alb & 1 & 0 & Optimal &  0.17 & 3 &  3.00 &  0.00\\
instance n=20 293.alb & 1 & 0 & Optimal &  0.14 & 3 &  3.00 &  0.00\\
instance n=20 294.alb & 1 & 0 & Optimal &  0.24 & 3 &  3.00 &  0.00\\
instance n=20 295.alb & 1 & 0 & Optimal &  0.14 & 3 &  3.00 &  0.00\\
instance n=20 296.alb & 1 & 0 & Optimal &  0.13 & 3 &  3.00 &  0.00\\
instance n=20 297.alb & 1 & 0 & Optimal &  0.19 & 3 &  3.00 &  0.00\\
instance n=20 298.alb & 1 & 0 & Optimal &  0.19 & 3 &  3.00 &  0.00\\
instance n=20 299.alb & 1 & 0 & Optimal &  0.16 & 3 &  3.00 &  0.00\\
instance n=20 3.alb & 1 & 0 & Optimal &  0.04 & 3 &  3.00 &  0.00\\
instance n=20 30.alb & 1 & 0 & Optimal & 14.70 & 16 & 16.00 &  0.00\\
instance n=20 300.alb & 1 & 0 & Optimal &  0.16 & 4 &  4.00 &  0.00\\
instance n=20 301.alb & 1 & 0 & Optimal &  0.16 & 3 &  3.00 &  0.00\\
instance n=20 302.alb & 1 & 0 & Optimal &  0.21 & 3 &  3.00 &  0.00\\
instance n=20 303.alb & 1 & 0 & Optimal &  0.19 & 3 &  3.00 &  0.00\\
instance n=20 304.alb & 1 & 0 & Optimal &  0.19 & 3 &  3.00 &  0.00\\
instance n=20 305.alb & 1 & 0 & Optimal &  0.22 & 3 &  3.00 &  0.00\\
instance n=20 306.alb & 1 & 0 & Optimal &  0.13 & 3 &  3.00 &  0.00\\
instance n=20 307.alb & 1 & 0 & Optimal &  0.22 & 3 &  3.00 &  0.00\\
instance n=20 308.alb & 1 & 0 & Optimal &  0.22 & 3 &  3.00 &  0.00\\
instance n=20 309.alb & 1 & 0 & Optimal &  0.14 & 3 &  3.00 &  0.00\\
instance n=20 31.alb & 1 & 0 & Optimal &  0.56 & 12 & 12.00 &  0.00\\
instance n=20 310.alb & 1 & 0 & Optimal &  0.17 & 3 &  3.00 &  0.00\\
instance n=20 311.alb & 1 & 0 & Optimal &  0.16 & 3 &  3.00 &  0.00\\
instance n=20 312.alb & 1 & 0 & Optimal &  0.19 & 4 &  4.00 &  0.00\\
instance n=20 313.alb & 1 & 0 & Optimal &  0.18 & 3 &  3.00 &  0.00\\
instance n=20 314.alb & 1 & 0 & Optimal &  0.16 & 3 &  3.00 &  0.00\\
instance n=20 315.alb & 1 & 0 & Optimal &  0.22 & 3 &  3.00 &  0.00\\
instance n=20 316.alb & 1 & 0 & Optimal &  0.17 & 10 & 10.00 &  0.00\\
instance n=20 317.alb & 1 & 0 & Optimal &  1.81 & 10 & 10.00 &  0.00\\
instance n=20 318.alb & 1 & 0 & Optimal &  0.30 & 10 & 10.00 &  0.00\\
instance n=20 319.alb & 1 & 0 & Optimal & 19.16 & 14 & 14.00 &  0.00\\
instance n=20 32.alb & 1 & 0 & Optimal & 15.54 & 13 & 13.00 &  0.00\\
instance n=20 320.alb & 1 & 0 & Optimal &  2.91 & 12 & 12.00 &  0.00\\
instance n=20 321.alb & 1 & 0 & Solution & 120.04 & 14 & 11.00 & 21.43\\
instance n=20 322.alb & 1 & 0 & Optimal & 21.71 & 12 & 12.00 &  0.00\\
instance n=20 323.alb & 1 & 0 & Optimal & 15.25 & 13 & 13.00 &  0.00\\
instance n=20 324.alb & 1 & 0 & Optimal &  0.57 & 9 &  9.00 &  0.00\\
instance n=20 325.alb & 1 & 0 & Solution & 120.04 & 14 & 12.00 & 14.29\\
instance n=20 326.alb & 1 & 0 & Optimal & 40.65 & 14 & 14.00 &  0.00\\
instance n=20 327.alb & 1 & 0 & Optimal & 42.57 & 13 & 13.00 &  0.00\\
instance n=20 328.alb & 1 & 0 & Optimal & 28.06 & 13 & 13.00 &  0.00\\
instance n=20 329.alb & 1 & 0 & Optimal &  0.33 & 10 & 10.00 &  0.00\\
instance n=20 33.alb & 1 & 0 & Optimal &  0.11 & 11 & 11.00 &  0.00\\
instance n=20 330.alb & 1 & 0 & Optimal & 21.95 & 12 & 12.00 &  0.00\\
instance n=20 331.alb & 1 & 0 & Optimal & 40.32 & 13 & 13.00 &  0.00\\
instance n=20 332.alb & 1 & 0 & Optimal &  6.12 & 13 & 13.00 &  0.00\\
instance n=20 333.alb & 1 & 0 & Optimal &  1.48 & 11 & 11.00 &  0.00\\
instance n=20 334.alb & 1 & 0 & Optimal &  0.21 & 10 & 10.00 &  0.00\\
instance n=20 335.alb & 1 & 0 & Solution & 120.05 & 14 & 11.00 & 21.43\\
instance n=20 336.alb & 1 & 0 & Optimal &  1.06 & 11 & 11.00 &  0.00\\
instance n=20 337.alb & 1 & 0 & Optimal &  0.18 & 10 & 10.00 &  0.00\\
instance n=20 338.alb & 1 & 0 & Optimal & 27.74 & 14 & 14.00 &  0.00\\
instance n=20 339.alb & 1 & 0 & Optimal & 35.97 & 13 & 13.00 &  0.00\\
instance n=20 34.alb & 1 & 0 & Optimal &  1.17 & 12 & 12.00 &  0.00\\
instance n=20 340.alb & 1 & 0 & Optimal &  2.52 & 11 & 11.00 &  0.00\\
instance n=20 341.alb & 1 & 0 & Optimal &  0.17 & 6 &  6.00 &  0.00\\
instance n=20 342.alb & 1 & 0 & Optimal &  0.18 & 6 &  6.00 &  0.00\\
instance n=20 343.alb & 1 & 0 & Optimal &  0.19 & 6 &  6.00 &  0.00\\
instance n=20 344.alb & 1 & 0 & Optimal &  0.17 & 6 &  6.00 &  0.00\\
instance n=20 345.alb & 1 & 0 & Optimal &  0.26 & 4 &  4.00 &  0.00\\
instance n=20 346.alb & 1 & 0 & Optimal &  0.32 & 5 &  5.00 &  0.00\\
instance n=20 347.alb & 1 & 0 & Optimal &  0.27 & 6 &  6.00 &  0.00\\
instance n=20 348.alb & 1 & 0 & Optimal &  0.22 & 5 &  5.00 &  0.00\\
instance n=20 349.alb & 1 & 0 & Optimal &  0.20 & 5 &  5.00 &  0.00\\
instance n=20 35.alb & 1 & 0 & Optimal &  0.43 & 12 & 12.00 &  0.00\\
instance n=20 350.alb & 1 & 0 & Optimal &  0.22 & 5 &  5.00 &  0.00\\
instance n=20 351.alb & 1 & 0 & Optimal &  0.21 & 5 &  5.00 &  0.00\\
instance n=20 352.alb & 1 & 0 & Optimal &  0.24 & 4 &  4.00 &  0.00\\
instance n=20 353.alb & 1 & 0 & Optimal &  0.19 & 6 &  6.00 &  0.00\\
instance n=20 354.alb & 1 & 0 & Optimal &  0.27 & 6 &  6.00 &  0.00\\
instance n=20 355.alb & 1 & 0 & Optimal &  0.17 & 5 &  5.00 &  0.00\\
instance n=20 356.alb & 1 & 0 & Optimal &  0.24 & 5 &  5.00 &  0.00\\
instance n=20 357.alb & 1 & 0 & Optimal &  0.27 & 5 &  5.00 &  0.00\\
instance n=20 358.alb & 1 & 0 & Optimal &  0.19 & 4 &  4.00 &  0.00\\
instance n=20 359.alb & 1 & 0 & Optimal &  0.21 & 4 &  4.00 &  0.00\\
instance n=20 36.alb & 1 & 0 & Optimal &  0.92 & 13 & 13.00 &  0.00\\
instance n=20 360.alb & 1 & 0 & Optimal &  0.27 & 6 &  6.00 &  0.00\\
instance n=20 361.alb & 1 & 0 & Optimal &  0.20 & 5 &  5.00 &  0.00\\
instance n=20 362.alb & 1 & 0 & Optimal &  0.17 & 5 &  5.00 &  0.00\\
instance n=20 363.alb & 1 & 0 & Optimal &  0.27 & 7 &  7.00 &  0.00\\
instance n=20 364.alb & 1 & 0 & Optimal &  0.20 & 4 &  4.00 &  0.00\\
instance n=20 365.alb & 1 & 0 & Optimal &  0.20 & 5 &  5.00 &  0.00\\
instance n=20 366.alb & 1 & 0 & Optimal &  0.15 & 3 &  3.00 &  0.00\\
instance n=20 367.alb & 1 & 0 & Optimal &  0.14 & 3 &  3.00 &  0.00\\
instance n=20 368.alb & 1 & 0 & Optimal &  0.14 & 3 &  3.00 &  0.00\\
instance n=20 369.alb & 1 & 0 & Optimal &  0.16 & 3 &  3.00 &  0.00\\
instance n=20 37.alb & 1 & 0 & Optimal &  0.67 & 12 & 12.00 &  0.00\\
instance n=20 370.alb & 1 & 0 & Optimal &  0.15 & 3 &  3.00 &  0.00\\
instance n=20 371.alb & 1 & 0 & Optimal &  0.16 & 3 &  3.00 &  0.00\\
instance n=20 372.alb & 1 & 0 & Optimal &  0.16 & 3 &  3.00 &  0.00\\
instance n=20 373.alb & 1 & 0 & Optimal &  0.24 & 3 &  3.00 &  0.00\\
instance n=20 374.alb & 1 & 0 & Optimal &  0.16 & 3 &  3.00 &  0.00\\
instance n=20 375.alb & 1 & 0 & Optimal &  0.21 & 3 &  3.00 &  0.00\\
instance n=20 376.alb & 1 & 0 & Optimal &  0.14 & 3 &  3.00 &  0.00\\
instance n=20 377.alb & 1 & 0 & Optimal &  0.17 & 3 &  3.00 &  0.00\\
instance n=20 378.alb & 1 & 0 & Optimal &  0.14 & 3 &  3.00 &  0.00\\
instance n=20 379.alb & 1 & 0 & Optimal &  0.24 & 4 &  4.00 &  0.00\\
instance n=20 38.alb & 1 & 0 & Optimal &  0.19 & 12 & 12.00 &  0.00\\
instance n=20 380.alb & 1 & 0 & Optimal &  0.15 & 3 &  3.00 &  0.00\\
instance n=20 381.alb & 1 & 0 & Optimal &  0.14 & 3 &  3.00 &  0.00\\
instance n=20 382.alb & 1 & 0 & Optimal &  0.24 & 4 &  4.00 &  0.00\\
instance n=20 383.alb & 1 & 0 & Optimal &  0.17 & 3 &  3.00 &  0.00\\
instance n=20 384.alb & 1 & 0 & Optimal &  0.16 & 3 &  3.00 &  0.00\\
instance n=20 385.alb & 1 & 0 & Optimal &  0.14 & 3 &  3.00 &  0.00\\
instance n=20 386.alb & 1 & 0 & Optimal &  0.14 & 3 &  3.00 &  0.00\\
instance n=20 387.alb & 1 & 0 & Optimal &  0.16 & 3 &  3.00 &  0.00\\
instance n=20 388.alb & 1 & 0 & Optimal &  0.18 & 3 &  3.00 &  0.00\\
instance n=20 389.alb & 1 & 0 & Optimal &  0.17 & 3 &  3.00 &  0.00\\
instance n=20 39.alb & 1 & 0 & Optimal &  0.36 & 13 & 13.00 &  0.00\\
instance n=20 390.alb & 1 & 0 & Optimal &  0.16 & 3 &  3.00 &  0.00\\
instance n=20 391.alb & 1 & 0 & Optimal &  0.74 & 11 & 10.00 &  9.09\\
instance n=20 392.alb & 1 & 0 & Optimal &  2.22 & 14 & 14.00 &  0.00\\
instance n=20 393.alb & 1 & 0 & Optimal &  2.23 & 11 & 10.00 &  9.09\\
instance n=20 394.alb & 1 & 0 & Optimal &  1.68 & 12 & 12.00 &  0.00\\
instance n=20 395.alb & 1 & 0 & Optimal &  0.72 & 12 & 12.00 &  0.00\\
instance n=20 396.alb & 1 & 0 & Optimal &  2.73 & 13 & 13.00 &  0.00\\
instance n=20 397.alb & 1 & 0 & Optimal &  0.94 & 10 & 10.00 &  0.00\\
instance n=20 398.alb & 1 & 0 & Optimal &  0.68 & 11 & 11.00 &  0.00\\
instance n=20 399.alb & 1 & 0 & Optimal &  1.61 & 13 & 13.00 &  0.00\\
instance n=20 4.alb & 1 & 0 & Optimal &  0.02 & 3 &  3.00 &  0.00\\
instance n=20 40.alb & 1 & 0 & Optimal &  1.37 & 12 & 12.00 &  0.00\\
instance n=20 400.alb & 1 & 0 & Optimal &  1.55 & 12 & 12.00 &  0.00\\
instance n=20 401.alb & 1 & 0 & Optimal &  1.60 & 12 & 12.00 &  0.00\\
instance n=20 402.alb & 1 & 0 & Optimal &  0.82 & 12 & 12.00 &  0.00\\
instance n=20 403.alb & 1 & 0 & Optimal &  1.63 & 12 & 12.00 &  0.00\\
instance n=20 404.alb & 1 & 0 & Optimal &  1.73 & 10 & 10.00 &  0.00\\
instance n=20 405.alb & 1 & 0 & Optimal &  1.63 & 12 & 12.00 &  0.00\\
instance n=20 406.alb & 1 & 0 & Optimal &  5.09 & 14 & 14.00 &  0.00\\
instance n=20 407.alb & 1 & 0 & Optimal &  0.24 & 10 & 10.00 &  0.00\\
instance n=20 408.alb & 1 & 0 & Optimal &  3.75 & 14 & 14.00 &  0.00\\
instance n=20 409.alb & 1 & 0 & Optimal &  1.19 & 12 & 12.00 &  0.00\\
instance n=20 41.alb & 1 & 0 & Optimal &  0.01 & 6 &  6.00 &  0.00\\
instance n=20 410.alb & 1 & 0 & Optimal &  1.26 & 11 & 11.00 &  0.00\\
instance n=20 411.alb & 1 & 0 & Optimal &  7.00 & 15 & 15.00 &  0.00\\
instance n=20 412.alb & 1 & 0 & Optimal &  0.99 & 11 & 11.00 &  0.00\\
instance n=20 413.alb & 1 & 0 & Optimal &  0.25 & 10 & 10.00 &  0.00\\
instance n=20 414.alb & 1 & 0 & Optimal &  3.08 & 12 & 12.00 &  0.00\\
instance n=20 415.alb & 1 & 0 & Optimal &  0.19 & 10 & 10.00 &  0.00\\
instance n=20 416.alb & 1 & 0 & Optimal &  0.24 & 6 &  6.00 &  0.00\\
instance n=20 417.alb & 1 & 0 & Optimal &  0.24 & 5 &  5.00 &  0.00\\
instance n=20 418.alb & 1 & 0 & Optimal &  0.19 & 6 &  6.00 &  0.00\\
instance n=20 419.alb & 1 & 0 & Optimal &  0.19 & 4 &  4.00 &  0.00\\
instance n=20 42.alb & 1 & 0 & Optimal &  0.03 & 5 &  5.00 &  0.00\\
instance n=20 420.alb & 1 & 0 & Optimal &  0.27 & 5 &  5.00 &  0.00\\
instance n=20 421.alb & 1 & 0 & Optimal &  0.25 & 6 &  6.00 &  0.00\\
instance n=20 422.alb & 1 & 0 & Optimal &  0.17 & 4 &  4.00 &  0.00\\
instance n=20 423.alb & 1 & 0 & Optimal &  0.20 & 6 &  6.00 &  0.00\\
instance n=20 424.alb & 1 & 0 & Optimal &  0.31 & 5 &  5.00 &  0.00\\
instance n=20 425.alb & 1 & 0 & Optimal &  0.27 & 6 &  6.00 &  0.00\\
instance n=20 426.alb & 1 & 0 & Optimal &  0.24 & 5 &  5.00 &  0.00\\
instance n=20 427.alb & 1 & 0 & Optimal &  0.20 & 6 &  6.00 &  0.00\\
instance n=20 428.alb & 1 & 0 & Optimal &  0.22 & 5 &  5.00 &  0.00\\
instance n=20 429.alb & 1 & 0 & Optimal &  0.22 & 4 &  4.00 &  0.00\\
instance n=20 43.alb & 1 & 0 & Optimal &  0.03 & 5 &  5.00 &  0.00\\
instance n=20 430.alb & 1 & 0 & Optimal &  0.28 & 5 &  5.00 &  0.00\\
instance n=20 431.alb & 1 & 0 & Optimal &  0.28 & 6 &  6.00 &  0.00\\
instance n=20 432.alb & 1 & 0 & Optimal &  0.17 & 5 &  5.00 &  0.00\\
instance n=20 433.alb & 1 & 0 & Optimal &  0.22 & 5 &  5.00 &  0.00\\
instance n=20 434.alb & 1 & 0 & Optimal &  0.21 & 5 &  5.00 &  0.00\\
instance n=20 435.alb & 1 & 0 & Optimal &  1.18 & 7 &  7.00 &  0.00\\
instance n=20 436.alb & 1 & 0 & Optimal &  0.20 & 5 &  5.00 &  0.00\\
instance n=20 437.alb & 1 & 0 & Optimal &  0.27 & 5 &  5.00 &  0.00\\
instance n=20 438.alb & 1 & 0 & Optimal &  0.22 & 6 &  6.00 &  0.00\\
instance n=20 439.alb & 1 & 0 & Optimal &  0.20 & 5 &  5.00 &  0.00\\
instance n=20 44.alb & 1 & 0 & Optimal &  0.04 & 5 &  5.00 &  0.00\\
instance n=20 440.alb & 1 & 0 & Optimal &  0.22 & 5 &  5.00 &  0.00\\
instance n=20 441.alb & 1 & 0 & Optimal &  0.17 & 3 &  3.00 &  0.00\\
instance n=20 442.alb & 1 & 0 & Optimal &  0.23 & 3 &  3.00 &  0.00\\
instance n=20 443.alb & 1 & 0 & Optimal &  0.20 & 3 &  3.00 &  0.00\\
instance n=20 444.alb & 1 & 0 & Optimal &  0.24 & 3 &  3.00 &  0.00\\
instance n=20 445.alb & 1 & 0 & Optimal &  0.18 & 3 &  3.00 &  0.00\\
instance n=20 446.alb & 1 & 0 & Optimal &  0.17 & 3 &  3.00 &  0.00\\
instance n=20 447.alb & 1 & 0 & Optimal &  0.17 & 3 &  3.00 &  0.00\\
instance n=20 448.alb & 1 & 0 & Optimal &  0.17 & 3 &  3.00 &  0.00\\
instance n=20 449.alb & 1 & 0 & Optimal &  0.25 & 3 &  3.00 &  0.00\\
instance n=20 45.alb & 1 & 0 & Optimal &  0.03 & 6 &  6.00 &  0.00\\
instance n=20 450.alb & 1 & 0 & Optimal &  0.24 & 3 &  3.00 &  0.00\\
instance n=20 451.alb & 1 & 0 & Optimal &  0.22 & 3 &  3.00 &  0.00\\
instance n=20 452.alb & 1 & 0 & Optimal &  0.20 & 3 &  3.00 &  0.00\\
instance n=20 453.alb & 1 & 0 & Optimal &  0.23 & 3 &  3.00 &  0.00\\
instance n=20 454.alb & 1 & 0 & Optimal &  0.20 & 3 &  3.00 &  0.00\\
instance n=20 455.alb & 1 & 0 & Optimal &  0.17 & 3 &  3.00 &  0.00\\
instance n=20 456.alb & 1 & 0 & Optimal &  0.17 & 4 &  4.00 &  0.00\\
instance n=20 457.alb & 1 & 0 & Optimal &  0.22 & 3 &  3.00 &  0.00\\
instance n=20 458.alb & 1 & 0 & Optimal &  0.17 & 3 &  3.00 &  0.00\\
instance n=20 459.alb & 1 & 0 & Optimal &  0.19 & 3 &  3.00 &  0.00\\
instance n=20 46.alb & 1 & 0 & Optimal &  0.02 & 4 &  4.00 &  0.00\\
instance n=20 460.alb & 1 & 0 & Optimal &  0.25 & 3 &  3.00 &  0.00\\
instance n=20 461.alb & 1 & 0 & Optimal &  0.22 & 3 &  3.00 &  0.00\\
instance n=20 462.alb & 1 & 0 & Optimal &  0.22 & 3 &  3.00 &  0.00\\
instance n=20 463.alb & 1 & 0 & Optimal &  0.19 & 3 &  3.00 &  0.00\\
instance n=20 464.alb & 1 & 0 & Optimal &  0.19 & 3 &  3.00 &  0.00\\
instance n=20 465.alb & 1 & 0 & Optimal &  0.16 & 3 &  3.00 &  0.00\\
instance n=20 466.alb & 1 & 0 & Optimal &  1.11 & 13 & 13.00 &  0.00\\
instance n=20 467.alb & 1 & 0 & Optimal &  1.08 & 14 & 14.00 &  0.00\\
instance n=20 468.alb & 1 & 0 & Optimal &  1.15 & 13 & 13.00 &  0.00\\
instance n=20 469.alb & 1 & 0 & Optimal &  0.86 & 14 & 14.00 &  0.00\\
instance n=20 47.alb & 1 & 0 & Optimal &  0.04 & 4 &  4.00 &  0.00\\
instance n=20 470.alb & 1 & 0 & Optimal &  1.19 & 12 & 12.00 &  0.00\\
instance n=20 471.alb & 1 & 0 & Optimal &  1.01 & 12 & 12.00 &  0.00\\
instance n=20 472.alb & 1 & 0 & Optimal &  0.97 & 13 & 13.00 &  0.00\\
instance n=20 473.alb & 1 & 0 & Optimal &  1.08 & 10 & 10.00 &  0.00\\
instance n=20 474.alb & 1 & 0 & Optimal &  1.07 & 14 & 14.00 &  0.00\\
instance n=20 475.alb & 1 & 0 & Optimal &  1.13 & 11 & 11.00 &  0.00\\
instance n=20 476.alb & 1 & 0 & Optimal &  1.00 & 11 & 11.00 &  0.00\\
instance n=20 477.alb & 1 & 0 & Optimal &  1.13 & 11 & 11.00 &  0.00\\
instance n=20 478.alb & 1 & 0 & Optimal &  1.29 & 12 & 12.00 &  0.00\\
instance n=20 479.alb & 1 & 0 & Optimal &  1.15 & 13 & 13.00 &  0.00\\
instance n=20 48.alb & 1 & 0 & Optimal &  0.05 & 5 &  5.00 &  0.00\\
instance n=20 480.alb & 1 & 0 & Optimal &  1.15 & 13 & 13.00 &  0.00\\
instance n=20 481.alb & 1 & 0 & Optimal &  0.94 & 13 & 13.00 &  0.00\\
instance n=20 482.alb & 1 & 0 & Optimal &  1.30 & 13 & 13.00 &  0.00\\
instance n=20 483.alb & 1 & 0 & Optimal &  1.07 & 12 & 12.00 &  0.00\\
instance n=20 484.alb & 1 & 0 & Optimal &  1.15 & 13 & 13.00 &  0.00\\
instance n=20 485.alb & 1 & 0 & Optimal &  0.99 & 15 & 15.00 &  0.00\\
instance n=20 486.alb & 1 & 0 & Optimal &  1.14 & 11 & 11.00 &  0.00\\
instance n=20 487.alb & 1 & 0 & Optimal &  1.23 & 12 & 12.00 &  0.00\\
instance n=20 488.alb & 1 & 0 & Optimal &  1.10 & 15 & 15.00 &  0.00\\
instance n=20 489.alb & 1 & 0 & Optimal &  1.04 & 12 & 12.00 &  0.00\\
instance n=20 49.alb & 1 & 0 & Optimal &  0.03 & 4 &  4.00 &  0.00\\
instance n=20 490.alb & 1 & 0 & Optimal &  1.22 & 12 & 12.00 &  0.00\\
instance n=20 491.alb & 1 & 0 & Optimal &  0.23 & 6 &  6.00 &  0.00\\
instance n=20 492.alb & 1 & 0 & Optimal &  0.24 & 5 &  5.00 &  0.00\\
instance n=20 493.alb & 1 & 0 & Optimal &  0.27 & 5 &  5.00 &  0.00\\
instance n=20 494.alb & 1 & 0 & Optimal &  0.22 & 6 &  6.00 &  0.00\\
instance n=20 495.alb & 1 & 0 & Optimal &  0.27 & 6 &  6.00 &  0.00\\
instance n=20 496.alb & 1 & 0 & Optimal &  0.27 & 5 &  5.00 &  0.00\\
instance n=20 497.alb & 1 & 0 & Optimal &  0.25 & 6 &  6.00 &  0.00\\
instance n=20 498.alb & 1 & 0 & Optimal &  0.39 & 6 &  6.00 &  0.00\\
instance n=20 499.alb & 1 & 0 & Optimal &  0.25 & 5 &  5.00 &  0.00\\
instance n=20 5.alb & 1 & 0 & Optimal &  0.03 & 3 &  3.00 &  0.00\\
instance n=20 50.alb & 1 & 0 & Optimal &  0.05 & 4 &  4.00 &  0.00\\
instance n=20 500.alb & 1 & 0 & Optimal &  1.33 & 8 &  8.00 &  0.00\\
instance n=20 501.alb & 1 & 0 & Optimal &  0.28 & 5 &  5.00 &  0.00\\
instance n=20 502.alb & 1 & 0 & Optimal &  0.19 & 4 &  4.00 &  0.00\\
instance n=20 503.alb & 1 & 0 & Optimal &  0.24 & 6 &  6.00 &  0.00\\
instance n=20 504.alb & 1 & 0 & Optimal &  0.28 & 6 &  6.00 &  0.00\\
instance n=20 505.alb & 1 & 0 & Optimal &  0.36 & 6 &  6.00 &  0.00\\
instance n=20 506.alb & 1 & 0 & Optimal &  0.31 & 5 &  5.00 &  0.00\\
instance n=20 507.alb & 1 & 0 & Optimal &  0.24 & 5 &  5.00 &  0.00\\
instance n=20 508.alb & 1 & 0 & Optimal &  0.28 & 5 &  5.00 &  0.00\\
instance n=20 509.alb & 1 & 0 & Optimal &  0.20 & 4 &  4.00 &  0.00\\
instance n=20 51.alb & 1 & 0 & Optimal &  0.03 & 4 &  4.00 &  0.00\\
instance n=20 510.alb & 1 & 0 & Optimal &  0.25 & 5 &  5.00 &  0.00\\
instance n=20 511.alb & 1 & 0 & Optimal &  0.44 & 5 &  5.00 &  0.00\\
instance n=20 512.alb & 1 & 0 & Optimal &  0.31 & 5 &  5.00 &  0.00\\
instance n=20 513.alb & 1 & 0 & Optimal &  0.27 & 5 &  5.00 &  0.00\\
instance n=20 514.alb & 1 & 0 & Optimal &  0.22 & 5 &  5.00 &  0.00\\
instance n=20 515.alb & 1 & 0 & Optimal &  1.67 & 6 &  6.00 &  0.00\\
instance n=20 516.alb & 1 & 0 & Optimal &  0.36 & 3 &  3.00 &  0.00\\
instance n=20 517.alb & 1 & 0 & Optimal &  0.31 & 3 &  3.00 &  0.00\\
instance n=20 518.alb & 1 & 0 & Optimal &  0.35 & 3 &  3.00 &  0.00\\
instance n=20 519.alb & 1 & 0 & Optimal &  0.35 & 3 &  3.00 &  0.00\\
instance n=20 52.alb & 1 & 0 & Optimal &  0.05 & 4 &  4.00 &  0.00\\
instance n=20 520.alb & 1 & 0 & Optimal &  0.39 & 3 &  3.00 &  0.00\\
instance n=20 521.alb & 1 & 0 & Optimal &  0.36 & 3 &  3.00 &  0.00\\
instance n=20 522.alb & 1 & 0 & Optimal &  0.30 & 3 &  3.00 &  0.00\\
instance n=20 523.alb & 1 & 0 & Optimal &  0.28 & 3 &  3.00 &  0.00\\
instance n=20 524.alb & 1 & 0 & Optimal &  0.33 & 3 &  3.00 &  0.00\\
instance n=20 525.alb & 1 & 0 & Optimal &  0.35 & 3 &  3.00 &  0.00\\
instance n=20 53.alb & 1 & 0 & Optimal &  0.03 & 5 &  5.00 &  0.00\\
instance n=20 54.alb & 1 & 0 & Optimal &  0.03 & 5 &  5.00 &  0.00\\
instance n=20 55.alb & 1 & 0 & Optimal &  0.04 & 5 &  5.00 &  0.00\\
instance n=20 56.alb & 1 & 0 & Optimal &  0.03 & 4 &  4.00 &  0.00\\
instance n=20 57.alb & 1 & 0 & Optimal &  0.03 & 4 &  4.00 &  0.00\\
instance n=20 58.alb & 1 & 0 & Optimal &  0.03 & 5 &  5.00 &  0.00\\
instance n=20 59.alb & 1 & 0 & Optimal &  0.04 & 4 &  4.00 &  0.00\\
instance n=20 6.alb & 1 & 0 & Optimal &  0.02 & 3 &  3.00 &  0.00\\
instance n=20 60.alb & 1 & 0 & Optimal &  0.05 & 6 &  6.00 &  0.00\\
instance n=20 61.alb & 1 & 0 & Optimal &  0.04 & 7 &  7.00 &  0.00\\
instance n=20 62.alb & 1 & 0 & Optimal &  0.03 & 5 &  5.00 &  0.00\\
instance n=20 63.alb & 1 & 0 & Optimal &  0.05 & 5 &  5.00 &  0.00\\
instance n=20 64.alb & 1 & 0 & Optimal &  0.04 & 5 &  5.00 &  0.00\\
instance n=20 65.alb & 1 & 0 & Optimal &  0.04 & 5 &  5.00 &  0.00\\
instance n=20 66.alb & 1 & 0 & Optimal &  0.02 & 3 &  3.00 &  0.00\\
instance n=20 67.alb & 1 & 0 & Optimal &  0.03 & 3 &  3.00 &  0.00\\
instance n=20 68.alb & 1 & 0 & Optimal &  0.05 & 3 &  3.00 &  0.00\\
instance n=20 69.alb & 1 & 0 & Optimal &  0.01 & 2 &  2.00 &  0.00\\
instance n=20 7.alb & 1 & 0 & Optimal &  0.02 & 3 &  3.00 &  0.00\\
instance n=20 70.alb & 1 & 0 & Optimal &  0.06 & 3 &  3.00 &  0.00\\
instance n=20 71.alb & 1 & 0 & Optimal &  0.03 & 3 &  3.00 &  0.00\\
instance n=20 72.alb & 1 & 0 & Optimal &  0.03 & 3 &  3.00 &  0.00\\
instance n=20 73.alb & 1 & 0 & Optimal &  0.01 & 2 &  2.00 &  0.00\\
instance n=20 74.alb & 1 & 0 & Optimal &  0.05 & 3 &  3.00 &  0.00\\
instance n=20 75.alb & 1 & 0 & Optimal &  0.03 & 3 &  3.00 &  0.00\\
instance n=20 76.alb & 1 & 0 & Optimal &  0.05 & 3 &  3.00 &  0.00\\
instance n=20 77.alb & 1 & 0 & Optimal &  0.05 & 3 &  3.00 &  0.00\\
instance n=20 78.alb & 1 & 0 & Optimal &  0.03 & 3 &  3.00 &  0.00\\
instance n=20 79.alb & 1 & 0 & Optimal &  0.03 & 3 &  3.00 &  0.00\\
instance n=20 8.alb & 1 & 0 & Optimal &  0.03 & 3 &  3.00 &  0.00\\
instance n=20 80.alb & 1 & 0 & Optimal &  0.05 & 3 &  3.00 &  0.00\\
instance n=20 81.alb & 1 & 0 & Optimal &  0.05 & 3 &  3.00 &  0.00\\
instance n=20 82.alb & 1 & 0 & Optimal &  0.05 & 4 &  4.00 &  0.00\\
instance n=20 83.alb & 1 & 0 & Optimal &  0.03 & 3 &  3.00 &  0.00\\
instance n=20 84.alb & 1 & 0 & Optimal &  0.05 & 3 &  3.00 &  0.00\\
instance n=20 85.alb & 1 & 0 & Optimal &  0.05 & 3 &  3.00 &  0.00\\
instance n=20 86.alb & 1 & 0 & Optimal &  0.04 & 3 &  3.00 &  0.00\\
instance n=20 87.alb & 1 & 0 & Optimal &  0.04 & 3 &  3.00 &  0.00\\
instance n=20 88.alb & 1 & 0 & Optimal &  0.05 & 3 &  3.00 &  0.00\\
instance n=20 89.alb & 1 & 0 & Optimal &  0.06 & 3 &  3.00 &  0.00\\
instance n=20 9.alb & 1 & 0 & Optimal &  0.02 & 3 &  3.00 &  0.00\\
instance n=20 90.alb & 1 & 0 & Optimal &  0.05 & 3 &  3.00 &  0.00\\
instance n=20 91.alb & 1 & 0 & Optimal &  0.20 & 11 & 11.00 &  0.00\\
instance n=20 92.alb & 1 & 0 & Optimal &  0.26 & 11 & 11.00 &  0.00\\
instance n=20 93.alb & 1 & 0 & Optimal &  0.36 & 13 & 13.00 &  0.00\\
instance n=20 94.alb & 1 & 0 & Optimal &  0.06 & 10 & 10.00 &  0.00\\
instance n=20 95.alb & 1 & 0 & Optimal &  0.23 & 12 & 12.00 &  0.00\\
instance n=20 96.alb & 1 & 0 & Optimal &  0.22 & 10 & 10.00 &  0.00\\
instance n=20 97.alb & 1 & 0 & Optimal &  2.52 & 15 & 15.00 &  0.00\\
instance n=20 98.alb & 1 & 0 & Optimal &  0.54 & 13 & 13.00 &  0.00\\
instance n=20 99.alb & 1 & 0 & Optimal &  0.59 & 12 & 12.00 &  0.00\\
instance n=50 1.alb & 1 & 0 & Optimal &  0.02 & 8 &  8.00 &  0.00\\
instance n=50 10.alb & 1 & 0 & Optimal &  0.03 & 7 &  7.00 &  0.00\\
instance n=50 100.alb & 1 & 0 & Optimal &  0.07 & 7 &  7.00 &  0.00\\
instance n=50 101.alb & 1 & 0 & Solution & 120.02 & 30 & 27.00 & 10.00\\
instance n=50 102.alb & 1 & 0 & Solution & 120.02 & 32 & 28.00 & 12.50\\
instance n=50 103.alb & 1 & 0 & Solution & 120.04 & 29 & 26.00 & 10.34\\
instance n=50 104.alb & 1 & 0 & Solution & 120.03 & 27 & 25.00 &  7.41\\
instance n=50 105.alb & 1 & 0 & Optimal & 93.42 & 24 & 24.00 &  0.00\\
instance n=50 106.alb & 1 & 0 & Solution & 120.04 & 28 & 26.00 &  7.14\\
instance n=50 107.alb & 1 & 0 & Solution & 120.02 & 28 & 27.00 &  3.57\\
instance n=50 108.alb & 1 & 0 & Solution & 120.02 & 30 & 27.00 & 10.00\\
instance n=50 109.alb & 1 & 0 & Solution & 120.04 & 30 & 26.00 & 13.33\\
instance n=50 11.alb & 1 & 0 & Optimal &  0.02 & 7 &  7.00 &  0.00\\
instance n=50 110.alb & 1 & 0 & Solution & 120.02 & 26 & 25.00 &  3.85\\
instance n=50 111.alb & 1 & 0 & Solution & 120.01 & 28 & 26.00 &  7.14\\
instance n=50 112.alb & 1 & 0 & Solution & 120.03 & 27 & 25.00 &  7.41\\
instance n=50 113.alb & 1 & 0 & Solution & 120.03 & 28 & 26.00 &  7.14\\
instance n=50 114.alb & 1 & 0 & Solution & 120.03 & 27 & 25.00 &  7.41\\
instance n=50 115.alb & 1 & 0 & Solution & 120.03 & 28 & 26.00 &  7.14\\
instance n=50 116.alb & 1 & 0 & Solution & 120.03 & 32 & 29.00 &  9.38\\
instance n=50 117.alb & 1 & 0 & Solution & 120.01 & 27 & 25.00 &  7.41\\
instance n=50 118.alb & 1 & 0 & Solution & 120.03 & 29 & 26.00 & 10.34\\
instance n=50 119.alb & 1 & 0 & Optimal &  5.79 & 25 & 25.00 &  0.00\\
instance n=50 12.alb & 1 & 0 & Optimal &  0.05 & 6 &  6.00 &  0.00\\
instance n=50 120.alb & 1 & 0 & Solution & 120.01 & 27 & 26.00 &  3.70\\
instance n=50 121.alb & 1 & 0 & Solution & 120.02 & 32 & 27.00 & 15.63\\
instance n=50 122.alb & 1 & 0 & Solution & 120.02 & 29 & 28.00 &  3.45\\
instance n=50 123.alb & 1 & 0 & Solution & 120.02 & 32 & 27.00 & 15.63\\
instance n=50 124.alb & 1 & 0 & Solution & 120.02 & 29 & 27.00 &  6.90\\
instance n=50 125.alb & 1 & 0 & Solution & 120.02 & 33 & 28.00 & 15.15\\
instance n=50 126.alb & 1 & 0 & Optimal &  0.10 & 12 & 12.00 &  0.00\\
instance n=50 127.alb & 1 & 0 & Optimal &  0.09 & 14 & 14.00 &  0.00\\
instance n=50 128.alb & 1 & 0 & Optimal &  0.42 & 12 & 12.00 &  0.00\\
instance n=50 129.alb & 1 & 0 & Optimal &  0.11 & 13 & 13.00 &  0.00\\
instance n=50 13.alb & 1 & 0 & Optimal &  0.02 & 6 &  6.00 &  0.00\\
instance n=50 130.alb & 1 & 0 & Optimal &  0.11 & 13 & 13.00 &  0.00\\
instance n=50 131.alb & 1 & 0 & Optimal &  0.11 & 12 & 12.00 &  0.00\\
instance n=50 132.alb & 1 & 0 & Optimal &  1.44 & 12 & 12.00 &  0.00\\
instance n=50 133.alb & 1 & 0 & Optimal &  0.07 & 12 & 12.00 &  0.00\\
instance n=50 134.alb & 1 & 0 & Optimal &  1.10 & 14 & 14.00 &  0.00\\
instance n=50 135.alb & 1 & 0 & Optimal &  0.47 & 13 & 13.00 &  0.00\\
instance n=50 136.alb & 1 & 0 & Optimal &  0.11 & 11 & 11.00 &  0.00\\
instance n=50 137.alb & 1 & 0 & Optimal &  0.11 & 11 & 11.00 &  0.00\\
instance n=50 138.alb & 1 & 0 & Optimal &  0.10 & 12 & 12.00 &  0.00\\
instance n=50 139.alb & 1 & 0 & Optimal &  3.67 & 11 & 11.00 &  0.00\\
instance n=50 14.alb & 1 & 0 & Optimal &  0.03 & 7 &  7.00 &  0.00\\
instance n=50 140.alb & 1 & 0 & Optimal &  0.20 & 12 & 12.00 &  0.00\\
instance n=50 141.alb & 1 & 0 & Optimal &  0.17 & 13 & 13.00 &  0.00\\
instance n=50 142.alb & 1 & 0 & Optimal &  0.11 & 11 & 11.00 &  0.00\\
instance n=50 143.alb & 1 & 0 & Optimal &  0.28 & 12 & 12.00 &  0.00\\
instance n=50 144.alb & 1 & 0 & Optimal &  0.23 & 13 & 13.00 &  0.00\\
instance n=50 145.alb & 1 & 0 & Optimal &  0.25 & 10 & 10.00 &  0.00\\
instance n=50 146.alb & 1 & 0 & Optimal &  0.20 & 13 & 13.00 &  0.00\\
instance n=50 147.alb & 1 & 0 & Optimal &  0.29 & 13 & 13.00 &  0.00\\
instance n=50 148.alb & 1 & 0 & Optimal &  0.17 & 10 & 10.00 &  0.00\\
instance n=50 149.alb & 1 & 0 & Optimal &  0.18 & 12 & 12.00 &  0.00\\
instance n=50 15.alb & 1 & 0 & Optimal &  0.02 & 8 &  8.00 &  0.00\\
instance n=50 150.alb & 1 & 0 & Optimal &  0.14 & 11 & 11.00 &  0.00\\
instance n=50 151.alb & 1 & 0 & Optimal &  0.14 & 7 &  7.00 &  0.00\\
instance n=50 152.alb & 1 & 0 & Optimal &  0.10 & 7 &  7.00 &  0.00\\
instance n=50 153.alb & 1 & 0 & Optimal &  0.56 & 7 &  7.00 &  0.00\\
instance n=50 154.alb & 1 & 0 & Optimal &  0.12 & 8 &  8.00 &  0.00\\
instance n=50 155.alb & 1 & 0 & Optimal &  0.09 & 7 &  7.00 &  0.00\\
instance n=50 156.alb & 1 & 0 & Optimal &  0.09 & 7 &  7.00 &  0.00\\
instance n=50 157.alb & 1 & 0 & Optimal &  0.10 & 8 &  8.00 &  0.00\\
instance n=50 158.alb & 1 & 0 & Optimal &  0.09 & 7 &  7.00 &  0.00\\
instance n=50 159.alb & 1 & 0 & Optimal &  0.12 & 7 &  7.00 &  0.00\\
instance n=50 16.alb & 1 & 0 & Optimal &  0.04 & 8 &  8.00 &  0.00\\
instance n=50 160.alb & 1 & 0 & Optimal &  0.11 & 8 &  8.00 &  0.00\\
instance n=50 161.alb & 1 & 0 & Optimal &  0.13 & 7 &  7.00 &  0.00\\
instance n=50 162.alb & 1 & 0 & Optimal &  0.11 & 8 &  8.00 &  0.00\\
instance n=50 163.alb & 1 & 0 & Optimal &  0.11 & 7 &  7.00 &  0.00\\
instance n=50 164.alb & 1 & 0 & Optimal &  0.12 & 7 &  7.00 &  0.00\\
instance n=50 165.alb & 1 & 0 & Optimal &  0.13 & 8 &  8.00 &  0.00\\
instance n=50 166.alb & 1 & 0 & Optimal &  0.12 & 8 &  8.00 &  0.00\\
instance n=50 167.alb & 1 & 0 & Optimal &  0.60 & 7 &  7.00 &  0.00\\
instance n=50 168.alb & 1 & 0 & Optimal &  0.69 & 8 &  8.00 &  0.00\\
instance n=50 169.alb & 1 & 0 & Optimal &  0.10 & 8 &  8.00 &  0.00\\
instance n=50 17.alb & 1 & 0 & Optimal &  0.03 & 7 &  7.00 &  0.00\\
instance n=50 170.alb & 1 & 0 & Optimal &  0.38 & 7 &  7.00 &  0.00\\
instance n=50 171.alb & 1 & 0 & Optimal &  0.12 & 8 &  8.00 &  0.00\\
instance n=50 172.alb & 1 & 0 & Optimal &  0.12 & 7 &  7.00 &  0.00\\
instance n=50 173.alb & 1 & 0 & Optimal &  0.36 & 7 &  7.00 &  0.00\\
instance n=50 174.alb & 1 & 0 & Optimal &  0.13 & 7 &  7.00 &  0.00\\
instance n=50 175.alb & 1 & 0 & Optimal &  0.09 & 7 &  7.00 &  0.00\\
instance n=50 176.alb & 1 & 0 & Solution & 120.03 & 27 & 25.00 &  7.41\\
instance n=50 177.alb & 1 & 0 & Solution & 120.03 & 28 & 26.00 &  7.14\\
instance n=50 178.alb & 1 & 0 & Solution & 120.05 & 28 & 26.00 &  7.14\\
instance n=50 179.alb & 1 & 0 & Solution & 120.03 & 27 & 25.00 &  7.41\\
instance n=50 18.alb & 1 & 0 & Optimal &  0.04 & 7 &  7.00 &  0.00\\
instance n=50 180.alb & 1 & 0 & Solution & 120.02 & 26 & 25.00 &  3.85\\
instance n=50 181.alb & 1 & 0 & Solution & 120.04 & 29 & 27.00 &  6.90\\
instance n=50 182.alb & 1 & 0 & Solution & 120.01 & 27 & 25.00 &  7.41\\
instance n=50 183.alb & 1 & 0 & Solution & 120.03 & 29 & 26.00 & 10.34\\
instance n=50 184.alb & 1 & 0 & Solution & 120.02 & 38 & 29.00 & 23.68\\
instance n=50 185.alb & 1 & 0 & Solution & 120.03 & 27 & 25.00 &  7.41\\
instance n=50 186.alb & 1 & 0 & Solution & 120.03 & 26 & 25.00 &  3.85\\
instance n=50 187.alb & 1 & 0 & Solution & 120.03 & 26 & 25.00 &  3.85\\
instance n=50 188.alb & 1 & 0 & Solution & 120.05 & 25 & 24.00 &  4.00\\
instance n=50 189.alb & 1 & 0 & Solution & 120.03 & 26 & 25.00 &  3.85\\
instance n=50 19.alb & 1 & 0 & Optimal &  0.03 & 8 &  8.00 &  0.00\\
instance n=50 190.alb & 1 & 0 & Solution & 120.03 & 30 & 26.00 & 13.33\\
instance n=50 191.alb & 1 & 0 & Solution & 120.02 & 28 & 26.00 &  7.14\\
instance n=50 192.alb & 1 & 0 & Solution & 120.02 & 27 & 26.00 &  3.70\\
instance n=50 193.alb & 1 & 0 & Solution & 120.02 & 28 & 27.00 &  3.57\\
instance n=50 194.alb & 1 & 0 & Solution & 120.03 & 28 & 26.00 &  7.14\\
instance n=50 195.alb & 1 & 0 & Solution & 120.05 & 28 & 26.00 &  7.14\\
instance n=50 196.alb & 1 & 0 & Solution & 120.03 & 27 & 26.00 &  3.70\\
instance n=50 197.alb & 1 & 0 & Solution & 120.03 & 28 & 27.00 &  3.57\\
instance n=50 198.alb & 1 & 0 & Solution & 120.03 & 28 & 26.00 &  7.14\\
instance n=50 199.alb & 1 & 0 & Solution & 120.04 & 29 & 27.00 &  6.90\\
instance n=50 2.alb & 1 & 0 & Optimal &  0.02 & 6 &  6.00 &  0.00\\
instance n=50 20.alb & 1 & 0 & Optimal &  0.03 & 8 &  8.00 &  0.00\\
instance n=50 200.alb & 1 & 0 & Solution & 120.03 & 25 & 24.00 &  4.00\\
instance n=50 201.alb & 1 & 0 & Optimal &  0.20 & 13 & 13.00 &  0.00\\
instance n=50 202.alb & 1 & 0 & Optimal &  0.47 & 9 &  9.00 &  0.00\\
instance n=50 203.alb & 1 & 0 & Optimal &  0.37 & 11 & 11.00 &  0.00\\
instance n=50 204.alb & 1 & 0 & Optimal &  1.01 & 10 & 10.00 &  0.00\\
instance n=50 205.alb & 1 & 0 & Optimal &  0.20 & 13 & 13.00 &  0.00\\
instance n=50 206.alb & 1 & 0 & Optimal & 13.25 & 11 & 11.00 &  0.00\\
instance n=50 207.alb & 1 & 0 & Optimal &  0.13 & 10 & 10.00 &  0.00\\
instance n=50 208.alb & 1 & 0 & Optimal &  0.32 & 13 & 13.00 &  0.00\\
instance n=50 209.alb & 1 & 0 & Optimal &  0.22 & 11 & 11.00 &  0.00\\
instance n=50 21.alb & 1 & 0 & Optimal &  0.03 & 6 &  6.00 &  0.00\\
instance n=50 210.alb & 1 & 0 & Optimal &  0.25 & 13 & 13.00 &  0.00\\
instance n=50 211.alb & 1 & 0 & Optimal &  0.14 & 12 & 12.00 &  0.00\\
instance n=50 212.alb & 1 & 0 & Optimal &  0.18 & 10 & 10.00 &  0.00\\
instance n=50 213.alb & 1 & 0 & Optimal &  0.15 & 13 & 13.00 &  0.00\\
instance n=50 214.alb & 1 & 0 & Optimal &  0.14 & 11 & 11.00 &  0.00\\
instance n=50 215.alb & 1 & 0 & Optimal &  0.23 & 11 & 11.00 &  0.00\\
instance n=50 216.alb & 1 & 0 & Optimal &  0.41 & 12 & 12.00 &  0.00\\
instance n=50 217.alb & 1 & 0 & Optimal &  1.16 & 13 & 13.00 &  0.00\\
instance n=50 218.alb & 1 & 0 & Optimal &  0.12 & 12 & 12.00 &  0.00\\
instance n=50 219.alb & 1 & 0 & Optimal &  0.20 & 11 & 11.00 &  0.00\\
instance n=50 22.alb & 1 & 0 & Optimal &  0.03 & 7 &  7.00 &  0.00\\
instance n=50 220.alb & 1 & 0 & Optimal &  0.14 & 11 & 11.00 &  0.00\\
instance n=50 221.alb & 1 & 0 & Optimal &  1.02 & 11 & 11.00 &  0.00\\
instance n=50 222.alb & 1 & 0 & Optimal &  0.16 & 14 & 14.00 &  0.00\\
instance n=50 223.alb & 1 & 0 & Optimal &  1.67 & 11 & 11.00 &  0.00\\
instance n=50 224.alb & 1 & 0 & Optimal &  0.12 & 11 & 11.00 &  0.00\\
instance n=50 225.alb & 1 & 0 & Optimal &  0.20 & 12 & 12.00 &  0.00\\
instance n=50 226.alb & 1 & 0 & Optimal &  0.14 & 7 &  7.00 &  0.00\\
instance n=50 227.alb & 1 & 0 & Optimal &  0.22 & 6 &  6.00 &  0.00\\
instance n=50 228.alb & 1 & 0 & Optimal &  0.21 & 6 &  6.00 &  0.00\\
instance n=50 229.alb & 1 & 0 & Optimal &  0.16 & 6 &  6.00 &  0.00\\
instance n=50 23.alb & 1 & 0 & Optimal &  0.03 & 7 &  7.00 &  0.00\\
instance n=50 230.alb & 1 & 0 & Optimal &  0.24 & 7 &  7.00 &  0.00\\
instance n=50 231.alb & 1 & 0 & Optimal &  0.28 & 7 &  7.00 &  0.00\\
instance n=50 232.alb & 1 & 0 & Optimal &  0.98 & 7 &  7.00 &  0.00\\
instance n=50 233.alb & 1 & 0 & Optimal &  0.20 & 6 &  6.00 &  0.00\\
instance n=50 234.alb & 1 & 0 & Optimal &  0.22 & 8 &  8.00 &  0.00\\
instance n=50 235.alb & 1 & 0 & Optimal &  0.26 & 7 &  7.00 &  0.00\\
instance n=50 236.alb & 1 & 0 & Optimal &  0.48 & 7 &  7.00 &  0.00\\
instance n=50 237.alb & 1 & 0 & Optimal &  0.20 & 8 &  8.00 &  0.00\\
instance n=50 238.alb & 1 & 0 & Optimal &  0.26 & 7 &  7.00 &  0.00\\
instance n=50 239.alb & 1 & 0 & Optimal &  0.36 & 7 &  7.00 &  0.00\\
instance n=50 24.alb & 1 & 0 & Optimal &  0.03 & 7 &  7.00 &  0.00\\
instance n=50 240.alb & 1 & 0 & Optimal &  0.14 & 7 &  7.00 &  0.00\\
instance n=50 241.alb & 1 & 0 & Optimal &  0.19 & 7 &  7.00 &  0.00\\
instance n=50 242.alb & 1 & 0 & Optimal &  0.24 & 8 &  8.00 &  0.00\\
instance n=50 243.alb & 1 & 0 & Optimal &  0.17 & 7 &  7.00 &  0.00\\
instance n=50 244.alb & 1 & 0 & Optimal &  0.55 & 7 &  7.00 &  0.00\\
instance n=50 245.alb & 1 & 0 & Optimal &  0.30 & 7 &  7.00 &  0.00\\
instance n=50 246.alb & 1 & 0 & Optimal &  0.21 & 8 &  8.00 &  0.00\\
instance n=50 247.alb & 1 & 0 & Optimal &  0.24 & 7 &  7.00 &  0.00\\
instance n=50 248.alb & 1 & 0 & Optimal &  0.13 & 7 &  7.00 &  0.00\\
instance n=50 249.alb & 1 & 0 & Optimal &  0.54 & 7 &  7.00 &  0.00\\
instance n=50 25.alb & 1 & 0 & Optimal &  0.03 & 6 &  6.00 &  0.00\\
instance n=50 250.alb & 1 & 0 & Optimal &  0.18 & 7 &  7.00 &  0.00\\
instance n=50 251.alb & 1 & 0 & Solution & 120.03 & 27 & 26.00 &  3.70\\
instance n=50 252.alb & 1 & 0 & Solution & 120.05 & 32 & 28.00 & 12.50\\
instance n=50 253.alb & 1 & 0 & Solution & 120.03 & 28 & 26.00 &  7.14\\
instance n=50 254.alb & 1 & 0 & Solution & 120.05 & 30 & 28.00 &  6.67\\
instance n=50 255.alb & 1 & 0 & Solution & 120.05 & 29 & 27.00 &  6.90\\
instance n=50 256.alb & 1 & 0 & Solution & 120.04 & 30 & 28.00 &  6.67\\
instance n=50 257.alb & 1 & 0 & Solution & 120.05 & 33 & 29.00 & 12.12\\
instance n=50 258.alb & 1 & 0 & Solution & 120.03 & 28 & 27.00 &  3.57\\
instance n=50 259.alb & 1 & 0 & Solution & 120.04 & 31 & 26.00 & 16.13\\
instance n=50 26.alb & 1 & 0 & Solution & 120.01 & 27 & 25.00 &  7.41\\
instance n=50 260.alb & 1 & 0 & Solution & 120.05 & 29 & 27.00 &  6.90\\
instance n=50 261.alb & 1 & 0 & Solution & 120.05 & 28 & 27.00 &  3.57\\
instance n=50 262.alb & 1 & 0 & Solution & 120.04 & 31 & 26.00 & 16.13\\
instance n=50 263.alb & 1 & 0 & Optimal & 118.45 & 29 & 29.00 &  0.00\\
instance n=50 264.alb & 1 & 0 & Solution & 120.04 & 27 & 26.00 &  3.70\\
instance n=50 265.alb & 1 & 0 & Solution & 120.05 & 27 & 26.00 &  3.70\\
instance n=50 266.alb & 1 & 0 & Optimal & 89.34 & 29 & 29.00 &  0.00\\
instance n=50 267.alb & 1 & 0 & Solution & 120.04 & 28 & 27.00 &  3.57\\
instance n=50 268.alb & 1 & 0 & Solution & 120.04 & 29 & 27.00 &  6.90\\
instance n=50 269.alb & 1 & 0 & Optimal & 37.05 & 26 & 26.00 &  0.00\\
instance n=50 27.alb & 1 & 0 & Solution & 120.01 & 30 & 27.00 & 10.00\\
instance n=50 270.alb & 1 & 0 & Solution & 120.04 & 28 & 27.00 &  3.57\\
instance n=50 271.alb & 1 & 0 & Solution & 120.04 & 31 & 29.00 &  6.45\\
instance n=50 272.alb & 1 & 0 & Solution & 120.03 & 27 & 26.00 &  3.70\\
instance n=50 273.alb & 1 & 0 & Solution & 120.04 & 27 & 26.00 &  3.70\\
instance n=50 274.alb & 1 & 0 & Solution & 120.04 & 29 & 27.00 &  6.90\\
instance n=50 275.alb & 1 & 0 & Optimal &  7.83 & 27 & 27.00 &  0.00\\
instance n=50 276.alb & 1 & 0 & Optimal &  0.88 & 12 & 12.00 &  0.00\\
instance n=50 277.alb & 1 & 0 & Optimal &  0.16 & 13 & 13.00 &  0.00\\
instance n=50 278.alb & 1 & 0 & Optimal &  0.63 & 12 & 12.00 &  0.00\\
instance n=50 279.alb & 1 & 0 & Optimal &  0.16 & 11 & 11.00 &  0.00\\
instance n=50 28.alb & 1 & 0 & Solution & 120.01 & 28 & 26.00 &  7.14\\
instance n=50 280.alb & 1 & 0 & Optimal &  0.19 & 13 & 13.00 &  0.00\\
instance n=50 281.alb & 1 & 0 & Optimal &  0.36 & 11 & 11.00 &  0.00\\
instance n=50 282.alb & 1 & 0 & Optimal &  4.74 & 12 & 12.00 &  0.00\\
instance n=50 283.alb & 1 & 0 & Optimal &  0.50 & 12 & 12.00 &  0.00\\
instance n=50 284.alb & 1 & 0 & Optimal &  0.20 & 11 & 11.00 &  0.00\\
instance n=50 285.alb & 1 & 0 & Optimal &  0.76 & 13 & 13.00 &  0.00\\
instance n=50 286.alb & 1 & 0 & Optimal &  0.94 & 11 & 11.00 &  0.00\\
instance n=50 287.alb & 1 & 0 & Optimal &  0.91 & 12 & 12.00 &  0.00\\
instance n=50 288.alb & 1 & 0 & Optimal &  0.49 & 10 & 10.00 &  0.00\\
instance n=50 289.alb & 1 & 0 & Optimal &  0.79 & 11 & 11.00 &  0.00\\
instance n=50 29.alb & 1 & 0 & Solution & 120.01 & 29 & 25.00 & 13.79\\
instance n=50 290.alb & 1 & 0 & Optimal &  0.52 & 14 & 14.00 &  0.00\\
instance n=50 291.alb & 1 & 0 & Optimal &  0.20 & 12 & 12.00 &  0.00\\
instance n=50 292.alb & 1 & 0 & Optimal &  0.17 & 13 & 13.00 &  0.00\\
instance n=50 293.alb & 1 & 0 & Optimal &  0.20 & 12 & 12.00 &  0.00\\
instance n=50 294.alb & 1 & 0 & Optimal &  0.20 & 13 & 13.00 &  0.00\\
instance n=50 295.alb & 1 & 0 & Optimal &  1.49 & 16 & 16.00 &  0.00\\
instance n=50 296.alb & 1 & 0 & Optimal &  0.24 & 13 & 13.00 &  0.00\\
instance n=50 297.alb & 1 & 0 & Optimal &  0.24 & 13 & 13.00 &  0.00\\
instance n=50 298.alb & 1 & 0 & Optimal &  0.55 & 11 & 11.00 &  0.00\\
instance n=50 299.alb & 1 & 0 & Optimal &  2.84 & 12 & 12.00 &  0.00\\
instance n=50 3.alb & 1 & 0 & Optimal &  0.04 & 8 &  8.00 &  0.00\\
instance n=50 30.alb & 1 & 0 & Solution & 120.01 & 27 & 25.00 &  7.41\\
instance n=50 300.alb & 1 & 0 & Optimal &  0.25 & 12 & 12.00 &  0.00\\
instance n=50 301.alb & 1 & 0 & Optimal &  0.33 & 6 &  6.00 &  0.00\\
instance n=50 302.alb & 1 & 0 & Optimal &  0.23 & 7 &  7.00 &  0.00\\
instance n=50 303.alb & 1 & 0 & Optimal &  0.20 & 8 &  8.00 &  0.00\\
instance n=50 304.alb & 1 & 0 & Optimal &  0.20 & 7 &  7.00 &  0.00\\
instance n=50 305.alb & 1 & 0 & Optimal &  0.20 & 8 &  8.00 &  0.00\\
instance n=50 306.alb & 1 & 0 & Optimal &  0.28 & 7 &  7.00 &  0.00\\
instance n=50 307.alb & 1 & 0 & Optimal &  0.27 & 7 &  7.00 &  0.00\\
instance n=50 308.alb & 1 & 0 & Optimal &  0.40 & 8 &  8.00 &  0.00\\
instance n=50 309.alb & 1 & 0 & Optimal &  0.49 & 7 &  7.00 &  0.00\\
instance n=50 31.alb & 1 & 0 & Solution & 120.01 & 28 & 25.00 & 10.71\\
instance n=50 310.alb & 1 & 0 & Optimal &  0.20 & 8 &  8.00 &  0.00\\
instance n=50 311.alb & 1 & 0 & Optimal &  0.20 & 8 &  8.00 &  0.00\\
instance n=50 312.alb & 1 & 0 & Optimal &  0.22 & 6 &  6.00 &  0.00\\
instance n=50 313.alb & 1 & 0 & Optimal &  0.20 & 8 &  8.00 &  0.00\\
instance n=50 314.alb & 1 & 0 & Optimal &  0.22 & 7 &  7.00 &  0.00\\
instance n=50 315.alb & 1 & 0 & Optimal &  0.30 & 8 &  8.00 &  0.00\\
instance n=50 316.alb & 1 & 0 & Optimal &  0.17 & 8 &  8.00 &  0.00\\
instance n=50 317.alb & 1 & 0 & Optimal &  0.17 & 6 &  6.00 &  0.00\\
instance n=50 318.alb & 1 & 0 & Optimal &  0.33 & 8 &  8.00 &  0.00\\
instance n=50 319.alb & 1 & 0 & Optimal &  0.20 & 7 &  7.00 &  0.00\\
instance n=50 32.alb & 1 & 0 & Optimal &  2.07 & 25 & 25.00 &  0.00\\
instance n=50 320.alb & 1 & 0 & Optimal &  0.22 & 8 &  8.00 &  0.00\\
instance n=50 321.alb & 1 & 0 & Optimal &  0.28 & 6 &  6.00 &  0.00\\
instance n=50 322.alb & 1 & 0 & Optimal &  0.20 & 7 &  7.00 &  0.00\\
instance n=50 323.alb & 1 & 0 & Optimal &  0.27 & 7 &  7.00 &  0.00\\
instance n=50 324.alb & 1 & 0 & Optimal &  0.30 & 7 &  7.00 &  0.00\\
instance n=50 325.alb & 1 & 0 & Optimal &  0.19 & 7 &  7.00 &  0.00\\
instance n=50 326.alb & 1 & 0 & Solution & 120.04 & 33 & 28.00 & 15.15\\
instance n=50 327.alb & 1 & 0 & Solution & 120.06 & 28 & 25.00 & 10.71\\
instance n=50 328.alb & 1 & 0 & Solution & 120.04 & 32 & 28.00 & 12.50\\
instance n=50 329.alb & 1 & 0 & Solution & 120.03 & 25 & 24.00 &  4.00\\
instance n=50 33.alb & 1 & 0 & Solution & 120.00 & 25 & 24.00 &  4.00\\
instance n=50 330.alb & 1 & 0 & Solution & 120.04 & 29 & 25.00 & 13.79\\
instance n=50 331.alb & 1 & 0 & Solution & 120.05 & 29 & 27.00 &  6.90\\
instance n=50 332.alb & 1 & 0 & Solution & 120.03 & 25 & 24.00 &  4.00\\
instance n=50 333.alb & 1 & 0 & Solution & 120.06 & 28 & 26.00 &  7.14\\
instance n=50 334.alb & 1 & 0 & Solution & 120.06 & 29 & 25.00 & 13.79\\
instance n=50 335.alb & 1 & 0 & Solution & 120.05 & 27 & 26.00 &  3.70\\
instance n=50 336.alb & 1 & 0 & Solution & 120.03 & 26 & 25.00 &  3.85\\
instance n=50 337.alb & 1 & 0 & Solution & 120.04 & 26 & 25.00 &  3.85\\
instance n=50 338.alb & 1 & 0 & Solution & 120.05 & 27 & 26.00 &  3.70\\
instance n=50 339.alb & 1 & 0 & Solution & 120.05 & 27 & 26.00 &  3.70\\
instance n=50 34.alb & 1 & 0 & Solution & 120.01 & 30 & 27.00 & 10.00\\
instance n=50 340.alb & 1 & 0 & Solution & 120.03 & 28 & 26.00 &  7.14\\
instance n=50 341.alb & 1 & 0 & Solution & 120.05 & 27 & 25.00 &  7.41\\
instance n=50 342.alb & 1 & 0 & Solution & 120.05 & 28 & 26.00 &  7.14\\
instance n=50 343.alb & 1 & 0 & Solution & 120.04 & 27 & 25.00 &  7.41\\
instance n=50 344.alb & 1 & 0 & Solution & 120.06 & 30 & 27.00 & 10.00\\
instance n=50 345.alb & 1 & 0 & Solution & 120.03 & 29 & 27.00 &  6.90\\
instance n=50 346.alb & 1 & 0 & Solution & 120.06 & 27 & 25.00 &  7.41\\
instance n=50 347.alb & 1 & 0 & Solution & 120.06 & 26 & 25.00 &  3.85\\
instance n=50 348.alb & 1 & 0 & Solution & 120.06 & 30 & 25.00 & 16.67\\
instance n=50 349.alb & 1 & 0 & Solution & 120.06 & 28 & 26.00 &  7.14\\
instance n=50 35.alb & 1 & 0 & Solution & 120.00 & 32 & 27.00 & 15.63\\
instance n=50 350.alb & 1 & 0 & Solution & 120.03 & 24 & 23.00 &  4.17\\
instance n=50 351.alb & 1 & 0 & Optimal &  0.19 & 12 & 12.00 &  0.00\\
instance n=50 352.alb & 1 & 0 & Optimal &  3.50 & 10 & 10.00 &  0.00\\
instance n=50 353.alb & 1 & 0 & Optimal &  0.42 & 13 & 13.00 &  0.00\\
instance n=50 354.alb & 1 & 0 & Solution & 120.04 & 14 & 13.00 &  7.14\\
instance n=50 355.alb & 1 & 0 & Optimal &  0.27 & 11 & 11.00 &  0.00\\
instance n=50 356.alb & 1 & 0 & Optimal &  0.20 & 15 & 15.00 &  0.00\\
instance n=50 357.alb & 1 & 0 & Optimal &  0.24 & 12 & 12.00 &  0.00\\
instance n=50 358.alb & 1 & 0 & Optimal &  0.25 & 11 & 11.00 &  0.00\\
instance n=50 359.alb & 1 & 0 & Optimal &  0.24 & 10 & 10.00 &  0.00\\
instance n=50 36.alb & 1 & 0 & Solution & 120.01 & 31 & 27.00 & 12.90\\
instance n=50 360.alb & 1 & 0 & Optimal &  0.61 & 12 & 12.00 &  0.00\\
instance n=50 361.alb & 1 & 0 & Optimal &  0.25 & 11 & 11.00 &  0.00\\
instance n=50 362.alb & 1 & 0 & Optimal &  0.32 & 10 & 10.00 &  0.00\\
instance n=50 363.alb & 1 & 0 & Solution & 120.05 & 12 & 11.00 &  8.33\\
instance n=50 364.alb & 1 & 0 & Optimal &  0.19 & 13 & 13.00 &  0.00\\
instance n=50 365.alb & 1 & 0 & Optimal &  0.28 & 11 & 11.00 &  0.00\\
instance n=50 366.alb & 1 & 0 & Optimal &  0.27 & 13 & 13.00 &  0.00\\
instance n=50 367.alb & 1 & 0 & Optimal &  0.25 & 12 & 12.00 &  0.00\\
instance n=50 368.alb & 1 & 0 & Optimal &  0.27 & 12 & 12.00 &  0.00\\
instance n=50 369.alb & 1 & 0 & Optimal &  0.58 & 12 & 12.00 &  0.00\\
instance n=50 37.alb & 1 & 0 & Solution & 120.01 & 32 & 27.00 & 15.63\\
instance n=50 370.alb & 1 & 0 & Optimal &  0.33 & 12 & 12.00 &  0.00\\
instance n=50 371.alb & 1 & 0 & Optimal &  2.56 & 11 & 11.00 &  0.00\\
instance n=50 372.alb & 1 & 0 & Optimal &  1.71 & 10 & 10.00 &  0.00\\
instance n=50 373.alb & 1 & 0 & Optimal &  0.22 & 12 & 12.00 &  0.00\\
instance n=50 374.alb & 1 & 0 & Optimal &  0.27 & 11 & 11.00 &  0.00\\
instance n=50 375.alb & 1 & 0 & Optimal &  1.08 & 13 & 13.00 &  0.00\\
instance n=50 376.alb & 1 & 0 & Optimal &  0.28 & 7 &  7.00 &  0.00\\
instance n=50 377.alb & 1 & 0 & Optimal &  0.21 & 7 &  7.00 &  0.00\\
instance n=50 378.alb & 1 & 0 & Optimal &  0.21 & 8 &  8.00 &  0.00\\
instance n=50 379.alb & 1 & 0 & Optimal &  0.25 & 7 &  7.00 &  0.00\\
instance n=50 38.alb & 1 & 0 & Solution & 120.00 & 31 & 28.00 &  9.68\\
instance n=50 380.alb & 1 & 0 & Optimal &  0.30 & 7 &  7.00 &  0.00\\
instance n=50 381.alb & 1 & 0 & Optimal &  0.33 & 8 &  8.00 &  0.00\\
instance n=50 382.alb & 1 & 0 & Optimal &  0.25 & 6 &  6.00 &  0.00\\
instance n=50 383.alb & 1 & 0 & Optimal &  0.25 & 7 &  7.00 &  0.00\\
instance n=50 384.alb & 1 & 0 & Optimal &  1.32 & 8 &  8.00 &  0.00\\
instance n=50 385.alb & 1 & 0 & Optimal &  0.22 & 7 &  7.00 &  0.00\\
instance n=50 386.alb & 1 & 0 & Optimal &  0.33 & 7 &  7.00 &  0.00\\
instance n=50 387.alb & 1 & 0 & Optimal &  0.27 & 8 &  8.00 &  0.00\\
instance n=50 388.alb & 1 & 0 & Optimal &  0.30 & 7 &  7.00 &  0.00\\
instance n=50 389.alb & 1 & 0 & Optimal &  0.24 & 8 &  8.00 &  0.00\\
instance n=50 39.alb & 1 & 0 & Solution & 120.00 & 29 & 26.00 & 10.34\\
instance n=50 390.alb & 1 & 0 & Optimal &  1.56 & 7 &  7.00 &  0.00\\
instance n=50 391.alb & 1 & 0 & Optimal &  0.30 & 7 &  7.00 &  0.00\\
instance n=50 392.alb & 1 & 0 & Optimal &  0.22 & 8 &  8.00 &  0.00\\
instance n=50 393.alb & 1 & 0 & Optimal &  0.30 & 7 &  7.00 &  0.00\\
instance n=50 394.alb & 1 & 0 & Optimal &  0.29 & 8 &  8.00 &  0.00\\
instance n=50 395.alb & 1 & 0 & Optimal &  0.27 & 7 &  7.00 &  0.00\\
instance n=50 396.alb & 1 & 0 & Optimal &  0.39 & 8 &  8.00 &  0.00\\
instance n=50 397.alb & 1 & 0 & Optimal &  0.22 & 7 &  7.00 &  0.00\\
instance n=50 398.alb & 1 & 0 & Optimal &  0.96 & 6 &  6.00 &  0.00\\
instance n=50 399.alb & 1 & 0 & Optimal &  2.02 & 7 &  7.00 &  0.00\\
instance n=50 4.alb & 1 & 0 & Optimal &  0.04 & 7 &  7.00 &  0.00\\
instance n=50 40.alb & 1 & 0 & Solution & 120.00 & 26 & 25.00 &  3.85\\
instance n=50 400.alb & 1 & 0 & Optimal &  0.24 & 8 &  8.00 &  0.00\\
instance n=50 401.alb & 1 & 0 & Solution & 120.04 & 28 & 26.00 &  7.14\\
instance n=50 402.alb & 1 & 0 & Solution & 120.04 & 27 & 26.00 &  3.70\\
instance n=50 403.alb & 1 & 0 & Solution & 120.06 & 34 & 30.00 & 11.76\\
instance n=50 404.alb & 1 & 0 & Solution & 120.07 & 31 & 26.00 & 16.13\\
instance n=50 405.alb & 1 & 0 & Solution & 120.05 & 27 & 26.00 &  3.70\\
instance n=50 406.alb & 1 & 0 & Solution & 120.06 & 32 & 30.00 &  6.25\\
instance n=50 407.alb & 1 & 0 & Solution & 120.06 & 29 & 26.00 & 10.34\\
instance n=50 408.alb & 1 & 0 & Optimal & 37.74 & 26 & 26.00 &  0.00\\
instance n=50 409.alb & 1 & 0 & Solution & 120.07 & 33 & 27.00 & 18.18\\
instance n=50 41.alb & 1 & 0 & Solution & 120.01 & 26 & 25.00 &  3.85\\
instance n=50 410.alb & 1 & 0 & Solution & 120.05 & 28 & 26.00 &  7.14\\
instance n=50 411.alb & 1 & 0 & Solution & 120.06 & 29 & 27.00 &  6.90\\
instance n=50 412.alb & 1 & 0 & Optimal & 109.80 & 26 & 26.00 &  0.00\\
instance n=50 413.alb & 1 & 0 & Solution & 120.07 & 30 & 26.00 & 13.33\\
instance n=50 414.alb & 1 & 0 & Solution & 120.05 & 27 & 25.00 &  7.41\\
instance n=50 415.alb & 1 & 0 & Solution & 120.07 & 28 & 26.00 &  7.14\\
instance n=50 416.alb & 1 & 0 & Solution & 120.10 & 27 & 26.00 &  3.70\\
instance n=50 417.alb & 1 & 0 & Solution & 120.06 & 30 & 27.00 & 10.00\\
instance n=50 418.alb & 1 & 0 & Solution & 120.07 & 27 & 25.00 &  7.41\\
instance n=50 419.alb & 1 & 0 & Solution & 120.08 & 33 & 28.00 & 15.15\\
instance n=50 42.alb & 1 & 0 & Solution & 120.00 & 24 & 23.00 &  4.17\\
instance n=50 420.alb & 1 & 0 & Solution & 120.05 & 28 & 26.00 &  7.14\\
instance n=50 421.alb & 1 & 0 & Solution & 120.06 & 34 & 29.00 & 14.71\\
instance n=50 422.alb & 1 & 0 & Solution & 120.05 & 29 & 26.00 & 10.34\\
instance n=50 423.alb & 1 & 0 & Solution & 120.03 & 29 & 26.00 & 10.34\\
instance n=50 424.alb & 1 & 0 & Solution & 120.05 & 27 & 26.00 &  3.70\\
instance n=50 425.alb & 1 & 0 & Solution & 120.07 & 34 & 30.00 & 11.76\\
instance n=50 426.alb & 1 & 0 & Optimal &  1.30 & 11 & 11.00 &  0.00\\
instance n=50 427.alb & 1 & 0 & Optimal &  0.37 & 12 & 12.00 &  0.00\\
instance n=50 428.alb & 1 & 0 & Optimal &  0.31 & 13 & 13.00 &  0.00\\
instance n=50 429.alb & 1 & 0 & Optimal &  0.31 & 11 & 11.00 &  0.00\\
instance n=50 43.alb & 1 & 0 & Optimal &  1.60 & 25 & 25.00 &  0.00\\
instance n=50 430.alb & 1 & 0 & Optimal &  1.26 & 14 & 14.00 &  0.00\\
instance n=50 431.alb & 1 & 0 & Optimal &  0.36 & 11 & 11.00 &  0.00\\
instance n=50 432.alb & 1 & 0 & Optimal &  1.30 & 12 & 12.00 &  0.00\\
instance n=50 433.alb & 1 & 0 & Optimal &  0.35 & 12 & 12.00 &  0.00\\
instance n=50 434.alb & 1 & 0 & Optimal &  0.57 & 11 & 11.00 &  0.00\\
instance n=50 435.alb & 1 & 0 & Optimal &  0.32 & 11 & 11.00 &  0.00\\
instance n=50 436.alb & 1 & 0 & Optimal &  0.24 & 11 & 11.00 &  0.00\\
instance n=50 437.alb & 1 & 0 & Optimal &  6.40 & 12 & 12.00 &  0.00\\
instance n=50 438.alb & 1 & 0 & Optimal &  4.98 & 10 & 10.00 &  0.00\\
instance n=50 439.alb & 1 & 0 & Optimal &  2.31 & 12 & 12.00 &  0.00\\
instance n=50 44.alb & 1 & 0 & Solution & 120.00 & 25 & 24.00 &  4.00\\
instance n=50 440.alb & 1 & 0 & Optimal &  8.07 & 13 & 13.00 &  0.00\\
instance n=50 441.alb & 1 & 0 & Optimal &  0.28 & 11 & 11.00 &  0.00\\
instance n=50 442.alb & 1 & 0 & Optimal &  0.64 & 12 & 12.00 &  0.00\\
instance n=50 443.alb & 1 & 0 & Optimal &  1.42 & 11 & 11.00 &  0.00\\
instance n=50 444.alb & 1 & 0 & Optimal &  0.36 & 12 & 12.00 &  0.00\\
instance n=50 445.alb & 1 & 0 & Optimal &  0.40 & 12 & 12.00 &  0.00\\
instance n=50 446.alb & 1 & 0 & Optimal &  0.71 & 12 & 12.00 &  0.00\\
instance n=50 447.alb & 1 & 0 & Optimal &  0.61 & 13 & 13.00 &  0.00\\
instance n=50 448.alb & 1 & 0 & Optimal &  7.05 & 12 & 12.00 &  0.00\\
instance n=50 449.alb & 1 & 0 & Optimal &  0.42 & 11 & 11.00 &  0.00\\
instance n=50 45.alb & 1 & 0 & Solution & 120.01 & 25 & 24.00 &  4.00\\
instance n=50 450.alb & 1 & 0 & Optimal &  0.33 & 11 & 11.00 &  0.00\\
instance n=50 451.alb & 1 & 0 & Optimal &  0.53 & 8 &  8.00 &  0.00\\
instance n=50 452.alb & 1 & 0 & Optimal &  0.28 & 8 &  8.00 &  0.00\\
instance n=50 453.alb & 1 & 0 & Optimal &  0.37 & 7 &  7.00 &  0.00\\
instance n=50 454.alb & 1 & 0 & Optimal &  1.08 & 8 &  8.00 &  0.00\\
instance n=50 455.alb & 1 & 0 & Optimal &  0.37 & 6 &  6.00 &  0.00\\
instance n=50 456.alb & 1 & 0 & Optimal &  0.46 & 8 &  8.00 &  0.00\\
instance n=50 457.alb & 1 & 0 & Optimal &  0.49 & 8 &  8.00 &  0.00\\
instance n=50 458.alb & 1 & 0 & Optimal &  0.60 & 7 &  7.00 &  0.00\\
instance n=50 459.alb & 1 & 0 & Optimal &  0.50 & 7 &  7.00 &  0.00\\
instance n=50 46.alb & 1 & 0 & Solution & 120.02 & 28 & 26.00 &  7.14\\
instance n=50 460.alb & 1 & 0 & Optimal &  0.53 & 7 &  7.00 &  0.00\\
instance n=50 461.alb & 1 & 0 & Optimal &  0.60 & 6 &  6.00 &  0.00\\
instance n=50 462.alb & 1 & 0 & Optimal &  0.33 & 7 &  7.00 &  0.00\\
instance n=50 463.alb & 1 & 0 & Optimal &  0.44 & 8 &  8.00 &  0.00\\
instance n=50 464.alb & 1 & 0 & Optimal &  0.50 & 6 &  6.00 &  0.00\\
instance n=50 465.alb & 1 & 0 & Optimal &  0.41 & 8 &  8.00 &  0.00\\
instance n=50 466.alb & 1 & 0 & Optimal &  0.63 & 7 &  7.00 &  0.00\\
instance n=50 467.alb & 1 & 0 & Optimal &  1.20 & 9 &  9.00 &  0.00\\
instance n=50 468.alb & 1 & 0 & Optimal &  0.41 & 7 &  7.00 &  0.00\\
instance n=50 469.alb & 1 & 0 & Optimal &  0.48 & 8 &  8.00 &  0.00\\
instance n=50 47.alb & 1 & 0 & Solution & 119.99 & 28 & 26.00 &  7.14\\
instance n=50 470.alb & 1 & 0 & Optimal &  0.36 & 8 &  8.00 &  0.00\\
instance n=50 471.alb & 1 & 0 & Optimal &  0.46 & 7 &  7.00 &  0.00\\
instance n=50 472.alb & 1 & 0 & Optimal &  0.38 & 8 &  8.00 &  0.00\\
instance n=50 473.alb & 1 & 0 & Optimal &  0.47 & 7 &  7.00 &  0.00\\
instance n=50 474.alb & 1 & 0 & Optimal &  0.47 & 7 &  7.00 &  0.00\\
instance n=50 475.alb & 1 & 0 & Optimal &  1.04 & 6 &  6.00 &  0.00\\
instance n=50 476.alb & 1 & 0 & Optimal &  1.13 & 28 & 28.00 &  0.00\\
instance n=50 477.alb & 1 & 0 & Optimal &  7.35 & 29 & 29.00 &  0.00\\
instance n=50 478.alb & 1 & 0 & Optimal & 10.16 & 32 & 32.00 &  0.00\\
instance n=50 479.alb & 1 & 0 & Optimal &  0.75 & 28 & 28.00 &  0.00\\
instance n=50 48.alb & 1 & 0 & Solution & 120.02 & 27 & 26.00 &  3.70\\
instance n=50 480.alb & 1 & 0 & Optimal &  1.32 & 34 & 34.00 &  0.00\\
instance n=50 481.alb & 1 & 0 & Optimal &  2.48 & 28 & 28.00 &  0.00\\
instance n=50 482.alb & 1 & 0 & Optimal &  1.79 & 27 & 27.00 &  0.00\\
instance n=50 483.alb & 1 & 0 & Optimal &  6.74 & 30 & 30.00 &  0.00\\
instance n=50 484.alb & 1 & 0 & Optimal &  1.87 & 32 & 32.00 &  0.00\\
instance n=50 485.alb & 1 & 0 & Optimal &  2.69 & 31 & 31.00 &  0.00\\
instance n=50 486.alb & 1 & 0 & Optimal &  1.51 & 32 & 31.00 &  3.13\\
instance n=50 487.alb & 1 & 0 & Optimal &  2.58 & 31 & 31.00 &  0.00\\
instance n=50 488.alb & 1 & 0 & Optimal &  6.63 & 31 & 31.00 &  0.00\\
instance n=50 489.alb & 1 & 0 & Optimal &  5.94 & 35 & 35.00 &  0.00\\
instance n=50 49.alb & 1 & 0 & Solution & 120.02 & 25 & 24.00 &  4.00\\
instance n=50 490.alb & 1 & 0 & Optimal &  2.56 & 29 & 29.00 &  0.00\\
instance n=50 491.alb & 1 & 0 & Optimal & 70.18 & 35 & 35.00 &  0.00\\
instance n=50 492.alb & 1 & 0 & Optimal &  7.01 & 29 & 29.00 &  0.00\\
instance n=50 493.alb & 1 & 0 & Optimal &  6.23 & 30 & 30.00 &  0.00\\
instance n=50 494.alb & 1 & 0 & Optimal &  3.77 & 32 & 32.00 &  0.00\\
instance n=50 495.alb & 1 & 0 & Optimal &  3.83 & 34 & 34.00 &  0.00\\
instance n=50 496.alb & 1 & 0 & Optimal &  3.96 & 29 & 29.00 &  0.00\\
instance n=50 497.alb & 1 & 0 & Optimal &  5.59 & 30 & 30.00 &  0.00\\
instance n=50 498.alb & 1 & 0 & Optimal &  1.51 & 30 & 30.00 &  0.00\\
instance n=50 499.alb & 1 & 0 & Optimal &  1.72 & 33 & 33.00 &  0.00\\
instance n=50 5.alb & 1 & 0 & Optimal &  0.03 & 7 &  7.00 &  0.00\\
instance n=50 50.alb & 1 & 0 & Solution & 120.00 & 27 & 25.00 &  7.41\\
instance n=50 500.alb & 1 & 0 & Optimal &  3.03 & 34 & 34.00 &  0.00\\
instance n=50 501.alb & 1 & 0 & Optimal &  1.21 & 12 & 12.00 &  0.00\\
instance n=50 502.alb & 1 & 0 & Optimal &  0.85 & 10 & 10.00 &  0.00\\
instance n=50 503.alb & 1 & 0 & Optimal &  1.21 & 13 & 13.00 &  0.00\\
instance n=50 504.alb & 1 & 0 & Optimal &  0.97 & 11 & 11.00 &  0.00\\
instance n=50 505.alb & 1 & 0 & Optimal &  0.94 & 12 & 12.00 &  0.00\\
instance n=50 506.alb & 1 & 0 & Optimal &  0.39 & 11 & 11.00 &  0.00\\
instance n=50 507.alb & 1 & 0 & Optimal &  0.64 & 13 & 13.00 &  0.00\\
instance n=50 508.alb & 1 & 0 & Optimal &  1.02 & 14 & 14.00 &  0.00\\
instance n=50 509.alb & 1 & 0 & Optimal &  0.36 & 13 & 13.00 &  0.00\\
instance n=50 51.alb & 1 & 0 & Optimal &  0.05 & 12 & 12.00 &  0.00\\
instance n=50 510.alb & 1 & 0 & Optimal &  1.22 & 11 & 11.00 &  0.00\\
instance n=50 511.alb & 1 & 0 & Optimal &  1.35 & 13 & 13.00 &  0.00\\
instance n=50 512.alb & 1 & 0 & Optimal &  1.16 & 13 & 13.00 &  0.00\\
instance n=50 513.alb & 1 & 0 & Optimal &  0.60 & 12 & 12.00 &  0.00\\
instance n=50 514.alb & 1 & 0 & Optimal &  1.32 & 12 & 12.00 &  0.00\\
instance n=50 515.alb & 1 & 0 & Optimal &  1.21 & 11 & 11.00 &  0.00\\
instance n=50 516.alb & 1 & 0 & Optimal &  0.88 & 13 & 13.00 &  0.00\\
instance n=50 517.alb & 1 & 0 & Optimal &  0.88 & 14 & 14.00 &  0.00\\
instance n=50 518.alb & 1 & 0 & Optimal &  1.19 & 11 & 11.00 &  0.00\\
instance n=50 519.alb & 1 & 0 & Optimal &  0.42 & 12 & 12.00 &  0.00\\
instance n=50 52.alb & 1 & 0 & Optimal &  0.05 & 11 & 11.00 &  0.00\\
instance n=50 520.alb & 1 & 0 & Optimal &  0.57 & 11 & 11.00 &  0.00\\
instance n=50 521.alb & 1 & 0 & Optimal &  0.33 & 10 & 10.00 &  0.00\\
instance n=50 522.alb & 1 & 0 & Optimal &  0.47 & 11 & 11.00 &  0.00\\
instance n=50 523.alb & 1 & 0 & Optimal &  1.01 & 11 & 11.00 &  0.00\\
instance n=50 524.alb & 1 & 0 & Optimal &  1.04 & 14 & 14.00 &  0.00\\
instance n=50 525.alb & 1 & 0 & Optimal &  1.29 & 11 & 11.00 &  0.00\\
instance n=50 53.alb & 1 & 0 & Solution & 120.01 & 13 & 12.00 &  7.69\\
instance n=50 54.alb & 1 & 0 & Optimal &  0.05 & 11 & 11.00 &  0.00\\
instance n=50 55.alb & 1 & 0 & Optimal &  0.07 & 13 & 13.00 &  0.00\\
instance n=50 56.alb & 1 & 0 & Optimal &  0.06 & 11 & 11.00 &  0.00\\
instance n=50 57.alb & 1 & 0 & Optimal &  0.06 & 13 & 13.00 &  0.00\\
instance n=50 58.alb & 1 & 0 & Optimal &  0.06 & 11 & 11.00 &  0.00\\
instance n=50 59.alb & 1 & 0 & Optimal &  0.06 & 11 & 11.00 &  0.00\\
instance n=50 6.alb & 1 & 0 & Optimal &  0.05 & 6 &  6.00 &  0.00\\
instance n=50 60.alb & 1 & 0 & Optimal &  0.23 & 12 & 12.00 &  0.00\\
instance n=50 61.alb & 1 & 0 & Optimal &  0.05 & 13 & 13.00 &  0.00\\
instance n=50 62.alb & 1 & 0 & Optimal &  0.06 & 13 & 13.00 &  0.00\\
instance n=50 63.alb & 1 & 0 & Optimal &  0.05 & 12 & 12.00 &  0.00\\
instance n=50 64.alb & 1 & 0 & Optimal &  0.05 & 13 & 13.00 &  0.00\\
instance n=50 65.alb & 1 & 0 & Optimal &  0.04 & 12 & 12.00 &  0.00\\
instance n=50 66.alb & 1 & 0 & Optimal &  0.25 & 12 & 12.00 &  0.00\\
instance n=50 67.alb & 1 & 0 & Optimal &  0.37 & 12 & 12.00 &  0.00\\
instance n=50 68.alb & 1 & 0 & Optimal &  0.08 & 12 & 12.00 &  0.00\\
instance n=50 69.alb & 1 & 0 & Optimal &  0.29 & 12 & 12.00 &  0.00\\
instance n=50 7.alb & 1 & 0 & Optimal &  0.03 & 7 &  7.00 &  0.00\\
instance n=50 70.alb & 1 & 0 & Optimal &  0.06 & 10 & 10.00 &  0.00\\
instance n=50 71.alb & 1 & 0 & Optimal &  0.09 & 13 & 13.00 &  0.00\\
instance n=50 72.alb & 1 & 0 & Optimal &  0.07 & 11 & 11.00 &  0.00\\
instance n=50 73.alb & 1 & 0 & Optimal &  0.07 & 11 & 11.00 &  0.00\\
instance n=50 74.alb & 1 & 0 & Optimal &  0.06 & 12 & 12.00 &  0.00\\
instance n=50 75.alb & 1 & 0 & Optimal &  0.74 & 11 & 11.00 &  0.00\\
instance n=50 76.alb & 1 & 0 & Optimal &  0.09 & 7 &  7.00 &  0.00\\
instance n=50 77.alb & 1 & 0 & Optimal &  0.06 & 7 &  7.00 &  0.00\\
instance n=50 78.alb & 1 & 0 & Optimal &  0.09 & 7 &  7.00 &  0.00\\
instance n=50 79.alb & 1 & 0 & Optimal &  0.20 & 8 &  8.00 &  0.00\\
instance n=50 8.alb & 1 & 0 & Optimal &  0.05 & 7 &  7.00 &  0.00\\
instance n=50 80.alb & 1 & 0 & Optimal &  0.08 & 7 &  7.00 &  0.00\\
instance n=50 81.alb & 1 & 0 & Optimal &  0.09 & 7 &  7.00 &  0.00\\
instance n=50 82.alb & 1 & 0 & Optimal &  0.08 & 6 &  6.00 &  0.00\\
instance n=50 83.alb & 1 & 0 & Optimal &  0.08 & 8 &  8.00 &  0.00\\
instance n=50 84.alb & 1 & 0 & Optimal &  0.08 & 7 &  7.00 &  0.00\\
instance n=50 85.alb & 1 & 0 & Optimal &  0.08 & 8 &  8.00 &  0.00\\
instance n=50 86.alb & 1 & 0 & Optimal &  0.08 & 7 &  7.00 &  0.00\\
instance n=50 87.alb & 1 & 0 & Optimal &  0.08 & 8 &  8.00 &  0.00\\
instance n=50 88.alb & 1 & 0 & Optimal &  0.08 & 8 &  8.00 &  0.00\\
instance n=50 89.alb & 1 & 0 & Optimal &  0.09 & 7 &  7.00 &  0.00\\
instance n=50 9.alb & 1 & 0 & Optimal &  0.03 & 9 &  9.00 &  0.00\\
instance n=50 90.alb & 1 & 0 & Optimal &  0.42 & 7 &  7.00 &  0.00\\
instance n=50 91.alb & 1 & 0 & Optimal &  0.08 & 7 &  7.00 &  0.00\\
instance n=50 92.alb & 1 & 0 & Optimal &  0.08 & 7 &  7.00 &  0.00\\
instance n=50 93.alb & 1 & 0 & Optimal &  0.06 & 7 &  7.00 &  0.00\\
instance n=50 94.alb & 1 & 0 & Optimal &  0.11 & 7 &  7.00 &  0.00\\
instance n=50 95.alb & 1 & 0 & Optimal &  0.08 & 7 &  7.00 &  0.00\\
instance n=50 96.alb & 1 & 0 & Optimal &  0.10 & 7 &  7.00 &  0.00\\
instance n=50 97.alb & 1 & 0 & Optimal &  0.22 & 7 &  7.00 &  0.00\\
instance n=50 98.alb & 1 & 0 & Optimal &  0.11 & 8 &  8.00 &  0.00\\
instance n=50 99.alb & 1 & 0 & Optimal &  0.11 & 7 &  7.00 &  0.00\\
\end{longtable}



\section{Results for CPSat}

\begin{longtable}{lrrlrrrr}
\caption{Results for SALBP-1 Problems (CPSat) (2100 Instances)}\\\toprule
Name & \shortstack{Nr\\Jobs} & \shortstack{Nr\\Machines} & Status & Time & Makespan & Bound & \shortstack{Gap\\Percent}\\ \midrule
\endhead
\bottomrule
\endfoot
instance n=1000 1.alb & 1 & 0 & Solution & 120.07 & 136 & 135.00 &  0.74\\
instance n=1000 10.alb & 1 & 0 & Solution & 120.08 & 141 & 140.00 &  0.71\\
instance n=1000 100.alb & 1 & 0 & Solution & 120.08 & 139 & 137.00 &  1.44\\
instance n=1000 101.alb & 1 & 0 & Solution & 120.17 & 554 & 430.00 & 22.38\\
instance n=1000 102.alb & 1 & 0 & Solution & 120.18 & 556 & 446.00 & 19.78\\
instance n=1000 103.alb & 1 & 0 & Solution & 120.23 & 560 & 469.00 & 16.25\\
instance n=1000 104.alb & 1 & 0 & Solution & 120.16 & 550 & 439.00 & 20.18\\
instance n=1000 105.alb & 1 & 0 & Solution & 120.16 & 545 & 439.00 & 19.45\\
instance n=1000 106.alb & 1 & 0 & Solution & 120.19 & 552 & 432.00 & 21.74\\
instance n=1000 107.alb & 1 & 0 & Solution & 120.14 & 540 & 444.00 & 17.78\\
instance n=1000 108.alb & 1 & 0 & Solution & 120.13 & 543 & 461.00 & 15.10\\
instance n=1000 109.alb & 1 & 0 & Solution & 120.19 & 546 & 427.00 & 21.79\\
instance n=1000 11.alb & 1 & 0 & Solution & 120.08 & 135 & 134.00 &  0.74\\
instance n=1000 110.alb & 1 & 0 & Solution & 120.18 & 557 & 430.00 & 22.80\\
instance n=1000 111.alb & 1 & 0 & Solution & 120.13 & 544 & 449.00 & 17.46\\
instance n=1000 112.alb & 1 & 0 & Solution & 120.18 & 549 & 449.00 & 18.21\\
instance n=1000 113.alb & 1 & 0 & Solution & 120.16 & 537 & 459.00 & 14.53\\
instance n=1000 114.alb & 1 & 0 & Solution & 120.13 & 548 & 425.00 & 22.45\\
instance n=1000 115.alb & 1 & 0 & Solution & 120.16 & 541 & 430.00 & 20.52\\
instance n=1000 116.alb & 1 & 0 & Solution & 120.17 & 543 & 449.00 & 17.31\\
instance n=1000 117.alb & 1 & 0 & Solution & 120.17 & 548 & 443.00 & 19.16\\
instance n=1000 118.alb & 1 & 0 & Solution & 120.13 & 564 & 452.00 & 19.86\\
instance n=1000 119.alb & 1 & 0 & Solution & 120.14 & 534 & 424.00 & 20.60\\
instance n=1000 12.alb & 1 & 0 & Solution & 120.07 & 135 & 134.00 &  0.74\\
instance n=1000 120.alb & 1 & 0 & Solution & 120.15 & 549 & 466.00 & 15.12\\
instance n=1000 121.alb & 1 & 0 & Solution & 120.15 & 543 & 470.00 & 13.44\\
instance n=1000 122.alb & 1 & 0 & Solution & 120.16 & 533 & 465.00 & 12.76\\
instance n=1000 123.alb & 1 & 0 & Solution & 120.15 & 556 & 441.00 & 20.68\\
instance n=1000 124.alb & 1 & 0 & Solution & 120.20 & 543 & 452.00 & 16.76\\
instance n=1000 125.alb & 1 & 0 & Solution & 120.16 & 545 & 440.00 & 19.27\\
instance n=1000 126.alb & 1 & 0 & Solution & 120.15 & 232 & 228.00 &  1.72\\
instance n=1000 127.alb & 1 & 0 & Solution & 120.09 & 224 & 221.00 &  1.34\\
instance n=1000 128.alb & 1 & 0 & Solution & 120.11 & 225 & 222.00 &  1.33\\
instance n=1000 129.alb & 1 & 0 & Solution & 120.12 & 226 & 223.00 &  1.33\\
instance n=1000 13.alb & 1 & 0 & Solution & 120.08 & 132 & 131.00 &  0.76\\
instance n=1000 130.alb & 1 & 0 & Solution & 120.09 & 225 & 221.00 &  1.78\\
instance n=1000 131.alb & 1 & 0 & Solution & 120.10 & 224 & 220.00 &  1.79\\
instance n=1000 132.alb & 1 & 0 & Solution & 120.09 & 218 & 214.00 &  1.83\\
instance n=1000 133.alb & 1 & 0 & Solution & 120.11 & 229 & 226.00 &  1.31\\
instance n=1000 134.alb & 1 & 0 & Solution & 120.12 & 219 & 215.00 &  1.83\\
instance n=1000 135.alb & 1 & 0 & Solution & 120.13 & 229 & 225.00 &  1.75\\
instance n=1000 136.alb & 1 & 0 & Solution & 120.11 & 232 & 228.00 &  1.72\\
instance n=1000 137.alb & 1 & 0 & Solution & 120.09 & 216 & 213.00 &  1.39\\
instance n=1000 138.alb & 1 & 0 & Solution & 120.12 & 225 & 221.00 &  1.78\\
instance n=1000 139.alb & 1 & 0 & Solution & 120.09 & 228 & 224.00 &  1.75\\
instance n=1000 14.alb & 1 & 0 & Solution & 120.08 & 138 & 136.00 &  1.45\\
instance n=1000 140.alb & 1 & 0 & Solution & 120.11 & 230 & 226.00 &  1.74\\
instance n=1000 141.alb & 1 & 0 & Solution & 120.10 & 219 & 215.00 &  1.83\\
instance n=1000 142.alb & 1 & 0 & Solution & 120.09 & 224 & 220.00 &  1.79\\
instance n=1000 143.alb & 1 & 0 & Solution & 120.10 & 217 & 213.00 &  1.84\\
instance n=1000 144.alb & 1 & 0 & Solution & 120.11 & 220 & 217.00 &  1.36\\
instance n=1000 145.alb & 1 & 0 & Solution & 120.10 & 223 & 220.00 &  1.35\\
instance n=1000 146.alb & 1 & 0 & Solution & 120.11 & 223 & 219.00 &  1.79\\
instance n=1000 147.alb & 1 & 0 & Solution & 120.13 & 233 & 229.00 &  1.72\\
instance n=1000 148.alb & 1 & 0 & Solution & 120.37 & 223 & 219.00 &  1.79\\
instance n=1000 149.alb & 1 & 0 & Solution & 120.10 & 241 & 237.00 &  1.66\\
instance n=1000 15.alb & 1 & 0 & Solution & 120.08 & 137 & 136.00 &  0.73\\
instance n=1000 150.alb & 1 & 0 & Solution & 120.09 & 225 & 222.00 &  1.33\\
instance n=1000 151.alb & 1 & 0 & Solution & 120.08 & 140 & 138.00 &  1.43\\
instance n=1000 152.alb & 1 & 0 & Solution & 120.07 & 138 & 136.00 &  1.45\\
instance n=1000 153.alb & 1 & 0 & Solution & 120.08 & 139 & 137.00 &  1.44\\
instance n=1000 154.alb & 1 & 0 & Solution & 120.10 & 142 & 140.00 &  1.41\\
instance n=1000 155.alb & 1 & 0 & Solution & 120.08 & 141 & 139.00 &  1.42\\
instance n=1000 156.alb & 1 & 0 & Solution & 120.10 & 143 & 141.00 &  1.40\\
instance n=1000 157.alb & 1 & 0 & Solution & 120.09 & 142 & 140.00 &  1.41\\
instance n=1000 158.alb & 1 & 0 & Solution & 120.10 & 137 & 136.00 &  0.73\\
instance n=1000 159.alb & 1 & 0 & Solution & 120.09 & 139 & 138.00 &  0.72\\
instance n=1000 16.alb & 1 & 0 & Solution & 120.09 & 138 & 137.00 &  0.72\\
instance n=1000 160.alb & 1 & 0 & Solution & 120.08 & 140 & 138.00 &  1.43\\
instance n=1000 161.alb & 1 & 0 & Solution & 120.08 & 134 & 133.00 &  0.75\\
instance n=1000 162.alb & 1 & 0 & Solution & 120.09 & 137 & 136.00 &  0.73\\
instance n=1000 163.alb & 1 & 0 & Solution & 120.09 & 141 & 139.00 &  1.42\\
instance n=1000 164.alb & 1 & 0 & Solution & 120.09 & 143 & 141.00 &  1.40\\
instance n=1000 165.alb & 1 & 0 & Solution & 120.09 & 137 & 135.00 &  1.46\\
instance n=1000 166.alb & 1 & 0 & Solution & 120.08 & 141 & 139.00 &  1.42\\
instance n=1000 167.alb & 1 & 0 & Solution & 120.11 & 140 & 139.00 &  0.71\\
instance n=1000 168.alb & 1 & 0 & Solution & 120.09 & 140 & 138.00 &  1.43\\
instance n=1000 169.alb & 1 & 0 & Solution & 120.10 & 136 & 134.00 &  1.47\\
instance n=1000 17.alb & 1 & 0 & Solution & 120.09 & 136 & 135.00 &  0.74\\
instance n=1000 170.alb & 1 & 0 & Solution & 120.08 & 136 & 134.00 &  1.47\\
instance n=1000 171.alb & 1 & 0 & Solution & 120.09 & 138 & 137.00 &  0.72\\
instance n=1000 172.alb & 1 & 0 & Solution & 120.09 & 136 & 135.00 &  0.74\\
instance n=1000 173.alb & 1 & 0 & Solution & 120.10 & 136 & 135.00 &  0.74\\
instance n=1000 174.alb & 1 & 0 & Solution & 120.08 & 137 & 136.00 &  0.73\\
instance n=1000 175.alb & 1 & 0 & Solution & 120.09 & 140 & 138.00 &  1.43\\
instance n=1000 176.alb & 1 & 0 & Solution & 120.11 & 562 & 322.00 & 42.70\\
instance n=1000 177.alb & 1 & 0 & Solution & 120.09 & 563 & 326.00 & 42.10\\
instance n=1000 178.alb & 1 & 0 & Solution & 120.11 & 567 & 325.00 & 42.68\\
instance n=1000 179.alb & 1 & 0 & Solution & 120.11 & 571 & 315.00 & 44.83\\
instance n=1000 18.alb & 1 & 0 & Solution & 120.08 & 135 & 134.00 &  0.74\\
instance n=1000 180.alb & 1 & 0 & Solution & 120.11 & 567 & 317.00 & 44.09\\
instance n=1000 181.alb & 1 & 0 & Solution & 120.16 & 567 & 320.00 & 43.56\\
instance n=1000 182.alb & 1 & 0 & Solution & 120.11 & 565 & 315.00 & 44.25\\
instance n=1000 183.alb & 1 & 0 & Solution & 120.11 & 554 & 317.00 & 42.78\\
instance n=1000 184.alb & 1 & 0 & Solution & 120.14 & 561 & 317.00 & 43.49\\
instance n=1000 185.alb & 1 & 0 & Solution & 120.13 & 557 & 319.00 & 42.73\\
instance n=1000 186.alb & 1 & 0 & Solution & 120.12 & 562 & 325.00 & 42.17\\
instance n=1000 187.alb & 1 & 0 & Solution & 120.13 & 565 & 331.00 & 41.42\\
instance n=1000 188.alb & 1 & 0 & Solution & 120.12 & 555 & 332.00 & 40.18\\
instance n=1000 189.alb & 1 & 0 & Solution & 120.12 & 555 & 317.00 & 42.88\\
instance n=1000 19.alb & 1 & 0 & Solution & 120.08 & 138 & 137.00 &  0.72\\
instance n=1000 190.alb & 1 & 0 & Solution & 120.11 & 563 & 313.00 & 44.40\\
instance n=1000 191.alb & 1 & 0 & Solution & 120.11 & 560 & 328.00 & 41.43\\
instance n=1000 192.alb & 1 & 0 & Solution & 120.13 & 563 & 326.00 & 42.10\\
instance n=1000 193.alb & 1 & 0 & Solution & 120.08 & 568 & 327.00 & 42.43\\
instance n=1000 194.alb & 1 & 0 & Solution & 120.13 & 568 & 319.00 & 43.84\\
instance n=1000 195.alb & 1 & 0 & Solution & 120.08 & 568 & 315.00 & 44.54\\
instance n=1000 196.alb & 1 & 0 & Solution & 120.11 & 560 & 320.00 & 42.86\\
instance n=1000 197.alb & 1 & 0 & Solution & 120.12 & 546 & 336.00 & 38.46\\
instance n=1000 198.alb & 1 & 0 & Solution & 120.13 & 567 & 318.00 & 43.92\\
instance n=1000 199.alb & 1 & 0 & Solution & 120.12 & 547 & 328.00 & 40.04\\
instance n=1000 2.alb & 1 & 0 & Solution & 120.07 & 138 & 137.00 &  0.72\\
instance n=1000 20.alb & 1 & 0 & Solution & 120.09 & 139 & 138.00 &  0.72\\
instance n=1000 200.alb & 1 & 0 & Solution & 120.09 & 556 & 322.00 & 42.09\\
instance n=1000 201.alb & 1 & 0 & Solution & 120.09 & 233 & 215.00 &  7.73\\
instance n=1000 202.alb & 1 & 0 & Solution & 120.11 & 230 & 188.00 & 18.26\\
instance n=1000 203.alb & 1 & 0 & Solution & 120.09 & 234 & 210.00 & 10.26\\
instance n=1000 204.alb & 1 & 0 & Solution & 120.10 & 232 & 218.00 &  6.03\\
instance n=1000 205.alb & 1 & 0 & Solution & 120.10 & 234 & 188.00 & 19.66\\
instance n=1000 206.alb & 1 & 0 & Solution & 120.10 & 233 & 192.00 & 17.60\\
instance n=1000 207.alb & 1 & 0 & Solution & 120.10 & 235 & 197.00 & 16.17\\
instance n=1000 208.alb & 1 & 0 & Solution & 120.12 & 234 & 226.00 &  3.42\\
instance n=1000 209.alb & 1 & 0 & Solution & 120.11 & 232 & 194.00 & 16.38\\
instance n=1000 21.alb & 1 & 0 & Solution & 120.10 & 139 & 138.00 &  0.72\\
instance n=1000 210.alb & 1 & 0 & Solution & 120.11 & 229 & 205.00 & 10.48\\
instance n=1000 211.alb & 1 & 0 & Solution & 120.10 & 224 & 189.00 & 15.63\\
instance n=1000 212.alb & 1 & 0 & Solution & 120.10 & 221 & 182.00 & 17.65\\
instance n=1000 213.alb & 1 & 0 & Solution & 120.12 & 238 & 211.00 & 11.34\\
instance n=1000 214.alb & 1 & 0 & Solution & 120.10 & 230 & 206.00 & 10.43\\
instance n=1000 215.alb & 1 & 0 & Solution & 120.07 & 227 & 218.00 &  3.96\\
instance n=1000 216.alb & 1 & 0 & Solution & 120.27 & 225 & 190.00 & 15.56\\
instance n=1000 217.alb & 1 & 0 & Solution & 120.10 & 229 & 191.00 & 16.59\\
instance n=1000 218.alb & 1 & 0 & Solution & 120.09 & 223 & 212.00 &  4.93\\
instance n=1000 219.alb & 1 & 0 & Solution & 120.10 & 237 & 223.00 &  5.91\\
instance n=1000 22.alb & 1 & 0 & Solution & 120.09 & 139 & 137.00 &  1.44\\
instance n=1000 220.alb & 1 & 0 & Solution & 120.09 & 229 & 200.00 & 12.66\\
instance n=1000 221.alb & 1 & 0 & Solution & 120.10 & 236 & 211.00 & 10.59\\
instance n=1000 222.alb & 1 & 0 & Solution & 120.12 & 226 & 212.00 &  6.19\\
instance n=1000 223.alb & 1 & 0 & Solution & 120.10 & 226 & 205.00 &  9.29\\
instance n=1000 224.alb & 1 & 0 & Solution & 120.11 & 231 & 219.00 &  5.19\\
instance n=1000 225.alb & 1 & 0 & Solution & 120.11 & 234 & 210.00 & 10.26\\
instance n=1000 226.alb & 1 & 0 & Solution & 120.10 & 138 & 136.00 &  1.45\\
instance n=1000 227.alb & 1 & 0 & Solution & 120.10 & 140 & 138.00 &  1.43\\
instance n=1000 228.alb & 1 & 0 & Solution & 120.09 & 135 & 133.00 &  1.48\\
instance n=1000 229.alb & 1 & 0 & Solution & 120.11 & 136 & 134.00 &  1.47\\
instance n=1000 23.alb & 1 & 0 & Solution & 120.35 & 137 & 136.00 &  0.73\\
instance n=1000 230.alb & 1 & 0 & Solution & 120.09 & 133 & 131.00 &  1.50\\
instance n=1000 231.alb & 1 & 0 & Solution & 120.11 & 140 & 138.00 &  1.43\\
instance n=1000 232.alb & 1 & 0 & Solution & 120.10 & 135 & 133.00 &  1.48\\
instance n=1000 233.alb & 1 & 0 & Solution & 120.13 & 137 & 135.00 &  1.46\\
instance n=1000 234.alb & 1 & 0 & Solution & 120.10 & 139 & 137.00 &  1.44\\
instance n=1000 235.alb & 1 & 0 & Solution & 120.09 & 134 & 133.00 &  0.75\\
instance n=1000 236.alb & 1 & 0 & Solution & 120.10 & 138 & 136.00 &  1.45\\
instance n=1000 237.alb & 1 & 0 & Solution & 120.08 & 140 & 138.00 &  1.43\\
instance n=1000 238.alb & 1 & 0 & Solution & 120.12 & 139 & 138.00 &  0.72\\
instance n=1000 239.alb & 1 & 0 & Solution & 120.09 & 136 & 135.00 &  0.74\\
instance n=1000 24.alb & 1 & 0 & Solution & 120.07 & 141 & 140.00 &  0.71\\
instance n=1000 240.alb & 1 & 0 & Solution & 120.09 & 137 & 135.00 &  1.46\\
instance n=1000 241.alb & 1 & 0 & Solution & 120.10 & 140 & 138.00 &  1.43\\
instance n=1000 242.alb & 1 & 0 & Solution & 120.21 & 137 & 135.00 &  1.46\\
instance n=1000 243.alb & 1 & 0 & Solution & 120.10 & 139 & 137.00 &  1.44\\
instance n=1000 244.alb & 1 & 0 & Solution & 120.12 & 138 & 137.00 &  0.72\\
instance n=1000 245.alb & 1 & 0 & Solution & 120.09 & 137 & 135.00 &  1.46\\
instance n=1000 246.alb & 1 & 0 & Solution & 120.10 & 137 & 135.00 &  1.46\\
instance n=1000 247.alb & 1 & 0 & Solution & 120.10 & 140 & 138.00 &  1.43\\
instance n=1000 248.alb & 1 & 0 & Solution & 120.13 & 141 & 138.00 &  2.13\\
instance n=1000 249.alb & 1 & 0 & Solution & 120.09 & 140 & 138.00 &  1.43\\
instance n=1000 25.alb & 1 & 0 & Solution & 120.08 & 137 & 136.00 &  0.73\\
instance n=1000 250.alb & 1 & 0 & Solution & 120.09 & 142 & 140.00 &  1.41\\
instance n=1000 251.alb & 1 & 0 & Solution & 120.18 & 577 & 445.00 & 22.88\\
instance n=1000 252.alb & 1 & 0 & Solution & 120.17 & 569 & 453.00 & 20.39\\
instance n=1000 253.alb & 1 & 0 & Solution & 120.18 & 577 & 403.00 & 30.16\\
instance n=1000 254.alb & 1 & 0 & Solution & 120.17 & 568 & 409.00 & 27.99\\
instance n=1000 255.alb & 1 & 0 & Solution & 120.16 & 556 & 440.00 & 20.86\\
instance n=1000 256.alb & 1 & 0 & Solution & 120.16 & 561 & 427.00 & 23.89\\
instance n=1000 257.alb & 1 & 0 & Solution & 120.17 & 571 & 407.00 & 28.72\\
instance n=1000 258.alb & 1 & 0 & Solution & 120.15 & 562 & 423.00 & 24.73\\
instance n=1000 259.alb & 1 & 0 & Solution & 120.20 & 561 & 444.00 & 20.86\\
instance n=1000 26.alb & 1 & 0 & Solution & 120.14 & 554 & 316.00 & 42.96\\
instance n=1000 260.alb & 1 & 0 & Solution & 120.16 & 556 & 437.00 & 21.40\\
instance n=1000 261.alb & 1 & 0 & Solution & 120.18 & 566 & 420.00 & 25.80\\
instance n=1000 262.alb & 1 & 0 & Solution & 120.19 & 554 & 441.00 & 20.40\\
instance n=1000 263.alb & 1 & 0 & Solution & 120.16 & 564 & 443.00 & 21.45\\
instance n=1000 264.alb & 1 & 0 & Solution & 120.18 & 560 & 434.00 & 22.50\\
instance n=1000 265.alb & 1 & 0 & Solution & 120.17 & 580 & 429.00 & 26.03\\
instance n=1000 266.alb & 1 & 0 & Solution & 120.12 & 560 & 416.00 & 25.71\\
instance n=1000 267.alb & 1 & 0 & Solution & 120.15 & 584 & 413.00 & 29.28\\
instance n=1000 268.alb & 1 & 0 & Solution & 120.14 & 558 & 444.00 & 20.43\\
instance n=1000 269.alb & 1 & 0 & Solution & 120.20 & 564 & 434.00 & 23.05\\
instance n=1000 27.alb & 1 & 0 & Solution & 120.13 & 554 & 314.00 & 43.32\\
instance n=1000 270.alb & 1 & 0 & Solution & 120.16 & 590 & 440.00 & 25.42\\
instance n=1000 271.alb & 1 & 0 & Solution & 120.16 & 555 & 410.00 & 26.13\\
instance n=1000 272.alb & 1 & 0 & Solution & 120.16 & 577 & 430.00 & 25.48\\
instance n=1000 273.alb & 1 & 0 & Solution & 120.15 & 566 & 428.00 & 24.38\\
instance n=1000 274.alb & 1 & 0 & Solution & 120.15 & 566 & 429.00 & 24.20\\
instance n=1000 275.alb & 1 & 0 & Solution & 120.16 & 571 & 445.00 & 22.07\\
instance n=1000 276.alb & 1 & 0 & Solution & 120.12 & 222 & 217.00 &  2.25\\
instance n=1000 277.alb & 1 & 0 & Solution & 120.12 & 230 & 225.00 &  2.17\\
instance n=1000 278.alb & 1 & 0 & Solution & 120.11 & 225 & 220.00 &  2.22\\
instance n=1000 279.alb & 1 & 0 & Solution & 120.11 & 220 & 215.00 &  2.27\\
instance n=1000 28.alb & 1 & 0 & Solution & 120.10 & 541 & 300.00 & 44.55\\
instance n=1000 280.alb & 1 & 0 & Solution & 120.09 & 230 & 226.00 &  1.74\\
instance n=1000 281.alb & 1 & 0 & Solution & 120.09 & 224 & 219.00 &  2.23\\
instance n=1000 282.alb & 1 & 0 & Solution & 120.11 & 219 & 214.00 &  2.28\\
instance n=1000 283.alb & 1 & 0 & Solution & 120.12 & 229 & 224.00 &  2.18\\
instance n=1000 284.alb & 1 & 0 & Solution & 120.11 & 222 & 217.00 &  2.25\\
instance n=1000 285.alb & 1 & 0 & Solution & 120.10 & 225 & 221.00 &  1.78\\
instance n=1000 286.alb & 1 & 0 & Solution & 120.21 & 226 & 221.00 &  2.21\\
instance n=1000 287.alb & 1 & 0 & Solution & 120.11 & 229 & 224.00 &  2.18\\
instance n=1000 288.alb & 1 & 0 & Solution & 120.14 & 224 & 219.00 &  2.23\\
instance n=1000 289.alb & 1 & 0 & Solution & 120.15 & 225 & 220.00 &  2.22\\
instance n=1000 29.alb & 1 & 0 & Solution & 120.09 & 544 & 317.00 & 41.73\\
instance n=1000 290.alb & 1 & 0 & Solution & 120.10 & 227 & 222.00 &  2.20\\
instance n=1000 291.alb & 1 & 0 & Solution & 120.11 & 230 & 225.00 &  2.17\\
instance n=1000 292.alb & 1 & 0 & Solution & 120.13 & 231 & 226.00 &  2.16\\
instance n=1000 293.alb & 1 & 0 & Solution & 120.10 & 231 & 225.00 &  2.60\\
instance n=1000 294.alb & 1 & 0 & Solution & 120.09 & 235 & 230.00 &  2.13\\
instance n=1000 295.alb & 1 & 0 & Solution & 120.14 & 233 & 227.00 &  2.58\\
instance n=1000 296.alb & 1 & 0 & Solution & 120.12 & 212 & 208.00 &  1.89\\
instance n=1000 297.alb & 1 & 0 & Solution & 120.13 & 221 & 217.00 &  1.81\\
instance n=1000 298.alb & 1 & 0 & Solution & 120.10 & 219 & 214.00 &  2.28\\
instance n=1000 299.alb & 1 & 0 & Solution & 120.12 & 231 & 226.00 &  2.16\\
instance n=1000 3.alb & 1 & 0 & Solution & 120.09 & 138 & 136.00 &  1.45\\
instance n=1000 30.alb & 1 & 0 & Solution & 120.11 & 570 & 314.00 & 44.91\\
instance n=1000 300.alb & 1 & 0 & Solution & 120.11 & 234 & 228.00 &  2.56\\
instance n=1000 301.alb & 1 & 0 & Solution & 120.11 & 138 & 137.00 &  0.72\\
instance n=1000 302.alb & 1 & 0 & Solution & 120.09 & 140 & 139.00 &  0.71\\
instance n=1000 303.alb & 1 & 0 & Solution & 120.09 & 140 & 138.00 &  1.43\\
instance n=1000 304.alb & 1 & 0 & Solution & 120.08 & 138 & 136.00 &  1.45\\
instance n=1000 305.alb & 1 & 0 & Solution & 120.11 & 141 & 140.00 &  0.71\\
instance n=1000 306.alb & 1 & 0 & Solution & 120.09 & 136 & 135.00 &  0.74\\
instance n=1000 307.alb & 1 & 0 & Solution & 120.10 & 137 & 136.00 &  0.73\\
instance n=1000 308.alb & 1 & 0 & Solution & 120.12 & 139 & 137.00 &  1.44\\
instance n=1000 309.alb & 1 & 0 & Solution & 120.10 & 136 & 135.00 &  0.74\\
instance n=1000 31.alb & 1 & 0 & Solution & 120.10 & 556 & 317.00 & 42.99\\
instance n=1000 310.alb & 1 & 0 & Solution & 120.10 & 143 & 141.00 &  1.40\\
instance n=1000 311.alb & 1 & 0 & Solution & 120.09 & 140 & 139.00 &  0.71\\
instance n=1000 312.alb & 1 & 0 & Solution & 120.09 & 136 & 135.00 &  0.74\\
instance n=1000 313.alb & 1 & 0 & Solution & 120.08 & 139 & 138.00 &  0.72\\
instance n=1000 314.alb & 1 & 0 & Solution & 120.09 & 143 & 142.00 &  0.70\\
instance n=1000 315.alb & 1 & 0 & Solution & 120.09 & 138 & 136.00 &  1.45\\
instance n=1000 316.alb & 1 & 0 & Solution & 120.10 & 138 & 137.00 &  0.72\\
instance n=1000 317.alb & 1 & 0 & Solution & 120.10 & 137 & 136.00 &  0.73\\
instance n=1000 318.alb & 1 & 0 & Solution & 120.09 & 139 & 138.00 &  0.72\\
instance n=1000 319.alb & 1 & 0 & Solution & 120.09 & 142 & 140.00 &  1.41\\
instance n=1000 32.alb & 1 & 0 & Solution & 120.11 & 554 & 318.00 & 42.60\\
instance n=1000 320.alb & 1 & 0 & Solution & 120.09 & 143 & 141.00 &  1.40\\
instance n=1000 321.alb & 1 & 0 & Solution & 120.07 & 141 & 140.00 &  0.71\\
instance n=1000 322.alb & 1 & 0 & Solution & 120.09 & 140 & 139.00 &  0.71\\
instance n=1000 323.alb & 1 & 0 & Solution & 120.09 & 139 & 138.00 &  0.72\\
instance n=1000 324.alb & 1 & 0 & Solution & 120.11 & 141 & 140.00 &  0.71\\
instance n=1000 325.alb & 1 & 0 & Solution & 120.12 & 139 & 138.00 &  0.72\\
instance n=1000 326.alb & 1 & 0 & Solution & 120.11 & 551 & 304.00 & 44.83\\
instance n=1000 327.alb & 1 & 0 & Solution & 120.12 & 564 & 325.00 & 42.38\\
instance n=1000 328.alb & 1 & 0 & Solution & 120.14 & 553 & 322.00 & 41.77\\
instance n=1000 329.alb & 1 & 0 & Solution & 120.11 & 565 & 324.00 & 42.65\\
instance n=1000 33.alb & 1 & 0 & Solution & 120.13 & 551 & 322.00 & 41.56\\
instance n=1000 330.alb & 1 & 0 & Solution & 120.12 & 545 & 319.00 & 41.47\\
instance n=1000 331.alb & 1 & 0 & Solution & 120.11 & 550 & 318.00 & 42.18\\
instance n=1000 332.alb & 1 & 0 & Solution & 120.10 & 543 & 323.00 & 40.52\\
instance n=1000 333.alb & 1 & 0 & Solution & 120.10 & 555 & 318.00 & 42.70\\
instance n=1000 334.alb & 1 & 0 & Solution & 120.11 & 544 & 332.00 & 38.97\\
instance n=1000 335.alb & 1 & 0 & Solution & 120.12 & 548 & 307.00 & 43.98\\
instance n=1000 336.alb & 1 & 0 & Solution & 120.16 & 544 & 328.00 & 39.71\\
instance n=1000 337.alb & 1 & 0 & Solution & 120.18 & 554 & 312.00 & 43.68\\
instance n=1000 338.alb & 1 & 0 & Solution & 120.14 & 554 & 320.00 & 42.24\\
instance n=1000 339.alb & 1 & 0 & Solution & 120.12 & 557 & 309.00 & 44.52\\
instance n=1000 34.alb & 1 & 0 & Solution & 120.11 & 575 & 325.00 & 43.48\\
instance n=1000 340.alb & 1 & 0 & Solution & 120.10 & 567 & 318.00 & 43.92\\
instance n=1000 341.alb & 1 & 0 & Solution & 120.17 & 555 & 313.00 & 43.60\\
instance n=1000 342.alb & 1 & 0 & Solution & 120.15 & 552 & 312.00 & 43.48\\
instance n=1000 343.alb & 1 & 0 & Solution & 120.13 & 552 & 328.00 & 40.58\\
instance n=1000 344.alb & 1 & 0 & Solution & 120.15 & 552 & 318.00 & 42.39\\
instance n=1000 345.alb & 1 & 0 & Solution & 120.11 & 560 & 315.00 & 43.75\\
instance n=1000 346.alb & 1 & 0 & Solution & 120.12 & 550 & 316.00 & 42.55\\
instance n=1000 347.alb & 1 & 0 & Solution & 120.18 & 549 & 316.00 & 42.44\\
instance n=1000 348.alb & 1 & 0 & Solution & 120.14 & 570 & 321.00 & 43.68\\
instance n=1000 349.alb & 1 & 0 & Solution & 120.13 & 559 & 335.00 & 40.07\\
instance n=1000 35.alb & 1 & 0 & Solution & 120.13 & 544 & 321.00 & 40.99\\
instance n=1000 350.alb & 1 & 0 & Solution & 120.14 & 539 & 307.00 & 43.04\\
instance n=1000 351.alb & 1 & 0 & Solution & 120.12 & 232 & 216.00 &  6.90\\
instance n=1000 352.alb & 1 & 0 & Solution & 120.10 & 231 & 208.00 &  9.96\\
instance n=1000 353.alb & 1 & 0 & Solution & 120.09 & 220 & 210.00 &  4.55\\
instance n=1000 354.alb & 1 & 0 & Solution & 120.11 & 226 & 212.00 &  6.19\\
instance n=1000 355.alb & 1 & 0 & Solution & 120.08 & 224 & 220.00 &  1.79\\
instance n=1000 356.alb & 1 & 0 & Solution & 120.10 & 230 & 221.00 &  3.91\\
instance n=1000 357.alb & 1 & 0 & Solution & 120.10 & 216 & 121.00 & 43.98\\
instance n=1000 358.alb & 1 & 0 & Solution & 120.09 & 223 & 218.00 &  2.24\\
instance n=1000 359.alb & 1 & 0 & Solution & 120.10 & 226 & 215.00 &  4.87\\
instance n=1000 36.alb & 1 & 0 & Solution & 120.12 & 547 & 316.00 & 42.23\\
instance n=1000 360.alb & 1 & 0 & Solution & 120.11 & 233 & 215.00 &  7.73\\
instance n=1000 361.alb & 1 & 0 & Solution & 120.11 & 219 & 215.00 &  1.83\\
instance n=1000 362.alb & 1 & 0 & Solution & 120.11 & 227 & 204.00 & 10.13\\
instance n=1000 363.alb & 1 & 0 & Solution & 120.09 & 219 & 206.00 &  5.94\\
instance n=1000 364.alb & 1 & 0 & Solution & 120.12 & 224 & 220.00 &  1.79\\
instance n=1000 365.alb & 1 & 0 & Solution & 120.11 & 231 & 217.00 &  6.06\\
instance n=1000 366.alb & 1 & 0 & Solution & 120.10 & 231 & 214.00 &  7.36\\
instance n=1000 367.alb & 1 & 0 & Solution & 120.13 & 231 & 217.00 &  6.06\\
instance n=1000 368.alb & 1 & 0 & Solution & 120.11 & 230 & 219.00 &  4.78\\
instance n=1000 369.alb & 1 & 0 & Solution & 120.11 & 224 & 181.00 & 19.20\\
instance n=1000 37.alb & 1 & 0 & Solution & 120.14 & 566 & 314.00 & 44.52\\
instance n=1000 370.alb & 1 & 0 & Solution & 120.10 & 227 & 210.00 &  7.49\\
instance n=1000 371.alb & 1 & 0 & Solution & 120.15 & 223 & 215.00 &  3.59\\
instance n=1000 372.alb & 1 & 0 & Solution & 120.10 & 234 & 208.00 & 11.11\\
instance n=1000 373.alb & 1 & 0 & Solution & 120.09 & 222 & 215.00 &  3.15\\
instance n=1000 374.alb & 1 & 0 & Solution & 120.10 & 222 & 219.00 &  1.35\\
instance n=1000 375.alb & 1 & 0 & Solution & 120.11 & 230 & 214.00 &  6.96\\
instance n=1000 376.alb & 1 & 0 & Solution & 120.09 & 134 & 132.00 &  1.49\\
instance n=1000 377.alb & 1 & 0 & Solution & 120.09 & 138 & 137.00 &  0.72\\
instance n=1000 378.alb & 1 & 0 & Solution & 120.10 & 135 & 134.00 &  0.74\\
instance n=1000 379.alb & 1 & 0 & Solution & 120.11 & 139 & 137.00 &  1.44\\
instance n=1000 38.alb & 1 & 0 & Solution & 120.12 & 564 & 309.00 & 45.21\\
instance n=1000 380.alb & 1 & 0 & Solution & 120.11 & 136 & 134.00 &  1.47\\
instance n=1000 381.alb & 1 & 0 & Solution & 120.10 & 140 & 138.00 &  1.43\\
instance n=1000 382.alb & 1 & 0 & Solution & 120.09 & 132 & 131.00 &  0.76\\
instance n=1000 383.alb & 1 & 0 & Solution & 120.09 & 140 & 138.00 &  1.43\\
instance n=1000 384.alb & 1 & 0 & Solution & 120.10 & 141 & 139.00 &  1.42\\
instance n=1000 385.alb & 1 & 0 & Solution & 120.10 & 137 & 135.00 &  1.46\\
instance n=1000 386.alb & 1 & 0 & Solution & 120.12 & 141 & 139.00 &  1.42\\
instance n=1000 387.alb & 1 & 0 & Solution & 120.11 & 139 & 137.00 &  1.44\\
instance n=1000 388.alb & 1 & 0 & Solution & 120.10 & 138 & 137.00 &  0.72\\
instance n=1000 389.alb & 1 & 0 & Solution & 120.11 & 137 & 136.00 &  0.73\\
instance n=1000 39.alb & 1 & 0 & Solution & 120.12 & 565 & 313.00 & 44.60\\
instance n=1000 390.alb & 1 & 0 & Solution & 120.11 & 137 & 136.00 &  0.73\\
instance n=1000 391.alb & 1 & 0 & Solution & 120.11 & 137 & 135.00 &  1.46\\
instance n=1000 392.alb & 1 & 0 & Solution & 120.11 & 137 & 136.00 &  0.73\\
instance n=1000 393.alb & 1 & 0 & Solution & 120.10 & 137 & 136.00 &  0.73\\
instance n=1000 394.alb & 1 & 0 & Solution & 120.10 & 140 & 138.00 &  1.43\\
instance n=1000 395.alb & 1 & 0 & Solution & 120.10 & 141 & 139.00 &  1.42\\
instance n=1000 396.alb & 1 & 0 & Solution & 120.10 & 138 & 136.00 &  1.45\\
instance n=1000 397.alb & 1 & 0 & Solution & 120.13 & 142 & 140.00 &  1.41\\
instance n=1000 398.alb & 1 & 0 & Solution & 120.11 & 136 & 134.00 &  1.47\\
instance n=1000 399.alb & 1 & 0 & Solution & 120.11 & 141 & 139.00 &  1.42\\
instance n=1000 4.alb & 1 & 0 & Solution & 120.10 & 139 & 138.00 &  0.72\\
instance n=1000 40.alb & 1 & 0 & Solution & 120.11 & 529 & 318.00 & 39.89\\
instance n=1000 400.alb & 1 & 0 & Solution & 120.11 & 142 & 140.00 &  1.41\\
instance n=1000 401.alb & 1 & 0 & Solution & 120.11 & 553 & 413.00 & 25.32\\
instance n=1000 402.alb & 1 & 0 & Solution & 120.19 & 556 & 424.00 & 23.74\\
instance n=1000 403.alb & 1 & 0 & Solution & 120.18 & 557 & 396.00 & 28.90\\
instance n=1000 404.alb & 1 & 0 & Solution & 120.18 & 554 & 430.00 & 22.38\\
instance n=1000 405.alb & 1 & 0 & Solution & 120.17 & 562 & 458.00 & 18.51\\
instance n=1000 406.alb & 1 & 0 & Solution & 120.16 & 547 & 403.00 & 26.33\\
instance n=1000 407.alb & 1 & 0 & Solution & 120.19 & 555 & 399.00 & 28.11\\
instance n=1000 408.alb & 1 & 0 & Solution & 120.18 & 563 & 412.00 & 26.82\\
instance n=1000 409.alb & 1 & 0 & Solution & 120.16 & 566 & 413.00 & 27.03\\
instance n=1000 41.alb & 1 & 0 & Solution & 120.13 & 555 & 336.00 & 39.46\\
instance n=1000 410.alb & 1 & 0 & Solution & 120.18 & 575 & 431.00 & 25.04\\
instance n=1000 411.alb & 1 & 0 & Solution & 120.22 & 558 & 422.00 & 24.37\\
instance n=1000 412.alb & 1 & 0 & Solution & 120.15 & 558 & 393.00 & 29.57\\
instance n=1000 413.alb & 1 & 0 & Solution & 120.19 & 558 & 411.00 & 26.34\\
instance n=1000 414.alb & 1 & 0 & Solution & 120.18 & 562 & 406.00 & 27.76\\
instance n=1000 415.alb & 1 & 0 & Solution & 120.17 & 561 & 413.00 & 26.38\\
instance n=1000 416.alb & 1 & 0 & Solution & 120.19 & 562 & 398.00 & 29.18\\
instance n=1000 417.alb & 1 & 0 & Solution & 120.12 & 594 & 406.00 & 31.65\\
instance n=1000 418.alb & 1 & 0 & Solution & 120.16 & 552 & 438.00 & 20.65\\
instance n=1000 419.alb & 1 & 0 & Solution & 120.19 & 577 & 423.00 & 26.69\\
instance n=1000 42.alb & 1 & 0 & Solution & 120.12 & 534 & 306.00 & 42.70\\
instance n=1000 420.alb & 1 & 0 & Solution & 120.16 & 556 & 429.00 & 22.84\\
instance n=1000 421.alb & 1 & 0 & Solution & 120.18 & 556 & 402.00 & 27.70\\
instance n=1000 422.alb & 1 & 0 & Solution & 120.11 & 552 & 420.00 & 23.91\\
instance n=1000 423.alb & 1 & 0 & Solution & 120.11 & 561 & 396.00 & 29.41\\
instance n=1000 424.alb & 1 & 0 & Solution & 120.10 & 548 & 431.00 & 21.35\\
instance n=1000 425.alb & 1 & 0 & Solution & 120.19 & 567 & 395.00 & 30.34\\
instance n=1000 426.alb & 1 & 0 & Solution & 120.10 & 229 & 224.00 &  2.18\\
instance n=1000 427.alb & 1 & 0 & Solution & 120.12 & 234 & 229.00 &  2.14\\
instance n=1000 428.alb & 1 & 0 & Solution & 120.12 & 228 & 224.00 &  1.75\\
instance n=1000 429.alb & 1 & 0 & Solution & 120.11 & 239 & 235.00 &  1.67\\
instance n=1000 43.alb & 1 & 0 & Solution & 120.13 & 541 & 325.00 & 39.93\\
instance n=1000 430.alb & 1 & 0 & Solution & 120.10 & 224 & 220.00 &  1.79\\
instance n=1000 431.alb & 1 & 0 & Solution & 120.13 & 234 & 230.00 &  1.71\\
instance n=1000 432.alb & 1 & 0 & Solution & 120.13 & 232 & 227.00 &  2.16\\
instance n=1000 433.alb & 1 & 0 & Solution & 120.10 & 234 & 229.00 &  2.14\\
instance n=1000 434.alb & 1 & 0 & Solution & 120.10 & 215 & 212.00 &  1.40\\
instance n=1000 435.alb & 1 & 0 & Solution & 120.10 & 231 & 227.00 &  1.73\\
instance n=1000 436.alb & 1 & 0 & Solution & 120.12 & 231 & 226.00 &  2.16\\
instance n=1000 437.alb & 1 & 0 & Solution & 120.12 & 226 & 222.00 &  1.77\\
instance n=1000 438.alb & 1 & 0 & Solution & 120.14 & 225 & 221.00 &  1.78\\
instance n=1000 439.alb & 1 & 0 & Solution & 120.13 & 230 & 225.00 &  2.17\\
instance n=1000 44.alb & 1 & 0 & Solution & 120.12 & 554 & 313.00 & 43.50\\
instance n=1000 440.alb & 1 & 0 & Solution & 120.12 & 230 & 225.00 &  2.17\\
instance n=1000 441.alb & 1 & 0 & Solution & 120.11 & 226 & 221.00 &  2.21\\
instance n=1000 442.alb & 1 & 0 & Solution & 120.12 & 235 & 230.00 &  2.13\\
instance n=1000 443.alb & 1 & 0 & Solution & 120.11 & 221 & 217.00 &  1.81\\
instance n=1000 444.alb & 1 & 0 & Solution & 120.12 & 227 & 222.00 &  2.20\\
instance n=1000 445.alb & 1 & 0 & Solution & 120.10 & 235 & 229.00 &  2.55\\
instance n=1000 446.alb & 1 & 0 & Solution & 120.12 & 232 & 228.00 &  1.72\\
instance n=1000 447.alb & 1 & 0 & Solution & 120.11 & 226 & 221.00 &  2.21\\
instance n=1000 448.alb & 1 & 0 & Solution & 120.11 & 226 & 222.00 &  1.77\\
instance n=1000 449.alb & 1 & 0 & Solution & 120.11 & 238 & 232.00 &  2.52\\
instance n=1000 45.alb & 1 & 0 & Solution & 120.14 & 534 & 318.00 & 40.45\\
instance n=1000 450.alb & 1 & 0 & Solution & 120.10 & 224 & 220.00 &  1.79\\
instance n=1000 451.alb & 1 & 0 & Solution & 120.12 & 139 & 136.00 &  2.16\\
instance n=1000 452.alb & 1 & 0 & Solution & 120.09 & 134 & 132.00 &  1.49\\
instance n=1000 453.alb & 1 & 0 & Solution & 120.10 & 140 & 138.00 &  1.43\\
instance n=1000 454.alb & 1 & 0 & Solution & 120.11 & 142 & 139.00 &  2.11\\
instance n=1000 455.alb & 1 & 0 & Solution & 120.09 & 139 & 136.00 &  2.16\\
instance n=1000 456.alb & 1 & 0 & Solution & 120.10 & 137 & 135.00 &  1.46\\
instance n=1000 457.alb & 1 & 0 & Solution & 120.10 & 139 & 137.00 &  1.44\\
instance n=1000 458.alb & 1 & 0 & Solution & 120.10 & 137 & 135.00 &  1.46\\
instance n=1000 459.alb & 1 & 0 & Solution & 120.10 & 140 & 137.00 &  2.14\\
instance n=1000 46.alb & 1 & 0 & Solution & 120.09 & 545 & 314.00 & 42.39\\
instance n=1000 460.alb & 1 & 0 & Solution & 120.12 & 140 & 138.00 &  1.43\\
instance n=1000 461.alb & 1 & 0 & Solution & 120.11 & 139 & 137.00 &  1.44\\
instance n=1000 462.alb & 1 & 0 & Solution & 120.10 & 138 & 136.00 &  1.45\\
instance n=1000 463.alb & 1 & 0 & Solution & 120.10 & 138 & 136.00 &  1.45\\
instance n=1000 464.alb & 1 & 0 & Solution & 120.10 & 141 & 138.00 &  2.13\\
instance n=1000 465.alb & 1 & 0 & Solution & 120.10 & 141 & 138.00 &  2.13\\
instance n=1000 466.alb & 1 & 0 & Solution & 120.09 & 136 & 133.00 &  2.21\\
instance n=1000 467.alb & 1 & 0 & Solution & 120.12 & 140 & 138.00 &  1.43\\
instance n=1000 468.alb & 1 & 0 & Solution & 120.10 & 139 & 137.00 &  1.44\\
instance n=1000 469.alb & 1 & 0 & Solution & 120.11 & 139 & 137.00 &  1.44\\
instance n=1000 47.alb & 1 & 0 & Solution & 120.08 & 547 & 303.00 & 44.61\\
instance n=1000 470.alb & 1 & 0 & Solution & 120.11 & 137 & 135.00 &  1.46\\
instance n=1000 471.alb & 1 & 0 & Solution & 120.10 & 138 & 135.00 &  2.17\\
instance n=1000 472.alb & 1 & 0 & Solution & 120.10 & 142 & 140.00 &  1.41\\
instance n=1000 473.alb & 1 & 0 & Solution & 120.10 & 138 & 135.00 &  2.17\\
instance n=1000 474.alb & 1 & 0 & Solution & 120.08 & 139 & 136.00 &  2.16\\
instance n=1000 475.alb & 1 & 0 & Solution & 120.10 & 138 & 136.00 &  1.45\\
instance n=1000 476.alb & 1 & 0 & Solution & 120.13 & 575 & 494.00 & 14.09\\
instance n=1000 477.alb & 1 & 0 & Solution & 120.13 & 585 & 524.00 & 10.43\\
instance n=1000 478.alb & 1 & 0 & Solution & 120.14 & 594 & 545.00 &  8.25\\
instance n=1000 479.alb & 1 & 0 & Solution & 120.16 & 577 & 490.00 & 15.08\\
instance n=1000 48.alb & 1 & 0 & Solution & 120.13 & 573 & 329.00 & 42.58\\
instance n=1000 480.alb & 1 & 0 & Solution & 120.12 & 566 & 507.00 & 10.42\\
instance n=1000 481.alb & 1 & 0 & Solution & 120.22 & 580 & 519.00 & 10.52\\
instance n=1000 482.alb & 1 & 0 & Solution & 120.20 & 603 & 498.00 & 17.41\\
instance n=1000 483.alb & 1 & 0 & Solution & 120.19 & 571 & 502.00 & 12.08\\
instance n=1000 484.alb & 1 & 0 & Solution & 120.26 & 588 & 512.00 & 12.93\\
instance n=1000 485.alb & 1 & 0 & Solution & 120.12 & 584 & 518.00 & 11.30\\
instance n=1000 486.alb & 1 & 0 & Solution & 120.12 & 575 & 504.00 & 12.35\\
instance n=1000 487.alb & 1 & 0 & Solution & 120.20 & 582 & 492.00 & 15.46\\
instance n=1000 488.alb & 1 & 0 & Solution & 120.13 & 575 & 511.00 & 11.13\\
instance n=1000 489.alb & 1 & 0 & Solution & 120.13 & 568 & 487.00 & 14.26\\
instance n=1000 49.alb & 1 & 0 & Solution & 120.11 & 546 & 323.00 & 40.84\\
instance n=1000 490.alb & 1 & 0 & Solution & 120.20 & 576 & 499.00 & 13.37\\
instance n=1000 491.alb & 1 & 0 & Solution & 120.23 & 571 & 495.00 & 13.31\\
instance n=1000 492.alb & 1 & 0 & Solution & 120.21 & 592 & 515.00 & 13.01\\
instance n=1000 493.alb & 1 & 0 & Solution & 120.12 & 564 & 498.00 & 11.70\\
instance n=1000 494.alb & 1 & 0 & Solution & 120.29 & 579 & 515.00 & 11.05\\
instance n=1000 495.alb & 1 & 0 & Solution & 120.24 & 595 & 508.00 & 14.62\\
instance n=1000 496.alb & 1 & 0 & Solution & 120.11 & 563 & 505.00 & 10.30\\
instance n=1000 497.alb & 1 & 0 & Solution & 120.22 & 569 & 499.00 & 12.30\\
instance n=1000 498.alb & 1 & 0 & Solution & 120.13 & 585 & 523.00 & 10.60\\
instance n=1000 499.alb & 1 & 0 & Solution & 120.12 & 567 & 505.00 & 10.93\\
instance n=1000 5.alb & 1 & 0 & Solution & 120.08 & 136 & 135.00 &  0.74\\
instance n=1000 50.alb & 1 & 0 & Solution & 120.10 & 535 & 303.00 & 43.36\\
instance n=1000 500.alb & 1 & 0 & Solution & 120.22 & 584 & 507.00 & 13.18\\
instance n=1000 501.alb & 1 & 0 & Solution & 120.10 & 233 & 227.00 &  2.58\\
instance n=1000 502.alb & 1 & 0 & Solution & 120.11 & 229 & 224.00 &  2.18\\
instance n=1000 503.alb & 1 & 0 & Solution & 120.16 & 232 & 225.00 &  3.02\\
instance n=1000 504.alb & 1 & 0 & Solution & 120.11 & 233 & 227.00 &  2.58\\
instance n=1000 505.alb & 1 & 0 & Solution & 120.11 & 219 & 213.00 &  2.74\\
instance n=1000 506.alb & 1 & 0 & Solution & 120.10 & 229 & 223.00 &  2.62\\
instance n=1000 507.alb & 1 & 0 & Solution & 120.11 & 226 & 220.00 &  2.65\\
instance n=1000 508.alb & 1 & 0 & Solution & 120.11 & 224 & 219.00 &  2.23\\
instance n=1000 509.alb & 1 & 0 & Solution & 120.15 & 231 & 225.00 &  2.60\\
instance n=1000 51.alb & 1 & 0 & Solution & 120.09 & 230 & 226.00 &  1.74\\
instance n=1000 510.alb & 1 & 0 & Solution & 120.11 & 233 & 226.00 &  3.00\\
instance n=1000 511.alb & 1 & 0 & Solution & 120.11 & 237 & 230.00 &  2.95\\
instance n=1000 512.alb & 1 & 0 & Solution & 120.11 & 224 & 219.00 &  2.23\\
instance n=1000 513.alb & 1 & 0 & Solution & 120.16 & 226 & 219.00 &  3.10\\
instance n=1000 514.alb & 1 & 0 & Solution & 120.13 & 233 & 226.00 &  3.00\\
instance n=1000 515.alb & 1 & 0 & Solution & 120.11 & 228 & 221.00 &  3.07\\
instance n=1000 516.alb & 1 & 0 & Solution & 120.11 & 235 & 229.00 &  2.55\\
instance n=1000 517.alb & 1 & 0 & Solution & 120.11 & 227 & 221.00 &  2.64\\
instance n=1000 518.alb & 1 & 0 & Solution & 120.12 & 226 & 220.00 &  2.65\\
instance n=1000 519.alb & 1 & 0 & Solution & 120.11 & 228 & 221.00 &  3.07\\
instance n=1000 52.alb & 1 & 0 & Solution & 120.15 & 232 & 197.00 & 15.09\\
instance n=1000 520.alb & 1 & 0 & Solution & 120.12 & 232 & 226.00 &  2.59\\
instance n=1000 521.alb & 1 & 0 & Solution & 120.14 & 236 & 229.00 &  2.97\\
instance n=1000 522.alb & 1 & 0 & Solution & 120.11 & 221 & 215.00 &  2.71\\
instance n=1000 523.alb & 1 & 0 & Solution & 120.12 & 226 & 220.00 &  2.65\\
instance n=1000 524.alb & 1 & 0 & Solution & 120.11 & 233 & 226.00 &  3.00\\
instance n=1000 525.alb & 1 & 0 & Solution & 120.15 & 227 & 221.00 &  2.64\\
instance n=1000 53.alb & 1 & 0 & Solution & 120.10 & 231 & 209.00 &  9.52\\
instance n=1000 54.alb & 1 & 0 & Solution & 120.14 & 223 & 203.00 &  8.97\\
instance n=1000 55.alb & 1 & 0 & Solution & 120.08 & 221 & 212.00 &  4.07\\
instance n=1000 56.alb & 1 & 0 & Solution & 120.09 & 231 & 198.00 & 14.29\\
instance n=1000 57.alb & 1 & 0 & Solution & 120.09 & 227 & 196.00 & 13.66\\
instance n=1000 58.alb & 1 & 0 & Solution & 120.08 & 227 & 200.00 & 11.89\\
instance n=1000 59.alb & 1 & 0 & Solution & 120.09 & 226 & 204.00 &  9.73\\
instance n=1000 6.alb & 1 & 0 & Solution & 120.08 & 143 & 141.00 &  1.40\\
instance n=1000 60.alb & 1 & 0 & Solution & 120.10 & 234 & 215.00 &  8.12\\
instance n=1000 61.alb & 1 & 0 & Solution & 120.09 & 233 & 196.00 & 15.88\\
instance n=1000 62.alb & 1 & 0 & Solution & 120.08 & 226 & 197.00 & 12.83\\
instance n=1000 63.alb & 1 & 0 & Solution & 120.08 & 230 & 197.00 & 14.35\\
instance n=1000 64.alb & 1 & 0 & Solution & 120.09 & 233 & 209.00 & 10.30\\
instance n=1000 65.alb & 1 & 0 & Solution & 120.11 & 228 & 210.00 &  7.89\\
instance n=1000 66.alb & 1 & 0 & Solution & 120.10 & 230 & 224.00 &  2.61\\
instance n=1000 67.alb & 1 & 0 & Solution & 120.10 & 227 & 192.00 & 15.42\\
instance n=1000 68.alb & 1 & 0 & Solution & 120.08 & 231 & 201.00 & 12.99\\
instance n=1000 69.alb & 1 & 0 & Solution & 120.10 & 227 & 206.00 &  9.25\\
instance n=1000 7.alb & 1 & 0 & Solution & 120.07 & 138 & 136.00 &  1.45\\
instance n=1000 70.alb & 1 & 0 & Solution & 120.11 & 232 & 203.00 & 12.50\\
instance n=1000 71.alb & 1 & 0 & Solution & 120.10 & 233 & 188.00 & 19.31\\
instance n=1000 72.alb & 1 & 0 & Solution & 120.08 & 226 & 206.00 &  8.85\\
instance n=1000 73.alb & 1 & 0 & Solution & 120.10 & 225 & 212.00 &  5.78\\
instance n=1000 74.alb & 1 & 0 & Solution & 120.08 & 231 & 218.00 &  5.63\\
instance n=1000 75.alb & 1 & 0 & Solution & 120.09 & 231 & 222.00 &  3.90\\
instance n=1000 76.alb & 1 & 0 & Solution & 120.07 & 138 & 136.00 &  1.45\\
instance n=1000 77.alb & 1 & 0 & Solution & 120.08 & 137 & 136.00 &  0.73\\
instance n=1000 78.alb & 1 & 0 & Solution & 120.07 & 140 & 138.00 &  1.43\\
instance n=1000 79.alb & 1 & 0 & Solution & 120.09 & 143 & 142.00 &  0.70\\
instance n=1000 8.alb & 1 & 0 & Solution & 120.08 & 140 & 138.00 &  1.43\\
instance n=1000 80.alb & 1 & 0 & Solution & 120.09 & 141 & 140.00 &  0.71\\
instance n=1000 81.alb & 1 & 0 & Solution & 120.08 & 137 & 136.00 &  0.73\\
instance n=1000 82.alb & 1 & 0 & Solution & 120.11 & 137 & 136.00 &  0.73\\
instance n=1000 83.alb & 1 & 0 & Solution & 120.92 & 141 & 140.00 &  0.71\\
instance n=1000 84.alb & 1 & 0 & Solution & 120.14 & 136 & 135.00 &  0.74\\
instance n=1000 85.alb & 1 & 0 & Solution & 120.10 & 137 & 136.00 &  0.73\\
instance n=1000 86.alb & 1 & 0 & Solution & 120.08 & 139 & 138.00 &  0.72\\
instance n=1000 87.alb & 1 & 0 & Solution & 120.09 & 142 & 140.00 &  1.41\\
instance n=1000 88.alb & 1 & 0 & Solution & 120.09 & 142 & 140.00 &  1.41\\
instance n=1000 89.alb & 1 & 0 & Solution & 120.10 & 141 & 140.00 &  0.71\\
instance n=1000 9.alb & 1 & 0 & Solution & 120.07 & 136 & 134.00 &  1.47\\
instance n=1000 90.alb & 1 & 0 & Solution & 120.10 & 139 & 138.00 &  0.72\\
instance n=1000 91.alb & 1 & 0 & Solution & 120.09 & 142 & 141.00 &  0.70\\
instance n=1000 92.alb & 1 & 0 & Solution & 120.10 & 137 & 136.00 &  0.73\\
instance n=1000 93.alb & 1 & 0 & Solution & 120.08 & 138 & 137.00 &  0.72\\
instance n=1000 94.alb & 1 & 0 & Solution & 120.09 & 139 & 137.00 &  1.44\\
instance n=1000 95.alb & 1 & 0 & Solution & 120.10 & 137 & 136.00 &  0.73\\
instance n=1000 96.alb & 1 & 0 & Solution & 120.07 & 139 & 137.00 &  1.44\\
instance n=1000 97.alb & 1 & 0 & Solution & 120.10 & 140 & 138.00 &  1.43\\
instance n=1000 98.alb & 1 & 0 & Solution & 120.12 & 137 & 136.00 &  0.73\\
instance n=1000 99.alb & 1 & 0 & Solution & 120.09 & 137 & 136.00 &  0.73\\
instance n=100 1.alb & 1 & 0 & Optimal & 17.05 & 23 & 23.00 &  0.00\\
instance n=100 10.alb & 1 & 0 & Optimal &  1.16 & 22 & 22.00 &  0.00\\
instance n=100 100.alb & 1 & 0 & Optimal & 120.03 & 25 & 25.00 &  0.00\\
instance n=100 101.alb & 1 & 0 & Optimal & 120.03 & 15 & 15.00 &  0.00\\
instance n=100 102.alb & 1 & 0 & Optimal &  0.39 & 14 & 14.00 &  0.00\\
instance n=100 103.alb & 1 & 0 & Optimal &  0.14 & 14 & 14.00 &  0.00\\
instance n=100 104.alb & 1 & 0 & Optimal &  0.09 & 14 & 14.00 &  0.00\\
instance n=100 105.alb & 1 & 0 & Optimal &  0.52 & 13 & 13.00 &  0.00\\
instance n=100 106.alb & 1 & 0 & Optimal &  0.15 & 14 & 14.00 &  0.00\\
instance n=100 107.alb & 1 & 0 & Optimal &  0.09 & 14 & 14.00 &  0.00\\
instance n=100 108.alb & 1 & 0 & Optimal & 120.02 & 14 & 14.00 &  0.00\\
instance n=100 109.alb & 1 & 0 & Optimal &  0.13 & 15 & 15.00 &  0.00\\
instance n=100 11.alb & 1 & 0 & Optimal & 12.16 & 24 & 24.00 &  0.00\\
instance n=100 110.alb & 1 & 0 & Optimal &  0.13 & 13 & 13.00 &  0.00\\
instance n=100 111.alb & 1 & 0 & Optimal &  0.11 & 16 & 16.00 &  0.00\\
instance n=100 112.alb & 1 & 0 & Optimal & 20.93 & 13 & 13.00 &  0.00\\
instance n=100 113.alb & 1 & 0 & Optimal &  0.39 & 14 & 14.00 &  0.00\\
instance n=100 114.alb & 1 & 0 & Optimal &  0.12 & 13 & 13.00 &  0.00\\
instance n=100 115.alb & 1 & 0 & Optimal & 120.01 & 14 & 14.00 &  0.00\\
instance n=100 116.alb & 1 & 0 & Optimal &  0.13 & 16 & 16.00 &  0.00\\
instance n=100 117.alb & 1 & 0 & Optimal & 120.03 & 15 & 15.00 &  0.00\\
instance n=100 118.alb & 1 & 0 & Optimal &  0.30 & 15 & 15.00 &  0.00\\
instance n=100 119.alb & 1 & 0 & Optimal &  0.11 & 14 & 14.00 &  0.00\\
instance n=100 12.alb & 1 & 0 & Optimal & 66.54 & 25 & 25.00 &  0.00\\
instance n=100 120.alb & 1 & 0 & Optimal &  0.12 & 14 & 14.00 &  0.00\\
instance n=100 121.alb & 1 & 0 & Optimal &  0.14 & 15 & 15.00 &  0.00\\
instance n=100 122.alb & 1 & 0 & Optimal &  0.26 & 13 & 13.00 &  0.00\\
instance n=100 123.alb & 1 & 0 & Optimal &  0.13 & 15 & 15.00 &  0.00\\
instance n=100 124.alb & 1 & 0 & Optimal & 120.01 & 15 & 15.00 &  0.00\\
instance n=100 125.alb & 1 & 0 & Optimal &  0.12 & 14 & 14.00 &  0.00\\
instance n=100 126.alb & 1 & 0 & Solution & 120.12 & 51 & 50.00 &  1.96\\
instance n=100 127.alb & 1 & 0 & Solution & 120.33 & 52 & 50.00 &  3.85\\
instance n=100 128.alb & 1 & 0 & Solution & 120.11 & 57 & 56.00 &  1.75\\
instance n=100 129.alb & 1 & 0 & Optimal &  1.96 & 54 & 54.00 &  0.00\\
instance n=100 13.alb & 1 & 0 & Optimal &  0.33 & 24 & 24.00 &  0.00\\
instance n=100 130.alb & 1 & 0 & Solution & 120.12 & 55 & 52.00 &  5.45\\
instance n=100 131.alb & 1 & 0 & Solution & 120.13 & 53 & 51.00 &  3.77\\
instance n=100 132.alb & 1 & 0 & Solution & 120.27 & 58 & 56.00 &  3.45\\
instance n=100 133.alb & 1 & 0 & Solution & 120.20 & 55 & 53.00 &  3.64\\
instance n=100 134.alb & 1 & 0 & Solution & 120.13 & 54 & 52.00 &  3.70\\
instance n=100 135.alb & 1 & 0 & Solution & 120.13 & 55 & 53.00 &  3.64\\
instance n=100 136.alb & 1 & 0 & Solution & 120.09 & 52 & 50.00 &  3.85\\
instance n=100 137.alb & 1 & 0 & Solution & 120.22 & 54 & 51.00 &  5.56\\
instance n=100 138.alb & 1 & 0 & Optimal &  9.65 & 56 & 56.00 &  0.00\\
instance n=100 139.alb & 1 & 0 & Optimal & 120.02 & 51 & 51.00 &  0.00\\
instance n=100 14.alb & 1 & 0 & Optimal & 120.03 & 20 & 20.00 &  0.00\\
instance n=100 140.alb & 1 & 0 & Solution & 120.45 & 55 & 54.00 &  1.82\\
instance n=100 141.alb & 1 & 0 & Solution & 120.24 & 51 & 49.00 &  3.92\\
instance n=100 142.alb & 1 & 0 & Solution & 120.15 & 55 & 52.00 &  5.45\\
instance n=100 143.alb & 1 & 0 & Solution & 120.10 & 53 & 51.00 &  3.77\\
instance n=100 144.alb & 1 & 0 & Solution & 120.09 & 49 & 47.00 &  4.08\\
instance n=100 145.alb & 1 & 0 & Solution & 120.26 & 56 & 53.00 &  5.36\\
instance n=100 146.alb & 1 & 0 & Optimal &  3.64 & 53 & 53.00 &  0.00\\
instance n=100 147.alb & 1 & 0 & Solution & 120.11 & 59 & 58.00 &  1.69\\
instance n=100 148.alb & 1 & 0 & Solution & 120.12 & 52 & 50.00 &  3.85\\
instance n=100 149.alb & 1 & 0 & Solution & 120.12 & 55 & 54.00 &  1.82\\
instance n=100 15.alb & 1 & 0 & Optimal &  0.08 & 24 & 24.00 &  0.00\\
instance n=100 150.alb & 1 & 0 & Solution & 120.13 & 57 & 54.00 &  5.26\\
instance n=100 151.alb & 1 & 0 & Solution & 120.10 & 22 & 21.00 &  4.55\\
instance n=100 152.alb & 1 & 0 & Optimal &  0.58 & 22 & 22.00 &  0.00\\
instance n=100 153.alb & 1 & 0 & Optimal & 120.02 & 21 & 21.00 &  0.00\\
instance n=100 154.alb & 1 & 0 & Optimal &  0.15 & 25 & 25.00 &  0.00\\
instance n=100 155.alb & 1 & 0 & Optimal &  0.78 & 22 & 22.00 &  0.00\\
instance n=100 156.alb & 1 & 0 & Optimal &  0.63 & 23 & 23.00 &  0.00\\
instance n=100 157.alb & 1 & 0 & Optimal &  0.79 & 26 & 26.00 &  0.00\\
instance n=100 158.alb & 1 & 0 & Optimal &  0.39 & 23 & 23.00 &  0.00\\
instance n=100 159.alb & 1 & 0 & Optimal &  0.14 & 19 & 19.00 &  0.00\\
instance n=100 16.alb & 1 & 0 & Optimal & 120.03 & 23 & 23.00 &  0.00\\
instance n=100 160.alb & 1 & 0 & Optimal &  0.68 & 22 & 22.00 &  0.00\\
instance n=100 161.alb & 1 & 0 & Solution & 120.11 & 23 & 22.00 &  4.35\\
instance n=100 162.alb & 1 & 0 & Optimal & 120.04 & 22 & 22.00 &  0.00\\
instance n=100 163.alb & 1 & 0 & Optimal &  0.15 & 25 & 25.00 &  0.00\\
instance n=100 164.alb & 1 & 0 & Optimal &  0.10 & 23 & 23.00 &  0.00\\
instance n=100 165.alb & 1 & 0 & Solution & 120.10 & 25 & 24.00 &  4.00\\
instance n=100 166.alb & 1 & 0 & Optimal &  0.59 & 24 & 24.00 &  0.00\\
instance n=100 167.alb & 1 & 0 & Optimal &  0.13 & 22 & 22.00 &  0.00\\
instance n=100 168.alb & 1 & 0 & Optimal & 120.03 & 21 & 21.00 &  0.00\\
instance n=100 169.alb & 1 & 0 & Optimal &  0.61 & 21 & 21.00 &  0.00\\
instance n=100 17.alb & 1 & 0 & Solution & 120.08 & 22 & 21.00 &  4.55\\
instance n=100 170.alb & 1 & 0 & Optimal & 24.40 & 24 & 24.00 &  0.00\\
instance n=100 171.alb & 1 & 0 & Solution & 120.12 & 25 & 24.00 &  4.00\\
instance n=100 172.alb & 1 & 0 & Optimal &  0.66 & 24 & 24.00 &  0.00\\
instance n=100 173.alb & 1 & 0 & Solution & 120.12 & 25 & 24.00 &  4.00\\
instance n=100 174.alb & 1 & 0 & Optimal & 120.03 & 22 & 22.00 &  0.00\\
instance n=100 175.alb & 1 & 0 & Solution & 120.11 & 27 & 26.00 &  3.70\\
instance n=100 176.alb & 1 & 0 & Optimal &  0.11 & 13 & 13.00 &  0.00\\
instance n=100 177.alb & 1 & 0 & Optimal & 120.03 & 14 & 14.00 &  0.00\\
instance n=100 178.alb & 1 & 0 & Optimal & 120.02 & 15 & 15.00 &  0.00\\
instance n=100 179.alb & 1 & 0 & Optimal &  0.12 & 15 & 15.00 &  0.00\\
instance n=100 18.alb & 1 & 0 & Solution & 120.09 & 20 & 19.00 &  5.00\\
instance n=100 180.alb & 1 & 0 & Optimal & 120.02 & 15 & 15.00 &  0.00\\
instance n=100 181.alb & 1 & 0 & Optimal & 120.03 & 13 & 13.00 &  0.00\\
instance n=100 182.alb & 1 & 0 & Optimal &  0.12 & 15 & 15.00 &  0.00\\
instance n=100 183.alb & 1 & 0 & Optimal & 120.01 & 14 & 14.00 &  0.00\\
instance n=100 184.alb & 1 & 0 & Optimal & 120.02 & 14 & 14.00 &  0.00\\
instance n=100 185.alb & 1 & 0 & Optimal & 26.08 & 15 & 15.00 &  0.00\\
instance n=100 186.alb & 1 & 0 & Optimal & 120.02 & 14 & 14.00 &  0.00\\
instance n=100 187.alb & 1 & 0 & Optimal & 66.69 & 13 & 13.00 &  0.00\\
instance n=100 188.alb & 1 & 0 & Optimal &  0.11 & 16 & 16.00 &  0.00\\
instance n=100 189.alb & 1 & 0 & Optimal &  6.38 & 14 & 14.00 &  0.00\\
instance n=100 19.alb & 1 & 0 & Optimal & 120.03 & 23 & 23.00 &  0.00\\
instance n=100 190.alb & 1 & 0 & Optimal & 120.02 & 13 & 13.00 &  0.00\\
instance n=100 191.alb & 1 & 0 & Optimal & 120.02 & 14 & 14.00 &  0.00\\
instance n=100 192.alb & 1 & 0 & Optimal & 120.02 & 13 & 13.00 &  0.00\\
instance n=100 193.alb & 1 & 0 & Optimal & 40.76 & 15 & 15.00 &  0.00\\
instance n=100 194.alb & 1 & 0 & Optimal &  0.23 & 15 & 15.00 &  0.00\\
instance n=100 195.alb & 1 & 0 & Optimal &  0.27 & 15 & 15.00 &  0.00\\
instance n=100 196.alb & 1 & 0 & Optimal & 120.02 & 15 & 15.00 &  0.00\\
instance n=100 197.alb & 1 & 0 & Optimal &  3.96 & 15 & 15.00 &  0.00\\
instance n=100 198.alb & 1 & 0 & Optimal & 120.03 & 13 & 13.00 &  0.00\\
instance n=100 199.alb & 1 & 0 & Optimal &  0.11 & 14 & 14.00 &  0.00\\
instance n=100 2.alb & 1 & 0 & Optimal & 120.04 & 21 & 21.00 &  0.00\\
instance n=100 20.alb & 1 & 0 & Optimal & 120.02 & 21 & 21.00 &  0.00\\
instance n=100 200.alb & 1 & 0 & Optimal & 43.68 & 15 & 15.00 &  0.00\\
instance n=100 201.alb & 1 & 0 & Solution & 120.11 & 53 & 52.00 &  1.89\\
instance n=100 202.alb & 1 & 0 & Optimal & 120.06 & 61 & 61.00 &  0.00\\
instance n=100 203.alb & 1 & 0 & Optimal & 120.04 & 52 & 52.00 &  0.00\\
instance n=100 204.alb & 1 & 0 & Solution & 120.33 & 51 & 49.00 &  3.92\\
instance n=100 205.alb & 1 & 0 & Solution & 120.11 & 57 & 56.00 &  1.75\\
instance n=100 206.alb & 1 & 0 & Solution & 120.17 & 52 & 50.00 &  3.85\\
instance n=100 207.alb & 1 & 0 & Solution & 120.12 & 51 & 50.00 &  1.96\\
instance n=100 208.alb & 1 & 0 & Solution & 120.10 & 57 & 56.00 &  1.75\\
instance n=100 209.alb & 1 & 0 & Solution & 120.16 & 55 & 54.00 &  1.82\\
instance n=100 21.alb & 1 & 0 & Optimal &  0.59 & 21 & 21.00 &  0.00\\
instance n=100 210.alb & 1 & 0 & Solution & 120.12 & 52 & 51.00 &  1.92\\
instance n=100 211.alb & 1 & 0 & Optimal & 46.45 & 51 & 51.00 &  0.00\\
instance n=100 212.alb & 1 & 0 & Solution & 120.12 & 52 & 51.00 &  1.92\\
instance n=100 213.alb & 1 & 0 & Solution & 120.13 & 52 & 51.00 &  1.92\\
instance n=100 214.alb & 1 & 0 & Solution & 120.48 & 55 & 53.00 &  3.64\\
instance n=100 215.alb & 1 & 0 & Solution & 120.43 & 50 & 48.00 &  4.00\\
instance n=100 216.alb & 1 & 0 & Solution & 120.12 & 52 & 51.00 &  1.92\\
instance n=100 217.alb & 1 & 0 & Solution & 120.27 & 52 & 51.00 &  1.92\\
instance n=100 218.alb & 1 & 0 & Solution & 120.11 & 53 & 52.00 &  1.89\\
instance n=100 219.alb & 1 & 0 & Solution & 120.14 & 52 & 51.00 &  1.92\\
instance n=100 22.alb & 1 & 0 & Solution & 120.09 & 25 & 24.00 &  4.00\\
instance n=100 220.alb & 1 & 0 & Solution & 120.11 & 53 & 52.00 &  1.89\\
instance n=100 221.alb & 1 & 0 & Solution & 120.12 & 57 & 56.00 &  1.75\\
instance n=100 222.alb & 1 & 0 & Solution & 120.29 & 53 & 51.00 &  3.77\\
instance n=100 223.alb & 1 & 0 & Solution & 120.13 & 51 & 50.00 &  1.96\\
instance n=100 224.alb & 1 & 0 & Optimal & 120.06 & 55 & 55.00 &  0.00\\
instance n=100 225.alb & 1 & 0 & Solution & 120.39 & 53 & 52.00 &  1.89\\
instance n=100 226.alb & 1 & 0 & Solution & 120.11 & 25 & 24.00 &  4.00\\
instance n=100 227.alb & 1 & 0 & Solution & 120.10 & 27 & 26.00 &  3.70\\
instance n=100 228.alb & 1 & 0 & Optimal &  4.63 & 22 & 22.00 &  0.00\\
instance n=100 229.alb & 1 & 0 & Optimal &  0.46 & 24 & 24.00 &  0.00\\
instance n=100 23.alb & 1 & 0 & Optimal &  0.11 & 24 & 24.00 &  0.00\\
instance n=100 230.alb & 1 & 0 & Optimal & 120.05 & 23 & 23.00 &  0.00\\
instance n=100 231.alb & 1 & 0 & Optimal &  0.83 & 22 & 22.00 &  0.00\\
instance n=100 232.alb & 1 & 0 & Optimal &  0.47 & 22 & 22.00 &  0.00\\
instance n=100 233.alb & 1 & 0 & Solution & 120.10 & 23 & 22.00 &  4.35\\
instance n=100 234.alb & 1 & 0 & Optimal &  0.48 & 23 & 23.00 &  0.00\\
instance n=100 235.alb & 1 & 0 & Optimal &  0.64 & 26 & 26.00 &  0.00\\
instance n=100 236.alb & 1 & 0 & Solution & 120.10 & 23 & 22.00 &  4.35\\
instance n=100 237.alb & 1 & 0 & Optimal &  0.47 & 23 & 23.00 &  0.00\\
instance n=100 238.alb & 1 & 0 & Optimal &  4.08 & 23 & 23.00 &  0.00\\
instance n=100 239.alb & 1 & 0 & Optimal &  0.24 & 21 & 21.00 &  0.00\\
instance n=100 24.alb & 1 & 0 & Optimal &  4.05 & 24 & 24.00 &  0.00\\
instance n=100 240.alb & 1 & 0 & Optimal &  2.53 & 22 & 22.00 &  0.00\\
instance n=100 241.alb & 1 & 0 & Optimal & 38.36 & 22 & 22.00 &  0.00\\
instance n=100 242.alb & 1 & 0 & Optimal &  2.22 & 23 & 23.00 &  0.00\\
instance n=100 243.alb & 1 & 0 & Solution & 120.09 & 24 & 23.00 &  4.17\\
instance n=100 244.alb & 1 & 0 & Optimal &  3.28 & 21 & 21.00 &  0.00\\
instance n=100 245.alb & 1 & 0 & Solution & 120.53 & 24 & 23.00 &  4.17\\
instance n=100 246.alb & 1 & 0 & Optimal &  1.04 & 26 & 26.00 &  0.00\\
instance n=100 247.alb & 1 & 0 & Optimal & 12.65 & 22 & 22.00 &  0.00\\
instance n=100 248.alb & 1 & 0 & Optimal & 34.41 & 19 & 19.00 &  0.00\\
instance n=100 249.alb & 1 & 0 & Optimal &  1.41 & 21 & 21.00 &  0.00\\
instance n=100 25.alb & 1 & 0 & Optimal & 120.03 & 22 & 22.00 &  0.00\\
instance n=100 250.alb & 1 & 0 & Optimal &  0.56 & 24 & 24.00 &  0.00\\
instance n=100 251.alb & 1 & 0 & Optimal &  0.11 & 15 & 15.00 &  0.00\\
instance n=100 252.alb & 1 & 0 & Optimal &  0.44 & 14 & 14.00 &  0.00\\
instance n=100 253.alb & 1 & 0 & Optimal &  0.10 & 14 & 14.00 &  0.00\\
instance n=100 254.alb & 1 & 0 & Optimal &  0.34 & 14 & 14.00 &  0.00\\
instance n=100 255.alb & 1 & 0 & Optimal &  0.14 & 14 & 14.00 &  0.00\\
instance n=100 256.alb & 1 & 0 & Optimal & 34.15 & 15 & 15.00 &  0.00\\
instance n=100 257.alb & 1 & 0 & Optimal & 120.03 & 12 & 12.00 &  0.00\\
instance n=100 258.alb & 1 & 0 & Optimal &  3.00 & 14 & 14.00 &  0.00\\
instance n=100 259.alb & 1 & 0 & Optimal &  0.57 & 15 & 15.00 &  0.00\\
instance n=100 26.alb & 1 & 0 & Optimal & 120.03 & 14 & 14.00 &  0.00\\
instance n=100 260.alb & 1 & 0 & Optimal &  5.78 & 15 & 15.00 &  0.00\\
instance n=100 261.alb & 1 & 0 & Optimal &  0.09 & 14 & 14.00 &  0.00\\
instance n=100 262.alb & 1 & 0 & Optimal &  0.03 & 14 & 14.00 &  0.00\\
instance n=100 263.alb & 1 & 0 & Optimal &  0.12 & 14 & 14.00 &  0.00\\
instance n=100 264.alb & 1 & 0 & Optimal &  2.65 & 15 & 15.00 &  0.00\\
instance n=100 265.alb & 1 & 0 & Optimal &  0.10 & 14 & 14.00 &  0.00\\
instance n=100 266.alb & 1 & 0 & Optimal &  0.44 & 13 & 13.00 &  0.00\\
instance n=100 267.alb & 1 & 0 & Optimal &  0.58 & 13 & 13.00 &  0.00\\
instance n=100 268.alb & 1 & 0 & Optimal &  0.09 & 15 & 15.00 &  0.00\\
instance n=100 269.alb & 1 & 0 & Optimal &  0.10 & 15 & 15.00 &  0.00\\
instance n=100 27.alb & 1 & 0 & Optimal & 120.02 & 13 & 13.00 &  0.00\\
instance n=100 270.alb & 1 & 0 & Optimal &  0.10 & 13 & 13.00 &  0.00\\
instance n=100 271.alb & 1 & 0 & Optimal & 120.02 & 13 & 13.00 &  0.00\\
instance n=100 272.alb & 1 & 0 & Optimal &  0.11 & 14 & 14.00 &  0.00\\
instance n=100 273.alb & 1 & 0 & Optimal & 11.72 & 13 & 13.00 &  0.00\\
instance n=100 274.alb & 1 & 0 & Optimal &  2.00 & 13 & 13.00 &  0.00\\
instance n=100 275.alb & 1 & 0 & Optimal &  0.09 & 13 & 13.00 &  0.00\\
instance n=100 276.alb & 1 & 0 & Solution & 120.12 & 60 & 58.00 &  3.33\\
instance n=100 277.alb & 1 & 0 & Solution & 120.14 & 57 & 54.00 &  5.26\\
instance n=100 278.alb & 1 & 0 & Solution & 120.13 & 57 & 55.00 &  3.51\\
instance n=100 279.alb & 1 & 0 & Solution & 120.11 & 53 & 52.00 &  1.89\\
instance n=100 28.alb & 1 & 0 & Optimal &  0.44 & 14 & 14.00 &  0.00\\
instance n=100 280.alb & 1 & 0 & Solution & 120.11 & 55 & 52.00 &  5.45\\
instance n=100 281.alb & 1 & 0 & Solution & 121.37 & 62 & 60.00 &  3.23\\
instance n=100 282.alb & 1 & 0 & Solution & 120.14 & 60 & 57.00 &  5.00\\
instance n=100 283.alb & 1 & 0 & Solution & 120.14 & 55 & 53.00 &  3.64\\
instance n=100 284.alb & 1 & 0 & Solution & 120.12 & 55 & 54.00 &  1.82\\
instance n=100 285.alb & 1 & 0 & Solution & 120.13 & 55 & 52.00 &  5.45\\
instance n=100 286.alb & 1 & 0 & Solution & 120.16 & 56 & 55.00 &  1.79\\
instance n=100 287.alb & 1 & 0 & Optimal &  9.49 & 54 & 54.00 &  0.00\\
instance n=100 288.alb & 1 & 0 & Solution & 120.35 & 56 & 53.00 &  5.36\\
instance n=100 289.alb & 1 & 0 & Optimal & 45.20 & 62 & 62.00 &  0.00\\
instance n=100 29.alb & 1 & 0 & Optimal & 74.27 & 14 & 14.00 &  0.00\\
instance n=100 290.alb & 1 & 0 & Solution & 120.12 & 54 & 52.00 &  3.70\\
instance n=100 291.alb & 1 & 0 & Solution & 120.11 & 52 & 49.00 &  5.77\\
instance n=100 292.alb & 1 & 0 & Solution & 120.12 & 57 & 55.00 &  3.51\\
instance n=100 293.alb & 1 & 0 & Solution & 120.14 & 52 & 50.00 &  3.85\\
instance n=100 294.alb & 1 & 0 & Solution & 120.12 & 57 & 54.00 &  5.26\\
instance n=100 295.alb & 1 & 0 & Solution & 120.12 & 56 & 55.00 &  1.79\\
instance n=100 296.alb & 1 & 0 & Solution & 120.11 & 55 & 53.00 &  3.64\\
instance n=100 297.alb & 1 & 0 & Optimal & 93.01 & 58 & 58.00 &  0.00\\
instance n=100 298.alb & 1 & 0 & Solution & 120.22 & 58 & 57.00 &  1.72\\
instance n=100 299.alb & 1 & 0 & Solution & 120.13 & 55 & 54.00 &  1.82\\
instance n=100 3.alb & 1 & 0 & Optimal &  0.33 & 20 & 20.00 &  0.00\\
instance n=100 30.alb & 1 & 0 & Optimal & 120.03 & 15 & 15.00 &  0.00\\
instance n=100 300.alb & 1 & 0 & Solution & 120.14 & 54 & 51.00 &  5.56\\
instance n=100 301.alb & 1 & 0 & Optimal & 120.03 & 23 & 23.00 &  0.00\\
instance n=100 302.alb & 1 & 0 & Optimal &  0.56 & 24 & 24.00 &  0.00\\
instance n=100 303.alb & 1 & 0 & Optimal & 120.04 & 24 & 24.00 &  0.00\\
instance n=100 304.alb & 1 & 0 & Optimal &  0.48 & 21 & 21.00 &  0.00\\
instance n=100 305.alb & 1 & 0 & Optimal & 63.57 & 22 & 22.00 &  0.00\\
instance n=100 306.alb & 1 & 0 & Optimal &  1.61 & 24 & 24.00 &  0.00\\
instance n=100 307.alb & 1 & 0 & Solution & 120.09 & 24 & 23.00 &  4.17\\
instance n=100 308.alb & 1 & 0 & Optimal & 120.07 & 20 & 20.00 &  0.00\\
instance n=100 309.alb & 1 & 0 & Solution & 120.10 & 22 & 21.00 &  4.55\\
instance n=100 31.alb & 1 & 0 & Optimal &  0.10 & 14 & 14.00 &  0.00\\
instance n=100 310.alb & 1 & 0 & Optimal &  0.15 & 23 & 23.00 &  0.00\\
instance n=100 311.alb & 1 & 0 & Optimal &  2.57 & 21 & 21.00 &  0.00\\
instance n=100 312.alb & 1 & 0 & Optimal & 120.03 & 22 & 22.00 &  0.00\\
instance n=100 313.alb & 1 & 0 & Optimal & 27.59 & 23 & 23.00 &  0.00\\
instance n=100 314.alb & 1 & 0 & Optimal &  0.59 & 19 & 19.00 &  0.00\\
instance n=100 315.alb & 1 & 0 & Optimal & 120.03 & 22 & 22.00 &  0.00\\
instance n=100 316.alb & 1 & 0 & Optimal & 120.04 & 24 & 24.00 &  0.00\\
instance n=100 317.alb & 1 & 0 & Optimal &  0.26 & 26 & 26.00 &  0.00\\
instance n=100 318.alb & 1 & 0 & Optimal &  0.24 & 21 & 21.00 &  0.00\\
instance n=100 319.alb & 1 & 0 & Optimal &  0.43 & 23 & 23.00 &  0.00\\
instance n=100 32.alb & 1 & 0 & Optimal & 120.01 & 14 & 14.00 &  0.00\\
instance n=100 320.alb & 1 & 0 & Optimal &  0.11 & 22 & 22.00 &  0.00\\
instance n=100 321.alb & 1 & 0 & Optimal &  3.31 & 26 & 26.00 &  0.00\\
instance n=100 322.alb & 1 & 0 & Solution & 120.11 & 24 & 23.00 &  4.17\\
instance n=100 323.alb & 1 & 0 & Optimal & 13.16 & 24 & 24.00 &  0.00\\
instance n=100 324.alb & 1 & 0 & Optimal &  0.11 & 23 & 23.00 &  0.00\\
instance n=100 325.alb & 1 & 0 & Optimal & 120.08 & 25 & 25.00 &  0.00\\
instance n=100 326.alb & 1 & 0 & Optimal & 120.01 & 13 & 13.00 &  0.00\\
instance n=100 327.alb & 1 & 0 & Optimal & 120.03 & 14 & 14.00 &  0.00\\
instance n=100 328.alb & 1 & 0 & Solution & 120.18 & 15 & 14.00 &  6.67\\
instance n=100 329.alb & 1 & 0 & Optimal & 120.02 & 14 & 14.00 &  0.00\\
instance n=100 33.alb & 1 & 0 & Optimal &  1.81 & 15 & 15.00 &  0.00\\
instance n=100 330.alb & 1 & 0 & Optimal & 39.16 & 14 & 14.00 &  0.00\\
instance n=100 331.alb & 1 & 0 & Optimal & 120.03 & 14 & 14.00 &  0.00\\
instance n=100 332.alb & 1 & 0 & Optimal & 120.01 & 14 & 14.00 &  0.00\\
instance n=100 333.alb & 1 & 0 & Optimal & 120.02 & 15 & 15.00 &  0.00\\
instance n=100 334.alb & 1 & 0 & Optimal & 120.03 & 14 & 14.00 &  0.00\\
instance n=100 335.alb & 1 & 0 & Optimal & 120.01 & 13 & 13.00 &  0.00\\
instance n=100 336.alb & 1 & 0 & Optimal & 120.03 & 15 & 15.00 &  0.00\\
instance n=100 337.alb & 1 & 0 & Optimal & 120.02 & 13 & 13.00 &  0.00\\
instance n=100 338.alb & 1 & 0 & Optimal & 120.03 & 14 & 14.00 &  0.00\\
instance n=100 339.alb & 1 & 0 & Optimal &  6.68 & 14 & 14.00 &  0.00\\
instance n=100 34.alb & 1 & 0 & Optimal & 29.63 & 15 & 15.00 &  0.00\\
instance n=100 340.alb & 1 & 0 & Optimal & 120.01 & 14 & 14.00 &  0.00\\
instance n=100 341.alb & 1 & 0 & Optimal & 120.01 & 16 & 16.00 &  0.00\\
instance n=100 342.alb & 1 & 0 & Optimal & 120.01 & 14 & 14.00 &  0.00\\
instance n=100 343.alb & 1 & 0 & Optimal & 120.03 & 16 & 16.00 &  0.00\\
instance n=100 344.alb & 1 & 0 & Optimal & 57.97 & 15 & 15.00 &  0.00\\
instance n=100 345.alb & 1 & 0 & Optimal & 120.02 & 14 & 14.00 &  0.00\\
instance n=100 346.alb & 1 & 0 & Optimal & 120.03 & 14 & 14.00 &  0.00\\
instance n=100 347.alb & 1 & 0 & Optimal & 120.02 & 14 & 14.00 &  0.00\\
instance n=100 348.alb & 1 & 0 & Optimal & 120.01 & 14 & 14.00 &  0.00\\
instance n=100 349.alb & 1 & 0 & Optimal & 120.02 & 13 & 13.00 &  0.00\\
instance n=100 35.alb & 1 & 0 & Optimal & 120.02 & 15 & 15.00 &  0.00\\
instance n=100 350.alb & 1 & 0 & Optimal & 27.41 & 14 & 14.00 &  0.00\\
instance n=100 351.alb & 1 & 0 & Solution & 120.11 & 59 & 58.00 &  1.69\\
instance n=100 352.alb & 1 & 0 & Optimal &  0.14 & 63 & 63.00 &  0.00\\
instance n=100 353.alb & 1 & 0 & Solution & 120.25 & 52 & 50.00 &  3.85\\
instance n=100 354.alb & 1 & 0 & Solution & 120.12 & 52 & 51.00 &  1.92\\
instance n=100 355.alb & 1 & 0 & Solution & 120.56 & 55 & 53.00 &  3.64\\
instance n=100 356.alb & 1 & 0 & Solution & 120.15 & 60 & 59.00 &  1.67\\
instance n=100 357.alb & 1 & 0 & Optimal & 120.07 & 53 & 53.00 &  0.00\\
instance n=100 358.alb & 1 & 0 & Solution & 120.12 & 52 & 51.00 &  1.92\\
instance n=100 359.alb & 1 & 0 & Solution & 120.11 & 53 & 52.00 &  1.89\\
instance n=100 36.alb & 1 & 0 & Optimal & 120.03 & 14 & 14.00 &  0.00\\
instance n=100 360.alb & 1 & 0 & Optimal & 46.00 & 54 & 54.00 &  0.00\\
instance n=100 361.alb & 1 & 0 & Solution & 120.12 & 52 & 50.00 &  3.85\\
instance n=100 362.alb & 1 & 0 & Optimal & 120.07 & 57 & 57.00 &  0.00\\
instance n=100 363.alb & 1 & 0 & Solution & 120.11 & 53 & 51.00 &  3.77\\
instance n=100 364.alb & 1 & 0 & Solution & 120.14 & 52 & 51.00 &  1.92\\
instance n=100 365.alb & 1 & 0 & Solution & 120.13 & 53 & 52.00 &  1.89\\
instance n=100 366.alb & 1 & 0 & Optimal & 120.03 & 61 & 61.00 &  0.00\\
instance n=100 367.alb & 1 & 0 & Optimal & 120.06 & 55 & 55.00 &  0.00\\
instance n=100 368.alb & 1 & 0 & Optimal & 120.05 & 58 & 58.00 &  0.00\\
instance n=100 369.alb & 1 & 0 & Solution & 120.12 & 51 & 50.00 &  1.96\\
instance n=100 37.alb & 1 & 0 & Optimal & 120.02 & 14 & 14.00 &  0.00\\
instance n=100 370.alb & 1 & 0 & Solution & 120.12 & 57 & 56.00 &  1.75\\
instance n=100 371.alb & 1 & 0 & Solution & 120.13 & 53 & 51.00 &  3.77\\
instance n=100 372.alb & 1 & 0 & Solution & 120.13 & 49 & 48.00 &  2.04\\
instance n=100 373.alb & 1 & 0 & Solution & 120.53 & 51 & 50.00 &  1.96\\
instance n=100 374.alb & 1 & 0 & Solution & 120.12 & 52 & 51.00 &  1.92\\
instance n=100 375.alb & 1 & 0 & Optimal & 120.03 & 57 & 57.00 &  0.00\\
instance n=100 376.alb & 1 & 0 & Optimal &  0.21 & 23 & 23.00 &  0.00\\
instance n=100 377.alb & 1 & 0 & Solution & 120.39 & 21 & 20.00 &  4.76\\
instance n=100 378.alb & 1 & 0 & Optimal & 13.14 & 22 & 22.00 &  0.00\\
instance n=100 379.alb & 1 & 0 & Optimal & 120.03 & 23 & 23.00 &  0.00\\
instance n=100 38.alb & 1 & 0 & Optimal & 120.02 & 14 & 14.00 &  0.00\\
instance n=100 380.alb & 1 & 0 & Solution & 120.10 & 23 & 22.00 &  4.35\\
instance n=100 381.alb & 1 & 0 & Optimal &  0.59 & 24 & 24.00 &  0.00\\
instance n=100 382.alb & 1 & 0 & Optimal & 120.03 & 25 & 25.00 &  0.00\\
instance n=100 383.alb & 1 & 0 & Optimal &  0.23 & 25 & 25.00 &  0.00\\
instance n=100 384.alb & 1 & 0 & Optimal &  1.10 & 25 & 25.00 &  0.00\\
instance n=100 385.alb & 1 & 0 & Optimal &  0.10 & 22 & 22.00 &  0.00\\
instance n=100 386.alb & 1 & 0 & Solution & 120.10 & 24 & 23.00 &  4.17\\
instance n=100 387.alb & 1 & 0 & Optimal &  0.11 & 22 & 22.00 &  0.00\\
instance n=100 388.alb & 1 & 0 & Optimal & 94.23 & 25 & 25.00 &  0.00\\
instance n=100 389.alb & 1 & 0 & Optimal &  1.34 & 23 & 23.00 &  0.00\\
instance n=100 39.alb & 1 & 0 & Optimal & 120.02 & 14 & 14.00 &  0.00\\
instance n=100 390.alb & 1 & 0 & Optimal & 26.72 & 22 & 22.00 &  0.00\\
instance n=100 391.alb & 1 & 0 & Optimal &  0.45 & 20 & 20.00 &  0.00\\
instance n=100 392.alb & 1 & 0 & Optimal &  0.12 & 22 & 22.00 &  0.00\\
instance n=100 393.alb & 1 & 0 & Solution & 120.10 & 24 & 23.00 &  4.17\\
instance n=100 394.alb & 1 & 0 & Optimal &  0.58 & 22 & 22.00 &  0.00\\
instance n=100 395.alb & 1 & 0 & Optimal &  2.06 & 24 & 24.00 &  0.00\\
instance n=100 396.alb & 1 & 0 & Optimal & 75.42 & 20 & 20.00 &  0.00\\
instance n=100 397.alb & 1 & 0 & Solution & 120.14 & 26 & 25.00 &  3.85\\
instance n=100 398.alb & 1 & 0 & Optimal & 120.04 & 25 & 25.00 &  0.00\\
instance n=100 399.alb & 1 & 0 & Optimal &  1.48 & 23 & 23.00 &  0.00\\
instance n=100 4.alb & 1 & 0 & Optimal &  1.52 & 24 & 24.00 &  0.00\\
instance n=100 40.alb & 1 & 0 & Optimal &  1.30 & 14 & 14.00 &  0.00\\
instance n=100 400.alb & 1 & 0 & Optimal &  0.56 & 24 & 24.00 &  0.00\\
instance n=100 401.alb & 1 & 0 & Optimal &  0.09 & 15 & 15.00 &  0.00\\
instance n=100 402.alb & 1 & 0 & Optimal &  0.57 & 15 & 15.00 &  0.00\\
instance n=100 403.alb & 1 & 0 & Optimal &  0.97 & 14 & 14.00 &  0.00\\
instance n=100 404.alb & 1 & 0 & Optimal &  0.09 & 15 & 15.00 &  0.00\\
instance n=100 405.alb & 1 & 0 & Optimal &  1.16 & 13 & 13.00 &  0.00\\
instance n=100 406.alb & 1 & 0 & Optimal &  0.10 & 14 & 14.00 &  0.00\\
instance n=100 407.alb & 1 & 0 & Optimal &  0.16 & 15 & 15.00 &  0.00\\
instance n=100 408.alb & 1 & 0 & Optimal & 120.03 & 14 & 14.00 &  0.00\\
instance n=100 409.alb & 1 & 0 & Optimal &  0.10 & 15 & 15.00 &  0.00\\
instance n=100 41.alb & 1 & 0 & Optimal & 120.03 & 13 & 13.00 &  0.00\\
instance n=100 410.alb & 1 & 0 & Optimal &  0.10 & 14 & 14.00 &  0.00\\
instance n=100 411.alb & 1 & 0 & Optimal &  5.46 & 14 & 14.00 &  0.00\\
instance n=100 412.alb & 1 & 0 & Optimal &  0.11 & 14 & 14.00 &  0.00\\
instance n=100 413.alb & 1 & 0 & Optimal &  0.29 & 14 & 14.00 &  0.00\\
instance n=100 414.alb & 1 & 0 & Optimal & 120.04 & 14 & 14.00 &  0.00\\
instance n=100 415.alb & 1 & 0 & Optimal & 15.44 & 13 & 13.00 &  0.00\\
instance n=100 416.alb & 1 & 0 & Optimal &  0.23 & 14 & 14.00 &  0.00\\
instance n=100 417.alb & 1 & 0 & Optimal &  0.11 & 15 & 15.00 &  0.00\\
instance n=100 418.alb & 1 & 0 & Optimal &  0.11 & 16 & 16.00 &  0.00\\
instance n=100 419.alb & 1 & 0 & Optimal &  1.55 & 14 & 14.00 &  0.00\\
instance n=100 42.alb & 1 & 0 & Optimal & 120.02 & 14 & 14.00 &  0.00\\
instance n=100 420.alb & 1 & 0 & Optimal &  0.11 & 14 & 14.00 &  0.00\\
instance n=100 421.alb & 1 & 0 & Optimal &  1.01 & 14 & 14.00 &  0.00\\
instance n=100 422.alb & 1 & 0 & Optimal &  0.10 & 15 & 15.00 &  0.00\\
instance n=100 423.alb & 1 & 0 & Optimal &  1.00 & 14 & 14.00 &  0.00\\
instance n=100 424.alb & 1 & 0 & Optimal &  0.12 & 14 & 14.00 &  0.00\\
instance n=100 425.alb & 1 & 0 & Optimal & 39.57 & 15 & 15.00 &  0.00\\
instance n=100 426.alb & 1 & 0 & Solution & 120.12 & 60 & 58.00 &  3.33\\
instance n=100 427.alb & 1 & 0 & Solution & 120.13 & 55 & 54.00 &  1.82\\
instance n=100 428.alb & 1 & 0 & Solution & 120.13 & 55 & 54.00 &  1.82\\
instance n=100 429.alb & 1 & 0 & Solution & 120.12 & 58 & 57.00 &  1.72\\
instance n=100 43.alb & 1 & 0 & Optimal & 120.03 & 14 & 14.00 &  0.00\\
instance n=100 430.alb & 1 & 0 & Solution & 120.13 & 53 & 52.00 &  1.89\\
instance n=100 431.alb & 1 & 0 & Solution & 120.13 & 54 & 52.00 &  3.70\\
instance n=100 432.alb & 1 & 0 & Solution & 120.11 & 56 & 54.00 &  3.57\\
instance n=100 433.alb & 1 & 0 & Optimal & 62.53 & 52 & 52.00 &  0.00\\
instance n=100 434.alb & 1 & 0 & Solution & 120.14 & 56 & 55.00 &  1.79\\
instance n=100 435.alb & 1 & 0 & Solution & 121.23 & 56 & 52.00 &  7.14\\
instance n=100 436.alb & 1 & 0 & Solution & 120.11 & 52 & 49.00 &  5.77\\
instance n=100 437.alb & 1 & 0 & Solution & 120.15 & 53 & 51.00 &  3.77\\
instance n=100 438.alb & 1 & 0 & Solution & 120.36 & 55 & 52.00 &  5.45\\
instance n=100 439.alb & 1 & 0 & Solution & 120.38 & 55 & 54.00 &  1.82\\
instance n=100 44.alb & 1 & 0 & Optimal &  0.09 & 14 & 14.00 &  0.00\\
instance n=100 440.alb & 1 & 0 & Solution & 120.24 & 53 & 51.00 &  3.77\\
instance n=100 441.alb & 1 & 0 & Solution & 120.13 & 52 & 51.00 &  1.92\\
instance n=100 442.alb & 1 & 0 & Solution & 120.11 & 52 & 49.00 &  5.77\\
instance n=100 443.alb & 1 & 0 & Solution & 120.14 & 55 & 53.00 &  3.64\\
instance n=100 444.alb & 1 & 0 & Solution & 120.14 & 54 & 50.00 &  7.41\\
instance n=100 445.alb & 1 & 0 & Solution & 120.13 & 55 & 54.00 &  1.82\\
instance n=100 446.alb & 1 & 0 & Solution & 120.15 & 57 & 54.00 &  5.26\\
instance n=100 447.alb & 1 & 0 & Solution & 120.39 & 54 & 52.00 &  3.70\\
instance n=100 448.alb & 1 & 0 & Solution & 120.19 & 55 & 54.00 &  1.82\\
instance n=100 449.alb & 1 & 0 & Solution & 120.15 & 55 & 52.00 &  5.45\\
instance n=100 45.alb & 1 & 0 & Optimal & 120.03 & 14 & 14.00 &  0.00\\
instance n=100 450.alb & 1 & 0 & Solution & 121.22 & 53 & 52.00 &  1.89\\
instance n=100 451.alb & 1 & 0 & Optimal &  0.19 & 26 & 26.00 &  0.00\\
instance n=100 452.alb & 1 & 0 & Optimal &  0.45 & 22 & 22.00 &  0.00\\
instance n=100 453.alb & 1 & 0 & Optimal &  0.40 & 24 & 24.00 &  0.00\\
instance n=100 454.alb & 1 & 0 & Optimal &  0.15 & 23 & 23.00 &  0.00\\
instance n=100 455.alb & 1 & 0 & Optimal &  0.51 & 23 & 23.00 &  0.00\\
instance n=100 456.alb & 1 & 0 & Optimal &  0.39 & 26 & 26.00 &  0.00\\
instance n=100 457.alb & 1 & 0 & Optimal &  0.38 & 23 & 23.00 &  0.00\\
instance n=100 458.alb & 1 & 0 & Optimal &  0.18 & 24 & 24.00 &  0.00\\
instance n=100 459.alb & 1 & 0 & Optimal &  0.31 & 23 & 23.00 &  0.00\\
instance n=100 46.alb & 1 & 0 & Optimal & 120.03 & 14 & 14.00 &  0.00\\
instance n=100 460.alb & 1 & 0 & Optimal &  0.16 & 23 & 23.00 &  0.00\\
instance n=100 461.alb & 1 & 0 & Optimal &  1.55 & 23 & 23.00 &  0.00\\
instance n=100 462.alb & 1 & 0 & Optimal &  0.32 & 23 & 23.00 &  0.00\\
instance n=100 463.alb & 1 & 0 & Optimal &  0.68 & 26 & 26.00 &  0.00\\
instance n=100 464.alb & 1 & 0 & Optimal &  0.14 & 25 & 25.00 &  0.00\\
instance n=100 465.alb & 1 & 0 & Optimal &  0.65 & 22 & 22.00 &  0.00\\
instance n=100 466.alb & 1 & 0 & Optimal &  0.40 & 26 & 26.00 &  0.00\\
instance n=100 467.alb & 1 & 0 & Optimal &  1.69 & 21 & 21.00 &  0.00\\
instance n=100 468.alb & 1 & 0 & Optimal &  0.58 & 25 & 25.00 &  0.00\\
instance n=100 469.alb & 1 & 0 & Optimal &  0.14 & 22 & 22.00 &  0.00\\
instance n=100 47.alb & 1 & 0 & Optimal & 120.03 & 14 & 14.00 &  0.00\\
instance n=100 470.alb & 1 & 0 & Optimal &  1.43 & 26 & 26.00 &  0.00\\
instance n=100 471.alb & 1 & 0 & Optimal &  0.55 & 26 & 26.00 &  0.00\\
instance n=100 472.alb & 1 & 0 & Optimal &  0.24 & 23 & 23.00 &  0.00\\
instance n=100 473.alb & 1 & 0 & Optimal &  0.48 & 28 & 28.00 &  0.00\\
instance n=100 474.alb & 1 & 0 & Optimal &  0.44 & 23 & 23.00 &  0.00\\
instance n=100 475.alb & 1 & 0 & Optimal &  1.21 & 24 & 24.00 &  0.00\\
instance n=100 476.alb & 1 & 0 & Optimal &  0.12 & 14 & 14.00 &  0.00\\
instance n=100 477.alb & 1 & 0 & Optimal &  0.11 & 14 & 14.00 &  0.00\\
instance n=100 478.alb & 1 & 0 & Optimal &  0.12 & 14 & 14.00 &  0.00\\
instance n=100 479.alb & 1 & 0 & Optimal &  0.30 & 16 & 16.00 &  0.00\\
instance n=100 48.alb & 1 & 0 & Optimal & 120.03 & 15 & 15.00 &  0.00\\
instance n=100 480.alb & 1 & 0 & Optimal &  0.09 & 15 & 15.00 &  0.00\\
instance n=100 481.alb & 1 & 0 & Optimal &  0.13 & 15 & 15.00 &  0.00\\
instance n=100 482.alb & 1 & 0 & Optimal &  0.46 & 15 & 15.00 &  0.00\\
instance n=100 483.alb & 1 & 0 & Optimal &  0.18 & 14 & 14.00 &  0.00\\
instance n=100 484.alb & 1 & 0 & Optimal &  0.12 & 14 & 14.00 &  0.00\\
instance n=100 485.alb & 1 & 0 & Optimal &  1.44 & 16 & 16.00 &  0.00\\
instance n=100 486.alb & 1 & 0 & Optimal &  0.09 & 15 & 15.00 &  0.00\\
instance n=100 487.alb & 1 & 0 & Optimal &  0.19 & 15 & 15.00 &  0.00\\
instance n=100 488.alb & 1 & 0 & Optimal &  0.41 & 16 & 16.00 &  0.00\\
instance n=100 489.alb & 1 & 0 & Optimal &  0.67 & 13 & 13.00 &  0.00\\
instance n=100 49.alb & 1 & 0 & Optimal & 120.01 & 14 & 14.00 &  0.00\\
instance n=100 490.alb & 1 & 0 & Optimal &  0.10 & 15 & 15.00 &  0.00\\
instance n=100 491.alb & 1 & 0 & Optimal &  1.63 & 16 & 16.00 &  0.00\\
instance n=100 492.alb & 1 & 0 & Optimal &  0.48 & 14 & 14.00 &  0.00\\
instance n=100 493.alb & 1 & 0 & Optimal &  0.34 & 14 & 14.00 &  0.00\\
instance n=100 494.alb & 1 & 0 & Optimal &  0.09 & 14 & 14.00 &  0.00\\
instance n=100 495.alb & 1 & 0 & Optimal &  0.09 & 15 & 15.00 &  0.00\\
instance n=100 496.alb & 1 & 0 & Optimal &  0.24 & 14 & 14.00 &  0.00\\
instance n=100 497.alb & 1 & 0 & Optimal &  0.11 & 13 & 13.00 &  0.00\\
instance n=100 498.alb & 1 & 0 & Optimal &  0.10 & 14 & 14.00 &  0.00\\
instance n=100 499.alb & 1 & 0 & Optimal &  0.13 & 14 & 14.00 &  0.00\\
instance n=100 5.alb & 1 & 0 & Optimal &  0.10 & 22 & 22.00 &  0.00\\
instance n=100 50.alb & 1 & 0 & Optimal & 120.02 & 14 & 14.00 &  0.00\\
instance n=100 500.alb & 1 & 0 & Optimal &  0.12 & 14 & 14.00 &  0.00\\
instance n=100 501.alb & 1 & 0 & Optimal &  1.75 & 62 & 62.00 &  0.00\\
instance n=100 502.alb & 1 & 0 & Optimal &  0.37 & 64 & 64.00 &  0.00\\
instance n=100 503.alb & 1 & 0 & Optimal &  1.27 & 60 & 60.00 &  0.00\\
instance n=100 504.alb & 1 & 0 & Optimal &  9.11 & 60 & 60.00 &  0.00\\
instance n=100 505.alb & 1 & 0 & Optimal &  0.40 & 61 & 61.00 &  0.00\\
instance n=100 506.alb & 1 & 0 & Optimal &  0.67 & 57 & 57.00 &  0.00\\
instance n=100 507.alb & 1 & 0 & Optimal &  4.43 & 59 & 59.00 &  0.00\\
instance n=100 508.alb & 1 & 0 & Optimal &  2.32 & 56 & 56.00 &  0.00\\
instance n=100 509.alb & 1 & 0 & Optimal &  0.98 & 57 & 57.00 &  0.00\\
instance n=100 51.alb & 1 & 0 & Solution & 120.12 & 50 & 49.00 &  2.00\\
instance n=100 510.alb & 1 & 0 & Optimal &  5.09 & 58 & 58.00 &  0.00\\
instance n=100 511.alb & 1 & 0 & Optimal &  3.50 & 59 & 59.00 &  0.00\\
instance n=100 512.alb & 1 & 0 & Optimal &  0.33 & 60 & 60.00 &  0.00\\
instance n=100 513.alb & 1 & 0 & Optimal &  6.71 & 62 & 62.00 &  0.00\\
instance n=100 514.alb & 1 & 0 & Optimal &  4.16 & 58 & 58.00 &  0.00\\
instance n=100 515.alb & 1 & 0 & Optimal &  5.58 & 61 & 61.00 &  0.00\\
instance n=100 516.alb & 1 & 0 & Optimal &  0.13 & 70 & 70.00 &  0.00\\
instance n=100 517.alb & 1 & 0 & Optimal &  1.97 & 62 & 62.00 &  0.00\\
instance n=100 518.alb & 1 & 0 & Optimal &  1.34 & 57 & 57.00 &  0.00\\
instance n=100 519.alb & 1 & 0 & Optimal &  0.83 & 61 & 61.00 &  0.00\\
instance n=100 52.alb & 1 & 0 & Solution & 120.29 & 53 & 52.00 &  1.89\\
instance n=100 520.alb & 1 & 0 & Optimal &  5.41 & 60 & 60.00 &  0.00\\
instance n=100 521.alb & 1 & 0 & Optimal &  0.94 & 70 & 70.00 &  0.00\\
instance n=100 522.alb & 1 & 0 & Optimal & 10.23 & 59 & 59.00 &  0.00\\
instance n=100 523.alb & 1 & 0 & Optimal &  4.62 & 55 & 55.00 &  0.00\\
instance n=100 524.alb & 1 & 0 & Optimal &  3.83 & 59 & 59.00 &  0.00\\
instance n=100 525.alb & 1 & 0 & Optimal &  5.42 & 62 & 62.00 &  0.00\\
instance n=100 53.alb & 1 & 0 & Optimal & 25.98 & 52 & 52.00 &  0.00\\
instance n=100 54.alb & 1 & 0 & Optimal & 120.05 & 51 & 51.00 &  0.00\\
instance n=100 55.alb & 1 & 0 & Solution & 120.10 & 53 & 52.00 &  1.89\\
instance n=100 56.alb & 1 & 0 & Solution & 120.12 & 52 & 51.00 &  1.92\\
instance n=100 57.alb & 1 & 0 & Solution & 120.11 & 54 & 53.00 &  1.85\\
instance n=100 58.alb & 1 & 0 & Solution & 120.11 & 57 & 56.00 &  1.75\\
instance n=100 59.alb & 1 & 0 & Optimal & 120.05 & 57 & 57.00 &  0.00\\
instance n=100 6.alb & 1 & 0 & Optimal & 120.04 & 22 & 22.00 &  0.00\\
instance n=100 60.alb & 1 & 0 & Solution & 120.18 & 54 & 53.00 &  1.85\\
instance n=100 61.alb & 1 & 0 & Solution & 120.09 & 55 & 54.00 &  1.82\\
instance n=100 62.alb & 1 & 0 & Solution & 120.17 & 52 & 50.00 &  3.85\\
instance n=100 63.alb & 1 & 0 & Optimal & 120.05 & 61 & 61.00 &  0.00\\
instance n=100 64.alb & 1 & 0 & Solution & 120.20 & 56 & 55.00 &  1.79\\
instance n=100 65.alb & 1 & 0 & Solution & 120.11 & 62 & 61.00 &  1.61\\
instance n=100 66.alb & 1 & 0 & Solution & 120.66 & 51 & 50.00 &  1.96\\
instance n=100 67.alb & 1 & 0 & Solution & 120.14 & 55 & 54.00 &  1.82\\
instance n=100 68.alb & 1 & 0 & Optimal &  0.23 & 57 & 57.00 &  0.00\\
instance n=100 69.alb & 1 & 0 & Optimal & 120.03 & 53 & 53.00 &  0.00\\
instance n=100 7.alb & 1 & 0 & Optimal &  6.66 & 26 & 26.00 &  0.00\\
instance n=100 70.alb & 1 & 0 & Solution & 120.11 & 53 & 51.00 &  3.77\\
instance n=100 71.alb & 1 & 0 & Solution & 120.11 & 53 & 52.00 &  1.89\\
instance n=100 72.alb & 1 & 0 & Solution & 120.16 & 53 & 52.00 &  1.89\\
instance n=100 73.alb & 1 & 0 & Solution & 120.12 & 56 & 55.00 &  1.79\\
instance n=100 74.alb & 1 & 0 & Solution & 120.12 & 51 & 50.00 &  1.96\\
instance n=100 75.alb & 1 & 0 & Optimal & 120.05 & 54 & 54.00 &  0.00\\
instance n=100 76.alb & 1 & 0 & Optimal &  0.10 & 23 & 23.00 &  0.00\\
instance n=100 77.alb & 1 & 0 & Optimal &  0.67 & 20 & 20.00 &  0.00\\
instance n=100 78.alb & 1 & 0 & Optimal &  3.46 & 21 & 21.00 &  0.00\\
instance n=100 79.alb & 1 & 0 & Optimal &  0.47 & 21 & 21.00 &  0.00\\
instance n=100 8.alb & 1 & 0 & Optimal &  0.40 & 24 & 24.00 &  0.00\\
instance n=100 80.alb & 1 & 0 & Optimal & 120.05 & 22 & 22.00 &  0.00\\
instance n=100 81.alb & 1 & 0 & Optimal & 46.05 & 20 & 20.00 &  0.00\\
instance n=100 82.alb & 1 & 0 & Optimal &  0.12 & 21 & 21.00 &  0.00\\
instance n=100 83.alb & 1 & 0 & Optimal & 35.51 & 22 & 22.00 &  0.00\\
instance n=100 84.alb & 1 & 0 & Solution & 120.07 & 27 & 26.00 &  3.70\\
instance n=100 85.alb & 1 & 0 & Solution & 120.06 & 25 & 24.00 &  4.00\\
instance n=100 86.alb & 1 & 0 & Optimal &  0.71 & 23 & 23.00 &  0.00\\
instance n=100 87.alb & 1 & 0 & Optimal &  0.54 & 22 & 22.00 &  0.00\\
instance n=100 88.alb & 1 & 0 & Solution & 120.08 & 24 & 23.00 &  4.17\\
instance n=100 89.alb & 1 & 0 & Optimal &  9.69 & 24 & 24.00 &  0.00\\
instance n=100 9.alb & 1 & 0 & Optimal & 23.46 & 23 & 23.00 &  0.00\\
instance n=100 90.alb & 1 & 0 & Solution & 120.06 & 21 & 20.00 &  4.76\\
instance n=100 91.alb & 1 & 0 & Optimal &  0.50 & 25 & 25.00 &  0.00\\
instance n=100 92.alb & 1 & 0 & Optimal &  0.12 & 24 & 24.00 &  0.00\\
instance n=100 93.alb & 1 & 0 & Optimal & 120.03 & 27 & 27.00 &  0.00\\
instance n=100 94.alb & 1 & 0 & Optimal & 120.04 & 22 & 22.00 &  0.00\\
instance n=100 95.alb & 1 & 0 & Optimal &  2.14 & 21 & 21.00 &  0.00\\
instance n=100 96.alb & 1 & 0 & Optimal & 120.02 & 21 & 21.00 &  0.00\\
instance n=100 97.alb & 1 & 0 & Optimal &  0.55 & 22 & 22.00 &  0.00\\
instance n=100 98.alb & 1 & 0 & Optimal & 15.98 & 22 & 22.00 &  0.00\\
instance n=100 99.alb & 1 & 0 & Optimal &  0.51 & 22 & 22.00 &  0.00\\
instance n=20 1.alb & 1 & 0 & Optimal &  0.02 & 3 &  3.00 &  0.00\\
instance n=20 10.alb & 1 & 0 & Optimal &  0.02 & 3 &  3.00 &  0.00\\
instance n=20 100.alb & 1 & 0 & Optimal &  0.03 & 11 & 11.00 &  0.00\\
instance n=20 101.alb & 1 & 0 & Optimal &  0.29 & 13 & 13.00 &  0.00\\
instance n=20 102.alb & 1 & 0 & Optimal &  0.12 & 13 & 13.00 &  0.00\\
instance n=20 103.alb & 1 & 0 & Optimal &  0.12 & 12 & 12.00 &  0.00\\
instance n=20 104.alb & 1 & 0 & Optimal &  0.01 & 11 & 11.00 &  0.00\\
instance n=20 105.alb & 1 & 0 & Optimal &  0.02 & 12 & 12.00 &  0.00\\
instance n=20 106.alb & 1 & 0 & Optimal &  0.13 & 10 & 10.00 &  0.00\\
instance n=20 107.alb & 1 & 0 & Optimal &  0.06 & 14 & 14.00 &  0.00\\
instance n=20 108.alb & 1 & 0 & Optimal &  0.02 & 15 & 15.00 &  0.00\\
instance n=20 109.alb & 1 & 0 & Optimal &  0.03 & 12 & 12.00 &  0.00\\
instance n=20 11.alb & 1 & 0 & Optimal &  0.02 & 3 &  3.00 &  0.00\\
instance n=20 110.alb & 1 & 0 & Optimal &  0.02 & 11 & 11.00 &  0.00\\
instance n=20 111.alb & 1 & 0 & Optimal &  0.04 & 13 & 13.00 &  0.00\\
instance n=20 112.alb & 1 & 0 & Optimal &  0.02 & 11 & 11.00 &  0.00\\
instance n=20 113.alb & 1 & 0 & Optimal &  0.02 & 12 & 12.00 &  0.00\\
instance n=20 114.alb & 1 & 0 & Optimal &  0.02 & 13 & 13.00 &  0.00\\
instance n=20 115.alb & 1 & 0 & Optimal &  0.02 & 11 & 11.00 &  0.00\\
instance n=20 116.alb & 1 & 0 & Optimal &  0.03 & 5 &  5.00 &  0.00\\
instance n=20 117.alb & 1 & 0 & Optimal &  0.02 & 5 &  5.00 &  0.00\\
instance n=20 118.alb & 1 & 0 & Optimal &  0.01 & 5 &  5.00 &  0.00\\
instance n=20 119.alb & 1 & 0 & Optimal &  0.01 & 6 &  6.00 &  0.00\\
instance n=20 12.alb & 1 & 0 & Optimal &  0.01 & 3 &  3.00 &  0.00\\
instance n=20 120.alb & 1 & 0 & Optimal &  0.03 & 6 &  6.00 &  0.00\\
instance n=20 121.alb & 1 & 0 & Optimal &  0.02 & 5 &  5.00 &  0.00\\
instance n=20 122.alb & 1 & 0 & Optimal &  0.01 & 6 &  6.00 &  0.00\\
instance n=20 123.alb & 1 & 0 & Optimal &  0.02 & 5 &  5.00 &  0.00\\
instance n=20 124.alb & 1 & 0 & Optimal &  0.02 & 5 &  5.00 &  0.00\\
instance n=20 125.alb & 1 & 0 & Optimal &  0.01 & 5 &  5.00 &  0.00\\
instance n=20 126.alb & 1 & 0 & Optimal &  0.02 & 5 &  5.00 &  0.00\\
instance n=20 127.alb & 1 & 0 & Optimal &  0.02 & 4 &  4.00 &  0.00\\
instance n=20 128.alb & 1 & 0 & Optimal &  0.02 & 5 &  5.00 &  0.00\\
instance n=20 129.alb & 1 & 0 & Optimal &  0.02 & 5 &  5.00 &  0.00\\
instance n=20 13.alb & 1 & 0 & Optimal &  0.01 & 3 &  3.00 &  0.00\\
instance n=20 130.alb & 1 & 0 & Optimal &  0.02 & 6 &  6.00 &  0.00\\
instance n=20 131.alb & 1 & 0 & Optimal &  0.02 & 7 &  7.00 &  0.00\\
instance n=20 132.alb & 1 & 0 & Optimal &  0.02 & 4 &  4.00 &  0.00\\
instance n=20 133.alb & 1 & 0 & Optimal &  0.01 & 5 &  5.00 &  0.00\\
instance n=20 134.alb & 1 & 0 & Optimal &  0.11 & 6 &  6.00 &  0.00\\
instance n=20 135.alb & 1 & 0 & Optimal &  0.11 & 6 &  6.00 &  0.00\\
instance n=20 136.alb & 1 & 0 & Optimal &  0.01 & 6 &  6.00 &  0.00\\
instance n=20 137.alb & 1 & 0 & Optimal &  0.02 & 5 &  5.00 &  0.00\\
instance n=20 138.alb & 1 & 0 & Optimal &  0.01 & 5 &  5.00 &  0.00\\
instance n=20 139.alb & 1 & 0 & Optimal &  0.02 & 5 &  5.00 &  0.00\\
instance n=20 14.alb & 1 & 0 & Optimal &  0.01 & 3 &  3.00 &  0.00\\
instance n=20 140.alb & 1 & 0 & Optimal &  0.01 & 5 &  5.00 &  0.00\\
instance n=20 141.alb & 1 & 0 & Optimal &  0.02 & 3 &  3.00 &  0.00\\
instance n=20 142.alb & 1 & 0 & Optimal &  0.01 & 3 &  3.00 &  0.00\\
instance n=20 143.alb & 1 & 0 & Optimal &  0.01 & 3 &  3.00 &  0.00\\
instance n=20 144.alb & 1 & 0 & Optimal &  0.01 & 4 &  4.00 &  0.00\\
instance n=20 145.alb & 1 & 0 & Optimal &  0.02 & 3 &  3.00 &  0.00\\
instance n=20 146.alb & 1 & 0 & Optimal &  0.02 & 3 &  3.00 &  0.00\\
instance n=20 147.alb & 1 & 0 & Optimal &  0.01 & 3 &  3.00 &  0.00\\
instance n=20 148.alb & 1 & 0 & Optimal &  0.02 & 3 &  3.00 &  0.00\\
instance n=20 149.alb & 1 & 0 & Optimal &  0.01 & 3 &  3.00 &  0.00\\
instance n=20 15.alb & 1 & 0 & Optimal &  0.02 & 3 &  3.00 &  0.00\\
instance n=20 150.alb & 1 & 0 & Optimal &  0.02 & 3 &  3.00 &  0.00\\
instance n=20 151.alb & 1 & 0 & Optimal &  0.02 & 3 &  3.00 &  0.00\\
instance n=20 152.alb & 1 & 0 & Optimal &  0.01 & 3 &  3.00 &  0.00\\
instance n=20 153.alb & 1 & 0 & Optimal &  0.01 & 3 &  3.00 &  0.00\\
instance n=20 154.alb & 1 & 0 & Optimal &  0.02 & 3 &  3.00 &  0.00\\
instance n=20 155.alb & 1 & 0 & Optimal &  0.02 & 3 &  3.00 &  0.00\\
instance n=20 156.alb & 1 & 0 & Optimal &  0.01 & 3 &  3.00 &  0.00\\
instance n=20 157.alb & 1 & 0 & Optimal &  0.02 & 3 &  3.00 &  0.00\\
instance n=20 158.alb & 1 & 0 & Optimal &  0.01 & 3 &  3.00 &  0.00\\
instance n=20 159.alb & 1 & 0 & Optimal &  0.01 & 3 &  3.00 &  0.00\\
instance n=20 16.alb & 1 & 0 & Optimal &  0.02 & 12 & 12.00 &  0.00\\
instance n=20 160.alb & 1 & 0 & Optimal &  0.02 & 3 &  3.00 &  0.00\\
instance n=20 161.alb & 1 & 0 & Optimal &  0.02 & 3 &  3.00 &  0.00\\
instance n=20 162.alb & 1 & 0 & Optimal &  0.01 & 3 &  3.00 &  0.00\\
instance n=20 163.alb & 1 & 0 & Optimal &  0.02 & 3 &  3.00 &  0.00\\
instance n=20 164.alb & 1 & 0 & Optimal &  0.11 & 4 &  4.00 &  0.00\\
instance n=20 165.alb & 1 & 0 & Optimal &  0.01 & 3 &  3.00 &  0.00\\
instance n=20 166.alb & 1 & 0 & Optimal &  0.13 & 12 & 12.00 &  0.00\\
instance n=20 167.alb & 1 & 0 & Optimal &  0.03 & 11 & 11.00 &  0.00\\
instance n=20 168.alb & 1 & 0 & Optimal &  0.02 & 10 & 10.00 &  0.00\\
instance n=20 169.alb & 1 & 0 & Optimal &  0.03 & 11 & 11.00 &  0.00\\
instance n=20 17.alb & 1 & 0 & Optimal &  0.03 & 10 & 10.00 &  0.00\\
instance n=20 170.alb & 1 & 0 & Optimal &  0.02 & 11 & 11.00 &  0.00\\
instance n=20 171.alb & 1 & 0 & Optimal &  0.16 & 13 & 13.00 &  0.00\\
instance n=20 172.alb & 1 & 0 & Optimal &  0.01 & 11 & 11.00 &  0.00\\
instance n=20 173.alb & 1 & 0 & Optimal &  0.05 & 11 & 11.00 &  0.00\\
instance n=20 174.alb & 1 & 0 & Optimal &  0.04 & 12 & 12.00 &  0.00\\
instance n=20 175.alb & 1 & 0 & Optimal &  0.11 & 10 & 10.00 &  0.00\\
instance n=20 176.alb & 1 & 0 & Optimal &  0.02 & 11 & 11.00 &  0.00\\
instance n=20 177.alb & 1 & 0 & Optimal &  0.36 & 10 & 10.00 &  0.00\\
instance n=20 178.alb & 1 & 0 & Optimal &  0.02 & 11 & 11.00 &  0.00\\
instance n=20 179.alb & 1 & 0 & Optimal &  0.01 & 11 & 11.00 &  0.00\\
instance n=20 18.alb & 1 & 0 & Optimal &  0.02 & 11 & 11.00 &  0.00\\
instance n=20 180.alb & 1 & 0 & Optimal &  0.02 & 13 & 13.00 &  0.00\\
instance n=20 181.alb & 1 & 0 & Optimal &  0.02 & 11 & 11.00 &  0.00\\
instance n=20 182.alb & 1 & 0 & Optimal &  0.02 & 11 & 11.00 &  0.00\\
instance n=20 183.alb & 1 & 0 & Optimal &  0.12 & 13 & 13.00 &  0.00\\
instance n=20 184.alb & 1 & 0 & Optimal &  0.01 & 12 & 12.00 &  0.00\\
instance n=20 185.alb & 1 & 0 & Optimal &  0.02 & 15 & 15.00 &  0.00\\
instance n=20 186.alb & 1 & 0 & Optimal &  0.82 & 14 & 14.00 &  0.00\\
instance n=20 187.alb & 1 & 0 & Optimal &  0.03 & 10 & 10.00 &  0.00\\
instance n=20 188.alb & 1 & 0 & Optimal &  0.04 & 11 & 11.00 &  0.00\\
instance n=20 189.alb & 1 & 0 & Optimal &  0.01 & 13 & 13.00 &  0.00\\
instance n=20 19.alb & 1 & 0 & Optimal &  0.05 & 14 & 14.00 &  0.00\\
instance n=20 190.alb & 1 & 0 & Optimal &  0.05 & 15 & 15.00 &  0.00\\
instance n=20 191.alb & 1 & 0 & Optimal &  0.01 & 4 &  4.00 &  0.00\\
instance n=20 192.alb & 1 & 0 & Optimal &  0.01 & 5 &  5.00 &  0.00\\
instance n=20 193.alb & 1 & 0 & Optimal &  0.01 & 5 &  5.00 &  0.00\\
instance n=20 194.alb & 1 & 0 & Optimal &  0.04 & 6 &  6.00 &  0.00\\
instance n=20 195.alb & 1 & 0 & Optimal &  0.02 & 6 &  6.00 &  0.00\\
instance n=20 196.alb & 1 & 0 & Optimal &  0.03 & 5 &  5.00 &  0.00\\
instance n=20 197.alb & 1 & 0 & Optimal &  0.02 & 4 &  4.00 &  0.00\\
instance n=20 198.alb & 1 & 0 & Optimal &  0.02 & 6 &  6.00 &  0.00\\
instance n=20 199.alb & 1 & 0 & Optimal &  0.10 & 5 &  5.00 &  0.00\\
instance n=20 2.alb & 1 & 0 & Optimal &  0.01 & 3 &  3.00 &  0.00\\
instance n=20 20.alb & 1 & 0 & Optimal &  0.03 & 11 & 11.00 &  0.00\\
instance n=20 200.alb & 1 & 0 & Optimal &  0.01 & 6 &  6.00 &  0.00\\
instance n=20 201.alb & 1 & 0 & Optimal &  0.02 & 6 &  6.00 &  0.00\\
instance n=20 202.alb & 1 & 0 & Optimal &  0.10 & 4 &  4.00 &  0.00\\
instance n=20 203.alb & 1 & 0 & Optimal &  0.02 & 4 &  4.00 &  0.00\\
instance n=20 204.alb & 1 & 0 & Optimal &  0.11 & 5 &  5.00 &  0.00\\
instance n=20 205.alb & 1 & 0 & Optimal &  0.02 & 6 &  6.00 &  0.00\\
instance n=20 206.alb & 1 & 0 & Optimal &  0.02 & 5 &  5.00 &  0.00\\
instance n=20 207.alb & 1 & 0 & Optimal &  0.06 & 6 &  6.00 &  0.00\\
instance n=20 208.alb & 1 & 0 & Optimal &  0.02 & 5 &  5.00 &  0.00\\
instance n=20 209.alb & 1 & 0 & Optimal &  0.03 & 4 &  4.00 &  0.00\\
instance n=20 21.alb & 1 & 0 & Optimal &  0.02 & 14 & 14.00 &  0.00\\
instance n=20 210.alb & 1 & 0 & Optimal &  0.01 & 5 &  5.00 &  0.00\\
instance n=20 211.alb & 1 & 0 & Optimal &  0.01 & 5 &  5.00 &  0.00\\
instance n=20 212.alb & 1 & 0 & Optimal &  0.01 & 5 &  5.00 &  0.00\\
instance n=20 213.alb & 1 & 0 & Optimal &  0.01 & 5 &  5.00 &  0.00\\
instance n=20 214.alb & 1 & 0 & Optimal &  0.02 & 5 &  5.00 &  0.00\\
instance n=20 215.alb & 1 & 0 & Optimal &  0.02 & 5 &  5.00 &  0.00\\
instance n=20 216.alb & 1 & 0 & Optimal &  0.01 & 3 &  3.00 &  0.00\\
instance n=20 217.alb & 1 & 0 & Optimal &  0.01 & 4 &  4.00 &  0.00\\
instance n=20 218.alb & 1 & 0 & Optimal &  0.01 & 3 &  3.00 &  0.00\\
instance n=20 219.alb & 1 & 0 & Optimal &  0.02 & 3 &  3.00 &  0.00\\
instance n=20 22.alb & 1 & 0 & Optimal &  0.02 & 12 & 12.00 &  0.00\\
instance n=20 220.alb & 1 & 0 & Optimal &  0.03 & 3 &  3.00 &  0.00\\
instance n=20 221.alb & 1 & 0 & Optimal &  0.01 & 3 &  3.00 &  0.00\\
instance n=20 222.alb & 1 & 0 & Optimal &  0.02 & 3 &  3.00 &  0.00\\
instance n=20 223.alb & 1 & 0 & Optimal &  0.01 & 3 &  3.00 &  0.00\\
instance n=20 224.alb & 1 & 0 & Optimal &  0.01 & 3 &  3.00 &  0.00\\
instance n=20 225.alb & 1 & 0 & Optimal &  0.02 & 3 &  3.00 &  0.00\\
instance n=20 226.alb & 1 & 0 & Optimal &  0.01 & 3 &  3.00 &  0.00\\
instance n=20 227.alb & 1 & 0 & Optimal &  0.01 & 3 &  3.00 &  0.00\\
instance n=20 228.alb & 1 & 0 & Optimal &  0.01 & 2 &  2.00 &  0.00\\
instance n=20 229.alb & 1 & 0 & Optimal &  0.02 & 3 &  3.00 &  0.00\\
instance n=20 23.alb & 1 & 0 & Optimal &  0.09 & 13 & 13.00 &  0.00\\
instance n=20 230.alb & 1 & 0 & Optimal &  0.02 & 3 &  3.00 &  0.00\\
instance n=20 231.alb & 1 & 0 & Optimal &  0.02 & 3 &  3.00 &  0.00\\
instance n=20 232.alb & 1 & 0 & Optimal &  0.01 & 3 &  3.00 &  0.00\\
instance n=20 233.alb & 1 & 0 & Optimal &  0.02 & 3 &  3.00 &  0.00\\
instance n=20 234.alb & 1 & 0 & Optimal &  0.01 & 3 &  3.00 &  0.00\\
instance n=20 235.alb & 1 & 0 & Optimal &  0.01 & 3 &  3.00 &  0.00\\
instance n=20 236.alb & 1 & 0 & Optimal &  0.02 & 3 &  3.00 &  0.00\\
instance n=20 237.alb & 1 & 0 & Optimal &  0.01 & 3 &  3.00 &  0.00\\
instance n=20 238.alb & 1 & 0 & Optimal &  0.01 & 3 &  3.00 &  0.00\\
instance n=20 239.alb & 1 & 0 & Optimal &  0.01 & 3 &  3.00 &  0.00\\
instance n=20 24.alb & 1 & 0 & Optimal &  0.02 & 11 & 11.00 &  0.00\\
instance n=20 240.alb & 1 & 0 & Optimal &  0.02 & 3 &  3.00 &  0.00\\
instance n=20 241.alb & 1 & 0 & Optimal &  0.10 & 13 & 13.00 &  0.00\\
instance n=20 242.alb & 1 & 0 & Optimal &  0.02 & 12 & 12.00 &  0.00\\
instance n=20 243.alb & 1 & 0 & Optimal &  0.11 & 10 & 10.00 &  0.00\\
instance n=20 244.alb & 1 & 0 & Optimal &  0.02 & 11 & 11.00 &  0.00\\
instance n=20 245.alb & 1 & 0 & Optimal &  0.02 & 13 & 13.00 &  0.00\\
instance n=20 246.alb & 1 & 0 & Optimal &  0.03 & 13 & 13.00 &  0.00\\
instance n=20 247.alb & 1 & 0 & Optimal &  0.12 & 11 & 11.00 &  0.00\\
instance n=20 248.alb & 1 & 0 & Optimal &  0.02 & 11 & 11.00 &  0.00\\
instance n=20 249.alb & 1 & 0 & Optimal &  0.02 & 13 & 13.00 &  0.00\\
instance n=20 25.alb & 1 & 0 & Optimal &  0.11 & 11 & 11.00 &  0.00\\
instance n=20 250.alb & 1 & 0 & Optimal &  0.02 & 10 & 10.00 &  0.00\\
instance n=20 251.alb & 1 & 0 & Optimal &  0.01 & 12 & 12.00 &  0.00\\
instance n=20 252.alb & 1 & 0 & Optimal &  0.03 & 11 & 11.00 &  0.00\\
instance n=20 253.alb & 1 & 0 & Optimal &  0.01 & 13 & 13.00 &  0.00\\
instance n=20 254.alb & 1 & 0 & Optimal &  0.03 & 12 & 12.00 &  0.00\\
instance n=20 255.alb & 1 & 0 & Optimal &  0.03 & 13 & 13.00 &  0.00\\
instance n=20 256.alb & 1 & 0 & Optimal &  0.02 & 14 & 14.00 &  0.00\\
instance n=20 257.alb & 1 & 0 & Optimal &  0.10 & 10 & 10.00 &  0.00\\
instance n=20 258.alb & 1 & 0 & Optimal &  0.02 & 13 & 13.00 &  0.00\\
instance n=20 259.alb & 1 & 0 & Optimal &  0.02 & 13 & 13.00 &  0.00\\
instance n=20 26.alb & 1 & 0 & Optimal &  0.02 & 12 & 12.00 &  0.00\\
instance n=20 260.alb & 1 & 0 & Optimal &  0.01 & 12 & 12.00 &  0.00\\
instance n=20 261.alb & 1 & 0 & Optimal &  0.03 & 12 & 12.00 &  0.00\\
instance n=20 262.alb & 1 & 0 & Optimal &  0.02 & 11 & 11.00 &  0.00\\
instance n=20 263.alb & 1 & 0 & Optimal &  0.03 & 12 & 12.00 &  0.00\\
instance n=20 264.alb & 1 & 0 & Optimal &  0.10 & 12 & 12.00 &  0.00\\
instance n=20 265.alb & 1 & 0 & Optimal &  0.02 & 12 & 12.00 &  0.00\\
instance n=20 266.alb & 1 & 0 & Optimal &  0.03 & 5 &  5.00 &  0.00\\
instance n=20 267.alb & 1 & 0 & Optimal &  0.01 & 6 &  6.00 &  0.00\\
instance n=20 268.alb & 1 & 0 & Optimal &  0.01 & 6 &  6.00 &  0.00\\
instance n=20 269.alb & 1 & 0 & Optimal &  0.10 & 7 &  7.00 &  0.00\\
instance n=20 27.alb & 1 & 0 & Optimal &  0.03 & 13 & 13.00 &  0.00\\
instance n=20 270.alb & 1 & 0 & Optimal &  0.10 & 7 &  7.00 &  0.00\\
instance n=20 271.alb & 1 & 0 & Optimal &  0.01 & 6 &  6.00 &  0.00\\
instance n=20 272.alb & 1 & 0 & Optimal &  0.01 & 5 &  5.00 &  0.00\\
instance n=20 273.alb & 1 & 0 & Optimal &  0.02 & 5 &  5.00 &  0.00\\
instance n=20 274.alb & 1 & 0 & Optimal &  0.09 & 6 &  6.00 &  0.00\\
instance n=20 275.alb & 1 & 0 & Optimal &  0.03 & 5 &  5.00 &  0.00\\
instance n=20 276.alb & 1 & 0 & Optimal &  0.01 & 4 &  4.00 &  0.00\\
instance n=20 277.alb & 1 & 0 & Optimal &  0.01 & 4 &  4.00 &  0.00\\
instance n=20 278.alb & 1 & 0 & Optimal &  0.09 & 6 &  6.00 &  0.00\\
instance n=20 279.alb & 1 & 0 & Optimal &  0.02 & 6 &  6.00 &  0.00\\
instance n=20 28.alb & 1 & 0 & Optimal &  0.02 & 12 & 12.00 &  0.00\\
instance n=20 280.alb & 1 & 0 & Optimal &  0.01 & 5 &  5.00 &  0.00\\
instance n=20 281.alb & 1 & 0 & Optimal &  0.02 & 4 &  4.00 &  0.00\\
instance n=20 282.alb & 1 & 0 & Optimal &  0.02 & 4 &  4.00 &  0.00\\
instance n=20 283.alb & 1 & 0 & Optimal &  0.01 & 5 &  5.00 &  0.00\\
instance n=20 284.alb & 1 & 0 & Optimal &  0.02 & 5 &  5.00 &  0.00\\
instance n=20 285.alb & 1 & 0 & Optimal &  0.01 & 5 &  5.00 &  0.00\\
instance n=20 286.alb & 1 & 0 & Optimal &  0.02 & 5 &  5.00 &  0.00\\
instance n=20 287.alb & 1 & 0 & Optimal &  0.02 & 5 &  5.00 &  0.00\\
instance n=20 288.alb & 1 & 0 & Optimal &  0.02 & 6 &  6.00 &  0.00\\
instance n=20 289.alb & 1 & 0 & Optimal &  0.01 & 5 &  5.00 &  0.00\\
instance n=20 29.alb & 1 & 0 & Optimal &  0.05 & 10 & 10.00 &  0.00\\
instance n=20 290.alb & 1 & 0 & Optimal &  0.01 & 5 &  5.00 &  0.00\\
instance n=20 291.alb & 1 & 0 & Optimal &  0.02 & 3 &  3.00 &  0.00\\
instance n=20 292.alb & 1 & 0 & Optimal &  0.03 & 3 &  3.00 &  0.00\\
instance n=20 293.alb & 1 & 0 & Optimal &  0.01 & 3 &  3.00 &  0.00\\
instance n=20 294.alb & 1 & 0 & Optimal &  0.03 & 3 &  3.00 &  0.00\\
instance n=20 295.alb & 1 & 0 & Optimal &  0.02 & 3 &  3.00 &  0.00\\
instance n=20 296.alb & 1 & 0 & Optimal &  0.01 & 3 &  3.00 &  0.00\\
instance n=20 297.alb & 1 & 0 & Optimal &  0.01 & 3 &  3.00 &  0.00\\
instance n=20 298.alb & 1 & 0 & Optimal &  0.01 & 3 &  3.00 &  0.00\\
instance n=20 299.alb & 1 & 0 & Optimal &  0.02 & 3 &  3.00 &  0.00\\
instance n=20 3.alb & 1 & 0 & Optimal &  0.01 & 3 &  3.00 &  0.00\\
instance n=20 30.alb & 1 & 0 & Optimal &  0.04 & 16 & 16.00 &  0.00\\
instance n=20 300.alb & 1 & 0 & Optimal &  0.02 & 4 &  4.00 &  0.00\\
instance n=20 301.alb & 1 & 0 & Optimal &  0.01 & 3 &  3.00 &  0.00\\
instance n=20 302.alb & 1 & 0 & Optimal &  0.02 & 3 &  3.00 &  0.00\\
instance n=20 303.alb & 1 & 0 & Optimal &  0.01 & 3 &  3.00 &  0.00\\
instance n=20 304.alb & 1 & 0 & Optimal &  0.01 & 3 &  3.00 &  0.00\\
instance n=20 305.alb & 1 & 0 & Optimal &  0.01 & 3 &  3.00 &  0.00\\
instance n=20 306.alb & 1 & 0 & Optimal &  0.01 & 3 &  3.00 &  0.00\\
instance n=20 307.alb & 1 & 0 & Optimal &  0.02 & 3 &  3.00 &  0.00\\
instance n=20 308.alb & 1 & 0 & Optimal &  0.01 & 3 &  3.00 &  0.00\\
instance n=20 309.alb & 1 & 0 & Optimal &  0.02 & 3 &  3.00 &  0.00\\
instance n=20 31.alb & 1 & 0 & Optimal &  0.07 & 12 & 12.00 &  0.00\\
instance n=20 310.alb & 1 & 0 & Optimal &  0.02 & 3 &  3.00 &  0.00\\
instance n=20 311.alb & 1 & 0 & Optimal &  0.01 & 3 &  3.00 &  0.00\\
instance n=20 312.alb & 1 & 0 & Optimal &  0.02 & 4 &  4.00 &  0.00\\
instance n=20 313.alb & 1 & 0 & Optimal &  0.01 & 3 &  3.00 &  0.00\\
instance n=20 314.alb & 1 & 0 & Optimal &  0.02 & 3 &  3.00 &  0.00\\
instance n=20 315.alb & 1 & 0 & Optimal &  0.02 & 3 &  3.00 &  0.00\\
instance n=20 316.alb & 1 & 0 & Optimal &  0.06 & 10 & 10.00 &  0.00\\
instance n=20 317.alb & 1 & 0 & Optimal &  0.06 & 10 & 10.00 &  0.00\\
instance n=20 318.alb & 1 & 0 & Optimal &  0.02 & 10 & 10.00 &  0.00\\
instance n=20 319.alb & 1 & 0 & Optimal &  0.18 & 14 & 14.00 &  0.00\\
instance n=20 32.alb & 1 & 0 & Optimal &  0.07 & 13 & 13.00 &  0.00\\
instance n=20 320.alb & 1 & 0 & Optimal &  0.01 & 12 & 12.00 &  0.00\\
instance n=20 321.alb & 1 & 0 & Optimal &  1.08 & 14 & 14.00 &  0.00\\
instance n=20 322.alb & 1 & 0 & Optimal &  0.31 & 12 & 12.00 &  0.00\\
instance n=20 323.alb & 1 & 0 & Optimal &  0.02 & 13 & 13.00 &  0.00\\
instance n=20 324.alb & 1 & 0 & Optimal &  0.07 & 9 &  9.00 &  0.00\\
instance n=20 325.alb & 1 & 0 & Optimal &  0.02 & 14 & 14.00 &  0.00\\
instance n=20 326.alb & 1 & 0 & Optimal &  0.53 & 14 & 14.00 &  0.00\\
instance n=20 327.alb & 1 & 0 & Optimal &  1.12 & 13 & 13.00 &  0.00\\
instance n=20 328.alb & 1 & 0 & Optimal &  0.01 & 13 & 13.00 &  0.00\\
instance n=20 329.alb & 1 & 0 & Optimal &  0.06 & 10 & 10.00 &  0.00\\
instance n=20 33.alb & 1 & 0 & Optimal &  0.03 & 11 & 11.00 &  0.00\\
instance n=20 330.alb & 1 & 0 & Optimal &  0.09 & 12 & 12.00 &  0.00\\
instance n=20 331.alb & 1 & 0 & Optimal &  0.08 & 13 & 13.00 &  0.00\\
instance n=20 332.alb & 1 & 0 & Optimal &  0.05 & 13 & 13.00 &  0.00\\
instance n=20 333.alb & 1 & 0 & Optimal &  0.03 & 11 & 11.00 &  0.00\\
instance n=20 334.alb & 1 & 0 & Optimal &  0.07 & 10 & 10.00 &  0.00\\
instance n=20 335.alb & 1 & 0 & Optimal &  0.02 & 14 & 14.00 &  0.00\\
instance n=20 336.alb & 1 & 0 & Optimal &  0.01 & 11 & 11.00 &  0.00\\
instance n=20 337.alb & 1 & 0 & Optimal &  0.05 & 10 & 10.00 &  0.00\\
instance n=20 338.alb & 1 & 0 & Optimal &  0.10 & 14 & 14.00 &  0.00\\
instance n=20 339.alb & 1 & 0 & Optimal &  0.01 & 13 & 13.00 &  0.00\\
instance n=20 34.alb & 1 & 0 & Optimal &  0.11 & 12 & 12.00 &  0.00\\
instance n=20 340.alb & 1 & 0 & Optimal &  0.11 & 11 & 11.00 &  0.00\\
instance n=20 341.alb & 1 & 0 & Optimal &  0.02 & 6 &  6.00 &  0.00\\
instance n=20 342.alb & 1 & 0 & Optimal &  0.02 & 6 &  6.00 &  0.00\\
instance n=20 343.alb & 1 & 0 & Optimal &  0.06 & 6 &  6.00 &  0.00\\
instance n=20 344.alb & 1 & 0 & Optimal &  0.02 & 6 &  6.00 &  0.00\\
instance n=20 345.alb & 1 & 0 & Optimal &  0.02 & 4 &  4.00 &  0.00\\
instance n=20 346.alb & 1 & 0 & Optimal &  0.02 & 5 &  5.00 &  0.00\\
instance n=20 347.alb & 1 & 0 & Optimal &  0.01 & 6 &  6.00 &  0.00\\
instance n=20 348.alb & 1 & 0 & Optimal &  0.02 & 5 &  5.00 &  0.00\\
instance n=20 349.alb & 1 & 0 & Optimal &  0.02 & 5 &  5.00 &  0.00\\
instance n=20 35.alb & 1 & 0 & Optimal &  0.04 & 12 & 12.00 &  0.00\\
instance n=20 350.alb & 1 & 0 & Optimal &  0.01 & 5 &  5.00 &  0.00\\
instance n=20 351.alb & 1 & 0 & Optimal &  0.02 & 5 &  5.00 &  0.00\\
instance n=20 352.alb & 1 & 0 & Optimal &  0.01 & 4 &  4.00 &  0.00\\
instance n=20 353.alb & 1 & 0 & Optimal &  0.02 & 6 &  6.00 &  0.00\\
instance n=20 354.alb & 1 & 0 & Optimal &  0.02 & 6 &  6.00 &  0.00\\
instance n=20 355.alb & 1 & 0 & Optimal &  0.02 & 5 &  5.00 &  0.00\\
instance n=20 356.alb & 1 & 0 & Optimal &  0.02 & 5 &  5.00 &  0.00\\
instance n=20 357.alb & 1 & 0 & Optimal &  0.02 & 5 &  5.00 &  0.00\\
instance n=20 358.alb & 1 & 0 & Optimal &  0.03 & 4 &  4.00 &  0.00\\
instance n=20 359.alb & 1 & 0 & Optimal &  0.02 & 4 &  4.00 &  0.00\\
instance n=20 36.alb & 1 & 0 & Optimal &  0.02 & 13 & 13.00 &  0.00\\
instance n=20 360.alb & 1 & 0 & Optimal &  0.03 & 6 &  6.00 &  0.00\\
instance n=20 361.alb & 1 & 0 & Optimal &  0.03 & 5 &  5.00 &  0.00\\
instance n=20 362.alb & 1 & 0 & Optimal &  0.02 & 5 &  5.00 &  0.00\\
instance n=20 363.alb & 1 & 0 & Optimal &  0.92 & 7 &  7.00 &  0.00\\
instance n=20 364.alb & 1 & 0 & Optimal &  0.01 & 4 &  4.00 &  0.00\\
instance n=20 365.alb & 1 & 0 & Optimal &  0.01 & 5 &  5.00 &  0.00\\
instance n=20 366.alb & 1 & 0 & Optimal &  0.01 & 3 &  3.00 &  0.00\\
instance n=20 367.alb & 1 & 0 & Optimal &  0.02 & 3 &  3.00 &  0.00\\
instance n=20 368.alb & 1 & 0 & Optimal &  0.02 & 3 &  3.00 &  0.00\\
instance n=20 369.alb & 1 & 0 & Optimal &  0.02 & 3 &  3.00 &  0.00\\
instance n=20 37.alb & 1 & 0 & Optimal &  0.01 & 12 & 12.00 &  0.00\\
instance n=20 370.alb & 1 & 0 & Optimal &  0.02 & 3 &  3.00 &  0.00\\
instance n=20 371.alb & 1 & 0 & Optimal &  0.02 & 3 &  3.00 &  0.00\\
instance n=20 372.alb & 1 & 0 & Optimal &  0.02 & 3 &  3.00 &  0.00\\
instance n=20 373.alb & 1 & 0 & Optimal &  0.02 & 3 &  3.00 &  0.00\\
instance n=20 374.alb & 1 & 0 & Optimal &  0.01 & 3 &  3.00 &  0.00\\
instance n=20 375.alb & 1 & 0 & Optimal &  0.01 & 3 &  3.00 &  0.00\\
instance n=20 376.alb & 1 & 0 & Optimal &  0.01 & 3 &  3.00 &  0.00\\
instance n=20 377.alb & 1 & 0 & Optimal &  0.02 & 3 &  3.00 &  0.00\\
instance n=20 378.alb & 1 & 0 & Optimal &  0.02 & 3 &  3.00 &  0.00\\
instance n=20 379.alb & 1 & 0 & Optimal &  0.02 & 4 &  4.00 &  0.00\\
instance n=20 38.alb & 1 & 0 & Optimal &  0.02 & 12 & 12.00 &  0.00\\
instance n=20 380.alb & 1 & 0 & Optimal &  0.02 & 3 &  3.00 &  0.00\\
instance n=20 381.alb & 1 & 0 & Optimal &  0.01 & 3 &  3.00 &  0.00\\
instance n=20 382.alb & 1 & 0 & Optimal &  0.02 & 4 &  4.00 &  0.00\\
instance n=20 383.alb & 1 & 0 & Optimal &  0.01 & 3 &  3.00 &  0.00\\
instance n=20 384.alb & 1 & 0 & Optimal &  0.02 & 3 &  3.00 &  0.00\\
instance n=20 385.alb & 1 & 0 & Optimal &  0.01 & 3 &  3.00 &  0.00\\
instance n=20 386.alb & 1 & 0 & Optimal &  0.01 & 3 &  3.00 &  0.00\\
instance n=20 387.alb & 1 & 0 & Optimal &  0.02 & 3 &  3.00 &  0.00\\
instance n=20 388.alb & 1 & 0 & Optimal &  0.01 & 3 &  3.00 &  0.00\\
instance n=20 389.alb & 1 & 0 & Optimal &  0.01 & 3 &  3.00 &  0.00\\
instance n=20 39.alb & 1 & 0 & Optimal &  0.05 & 13 & 13.00 &  0.00\\
instance n=20 390.alb & 1 & 0 & Optimal &  0.02 & 3 &  3.00 &  0.00\\
instance n=20 391.alb & 1 & 0 & Optimal &  0.03 & 11 & 11.00 &  0.00\\
instance n=20 392.alb & 1 & 0 & Optimal &  0.12 & 14 & 14.00 &  0.00\\
instance n=20 393.alb & 1 & 0 & Optimal &  0.11 & 11 & 11.00 &  0.00\\
instance n=20 394.alb & 1 & 0 & Optimal &  0.12 & 12 & 12.00 &  0.00\\
instance n=20 395.alb & 1 & 0 & Optimal &  0.02 & 12 & 12.00 &  0.00\\
instance n=20 396.alb & 1 & 0 & Optimal &  0.10 & 13 & 13.00 &  0.00\\
instance n=20 397.alb & 1 & 0 & Optimal &  0.11 & 10 & 10.00 &  0.00\\
instance n=20 398.alb & 1 & 0 & Optimal &  0.02 & 11 & 11.00 &  0.00\\
instance n=20 399.alb & 1 & 0 & Optimal &  0.01 & 13 & 13.00 &  0.00\\
instance n=20 4.alb & 1 & 0 & Optimal &  0.02 & 3 &  3.00 &  0.00\\
instance n=20 40.alb & 1 & 0 & Optimal &  0.03 & 12 & 12.00 &  0.00\\
instance n=20 400.alb & 1 & 0 & Optimal &  0.03 & 12 & 12.00 &  0.00\\
instance n=20 401.alb & 1 & 0 & Optimal &  0.12 & 12 & 12.00 &  0.00\\
instance n=20 402.alb & 1 & 0 & Optimal &  0.02 & 12 & 12.00 &  0.00\\
instance n=20 403.alb & 1 & 0 & Optimal &  0.01 & 12 & 12.00 &  0.00\\
instance n=20 404.alb & 1 & 0 & Optimal &  0.10 & 10 & 10.00 &  0.00\\
instance n=20 405.alb & 1 & 0 & Optimal &  0.02 & 12 & 12.00 &  0.00\\
instance n=20 406.alb & 1 & 0 & Optimal &  0.01 & 14 & 14.00 &  0.00\\
instance n=20 407.alb & 1 & 0 & Optimal &  0.03 & 10 & 10.00 &  0.00\\
instance n=20 408.alb & 1 & 0 & Optimal &  0.11 & 14 & 14.00 &  0.00\\
instance n=20 409.alb & 1 & 0 & Optimal &  0.10 & 12 & 12.00 &  0.00\\
instance n=20 41.alb & 1 & 0 & Optimal &  0.01 & 6 &  6.00 &  0.00\\
instance n=20 410.alb & 1 & 0 & Optimal &  0.03 & 11 & 11.00 &  0.00\\
instance n=20 411.alb & 1 & 0 & Optimal &  0.11 & 15 & 15.00 &  0.00\\
instance n=20 412.alb & 1 & 0 & Optimal &  0.12 & 11 & 11.00 &  0.00\\
instance n=20 413.alb & 1 & 0 & Optimal &  0.03 & 10 & 10.00 &  0.00\\
instance n=20 414.alb & 1 & 0 & Optimal &  0.11 & 12 & 12.00 &  0.00\\
instance n=20 415.alb & 1 & 0 & Optimal &  0.02 & 10 & 10.00 &  0.00\\
instance n=20 416.alb & 1 & 0 & Optimal &  0.02 & 6 &  6.00 &  0.00\\
instance n=20 417.alb & 1 & 0 & Optimal &  0.01 & 5 &  5.00 &  0.00\\
instance n=20 418.alb & 1 & 0 & Optimal &  0.01 & 6 &  6.00 &  0.00\\
instance n=20 419.alb & 1 & 0 & Optimal &  0.02 & 4 &  4.00 &  0.00\\
instance n=20 42.alb & 1 & 0 & Optimal &  0.02 & 5 &  5.00 &  0.00\\
instance n=20 420.alb & 1 & 0 & Optimal &  0.02 & 5 &  5.00 &  0.00\\
instance n=20 421.alb & 1 & 0 & Optimal &  0.03 & 6 &  6.00 &  0.00\\
instance n=20 422.alb & 1 & 0 & Optimal &  0.01 & 4 &  4.00 &  0.00\\
instance n=20 423.alb & 1 & 0 & Optimal &  0.02 & 6 &  6.00 &  0.00\\
instance n=20 424.alb & 1 & 0 & Optimal &  0.01 & 5 &  5.00 &  0.00\\
instance n=20 425.alb & 1 & 0 & Optimal &  0.02 & 6 &  6.00 &  0.00\\
instance n=20 426.alb & 1 & 0 & Optimal &  0.01 & 5 &  5.00 &  0.00\\
instance n=20 427.alb & 1 & 0 & Optimal &  0.02 & 6 &  6.00 &  0.00\\
instance n=20 428.alb & 1 & 0 & Optimal &  0.02 & 5 &  5.00 &  0.00\\
instance n=20 429.alb & 1 & 0 & Optimal &  0.02 & 4 &  4.00 &  0.00\\
instance n=20 43.alb & 1 & 0 & Optimal &  0.02 & 5 &  5.00 &  0.00\\
instance n=20 430.alb & 1 & 0 & Optimal &  0.02 & 5 &  5.00 &  0.00\\
instance n=20 431.alb & 1 & 0 & Optimal &  0.02 & 6 &  6.00 &  0.00\\
instance n=20 432.alb & 1 & 0 & Optimal &  0.01 & 5 &  5.00 &  0.00\\
instance n=20 433.alb & 1 & 0 & Optimal &  0.02 & 5 &  5.00 &  0.00\\
instance n=20 434.alb & 1 & 0 & Optimal &  0.01 & 5 &  5.00 &  0.00\\
instance n=20 435.alb & 1 & 0 & Optimal &  0.01 & 7 &  7.00 &  0.00\\
instance n=20 436.alb & 1 & 0 & Optimal &  0.02 & 5 &  5.00 &  0.00\\
instance n=20 437.alb & 1 & 0 & Optimal &  0.01 & 5 &  5.00 &  0.00\\
instance n=20 438.alb & 1 & 0 & Optimal &  0.02 & 6 &  6.00 &  0.00\\
instance n=20 439.alb & 1 & 0 & Optimal &  0.02 & 5 &  5.00 &  0.00\\
instance n=20 44.alb & 1 & 0 & Optimal &  0.01 & 5 &  5.00 &  0.00\\
instance n=20 440.alb & 1 & 0 & Optimal &  0.01 & 5 &  5.00 &  0.00\\
instance n=20 441.alb & 1 & 0 & Optimal &  0.01 & 3 &  3.00 &  0.00\\
instance n=20 442.alb & 1 & 0 & Optimal &  0.02 & 3 &  3.00 &  0.00\\
instance n=20 443.alb & 1 & 0 & Optimal &  0.02 & 3 &  3.00 &  0.00\\
instance n=20 444.alb & 1 & 0 & Optimal &  0.03 & 3 &  3.00 &  0.00\\
instance n=20 445.alb & 1 & 0 & Optimal &  0.01 & 3 &  3.00 &  0.00\\
instance n=20 446.alb & 1 & 0 & Optimal &  0.02 & 3 &  3.00 &  0.00\\
instance n=20 447.alb & 1 & 0 & Optimal &  0.02 & 3 &  3.00 &  0.00\\
instance n=20 448.alb & 1 & 0 & Optimal &  0.02 & 3 &  3.00 &  0.00\\
instance n=20 449.alb & 1 & 0 & Optimal &  0.02 & 3 &  3.00 &  0.00\\
instance n=20 45.alb & 1 & 0 & Optimal &  0.03 & 6 &  6.00 &  0.00\\
instance n=20 450.alb & 1 & 0 & Optimal &  0.01 & 3 &  3.00 &  0.00\\
instance n=20 451.alb & 1 & 0 & Optimal &  0.02 & 3 &  3.00 &  0.00\\
instance n=20 452.alb & 1 & 0 & Optimal &  0.01 & 3 &  3.00 &  0.00\\
instance n=20 453.alb & 1 & 0 & Optimal &  0.01 & 3 &  3.00 &  0.00\\
instance n=20 454.alb & 1 & 0 & Optimal &  0.02 & 3 &  3.00 &  0.00\\
instance n=20 455.alb & 1 & 0 & Optimal &  0.01 & 3 &  3.00 &  0.00\\
instance n=20 456.alb & 1 & 0 & Optimal &  0.01 & 4 &  4.00 &  0.00\\
instance n=20 457.alb & 1 & 0 & Optimal &  0.01 & 3 &  3.00 &  0.00\\
instance n=20 458.alb & 1 & 0 & Optimal &  0.02 & 3 &  3.00 &  0.00\\
instance n=20 459.alb & 1 & 0 & Optimal &  0.01 & 3 &  3.00 &  0.00\\
instance n=20 46.alb & 1 & 0 & Optimal &  0.01 & 4 &  4.00 &  0.00\\
instance n=20 460.alb & 1 & 0 & Optimal &  0.03 & 3 &  3.00 &  0.00\\
instance n=20 461.alb & 1 & 0 & Optimal &  0.02 & 3 &  3.00 &  0.00\\
instance n=20 462.alb & 1 & 0 & Optimal &  0.01 & 3 &  3.00 &  0.00\\
instance n=20 463.alb & 1 & 0 & Optimal &  0.01 & 3 &  3.00 &  0.00\\
instance n=20 464.alb & 1 & 0 & Optimal &  0.02 & 3 &  3.00 &  0.00\\
instance n=20 465.alb & 1 & 0 & Optimal &  0.01 & 3 &  3.00 &  0.00\\
instance n=20 466.alb & 1 & 0 & Optimal &  0.01 & 13 & 13.00 &  0.00\\
instance n=20 467.alb & 1 & 0 & Optimal &  0.01 & 14 & 14.00 &  0.00\\
instance n=20 468.alb & 1 & 0 & Optimal &  0.02 & 13 & 13.00 &  0.00\\
instance n=20 469.alb & 1 & 0 & Optimal &  0.02 & 14 & 14.00 &  0.00\\
instance n=20 47.alb & 1 & 0 & Optimal &  0.02 & 4 &  4.00 &  0.00\\
instance n=20 470.alb & 1 & 0 & Optimal &  0.02 & 12 & 12.00 &  0.00\\
instance n=20 471.alb & 1 & 0 & Optimal &  0.02 & 12 & 12.00 &  0.00\\
instance n=20 472.alb & 1 & 0 & Optimal &  0.01 & 13 & 13.00 &  0.00\\
instance n=20 473.alb & 1 & 0 & Optimal &  0.01 & 10 & 10.00 &  0.00\\
instance n=20 474.alb & 1 & 0 & Optimal &  0.02 & 14 & 14.00 &  0.00\\
instance n=20 475.alb & 1 & 0 & Optimal &  0.02 & 11 & 11.00 &  0.00\\
instance n=20 476.alb & 1 & 0 & Optimal &  0.02 & 11 & 11.00 &  0.00\\
instance n=20 477.alb & 1 & 0 & Optimal &  0.02 & 11 & 11.00 &  0.00\\
instance n=20 478.alb & 1 & 0 & Optimal &  0.01 & 12 & 12.00 &  0.00\\
instance n=20 479.alb & 1 & 0 & Optimal &  0.02 & 13 & 13.00 &  0.00\\
instance n=20 48.alb & 1 & 0 & Optimal &  0.02 & 5 &  5.00 &  0.00\\
instance n=20 480.alb & 1 & 0 & Optimal &  0.02 & 13 & 13.00 &  0.00\\
instance n=20 481.alb & 1 & 0 & Optimal &  0.02 & 13 & 13.00 &  0.00\\
instance n=20 482.alb & 1 & 0 & Optimal &  0.01 & 13 & 13.00 &  0.00\\
instance n=20 483.alb & 1 & 0 & Optimal &  0.02 & 12 & 12.00 &  0.00\\
instance n=20 484.alb & 1 & 0 & Optimal &  0.01 & 13 & 13.00 &  0.00\\
instance n=20 485.alb & 1 & 0 & Optimal &  0.02 & 15 & 15.00 &  0.00\\
instance n=20 486.alb & 1 & 0 & Optimal &  0.02 & 11 & 11.00 &  0.00\\
instance n=20 487.alb & 1 & 0 & Optimal &  0.02 & 12 & 12.00 &  0.00\\
instance n=20 488.alb & 1 & 0 & Optimal &  0.01 & 15 & 15.00 &  0.00\\
instance n=20 489.alb & 1 & 0 & Optimal &  0.02 & 12 & 12.00 &  0.00\\
instance n=20 49.alb & 1 & 0 & Optimal &  0.02 & 4 &  4.00 &  0.00\\
instance n=20 490.alb & 1 & 0 & Optimal &  0.02 & 12 & 12.00 &  0.00\\
instance n=20 491.alb & 1 & 0 & Optimal &  0.02 & 6 &  6.00 &  0.00\\
instance n=20 492.alb & 1 & 0 & Optimal &  0.02 & 5 &  5.00 &  0.00\\
instance n=20 493.alb & 1 & 0 & Optimal &  0.01 & 5 &  5.00 &  0.00\\
instance n=20 494.alb & 1 & 0 & Optimal &  0.01 & 6 &  6.00 &  0.00\\
instance n=20 495.alb & 1 & 0 & Optimal &  0.01 & 6 &  6.00 &  0.00\\
instance n=20 496.alb & 1 & 0 & Optimal &  0.02 & 5 &  5.00 &  0.00\\
instance n=20 497.alb & 1 & 0 & Optimal &  0.02 & 6 &  6.00 &  0.00\\
instance n=20 498.alb & 1 & 0 & Optimal &  0.01 & 6 &  6.00 &  0.00\\
instance n=20 499.alb & 1 & 0 & Optimal &  0.02 & 5 &  5.00 &  0.00\\
instance n=20 5.alb & 1 & 0 & Optimal &  0.01 & 3 &  3.00 &  0.00\\
instance n=20 50.alb & 1 & 0 & Optimal &  0.01 & 4 &  4.00 &  0.00\\
instance n=20 500.alb & 1 & 0 & Optimal &  0.02 & 8 &  8.00 &  0.00\\
instance n=20 501.alb & 1 & 0 & Optimal &  0.03 & 5 &  5.00 &  0.00\\
instance n=20 502.alb & 1 & 0 & Optimal &  0.02 & 4 &  4.00 &  0.00\\
instance n=20 503.alb & 1 & 0 & Optimal &  0.02 & 6 &  6.00 &  0.00\\
instance n=20 504.alb & 1 & 0 & Optimal &  0.02 & 6 &  6.00 &  0.00\\
instance n=20 505.alb & 1 & 0 & Optimal &  0.02 & 6 &  6.00 &  0.00\\
instance n=20 506.alb & 1 & 0 & Optimal &  0.01 & 5 &  5.00 &  0.00\\
instance n=20 507.alb & 1 & 0 & Optimal &  0.02 & 5 &  5.00 &  0.00\\
instance n=20 508.alb & 1 & 0 & Optimal &  0.02 & 5 &  5.00 &  0.00\\
instance n=20 509.alb & 1 & 0 & Optimal &  0.01 & 4 &  4.00 &  0.00\\
instance n=20 51.alb & 1 & 0 & Optimal &  0.02 & 4 &  4.00 &  0.00\\
instance n=20 510.alb & 1 & 0 & Optimal &  0.02 & 5 &  5.00 &  0.00\\
instance n=20 511.alb & 1 & 0 & Optimal &  0.01 & 5 &  5.00 &  0.00\\
instance n=20 512.alb & 1 & 0 & Optimal &  0.02 & 5 &  5.00 &  0.00\\
instance n=20 513.alb & 1 & 0 & Optimal &  0.02 & 5 &  5.00 &  0.00\\
instance n=20 514.alb & 1 & 0 & Optimal &  0.02 & 5 &  5.00 &  0.00\\
instance n=20 515.alb & 1 & 0 & Optimal &  0.02 & 6 &  6.00 &  0.00\\
instance n=20 516.alb & 1 & 0 & Optimal &  0.02 & 3 &  3.00 &  0.00\\
instance n=20 517.alb & 1 & 0 & Optimal &  0.01 & 3 &  3.00 &  0.00\\
instance n=20 518.alb & 1 & 0 & Optimal &  0.01 & 3 &  3.00 &  0.00\\
instance n=20 519.alb & 1 & 0 & Optimal &  0.01 & 3 &  3.00 &  0.00\\
instance n=20 52.alb & 1 & 0 & Optimal &  0.02 & 4 &  4.00 &  0.00\\
instance n=20 520.alb & 1 & 0 & Optimal &  0.01 & 3 &  3.00 &  0.00\\
instance n=20 521.alb & 1 & 0 & Optimal &  0.01 & 3 &  3.00 &  0.00\\
instance n=20 522.alb & 1 & 0 & Optimal &  0.02 & 3 &  3.00 &  0.00\\
instance n=20 523.alb & 1 & 0 & Optimal &  0.02 & 3 &  3.00 &  0.00\\
instance n=20 524.alb & 1 & 0 & Optimal &  0.02 & 3 &  3.00 &  0.00\\
instance n=20 525.alb & 1 & 0 & Optimal &  0.01 & 3 &  3.00 &  0.00\\
instance n=20 53.alb & 1 & 0 & Optimal &  0.01 & 5 &  5.00 &  0.00\\
instance n=20 54.alb & 1 & 0 & Optimal &  0.02 & 5 &  5.00 &  0.00\\
instance n=20 55.alb & 1 & 0 & Optimal &  0.02 & 5 &  5.00 &  0.00\\
instance n=20 56.alb & 1 & 0 & Optimal &  0.03 & 4 &  4.00 &  0.00\\
instance n=20 57.alb & 1 & 0 & Optimal &  0.01 & 4 &  4.00 &  0.00\\
instance n=20 58.alb & 1 & 0 & Optimal &  0.10 & 5 &  5.00 &  0.00\\
instance n=20 59.alb & 1 & 0 & Optimal &  0.11 & 4 &  4.00 &  0.00\\
instance n=20 6.alb & 1 & 0 & Optimal &  0.01 & 3 &  3.00 &  0.00\\
instance n=20 60.alb & 1 & 0 & Optimal &  0.11 & 6 &  6.00 &  0.00\\
instance n=20 61.alb & 1 & 0 & Optimal &  0.03 & 7 &  7.00 &  0.00\\
instance n=20 62.alb & 1 & 0 & Optimal &  0.02 & 5 &  5.00 &  0.00\\
instance n=20 63.alb & 1 & 0 & Optimal &  0.03 & 5 &  5.00 &  0.00\\
instance n=20 64.alb & 1 & 0 & Optimal &  0.02 & 5 &  5.00 &  0.00\\
instance n=20 65.alb & 1 & 0 & Optimal &  0.02 & 5 &  5.00 &  0.00\\
instance n=20 66.alb & 1 & 0 & Optimal &  0.01 & 3 &  3.00 &  0.00\\
instance n=20 67.alb & 1 & 0 & Optimal &  0.01 & 3 &  3.00 &  0.00\\
instance n=20 68.alb & 1 & 0 & Optimal &  0.01 & 3 &  3.00 &  0.00\\
instance n=20 69.alb & 1 & 0 & Optimal &  0.01 & 2 &  2.00 &  0.00\\
instance n=20 7.alb & 1 & 0 & Optimal &  0.02 & 3 &  3.00 &  0.00\\
instance n=20 70.alb & 1 & 0 & Optimal &  0.10 & 3 &  3.00 &  0.00\\
instance n=20 71.alb & 1 & 0 & Optimal &  0.02 & 3 &  3.00 &  0.00\\
instance n=20 72.alb & 1 & 0 & Optimal &  0.02 & 3 &  3.00 &  0.00\\
instance n=20 73.alb & 1 & 0 & Optimal &  0.01 & 2 &  2.00 &  0.00\\
instance n=20 74.alb & 1 & 0 & Optimal &  0.01 & 3 &  3.00 &  0.00\\
instance n=20 75.alb & 1 & 0 & Optimal &  0.01 & 3 &  3.00 &  0.00\\
instance n=20 76.alb & 1 & 0 & Optimal &  0.01 & 3 &  3.00 &  0.00\\
instance n=20 77.alb & 1 & 0 & Optimal &  0.01 & 3 &  3.00 &  0.00\\
instance n=20 78.alb & 1 & 0 & Optimal &  0.02 & 3 &  3.00 &  0.00\\
instance n=20 79.alb & 1 & 0 & Optimal &  0.01 & 3 &  3.00 &  0.00\\
instance n=20 8.alb & 1 & 0 & Optimal &  0.01 & 3 &  3.00 &  0.00\\
instance n=20 80.alb & 1 & 0 & Optimal &  0.02 & 3 &  3.00 &  0.00\\
instance n=20 81.alb & 1 & 0 & Optimal &  0.01 & 3 &  3.00 &  0.00\\
instance n=20 82.alb & 1 & 0 & Optimal &  0.03 & 4 &  4.00 &  0.00\\
instance n=20 83.alb & 1 & 0 & Optimal &  0.02 & 3 &  3.00 &  0.00\\
instance n=20 84.alb & 1 & 0 & Optimal &  0.02 & 3 &  3.00 &  0.00\\
instance n=20 85.alb & 1 & 0 & Optimal &  0.01 & 3 &  3.00 &  0.00\\
instance n=20 86.alb & 1 & 0 & Optimal &  0.02 & 3 &  3.00 &  0.00\\
instance n=20 87.alb & 1 & 0 & Optimal &  0.01 & 3 &  3.00 &  0.00\\
instance n=20 88.alb & 1 & 0 & Optimal &  0.02 & 3 &  3.00 &  0.00\\
instance n=20 89.alb & 1 & 0 & Optimal &  0.02 & 3 &  3.00 &  0.00\\
instance n=20 9.alb & 1 & 0 & Optimal &  0.01 & 3 &  3.00 &  0.00\\
instance n=20 90.alb & 1 & 0 & Optimal &  0.01 & 3 &  3.00 &  0.00\\
instance n=20 91.alb & 1 & 0 & Optimal &  0.03 & 11 & 11.00 &  0.00\\
instance n=20 92.alb & 1 & 0 & Optimal &  0.01 & 11 & 11.00 &  0.00\\
instance n=20 93.alb & 1 & 0 & Optimal &  0.04 & 13 & 13.00 &  0.00\\
instance n=20 94.alb & 1 & 0 & Optimal &  0.03 & 10 & 10.00 &  0.00\\
instance n=20 95.alb & 1 & 0 & Optimal &  0.10 & 12 & 12.00 &  0.00\\
instance n=20 96.alb & 1 & 0 & Optimal &  0.02 & 10 & 10.00 &  0.00\\
instance n=20 97.alb & 1 & 0 & Optimal &  0.10 & 15 & 15.00 &  0.00\\
instance n=20 98.alb & 1 & 0 & Optimal &  0.02 & 13 & 13.00 &  0.00\\
instance n=20 99.alb & 1 & 0 & Optimal &  0.12 & 12 & 12.00 &  0.00\\
instance n=50 1.alb & 1 & 0 & Optimal &  0.15 & 8 &  8.00 &  0.00\\
instance n=50 10.alb & 1 & 0 & Optimal & 120.02 & 7 &  7.00 &  0.00\\
instance n=50 100.alb & 1 & 0 & Optimal &  0.06 & 7 &  7.00 &  0.00\\
instance n=50 101.alb & 1 & 0 & Optimal & 17.99 & 30 & 30.00 &  0.00\\
instance n=50 102.alb & 1 & 0 & Optimal & 78.08 & 32 & 32.00 &  0.00\\
instance n=50 103.alb & 1 & 0 & Optimal &  0.17 & 29 & 29.00 &  0.00\\
instance n=50 104.alb & 1 & 0 & Optimal &  1.29 & 27 & 27.00 &  0.00\\
instance n=50 105.alb & 1 & 0 & Optimal & 23.50 & 24 & 24.00 &  0.00\\
instance n=50 106.alb & 1 & 0 & Optimal & 18.60 & 28 & 28.00 &  0.00\\
instance n=50 107.alb & 1 & 0 & Optimal &  4.01 & 28 & 28.00 &  0.00\\
instance n=50 108.alb & 1 & 0 & Optimal &  0.72 & 30 & 30.00 &  0.00\\
instance n=50 109.alb & 1 & 0 & Optimal &  0.12 & 30 & 30.00 &  0.00\\
instance n=50 11.alb & 1 & 0 & Optimal &  0.06 & 7 &  7.00 &  0.00\\
instance n=50 110.alb & 1 & 0 & Optimal &  0.44 & 26 & 26.00 &  0.00\\
instance n=50 111.alb & 1 & 0 & Optimal &  0.31 & 28 & 28.00 &  0.00\\
instance n=50 112.alb & 1 & 0 & Optimal &  1.07 & 27 & 27.00 &  0.00\\
instance n=50 113.alb & 1 & 0 & Optimal & 11.14 & 28 & 28.00 &  0.00\\
instance n=50 114.alb & 1 & 0 & Optimal &  0.68 & 27 & 27.00 &  0.00\\
instance n=50 115.alb & 1 & 0 & Optimal & 109.51 & 28 & 28.00 &  0.00\\
instance n=50 116.alb & 1 & 0 & Optimal &  0.31 & 32 & 32.00 &  0.00\\
instance n=50 117.alb & 1 & 0 & Optimal & 19.30 & 27 & 27.00 &  0.00\\
instance n=50 118.alb & 1 & 0 & Optimal &  0.74 & 29 & 29.00 &  0.00\\
instance n=50 119.alb & 1 & 0 & Optimal &  0.33 & 25 & 25.00 &  0.00\\
instance n=50 12.alb & 1 & 0 & Optimal & 120.02 & 6 &  6.00 &  0.00\\
instance n=50 120.alb & 1 & 0 & Optimal &  1.71 & 27 & 27.00 &  0.00\\
instance n=50 121.alb & 1 & 0 & Optimal &  8.70 & 32 & 32.00 &  0.00\\
instance n=50 122.alb & 1 & 0 & Optimal & 20.50 & 29 & 29.00 &  0.00\\
instance n=50 123.alb & 1 & 0 & Optimal &  2.65 & 32 & 32.00 &  0.00\\
instance n=50 124.alb & 1 & 0 & Optimal &  1.25 & 29 & 29.00 &  0.00\\
instance n=50 125.alb & 1 & 0 & Optimal &  0.08 & 33 & 33.00 &  0.00\\
instance n=50 126.alb & 1 & 0 & Optimal &  0.07 & 12 & 12.00 &  0.00\\
instance n=50 127.alb & 1 & 0 & Optimal &  0.27 & 14 & 14.00 &  0.00\\
instance n=50 128.alb & 1 & 0 & Optimal &  0.37 & 12 & 12.00 &  0.00\\
instance n=50 129.alb & 1 & 0 & Optimal &  0.09 & 13 & 13.00 &  0.00\\
instance n=50 13.alb & 1 & 0 & Optimal &  0.63 & 6 &  6.00 &  0.00\\
instance n=50 130.alb & 1 & 0 & Optimal &  0.17 & 13 & 13.00 &  0.00\\
instance n=50 131.alb & 1 & 0 & Optimal &  0.07 & 12 & 12.00 &  0.00\\
instance n=50 132.alb & 1 & 0 & Optimal &  0.80 & 12 & 12.00 &  0.00\\
instance n=50 133.alb & 1 & 0 & Optimal &  0.04 & 12 & 12.00 &  0.00\\
instance n=50 134.alb & 1 & 0 & Optimal &  0.13 & 14 & 14.00 &  0.00\\
instance n=50 135.alb & 1 & 0 & Optimal &  0.22 & 13 & 13.00 &  0.00\\
instance n=50 136.alb & 1 & 0 & Optimal &  0.05 & 11 & 11.00 &  0.00\\
instance n=50 137.alb & 1 & 0 & Optimal &  0.05 & 11 & 11.00 &  0.00\\
instance n=50 138.alb & 1 & 0 & Optimal &  0.05 & 12 & 12.00 &  0.00\\
instance n=50 139.alb & 1 & 0 & Optimal &  3.70 & 11 & 11.00 &  0.00\\
instance n=50 14.alb & 1 & 0 & Optimal &  0.04 & 7 &  7.00 &  0.00\\
instance n=50 140.alb & 1 & 0 & Optimal &  0.03 & 12 & 12.00 &  0.00\\
instance n=50 141.alb & 1 & 0 & Optimal &  0.61 & 13 & 13.00 &  0.00\\
instance n=50 142.alb & 1 & 0 & Optimal &  0.12 & 11 & 11.00 &  0.00\\
instance n=50 143.alb & 1 & 0 & Optimal &  0.08 & 12 & 12.00 &  0.00\\
instance n=50 144.alb & 1 & 0 & Optimal &  0.07 & 13 & 13.00 &  0.00\\
instance n=50 145.alb & 1 & 0 & Optimal &  0.24 & 10 & 10.00 &  0.00\\
instance n=50 146.alb & 1 & 0 & Optimal &  0.12 & 13 & 13.00 &  0.00\\
instance n=50 147.alb & 1 & 0 & Optimal &  0.26 & 13 & 13.00 &  0.00\\
instance n=50 148.alb & 1 & 0 & Optimal &  0.04 & 10 & 10.00 &  0.00\\
instance n=50 149.alb & 1 & 0 & Optimal &  0.08 & 12 & 12.00 &  0.00\\
instance n=50 15.alb & 1 & 0 & Optimal &  0.04 & 8 &  8.00 &  0.00\\
instance n=50 150.alb & 1 & 0 & Optimal &  0.07 & 11 & 11.00 &  0.00\\
instance n=50 151.alb & 1 & 0 & Optimal &  0.12 & 7 &  7.00 &  0.00\\
instance n=50 152.alb & 1 & 0 & Optimal &  0.71 & 7 &  7.00 &  0.00\\
instance n=50 153.alb & 1 & 0 & Optimal &  1.28 & 7 &  7.00 &  0.00\\
instance n=50 154.alb & 1 & 0 & Optimal &  0.06 & 8 &  8.00 &  0.00\\
instance n=50 155.alb & 1 & 0 & Optimal &  0.02 & 7 &  7.00 &  0.00\\
instance n=50 156.alb & 1 & 0 & Optimal &  0.04 & 7 &  7.00 &  0.00\\
instance n=50 157.alb & 1 & 0 & Optimal &  0.72 & 8 &  8.00 &  0.00\\
instance n=50 158.alb & 1 & 0 & Optimal &  5.39 & 7 &  7.00 &  0.00\\
instance n=50 159.alb & 1 & 0 & Optimal &  0.04 & 7 &  7.00 &  0.00\\
instance n=50 16.alb & 1 & 0 & Optimal &  0.03 & 8 &  8.00 &  0.00\\
instance n=50 160.alb & 1 & 0 & Optimal &  0.11 & 8 &  8.00 &  0.00\\
instance n=50 161.alb & 1 & 0 & Optimal & 120.01 & 7 &  7.00 &  0.00\\
instance n=50 162.alb & 1 & 0 & Optimal & 57.69 & 8 &  8.00 &  0.00\\
instance n=50 163.alb & 1 & 0 & Optimal &  5.10 & 7 &  7.00 &  0.00\\
instance n=50 164.alb & 1 & 0 & Optimal &  0.28 & 7 &  7.00 &  0.00\\
instance n=50 165.alb & 1 & 0 & Optimal &  0.22 & 8 &  8.00 &  0.00\\
instance n=50 166.alb & 1 & 0 & Optimal &  0.03 & 8 &  8.00 &  0.00\\
instance n=50 167.alb & 1 & 0 & Optimal &  2.69 & 7 &  7.00 &  0.00\\
instance n=50 168.alb & 1 & 0 & Optimal &  0.59 & 8 &  8.00 &  0.00\\
instance n=50 169.alb & 1 & 0 & Optimal &  6.11 & 8 &  8.00 &  0.00\\
instance n=50 17.alb & 1 & 0 & Optimal &  0.03 & 7 &  7.00 &  0.00\\
instance n=50 170.alb & 1 & 0 & Optimal &  2.96 & 7 &  7.00 &  0.00\\
instance n=50 171.alb & 1 & 0 & Optimal &  1.73 & 8 &  8.00 &  0.00\\
instance n=50 172.alb & 1 & 0 & Optimal &  0.23 & 7 &  7.00 &  0.00\\
instance n=50 173.alb & 1 & 0 & Optimal &  0.59 & 7 &  7.00 &  0.00\\
instance n=50 174.alb & 1 & 0 & Optimal &  4.45 & 7 &  7.00 &  0.00\\
instance n=50 175.alb & 1 & 0 & Optimal &  0.93 & 7 &  7.00 &  0.00\\
instance n=50 176.alb & 1 & 0 & Optimal & 21.25 & 27 & 27.00 &  0.00\\
instance n=50 177.alb & 1 & 0 & Solution & 120.13 & 28 & 27.00 &  3.57\\
instance n=50 178.alb & 1 & 0 & Solution & 120.12 & 28 & 27.00 &  3.57\\
instance n=50 179.alb & 1 & 0 & Optimal &  9.31 & 26 & 26.00 &  0.00\\
instance n=50 18.alb & 1 & 0 & Optimal &  0.04 & 7 &  7.00 &  0.00\\
instance n=50 180.alb & 1 & 0 & Optimal &  0.44 & 26 & 26.00 &  0.00\\
instance n=50 181.alb & 1 & 0 & Optimal &  3.32 & 29 & 29.00 &  0.00\\
instance n=50 182.alb & 1 & 0 & Optimal & 120.05 & 26 & 26.00 &  0.00\\
instance n=50 183.alb & 1 & 0 & Optimal & 29.62 & 28 & 28.00 &  0.00\\
instance n=50 184.alb & 1 & 0 & Optimal &  0.06 & 38 & 38.00 &  0.00\\
instance n=50 185.alb & 1 & 0 & Optimal & 41.90 & 26 & 26.00 &  0.00\\
instance n=50 186.alb & 1 & 0 & Optimal &  0.94 & 26 & 26.00 &  0.00\\
instance n=50 187.alb & 1 & 0 & Solution & 120.79 & 26 & 25.00 &  3.85\\
instance n=50 188.alb & 1 & 0 & Solution & 121.18 & 25 & 24.00 &  4.00\\
instance n=50 189.alb & 1 & 0 & Solution & 120.15 & 26 & 25.00 &  3.85\\
instance n=50 19.alb & 1 & 0 & Optimal &  0.19 & 8 &  8.00 &  0.00\\
instance n=50 190.alb & 1 & 0 & Optimal &  1.72 & 30 & 30.00 &  0.00\\
instance n=50 191.alb & 1 & 0 & Solution & 121.36 & 28 & 27.00 &  3.57\\
instance n=50 192.alb & 1 & 0 & Optimal &  2.66 & 27 & 27.00 &  0.00\\
instance n=50 193.alb & 1 & 0 & Optimal & 38.33 & 28 & 28.00 &  0.00\\
instance n=50 194.alb & 1 & 0 & Optimal & 23.55 & 28 & 28.00 &  0.00\\
instance n=50 195.alb & 1 & 0 & Optimal &  2.55 & 28 & 28.00 &  0.00\\
instance n=50 196.alb & 1 & 0 & Optimal & 31.41 & 27 & 27.00 &  0.00\\
instance n=50 197.alb & 1 & 0 & Optimal & 120.03 & 28 & 28.00 &  0.00\\
instance n=50 198.alb & 1 & 0 & Optimal &  0.08 & 28 & 28.00 &  0.00\\
instance n=50 199.alb & 1 & 0 & Optimal &  0.09 & 29 & 29.00 &  0.00\\
instance n=50 2.alb & 1 & 0 & Optimal & 67.63 & 6 &  6.00 &  0.00\\
instance n=50 20.alb & 1 & 0 & Optimal &  0.04 & 8 &  8.00 &  0.00\\
instance n=50 200.alb & 1 & 0 & Solution & 121.04 & 25 & 24.00 &  4.00\\
instance n=50 201.alb & 1 & 0 & Optimal &  0.04 & 13 & 13.00 &  0.00\\
instance n=50 202.alb & 1 & 0 & Optimal &  1.56 & 9 &  9.00 &  0.00\\
instance n=50 203.alb & 1 & 0 & Optimal &  0.04 & 11 & 11.00 &  0.00\\
instance n=50 204.alb & 1 & 0 & Optimal &  0.99 & 10 & 10.00 &  0.00\\
instance n=50 205.alb & 1 & 0 & Optimal &  0.04 & 13 & 13.00 &  0.00\\
instance n=50 206.alb & 1 & 0 & Optimal & 120.06 & 11 & 11.00 &  0.00\\
instance n=50 207.alb & 1 & 0 & Optimal &  0.65 & 10 & 10.00 &  0.00\\
instance n=50 208.alb & 1 & 0 & Optimal &  0.23 & 13 & 13.00 &  0.00\\
instance n=50 209.alb & 1 & 0 & Optimal &  1.16 & 11 & 11.00 &  0.00\\
instance n=50 21.alb & 1 & 0 & Optimal & 120.03 & 6 &  6.00 &  0.00\\
instance n=50 210.alb & 1 & 0 & Optimal &  0.04 & 13 & 13.00 &  0.00\\
instance n=50 211.alb & 1 & 0 & Optimal &  0.03 & 12 & 12.00 &  0.00\\
instance n=50 212.alb & 1 & 0 & Optimal &  0.08 & 10 & 10.00 &  0.00\\
instance n=50 213.alb & 1 & 0 & Optimal &  0.04 & 13 & 13.00 &  0.00\\
instance n=50 214.alb & 1 & 0 & Optimal &  4.68 & 11 & 11.00 &  0.00\\
instance n=50 215.alb & 1 & 0 & Optimal &  0.06 & 11 & 11.00 &  0.00\\
instance n=50 216.alb & 1 & 0 & Optimal &  0.30 & 12 & 12.00 &  0.00\\
instance n=50 217.alb & 1 & 0 & Optimal &  0.84 & 13 & 13.00 &  0.00\\
instance n=50 218.alb & 1 & 0 & Optimal &  0.04 & 12 & 12.00 &  0.00\\
instance n=50 219.alb & 1 & 0 & Optimal &  0.28 & 11 & 11.00 &  0.00\\
instance n=50 22.alb & 1 & 0 & Optimal & 120.02 & 7 &  7.00 &  0.00\\
instance n=50 220.alb & 1 & 0 & Optimal &  0.04 & 11 & 11.00 &  0.00\\
instance n=50 221.alb & 1 & 0 & Optimal &  2.64 & 11 & 11.00 &  0.00\\
instance n=50 222.alb & 1 & 0 & Optimal &  0.29 & 14 & 14.00 &  0.00\\
instance n=50 223.alb & 1 & 0 & Optimal &  0.28 & 11 & 11.00 &  0.00\\
instance n=50 224.alb & 1 & 0 & Optimal &  0.09 & 11 & 11.00 &  0.00\\
instance n=50 225.alb & 1 & 0 & Optimal &  0.03 & 12 & 12.00 &  0.00\\
instance n=50 226.alb & 1 & 0 & Optimal &  0.05 & 7 &  7.00 &  0.00\\
instance n=50 227.alb & 1 & 0 & Optimal &  0.09 & 6 &  6.00 &  0.00\\
instance n=50 228.alb & 1 & 0 & Optimal &  0.04 & 6 &  6.00 &  0.00\\
instance n=50 229.alb & 1 & 0 & Optimal &  0.03 & 6 &  6.00 &  0.00\\
instance n=50 23.alb & 1 & 0 & Optimal &  0.04 & 7 &  7.00 &  0.00\\
instance n=50 230.alb & 1 & 0 & Optimal &  0.06 & 7 &  7.00 &  0.00\\
instance n=50 231.alb & 1 & 0 & Optimal &  0.03 & 7 &  7.00 &  0.00\\
instance n=50 232.alb & 1 & 0 & Optimal &  0.05 & 7 &  7.00 &  0.00\\
instance n=50 233.alb & 1 & 0 & Optimal &  0.03 & 6 &  6.00 &  0.00\\
instance n=50 234.alb & 1 & 0 & Optimal &  0.09 & 8 &  8.00 &  0.00\\
instance n=50 235.alb & 1 & 0 & Optimal &  0.05 & 7 &  7.00 &  0.00\\
instance n=50 236.alb & 1 & 0 & Optimal &  0.40 & 7 &  7.00 &  0.00\\
instance n=50 237.alb & 1 & 0 & Optimal &  0.03 & 8 &  8.00 &  0.00\\
instance n=50 238.alb & 1 & 0 & Optimal &  0.06 & 7 &  7.00 &  0.00\\
instance n=50 239.alb & 1 & 0 & Optimal &  0.05 & 7 &  7.00 &  0.00\\
instance n=50 24.alb & 1 & 0 & Optimal & 120.02 & 7 &  7.00 &  0.00\\
instance n=50 240.alb & 1 & 0 & Optimal &  0.04 & 7 &  7.00 &  0.00\\
instance n=50 241.alb & 1 & 0 & Optimal &  0.08 & 7 &  7.00 &  0.00\\
instance n=50 242.alb & 1 & 0 & Optimal &  0.07 & 8 &  8.00 &  0.00\\
instance n=50 243.alb & 1 & 0 & Optimal &  0.12 & 7 &  7.00 &  0.00\\
instance n=50 244.alb & 1 & 0 & Optimal &  0.05 & 7 &  7.00 &  0.00\\
instance n=50 245.alb & 1 & 0 & Optimal &  0.04 & 7 &  7.00 &  0.00\\
instance n=50 246.alb & 1 & 0 & Optimal &  0.22 & 8 &  8.00 &  0.00\\
instance n=50 247.alb & 1 & 0 & Optimal &  0.05 & 7 &  7.00 &  0.00\\
instance n=50 248.alb & 1 & 0 & Optimal &  0.06 & 7 &  7.00 &  0.00\\
instance n=50 249.alb & 1 & 0 & Optimal &  0.18 & 7 &  7.00 &  0.00\\
instance n=50 25.alb & 1 & 0 & Optimal &  0.06 & 6 &  6.00 &  0.00\\
instance n=50 250.alb & 1 & 0 & Optimal &  0.04 & 7 &  7.00 &  0.00\\
instance n=50 251.alb & 1 & 0 & Optimal &  1.08 & 27 & 27.00 &  0.00\\
instance n=50 252.alb & 1 & 0 & Optimal &  4.59 & 32 & 32.00 &  0.00\\
instance n=50 253.alb & 1 & 0 & Optimal &  4.81 & 28 & 28.00 &  0.00\\
instance n=50 254.alb & 1 & 0 & Optimal &  0.06 & 30 & 30.00 &  0.00\\
instance n=50 255.alb & 1 & 0 & Optimal &  0.59 & 29 & 29.00 &  0.00\\
instance n=50 256.alb & 1 & 0 & Optimal &  0.39 & 30 & 30.00 &  0.00\\
instance n=50 257.alb & 1 & 0 & Optimal &  3.78 & 33 & 33.00 &  0.00\\
instance n=50 258.alb & 1 & 0 & Optimal &  4.75 & 28 & 28.00 &  0.00\\
instance n=50 259.alb & 1 & 0 & Optimal &  3.69 & 31 & 31.00 &  0.00\\
instance n=50 26.alb & 1 & 0 & Optimal & 83.72 & 27 & 27.00 &  0.00\\
instance n=50 260.alb & 1 & 0 & Optimal &  0.73 & 29 & 29.00 &  0.00\\
instance n=50 261.alb & 1 & 0 & Optimal &  2.79 & 28 & 28.00 &  0.00\\
instance n=50 262.alb & 1 & 0 & Optimal &  0.92 & 31 & 31.00 &  0.00\\
instance n=50 263.alb & 1 & 0 & Optimal &  0.92 & 29 & 29.00 &  0.00\\
instance n=50 264.alb & 1 & 0 & Optimal &  2.51 & 27 & 27.00 &  0.00\\
instance n=50 265.alb & 1 & 0 & Optimal &  0.81 & 27 & 27.00 &  0.00\\
instance n=50 266.alb & 1 & 0 & Optimal &  4.79 & 29 & 29.00 &  0.00\\
instance n=50 267.alb & 1 & 0 & Optimal &  5.15 & 28 & 28.00 &  0.00\\
instance n=50 268.alb & 1 & 0 & Optimal &  6.11 & 29 & 29.00 &  0.00\\
instance n=50 269.alb & 1 & 0 & Optimal &  0.56 & 26 & 26.00 &  0.00\\
instance n=50 27.alb & 1 & 0 & Optimal & 13.18 & 30 & 30.00 &  0.00\\
instance n=50 270.alb & 1 & 0 & Optimal &  0.31 & 28 & 28.00 &  0.00\\
instance n=50 271.alb & 1 & 0 & Optimal &  2.56 & 31 & 31.00 &  0.00\\
instance n=50 272.alb & 1 & 0 & Optimal &  2.09 & 27 & 27.00 &  0.00\\
instance n=50 273.alb & 1 & 0 & Optimal &  5.52 & 27 & 27.00 &  0.00\\
instance n=50 274.alb & 1 & 0 & Optimal &  0.07 & 29 & 29.00 &  0.00\\
instance n=50 275.alb & 1 & 0 & Optimal &  0.87 & 27 & 27.00 &  0.00\\
instance n=50 276.alb & 1 & 0 & Optimal &  0.06 & 12 & 12.00 &  0.00\\
instance n=50 277.alb & 1 & 0 & Optimal &  0.08 & 13 & 13.00 &  0.00\\
instance n=50 278.alb & 1 & 0 & Optimal &  0.10 & 12 & 12.00 &  0.00\\
instance n=50 279.alb & 1 & 0 & Optimal &  0.05 & 11 & 11.00 &  0.00\\
instance n=50 28.alb & 1 & 0 & Optimal &  0.08 & 28 & 28.00 &  0.00\\
instance n=50 280.alb & 1 & 0 & Optimal &  0.09 & 13 & 13.00 &  0.00\\
instance n=50 281.alb & 1 & 0 & Optimal &  0.08 & 11 & 11.00 &  0.00\\
instance n=50 282.alb & 1 & 0 & Optimal &  3.49 & 12 & 12.00 &  0.00\\
instance n=50 283.alb & 1 & 0 & Optimal &  0.27 & 12 & 12.00 &  0.00\\
instance n=50 284.alb & 1 & 0 & Optimal &  0.05 & 11 & 11.00 &  0.00\\
instance n=50 285.alb & 1 & 0 & Optimal &  0.20 & 13 & 13.00 &  0.00\\
instance n=50 286.alb & 1 & 0 & Optimal &  0.32 & 11 & 11.00 &  0.00\\
instance n=50 287.alb & 1 & 0 & Optimal &  0.96 & 12 & 12.00 &  0.00\\
instance n=50 288.alb & 1 & 0 & Optimal &  0.06 & 10 & 10.00 &  0.00\\
instance n=50 289.alb & 1 & 0 & Optimal &  0.24 & 11 & 11.00 &  0.00\\
instance n=50 29.alb & 1 & 0 & Optimal &  0.04 & 29 & 29.00 &  0.00\\
instance n=50 290.alb & 1 & 0 & Optimal &  0.09 & 14 & 14.00 &  0.00\\
instance n=50 291.alb & 1 & 0 & Optimal &  0.09 & 12 & 12.00 &  0.00\\
instance n=50 292.alb & 1 & 0 & Optimal &  0.07 & 13 & 13.00 &  0.00\\
instance n=50 293.alb & 1 & 0 & Optimal &  0.04 & 12 & 12.00 &  0.00\\
instance n=50 294.alb & 1 & 0 & Optimal &  0.07 & 13 & 13.00 &  0.00\\
instance n=50 295.alb & 1 & 0 & Optimal &  0.09 & 16 & 16.00 &  0.00\\
instance n=50 296.alb & 1 & 0 & Optimal &  0.15 & 13 & 13.00 &  0.00\\
instance n=50 297.alb & 1 & 0 & Optimal &  0.07 & 13 & 13.00 &  0.00\\
instance n=50 298.alb & 1 & 0 & Optimal &  0.07 & 11 & 11.00 &  0.00\\
instance n=50 299.alb & 1 & 0 & Optimal &  2.00 & 12 & 12.00 &  0.00\\
instance n=50 3.alb & 1 & 0 & Optimal &  0.31 & 8 &  8.00 &  0.00\\
instance n=50 30.alb & 1 & 0 & Optimal & 120.06 & 26 & 26.00 &  0.00\\
instance n=50 300.alb & 1 & 0 & Optimal &  0.04 & 12 & 12.00 &  0.00\\
instance n=50 301.alb & 1 & 0 & Optimal & 120.01 & 6 &  6.00 &  0.00\\
instance n=50 302.alb & 1 & 0 & Optimal & 120.03 & 7 &  7.00 &  0.00\\
instance n=50 303.alb & 1 & 0 & Optimal & 120.02 & 8 &  8.00 &  0.00\\
instance n=50 304.alb & 1 & 0 & Optimal &  0.47 & 7 &  7.00 &  0.00\\
instance n=50 305.alb & 1 & 0 & Optimal & 120.03 & 8 &  8.00 &  0.00\\
instance n=50 306.alb & 1 & 0 & Optimal & 36.00 & 7 &  7.00 &  0.00\\
instance n=50 307.alb & 1 & 0 & Optimal & 120.02 & 7 &  7.00 &  0.00\\
instance n=50 308.alb & 1 & 0 & Optimal &  2.72 & 8 &  8.00 &  0.00\\
instance n=50 309.alb & 1 & 0 & Optimal &  1.61 & 7 &  7.00 &  0.00\\
instance n=50 31.alb & 1 & 0 & Solution & 120.15 & 28 & 27.00 &  3.57\\
instance n=50 310.alb & 1 & 0 & Optimal &  0.04 & 8 &  8.00 &  0.00\\
instance n=50 311.alb & 1 & 0 & Optimal &  9.54 & 8 &  8.00 &  0.00\\
instance n=50 312.alb & 1 & 0 & Optimal &  0.26 & 6 &  6.00 &  0.00\\
instance n=50 313.alb & 1 & 0 & Optimal & 120.03 & 8 &  8.00 &  0.00\\
instance n=50 314.alb & 1 & 0 & Optimal & 17.64 & 7 &  7.00 &  0.00\\
instance n=50 315.alb & 1 & 0 & Optimal & 120.02 & 8 &  8.00 &  0.00\\
instance n=50 316.alb & 1 & 0 & Optimal &  0.70 & 8 &  8.00 &  0.00\\
instance n=50 317.alb & 1 & 0 & Optimal &  0.03 & 6 &  6.00 &  0.00\\
instance n=50 318.alb & 1 & 0 & Optimal &  0.16 & 8 &  8.00 &  0.00\\
instance n=50 319.alb & 1 & 0 & Optimal &  0.22 & 7 &  7.00 &  0.00\\
instance n=50 32.alb & 1 & 0 & Optimal & 26.75 & 25 & 25.00 &  0.00\\
instance n=50 320.alb & 1 & 0 & Optimal & 120.02 & 8 &  8.00 &  0.00\\
instance n=50 321.alb & 1 & 0 & Optimal &  0.03 & 6 &  6.00 &  0.00\\
instance n=50 322.alb & 1 & 0 & Optimal & 120.02 & 7 &  7.00 &  0.00\\
instance n=50 323.alb & 1 & 0 & Optimal & 120.02 & 7 &  7.00 &  0.00\\
instance n=50 324.alb & 1 & 0 & Optimal & 120.02 & 7 &  7.00 &  0.00\\
instance n=50 325.alb & 1 & 0 & Optimal &  0.24 & 7 &  7.00 &  0.00\\
instance n=50 326.alb & 1 & 0 & Optimal &  0.65 & 33 & 33.00 &  0.00\\
instance n=50 327.alb & 1 & 0 & Optimal & 113.77 & 28 & 28.00 &  0.00\\
instance n=50 328.alb & 1 & 0 & Optimal &  0.47 & 32 & 32.00 &  0.00\\
instance n=50 329.alb & 1 & 0 & Solution & 120.13 & 25 & 24.00 &  4.00\\
instance n=50 33.alb & 1 & 0 & Solution & 120.15 & 25 & 24.00 &  4.00\\
instance n=50 330.alb & 1 & 0 & Optimal &  0.07 & 29 & 29.00 &  0.00\\
instance n=50 331.alb & 1 & 0 & Optimal & 120.05 & 29 & 29.00 &  0.00\\
instance n=50 332.alb & 1 & 0 & Solution & 120.91 & 25 & 24.00 &  4.00\\
instance n=50 333.alb & 1 & 0 & Optimal &  5.15 & 28 & 28.00 &  0.00\\
instance n=50 334.alb & 1 & 0 & Optimal &  0.03 & 29 & 29.00 &  0.00\\
instance n=50 335.alb & 1 & 0 & Optimal & 120.06 & 27 & 27.00 &  0.00\\
instance n=50 336.alb & 1 & 0 & Solution & 120.11 & 26 & 25.00 &  3.85\\
instance n=50 337.alb & 1 & 0 & Optimal &  0.42 & 26 & 26.00 &  0.00\\
instance n=50 338.alb & 1 & 0 & Optimal & 82.25 & 26 & 26.00 &  0.00\\
instance n=50 339.alb & 1 & 0 & Optimal &  0.08 & 27 & 27.00 &  0.00\\
instance n=50 34.alb & 1 & 0 & Optimal &  0.11 & 30 & 30.00 &  0.00\\
instance n=50 340.alb & 1 & 0 & Solution & 120.13 & 28 & 27.00 &  3.57\\
instance n=50 341.alb & 1 & 0 & Optimal & 120.04 & 27 & 27.00 &  0.00\\
instance n=50 342.alb & 1 & 0 & Solution & 121.05 & 28 & 27.00 &  3.57\\
instance n=50 343.alb & 1 & 0 & Optimal & 120.05 & 27 & 27.00 &  0.00\\
instance n=50 344.alb & 1 & 0 & Optimal &  2.17 & 30 & 30.00 &  0.00\\
instance n=50 345.alb & 1 & 0 & Optimal & 120.04 & 29 & 29.00 &  0.00\\
instance n=50 346.alb & 1 & 0 & Optimal &  5.19 & 27 & 27.00 &  0.00\\
instance n=50 347.alb & 1 & 0 & Optimal & 110.19 & 25 & 25.00 &  0.00\\
instance n=50 348.alb & 1 & 0 & Optimal &  0.03 & 30 & 30.00 &  0.00\\
instance n=50 349.alb & 1 & 0 & Optimal &  0.95 & 28 & 28.00 &  0.00\\
instance n=50 35.alb & 1 & 0 & Optimal &  9.75 & 31 & 31.00 &  0.00\\
instance n=50 350.alb & 1 & 0 & Solution & 120.14 & 24 & 23.00 &  4.17\\
instance n=50 351.alb & 1 & 0 & Optimal &  0.03 & 12 & 12.00 &  0.00\\
instance n=50 352.alb & 1 & 0 & Optimal & 120.04 & 10 & 10.00 &  0.00\\
instance n=50 353.alb & 1 & 0 & Optimal &  0.06 & 13 & 13.00 &  0.00\\
instance n=50 354.alb & 1 & 0 & Solution & 120.11 & 14 & 13.00 &  7.14\\
instance n=50 355.alb & 1 & 0 & Optimal &  0.03 & 11 & 11.00 &  0.00\\
instance n=50 356.alb & 1 & 0 & Optimal &  0.05 & 15 & 15.00 &  0.00\\
instance n=50 357.alb & 1 & 0 & Optimal &  0.04 & 12 & 12.00 &  0.00\\
instance n=50 358.alb & 1 & 0 & Optimal &  0.17 & 11 & 11.00 &  0.00\\
instance n=50 359.alb & 1 & 0 & Optimal & 120.04 & 10 & 10.00 &  0.00\\
instance n=50 36.alb & 1 & 0 & Optimal &  0.26 & 31 & 31.00 &  0.00\\
instance n=50 360.alb & 1 & 0 & Optimal &  0.05 & 12 & 12.00 &  0.00\\
instance n=50 361.alb & 1 & 0 & Optimal &  0.27 & 11 & 11.00 &  0.00\\
instance n=50 362.alb & 1 & 0 & Optimal &  0.04 & 10 & 10.00 &  0.00\\
instance n=50 363.alb & 1 & 0 & Solution & 120.11 & 12 & 11.00 &  8.33\\
instance n=50 364.alb & 1 & 0 & Optimal &  0.55 & 13 & 13.00 &  0.00\\
instance n=50 365.alb & 1 & 0 & Optimal &  0.38 & 11 & 11.00 &  0.00\\
instance n=50 366.alb & 1 & 0 & Optimal &  0.03 & 13 & 13.00 &  0.00\\
instance n=50 367.alb & 1 & 0 & Optimal &  0.17 & 12 & 12.00 &  0.00\\
instance n=50 368.alb & 1 & 0 & Optimal &  0.05 & 12 & 12.00 &  0.00\\
instance n=50 369.alb & 1 & 0 & Optimal &  0.41 & 12 & 12.00 &  0.00\\
instance n=50 37.alb & 1 & 0 & Solution & 120.73 & 32 & 31.00 &  3.13\\
instance n=50 370.alb & 1 & 0 & Optimal &  0.05 & 12 & 12.00 &  0.00\\
instance n=50 371.alb & 1 & 0 & Optimal & 82.83 & 11 & 11.00 &  0.00\\
instance n=50 372.alb & 1 & 0 & Optimal & 120.04 & 10 & 10.00 &  0.00\\
instance n=50 373.alb & 1 & 0 & Optimal &  1.20 & 12 & 12.00 &  0.00\\
instance n=50 374.alb & 1 & 0 & Optimal &  0.04 & 11 & 11.00 &  0.00\\
instance n=50 375.alb & 1 & 0 & Optimal & 120.04 & 13 & 13.00 &  0.00\\
instance n=50 376.alb & 1 & 0 & Optimal &  0.05 & 7 &  7.00 &  0.00\\
instance n=50 377.alb & 1 & 0 & Optimal &  0.04 & 7 &  7.00 &  0.00\\
instance n=50 378.alb & 1 & 0 & Optimal &  0.06 & 8 &  8.00 &  0.00\\
instance n=50 379.alb & 1 & 0 & Optimal &  0.05 & 7 &  7.00 &  0.00\\
instance n=50 38.alb & 1 & 0 & Optimal &  0.36 & 31 & 31.00 &  0.00\\
instance n=50 380.alb & 1 & 0 & Optimal &  0.03 & 7 &  7.00 &  0.00\\
instance n=50 381.alb & 1 & 0 & Optimal &  0.33 & 8 &  8.00 &  0.00\\
instance n=50 382.alb & 1 & 0 & Optimal &  0.04 & 6 &  6.00 &  0.00\\
instance n=50 383.alb & 1 & 0 & Optimal &  0.08 & 7 &  7.00 &  0.00\\
instance n=50 384.alb & 1 & 0 & Optimal &  0.15 & 8 &  8.00 &  0.00\\
instance n=50 385.alb & 1 & 0 & Optimal &  0.04 & 7 &  7.00 &  0.00\\
instance n=50 386.alb & 1 & 0 & Optimal &  0.04 & 7 &  7.00 &  0.00\\
instance n=50 387.alb & 1 & 0 & Optimal &  0.06 & 8 &  8.00 &  0.00\\
instance n=50 388.alb & 1 & 0 & Optimal &  0.04 & 7 &  7.00 &  0.00\\
instance n=50 389.alb & 1 & 0 & Optimal &  0.04 & 8 &  8.00 &  0.00\\
instance n=50 39.alb & 1 & 0 & Solution & 120.14 & 29 & 28.00 &  3.45\\
instance n=50 390.alb & 1 & 0 & Optimal &  0.07 & 7 &  7.00 &  0.00\\
instance n=50 391.alb & 1 & 0 & Optimal &  0.03 & 7 &  7.00 &  0.00\\
instance n=50 392.alb & 1 & 0 & Optimal &  0.04 & 8 &  8.00 &  0.00\\
instance n=50 393.alb & 1 & 0 & Optimal &  0.06 & 7 &  7.00 &  0.00\\
instance n=50 394.alb & 1 & 0 & Optimal &  0.03 & 8 &  8.00 &  0.00\\
instance n=50 395.alb & 1 & 0 & Optimal &  0.04 & 7 &  7.00 &  0.00\\
instance n=50 396.alb & 1 & 0 & Optimal &  0.03 & 8 &  8.00 &  0.00\\
instance n=50 397.alb & 1 & 0 & Optimal &  0.04 & 7 &  7.00 &  0.00\\
instance n=50 398.alb & 1 & 0 & Optimal &  0.03 & 6 &  6.00 &  0.00\\
instance n=50 399.alb & 1 & 0 & Optimal &  4.46 & 7 &  7.00 &  0.00\\
instance n=50 4.alb & 1 & 0 & Optimal &  0.09 & 7 &  7.00 &  0.00\\
instance n=50 40.alb & 1 & 0 & Optimal & 120.04 & 26 & 26.00 &  0.00\\
instance n=50 400.alb & 1 & 0 & Optimal &  0.04 & 8 &  8.00 &  0.00\\
instance n=50 401.alb & 1 & 0 & Optimal & 59.04 & 28 & 28.00 &  0.00\\
instance n=50 402.alb & 1 & 0 & Optimal &  2.01 & 27 & 27.00 &  0.00\\
instance n=50 403.alb & 1 & 0 & Optimal &  2.30 & 34 & 34.00 &  0.00\\
instance n=50 404.alb & 1 & 0 & Optimal &  4.18 & 31 & 31.00 &  0.00\\
instance n=50 405.alb & 1 & 0 & Optimal &  3.45 & 27 & 27.00 &  0.00\\
instance n=50 406.alb & 1 & 0 & Optimal &  2.78 & 32 & 32.00 &  0.00\\
instance n=50 407.alb & 1 & 0 & Optimal &  6.19 & 29 & 29.00 &  0.00\\
instance n=50 408.alb & 1 & 0 & Optimal &  0.44 & 26 & 26.00 &  0.00\\
instance n=50 409.alb & 1 & 0 & Optimal &  5.51 & 33 & 33.00 &  0.00\\
instance n=50 41.alb & 1 & 0 & Optimal & 120.05 & 25 & 25.00 &  0.00\\
instance n=50 410.alb & 1 & 0 & Optimal &  0.30 & 28 & 28.00 &  0.00\\
instance n=50 411.alb & 1 & 0 & Optimal &  0.08 & 29 & 29.00 &  0.00\\
instance n=50 412.alb & 1 & 0 & Optimal &  0.12 & 26 & 26.00 &  0.00\\
instance n=50 413.alb & 1 & 0 & Optimal &  0.14 & 30 & 30.00 &  0.00\\
instance n=50 414.alb & 1 & 0 & Optimal & 33.49 & 27 & 27.00 &  0.00\\
instance n=50 415.alb & 1 & 0 & Optimal &  0.31 & 28 & 28.00 &  0.00\\
instance n=50 416.alb & 1 & 0 & Optimal &  0.19 & 27 & 27.00 &  0.00\\
instance n=50 417.alb & 1 & 0 & Optimal & 59.79 & 30 & 30.00 &  0.00\\
instance n=50 418.alb & 1 & 0 & Optimal &  1.04 & 27 & 27.00 &  0.00\\
instance n=50 419.alb & 1 & 0 & Optimal & 11.42 & 33 & 33.00 &  0.00\\
instance n=50 42.alb & 1 & 0 & Solution & 120.98 & 24 & 23.00 &  4.17\\
instance n=50 420.alb & 1 & 0 & Optimal & 13.02 & 28 & 28.00 &  0.00\\
instance n=50 421.alb & 1 & 0 & Optimal &  3.69 & 34 & 34.00 &  0.00\\
instance n=50 422.alb & 1 & 0 & Optimal &  3.03 & 29 & 29.00 &  0.00\\
instance n=50 423.alb & 1 & 0 & Optimal &  0.24 & 29 & 29.00 &  0.00\\
instance n=50 424.alb & 1 & 0 & Optimal &  0.80 & 27 & 27.00 &  0.00\\
instance n=50 425.alb & 1 & 0 & Optimal &  6.30 & 34 & 34.00 &  0.00\\
instance n=50 426.alb & 1 & 0 & Optimal &  0.30 & 11 & 11.00 &  0.00\\
instance n=50 427.alb & 1 & 0 & Optimal &  0.03 & 12 & 12.00 &  0.00\\
instance n=50 428.alb & 1 & 0 & Optimal &  0.14 & 13 & 13.00 &  0.00\\
instance n=50 429.alb & 1 & 0 & Optimal &  0.04 & 11 & 11.00 &  0.00\\
instance n=50 43.alb & 1 & 0 & Optimal &  1.58 & 25 & 25.00 &  0.00\\
instance n=50 430.alb & 1 & 0 & Optimal &  1.01 & 14 & 14.00 &  0.00\\
instance n=50 431.alb & 1 & 0 & Optimal &  0.05 & 11 & 11.00 &  0.00\\
instance n=50 432.alb & 1 & 0 & Optimal &  0.37 & 12 & 12.00 &  0.00\\
instance n=50 433.alb & 1 & 0 & Optimal &  0.04 & 12 & 12.00 &  0.00\\
instance n=50 434.alb & 1 & 0 & Optimal &  0.07 & 11 & 11.00 &  0.00\\
instance n=50 435.alb & 1 & 0 & Optimal &  0.54 & 11 & 11.00 &  0.00\\
instance n=50 436.alb & 1 & 0 & Optimal &  0.23 & 11 & 11.00 &  0.00\\
instance n=50 437.alb & 1 & 0 & Optimal &  1.92 & 12 & 12.00 &  0.00\\
instance n=50 438.alb & 1 & 0 & Optimal &  1.43 & 10 & 10.00 &  0.00\\
instance n=50 439.alb & 1 & 0 & Optimal &  0.45 & 12 & 12.00 &  0.00\\
instance n=50 44.alb & 1 & 0 & Solution & 120.15 & 25 & 24.00 &  4.00\\
instance n=50 440.alb & 1 & 0 & Optimal &  0.84 & 13 & 13.00 &  0.00\\
instance n=50 441.alb & 1 & 0 & Optimal &  0.06 & 11 & 11.00 &  0.00\\
instance n=50 442.alb & 1 & 0 & Optimal &  0.11 & 12 & 12.00 &  0.00\\
instance n=50 443.alb & 1 & 0 & Optimal &  0.06 & 11 & 11.00 &  0.00\\
instance n=50 444.alb & 1 & 0 & Optimal &  0.09 & 12 & 12.00 &  0.00\\
instance n=50 445.alb & 1 & 0 & Optimal &  0.24 & 12 & 12.00 &  0.00\\
instance n=50 446.alb & 1 & 0 & Optimal &  0.08 & 12 & 12.00 &  0.00\\
instance n=50 447.alb & 1 & 0 & Optimal &  0.08 & 13 & 13.00 &  0.00\\
instance n=50 448.alb & 1 & 0 & Optimal &  0.80 & 12 & 12.00 &  0.00\\
instance n=50 449.alb & 1 & 0 & Optimal &  0.07 & 11 & 11.00 &  0.00\\
instance n=50 45.alb & 1 & 0 & Solution & 120.12 & 25 & 24.00 &  4.00\\
instance n=50 450.alb & 1 & 0 & Optimal &  0.05 & 11 & 11.00 &  0.00\\
instance n=50 451.alb & 1 & 0 & Optimal &  0.06 & 8 &  8.00 &  0.00\\
instance n=50 452.alb & 1 & 0 & Optimal &  0.03 & 8 &  8.00 &  0.00\\
instance n=50 453.alb & 1 & 0 & Optimal &  0.02 & 7 &  7.00 &  0.00\\
instance n=50 454.alb & 1 & 0 & Optimal &  0.08 & 8 &  8.00 &  0.00\\
instance n=50 455.alb & 1 & 0 & Optimal &  0.02 & 6 &  6.00 &  0.00\\
instance n=50 456.alb & 1 & 0 & Optimal &  0.04 & 8 &  8.00 &  0.00\\
instance n=50 457.alb & 1 & 0 & Optimal &  0.03 & 8 &  8.00 &  0.00\\
instance n=50 458.alb & 1 & 0 & Optimal &  0.03 & 7 &  7.00 &  0.00\\
instance n=50 459.alb & 1 & 0 & Optimal &  0.01 & 7 &  7.00 &  0.00\\
instance n=50 46.alb & 1 & 0 & Optimal &  0.31 & 28 & 28.00 &  0.00\\
instance n=50 460.alb & 1 & 0 & Optimal &  0.04 & 7 &  7.00 &  0.00\\
instance n=50 461.alb & 1 & 0 & Optimal &  0.01 & 6 &  6.00 &  0.00\\
instance n=50 462.alb & 1 & 0 & Optimal &  0.04 & 7 &  7.00 &  0.00\\
instance n=50 463.alb & 1 & 0 & Optimal &  0.04 & 8 &  8.00 &  0.00\\
instance n=50 464.alb & 1 & 0 & Optimal &  0.03 & 6 &  6.00 &  0.00\\
instance n=50 465.alb & 1 & 0 & Optimal &  0.03 & 8 &  8.00 &  0.00\\
instance n=50 466.alb & 1 & 0 & Optimal &  0.03 & 7 &  7.00 &  0.00\\
instance n=50 467.alb & 1 & 0 & Optimal &  0.08 & 9 &  9.00 &  0.00\\
instance n=50 468.alb & 1 & 0 & Optimal &  0.05 & 7 &  7.00 &  0.00\\
instance n=50 469.alb & 1 & 0 & Optimal &  0.07 & 8 &  8.00 &  0.00\\
instance n=50 47.alb & 1 & 0 & Optimal &  3.94 & 28 & 28.00 &  0.00\\
instance n=50 470.alb & 1 & 0 & Optimal &  0.03 & 8 &  8.00 &  0.00\\
instance n=50 471.alb & 1 & 0 & Optimal &  0.06 & 7 &  7.00 &  0.00\\
instance n=50 472.alb & 1 & 0 & Optimal &  0.03 & 8 &  8.00 &  0.00\\
instance n=50 473.alb & 1 & 0 & Optimal &  0.02 & 7 &  7.00 &  0.00\\
instance n=50 474.alb & 1 & 0 & Optimal &  0.01 & 7 &  7.00 &  0.00\\
instance n=50 475.alb & 1 & 0 & Optimal &  0.05 & 6 &  6.00 &  0.00\\
instance n=50 476.alb & 1 & 0 & Optimal &  0.08 & 28 & 28.00 &  0.00\\
instance n=50 477.alb & 1 & 0 & Optimal &  0.06 & 29 & 29.00 &  0.00\\
instance n=50 478.alb & 1 & 0 & Optimal &  0.12 & 32 & 32.00 &  0.00\\
instance n=50 479.alb & 1 & 0 & Optimal &  0.06 & 28 & 28.00 &  0.00\\
instance n=50 48.alb & 1 & 0 & Optimal &  0.56 & 27 & 27.00 &  0.00\\
instance n=50 480.alb & 1 & 0 & Optimal &  0.04 & 34 & 34.00 &  0.00\\
instance n=50 481.alb & 1 & 0 & Optimal &  0.06 & 28 & 28.00 &  0.00\\
instance n=50 482.alb & 1 & 0 & Optimal &  0.03 & 27 & 27.00 &  0.00\\
instance n=50 483.alb & 1 & 0 & Optimal &  0.11 & 30 & 30.00 &  0.00\\
instance n=50 484.alb & 1 & 0 & Optimal &  0.03 & 32 & 32.00 &  0.00\\
instance n=50 485.alb & 1 & 0 & Optimal &  0.08 & 31 & 31.00 &  0.00\\
instance n=50 486.alb & 1 & 0 & Optimal &  0.04 & 32 & 32.00 &  0.00\\
instance n=50 487.alb & 1 & 0 & Optimal &  0.12 & 31 & 31.00 &  0.00\\
instance n=50 488.alb & 1 & 0 & Optimal &  0.06 & 31 & 31.00 &  0.00\\
instance n=50 489.alb & 1 & 0 & Optimal &  0.05 & 35 & 35.00 &  0.00\\
instance n=50 49.alb & 1 & 0 & Optimal &  0.58 & 25 & 25.00 &  0.00\\
instance n=50 490.alb & 1 & 0 & Optimal &  0.03 & 29 & 29.00 &  0.00\\
instance n=50 491.alb & 1 & 0 & Optimal &  0.05 & 35 & 35.00 &  0.00\\
instance n=50 492.alb & 1 & 0 & Optimal &  0.04 & 29 & 29.00 &  0.00\\
instance n=50 493.alb & 1 & 0 & Optimal &  0.11 & 30 & 30.00 &  0.00\\
instance n=50 494.alb & 1 & 0 & Optimal &  0.04 & 32 & 32.00 &  0.00\\
instance n=50 495.alb & 1 & 0 & Optimal &  0.06 & 34 & 34.00 &  0.00\\
instance n=50 496.alb & 1 & 0 & Optimal &  0.09 & 29 & 29.00 &  0.00\\
instance n=50 497.alb & 1 & 0 & Optimal &  0.12 & 30 & 30.00 &  0.00\\
instance n=50 498.alb & 1 & 0 & Optimal &  0.08 & 30 & 30.00 &  0.00\\
instance n=50 499.alb & 1 & 0 & Optimal &  0.06 & 33 & 33.00 &  0.00\\
instance n=50 5.alb & 1 & 0 & Optimal &  0.05 & 7 &  7.00 &  0.00\\
instance n=50 50.alb & 1 & 0 & Solution & 120.15 & 27 & 26.00 &  3.70\\
instance n=50 500.alb & 1 & 0 & Optimal &  0.04 & 34 & 34.00 &  0.00\\
instance n=50 501.alb & 1 & 0 & Optimal &  0.04 & 12 & 12.00 &  0.00\\
instance n=50 502.alb & 1 & 0 & Optimal &  0.03 & 10 & 10.00 &  0.00\\
instance n=50 503.alb & 1 & 0 & Optimal &  0.06 & 13 & 13.00 &  0.00\\
instance n=50 504.alb & 1 & 0 & Optimal &  0.05 & 11 & 11.00 &  0.00\\
instance n=50 505.alb & 1 & 0 & Optimal &  0.03 & 12 & 12.00 &  0.00\\
instance n=50 506.alb & 1 & 0 & Optimal &  0.04 & 11 & 11.00 &  0.00\\
instance n=50 507.alb & 1 & 0 & Optimal &  0.05 & 13 & 13.00 &  0.00\\
instance n=50 508.alb & 1 & 0 & Optimal &  0.04 & 14 & 14.00 &  0.00\\
instance n=50 509.alb & 1 & 0 & Optimal &  0.04 & 13 & 13.00 &  0.00\\
instance n=50 51.alb & 1 & 0 & Optimal &  0.53 & 12 & 12.00 &  0.00\\
instance n=50 510.alb & 1 & 0 & Optimal &  0.08 & 11 & 11.00 &  0.00\\
instance n=50 511.alb & 1 & 0 & Optimal &  0.03 & 13 & 13.00 &  0.00\\
instance n=50 512.alb & 1 & 0 & Optimal &  0.05 & 13 & 13.00 &  0.00\\
instance n=50 513.alb & 1 & 0 & Optimal &  0.07 & 12 & 12.00 &  0.00\\
instance n=50 514.alb & 1 & 0 & Optimal &  0.05 & 12 & 12.00 &  0.00\\
instance n=50 515.alb & 1 & 0 & Optimal &  0.05 & 11 & 11.00 &  0.00\\
instance n=50 516.alb & 1 & 0 & Optimal &  0.04 & 13 & 13.00 &  0.00\\
instance n=50 517.alb & 1 & 0 & Optimal &  0.04 & 14 & 14.00 &  0.00\\
instance n=50 518.alb & 1 & 0 & Optimal &  0.04 & 11 & 11.00 &  0.00\\
instance n=50 519.alb & 1 & 0 & Optimal &  0.04 & 12 & 12.00 &  0.00\\
instance n=50 52.alb & 1 & 0 & Optimal &  0.03 & 11 & 11.00 &  0.00\\
instance n=50 520.alb & 1 & 0 & Optimal &  0.04 & 11 & 11.00 &  0.00\\
instance n=50 521.alb & 1 & 0 & Optimal &  0.05 & 10 & 10.00 &  0.00\\
instance n=50 522.alb & 1 & 0 & Optimal &  0.04 & 11 & 11.00 &  0.00\\
instance n=50 523.alb & 1 & 0 & Optimal &  0.05 & 11 & 11.00 &  0.00\\
instance n=50 524.alb & 1 & 0 & Optimal &  0.04 & 14 & 14.00 &  0.00\\
instance n=50 525.alb & 1 & 0 & Optimal &  0.08 & 11 & 11.00 &  0.00\\
instance n=50 53.alb & 1 & 0 & Solution & 120.12 & 13 & 12.00 &  7.69\\
instance n=50 54.alb & 1 & 0 & Optimal &  0.04 & 11 & 11.00 &  0.00\\
instance n=50 55.alb & 1 & 0 & Optimal &  0.05 & 13 & 13.00 &  0.00\\
instance n=50 56.alb & 1 & 0 & Optimal &  0.04 & 11 & 11.00 &  0.00\\
instance n=50 57.alb & 1 & 0 & Optimal &  0.07 & 13 & 13.00 &  0.00\\
instance n=50 58.alb & 1 & 0 & Optimal &  0.12 & 11 & 11.00 &  0.00\\
instance n=50 59.alb & 1 & 0 & Optimal &  7.70 & 11 & 11.00 &  0.00\\
instance n=50 6.alb & 1 & 0 & Optimal &  0.06 & 6 &  6.00 &  0.00\\
instance n=50 60.alb & 1 & 0 & Optimal &  0.30 & 12 & 12.00 &  0.00\\
instance n=50 61.alb & 1 & 0 & Optimal &  0.06 & 13 & 13.00 &  0.00\\
instance n=50 62.alb & 1 & 0 & Optimal &  0.03 & 13 & 13.00 &  0.00\\
instance n=50 63.alb & 1 & 0 & Optimal & 120.04 & 12 & 12.00 &  0.00\\
instance n=50 64.alb & 1 & 0 & Optimal &  0.05 & 13 & 13.00 &  0.00\\
instance n=50 65.alb & 1 & 0 & Optimal &  1.96 & 12 & 12.00 &  0.00\\
instance n=50 66.alb & 1 & 0 & Optimal &  0.76 & 12 & 12.00 &  0.00\\
instance n=50 67.alb & 1 & 0 & Optimal &  0.33 & 12 & 12.00 &  0.00\\
instance n=50 68.alb & 1 & 0 & Optimal &  0.06 & 12 & 12.00 &  0.00\\
instance n=50 69.alb & 1 & 0 & Optimal &  0.14 & 12 & 12.00 &  0.00\\
instance n=50 7.alb & 1 & 0 & Optimal &  0.03 & 7 &  7.00 &  0.00\\
instance n=50 70.alb & 1 & 0 & Optimal &  0.04 & 10 & 10.00 &  0.00\\
instance n=50 71.alb & 1 & 0 & Optimal &  0.15 & 13 & 13.00 &  0.00\\
instance n=50 72.alb & 1 & 0 & Optimal & 37.28 & 11 & 11.00 &  0.00\\
instance n=50 73.alb & 1 & 0 & Optimal &  0.09 & 11 & 11.00 &  0.00\\
instance n=50 74.alb & 1 & 0 & Optimal & 32.81 & 12 & 12.00 &  0.00\\
instance n=50 75.alb & 1 & 0 & Optimal &  0.32 & 11 & 11.00 &  0.00\\
instance n=50 76.alb & 1 & 0 & Optimal &  0.03 & 7 &  7.00 &  0.00\\
instance n=50 77.alb & 1 & 0 & Optimal &  0.04 & 7 &  7.00 &  0.00\\
instance n=50 78.alb & 1 & 0 & Optimal &  1.81 & 7 &  7.00 &  0.00\\
instance n=50 79.alb & 1 & 0 & Optimal &  0.28 & 8 &  8.00 &  0.00\\
instance n=50 8.alb & 1 & 0 & Optimal &  4.95 & 7 &  7.00 &  0.00\\
instance n=50 80.alb & 1 & 0 & Optimal &  1.83 & 7 &  7.00 &  0.00\\
instance n=50 81.alb & 1 & 0 & Optimal &  0.03 & 7 &  7.00 &  0.00\\
instance n=50 82.alb & 1 & 0 & Optimal &  0.04 & 6 &  6.00 &  0.00\\
instance n=50 83.alb & 1 & 0 & Optimal &  0.07 & 8 &  8.00 &  0.00\\
instance n=50 84.alb & 1 & 0 & Optimal &  0.03 & 7 &  7.00 &  0.00\\
instance n=50 85.alb & 1 & 0 & Optimal &  0.05 & 8 &  8.00 &  0.00\\
instance n=50 86.alb & 1 & 0 & Optimal &  0.05 & 7 &  7.00 &  0.00\\
instance n=50 87.alb & 1 & 0 & Optimal &  0.08 & 8 &  8.00 &  0.00\\
instance n=50 88.alb & 1 & 0 & Optimal &  1.30 & 8 &  8.00 &  0.00\\
instance n=50 89.alb & 1 & 0 & Optimal &  0.07 & 7 &  7.00 &  0.00\\
instance n=50 9.alb & 1 & 0 & Optimal &  1.13 & 9 &  9.00 &  0.00\\
instance n=50 90.alb & 1 & 0 & Optimal &  0.05 & 7 &  7.00 &  0.00\\
instance n=50 91.alb & 1 & 0 & Optimal &  3.30 & 7 &  7.00 &  0.00\\
instance n=50 92.alb & 1 & 0 & Optimal &  0.06 & 7 &  7.00 &  0.00\\
instance n=50 93.alb & 1 & 0 & Optimal &  0.03 & 7 &  7.00 &  0.00\\
instance n=50 94.alb & 1 & 0 & Optimal &  0.92 & 7 &  7.00 &  0.00\\
instance n=50 95.alb & 1 & 0 & Optimal &  0.06 & 7 &  7.00 &  0.00\\
instance n=50 96.alb & 1 & 0 & Optimal &  0.06 & 7 &  7.00 &  0.00\\
instance n=50 97.alb & 1 & 0 & Optimal &  0.04 & 7 &  7.00 &  0.00\\
instance n=50 98.alb & 1 & 0 & Optimal &  0.12 & 8 &  8.00 &  0.00\\
instance n=50 99.alb & 1 & 0 & Optimal &  0.09 & 7 &  7.00 &  0.00\\
\end{longtable}



\section{Results for MiniZinc/Cplex}

\begin{longtable}{lrrlrrrr}
\caption{Results for SALBP-1 Problems (Cplex) (1575 Instances)}\\\toprule
Name & \shortstack{Nr\\Jobs} & \shortstack{Nr\\Machines} & Status & Time & Makespan & Bound & \shortstack{Gap\\Percent}\\ \midrule
\endhead
\bottomrule
\endfoot
instance n=100 1.alb & 1 & 0 & Solution & 120.54 & 24 &  0.00 &  0.00\\
instance n=100 10.alb & 1 & 0 & Unknown & 120491.00 & - & - & -\\
instance n=100 100.alb & 1 & 0 & Unknown & 120526.00 & - & - & -\\
instance n=100 101.alb & 1 & 0 & Unknown & 120439.00 & - & - & -\\
instance n=100 102.alb & 1 & 0 & Unknown & 120532.00 & - & - & -\\
instance n=100 103.alb & 1 & 0 & Unknown & 120496.00 & - & - & -\\
instance n=100 104.alb & 1 & 0 & Unknown & 120518.00 & - & - & -\\
instance n=100 105.alb & 1 & 0 & Unknown & 120488.00 & - & - & -\\
instance n=100 106.alb & 1 & 0 & Solution & 120.53 & 41 &  0.00 &  0.00\\
instance n=100 107.alb & 1 & 0 & Unknown & 120483.00 & - & - & -\\
instance n=100 108.alb & 1 & 0 & Unknown & 120512.00 & - & - & -\\
instance n=100 109.alb & 1 & 0 & Unknown & 120507.00 & - & - & -\\
instance n=100 11.alb & 1 & 0 & Solution & 120.52 & 53 &  0.00 &  0.00\\
instance n=100 110.alb & 1 & 0 & Solution & 120.53 & 45 &  0.00 &  0.00\\
instance n=100 111.alb & 1 & 0 & Unknown & 120525.00 & - & - & -\\
instance n=100 112.alb & 1 & 0 & Unknown & 120447.00 & - & - & -\\
instance n=100 113.alb & 1 & 0 & Solution & 120.52 & 34 &  0.00 &  0.00\\
instance n=100 114.alb & 1 & 0 & Unknown & 120519.00 & - & - & -\\
instance n=100 115.alb & 1 & 0 & Solution & 120.45 & 20 &  0.00 &  0.00\\
instance n=100 116.alb & 1 & 0 & Unknown & 120494.00 & - & - & -\\
instance n=100 117.alb & 1 & 0 & Unknown & 120501.00 & - & - & -\\
instance n=100 118.alb & 1 & 0 & Solution & 120.51 & 43 &  0.00 &  0.00\\
instance n=100 119.alb & 1 & 0 & Unknown & 120534.00 & - & - & -\\
instance n=100 12.alb & 1 & 0 & Solution & 120.44 & 27 &  0.00 &  0.00\\
instance n=100 120.alb & 1 & 0 & Unknown & 120529.00 & - & - & -\\
instance n=100 121.alb & 1 & 0 & Unknown & 120528.00 & - & - & -\\
instance n=100 122.alb & 1 & 0 & Solution & 120.55 & 41 &  0.00 &  0.00\\
instance n=100 123.alb & 1 & 0 & Unknown & 120507.00 & - & - & -\\
instance n=100 124.alb & 1 & 0 & Unknown & 120507.00 & - & - & -\\
instance n=100 125.alb & 1 & 0 & Solution & 120.51 & 48 &  0.00 &  0.00\\
instance n=100 126.alb & 1 & 0 & Solution & 120.46 & 77 &  0.00 &  0.00\\
instance n=100 127.alb & 1 & 0 & Solution & 120.45 & 78 &  0.00 &  0.00\\
instance n=100 128.alb & 1 & 0 & Solution & 120.50 & 70 &  0.00 &  0.00\\
instance n=100 129.alb & 1 & 0 & Solution & 120.49 & 67 &  0.00 &  0.00\\
instance n=100 13.alb & 1 & 0 & Solution & 120.52 & 25 &  0.00 &  0.00\\
instance n=100 130.alb & 1 & 0 & Solution & 120.50 & 79 &  0.00 &  0.00\\
instance n=100 131.alb & 1 & 0 & Solution & 120.46 & 69 &  0.00 &  0.00\\
instance n=100 132.alb & 1 & 0 & Solution & 120.46 & 70 &  0.00 &  0.00\\
instance n=100 133.alb & 1 & 0 & Solution & 120.46 & 86 &  0.00 &  0.00\\
instance n=100 134.alb & 1 & 0 & Solution & 120.47 & 72 &  0.00 &  0.00\\
instance n=100 135.alb & 1 & 0 & Solution & 120.48 & 71 &  0.00 &  0.00\\
instance n=100 136.alb & 1 & 0 & Solution & 120.46 & 72 &  0.00 &  0.00\\
instance n=100 137.alb & 1 & 0 & Solution & 120.48 & 75 &  0.00 &  0.00\\
instance n=100 138.alb & 1 & 0 & Solution & 120.49 & 70 &  0.00 &  0.00\\
instance n=100 139.alb & 1 & 0 & Solution & 120.47 & 70 &  0.00 &  0.00\\
instance n=100 14.alb & 1 & 0 & Unknown & 120503.00 & - & - & -\\
instance n=100 140.alb & 1 & 0 & Solution & 120.52 & 65 &  0.00 &  0.00\\
instance n=100 141.alb & 1 & 0 & Solution & 120.50 & 70 &  0.00 &  0.00\\
instance n=100 142.alb & 1 & 0 & Solution & 120.47 & 68 &  0.00 &  0.00\\
instance n=100 143.alb & 1 & 0 & Solution & 120.46 & 84 &  0.00 &  0.00\\
instance n=100 144.alb & 1 & 0 & Solution & 120.47 & 59 &  0.00 &  0.00\\
instance n=100 145.alb & 1 & 0 & Unknown & 120503.00 & - & - & -\\
instance n=100 146.alb & 1 & 0 & Solution & 120.46 & 63 &  0.00 &  0.00\\
instance n=100 147.alb & 1 & 0 & Solution & 120.45 & 73 &  0.00 &  0.00\\
instance n=100 148.alb & 1 & 0 & Solution & 120.47 & 65 &  0.00 &  0.00\\
instance n=100 149.alb & 1 & 0 & Solution & 120.47 & 69 &  0.00 &  0.00\\
instance n=100 15.alb & 1 & 0 & Solution & 120.51 & 93 &  0.00 &  0.00\\
instance n=100 150.alb & 1 & 0 & Solution & 120.47 & 72 &  0.00 &  0.00\\
instance n=100 151.alb & 1 & 0 & Unknown & 120511.00 & - & - & -\\
instance n=100 152.alb & 1 & 0 & Unknown & 120509.00 & - & - & -\\
instance n=100 153.alb & 1 & 0 & Unknown & 120490.00 & - & - & -\\
instance n=100 154.alb & 1 & 0 & Unknown & 120496.00 & - & - & -\\
instance n=100 155.alb & 1 & 0 & Unknown & 120520.00 & - & - & -\\
instance n=100 156.alb & 1 & 0 & Solution & 120.56 & 97 &  0.00 &  0.00\\
instance n=100 157.alb & 1 & 0 & Solution & 120.42 & 60 &  0.00 &  0.00\\
instance n=100 158.alb & 1 & 0 & Unknown & 120507.00 & - & - & -\\
instance n=100 159.alb & 1 & 0 & Unknown & 120463.00 & - & - & -\\
instance n=100 16.alb & 1 & 0 & Solution & 120.47 & 43 &  0.00 &  0.00\\
instance n=100 160.alb & 1 & 0 & Unknown & 120514.00 & - & - & -\\
instance n=100 161.alb & 1 & 0 & Unknown & 120488.00 & - & - & -\\
instance n=100 162.alb & 1 & 0 & Unknown & 120529.00 & - & - & -\\
instance n=100 163.alb & 1 & 0 & Unknown & 120538.00 & - & - & -\\
instance n=100 164.alb & 1 & 0 & Solution & 120.63 & 53 &  0.00 &  0.00\\
instance n=100 165.alb & 1 & 0 & Unknown & 120518.00 & - & - & -\\
instance n=100 166.alb & 1 & 0 & Unknown & 120593.00 & - & - & -\\
instance n=100 167.alb & 1 & 0 & Unknown & 120500.00 & - & - & -\\
instance n=100 168.alb & 1 & 0 & Unknown & 120511.00 & - & - & -\\
instance n=100 169.alb & 1 & 0 & Unknown & 120601.00 & - & - & -\\
instance n=100 17.alb & 1 & 0 & Unknown & 120858.00 & - & - & -\\
instance n=100 170.alb & 1 & 0 & Solution & 120.56 & 52 &  0.00 &  0.00\\
instance n=100 171.alb & 1 & 0 & Solution & 120.52 & 25 &  0.00 &  0.00\\
instance n=100 172.alb & 1 & 0 & Unknown & 120501.00 & - & - & -\\
instance n=100 173.alb & 1 & 0 & Solution & 120.59 & 26 &  0.00 &  0.00\\
instance n=100 174.alb & 1 & 0 & Unknown & 120500.00 & - & - & -\\
instance n=100 175.alb & 1 & 0 & Solution & 120.46 & 100 &  0.00 &  0.00\\
instance n=100 176.alb & 1 & 0 & Solution & 120.52 & 14 &  0.00 &  0.00\\
instance n=100 177.alb & 1 & 0 & Solution & 120.51 & 19 &  0.00 &  0.00\\
instance n=100 178.alb & 1 & 0 & Unknown & 120481.00 & - & - & -\\
instance n=100 179.alb & 1 & 0 & Solution & 120.50 & 16 &  0.00 &  0.00\\
instance n=100 18.alb & 1 & 0 & Unknown & 120510.00 & - & - & -\\
instance n=100 180.alb & 1 & 0 & Solution & 120.53 & 15 &  0.00 &  0.00\\
instance n=100 181.alb & 1 & 0 & Solution & 120.55 & 14 &  0.00 &  0.00\\
instance n=100 182.alb & 1 & 0 & Solution & 120.50 & 15 &  0.00 &  0.00\\
instance n=100 183.alb & 1 & 0 & Solution & 120.60 & 31 &  0.00 &  0.00\\
instance n=100 184.alb & 1 & 0 & Unknown & 120490.00 & - & - & -\\
instance n=100 185.alb & 1 & 0 & Unknown & 120550.00 & - & - & -\\
instance n=100 186.alb & 1 & 0 & Solution & 120.53 & 15 &  0.00 &  0.00\\
instance n=100 187.alb & 1 & 0 & Solution & 120.51 & 14 &  0.00 &  0.00\\
instance n=100 188.alb & 1 & 0 & Unknown & 120501.00 & - & - & -\\
instance n=100 189.alb & 1 & 0 & Unknown & 120517.00 & - & - & -\\
instance n=100 19.alb & 1 & 0 & Unknown & 120586.00 & - & - & -\\
instance n=100 190.alb & 1 & 0 & Solution & 120.58 & 14 &  0.00 &  0.00\\
instance n=100 191.alb & 1 & 0 & Solution & 120.50 & 14 &  0.00 &  0.00\\
instance n=100 192.alb & 1 & 0 & Solution & 120.55 & 14 &  0.00 &  0.00\\
instance n=100 193.alb & 1 & 0 & Solution & 120.58 & 62 &  0.00 &  0.00\\
instance n=100 194.alb & 1 & 0 & Unknown & 120489.00 & - & - & -\\
instance n=100 195.alb & 1 & 0 & Solution & 120.52 & 36 &  0.00 &  0.00\\
instance n=100 196.alb & 1 & 0 & Solution & 120.47 & 15 &  0.00 &  0.00\\
instance n=100 197.alb & 1 & 0 & Solution & 120.51 & 100 &  0.00 &  0.00\\
instance n=100 198.alb & 1 & 0 & Solution & 120.43 & 37 &  0.00 &  0.00\\
instance n=100 199.alb & 1 & 0 & Solution & 120.53 & 20 &  0.00 &  0.00\\
instance n=100 2.alb & 1 & 0 & Unknown & 120483.00 & - & - & -\\
instance n=100 20.alb & 1 & 0 & Unknown & 120514.00 & - & - & -\\
instance n=100 200.alb & 1 & 0 & Unknown & 120493.00 & - & - & -\\
instance n=100 201.alb & 1 & 0 & Solution & 120.51 & 72 &  0.00 &  0.00\\
instance n=100 202.alb & 1 & 0 & Solution & 120.45 & 100 &  0.00 &  0.00\\
instance n=100 203.alb & 1 & 0 & Unknown & 120441.00 & - & - & -\\
instance n=100 204.alb & 1 & 0 & Unknown & 120503.00 & - & - & -\\
instance n=100 205.alb & 1 & 0 & Solution & 120.52 & 75 &  0.00 &  0.00\\
instance n=100 206.alb & 1 & 0 & Unknown & 120537.00 & - & - & -\\
instance n=100 207.alb & 1 & 0 & Unknown & 120503.00 & - & - & -\\
instance n=100 208.alb & 1 & 0 & Unknown & 120503.00 & - & - & -\\
instance n=100 209.alb & 1 & 0 & Solution & 120.49 & 73 &  0.00 &  0.00\\
instance n=100 21.alb & 1 & 0 & Solution & 120.47 & 22 &  0.00 &  0.00\\
instance n=100 210.alb & 1 & 0 & Solution & 120.47 & 65 &  0.00 &  0.00\\
instance n=100 211.alb & 1 & 0 & Solution & 120.49 & 79 &  0.00 &  0.00\\
instance n=100 212.alb & 1 & 0 & Solution & 120.60 & 69 &  0.00 &  0.00\\
instance n=100 213.alb & 1 & 0 & Solution & 120.48 & 74 &  0.00 &  0.00\\
instance n=100 214.alb & 1 & 0 & Solution & 120.50 & 71 &  0.00 &  0.00\\
instance n=100 215.alb & 1 & 0 & Unknown & 120503.00 & - & - & -\\
instance n=100 216.alb & 1 & 0 & Unknown & 120495.00 & - & - & -\\
instance n=100 217.alb & 1 & 0 & Solution & 120.49 & 70 &  0.00 &  0.00\\
instance n=100 218.alb & 1 & 0 & Solution & 120.50 & 65 &  0.00 &  0.00\\
instance n=100 219.alb & 1 & 0 & Solution & 120.48 & 77 &  0.00 &  0.00\\
instance n=100 22.alb & 1 & 0 & Unknown & 120530.00 & - & - & -\\
instance n=100 220.alb & 1 & 0 & Solution & 120.48 & 88 &  0.00 &  0.00\\
instance n=100 221.alb & 1 & 0 & Solution & 120.49 & 97 &  0.00 &  0.00\\
instance n=100 222.alb & 1 & 0 & Solution & 120.51 & 82 &  0.00 &  0.00\\
instance n=100 223.alb & 1 & 0 & Unknown & 120476.00 & - & - & -\\
instance n=100 224.alb & 1 & 0 & Unknown & 120479.00 & - & - & -\\
instance n=100 225.alb & 1 & 0 & Solution & 120.49 & 74 &  0.00 &  0.00\\
instance n=100 226.alb & 1 & 0 & Unknown & 120503.00 & - & - & -\\
instance n=100 227.alb & 1 & 0 & Unknown & 120447.00 & - & - & -\\
instance n=100 228.alb & 1 & 0 & Unknown & 120506.00 & - & - & -\\
instance n=100 229.alb & 1 & 0 & Unknown & 120506.00 & - & - & -\\
instance n=100 23.alb & 1 & 0 & Unknown & 120499.00 & - & - & -\\
instance n=100 230.alb & 1 & 0 & Unknown & 120470.00 & - & - & -\\
instance n=100 231.alb & 1 & 0 & Unknown & 120532.00 & - & - & -\\
instance n=100 232.alb & 1 & 0 & Solution & 120.45 & 99 &  0.00 &  0.00\\
instance n=100 233.alb & 1 & 0 & Unknown & 120536.00 & - & - & -\\
instance n=100 234.alb & 1 & 0 & Unknown & 120628.00 & - & - & -\\
instance n=100 235.alb & 1 & 0 & Solution & 120.42 & 54 &  0.00 &  0.00\\
instance n=100 236.alb & 1 & 0 & Solution & 120.53 & 25 &  0.00 &  0.00\\
instance n=100 237.alb & 1 & 0 & Unknown & 120424.00 & - & - & -\\
instance n=100 238.alb & 1 & 0 & Unknown & 120542.00 & - & - & -\\
instance n=100 239.alb & 1 & 0 & Solution & 120.47 & 36 &  0.00 &  0.00\\
instance n=100 24.alb & 1 & 0 & Solution & 120.53 & 42 &  0.00 &  0.00\\
instance n=100 240.alb & 1 & 0 & Unknown & 120465.00 & - & - & -\\
instance n=100 241.alb & 1 & 0 & Unknown & 120512.00 & - & - & -\\
instance n=100 242.alb & 1 & 0 & Unknown & 120522.00 & - & - & -\\
instance n=100 243.alb & 1 & 0 & Unknown & 120429.00 & - & - & -\\
instance n=100 244.alb & 1 & 0 & Unknown & 120565.00 & - & - & -\\
instance n=100 245.alb & 1 & 0 & Unknown & 120493.00 & - & - & -\\
instance n=100 246.alb & 1 & 0 & Solution & 120.44 & 89 &  0.00 &  0.00\\
instance n=100 247.alb & 1 & 0 & Unknown & 120521.00 & - & - & -\\
instance n=100 248.alb & 1 & 0 & Solution & 120.52 & 36 &  0.00 &  0.00\\
instance n=100 249.alb & 1 & 0 & Unknown & 120506.00 & - & - & -\\
instance n=100 25.alb & 1 & 0 & Solution & 120.47 & 25 &  0.00 &  0.00\\
instance n=100 250.alb & 1 & 0 & Unknown & 120483.00 & - & - & -\\
instance n=100 251.alb & 1 & 0 & Solution & 120.47 & 25 &  0.00 &  0.00\\
instance n=100 252.alb & 1 & 0 & Unknown & 120954.00 & - & - & -\\
instance n=100 253.alb & 1 & 0 & Solution & 120.50 & 26 &  0.00 &  0.00\\
instance n=100 254.alb & 1 & 0 & Solution & 120.55 & 42 &  0.00 &  0.00\\
instance n=100 255.alb & 1 & 0 & Solution & 120.50 & 14 &  0.00 &  0.00\\
instance n=100 256.alb & 1 & 0 & Unknown & 120522.00 & - & - & -\\
instance n=100 257.alb & 1 & 0 & Solution & 120.47 & 100 &  0.00 &  0.00\\
instance n=100 258.alb & 1 & 0 & Unknown & 120524.00 & - & - & -\\
instance n=100 259.alb & 1 & 0 & Unknown & 120468.00 & - & - & -\\
instance n=100 26.alb & 1 & 0 & Unknown & 120493.00 & - & - & -\\
instance n=100 260.alb & 1 & 0 & Unknown & 120487.00 & - & - & -\\
instance n=100 261.alb & 1 & 0 & Solution & 120.54 & 99 &  0.00 &  0.00\\
instance n=100 262.alb & 1 & 0 & Solution & 120.52 & 100 &  0.00 &  0.00\\
instance n=100 263.alb & 1 & 0 & Unknown & 120491.00 & - & - & -\\
instance n=100 264.alb & 1 & 0 & Solution & 120.55 & 41 &  0.00 &  0.00\\
instance n=100 265.alb & 1 & 0 & Solution & 120.47 & 100 &  0.00 &  0.00\\
instance n=100 266.alb & 1 & 0 & Solution & 120.54 & 32 &  0.00 &  0.00\\
instance n=100 267.alb & 1 & 0 & Solution & 120.47 & 25 &  0.00 &  0.00\\
instance n=100 268.alb & 1 & 0 & Solution & 120.74 & 15 &  0.00 &  0.00\\
instance n=100 269.alb & 1 & 0 & Unknown & 120607.00 & - & - & -\\
instance n=100 27.alb & 1 & 0 & Solution & 120.54 & 35 &  0.00 &  0.00\\
instance n=100 270.alb & 1 & 0 & Solution & 120.49 & 99 &  0.00 &  0.00\\
instance n=100 271.alb & 1 & 0 & Solution & 120.43 & 39 &  0.00 &  0.00\\
instance n=100 272.alb & 1 & 0 & Unknown & 120547.00 & - & - & -\\
instance n=100 273.alb & 1 & 0 & Unknown & 120538.00 & - & - & -\\
instance n=100 274.alb & 1 & 0 & Unknown & 120500.00 & - & - & -\\
instance n=100 275.alb & 1 & 0 & Solution & 120.52 & 100 &  0.00 &  0.00\\
instance n=100 276.alb & 1 & 0 & Solution & 120.50 & 69 &  0.00 &  0.00\\
instance n=100 277.alb & 1 & 0 & Solution & 120.45 & 70 &  0.00 &  0.00\\
instance n=100 278.alb & 1 & 0 & Solution & 120.46 & 67 &  0.00 &  0.00\\
instance n=100 279.alb & 1 & 0 & Solution & 120.45 & 73 &  0.00 &  0.00\\
instance n=100 28.alb & 1 & 0 & Solution & 120.47 & 15 &  0.00 &  0.00\\
instance n=100 280.alb & 1 & 0 & Solution & 120.48 & 67 &  0.00 &  0.00\\
instance n=100 281.alb & 1 & 0 & Solution & 120.47 & 85 &  0.00 &  0.00\\
instance n=100 282.alb & 1 & 0 & Solution & 120.48 & 71 &  0.00 &  0.00\\
instance n=100 283.alb & 1 & 0 & Solution & 120.48 & 65 &  0.00 &  0.00\\
instance n=100 284.alb & 1 & 0 & Solution & 120.49 & 72 &  0.00 &  0.00\\
instance n=100 285.alb & 1 & 0 & Solution & 120.45 & 69 &  0.00 &  0.00\\
instance n=100 286.alb & 1 & 0 & Solution & 120.48 & 77 &  0.00 &  0.00\\
instance n=100 287.alb & 1 & 0 & Solution & 120.46 & 78 &  0.00 &  0.00\\
instance n=100 288.alb & 1 & 0 & Solution & 120.45 & 67 &  0.00 &  0.00\\
instance n=100 289.alb & 1 & 0 & Solution & 120.47 & 73 &  0.00 &  0.00\\
instance n=100 29.alb & 1 & 0 & Solution & 120.48 & 16 &  0.00 &  0.00\\
instance n=100 290.alb & 1 & 0 & Solution & 120.51 & 67 &  0.00 &  0.00\\
instance n=100 291.alb & 1 & 0 & Solution & 120.47 & 72 &  0.00 &  0.00\\
instance n=100 292.alb & 1 & 0 & Solution & 120.48 & 72 &  0.00 &  0.00\\
instance n=100 293.alb & 1 & 0 & Solution & 120.50 & 63 &  0.00 &  0.00\\
instance n=100 294.alb & 1 & 0 & Solution & 120.49 & 73 &  0.00 &  0.00\\
instance n=100 295.alb & 1 & 0 & Solution & 120.50 & 73 &  0.00 &  0.00\\
instance n=100 296.alb & 1 & 0 & Solution & 120.47 & 69 &  0.00 &  0.00\\
instance n=100 297.alb & 1 & 0 & Solution & 120.45 & 68 &  0.00 &  0.00\\
instance n=100 298.alb & 1 & 0 & Solution & 120.47 & 68 &  0.00 &  0.00\\
instance n=100 299.alb & 1 & 0 & Solution & 120.46 & 66 &  0.00 &  0.00\\
instance n=100 3.alb & 1 & 0 & Solution & 120.55 & 62 &  0.00 &  0.00\\
instance n=100 30.alb & 1 & 0 & Solution & 120.55 & 71 &  0.00 &  0.00\\
instance n=100 300.alb & 1 & 0 & Solution & 120.50 & 97 &  0.00 &  0.00\\
instance n=100 301.alb & 1 & 0 & Solution & 120.49 & 25 &  0.00 &  0.00\\
instance n=100 302.alb & 1 & 0 & Solution & 120.45 & 68 &  0.00 &  0.00\\
instance n=100 303.alb & 1 & 0 & Solution & 120.53 & 56 &  0.00 &  0.00\\
instance n=100 304.alb & 1 & 0 & Solution & 120.48 & 54 &  0.00 &  0.00\\
instance n=100 305.alb & 1 & 0 & Unknown & 120492.00 & - & - & -\\
instance n=100 306.alb & 1 & 0 & Unknown & 120525.00 & - & - & -\\
instance n=100 307.alb & 1 & 0 & Unknown & 120539.00 & - & - & -\\
instance n=100 308.alb & 1 & 0 & Solution & 120.52 & 22 &  0.00 &  0.00\\
instance n=100 309.alb & 1 & 0 & Unknown & 120490.00 & - & - & -\\
instance n=100 31.alb & 1 & 0 & Unknown & 120536.00 & - & - & -\\
instance n=100 310.alb & 1 & 0 & Unknown & 120478.00 & - & - & -\\
instance n=100 311.alb & 1 & 0 & Unknown & 120458.00 & - & - & -\\
instance n=100 312.alb & 1 & 0 & Solution & 120.53 & 44 &  0.00 &  0.00\\
instance n=100 313.alb & 1 & 0 & Unknown & 120496.00 & - & - & -\\
instance n=100 314.alb & 1 & 0 & Solution & 120.53 & 100 &  0.00 &  0.00\\
instance n=100 315.alb & 1 & 0 & Unknown & 120501.00 & - & - & -\\
instance n=100 316.alb & 1 & 0 & Unknown & 120474.00 & - & - & -\\
instance n=100 317.alb & 1 & 0 & Unknown & 120500.00 & - & - & -\\
instance n=100 318.alb & 1 & 0 & Unknown & 120429.00 & - & - & -\\
instance n=100 319.alb & 1 & 0 & Unknown & 120483.00 & - & - & -\\
instance n=100 32.alb & 1 & 0 & Solution & 120.54 & 100 &  0.00 &  0.00\\
instance n=100 320.alb & 1 & 0 & Unknown & 120520.00 & - & - & -\\
instance n=100 321.alb & 1 & 0 & Solution & 120.53 & 53 &  0.00 &  0.00\\
instance n=100 322.alb & 1 & 0 & Solution & 120.54 & 46 &  0.00 &  0.00\\
instance n=100 323.alb & 1 & 0 & Solution & 120.47 & 26 &  0.00 &  0.00\\
instance n=100 324.alb & 1 & 0 & Unknown & 120442.00 & - & - & -\\
instance n=100 325.alb & 1 & 0 & Unknown & 120504.00 & - & - & -\\
instance n=100 326.alb & 1 & 0 & Solution & 120.47 & 14 &  0.00 &  0.00\\
instance n=100 327.alb & 1 & 0 & Solution & 120.55 & 15 &  0.00 &  0.00\\
instance n=100 328.alb & 1 & 0 & Solution & 120.50 & 100 &  0.00 &  0.00\\
instance n=100 329.alb & 1 & 0 & Solution & 120.50 & 65 &  0.00 &  0.00\\
instance n=100 33.alb & 1 & 0 & Solution & 120.56 & 93 &  0.00 &  0.00\\
instance n=100 330.alb & 1 & 0 & Unknown & 120454.00 & - & - & -\\
instance n=100 331.alb & 1 & 0 & Solution & 120.56 & 92 &  0.00 &  0.00\\
instance n=100 332.alb & 1 & 0 & Solution & 120.44 & 14 &  0.00 &  0.00\\
instance n=100 333.alb & 1 & 0 & Solution & 120.52 & 49 &  0.00 &  0.00\\
instance n=100 334.alb & 1 & 0 & Unknown & 120512.00 & - & - & -\\
instance n=100 335.alb & 1 & 0 & Solution & 120.45 & 14 &  0.00 &  0.00\\
instance n=100 336.alb & 1 & 0 & Solution & 120.56 & 55 &  0.00 &  0.00\\
instance n=100 337.alb & 1 & 0 & Solution & 120.56 & 14 &  0.00 &  0.00\\
instance n=100 338.alb & 1 & 0 & Solution & 120.50 & 16 &  0.00 &  0.00\\
instance n=100 339.alb & 1 & 0 & Solution & 120.48 & 16 &  0.00 &  0.00\\
instance n=100 34.alb & 1 & 0 & Unknown & 120495.00 & - & - & -\\
instance n=100 340.alb & 1 & 0 & Solution & 120.51 & 15 &  0.00 &  0.00\\
instance n=100 341.alb & 1 & 0 & Unknown & 120504.00 & - & - & -\\
instance n=100 342.alb & 1 & 0 & Solution & 120.54 & 100 &  0.00 &  0.00\\
instance n=100 343.alb & 1 & 0 & Unknown & 120511.00 & - & - & -\\
instance n=100 344.alb & 1 & 0 & Solution & 120.52 & 100 &  0.00 &  0.00\\
instance n=100 345.alb & 1 & 0 & Solution & 120.60 & 98 &  0.00 &  0.00\\
instance n=100 346.alb & 1 & 0 & Solution & 120.54 & 99 &  0.00 &  0.00\\
instance n=100 347.alb & 1 & 0 & Solution & 120.49 & 15 &  0.00 &  0.00\\
instance n=100 348.alb & 1 & 0 & Unknown & 120508.00 & - & - & -\\
instance n=100 349.alb & 1 & 0 & Solution & 120.42 & 14 &  0.00 &  0.00\\
instance n=100 35.alb & 1 & 0 & Solution & 120.45 & 16 &  0.00 &  0.00\\
instance n=100 350.alb & 1 & 0 & Solution & 120.52 & 36 &  0.00 &  0.00\\
instance n=100 351.alb & 1 & 0 & Solution & 120.47 & 98 &  0.00 &  0.00\\
instance n=100 352.alb & 1 & 0 & Solution & 120.47 & 82 &  0.00 &  0.00\\
instance n=100 353.alb & 1 & 0 & Unknown & 120470.00 & - & - & -\\
instance n=100 354.alb & 1 & 0 & Solution & 120.53 & 67 &  0.00 &  0.00\\
instance n=100 355.alb & 1 & 0 & Solution & 120.47 & 68 &  0.00 &  0.00\\
instance n=100 356.alb & 1 & 0 & Unknown & 120499.00 & - & - & -\\
instance n=100 357.alb & 1 & 0 & Solution & 120.48 & 79 &  0.00 &  0.00\\
instance n=100 358.alb & 1 & 0 & Solution & 120.51 & 100 &  0.00 &  0.00\\
instance n=100 359.alb & 1 & 0 & Solution & 120.48 & 99 &  0.00 &  0.00\\
instance n=100 36.alb & 1 & 0 & Unknown & 120521.00 & - & - & -\\
instance n=100 360.alb & 1 & 0 & Unknown & 120492.00 & - & - & -\\
instance n=100 361.alb & 1 & 0 & Unknown & 120450.00 & - & - & -\\
instance n=100 362.alb & 1 & 0 & Solution & 120.47 & 98 &  0.00 &  0.00\\
instance n=100 363.alb & 1 & 0 & Solution & 120.47 & 65 &  0.00 &  0.00\\
instance n=100 364.alb & 1 & 0 & Solution & 120.45 & 79 &  0.00 &  0.00\\
instance n=100 365.alb & 1 & 0 & Solution & 120.49 & 76 &  0.00 &  0.00\\
instance n=100 366.alb & 1 & 0 & Solution & 120.47 & 88 &  0.00 &  0.00\\
instance n=100 367.alb & 1 & 0 & Solution & 120.49 & 82 &  0.00 &  0.00\\
instance n=100 368.alb & 1 & 0 & Solution & 120.47 & 100 &  0.00 &  0.00\\
instance n=100 369.alb & 1 & 0 & Unknown & 120467.00 & - & - & -\\
instance n=100 37.alb & 1 & 0 & Unknown & 120516.00 & - & - & -\\
instance n=100 370.alb & 1 & 0 & Solution & 120.47 & 91 &  0.00 &  0.00\\
instance n=100 371.alb & 1 & 0 & Solution & 120.46 & 100 &  0.00 &  0.00\\
instance n=100 372.alb & 1 & 0 & Unknown & 120519.00 & - & - & -\\
instance n=100 373.alb & 1 & 0 & Unknown & 120501.00 & - & - & -\\
instance n=100 374.alb & 1 & 0 & Unknown & 120502.00 & - & - & -\\
instance n=100 375.alb & 1 & 0 & Solution & 120.50 & 81 &  0.00 &  0.00\\
instance n=100 376.alb & 1 & 0 & Unknown & 120498.00 & - & - & -\\
instance n=100 377.alb & 1 & 0 & Unknown & 120526.00 & - & - & -\\
instance n=100 378.alb & 1 & 0 & Unknown & 120512.00 & - & - & -\\
instance n=100 379.alb & 1 & 0 & Unknown & 120515.00 & - & - & -\\
instance n=100 38.alb & 1 & 0 & Solution & 120.50 & 16 &  0.00 &  0.00\\
instance n=100 380.alb & 1 & 0 & Unknown & 120535.00 & - & - & -\\
instance n=100 381.alb & 1 & 0 & Unknown & 120527.00 & - & - & -\\
instance n=100 382.alb & 1 & 0 & Solution & 120.51 & 61 &  0.00 &  0.00\\
instance n=100 383.alb & 1 & 0 & Solution & 120.53 & 54 &  0.00 &  0.00\\
instance n=100 384.alb & 1 & 0 & Unknown & 120537.00 & - & - & -\\
instance n=100 385.alb & 1 & 0 & Unknown & 120436.00 & - & - & -\\
instance n=100 386.alb & 1 & 0 & Unknown & 120442.00 & - & - & -\\
instance n=100 387.alb & 1 & 0 & Unknown & 120528.00 & - & - & -\\
instance n=100 388.alb & 1 & 0 & Unknown & 120553.00 & - & - & -\\
instance n=100 389.alb & 1 & 0 & Unknown & 120497.00 & - & - & -\\
instance n=100 39.alb & 1 & 0 & Solution & 120.52 & 30 &  0.00 &  0.00\\
instance n=100 390.alb & 1 & 0 & Solution & 120.58 & 67 &  0.00 &  0.00\\
instance n=100 391.alb & 1 & 0 & Unknown & 120516.00 & - & - & -\\
instance n=100 392.alb & 1 & 0 & Solution & 120.57 & 47 &  0.00 &  0.00\\
instance n=100 393.alb & 1 & 0 & Unknown & 120445.00 & - & - & -\\
instance n=100 394.alb & 1 & 0 & Unknown & 120520.00 & - & - & -\\
instance n=100 395.alb & 1 & 0 & Unknown & 120427.00 & - & - & -\\
instance n=100 396.alb & 1 & 0 & Unknown & 120519.00 & - & - & -\\
instance n=100 397.alb & 1 & 0 & Solution & 120.45 & 57 &  0.00 &  0.00\\
instance n=100 398.alb & 1 & 0 & Unknown & 120505.00 & - & - & -\\
instance n=100 399.alb & 1 & 0 & Unknown & 120525.00 & - & - & -\\
instance n=100 4.alb & 1 & 0 & Unknown & 120478.00 & - & - & -\\
instance n=100 40.alb & 1 & 0 & Solution & 120.58 & 57 &  0.00 &  0.00\\
instance n=100 400.alb & 1 & 0 & Solution & 120.46 & 31 &  0.00 &  0.00\\
instance n=100 401.alb & 1 & 0 & Unknown & 120518.00 & - & - & -\\
instance n=100 402.alb & 1 & 0 & Unknown & 120505.00 & - & - & -\\
instance n=100 403.alb & 1 & 0 & Solution & 120.63 & 99 &  0.00 &  0.00\\
instance n=100 404.alb & 1 & 0 & Solution & 120.49 & 56 &  0.00 &  0.00\\
instance n=100 405.alb & 1 & 0 & Solution & 120.49 & 14 &  0.00 &  0.00\\
instance n=100 406.alb & 1 & 0 & Solution & 120.50 & 14 &  0.00 &  0.00\\
instance n=100 407.alb & 1 & 0 & Unknown & 120561.00 & - & - & -\\
instance n=100 408.alb & 1 & 0 & Solution & 120.51 & 90 &  0.00 &  0.00\\
instance n=100 409.alb & 1 & 0 & Unknown & 120503.00 & - & - & -\\
instance n=100 41.alb & 1 & 0 & Unknown & 120516.00 & - & - & -\\
instance n=100 410.alb & 1 & 0 & Unknown & 120491.00 & - & - & -\\
instance n=100 411.alb & 1 & 0 & Unknown & 120528.00 & - & - & -\\
instance n=100 412.alb & 1 & 0 & Unknown & 120447.00 & - & - & -\\
instance n=100 413.alb & 1 & 0 & Unknown & 120565.00 & - & - & -\\
instance n=100 414.alb & 1 & 0 & Solution & 120.51 & 47 &  0.00 &  0.00\\
instance n=100 415.alb & 1 & 0 & Solution & 120.51 & 99 &  0.00 &  0.00\\
instance n=100 416.alb & 1 & 0 & Solution & 120.52 & 52 &  0.00 &  0.00\\
instance n=100 417.alb & 1 & 0 & Unknown & 120496.00 & - & - & -\\
instance n=100 418.alb & 1 & 0 & Unknown & 120496.00 & - & - & -\\
instance n=100 419.alb & 1 & 0 & Unknown & 120523.00 & - & - & -\\
instance n=100 42.alb & 1 & 0 & Unknown & 120529.00 & - & - & -\\
instance n=100 420.alb & 1 & 0 & Solution & 120.53 & 37 &  0.00 &  0.00\\
instance n=100 421.alb & 1 & 0 & Solution & 120.57 & 14 &  0.00 &  0.00\\
instance n=100 422.alb & 1 & 0 & Solution & 120.52 & 39 &  0.00 &  0.00\\
instance n=100 423.alb & 1 & 0 & Solution & 120.52 & 42 &  0.00 &  0.00\\
instance n=100 424.alb & 1 & 0 & Solution & 120.51 & 57 &  0.00 &  0.00\\
instance n=100 425.alb & 1 & 0 & Solution & 120.50 & 98 &  0.00 &  0.00\\
instance n=100 426.alb & 1 & 0 & Solution & 120.46 & 74 &  0.00 &  0.00\\
instance n=100 427.alb & 1 & 0 & Solution & 120.48 & 78 &  0.00 &  0.00\\
instance n=100 428.alb & 1 & 0 & Solution & 120.47 & 69 &  0.00 &  0.00\\
instance n=100 429.alb & 1 & 0 & Solution & 120.46 & 73 &  0.00 &  0.00\\
instance n=100 43.alb & 1 & 0 & Solution & 120.55 & 15 &  0.00 &  0.00\\
instance n=100 430.alb & 1 & 0 & Solution & 120.48 & 68 &  0.00 &  0.00\\
instance n=100 431.alb & 1 & 0 & Solution & 120.48 & 68 &  0.00 &  0.00\\
instance n=100 432.alb & 1 & 0 & Solution & 120.47 & 74 &  0.00 &  0.00\\
instance n=100 433.alb & 1 & 0 & Solution & 120.47 & 65 &  0.00 &  0.00\\
instance n=100 434.alb & 1 & 0 & Solution & 120.45 & 70 &  0.00 &  0.00\\
instance n=100 435.alb & 1 & 0 & Solution & 120.47 & 69 &  0.00 &  0.00\\
instance n=100 436.alb & 1 & 0 & Solution & 120.46 & 66 &  0.00 &  0.00\\
instance n=100 437.alb & 1 & 0 & Solution & 120.46 & 66 &  0.00 &  0.00\\
instance n=100 438.alb & 1 & 0 & Solution & 120.47 & 66 &  0.00 &  0.00\\
instance n=100 439.alb & 1 & 0 & Solution & 120.48 & 79 &  0.00 &  0.00\\
instance n=100 44.alb & 1 & 0 & Solution & 120.53 & 37 &  0.00 &  0.00\\
instance n=100 440.alb & 1 & 0 & Solution & 120.46 & 63 &  0.00 &  0.00\\
instance n=100 441.alb & 1 & 0 & Solution & 120.44 & 66 &  0.00 &  0.00\\
instance n=100 442.alb & 1 & 0 & Solution & 120.46 & 68 &  0.00 &  0.00\\
instance n=100 443.alb & 1 & 0 & Solution & 120.51 & 66 &  0.00 &  0.00\\
instance n=100 444.alb & 1 & 0 & Solution & 120.46 & 99 &  0.00 &  0.00\\
instance n=100 445.alb & 1 & 0 & Solution & 120.47 & 65 &  0.00 &  0.00\\
instance n=100 446.alb & 1 & 0 & Solution & 120.48 & 73 &  0.00 &  0.00\\
instance n=100 447.alb & 1 & 0 & Solution & 120.47 & 64 &  0.00 &  0.00\\
instance n=100 448.alb & 1 & 0 & Solution & 120.53 & 84 &  0.00 &  0.00\\
instance n=100 449.alb & 1 & 0 & Solution & 120.44 & 71 &  0.00 &  0.00\\
instance n=100 45.alb & 1 & 0 & Solution & 120.52 & 95 &  0.00 &  0.00\\
instance n=100 450.alb & 1 & 0 & Solution & 120.47 & 68 &  0.00 &  0.00\\
instance n=100 451.alb & 1 & 0 & Solution & 120.44 & 30 &  0.00 &  0.00\\
instance n=100 452.alb & 1 & 0 & Solution & 120.47 & 24 &  0.00 &  0.00\\
instance n=100 453.alb & 1 & 0 & Unknown & 120528.00 & - & - & -\\
instance n=100 454.alb & 1 & 0 & Unknown & 120514.00 & - & - & -\\
instance n=100 455.alb & 1 & 0 & Unknown & 120513.00 & - & - & -\\
instance n=100 456.alb & 1 & 0 & Solution & 120.44 & 29 &  0.00 &  0.00\\
instance n=100 457.alb & 1 & 0 & Unknown & 120449.00 & - & - & -\\
instance n=100 458.alb & 1 & 0 & Unknown & 120509.00 & - & - & -\\
instance n=100 459.alb & 1 & 0 & Unknown & 120549.00 & - & - & -\\
instance n=100 46.alb & 1 & 0 & Solution & 120.53 & 14 &  0.00 &  0.00\\
instance n=100 460.alb & 1 & 0 & Solution & 120.43 & 26 &  0.00 &  0.00\\
instance n=100 461.alb & 1 & 0 & Solution & 120.46 & 31 &  0.00 &  0.00\\
instance n=100 462.alb & 1 & 0 & Unknown & 120489.00 & - & - & -\\
instance n=100 463.alb & 1 & 0 & Solution & 120.45 & 28 &  0.00 &  0.00\\
instance n=100 464.alb & 1 & 0 & Solution & 120.43 & 31 &  0.00 &  0.00\\
instance n=100 465.alb & 1 & 0 & Solution & 120.46 & 30 &  0.00 &  0.00\\
instance n=100 466.alb & 1 & 0 & Solution & 120.42 & 39 &  0.00 &  0.00\\
instance n=100 467.alb & 1 & 0 & Unknown & 120511.00 & - & - & -\\
instance n=100 468.alb & 1 & 0 & Solution & 120.43 & 32 &  0.00 &  0.00\\
instance n=100 469.alb & 1 & 0 & Solution & 120.45 & 24 &  0.00 &  0.00\\
instance n=100 47.alb & 1 & 0 & Unknown & 120615.00 & - & - & -\\
instance n=100 470.alb & 1 & 0 & Unknown & 120426.00 & - & - & -\\
instance n=100 471.alb & 1 & 0 & Solution & 120.44 & 88 &  0.00 &  0.00\\
instance n=100 472.alb & 1 & 0 & Solution & 120.52 & 32 &  0.00 &  0.00\\
instance n=100 473.alb & 1 & 0 & Unknown & 120418.00 & - & - & -\\
instance n=100 474.alb & 1 & 0 & Solution & 120.43 & 25 &  0.00 &  0.00\\
instance n=100 475.alb & 1 & 0 & Unknown & 120446.00 & - & - & -\\
instance n=100 476.alb & 1 & 0 & Solution & 120.51 & 15 &  0.00 &  0.00\\
instance n=100 477.alb & 1 & 0 & Unknown & 120456.00 & - & - & -\\
instance n=100 478.alb & 1 & 0 & Unknown & 120499.00 & - & - & -\\
instance n=100 479.alb & 1 & 0 & Unknown & 120487.00 & - & - & -\\
instance n=100 48.alb & 1 & 0 & Solution & 120.54 & 16 &  0.00 &  0.00\\
instance n=100 480.alb & 1 & 0 & Unknown & 120511.00 & - & - & -\\
instance n=100 481.alb & 1 & 0 & Unknown & 120550.00 & - & - & -\\
instance n=100 482.alb & 1 & 0 & Unknown & 120552.00 & - & - & -\\
instance n=100 483.alb & 1 & 0 & Solution & 120.49 & 26 &  0.00 &  0.00\\
instance n=100 484.alb & 1 & 0 & Unknown & 120549.00 & - & - & -\\
instance n=100 485.alb & 1 & 0 & Solution & 120.49 & 20 &  0.00 &  0.00\\
instance n=100 486.alb & 1 & 0 & Unknown & 120513.00 & - & - & -\\
instance n=100 487.alb & 1 & 0 & Unknown & 120489.00 & - & - & -\\
instance n=100 488.alb & 1 & 0 & Unknown & 120486.00 & - & - & -\\
instance n=100 489.alb & 1 & 0 & Solution & 120.51 & 33 &  0.00 &  0.00\\
instance n=100 49.alb & 1 & 0 & Solution & 120.52 & 15 &  0.00 &  0.00\\
instance n=100 490.alb & 1 & 0 & Unknown & 120541.00 & - & - & -\\
instance n=100 491.alb & 1 & 0 & Solution & 120.50 & 28 &  0.00 &  0.00\\
instance n=100 492.alb & 1 & 0 & Solution & 120.51 & 95 &  0.00 &  0.00\\
instance n=100 493.alb & 1 & 0 & Unknown & 120467.00 & - & - & -\\
instance n=100 494.alb & 1 & 0 & Unknown & 120499.00 & - & - & -\\
instance n=100 495.alb & 1 & 0 & Solution & 120.51 & 18 &  0.00 &  0.00\\
instance n=100 496.alb & 1 & 0 & Solution & 120.51 & 24 &  0.00 &  0.00\\
instance n=100 497.alb & 1 & 0 & Solution & 120.46 & 64 &  0.00 &  0.00\\
instance n=100 498.alb & 1 & 0 & Unknown & 120503.00 & - & - & -\\
instance n=100 499.alb & 1 & 0 & Solution & 120.53 & 25 &  0.00 &  0.00\\
instance n=100 5.alb & 1 & 0 & Solution & 120.48 & 24 &  0.00 &  0.00\\
instance n=100 50.alb & 1 & 0 & Solution & 120.49 & 14 &  0.00 &  0.00\\
instance n=100 500.alb & 1 & 0 & Solution & 120.49 & 17 &  0.00 &  0.00\\
instance n=100 501.alb & 1 & 0 & Solution & 120.46 & 67 &  0.00 &  0.00\\
instance n=100 502.alb & 1 & 0 & Solution & 120.45 & 69 &  0.00 &  0.00\\
instance n=100 503.alb & 1 & 0 & Solution & 120.45 & 65 &  0.00 &  0.00\\
instance n=100 504.alb & 1 & 0 & Solution & 120.46 & 64 &  0.00 &  0.00\\
instance n=100 505.alb & 1 & 0 & Solution & 120.47 & 63 &  0.00 &  0.00\\
instance n=100 506.alb & 1 & 0 & Solution & 120.51 & 64 &  0.00 &  0.00\\
instance n=100 507.alb & 1 & 0 & Solution & 120.47 & 62 &  0.00 &  0.00\\
instance n=100 508.alb & 1 & 0 & Solution & 120.43 & 61 &  0.00 &  0.00\\
instance n=100 509.alb & 1 & 0 & Solution & 120.47 & 63 &  0.00 &  0.00\\
instance n=100 51.alb & 1 & 0 & Unknown & 120462.00 & - & - & -\\
instance n=100 510.alb & 1 & 0 & Solution & 120.46 & 63 &  0.00 &  0.00\\
instance n=100 511.alb & 1 & 0 & Solution & 120.46 & 61 &  0.00 &  0.00\\
instance n=100 512.alb & 1 & 0 & Solution & 120.45 & 66 &  0.00 &  0.00\\
instance n=100 513.alb & 1 & 0 & Solution & 120.45 & 64 &  0.00 &  0.00\\
instance n=100 514.alb & 1 & 0 & Solution & 120.53 & 66 &  0.00 &  0.00\\
instance n=100 515.alb & 1 & 0 & Solution & 120.47 & 65 &  0.00 &  0.00\\
instance n=100 516.alb & 1 & 0 & Solution & 120.47 & 78 &  0.00 &  0.00\\
instance n=100 517.alb & 1 & 0 & Solution & 120.45 & 64 &  0.00 &  0.00\\
instance n=100 518.alb & 1 & 0 & Solution & 121.01 & 67 &  0.00 &  0.00\\
instance n=100 519.alb & 1 & 0 & Solution & 120.47 & 69 &  0.00 &  0.00\\
instance n=100 52.alb & 1 & 0 & Solution & 120.50 & 71 &  0.00 &  0.00\\
instance n=100 520.alb & 1 & 0 & Solution & 120.44 & 64 &  0.00 &  0.00\\
instance n=100 521.alb & 1 & 0 & Solution & 120.45 & 75 &  0.00 &  0.00\\
instance n=100 522.alb & 1 & 0 & Solution & 120.47 & 65 &  0.00 &  0.00\\
instance n=100 523.alb & 1 & 0 & Solution & 120.46 & 61 &  0.00 &  0.00\\
instance n=100 524.alb & 1 & 0 & Solution & 120.88 & 68 &  0.00 &  0.00\\
instance n=100 525.alb & 1 & 0 & Solution & 120.43 & 66 &  0.00 &  0.00\\
instance n=100 53.alb & 1 & 0 & Solution & 120.48 & 88 &  0.00 &  0.00\\
instance n=100 54.alb & 1 & 0 & Unknown & 120491.00 & - & - & -\\
instance n=100 55.alb & 1 & 0 & Unknown & 120513.00 & - & - & -\\
instance n=100 56.alb & 1 & 0 & Solution & 120.48 & 68 &  0.00 &  0.00\\
instance n=100 57.alb & 1 & 0 & Solution & 120.49 & 100 &  0.00 &  0.00\\
instance n=100 58.alb & 1 & 0 & Solution & 120.50 & 86 &  0.00 &  0.00\\
instance n=100 59.alb & 1 & 0 & Solution & 120.46 & 84 &  0.00 &  0.00\\
instance n=100 6.alb & 1 & 0 & Unknown & 120512.00 & - & - & -\\
instance n=100 60.alb & 1 & 0 & Unknown & 120471.00 & - & - & -\\
instance n=100 61.alb & 1 & 0 & Solution & 120.48 & 76 &  0.00 &  0.00\\
instance n=100 62.alb & 1 & 0 & Solution & 120.48 & 100 &  0.00 &  0.00\\
instance n=100 63.alb & 1 & 0 & Unknown & 120490.00 & - & - & -\\
instance n=100 64.alb & 1 & 0 & Solution & 120.51 & 71 &  0.00 &  0.00\\
instance n=100 65.alb & 1 & 0 & Solution & 120.46 & 66 &  0.00 &  0.00\\
instance n=100 66.alb & 1 & 0 & Solution & 120.46 & 78 &  0.00 &  0.00\\
instance n=100 67.alb & 1 & 0 & Unknown & 120490.00 & - & - & -\\
instance n=100 68.alb & 1 & 0 & Solution & 120.47 & 99 &  0.00 &  0.00\\
instance n=100 69.alb & 1 & 0 & Solution & 120.50 & 81 &  0.00 &  0.00\\
instance n=100 7.alb & 1 & 0 & Unknown & 120435.00 & - & - & -\\
instance n=100 70.alb & 1 & 0 & Unknown & 120494.00 & - & - & -\\
instance n=100 71.alb & 1 & 0 & Solution & 120.51 & 95 &  0.00 &  0.00\\
instance n=100 72.alb & 1 & 0 & Solution & 120.47 & 77 &  0.00 &  0.00\\
instance n=100 73.alb & 1 & 0 & Solution & 120.49 & 71 &  0.00 &  0.00\\
instance n=100 74.alb & 1 & 0 & Unknown & 120500.00 & - & - & -\\
instance n=100 75.alb & 1 & 0 & Solution & 120.47 & 85 &  0.00 &  0.00\\
instance n=100 76.alb & 1 & 0 & Unknown & 120523.00 & - & - & -\\
instance n=100 77.alb & 1 & 0 & Unknown & 120516.00 & - & - & -\\
instance n=100 78.alb & 1 & 0 & Solution & 120.52 & 65 &  0.00 &  0.00\\
instance n=100 79.alb & 1 & 0 & Unknown & 120491.00 & - & - & -\\
instance n=100 8.alb & 1 & 0 & Unknown & 120518.00 & - & - & -\\
instance n=100 80.alb & 1 & 0 & Solution & 120.44 & 99 &  0.00 &  0.00\\
instance n=100 81.alb & 1 & 0 & Unknown & 120499.00 & - & - & -\\
instance n=100 82.alb & 1 & 0 & Unknown & 120419.00 & - & - & -\\
instance n=100 83.alb & 1 & 0 & Unknown & 120432.00 & - & - & -\\
instance n=100 84.alb & 1 & 0 & Solution & 120.48 & 45 &  0.00 &  0.00\\
instance n=100 85.alb & 1 & 0 & Unknown & 120474.00 & - & - & -\\
instance n=100 86.alb & 1 & 0 & Unknown & 120508.00 & - & - & -\\
instance n=100 87.alb & 1 & 0 & Unknown & 120515.00 & - & - & -\\
instance n=100 88.alb & 1 & 0 & Unknown & 120470.00 & - & - & -\\
instance n=100 89.alb & 1 & 0 & Solution & 120.43 & 73 &  0.00 &  0.00\\
instance n=100 9.alb & 1 & 0 & Solution & 120.48 & 25 &  0.00 &  0.00\\
instance n=100 90.alb & 1 & 0 & Unknown & 120506.00 & - & - & -\\
instance n=100 91.alb & 1 & 0 & Unknown & 120527.00 & - & - & -\\
instance n=100 92.alb & 1 & 0 & Unknown & 120525.00 & - & - & -\\
instance n=100 93.alb & 1 & 0 & Unknown & 120470.00 & - & - & -\\
instance n=100 94.alb & 1 & 0 & Solution & 120.46 & 57 &  0.00 &  0.00\\
instance n=100 95.alb & 1 & 0 & Unknown & 120429.00 & - & - & -\\
instance n=100 96.alb & 1 & 0 & Solution & 120.44 & 100 &  0.00 &  0.00\\
instance n=100 97.alb & 1 & 0 & Unknown & 120491.00 & - & - & -\\
instance n=100 98.alb & 1 & 0 & Unknown & 120549.00 & - & - & -\\
instance n=100 99.alb & 1 & 0 & Unknown & 120507.00 & - & - & -\\
instance n=20 1.alb & 1 & 0 & Optimal &  0.33 & 3 &  0.00 &  0.00\\
instance n=20 10.alb & 1 & 0 & Optimal &  0.31 & 3 &  0.00 &  0.00\\
instance n=20 100.alb & 1 & 0 & Optimal &  0.38 & 11 &  0.00 &  0.00\\
instance n=20 101.alb & 1 & 0 & Optimal &  1.07 & 13 &  0.00 &  0.00\\
instance n=20 102.alb & 1 & 0 & Optimal &  0.55 & 13 &  0.00 &  0.00\\
instance n=20 103.alb & 1 & 0 & Optimal &  0.49 & 12 &  0.00 &  0.00\\
instance n=20 104.alb & 1 & 0 & Optimal &  0.40 & 11 &  0.00 &  0.00\\
instance n=20 105.alb & 1 & 0 & Optimal &  0.38 & 12 &  0.00 &  0.00\\
instance n=20 106.alb & 1 & 0 & Optimal &  0.40 & 10 &  0.00 &  0.00\\
instance n=20 107.alb & 1 & 0 & Optimal &  0.37 & 14 &  0.00 &  0.00\\
instance n=20 108.alb & 1 & 0 & Optimal &  0.41 & 15 &  0.00 &  0.00\\
instance n=20 109.alb & 1 & 0 & Optimal &  0.39 & 12 &  0.00 &  0.00\\
instance n=20 11.alb & 1 & 0 & Optimal &  0.31 & 3 &  0.00 &  0.00\\
instance n=20 110.alb & 1 & 0 & Optimal &  0.34 & 11 &  0.00 &  0.00\\
instance n=20 111.alb & 1 & 0 & Optimal &  0.39 & 13 &  0.00 &  0.00\\
instance n=20 112.alb & 1 & 0 & Optimal &  0.37 & 11 &  0.00 &  0.00\\
instance n=20 113.alb & 1 & 0 & Optimal &  0.44 & 12 &  0.00 &  0.00\\
instance n=20 114.alb & 1 & 0 & Optimal &  0.44 & 13 &  0.00 &  0.00\\
instance n=20 115.alb & 1 & 0 & Optimal &  0.40 & 11 &  0.00 &  0.00\\
instance n=20 116.alb & 1 & 0 & Optimal &  0.37 & 5 &  0.00 &  0.00\\
instance n=20 117.alb & 1 & 0 & Optimal &  0.33 & 5 &  0.00 &  0.00\\
instance n=20 118.alb & 1 & 0 & Optimal &  0.33 & 5 &  0.00 &  0.00\\
instance n=20 119.alb & 1 & 0 & Optimal &  0.37 & 6 &  0.00 &  0.00\\
instance n=20 12.alb & 1 & 0 & Optimal &  0.32 & 3 &  0.00 &  0.00\\
instance n=20 120.alb & 1 & 0 & Optimal &  0.33 & 6 &  0.00 &  0.00\\
instance n=20 121.alb & 1 & 0 & Optimal &  0.32 & 5 &  0.00 &  0.00\\
instance n=20 122.alb & 1 & 0 & Optimal &  0.35 & 6 &  0.00 &  0.00\\
instance n=20 123.alb & 1 & 0 & Optimal &  0.31 & 5 &  0.00 &  0.00\\
instance n=20 124.alb & 1 & 0 & Optimal &  0.32 & 5 &  0.00 &  0.00\\
instance n=20 125.alb & 1 & 0 & Optimal &  0.34 & 5 &  0.00 &  0.00\\
instance n=20 126.alb & 1 & 0 & Optimal &  0.31 & 5 &  0.00 &  0.00\\
instance n=20 127.alb & 1 & 0 & Optimal &  0.32 & 4 &  0.00 &  0.00\\
instance n=20 128.alb & 1 & 0 & Optimal &  0.32 & 5 &  0.00 &  0.00\\
instance n=20 129.alb & 1 & 0 & Optimal &  0.33 & 5 &  0.00 &  0.00\\
instance n=20 13.alb & 1 & 0 & Optimal &  0.32 & 3 &  0.00 &  0.00\\
instance n=20 130.alb & 1 & 0 & Optimal &  0.33 & 6 &  0.00 &  0.00\\
instance n=20 131.alb & 1 & 0 & Optimal &  0.39 & 7 &  0.00 &  0.00\\
instance n=20 132.alb & 1 & 0 & Optimal &  0.33 & 4 &  0.00 &  0.00\\
instance n=20 133.alb & 1 & 0 & Optimal &  0.33 & 5 &  0.00 &  0.00\\
instance n=20 134.alb & 1 & 0 & Optimal &  0.34 & 6 &  0.00 &  0.00\\
instance n=20 135.alb & 1 & 0 & Optimal &  0.36 & 6 &  0.00 &  0.00\\
instance n=20 136.alb & 1 & 0 & Optimal &  0.32 & 6 &  0.00 &  0.00\\
instance n=20 137.alb & 1 & 0 & Optimal &  0.33 & 5 &  0.00 &  0.00\\
instance n=20 138.alb & 1 & 0 & Optimal &  0.36 & 5 &  0.00 &  0.00\\
instance n=20 139.alb & 1 & 0 & Optimal &  0.36 & 5 &  0.00 &  0.00\\
instance n=20 14.alb & 1 & 0 & Optimal &  0.32 & 3 &  0.00 &  0.00\\
instance n=20 140.alb & 1 & 0 & Optimal &  0.34 & 5 &  0.00 &  0.00\\
instance n=20 141.alb & 1 & 0 & Optimal &  0.33 & 3 &  0.00 &  0.00\\
instance n=20 142.alb & 1 & 0 & Optimal &  0.31 & 3 &  0.00 &  0.00\\
instance n=20 143.alb & 1 & 0 & Optimal &  0.32 & 3 &  0.00 &  0.00\\
instance n=20 144.alb & 1 & 0 & Optimal &  0.31 & 4 &  0.00 &  0.00\\
instance n=20 145.alb & 1 & 0 & Optimal &  0.32 & 3 &  0.00 &  0.00\\
instance n=20 146.alb & 1 & 0 & Optimal &  0.32 & 3 &  0.00 &  0.00\\
instance n=20 147.alb & 1 & 0 & Optimal &  0.32 & 3 &  0.00 &  0.00\\
instance n=20 148.alb & 1 & 0 & Optimal &  0.32 & 3 &  0.00 &  0.00\\
instance n=20 149.alb & 1 & 0 & Optimal &  0.33 & 3 &  0.00 &  0.00\\
instance n=20 15.alb & 1 & 0 & Optimal &  0.33 & 3 &  0.00 &  0.00\\
instance n=20 150.alb & 1 & 0 & Optimal &  0.32 & 3 &  0.00 &  0.00\\
instance n=20 151.alb & 1 & 0 & Optimal &  0.31 & 3 &  0.00 &  0.00\\
instance n=20 152.alb & 1 & 0 & Optimal &  0.32 & 3 &  0.00 &  0.00\\
instance n=20 153.alb & 1 & 0 & Optimal &  0.32 & 3 &  0.00 &  0.00\\
instance n=20 154.alb & 1 & 0 & Optimal &  0.31 & 3 &  0.00 &  0.00\\
instance n=20 155.alb & 1 & 0 & Optimal &  0.31 & 3 &  0.00 &  0.00\\
instance n=20 156.alb & 1 & 0 & Optimal &  0.32 & 3 &  0.00 &  0.00\\
instance n=20 157.alb & 1 & 0 & Optimal &  0.31 & 3 &  0.00 &  0.00\\
instance n=20 158.alb & 1 & 0 & Optimal &  0.34 & 3 &  0.00 &  0.00\\
instance n=20 159.alb & 1 & 0 & Optimal &  0.30 & 3 &  0.00 &  0.00\\
instance n=20 16.alb & 1 & 0 & Optimal &  0.38 & 12 &  0.00 &  0.00\\
instance n=20 160.alb & 1 & 0 & Optimal &  0.33 & 3 &  0.00 &  0.00\\
instance n=20 161.alb & 1 & 0 & Optimal &  0.31 & 3 &  0.00 &  0.00\\
instance n=20 162.alb & 1 & 0 & Optimal &  0.32 & 3 &  0.00 &  0.00\\
instance n=20 163.alb & 1 & 0 & Optimal &  0.32 & 3 &  0.00 &  0.00\\
instance n=20 164.alb & 1 & 0 & Optimal &  0.31 & 4 &  0.00 &  0.00\\
instance n=20 165.alb & 1 & 0 & Optimal &  0.34 & 3 &  0.00 &  0.00\\
instance n=20 166.alb & 1 & 0 & Optimal &  0.43 & 12 &  0.00 &  0.00\\
instance n=20 167.alb & 1 & 0 & Optimal &  0.41 & 11 &  0.00 &  0.00\\
instance n=20 168.alb & 1 & 0 & Optimal &  0.37 & 10 &  0.00 &  0.00\\
instance n=20 169.alb & 1 & 0 & Optimal &  0.49 & 11 &  0.00 &  0.00\\
instance n=20 17.alb & 1 & 0 & Optimal &  0.40 & 10 &  0.00 &  0.00\\
instance n=20 170.alb & 1 & 0 & Optimal &  0.39 & 11 &  0.00 &  0.00\\
instance n=20 171.alb & 1 & 0 & Optimal &  0.52 & 13 &  0.00 &  0.00\\
instance n=20 172.alb & 1 & 0 & Optimal &  0.38 & 11 &  0.00 &  0.00\\
instance n=20 173.alb & 1 & 0 & Optimal &  0.39 & 11 &  0.00 &  0.00\\
instance n=20 174.alb & 1 & 0 & Optimal &  0.39 & 12 &  0.00 &  0.00\\
instance n=20 175.alb & 1 & 0 & Optimal &  0.38 & 10 &  0.00 &  0.00\\
instance n=20 176.alb & 1 & 0 & Optimal &  0.44 & 11 &  0.00 &  0.00\\
instance n=20 177.alb & 1 & 0 & Optimal &  0.69 & 10 &  0.00 &  0.00\\
instance n=20 178.alb & 1 & 0 & Optimal &  0.39 & 11 &  0.00 &  0.00\\
instance n=20 179.alb & 1 & 0 & Optimal &  0.53 & 11 &  0.00 &  0.00\\
instance n=20 18.alb & 1 & 0 & Optimal &  0.45 & 11 &  0.00 &  0.00\\
instance n=20 180.alb & 1 & 0 & Optimal &  0.46 & 13 &  0.00 &  0.00\\
instance n=20 181.alb & 1 & 0 & Optimal &  0.52 & 11 &  0.00 &  0.00\\
instance n=20 182.alb & 1 & 0 & Optimal &  0.38 & 11 &  0.00 &  0.00\\
instance n=20 183.alb & 1 & 0 & Optimal &  0.48 & 13 &  0.00 &  0.00\\
instance n=20 184.alb & 1 & 0 & Optimal &  0.39 & 12 &  0.00 &  0.00\\
instance n=20 185.alb & 1 & 0 & Optimal &  0.41 & 15 &  0.00 &  0.00\\
instance n=20 186.alb & 1 & 0 & Optimal &  0.83 & 14 &  0.00 &  0.00\\
instance n=20 187.alb & 1 & 0 & Optimal &  0.38 & 10 &  0.00 &  0.00\\
instance n=20 188.alb & 1 & 0 & Optimal &  0.38 & 11 &  0.00 &  0.00\\
instance n=20 189.alb & 1 & 0 & Optimal &  0.44 & 13 &  0.00 &  0.00\\
instance n=20 19.alb & 1 & 0 & Optimal &  0.46 & 14 &  0.00 &  0.00\\
instance n=20 190.alb & 1 & 0 & Optimal &  0.46 & 15 &  0.00 &  0.00\\
instance n=20 191.alb & 1 & 0 & Optimal &  0.32 & 4 &  0.00 &  0.00\\
instance n=20 192.alb & 1 & 0 & Optimal &  0.39 & 5 &  0.00 &  0.00\\
instance n=20 193.alb & 1 & 0 & Optimal &  0.36 & 5 &  0.00 &  0.00\\
instance n=20 194.alb & 1 & 0 & Optimal &  0.33 & 6 &  0.00 &  0.00\\
instance n=20 195.alb & 1 & 0 & Optimal &  0.33 & 6 &  0.00 &  0.00\\
instance n=20 196.alb & 1 & 0 & Optimal &  0.38 & 5 &  0.00 &  0.00\\
instance n=20 197.alb & 1 & 0 & Optimal &  0.36 & 4 &  0.00 &  0.00\\
instance n=20 198.alb & 1 & 0 & Optimal &  0.35 & 6 &  0.00 &  0.00\\
instance n=20 199.alb & 1 & 0 & Optimal &  0.37 & 5 &  0.00 &  0.00\\
instance n=20 2.alb & 1 & 0 & Optimal &  0.32 & 3 &  0.00 &  0.00\\
instance n=20 20.alb & 1 & 0 & Optimal &  0.38 & 11 &  0.00 &  0.00\\
instance n=20 200.alb & 1 & 0 & Optimal &  0.37 & 6 &  0.00 &  0.00\\
instance n=20 201.alb & 1 & 0 & Optimal &  0.35 & 6 &  0.00 &  0.00\\
instance n=20 202.alb & 1 & 0 & Optimal &  0.35 & 4 &  0.00 &  0.00\\
instance n=20 203.alb & 1 & 0 & Optimal &  0.36 & 4 &  0.00 &  0.00\\
instance n=20 204.alb & 1 & 0 & Optimal &  0.31 & 5 &  0.00 &  0.00\\
instance n=20 205.alb & 1 & 0 & Optimal &  0.36 & 6 &  0.00 &  0.00\\
instance n=20 206.alb & 1 & 0 & Optimal &  0.34 & 5 &  0.00 &  0.00\\
instance n=20 207.alb & 1 & 0 & Optimal &  0.34 & 6 &  0.00 &  0.00\\
instance n=20 208.alb & 1 & 0 & Optimal &  0.36 & 5 &  0.00 &  0.00\\
instance n=20 209.alb & 1 & 0 & Optimal &  0.37 & 4 &  0.00 &  0.00\\
instance n=20 21.alb & 1 & 0 & Optimal &  0.42 & 14 &  0.00 &  0.00\\
instance n=20 210.alb & 1 & 0 & Optimal &  0.37 & 5 &  0.00 &  0.00\\
instance n=20 211.alb & 1 & 0 & Optimal &  0.36 & 5 &  0.00 &  0.00\\
instance n=20 212.alb & 1 & 0 & Optimal &  0.32 & 5 &  0.00 &  0.00\\
instance n=20 213.alb & 1 & 0 & Optimal &  0.36 & 5 &  0.00 &  0.00\\
instance n=20 214.alb & 1 & 0 & Optimal &  0.34 & 5 &  0.00 &  0.00\\
instance n=20 215.alb & 1 & 0 & Optimal &  0.38 & 5 &  0.00 &  0.00\\
instance n=20 216.alb & 1 & 0 & Optimal &  0.34 & 3 &  0.00 &  0.00\\
instance n=20 217.alb & 1 & 0 & Optimal &  0.33 & 4 &  0.00 &  0.00\\
instance n=20 218.alb & 1 & 0 & Optimal &  0.31 & 3 &  0.00 &  0.00\\
instance n=20 219.alb & 1 & 0 & Optimal &  0.32 & 3 &  0.00 &  0.00\\
instance n=20 22.alb & 1 & 0 & Optimal &  0.42 & 12 &  0.00 &  0.00\\
instance n=20 220.alb & 1 & 0 & Optimal &  0.33 & 3 &  0.00 &  0.00\\
instance n=20 221.alb & 1 & 0 & Optimal &  0.34 & 3 &  0.00 &  0.00\\
instance n=20 222.alb & 1 & 0 & Optimal &  0.31 & 3 &  0.00 &  0.00\\
instance n=20 223.alb & 1 & 0 & Optimal &  0.32 & 3 &  0.00 &  0.00\\
instance n=20 224.alb & 1 & 0 & Optimal &  0.32 & 3 &  0.00 &  0.00\\
instance n=20 225.alb & 1 & 0 & Optimal &  0.32 & 3 &  0.00 &  0.00\\
instance n=20 226.alb & 1 & 0 & Optimal &  0.33 & 3 &  0.00 &  0.00\\
instance n=20 227.alb & 1 & 0 & Optimal &  0.32 & 3 &  0.00 &  0.00\\
instance n=20 228.alb & 1 & 0 & Optimal &  0.31 & 2 &  0.00 &  0.00\\
instance n=20 229.alb & 1 & 0 & Optimal &  0.32 & 3 &  0.00 &  0.00\\
instance n=20 23.alb & 1 & 0 & Optimal &  0.60 & 13 &  0.00 &  0.00\\
instance n=20 230.alb & 1 & 0 & Optimal &  0.32 & 3 &  0.00 &  0.00\\
instance n=20 231.alb & 1 & 0 & Optimal &  0.33 & 3 &  0.00 &  0.00\\
instance n=20 232.alb & 1 & 0 & Optimal &  0.32 & 3 &  0.00 &  0.00\\
instance n=20 233.alb & 1 & 0 & Optimal &  0.32 & 3 &  0.00 &  0.00\\
instance n=20 234.alb & 1 & 0 & Optimal &  0.33 & 3 &  0.00 &  0.00\\
instance n=20 235.alb & 1 & 0 & Optimal &  0.32 & 3 &  0.00 &  0.00\\
instance n=20 236.alb & 1 & 0 & Optimal &  0.32 & 3 &  0.00 &  0.00\\
instance n=20 237.alb & 1 & 0 & Optimal &  0.33 & 3 &  0.00 &  0.00\\
instance n=20 238.alb & 1 & 0 & Optimal &  0.33 & 3 &  0.00 &  0.00\\
instance n=20 239.alb & 1 & 0 & Optimal &  0.32 & 3 &  0.00 &  0.00\\
instance n=20 24.alb & 1 & 0 & Optimal &  0.38 & 11 &  0.00 &  0.00\\
instance n=20 240.alb & 1 & 0 & Optimal &  0.33 & 3 &  0.00 &  0.00\\
instance n=20 241.alb & 1 & 0 & Optimal &  0.47 & 13 &  0.00 &  0.00\\
instance n=20 242.alb & 1 & 0 & Optimal &  0.39 & 12 &  0.00 &  0.00\\
instance n=20 243.alb & 1 & 0 & Optimal &  0.39 & 10 &  0.00 &  0.00\\
instance n=20 244.alb & 1 & 0 & Optimal &  0.39 & 11 &  0.00 &  0.00\\
instance n=20 245.alb & 1 & 0 & Optimal &  0.38 & 13 &  0.00 &  0.00\\
instance n=20 246.alb & 1 & 0 & Optimal &  0.37 & 13 &  0.00 &  0.00\\
instance n=20 247.alb & 1 & 0 & Optimal &  0.40 & 11 &  0.00 &  0.00\\
instance n=20 248.alb & 1 & 0 & Optimal &  0.39 & 11 &  0.00 &  0.00\\
instance n=20 249.alb & 1 & 0 & Optimal &  0.44 & 13 &  0.00 &  0.00\\
instance n=20 25.alb & 1 & 0 & Optimal &  0.40 & 11 &  0.00 &  0.00\\
instance n=20 250.alb & 1 & 0 & Optimal &  0.40 & 10 &  0.00 &  0.00\\
instance n=20 251.alb & 1 & 0 & Optimal &  0.37 & 12 &  0.00 &  0.00\\
instance n=20 252.alb & 1 & 0 & Optimal &  0.40 & 11 &  0.00 &  0.00\\
instance n=20 253.alb & 1 & 0 & Optimal &  0.58 & 13 &  0.00 &  0.00\\
instance n=20 254.alb & 1 & 0 & Optimal &  0.40 & 12 &  0.00 &  0.00\\
instance n=20 255.alb & 1 & 0 & Optimal &  0.39 & 13 &  0.00 &  0.00\\
instance n=20 256.alb & 1 & 0 & Optimal &  0.40 & 14 &  0.00 &  0.00\\
instance n=20 257.alb & 1 & 0 & Optimal &  0.39 & 10 &  0.00 &  0.00\\
instance n=20 258.alb & 1 & 0 & Optimal &  0.40 & 13 &  0.00 &  0.00\\
instance n=20 259.alb & 1 & 0 & Optimal &  0.40 & 13 &  0.00 &  0.00\\
instance n=20 26.alb & 1 & 0 & Optimal &  0.38 & 12 &  0.00 &  0.00\\
instance n=20 260.alb & 1 & 0 & Optimal &  0.43 & 12 &  0.00 &  0.00\\
instance n=20 261.alb & 1 & 0 & Optimal &  0.41 & 12 &  0.00 &  0.00\\
instance n=20 262.alb & 1 & 0 & Optimal &  0.39 & 11 &  0.00 &  0.00\\
instance n=20 263.alb & 1 & 0 & Optimal &  0.39 & 12 &  0.00 &  0.00\\
instance n=20 264.alb & 1 & 0 & Optimal &  0.41 & 12 &  0.00 &  0.00\\
instance n=20 265.alb & 1 & 0 & Optimal &  0.44 & 12 &  0.00 &  0.00\\
instance n=20 266.alb & 1 & 0 & Optimal &  0.34 & 5 &  0.00 &  0.00\\
instance n=20 267.alb & 1 & 0 & Optimal &  0.34 & 6 &  0.00 &  0.00\\
instance n=20 268.alb & 1 & 0 & Optimal &  0.34 & 6 &  0.00 &  0.00\\
instance n=20 269.alb & 1 & 0 & Optimal &  0.38 & 7 &  0.00 &  0.00\\
instance n=20 27.alb & 1 & 0 & Optimal &  0.49 & 13 &  0.00 &  0.00\\
instance n=20 270.alb & 1 & 0 & Optimal &  0.35 & 7 &  0.00 &  0.00\\
instance n=20 271.alb & 1 & 0 & Optimal &  0.33 & 6 &  0.00 &  0.00\\
instance n=20 272.alb & 1 & 0 & Optimal &  0.34 & 5 &  0.00 &  0.00\\
instance n=20 273.alb & 1 & 0 & Optimal &  0.34 & 5 &  0.00 &  0.00\\
instance n=20 274.alb & 1 & 0 & Optimal &  0.38 & 6 &  0.00 &  0.00\\
instance n=20 275.alb & 1 & 0 & Optimal &  0.34 & 5 &  0.00 &  0.00\\
instance n=20 276.alb & 1 & 0 & Optimal &  0.33 & 4 &  0.00 &  0.00\\
instance n=20 277.alb & 1 & 0 & Optimal &  0.33 & 4 &  0.00 &  0.00\\
instance n=20 278.alb & 1 & 0 & Optimal &  0.37 & 6 &  0.00 &  0.00\\
instance n=20 279.alb & 1 & 0 & Optimal &  0.36 & 6 &  0.00 &  0.00\\
instance n=20 28.alb & 1 & 0 & Optimal &  0.43 & 12 &  0.00 &  0.00\\
instance n=20 280.alb & 1 & 0 & Optimal &  0.33 & 5 &  0.00 &  0.00\\
instance n=20 281.alb & 1 & 0 & Optimal &  0.34 & 4 &  0.00 &  0.00\\
instance n=20 282.alb & 1 & 0 & Optimal &  0.34 & 4 &  0.00 &  0.00\\
instance n=20 283.alb & 1 & 0 & Optimal &  0.33 & 5 &  0.00 &  0.00\\
instance n=20 284.alb & 1 & 0 & Optimal &  0.35 & 5 &  0.00 &  0.00\\
instance n=20 285.alb & 1 & 0 & Optimal &  0.32 & 5 &  0.00 &  0.00\\
instance n=20 286.alb & 1 & 0 & Optimal &  0.34 & 5 &  0.00 &  0.00\\
instance n=20 287.alb & 1 & 0 & Optimal &  0.34 & 5 &  0.00 &  0.00\\
instance n=20 288.alb & 1 & 0 & Optimal &  0.33 & 6 &  0.00 &  0.00\\
instance n=20 289.alb & 1 & 0 & Optimal &  0.33 & 5 &  0.00 &  0.00\\
instance n=20 29.alb & 1 & 0 & Optimal &  0.39 & 10 &  0.00 &  0.00\\
instance n=20 290.alb & 1 & 0 & Optimal &  0.33 & 5 &  0.00 &  0.00\\
instance n=20 291.alb & 1 & 0 & Optimal &  0.31 & 3 &  0.00 &  0.00\\
instance n=20 292.alb & 1 & 0 & Optimal &  0.31 & 3 &  0.00 &  0.00\\
instance n=20 293.alb & 1 & 0 & Optimal &  0.32 & 3 &  0.00 &  0.00\\
instance n=20 294.alb & 1 & 0 & Optimal &  0.33 & 3 &  0.00 &  0.00\\
instance n=20 295.alb & 1 & 0 & Optimal &  0.31 & 3 &  0.00 &  0.00\\
instance n=20 296.alb & 1 & 0 & Optimal &  0.33 & 3 &  0.00 &  0.00\\
instance n=20 297.alb & 1 & 0 & Optimal &  0.31 & 3 &  0.00 &  0.00\\
instance n=20 298.alb & 1 & 0 & Optimal &  0.30 & 3 &  0.00 &  0.00\\
instance n=20 299.alb & 1 & 0 & Optimal &  0.34 & 3 &  0.00 &  0.00\\
instance n=20 3.alb & 1 & 0 & Optimal &  0.33 & 3 &  0.00 &  0.00\\
instance n=20 30.alb & 1 & 0 & Optimal &  0.54 & 16 &  0.00 &  0.00\\
instance n=20 300.alb & 1 & 0 & Optimal &  0.32 & 4 &  0.00 &  0.00\\
instance n=20 301.alb & 1 & 0 & Optimal &  0.32 & 3 &  0.00 &  0.00\\
instance n=20 302.alb & 1 & 0 & Optimal &  0.32 & 3 &  0.00 &  0.00\\
instance n=20 303.alb & 1 & 0 & Optimal &  0.32 & 3 &  0.00 &  0.00\\
instance n=20 304.alb & 1 & 0 & Optimal &  0.32 & 3 &  0.00 &  0.00\\
instance n=20 305.alb & 1 & 0 & Optimal &  0.33 & 3 &  0.00 &  0.00\\
instance n=20 306.alb & 1 & 0 & Optimal &  0.32 & 3 &  0.00 &  0.00\\
instance n=20 307.alb & 1 & 0 & Optimal &  0.32 & 3 &  0.00 &  0.00\\
instance n=20 308.alb & 1 & 0 & Optimal &  0.31 & 3 &  0.00 &  0.00\\
instance n=20 309.alb & 1 & 0 & Optimal &  0.33 & 3 &  0.00 &  0.00\\
instance n=20 31.alb & 1 & 0 & Optimal &  0.41 & 12 &  0.00 &  0.00\\
instance n=20 310.alb & 1 & 0 & Optimal &  0.31 & 3 &  0.00 &  0.00\\
instance n=20 311.alb & 1 & 0 & Optimal &  0.31 & 3 &  0.00 &  0.00\\
instance n=20 312.alb & 1 & 0 & Optimal &  0.32 & 4 &  0.00 &  0.00\\
instance n=20 313.alb & 1 & 0 & Optimal &  0.35 & 3 &  0.00 &  0.00\\
instance n=20 314.alb & 1 & 0 & Optimal &  0.32 & 3 &  0.00 &  0.00\\
instance n=20 315.alb & 1 & 0 & Optimal &  0.32 & 3 &  0.00 &  0.00\\
instance n=20 316.alb & 1 & 0 & Optimal &  0.41 & 10 &  0.00 &  0.00\\
instance n=20 317.alb & 1 & 0 & Optimal &  0.44 & 10 &  0.00 &  0.00\\
instance n=20 318.alb & 1 & 0 & Optimal &  0.40 & 10 &  0.00 &  0.00\\
instance n=20 319.alb & 1 & 0 & Optimal &  0.51 & 14 &  0.00 &  0.00\\
instance n=20 32.alb & 1 & 0 & Optimal &  0.42 & 13 &  0.00 &  0.00\\
instance n=20 320.alb & 1 & 0 & Optimal &  0.44 & 12 &  0.00 &  0.00\\
instance n=20 321.alb & 1 & 0 & Optimal &  0.55 & 14 &  0.00 &  0.00\\
instance n=20 322.alb & 1 & 0 & Optimal &  0.43 & 12 &  0.00 &  0.00\\
instance n=20 323.alb & 1 & 0 & Optimal &  0.43 & 13 &  0.00 &  0.00\\
instance n=20 324.alb & 1 & 0 & Optimal &  0.49 & 9 &  0.00 &  0.00\\
instance n=20 325.alb & 1 & 0 & Optimal &  0.44 & 14 &  0.00 &  0.00\\
instance n=20 326.alb & 1 & 0 & Optimal &  0.55 & 14 &  0.00 &  0.00\\
instance n=20 327.alb & 1 & 0 & Optimal &  0.55 & 13 &  0.00 &  0.00\\
instance n=20 328.alb & 1 & 0 & Optimal &  0.43 & 13 &  0.00 &  0.00\\
instance n=20 329.alb & 1 & 0 & Optimal &  0.40 & 10 &  0.00 &  0.00\\
instance n=20 33.alb & 1 & 0 & Optimal &  0.39 & 11 &  0.00 &  0.00\\
instance n=20 330.alb & 1 & 0 & Optimal &  0.44 & 12 &  0.00 &  0.00\\
instance n=20 331.alb & 1 & 0 & Optimal &  0.40 & 13 &  0.00 &  0.00\\
instance n=20 332.alb & 1 & 0 & Optimal &  0.38 & 13 &  0.00 &  0.00\\
instance n=20 333.alb & 1 & 0 & Optimal &  0.49 & 11 &  0.00 &  0.00\\
instance n=20 334.alb & 1 & 0 & Optimal &  0.40 & 10 &  0.00 &  0.00\\
instance n=20 335.alb & 1 & 0 & Optimal &  0.45 & 14 &  0.00 &  0.00\\
instance n=20 336.alb & 1 & 0 & Optimal &  0.39 & 11 &  0.00 &  0.00\\
instance n=20 337.alb & 1 & 0 & Optimal &  0.38 & 10 &  0.00 &  0.00\\
instance n=20 338.alb & 1 & 0 & Optimal &  0.54 & 14 &  0.00 &  0.00\\
instance n=20 339.alb & 1 & 0 & Optimal &  0.46 & 13 &  0.00 &  0.00\\
instance n=20 34.alb & 1 & 0 & Optimal &  0.70 & 12 &  0.00 &  0.00\\
instance n=20 340.alb & 1 & 0 & Optimal &  0.70 & 11 &  0.00 &  0.00\\
instance n=20 341.alb & 1 & 0 & Optimal &  0.36 & 6 &  0.00 &  0.00\\
instance n=20 342.alb & 1 & 0 & Optimal &  0.34 & 6 &  0.00 &  0.00\\
instance n=20 343.alb & 1 & 0 & Optimal &  0.34 & 6 &  0.00 &  0.00\\
instance n=20 344.alb & 1 & 0 & Optimal &  0.35 & 6 &  0.00 &  0.00\\
instance n=20 345.alb & 1 & 0 & Optimal &  0.33 & 4 &  0.00 &  0.00\\
instance n=20 346.alb & 1 & 0 & Optimal &  0.41 & 5 &  0.00 &  0.00\\
instance n=20 347.alb & 1 & 0 & Optimal &  0.34 & 6 &  0.00 &  0.00\\
instance n=20 348.alb & 1 & 0 & Optimal &  0.39 & 5 &  0.00 &  0.00\\
instance n=20 349.alb & 1 & 0 & Optimal &  0.39 & 5 &  0.00 &  0.00\\
instance n=20 35.alb & 1 & 0 & Optimal &  0.45 & 12 &  0.00 &  0.00\\
instance n=20 350.alb & 1 & 0 & Optimal &  0.34 & 5 &  0.00 &  0.00\\
instance n=20 351.alb & 1 & 0 & Optimal &  0.34 & 5 &  0.00 &  0.00\\
instance n=20 352.alb & 1 & 0 & Optimal &  0.33 & 4 &  0.00 &  0.00\\
instance n=20 353.alb & 1 & 0 & Optimal &  0.38 & 6 &  0.00 &  0.00\\
instance n=20 354.alb & 1 & 0 & Optimal &  0.37 & 6 &  0.00 &  0.00\\
instance n=20 355.alb & 1 & 0 & Optimal &  0.32 & 5 &  0.00 &  0.00\\
instance n=20 356.alb & 1 & 0 & Optimal &  0.39 & 5 &  0.00 &  0.00\\
instance n=20 357.alb & 1 & 0 & Optimal &  0.39 & 5 &  0.00 &  0.00\\
instance n=20 358.alb & 1 & 0 & Optimal &  0.32 & 4 &  0.00 &  0.00\\
instance n=20 359.alb & 1 & 0 & Optimal &  0.31 & 4 &  0.00 &  0.00\\
instance n=20 36.alb & 1 & 0 & Optimal &  0.43 & 13 &  0.00 &  0.00\\
instance n=20 360.alb & 1 & 0 & Optimal &  0.36 & 6 &  0.00 &  0.00\\
instance n=20 361.alb & 1 & 0 & Optimal &  0.33 & 5 &  0.00 &  0.00\\
instance n=20 362.alb & 1 & 0 & Optimal &  0.34 & 5 &  0.00 &  0.00\\
instance n=20 363.alb & 1 & 0 & Optimal &  0.37 & 7 &  0.00 &  0.00\\
instance n=20 364.alb & 1 & 0 & Optimal &  0.31 & 4 &  0.00 &  0.00\\
instance n=20 365.alb & 1 & 0 & Optimal &  0.39 & 5 &  0.00 &  0.00\\
instance n=20 366.alb & 1 & 0 & Optimal &  0.33 & 3 &  0.00 &  0.00\\
instance n=20 367.alb & 1 & 0 & Optimal &  0.32 & 3 &  0.00 &  0.00\\
instance n=20 368.alb & 1 & 0 & Optimal &  0.32 & 3 &  0.00 &  0.00\\
instance n=20 369.alb & 1 & 0 & Optimal &  0.32 & 3 &  0.00 &  0.00\\
instance n=20 37.alb & 1 & 0 & Optimal &  0.39 & 12 &  0.00 &  0.00\\
instance n=20 370.alb & 1 & 0 & Optimal &  0.32 & 3 &  0.00 &  0.00\\
instance n=20 371.alb & 1 & 0 & Optimal &  0.34 & 3 &  0.00 &  0.00\\
instance n=20 372.alb & 1 & 0 & Optimal &  0.33 & 3 &  0.00 &  0.00\\
instance n=20 373.alb & 1 & 0 & Optimal &  0.33 & 3 &  0.00 &  0.00\\
instance n=20 374.alb & 1 & 0 & Optimal &  0.33 & 3 &  0.00 &  0.00\\
instance n=20 375.alb & 1 & 0 & Optimal &  0.33 & 3 &  0.00 &  0.00\\
instance n=20 376.alb & 1 & 0 & Optimal &  0.33 & 3 &  0.00 &  0.00\\
instance n=20 377.alb & 1 & 0 & Optimal &  0.32 & 3 &  0.00 &  0.00\\
instance n=20 378.alb & 1 & 0 & Optimal &  0.32 & 3 &  0.00 &  0.00\\
instance n=20 379.alb & 1 & 0 & Optimal &  0.33 & 4 &  0.00 &  0.00\\
instance n=20 38.alb & 1 & 0 & Optimal &  0.41 & 12 &  0.00 &  0.00\\
instance n=20 380.alb & 1 & 0 & Optimal &  0.34 & 3 &  0.00 &  0.00\\
instance n=20 381.alb & 1 & 0 & Optimal &  0.33 & 3 &  0.00 &  0.00\\
instance n=20 382.alb & 1 & 0 & Optimal &  0.33 & 4 &  0.00 &  0.00\\
instance n=20 383.alb & 1 & 0 & Optimal &  0.32 & 3 &  0.00 &  0.00\\
instance n=20 384.alb & 1 & 0 & Optimal &  0.33 & 3 &  0.00 &  0.00\\
instance n=20 385.alb & 1 & 0 & Optimal &  0.32 & 3 &  0.00 &  0.00\\
instance n=20 386.alb & 1 & 0 & Optimal &  0.32 & 3 &  0.00 &  0.00\\
instance n=20 387.alb & 1 & 0 & Optimal &  0.33 & 3 &  0.00 &  0.00\\
instance n=20 388.alb & 1 & 0 & Optimal &  0.33 & 3 &  0.00 &  0.00\\
instance n=20 389.alb & 1 & 0 & Optimal &  0.32 & 3 &  0.00 &  0.00\\
instance n=20 39.alb & 1 & 0 & Optimal &  0.42 & 13 &  0.00 &  0.00\\
instance n=20 390.alb & 1 & 0 & Optimal &  0.33 & 3 &  0.00 &  0.00\\
instance n=20 391.alb & 1 & 0 & Optimal &  0.39 & 11 &  0.00 &  0.00\\
instance n=20 392.alb & 1 & 0 & Optimal &  0.38 & 14 &  0.00 &  0.00\\
instance n=20 393.alb & 1 & 0 & Optimal &  0.40 & 11 &  0.00 &  0.00\\
instance n=20 394.alb & 1 & 0 & Optimal &  0.42 & 12 &  0.00 &  0.00\\
instance n=20 395.alb & 1 & 0 & Optimal &  0.40 & 12 &  0.00 &  0.00\\
instance n=20 396.alb & 1 & 0 & Optimal &  0.44 & 13 &  0.00 &  0.00\\
instance n=20 397.alb & 1 & 0 & Optimal &  0.38 & 10 &  0.00 &  0.00\\
instance n=20 398.alb & 1 & 0 & Optimal &  0.40 & 11 &  0.00 &  0.00\\
instance n=20 399.alb & 1 & 0 & Optimal &  0.42 & 13 &  0.00 &  0.00\\
instance n=20 4.alb & 1 & 0 & Optimal &  0.30 & 3 &  0.00 &  0.00\\
instance n=20 40.alb & 1 & 0 & Optimal &  0.38 & 12 &  0.00 &  0.00\\
instance n=20 400.alb & 1 & 0 & Optimal &  0.41 & 12 &  0.00 &  0.00\\
instance n=20 401.alb & 1 & 0 & Optimal &  0.46 & 12 &  0.00 &  0.00\\
instance n=20 402.alb & 1 & 0 & Optimal &  0.46 & 12 &  0.00 &  0.00\\
instance n=20 403.alb & 1 & 0 & Optimal &  0.42 & 12 &  0.00 &  0.00\\
instance n=20 404.alb & 1 & 0 & Optimal &  0.49 & 10 &  0.00 &  0.00\\
instance n=20 405.alb & 1 & 0 & Optimal &  0.40 & 12 &  0.00 &  0.00\\
instance n=20 406.alb & 1 & 0 & Optimal &  0.41 & 14 &  0.00 &  0.00\\
instance n=20 407.alb & 1 & 0 & Optimal &  0.40 & 10 &  0.00 &  0.00\\
instance n=20 408.alb & 1 & 0 & Optimal &  0.52 & 14 &  0.00 &  0.00\\
instance n=20 409.alb & 1 & 0 & Optimal &  0.46 & 12 &  0.00 &  0.00\\
instance n=20 41.alb & 1 & 0 & Optimal &  0.36 & 6 &  0.00 &  0.00\\
instance n=20 410.alb & 1 & 0 & Optimal &  0.40 & 11 &  0.00 &  0.00\\
instance n=20 411.alb & 1 & 0 & Optimal &  0.53 & 15 &  0.00 &  0.00\\
instance n=20 412.alb & 1 & 0 & Optimal &  0.40 & 11 &  0.00 &  0.00\\
instance n=20 413.alb & 1 & 0 & Optimal &  0.40 & 10 &  0.00 &  0.00\\
instance n=20 414.alb & 1 & 0 & Optimal &  0.50 & 12 &  0.00 &  0.00\\
instance n=20 415.alb & 1 & 0 & Optimal &  0.40 & 10 &  0.00 &  0.00\\
instance n=20 416.alb & 1 & 0 & Optimal &  0.36 & 6 &  0.00 &  0.00\\
instance n=20 417.alb & 1 & 0 & Optimal &  0.33 & 5 &  0.00 &  0.00\\
instance n=20 418.alb & 1 & 0 & Optimal &  0.35 & 6 &  0.00 &  0.00\\
instance n=20 419.alb & 1 & 0 & Optimal &  0.32 & 4 &  0.00 &  0.00\\
instance n=20 42.alb & 1 & 0 & Optimal &  0.39 & 5 &  0.00 &  0.00\\
instance n=20 420.alb & 1 & 0 & Optimal &  0.33 & 5 &  0.00 &  0.00\\
instance n=20 421.alb & 1 & 0 & Optimal &  0.34 & 6 &  0.00 &  0.00\\
instance n=20 422.alb & 1 & 0 & Optimal &  0.32 & 4 &  0.00 &  0.00\\
instance n=20 423.alb & 1 & 0 & Optimal &  0.39 & 6 &  0.00 &  0.00\\
instance n=20 424.alb & 1 & 0 & Optimal &  0.36 & 5 &  0.00 &  0.00\\
instance n=20 425.alb & 1 & 0 & Optimal &  0.34 & 6 &  0.00 &  0.00\\
instance n=20 426.alb & 1 & 0 & Optimal &  0.33 & 5 &  0.00 &  0.00\\
instance n=20 427.alb & 1 & 0 & Optimal &  0.35 & 6 &  0.00 &  0.00\\
instance n=20 428.alb & 1 & 0 & Optimal &  0.34 & 5 &  0.00 &  0.00\\
instance n=20 429.alb & 1 & 0 & Optimal &  0.33 & 4 &  0.00 &  0.00\\
instance n=20 43.alb & 1 & 0 & Optimal &  0.34 & 5 &  0.00 &  0.00\\
instance n=20 430.alb & 1 & 0 & Optimal &  0.33 & 5 &  0.00 &  0.00\\
instance n=20 431.alb & 1 & 0 & Optimal &  0.38 & 6 &  0.00 &  0.00\\
instance n=20 432.alb & 1 & 0 & Optimal &  0.35 & 5 &  0.00 &  0.00\\
instance n=20 433.alb & 1 & 0 & Optimal &  0.33 & 5 &  0.00 &  0.00\\
instance n=20 434.alb & 1 & 0 & Optimal &  0.34 & 5 &  0.00 &  0.00\\
instance n=20 435.alb & 1 & 0 & Optimal &  0.38 & 7 &  0.00 &  0.00\\
instance n=20 436.alb & 1 & 0 & Optimal &  0.34 & 5 &  0.00 &  0.00\\
instance n=20 437.alb & 1 & 0 & Optimal &  0.34 & 5 &  0.00 &  0.00\\
instance n=20 438.alb & 1 & 0 & Optimal &  0.35 & 6 &  0.00 &  0.00\\
instance n=20 439.alb & 1 & 0 & Optimal &  0.31 & 5 &  0.00 &  0.00\\
instance n=20 44.alb & 1 & 0 & Optimal &  0.34 & 5 &  0.00 &  0.00\\
instance n=20 440.alb & 1 & 0 & Optimal &  0.33 & 5 &  0.00 &  0.00\\
instance n=20 441.alb & 1 & 0 & Optimal &  0.34 & 3 &  0.00 &  0.00\\
instance n=20 442.alb & 1 & 0 & Optimal &  0.33 & 3 &  0.00 &  0.00\\
instance n=20 443.alb & 1 & 0 & Optimal &  0.32 & 3 &  0.00 &  0.00\\
instance n=20 444.alb & 1 & 0 & Optimal &  0.33 & 3 &  0.00 &  0.00\\
instance n=20 445.alb & 1 & 0 & Optimal &  0.31 & 3 &  0.00 &  0.00\\
instance n=20 446.alb & 1 & 0 & Optimal &  0.31 & 3 &  0.00 &  0.00\\
instance n=20 447.alb & 1 & 0 & Optimal &  0.32 & 3 &  0.00 &  0.00\\
instance n=20 448.alb & 1 & 0 & Optimal &  0.32 & 3 &  0.00 &  0.00\\
instance n=20 449.alb & 1 & 0 & Optimal &  0.32 & 3 &  0.00 &  0.00\\
instance n=20 45.alb & 1 & 0 & Optimal &  0.32 & 6 &  0.00 &  0.00\\
instance n=20 450.alb & 1 & 0 & Optimal &  0.32 & 3 &  0.00 &  0.00\\
instance n=20 451.alb & 1 & 0 & Optimal &  0.32 & 3 &  0.00 &  0.00\\
instance n=20 452.alb & 1 & 0 & Optimal &  0.31 & 3 &  0.00 &  0.00\\
instance n=20 453.alb & 1 & 0 & Optimal &  0.32 & 3 &  0.00 &  0.00\\
instance n=20 454.alb & 1 & 0 & Optimal &  0.32 & 3 &  0.00 &  0.00\\
instance n=20 455.alb & 1 & 0 & Optimal &  0.33 & 3 &  0.00 &  0.00\\
instance n=20 456.alb & 1 & 0 & Optimal &  0.33 & 4 &  0.00 &  0.00\\
instance n=20 457.alb & 1 & 0 & Optimal &  0.32 & 3 &  0.00 &  0.00\\
instance n=20 458.alb & 1 & 0 & Optimal &  0.32 & 3 &  0.00 &  0.00\\
instance n=20 459.alb & 1 & 0 & Optimal &  0.32 & 3 &  0.00 &  0.00\\
instance n=20 46.alb & 1 & 0 & Optimal &  0.32 & 4 &  0.00 &  0.00\\
instance n=20 460.alb & 1 & 0 & Optimal &  0.32 & 3 &  0.00 &  0.00\\
instance n=20 461.alb & 1 & 0 & Optimal &  0.32 & 3 &  0.00 &  0.00\\
instance n=20 462.alb & 1 & 0 & Optimal &  0.32 & 3 &  0.00 &  0.00\\
instance n=20 463.alb & 1 & 0 & Optimal &  0.32 & 3 &  0.00 &  0.00\\
instance n=20 464.alb & 1 & 0 & Optimal &  0.33 & 3 &  0.00 &  0.00\\
instance n=20 465.alb & 1 & 0 & Optimal &  0.32 & 3 &  0.00 &  0.00\\
instance n=20 466.alb & 1 & 0 & Optimal &  0.32 & 13 &  0.00 &  0.00\\
instance n=20 467.alb & 1 & 0 & Optimal &  0.33 & 14 &  0.00 &  0.00\\
instance n=20 468.alb & 1 & 0 & Optimal &  0.37 & 13 &  0.00 &  0.00\\
instance n=20 469.alb & 1 & 0 & Optimal &  0.36 & 14 &  0.00 &  0.00\\
instance n=20 47.alb & 1 & 0 & Optimal &  0.31 & 4 &  0.00 &  0.00\\
instance n=20 470.alb & 1 & 0 & Optimal &  0.37 & 12 &  0.00 &  0.00\\
instance n=20 471.alb & 1 & 0 & Optimal &  0.36 & 12 &  0.00 &  0.00\\
instance n=20 472.alb & 1 & 0 & Optimal &  0.35 & 13 &  0.00 &  0.00\\
instance n=20 473.alb & 1 & 0 & Optimal &  0.32 & 10 &  0.00 &  0.00\\
instance n=20 474.alb & 1 & 0 & Optimal &  0.34 & 14 &  0.00 &  0.00\\
instance n=20 475.alb & 1 & 0 & Optimal &  0.33 & 11 &  0.00 &  0.00\\
instance n=20 476.alb & 1 & 0 & Optimal &  0.35 & 11 &  0.00 &  0.00\\
instance n=20 477.alb & 1 & 0 & Optimal &  0.34 & 11 &  0.00 &  0.00\\
instance n=20 478.alb & 1 & 0 & Optimal &  0.33 & 12 &  0.00 &  0.00\\
instance n=20 479.alb & 1 & 0 & Optimal &  0.32 & 13 &  0.00 &  0.00\\
instance n=20 48.alb & 1 & 0 & Optimal &  0.34 & 5 &  0.00 &  0.00\\
instance n=20 480.alb & 1 & 0 & Optimal &  0.32 & 13 &  0.00 &  0.00\\
instance n=20 481.alb & 1 & 0 & Optimal &  0.34 & 13 &  0.00 &  0.00\\
instance n=20 482.alb & 1 & 0 & Optimal &  0.32 & 13 &  0.00 &  0.00\\
instance n=20 483.alb & 1 & 0 & Optimal &  0.36 & 12 &  0.00 &  0.00\\
instance n=20 484.alb & 1 & 0 & Optimal &  0.38 & 13 &  0.00 &  0.00\\
instance n=20 485.alb & 1 & 0 & Optimal &  0.34 & 15 &  0.00 &  0.00\\
instance n=20 486.alb & 1 & 0 & Optimal &  0.33 & 11 &  0.00 &  0.00\\
instance n=20 487.alb & 1 & 0 & Optimal &  0.34 & 12 &  0.00 &  0.00\\
instance n=20 488.alb & 1 & 0 & Optimal &  0.39 & 15 &  0.00 &  0.00\\
instance n=20 489.alb & 1 & 0 & Optimal &  0.35 & 12 &  0.00 &  0.00\\
instance n=20 49.alb & 1 & 0 & Optimal &  0.33 & 4 &  0.00 &  0.00\\
instance n=20 490.alb & 1 & 0 & Optimal &  0.35 & 12 &  0.00 &  0.00\\
instance n=20 491.alb & 1 & 0 & Optimal &  0.32 & 6 &  0.00 &  0.00\\
instance n=20 492.alb & 1 & 0 & Optimal &  0.33 & 5 &  0.00 &  0.00\\
instance n=20 493.alb & 1 & 0 & Optimal &  0.34 & 5 &  0.00 &  0.00\\
instance n=20 494.alb & 1 & 0 & Optimal &  0.33 & 6 &  0.00 &  0.00\\
instance n=20 495.alb & 1 & 0 & Optimal &  0.32 & 6 &  0.00 &  0.00\\
instance n=20 496.alb & 1 & 0 & Optimal &  0.31 & 5 &  0.00 &  0.00\\
instance n=20 497.alb & 1 & 0 & Optimal &  0.32 & 6 &  0.00 &  0.00\\
instance n=20 498.alb & 1 & 0 & Optimal &  0.32 & 6 &  0.00 &  0.00\\
instance n=20 499.alb & 1 & 0 & Optimal &  0.34 & 5 &  0.00 &  0.00\\
instance n=20 5.alb & 1 & 0 & Optimal &  0.33 & 3 &  0.00 &  0.00\\
instance n=20 50.alb & 1 & 0 & Optimal &  0.32 & 4 &  0.00 &  0.00\\
instance n=20 500.alb & 1 & 0 & Optimal &  0.32 & 8 &  0.00 &  0.00\\
instance n=20 501.alb & 1 & 0 & Optimal &  0.33 & 5 &  0.00 &  0.00\\
instance n=20 502.alb & 1 & 0 & Optimal &  0.32 & 4 &  0.00 &  0.00\\
instance n=20 503.alb & 1 & 0 & Optimal &  0.32 & 6 &  0.00 &  0.00\\
instance n=20 504.alb & 1 & 0 & Optimal &  0.35 & 6 &  0.00 &  0.00\\
instance n=20 505.alb & 1 & 0 & Optimal &  0.32 & 6 &  0.00 &  0.00\\
instance n=20 506.alb & 1 & 0 & Optimal &  0.31 & 5 &  0.00 &  0.00\\
instance n=20 507.alb & 1 & 0 & Optimal &  0.33 & 5 &  0.00 &  0.00\\
instance n=20 508.alb & 1 & 0 & Optimal &  0.32 & 5 &  0.00 &  0.00\\
instance n=20 509.alb & 1 & 0 & Optimal &  0.32 & 4 &  0.00 &  0.00\\
instance n=20 51.alb & 1 & 0 & Optimal &  0.35 & 4 &  0.00 &  0.00\\
instance n=20 510.alb & 1 & 0 & Optimal &  0.36 & 5 &  0.00 &  0.00\\
instance n=20 511.alb & 1 & 0 & Optimal &  0.33 & 5 &  0.00 &  0.00\\
instance n=20 512.alb & 1 & 0 & Optimal &  0.33 & 5 &  0.00 &  0.00\\
instance n=20 513.alb & 1 & 0 & Optimal &  0.33 & 5 &  0.00 &  0.00\\
instance n=20 514.alb & 1 & 0 & Optimal &  0.32 & 5 &  0.00 &  0.00\\
instance n=20 515.alb & 1 & 0 & Optimal &  0.32 & 6 &  0.00 &  0.00\\
instance n=20 516.alb & 1 & 0 & Optimal &  0.31 & 3 &  0.00 &  0.00\\
instance n=20 517.alb & 1 & 0 & Optimal &  0.33 & 3 &  0.00 &  0.00\\
instance n=20 518.alb & 1 & 0 & Optimal &  0.30 & 3 &  0.00 &  0.00\\
instance n=20 519.alb & 1 & 0 & Optimal &  0.31 & 3 &  0.00 &  0.00\\
instance n=20 52.alb & 1 & 0 & Optimal &  0.33 & 4 &  0.00 &  0.00\\
instance n=20 520.alb & 1 & 0 & Optimal &  0.32 & 3 &  0.00 &  0.00\\
instance n=20 521.alb & 1 & 0 & Optimal &  0.31 & 3 &  0.00 &  0.00\\
instance n=20 522.alb & 1 & 0 & Optimal &  0.32 & 3 &  0.00 &  0.00\\
instance n=20 523.alb & 1 & 0 & Optimal &  0.32 & 3 &  0.00 &  0.00\\
instance n=20 524.alb & 1 & 0 & Optimal &  0.33 & 3 &  0.00 &  0.00\\
instance n=20 525.alb & 1 & 0 & Optimal &  0.32 & 3 &  0.00 &  0.00\\
instance n=20 53.alb & 1 & 0 & Optimal &  0.34 & 5 &  0.00 &  0.00\\
instance n=20 54.alb & 1 & 0 & Optimal &  0.33 & 5 &  0.00 &  0.00\\
instance n=20 55.alb & 1 & 0 & Optimal &  0.31 & 5 &  0.00 &  0.00\\
instance n=20 56.alb & 1 & 0 & Optimal &  0.33 & 4 &  0.00 &  0.00\\
instance n=20 57.alb & 1 & 0 & Optimal &  0.32 & 4 &  0.00 &  0.00\\
instance n=20 58.alb & 1 & 0 & Optimal &  0.31 & 5 &  0.00 &  0.00\\
instance n=20 59.alb & 1 & 0 & Optimal &  0.36 & 4 &  0.00 &  0.00\\
instance n=20 6.alb & 1 & 0 & Optimal &  0.30 & 3 &  0.00 &  0.00\\
instance n=20 60.alb & 1 & 0 & Optimal &  0.36 & 6 &  0.00 &  0.00\\
instance n=20 61.alb & 1 & 0 & Optimal &  0.38 & 7 &  0.00 &  0.00\\
instance n=20 62.alb & 1 & 0 & Optimal &  0.37 & 5 &  0.00 &  0.00\\
instance n=20 63.alb & 1 & 0 & Optimal &  0.33 & 5 &  0.00 &  0.00\\
instance n=20 64.alb & 1 & 0 & Optimal &  0.33 & 5 &  0.00 &  0.00\\
instance n=20 65.alb & 1 & 0 & Optimal &  0.33 & 5 &  0.00 &  0.00\\
instance n=20 66.alb & 1 & 0 & Optimal &  0.31 & 3 &  0.00 &  0.00\\
instance n=20 67.alb & 1 & 0 & Optimal &  0.33 & 3 &  0.00 &  0.00\\
instance n=20 68.alb & 1 & 0 & Optimal &  0.33 & 3 &  0.00 &  0.00\\
instance n=20 69.alb & 1 & 0 & Optimal &  0.30 & 2 &  0.00 &  0.00\\
instance n=20 7.alb & 1 & 0 & Optimal &  0.32 & 3 &  0.00 &  0.00\\
instance n=20 70.alb & 1 & 0 & Optimal &  0.33 & 3 &  0.00 &  0.00\\
instance n=20 71.alb & 1 & 0 & Optimal &  0.32 & 3 &  0.00 &  0.00\\
instance n=20 72.alb & 1 & 0 & Optimal &  0.32 & 3 &  0.00 &  0.00\\
instance n=20 73.alb & 1 & 0 & Optimal &  0.30 & 2 &  0.00 &  0.00\\
instance n=20 74.alb & 1 & 0 & Optimal &  0.32 & 3 &  0.00 &  0.00\\
instance n=20 75.alb & 1 & 0 & Optimal &  0.32 & 3 &  0.00 &  0.00\\
instance n=20 76.alb & 1 & 0 & Optimal &  0.33 & 3 &  0.00 &  0.00\\
instance n=20 77.alb & 1 & 0 & Optimal &  0.33 & 3 &  0.00 &  0.00\\
instance n=20 78.alb & 1 & 0 & Optimal &  0.32 & 3 &  0.00 &  0.00\\
instance n=20 79.alb & 1 & 0 & Optimal &  0.32 & 3 &  0.00 &  0.00\\
instance n=20 8.alb & 1 & 0 & Optimal &  0.32 & 3 &  0.00 &  0.00\\
instance n=20 80.alb & 1 & 0 & Optimal &  0.32 & 3 &  0.00 &  0.00\\
instance n=20 81.alb & 1 & 0 & Optimal &  0.32 & 3 &  0.00 &  0.00\\
instance n=20 82.alb & 1 & 0 & Optimal &  0.32 & 4 &  0.00 &  0.00\\
instance n=20 83.alb & 1 & 0 & Optimal &  0.32 & 3 &  0.00 &  0.00\\
instance n=20 84.alb & 1 & 0 & Optimal &  0.32 & 3 &  0.00 &  0.00\\
instance n=20 85.alb & 1 & 0 & Optimal &  0.33 & 3 &  0.00 &  0.00\\
instance n=20 86.alb & 1 & 0 & Optimal &  0.32 & 3 &  0.00 &  0.00\\
instance n=20 87.alb & 1 & 0 & Optimal &  0.32 & 3 &  0.00 &  0.00\\
instance n=20 88.alb & 1 & 0 & Optimal &  0.33 & 3 &  0.00 &  0.00\\
instance n=20 89.alb & 1 & 0 & Optimal &  0.32 & 3 &  0.00 &  0.00\\
instance n=20 9.alb & 1 & 0 & Optimal &  0.32 & 3 &  0.00 &  0.00\\
instance n=20 90.alb & 1 & 0 & Optimal &  0.33 & 3 &  0.00 &  0.00\\
instance n=20 91.alb & 1 & 0 & Optimal &  0.40 & 11 &  0.00 &  0.00\\
instance n=20 92.alb & 1 & 0 & Optimal &  0.39 & 11 &  0.00 &  0.00\\
instance n=20 93.alb & 1 & 0 & Optimal &  0.38 & 13 &  0.00 &  0.00\\
instance n=20 94.alb & 1 & 0 & Optimal &  0.38 & 10 &  0.00 &  0.00\\
instance n=20 95.alb & 1 & 0 & Optimal &  0.41 & 12 &  0.00 &  0.00\\
instance n=20 96.alb & 1 & 0 & Optimal &  0.40 & 10 &  0.00 &  0.00\\
instance n=20 97.alb & 1 & 0 & Optimal &  0.63 & 15 &  0.00 &  0.00\\
instance n=20 98.alb & 1 & 0 & Optimal &  0.42 & 13 &  0.00 &  0.00\\
instance n=20 99.alb & 1 & 0 & Optimal &  0.41 & 12 &  0.00 &  0.00\\
instance n=50 1.alb & 1 & 0 & Solution & 30.12 & 8 &  0.00 &  0.00\\
instance n=50 10.alb & 1 & 0 & Solution & 30.10 & 7 &  0.00 &  0.00\\
instance n=50 100.alb & 1 & 0 & Solution & 30.10 & 7 &  0.00 &  0.00\\
instance n=50 101.alb & 1 & 0 & Solution & 30.10 & 33 &  0.00 &  0.00\\
instance n=50 102.alb & 1 & 0 & Solution & 30.10 & 34 &  0.00 &  0.00\\
instance n=50 103.alb & 1 & 0 & Solution & 30.10 & 30 &  0.00 &  0.00\\
instance n=50 104.alb & 1 & 0 & Solution & 30.10 & 29 &  0.00 &  0.00\\
instance n=50 105.alb & 1 & 0 & Solution & 30.11 & 27 &  0.00 &  0.00\\
instance n=50 106.alb & 1 & 0 & Solution & 30.14 & 29 &  0.00 &  0.00\\
instance n=50 107.alb & 1 & 0 & Solution & 30.10 & 31 &  0.00 &  0.00\\
instance n=50 108.alb & 1 & 0 & Solution & 30.10 & 33 &  0.00 &  0.00\\
instance n=50 109.alb & 1 & 0 & Solution & 30.10 & 31 &  0.00 &  0.00\\
instance n=50 11.alb & 1 & 0 & Solution & 30.13 & 7 &  0.00 &  0.00\\
instance n=50 110.alb & 1 & 0 & Solution & 30.10 & 28 &  0.00 &  0.00\\
instance n=50 111.alb & 1 & 0 & Solution & 30.10 & 29 &  0.00 &  0.00\\
instance n=50 112.alb & 1 & 0 & Solution & 30.10 & 29 &  0.00 &  0.00\\
instance n=50 113.alb & 1 & 0 & Solution & 30.08 & 31 &  0.00 &  0.00\\
instance n=50 114.alb & 1 & 0 & Solution & 30.09 & 30 &  0.00 &  0.00\\
instance n=50 115.alb & 1 & 0 & Solution & 30.09 & 31 &  0.00 &  0.00\\
instance n=50 116.alb & 1 & 0 & Solution & 30.10 & 34 &  0.00 &  0.00\\
instance n=50 117.alb & 1 & 0 & Solution & 30.11 & 27 &  0.00 &  0.00\\
instance n=50 118.alb & 1 & 0 & Solution & 30.09 & 32 &  0.00 &  0.00\\
instance n=50 119.alb & 1 & 0 & Solution & 30.10 & 27 &  0.00 &  0.00\\
instance n=50 12.alb & 1 & 0 & Solution & 30.11 & 7 &  0.00 &  0.00\\
instance n=50 120.alb & 1 & 0 & Solution & 30.11 & 29 &  0.00 &  0.00\\
instance n=50 121.alb & 1 & 0 & Solution & 30.11 & 32 &  0.00 &  0.00\\
instance n=50 122.alb & 1 & 0 & Solution & 30.09 & 32 &  0.00 &  0.00\\
instance n=50 123.alb & 1 & 0 & Solution & 30.11 & 33 &  0.00 &  0.00\\
instance n=50 124.alb & 1 & 0 & Solution & 30.10 & 31 &  0.00 &  0.00\\
instance n=50 125.alb & 1 & 0 & Solution & 30.10 & 34 &  0.00 &  0.00\\
instance n=50 126.alb & 1 & 0 & Solution & 30.18 & 12 &  0.00 &  0.00\\
instance n=50 127.alb & 1 & 0 & Solution & 30.10 & 14 &  0.00 &  0.00\\
instance n=50 128.alb & 1 & 0 & Solution & 30.12 & 13 &  0.00 &  0.00\\
instance n=50 129.alb & 1 & 0 & Solution & 30.10 & 13 &  0.00 &  0.00\\
instance n=50 13.alb & 1 & 0 & Solution & 30.11 & 6 &  0.00 &  0.00\\
instance n=50 130.alb & 1 & 0 & Solution & 30.10 & 13 &  0.00 &  0.00\\
instance n=50 131.alb & 1 & 0 & Solution & 30.11 & 12 &  0.00 &  0.00\\
instance n=50 132.alb & 1 & 0 & Solution & 30.11 & 13 &  0.00 &  0.00\\
instance n=50 133.alb & 1 & 0 & Solution & 30.10 & 12 &  0.00 &  0.00\\
instance n=50 134.alb & 1 & 0 & Solution & 30.11 & 15 &  0.00 &  0.00\\
instance n=50 135.alb & 1 & 0 & Solution & 30.10 & 14 &  0.00 &  0.00\\
instance n=50 136.alb & 1 & 0 & Solution & 30.10 & 11 &  0.00 &  0.00\\
instance n=50 137.alb & 1 & 0 & Solution & 30.10 & 11 &  0.00 &  0.00\\
instance n=50 138.alb & 1 & 0 & Solution & 30.09 & 12 &  0.00 &  0.00\\
instance n=50 139.alb & 1 & 0 & Solution & 30.10 & 12 &  0.00 &  0.00\\
instance n=50 14.alb & 1 & 0 & Solution & 30.13 & 7 &  0.00 &  0.00\\
instance n=50 140.alb & 1 & 0 & Solution & 30.09 & 12 &  0.00 &  0.00\\
instance n=50 141.alb & 1 & 0 & Solution & 30.11 & 13 &  0.00 &  0.00\\
instance n=50 142.alb & 1 & 0 & Solution & 30.10 & 11 &  0.00 &  0.00\\
instance n=50 143.alb & 1 & 0 & Solution & 30.10 & 12 &  0.00 &  0.00\\
instance n=50 144.alb & 1 & 0 & Solution & 30.10 & 13 &  0.00 &  0.00\\
instance n=50 145.alb & 1 & 0 & Solution & 30.10 & 10 &  0.00 &  0.00\\
instance n=50 146.alb & 1 & 0 & Solution & 30.14 & 13 &  0.00 &  0.00\\
instance n=50 147.alb & 1 & 0 & Solution & 30.11 & 13 &  0.00 &  0.00\\
instance n=50 148.alb & 1 & 0 & Solution & 30.10 & 10 &  0.00 &  0.00\\
instance n=50 149.alb & 1 & 0 & Solution & 30.10 & 12 &  0.00 &  0.00\\
instance n=50 15.alb & 1 & 0 & Solution & 30.12 & 8 &  0.00 &  0.00\\
instance n=50 150.alb & 1 & 0 & Solution & 30.10 & 11 &  0.00 &  0.00\\
instance n=50 151.alb & 1 & 0 & Solution & 30.13 & 7 &  0.00 &  0.00\\
instance n=50 152.alb & 1 & 0 & Solution & 30.12 & 7 &  0.00 &  0.00\\
instance n=50 153.alb & 1 & 0 & Solution & 30.12 & 8 &  0.00 &  0.00\\
instance n=50 154.alb & 1 & 0 & Solution & 30.13 & 8 &  0.00 &  0.00\\
instance n=50 155.alb & 1 & 0 & Solution & 30.32 & 7 &  0.00 &  0.00\\
instance n=50 156.alb & 1 & 0 & Solution & 30.10 & 7 &  0.00 &  0.00\\
instance n=50 157.alb & 1 & 0 & Solution & 30.10 & 8 &  0.00 &  0.00\\
instance n=50 158.alb & 1 & 0 & Solution & 30.09 & 7 &  0.00 &  0.00\\
instance n=50 159.alb & 1 & 0 & Solution & 30.08 & 7 &  0.00 &  0.00\\
instance n=50 16.alb & 1 & 0 & Solution & 30.11 & 8 &  0.00 &  0.00\\
instance n=50 160.alb & 1 & 0 & Solution & 30.10 & 8 &  0.00 &  0.00\\
instance n=50 161.alb & 1 & 0 & Solution & 30.10 & 7 &  0.00 &  0.00\\
instance n=50 162.alb & 1 & 0 & Solution & 30.11 & 8 &  0.00 &  0.00\\
instance n=50 163.alb & 1 & 0 & Solution & 30.14 & 7 &  0.00 &  0.00\\
instance n=50 164.alb & 1 & 0 & Solution & 30.11 & 7 &  0.00 &  0.00\\
instance n=50 165.alb & 1 & 0 & Solution & 30.11 & 8 &  0.00 &  0.00\\
instance n=50 166.alb & 1 & 0 & Solution & 30.11 & 8 &  0.00 &  0.00\\
instance n=50 167.alb & 1 & 0 & Solution & 30.10 & 8 &  0.00 &  0.00\\
instance n=50 168.alb & 1 & 0 & Solution & 30.11 & 9 &  0.00 &  0.00\\
instance n=50 169.alb & 1 & 0 & Solution & 30.10 & 8 &  0.00 &  0.00\\
instance n=50 17.alb & 1 & 0 & Solution & 30.11 & 7 &  0.00 &  0.00\\
instance n=50 170.alb & 1 & 0 & Solution & 30.13 & 8 &  0.00 &  0.00\\
instance n=50 171.alb & 1 & 0 & Solution & 30.10 & 8 &  0.00 &  0.00\\
instance n=50 172.alb & 1 & 0 & Solution & 30.07 & 7 &  0.00 &  0.00\\
instance n=50 173.alb & 1 & 0 & Solution & 30.11 & 8 &  0.00 &  0.00\\
instance n=50 174.alb & 1 & 0 & Solution & 30.09 & 7 &  0.00 &  0.00\\
instance n=50 175.alb & 1 & 0 & Solution & 30.10 & 8 &  0.00 &  0.00\\
instance n=50 176.alb & 1 & 0 & Solution & 30.12 & 33 &  0.00 &  0.00\\
instance n=50 177.alb & 1 & 0 & Solution & 30.10 & 33 &  0.00 &  0.00\\
instance n=50 178.alb & 1 & 0 & Solution & 30.11 & 32 &  0.00 &  0.00\\
instance n=50 179.alb & 1 & 0 & Solution & 30.10 & 32 &  0.00 &  0.00\\
instance n=50 18.alb & 1 & 0 & Solution & 30.10 & 7 &  0.00 &  0.00\\
instance n=50 180.alb & 1 & 0 & Solution & 30.11 & 30 &  0.00 &  0.00\\
instance n=50 181.alb & 1 & 0 & Solution & 30.12 & 33 &  0.00 &  0.00\\
instance n=50 182.alb & 1 & 0 & Solution & 30.11 & 30 &  0.00 &  0.00\\
instance n=50 183.alb & 1 & 0 & Solution & 30.11 & 33 &  0.00 &  0.00\\
instance n=50 184.alb & 1 & 0 & Solution & 30.11 & 39 &  0.00 &  0.00\\
instance n=50 185.alb & 1 & 0 & Solution & 30.12 & 32 &  0.00 &  0.00\\
instance n=50 186.alb & 1 & 0 & Solution & 30.11 & 32 &  0.00 &  0.00\\
instance n=50 187.alb & 1 & 0 & Solution & 30.11 & 31 &  0.00 &  0.00\\
instance n=50 188.alb & 1 & 0 & Solution & 30.11 & 27 &  0.00 &  0.00\\
instance n=50 189.alb & 1 & 0 & Solution & 30.11 & 31 &  0.00 &  0.00\\
instance n=50 19.alb & 1 & 0 & Solution & 30.13 & 8 &  0.00 &  0.00\\
instance n=50 190.alb & 1 & 0 & Solution & 30.10 & 34 &  0.00 &  0.00\\
instance n=50 191.alb & 1 & 0 & Solution & 30.11 & 33 &  0.00 &  0.00\\
instance n=50 192.alb & 1 & 0 & Solution & 30.11 & 31 &  0.00 &  0.00\\
instance n=50 193.alb & 1 & 0 & Solution & 30.10 & 35 &  0.00 &  0.00\\
instance n=50 194.alb & 1 & 0 & Solution & 30.09 & 32 &  0.00 &  0.00\\
instance n=50 195.alb & 1 & 0 & Solution & 30.09 & 33 &  0.00 &  0.00\\
instance n=50 196.alb & 1 & 0 & Solution & 30.10 & 33 &  0.00 &  0.00\\
instance n=50 197.alb & 1 & 0 & Solution & 30.12 & 32 &  0.00 &  0.00\\
instance n=50 198.alb & 1 & 0 & Solution & 30.10 & 32 &  0.00 &  0.00\\
instance n=50 199.alb & 1 & 0 & Solution & 30.09 & 34 &  0.00 &  0.00\\
instance n=50 2.alb & 1 & 0 & Solution & 30.10 & 6 &  0.00 &  0.00\\
instance n=50 20.alb & 1 & 0 & Solution & 30.10 & 8 &  0.00 &  0.00\\
instance n=50 200.alb & 1 & 0 & Solution & 30.10 & 30 &  0.00 &  0.00\\
instance n=50 201.alb & 1 & 0 & Solution & 30.10 & 13 &  0.00 &  0.00\\
instance n=50 202.alb & 1 & 0 & Solution & 30.11 & 10 &  0.00 &  0.00\\
instance n=50 203.alb & 1 & 0 & Solution & 30.12 & 12 &  0.00 &  0.00\\
instance n=50 204.alb & 1 & 0 & Solution & 30.10 & 11 &  0.00 &  0.00\\
instance n=50 205.alb & 1 & 0 & Solution & 30.11 & 13 &  0.00 &  0.00\\
instance n=50 206.alb & 1 & 0 & Solution & 30.10 & 13 &  0.00 &  0.00\\
instance n=50 207.alb & 1 & 0 & Solution & 30.11 & 10 &  0.00 &  0.00\\
instance n=50 208.alb & 1 & 0 & Solution & 30.11 & 14 &  0.00 &  0.00\\
instance n=50 209.alb & 1 & 0 & Solution & 30.11 & 11 &  0.00 &  0.00\\
instance n=50 21.alb & 1 & 0 & Solution & 30.11 & 6 &  0.00 &  0.00\\
instance n=50 210.alb & 1 & 0 & Solution & 30.11 & 14 &  0.00 &  0.00\\
instance n=50 211.alb & 1 & 0 & Solution & 30.11 & 12 &  0.00 &  0.00\\
instance n=50 212.alb & 1 & 0 & Solution & 30.10 & 11 &  0.00 &  0.00\\
instance n=50 213.alb & 1 & 0 & Solution & 30.11 & 14 &  0.00 &  0.00\\
instance n=50 214.alb & 1 & 0 & Solution & 30.12 & 11 &  0.00 &  0.00\\
instance n=50 215.alb & 1 & 0 & Solution & 30.11 & 11 &  0.00 &  0.00\\
instance n=50 216.alb & 1 & 0 & Solution & 30.11 & 13 &  0.00 &  0.00\\
instance n=50 217.alb & 1 & 0 & Solution & 30.11 & 14 &  0.00 &  0.00\\
instance n=50 218.alb & 1 & 0 & Solution & 30.11 & 13 &  0.00 &  0.00\\
instance n=50 219.alb & 1 & 0 & Solution & 30.11 & 11 &  0.00 &  0.00\\
instance n=50 22.alb & 1 & 0 & Solution & 30.12 & 7 &  0.00 &  0.00\\
instance n=50 220.alb & 1 & 0 & Solution & 30.11 & 12 &  0.00 &  0.00\\
instance n=50 221.alb & 1 & 0 & Solution & 30.11 & 12 &  0.00 &  0.00\\
instance n=50 222.alb & 1 & 0 & Solution & 30.11 & 16 &  0.00 &  0.00\\
instance n=50 223.alb & 1 & 0 & Solution & 30.09 & 12 &  0.00 &  0.00\\
instance n=50 224.alb & 1 & 0 & Solution & 30.13 & 11 &  0.00 &  0.00\\
instance n=50 225.alb & 1 & 0 & Solution & 30.11 & 12 &  0.00 &  0.00\\
instance n=50 226.alb & 1 & 0 & Solution & 30.10 & 7 &  0.00 &  0.00\\
instance n=50 227.alb & 1 & 0 & Optimal & 22.91 & 6 &  0.00 &  0.00\\
instance n=50 228.alb & 1 & 0 & Optimal &  9.93 & 6 &  0.00 &  0.00\\
instance n=50 229.alb & 1 & 0 & Optimal &  9.24 & 6 &  0.00 &  0.00\\
instance n=50 23.alb & 1 & 0 & Solution & 30.12 & 7 &  0.00 &  0.00\\
instance n=50 230.alb & 1 & 0 & Solution & 30.10 & 7 &  0.00 &  0.00\\
instance n=50 231.alb & 1 & 0 & Solution & 30.09 & 7 &  0.00 &  0.00\\
instance n=50 232.alb & 1 & 0 & Solution & 30.09 & 8 &  0.00 &  0.00\\
instance n=50 233.alb & 1 & 0 & Optimal &  4.03 & 6 &  0.00 &  0.00\\
instance n=50 234.alb & 1 & 0 & Solution & 30.10 & 8 &  0.00 &  0.00\\
instance n=50 235.alb & 1 & 0 & Solution & 30.09 & 7 &  0.00 &  0.00\\
instance n=50 236.alb & 1 & 0 & Solution & 30.10 & 8 &  0.00 &  0.00\\
instance n=50 237.alb & 1 & 0 & Solution & 30.10 & 8 &  0.00 &  0.00\\
instance n=50 238.alb & 1 & 0 & Solution & 30.09 & 7 &  0.00 &  0.00\\
instance n=50 239.alb & 1 & 0 & Solution & 30.10 & 7 &  0.00 &  0.00\\
instance n=50 24.alb & 1 & 0 & Solution & 30.10 & 7 &  0.00 &  0.00\\
instance n=50 240.alb & 1 & 0 & Solution & 30.10 & 7 &  0.00 &  0.00\\
instance n=50 241.alb & 1 & 0 & Solution & 30.10 & 7 &  0.00 &  0.00\\
instance n=50 242.alb & 1 & 0 & Solution & 30.11 & 8 &  0.00 &  0.00\\
instance n=50 243.alb & 1 & 0 & Solution & 30.10 & 7 &  0.00 &  0.00\\
instance n=50 244.alb & 1 & 0 & Optimal & 14.63 & 7 &  0.00 &  0.00\\
instance n=50 245.alb & 1 & 0 & Solution & 30.10 & 7 &  0.00 &  0.00\\
instance n=50 246.alb & 1 & 0 & Solution & 30.10 & 8 &  0.00 &  0.00\\
instance n=50 247.alb & 1 & 0 & Solution & 30.11 & 7 &  0.00 &  0.00\\
instance n=50 248.alb & 1 & 0 & Solution & 30.09 & 7 &  0.00 &  0.00\\
instance n=50 249.alb & 1 & 0 & Solution & 30.10 & 7 &  0.00 &  0.00\\
instance n=50 25.alb & 1 & 0 & Solution & 30.11 & 6 &  0.00 &  0.00\\
instance n=50 250.alb & 1 & 0 & Solution & 30.12 & 7 &  0.00 &  0.00\\
instance n=50 251.alb & 1 & 0 & Solution & 30.11 & 29 &  0.00 &  0.00\\
instance n=50 252.alb & 1 & 0 & Solution & 30.10 & 35 &  0.00 &  0.00\\
instance n=50 253.alb & 1 & 0 & Solution & 30.13 & 31 &  0.00 &  0.00\\
instance n=50 254.alb & 1 & 0 & Solution & 30.10 & 33 &  0.00 &  0.00\\
instance n=50 255.alb & 1 & 0 & Solution & 30.10 & 32 &  0.00 &  0.00\\
instance n=50 256.alb & 1 & 0 & Solution & 30.10 & 32 &  0.00 &  0.00\\
instance n=50 257.alb & 1 & 0 & Solution & 30.10 & 35 &  0.00 &  0.00\\
instance n=50 258.alb & 1 & 0 & Solution & 30.09 & 30 &  0.00 &  0.00\\
instance n=50 259.alb & 1 & 0 & Solution & 30.10 & 32 &  0.00 &  0.00\\
instance n=50 26.alb & 1 & 0 & Solution & 30.09 & 30 &  0.00 &  0.00\\
instance n=50 260.alb & 1 & 0 & Solution & 30.09 & 30 &  0.00 &  0.00\\
instance n=50 261.alb & 1 & 0 & Solution & 30.14 & 30 &  0.00 &  0.00\\
instance n=50 262.alb & 1 & 0 & Solution & 30.09 & 31 &  0.00 &  0.00\\
instance n=50 263.alb & 1 & 0 & Solution & 30.11 & 31 &  0.00 &  0.00\\
instance n=50 264.alb & 1 & 0 & Solution & 30.10 & 31 &  0.00 &  0.00\\
instance n=50 265.alb & 1 & 0 & Solution & 30.09 & 30 &  0.00 &  0.00\\
instance n=50 266.alb & 1 & 0 & Solution & 30.09 & 31 &  0.00 &  0.00\\
instance n=50 267.alb & 1 & 0 & Solution & 30.09 & 30 &  0.00 &  0.00\\
instance n=50 268.alb & 1 & 0 & Solution & 30.10 & 31 &  0.00 &  0.00\\
instance n=50 269.alb & 1 & 0 & Solution & 30.10 & 29 &  0.00 &  0.00\\
instance n=50 27.alb & 1 & 0 & Solution & 30.10 & 35 &  0.00 &  0.00\\
instance n=50 270.alb & 1 & 0 & Solution & 30.10 & 29 &  0.00 &  0.00\\
instance n=50 271.alb & 1 & 0 & Solution & 30.10 & 33 &  0.00 &  0.00\\
instance n=50 272.alb & 1 & 0 & Solution & 30.10 & 30 &  0.00 &  0.00\\
instance n=50 273.alb & 1 & 0 & Solution & 30.09 & 30 &  0.00 &  0.00\\
instance n=50 274.alb & 1 & 0 & Solution & 30.10 & 32 &  0.00 &  0.00\\
instance n=50 275.alb & 1 & 0 & Solution & 30.09 & 29 &  0.00 &  0.00\\
instance n=50 276.alb & 1 & 0 & Solution & 30.11 & 13 &  0.00 &  0.00\\
instance n=50 277.alb & 1 & 0 & Solution & 30.11 & 13 &  0.00 &  0.00\\
instance n=50 278.alb & 1 & 0 & Solution & 30.09 & 13 &  0.00 &  0.00\\
instance n=50 279.alb & 1 & 0 & Solution & 30.16 & 11 &  0.00 &  0.00\\
instance n=50 28.alb & 1 & 0 & Solution & 30.12 & 34 &  0.00 &  0.00\\
instance n=50 280.alb & 1 & 0 & Solution & 30.10 & 13 &  0.00 &  0.00\\
instance n=50 281.alb & 1 & 0 & Solution & 30.11 & 11 &  0.00 &  0.00\\
instance n=50 282.alb & 1 & 0 & Solution & 30.11 & 12 &  0.00 &  0.00\\
instance n=50 283.alb & 1 & 0 & Solution & 30.11 & 13 &  0.00 &  0.00\\
instance n=50 284.alb & 1 & 0 & Solution & 30.10 & 11 &  0.00 &  0.00\\
instance n=50 285.alb & 1 & 0 & Solution & 30.10 & 14 &  0.00 &  0.00\\
instance n=50 286.alb & 1 & 0 & Solution & 30.11 & 12 &  0.00 &  0.00\\
instance n=50 287.alb & 1 & 0 & Solution & 30.12 & 13 &  0.00 &  0.00\\
instance n=50 288.alb & 1 & 0 & Solution & 30.13 & 11 &  0.00 &  0.00\\
instance n=50 289.alb & 1 & 0 & Solution & 30.11 & 12 &  0.00 &  0.00\\
instance n=50 29.alb & 1 & 0 & Solution & 30.10 & 32 &  0.00 &  0.00\\
instance n=50 290.alb & 1 & 0 & Solution & 30.10 & 14 &  0.00 &  0.00\\
instance n=50 291.alb & 1 & 0 & Solution & 30.15 & 12 &  0.00 &  0.00\\
instance n=50 292.alb & 1 & 0 & Solution & 30.12 & 13 &  0.00 &  0.00\\
instance n=50 293.alb & 1 & 0 & Solution & 30.11 & 12 &  0.00 &  0.00\\
instance n=50 294.alb & 1 & 0 & Solution & 30.11 & 13 &  0.00 &  0.00\\
instance n=50 295.alb & 1 & 0 & Solution & 30.10 & 17 &  0.00 &  0.00\\
instance n=50 296.alb & 1 & 0 & Solution & 30.12 & 13 &  0.00 &  0.00\\
instance n=50 297.alb & 1 & 0 & Solution & 30.12 & 13 &  0.00 &  0.00\\
instance n=50 298.alb & 1 & 0 & Solution & 30.13 & 11 &  0.00 &  0.00\\
instance n=50 299.alb & 1 & 0 & Solution & 30.10 & 12 &  0.00 &  0.00\\
instance n=50 3.alb & 1 & 0 & Solution & 30.10 & 8 &  0.00 &  0.00\\
instance n=50 30.alb & 1 & 0 & Solution & 30.11 & 30 &  0.00 &  0.00\\
instance n=50 300.alb & 1 & 0 & Solution & 30.10 & 12 &  0.00 &  0.00\\
instance n=50 301.alb & 1 & 0 & Solution & 30.12 & 7 &  0.00 &  0.00\\
instance n=50 302.alb & 1 & 0 & Solution & 30.11 & 7 &  0.00 &  0.00\\
instance n=50 303.alb & 1 & 0 & Solution & 30.13 & 8 &  0.00 &  0.00\\
instance n=50 304.alb & 1 & 0 & Solution & 30.13 & 7 &  0.00 &  0.00\\
instance n=50 305.alb & 1 & 0 & Solution & 30.11 & 8 &  0.00 &  0.00\\
instance n=50 306.alb & 1 & 0 & Solution & 30.11 & 7 &  0.00 &  0.00\\
instance n=50 307.alb & 1 & 0 & Solution & 30.10 & 7 &  0.00 &  0.00\\
instance n=50 308.alb & 1 & 0 & Solution & 30.11 & 8 &  0.00 &  0.00\\
instance n=50 309.alb & 1 & 0 & Solution & 30.12 & 8 &  0.00 &  0.00\\
instance n=50 31.alb & 1 & 0 & Solution & 30.12 & 31 &  0.00 &  0.00\\
instance n=50 310.alb & 1 & 0 & Solution & 30.12 & 8 &  0.00 &  0.00\\
instance n=50 311.alb & 1 & 0 & Solution & 30.11 & 8 &  0.00 &  0.00\\
instance n=50 312.alb & 1 & 0 & Solution & 30.10 & 7 &  0.00 &  0.00\\
instance n=50 313.alb & 1 & 0 & Solution & 30.12 & 8 &  0.00 &  0.00\\
instance n=50 314.alb & 1 & 0 & Solution & 30.11 & 7 &  0.00 &  0.00\\
instance n=50 315.alb & 1 & 0 & Solution & 30.11 & 8 &  0.00 &  0.00\\
instance n=50 316.alb & 1 & 0 & Solution & 30.10 & 8 &  0.00 &  0.00\\
instance n=50 317.alb & 1 & 0 & Solution & 30.10 & 6 &  0.00 &  0.00\\
instance n=50 318.alb & 1 & 0 & Solution & 30.11 & 8 &  0.00 &  0.00\\
instance n=50 319.alb & 1 & 0 & Solution & 30.11 & 7 &  0.00 &  0.00\\
instance n=50 32.alb & 1 & 0 & Solution & 30.10 & 31 &  0.00 &  0.00\\
instance n=50 320.alb & 1 & 0 & Solution & 30.12 & 8 &  0.00 &  0.00\\
instance n=50 321.alb & 1 & 0 & Solution & 30.10 & 6 &  0.00 &  0.00\\
instance n=50 322.alb & 1 & 0 & Solution & 30.11 & 7 &  0.00 &  0.00\\
instance n=50 323.alb & 1 & 0 & Solution & 30.10 & 7 &  0.00 &  0.00\\
instance n=50 324.alb & 1 & 0 & Solution & 30.11 & 7 &  0.00 &  0.00\\
instance n=50 325.alb & 1 & 0 & Solution & 30.11 & 7 &  0.00 &  0.00\\
instance n=50 326.alb & 1 & 0 & Solution & 30.09 & 36 &  0.00 &  0.00\\
instance n=50 327.alb & 1 & 0 & Solution & 30.11 & 31 &  0.00 &  0.00\\
instance n=50 328.alb & 1 & 0 & Solution & 30.11 & 34 &  0.00 &  0.00\\
instance n=50 329.alb & 1 & 0 & Solution & 30.10 & 30 &  0.00 &  0.00\\
instance n=50 33.alb & 1 & 0 & Solution & 30.10 & 28 &  0.00 &  0.00\\
instance n=50 330.alb & 1 & 0 & Solution & 30.10 & 33 &  0.00 &  0.00\\
instance n=50 331.alb & 1 & 0 & Solution & 30.10 & 36 &  0.00 &  0.00\\
instance n=50 332.alb & 1 & 0 & Solution & 30.12 & 30 &  0.00 &  0.00\\
instance n=50 333.alb & 1 & 0 & Solution & 30.10 & 32 &  0.00 &  0.00\\
instance n=50 334.alb & 1 & 0 & Solution & 30.10 & 32 &  0.00 &  0.00\\
instance n=50 335.alb & 1 & 0 & Solution & 30.10 & 33 &  0.00 &  0.00\\
instance n=50 336.alb & 1 & 0 & Solution & 30.10 & 31 &  0.00 &  0.00\\
instance n=50 337.alb & 1 & 0 & Solution & 30.12 & 31 &  0.00 &  0.00\\
instance n=50 338.alb & 1 & 0 & Solution & 30.10 & 34 &  0.00 &  0.00\\
instance n=50 339.alb & 1 & 0 & Solution & 30.10 & 32 &  0.00 &  0.00\\
instance n=50 34.alb & 1 & 0 & Solution & 30.11 & 32 &  0.00 &  0.00\\
instance n=50 340.alb & 1 & 0 & Solution & 30.10 & 33 &  0.00 &  0.00\\
instance n=50 341.alb & 1 & 0 & Solution & 30.10 & 33 &  0.00 &  0.00\\
instance n=50 342.alb & 1 & 0 & Solution & 30.10 & 33 &  0.00 &  0.00\\
instance n=50 343.alb & 1 & 0 & Solution & 30.12 & 31 &  0.00 &  0.00\\
instance n=50 344.alb & 1 & 0 & Solution & 30.10 & 33 &  0.00 &  0.00\\
instance n=50 345.alb & 1 & 0 & Solution & 30.12 & 35 &  0.00 &  0.00\\
instance n=50 346.alb & 1 & 0 & Solution & 30.11 & 30 &  0.00 &  0.00\\
instance n=50 347.alb & 1 & 0 & Solution & 30.13 & 33 &  0.00 &  0.00\\
instance n=50 348.alb & 1 & 0 & Solution & 30.10 & 33 &  0.00 &  0.00\\
instance n=50 349.alb & 1 & 0 & Solution & 30.18 & 33 &  0.00 &  0.00\\
instance n=50 35.alb & 1 & 0 & Solution & 30.10 & 33 &  0.00 &  0.00\\
instance n=50 350.alb & 1 & 0 & Solution & 30.10 & 28 &  0.00 &  0.00\\
instance n=50 351.alb & 1 & 0 & Solution & 30.09 & 12 &  0.00 &  0.00\\
instance n=50 352.alb & 1 & 0 & Solution & 30.11 & 11 &  0.00 &  0.00\\
instance n=50 353.alb & 1 & 0 & Solution & 30.11 & 14 &  0.00 &  0.00\\
instance n=50 354.alb & 1 & 0 & Solution & 30.11 & 14 &  0.00 &  0.00\\
instance n=50 355.alb & 1 & 0 & Solution & 30.11 & 11 &  0.00 &  0.00\\
instance n=50 356.alb & 1 & 0 & Solution & 30.12 & 16 &  0.00 &  0.00\\
instance n=50 357.alb & 1 & 0 & Solution & 30.18 & 13 &  0.00 &  0.00\\
instance n=50 358.alb & 1 & 0 & Solution & 30.11 & 11 &  0.00 &  0.00\\
instance n=50 359.alb & 1 & 0 & Solution & 30.10 & 10 &  0.00 &  0.00\\
instance n=50 36.alb & 1 & 0 & Solution & 30.10 & 35 &  0.00 &  0.00\\
instance n=50 360.alb & 1 & 0 & Solution & 30.13 & 13 &  0.00 &  0.00\\
instance n=50 361.alb & 1 & 0 & Solution & 30.15 & 12 &  0.00 &  0.00\\
instance n=50 362.alb & 1 & 0 & Solution & 30.10 & 11 &  0.00 &  0.00\\
instance n=50 363.alb & 1 & 0 & Solution & 30.10 & 12 &  0.00 &  0.00\\
instance n=50 364.alb & 1 & 0 & Solution & 30.11 & 13 &  0.00 &  0.00\\
instance n=50 365.alb & 1 & 0 & Solution & 30.09 & 11 &  0.00 &  0.00\\
instance n=50 366.alb & 1 & 0 & Solution & 30.10 & 14 &  0.00 &  0.00\\
instance n=50 367.alb & 1 & 0 & Solution & 30.11 & 12 &  0.00 &  0.00\\
instance n=50 368.alb & 1 & 0 & Solution & 30.11 & 12 &  0.00 &  0.00\\
instance n=50 369.alb & 1 & 0 & Solution & 30.11 & 13 &  0.00 &  0.00\\
instance n=50 37.alb & 1 & 0 & Solution & 30.09 & 36 &  0.00 &  0.00\\
instance n=50 370.alb & 1 & 0 & Solution & 30.11 & 12 &  0.00 &  0.00\\
instance n=50 371.alb & 1 & 0 & Solution & 30.13 & 12 &  0.00 &  0.00\\
instance n=50 372.alb & 1 & 0 & Solution & 30.10 & 11 &  0.00 &  0.00\\
instance n=50 373.alb & 1 & 0 & Solution & 30.09 & 13 &  0.00 &  0.00\\
instance n=50 374.alb & 1 & 0 & Solution & 30.13 & 11 &  0.00 &  0.00\\
instance n=50 375.alb & 1 & 0 & Solution & 30.10 & 14 &  0.00 &  0.00\\
instance n=50 376.alb & 1 & 0 & Solution & 30.12 & 7 &  0.00 &  0.00\\
instance n=50 377.alb & 1 & 0 & Solution & 30.10 & 7 &  0.00 &  0.00\\
instance n=50 378.alb & 1 & 0 & Solution & 30.10 & 8 &  0.00 &  0.00\\
instance n=50 379.alb & 1 & 0 & Solution & 30.11 & 7 &  0.00 &  0.00\\
instance n=50 38.alb & 1 & 0 & Solution & 30.10 & 35 &  0.00 &  0.00\\
instance n=50 380.alb & 1 & 0 & Solution & 30.12 & 7 &  0.00 &  0.00\\
instance n=50 381.alb & 1 & 0 & Solution & 30.13 & 8 &  0.00 &  0.00\\
instance n=50 382.alb & 1 & 0 & Solution & 30.08 & 6 &  0.00 &  0.00\\
instance n=50 383.alb & 1 & 0 & Solution & 30.11 & 7 &  0.00 &  0.00\\
instance n=50 384.alb & 1 & 0 & Solution & 30.09 & 9 &  0.00 &  0.00\\
instance n=50 385.alb & 1 & 0 & Solution & 30.09 & 7 &  0.00 &  0.00\\
instance n=50 386.alb & 1 & 0 & Solution & 30.10 & 7 &  0.00 &  0.00\\
instance n=50 387.alb & 1 & 0 & Solution & 30.17 & 8 &  0.00 &  0.00\\
instance n=50 388.alb & 1 & 0 & Solution & 30.11 & 7 &  0.00 &  0.00\\
instance n=50 389.alb & 1 & 0 & Solution & 30.10 & 8 &  0.00 &  0.00\\
instance n=50 39.alb & 1 & 0 & Solution & 30.11 & 35 &  0.00 &  0.00\\
instance n=50 390.alb & 1 & 0 & Solution & 30.10 & 8 &  0.00 &  0.00\\
instance n=50 391.alb & 1 & 0 & Solution & 30.10 & 7 &  0.00 &  0.00\\
instance n=50 392.alb & 1 & 0 & Solution & 30.11 & 8 &  0.00 &  0.00\\
instance n=50 393.alb & 1 & 0 & Solution & 30.10 & 7 &  0.00 &  0.00\\
instance n=50 394.alb & 1 & 0 & Solution & 30.10 & 8 &  0.00 &  0.00\\
instance n=50 395.alb & 1 & 0 & Solution & 30.09 & 7 &  0.00 &  0.00\\
instance n=50 396.alb & 1 & 0 & Solution & 30.10 & 8 &  0.00 &  0.00\\
instance n=50 397.alb & 1 & 0 & Solution & 30.09 & 7 &  0.00 &  0.00\\
instance n=50 398.alb & 1 & 0 & Solution & 30.10 & 7 &  0.00 &  0.00\\
instance n=50 399.alb & 1 & 0 & Solution & 30.11 & 8 &  0.00 &  0.00\\
instance n=50 4.alb & 1 & 0 & Solution & 30.10 & 7 &  0.00 &  0.00\\
instance n=50 40.alb & 1 & 0 & Solution & 30.11 & 32 &  0.00 &  0.00\\
instance n=50 400.alb & 1 & 0 & Solution & 30.10 & 8 &  0.00 &  0.00\\
instance n=50 401.alb & 1 & 0 & Solution & 30.09 & 31 &  0.00 &  0.00\\
instance n=50 402.alb & 1 & 0 & Solution & 30.10 & 30 &  0.00 &  0.00\\
instance n=50 403.alb & 1 & 0 & Solution & 30.09 & 36 &  0.00 &  0.00\\
instance n=50 404.alb & 1 & 0 & Solution & 30.09 & 32 &  0.00 &  0.00\\
instance n=50 405.alb & 1 & 0 & Solution & 30.11 & 29 &  0.00 &  0.00\\
instance n=50 406.alb & 1 & 0 & Solution & 30.09 & 36 &  0.00 &  0.00\\
instance n=50 407.alb & 1 & 0 & Solution & 30.10 & 30 &  0.00 &  0.00\\
instance n=50 408.alb & 1 & 0 & Solution & 30.14 & 28 &  0.00 &  0.00\\
instance n=50 409.alb & 1 & 0 & Solution & 30.08 & 34 &  0.00 &  0.00\\
instance n=50 41.alb & 1 & 0 & Solution & 30.10 & 32 &  0.00 &  0.00\\
instance n=50 410.alb & 1 & 0 & Solution & 30.08 & 30 &  0.00 &  0.00\\
instance n=50 411.alb & 1 & 0 & Solution & 30.10 & 32 &  0.00 &  0.00\\
instance n=50 412.alb & 1 & 0 & Solution & 30.09 & 29 &  0.00 &  0.00\\
instance n=50 413.alb & 1 & 0 & Solution & 30.10 & 32 &  0.00 &  0.00\\
instance n=50 414.alb & 1 & 0 & Solution & 30.08 & 28 &  0.00 &  0.00\\
instance n=50 415.alb & 1 & 0 & Solution & 30.10 & 32 &  0.00 &  0.00\\
instance n=50 416.alb & 1 & 0 & Solution & 30.09 & 29 &  0.00 &  0.00\\
instance n=50 417.alb & 1 & 0 & Solution & 30.09 & 32 &  0.00 &  0.00\\
instance n=50 418.alb & 1 & 0 & Solution & 30.09 & 29 &  0.00 &  0.00\\
instance n=50 419.alb & 1 & 0 & Solution & 30.10 & 34 &  0.00 &  0.00\\
instance n=50 42.alb & 1 & 0 & Solution & 30.11 & 31 &  0.00 &  0.00\\
instance n=50 420.alb & 1 & 0 & Solution & 30.13 & 30 &  0.00 &  0.00\\
instance n=50 421.alb & 1 & 0 & Solution & 30.09 & 35 &  0.00 &  0.00\\
instance n=50 422.alb & 1 & 0 & Solution & 30.09 & 31 &  0.00 &  0.00\\
instance n=50 423.alb & 1 & 0 & Solution & 30.09 & 31 &  0.00 &  0.00\\
instance n=50 424.alb & 1 & 0 & Solution & 30.09 & 30 &  0.00 &  0.00\\
instance n=50 425.alb & 1 & 0 & Solution & 30.10 & 35 &  0.00 &  0.00\\
instance n=50 426.alb & 1 & 0 & Solution & 30.19 & 12 &  0.00 &  0.00\\
instance n=50 427.alb & 1 & 0 & Solution & 30.10 & 12 &  0.00 &  0.00\\
instance n=50 428.alb & 1 & 0 & Solution & 30.10 & 13 &  0.00 &  0.00\\
instance n=50 429.alb & 1 & 0 & Solution & 30.11 & 11 &  0.00 &  0.00\\
instance n=50 43.alb & 1 & 0 & Solution & 30.10 & 31 &  0.00 &  0.00\\
instance n=50 430.alb & 1 & 0 & Solution & 30.10 & 15 &  0.00 &  0.00\\
instance n=50 431.alb & 1 & 0 & Solution & 30.10 & 11 &  0.00 &  0.00\\
instance n=50 432.alb & 1 & 0 & Solution & 30.10 & 13 &  0.00 &  0.00\\
instance n=50 433.alb & 1 & 0 & Solution & 30.10 & 12 &  0.00 &  0.00\\
instance n=50 434.alb & 1 & 0 & Solution & 30.14 & 11 &  0.00 &  0.00\\
instance n=50 435.alb & 1 & 0 & Solution & 30.10 & 11 &  0.00 &  0.00\\
instance n=50 436.alb & 1 & 0 & Solution & 30.09 & 11 &  0.00 &  0.00\\
instance n=50 437.alb & 1 & 0 & Solution & 30.11 & 13 &  0.00 &  0.00\\
instance n=50 438.alb & 1 & 0 & Solution & 30.11 & 11 &  0.00 &  0.00\\
instance n=50 439.alb & 1 & 0 & Solution & 30.13 & 13 &  0.00 &  0.00\\
instance n=50 44.alb & 1 & 0 & Solution & 30.11 & 31 &  0.00 &  0.00\\
instance n=50 440.alb & 1 & 0 & Solution & 30.10 & 13 &  0.00 &  0.00\\
instance n=50 441.alb & 1 & 0 & Solution & 30.10 & 11 &  0.00 &  0.00\\
instance n=50 442.alb & 1 & 0 & Solution & 30.10 & 13 &  0.00 &  0.00\\
instance n=50 443.alb & 1 & 0 & Solution & 30.11 & 12 &  0.00 &  0.00\\
instance n=50 444.alb & 1 & 0 & Solution & 30.11 & 12 &  0.00 &  0.00\\
instance n=50 445.alb & 1 & 0 & Solution & 30.13 & 12 &  0.00 &  0.00\\
instance n=50 446.alb & 1 & 0 & Solution & 30.11 & 13 &  0.00 &  0.00\\
instance n=50 447.alb & 1 & 0 & Solution & 30.10 & 14 &  0.00 &  0.00\\
instance n=50 448.alb & 1 & 0 & Solution & 30.10 & 13 &  0.00 &  0.00\\
instance n=50 449.alb & 1 & 0 & Solution & 30.10 & 11 &  0.00 &  0.00\\
instance n=50 45.alb & 1 & 0 & Solution & 30.11 & 28 &  0.00 &  0.00\\
instance n=50 450.alb & 1 & 0 & Solution & 30.13 & 11 &  0.00 &  0.00\\
instance n=50 451.alb & 1 & 0 & Optimal &  3.72 & 8 &  0.00 &  0.00\\
instance n=50 452.alb & 1 & 0 & Optimal &  2.81 & 8 &  0.00 &  0.00\\
instance n=50 453.alb & 1 & 0 & Optimal &  2.55 & 7 &  0.00 &  0.00\\
instance n=50 454.alb & 1 & 0 & Optimal &  3.73 & 8 &  0.00 &  0.00\\
instance n=50 455.alb & 1 & 0 & Optimal &  2.59 & 6 &  0.00 &  0.00\\
instance n=50 456.alb & 1 & 0 & Optimal &  2.85 & 8 &  0.00 &  0.00\\
instance n=50 457.alb & 1 & 0 & Optimal &  2.71 & 8 &  0.00 &  0.00\\
instance n=50 458.alb & 1 & 0 & Optimal &  2.65 & 7 &  0.00 &  0.00\\
instance n=50 459.alb & 1 & 0 & Optimal &  2.54 & 7 &  0.00 &  0.00\\
instance n=50 46.alb & 1 & 0 & Solution & 30.09 & 33 &  0.00 &  0.00\\
instance n=50 460.alb & 1 & 0 & Optimal &  3.10 & 7 &  0.00 &  0.00\\
instance n=50 461.alb & 1 & 0 & Optimal &  2.66 & 6 &  0.00 &  0.00\\
instance n=50 462.alb & 1 & 0 & Optimal &  3.26 & 7 &  0.00 &  0.00\\
instance n=50 463.alb & 1 & 0 & Optimal &  2.83 & 8 &  0.00 &  0.00\\
instance n=50 464.alb & 1 & 0 & Optimal &  2.73 & 6 &  0.00 &  0.00\\
instance n=50 465.alb & 1 & 0 & Optimal &  2.67 & 8 &  0.00 &  0.00\\
instance n=50 466.alb & 1 & 0 & Optimal &  2.78 & 7 &  0.00 &  0.00\\
instance n=50 467.alb & 1 & 0 & Optimal &  5.73 & 9 &  0.00 &  0.00\\
instance n=50 468.alb & 1 & 0 & Optimal &  2.68 & 7 &  0.00 &  0.00\\
instance n=50 469.alb & 1 & 0 & Optimal &  2.67 & 8 &  0.00 &  0.00\\
instance n=50 47.alb & 1 & 0 & Solution & 30.10 & 33 &  0.00 &  0.00\\
instance n=50 470.alb & 1 & 0 & Optimal &  3.42 & 8 &  0.00 &  0.00\\
instance n=50 471.alb & 1 & 0 & Optimal &  2.64 & 7 &  0.00 &  0.00\\
instance n=50 472.alb & 1 & 0 & Optimal &  2.81 & 8 &  0.00 &  0.00\\
instance n=50 473.alb & 1 & 0 & Optimal &  2.52 & 7 &  0.00 &  0.00\\
instance n=50 474.alb & 1 & 0 & Optimal &  2.86 & 7 &  0.00 &  0.00\\
instance n=50 475.alb & 1 & 0 & Optimal &  2.49 & 6 &  0.00 &  0.00\\
instance n=50 476.alb & 1 & 0 & Optimal &  9.05 & 28 &  0.00 &  0.00\\
instance n=50 477.alb & 1 & 0 & Solution & 30.09 & 29 &  0.00 &  0.00\\
instance n=50 478.alb & 1 & 0 & Solution & 30.09 & 32 &  0.00 &  0.00\\
instance n=50 479.alb & 1 & 0 & Optimal & 19.68 & 28 &  0.00 &  0.00\\
instance n=50 48.alb & 1 & 0 & Solution & 30.11 & 32 &  0.00 &  0.00\\
instance n=50 480.alb & 1 & 0 & Optimal &  7.19 & 34 &  0.00 &  0.00\\
instance n=50 481.alb & 1 & 0 & Solution & 30.09 & 28 &  0.00 &  0.00\\
instance n=50 482.alb & 1 & 0 & Optimal &  5.35 & 27 &  0.00 &  0.00\\
instance n=50 483.alb & 1 & 0 & Solution & 30.08 & 30 &  0.00 &  0.00\\
instance n=50 484.alb & 1 & 0 & Optimal & 15.58 & 32 &  0.00 &  0.00\\
instance n=50 485.alb & 1 & 0 & Solution & 30.09 & 31 &  0.00 &  0.00\\
instance n=50 486.alb & 1 & 0 & Optimal &  5.60 & 32 &  0.00 &  0.00\\
instance n=50 487.alb & 1 & 0 & Solution & 30.09 & 31 &  0.00 &  0.00\\
instance n=50 488.alb & 1 & 0 & Solution & 30.07 & 31 &  0.00 &  0.00\\
instance n=50 489.alb & 1 & 0 & Solution & 30.09 & 35 &  0.00 &  0.00\\
instance n=50 49.alb & 1 & 0 & Solution & 30.10 & 31 &  0.00 &  0.00\\
instance n=50 490.alb & 1 & 0 & Solution & 30.08 & 29 &  0.00 &  0.00\\
instance n=50 491.alb & 1 & 0 & Solution & 30.09 & 35 &  0.00 &  0.00\\
instance n=50 492.alb & 1 & 0 & Solution & 30.08 & 29 &  0.00 &  0.00\\
instance n=50 493.alb & 1 & 0 & Solution & 30.08 & 30 &  0.00 &  0.00\\
instance n=50 494.alb & 1 & 0 & Solution & 30.08 & 32 &  0.00 &  0.00\\
instance n=50 495.alb & 1 & 0 & Solution & 30.08 & 34 &  0.00 &  0.00\\
instance n=50 496.alb & 1 & 0 & Solution & 30.08 & 29 &  0.00 &  0.00\\
instance n=50 497.alb & 1 & 0 & Solution & 30.10 & 30 &  0.00 &  0.00\\
instance n=50 498.alb & 1 & 0 & Solution & 30.09 & 30 &  0.00 &  0.00\\
instance n=50 499.alb & 1 & 0 & Solution & 30.09 & 33 &  0.00 &  0.00\\
instance n=50 5.alb & 1 & 0 & Solution & 30.09 & 7 &  0.00 &  0.00\\
instance n=50 50.alb & 1 & 0 & Solution & 30.11 & 32 &  0.00 &  0.00\\
instance n=50 500.alb & 1 & 0 & Solution & 30.09 & 34 &  0.00 &  0.00\\
instance n=50 501.alb & 1 & 0 & Optimal &  3.41 & 12 &  0.00 &  0.00\\
instance n=50 502.alb & 1 & 0 & Optimal &  2.80 & 10 &  0.00 &  0.00\\
instance n=50 503.alb & 1 & 0 & Optimal &  4.21 & 13 &  0.00 &  0.00\\
instance n=50 504.alb & 1 & 0 & Optimal &  4.68 & 11 &  0.00 &  0.00\\
instance n=50 505.alb & 1 & 0 & Optimal &  2.81 & 12 &  0.00 &  0.00\\
instance n=50 506.alb & 1 & 0 & Optimal &  4.35 & 11 &  0.00 &  0.00\\
instance n=50 507.alb & 1 & 0 & Optimal &  3.36 & 13 &  0.00 &  0.00\\
instance n=50 508.alb & 1 & 0 & Optimal &  2.87 & 14 &  0.00 &  0.00\\
instance n=50 509.alb & 1 & 0 & Optimal &  3.42 & 13 &  0.00 &  0.00\\
instance n=50 51.alb & 1 & 0 & Solution & 30.10 & 12 &  0.00 &  0.00\\
instance n=50 510.alb & 1 & 0 & Optimal &  5.55 & 11 &  0.00 &  0.00\\
instance n=50 511.alb & 1 & 0 & Optimal &  2.96 & 13 &  0.00 &  0.00\\
instance n=50 512.alb & 1 & 0 & Optimal &  7.95 & 13 &  0.00 &  0.00\\
instance n=50 513.alb & 1 & 0 & Optimal &  3.73 & 12 &  0.00 &  0.00\\
instance n=50 514.alb & 1 & 0 & Optimal &  3.78 & 12 &  0.00 &  0.00\\
instance n=50 515.alb & 1 & 0 & Optimal &  5.82 & 11 &  0.00 &  0.00\\
instance n=50 516.alb & 1 & 0 & Optimal &  3.69 & 13 &  0.00 &  0.00\\
instance n=50 517.alb & 1 & 0 & Optimal &  6.75 & 14 &  0.00 &  0.00\\
instance n=50 518.alb & 1 & 0 & Optimal &  3.19 & 11 &  0.00 &  0.00\\
instance n=50 519.alb & 1 & 0 & Optimal &  4.24 & 12 &  0.00 &  0.00\\
instance n=50 52.alb & 1 & 0 & Solution & 30.11 & 11 &  0.00 &  0.00\\
instance n=50 520.alb & 1 & 0 & Optimal &  3.62 & 11 &  0.00 &  0.00\\
instance n=50 521.alb & 1 & 0 & Optimal &  3.39 & 10 &  0.00 &  0.00\\
instance n=50 522.alb & 1 & 0 & Optimal &  2.92 & 11 &  0.00 &  0.00\\
instance n=50 523.alb & 1 & 0 & Optimal &  3.05 & 11 &  0.00 &  0.00\\
instance n=50 524.alb & 1 & 0 & Optimal &  4.63 & 14 &  0.00 &  0.00\\
instance n=50 525.alb & 1 & 0 & Optimal &  4.64 & 11 &  0.00 &  0.00\\
instance n=50 53.alb & 1 & 0 & Solution & 30.11 & 13 &  0.00 &  0.00\\
instance n=50 54.alb & 1 & 0 & Solution & 30.09 & 12 &  0.00 &  0.00\\
instance n=50 55.alb & 1 & 0 & Solution & 30.13 & 14 &  0.00 &  0.00\\
instance n=50 56.alb & 1 & 0 & Solution & 30.12 & 12 &  0.00 &  0.00\\
instance n=50 57.alb & 1 & 0 & Solution & 30.12 & 15 &  0.00 &  0.00\\
instance n=50 58.alb & 1 & 0 & Solution & 30.10 & 11 &  0.00 &  0.00\\
instance n=50 59.alb & 1 & 0 & Solution & 30.11 & 11 &  0.00 &  0.00\\
instance n=50 6.alb & 1 & 0 & Solution & 30.11 & 6 &  0.00 &  0.00\\
instance n=50 60.alb & 1 & 0 & Solution & 30.11 & 13 &  0.00 &  0.00\\
instance n=50 61.alb & 1 & 0 & Solution & 30.11 & 13 &  0.00 &  0.00\\
instance n=50 62.alb & 1 & 0 & Solution & 30.11 & 14 &  0.00 &  0.00\\
instance n=50 63.alb & 1 & 0 & Solution & 30.12 & 12 &  0.00 &  0.00\\
instance n=50 64.alb & 1 & 0 & Solution & 30.12 & 13 &  0.00 &  0.00\\
instance n=50 65.alb & 1 & 0 & Solution & 30.12 & 12 &  0.00 &  0.00\\
instance n=50 66.alb & 1 & 0 & Solution & 30.12 & 14 &  0.00 &  0.00\\
instance n=50 67.alb & 1 & 0 & Solution & 30.12 & 13 &  0.00 &  0.00\\
instance n=50 68.alb & 1 & 0 & Solution & 30.12 & 12 &  0.00 &  0.00\\
instance n=50 69.alb & 1 & 0 & Solution & 30.11 & 13 &  0.00 &  0.00\\
instance n=50 7.alb & 1 & 0 & Solution & 30.10 & 7 &  0.00 &  0.00\\
instance n=50 70.alb & 1 & 0 & Solution & 30.11 & 10 &  0.00 &  0.00\\
instance n=50 71.alb & 1 & 0 & Solution & 30.11 & 15 &  0.00 &  0.00\\
instance n=50 72.alb & 1 & 0 & Solution & 30.12 & 11 &  0.00 &  0.00\\
instance n=50 73.alb & 1 & 0 & Solution & 30.12 & 12 &  0.00 &  0.00\\
instance n=50 74.alb & 1 & 0 & Solution & 30.12 & 12 &  0.00 &  0.00\\
instance n=50 75.alb & 1 & 0 & Solution & 30.13 & 12 &  0.00 &  0.00\\
instance n=50 76.alb & 1 & 0 & Solution & 30.10 & 7 &  0.00 &  0.00\\
instance n=50 77.alb & 1 & 0 & Solution & 30.08 & 7 &  0.00 &  0.00\\
instance n=50 78.alb & 1 & 0 & Solution & 30.10 & 7 &  0.00 &  0.00\\
instance n=50 79.alb & 1 & 0 & Solution & 30.12 & 8 &  0.00 &  0.00\\
instance n=50 8.alb & 1 & 0 & Solution & 30.12 & 7 &  0.00 &  0.00\\
instance n=50 80.alb & 1 & 0 & Solution & 30.10 & 7 &  0.00 &  0.00\\
instance n=50 81.alb & 1 & 0 & Solution & 30.10 & 7 &  0.00 &  0.00\\
instance n=50 82.alb & 1 & 0 & Solution & 30.11 & 6 &  0.00 &  0.00\\
instance n=50 83.alb & 1 & 0 & Solution & 30.11 & 8 &  0.00 &  0.00\\
instance n=50 84.alb & 1 & 0 & Solution & 30.11 & 7 &  0.00 &  0.00\\
instance n=50 85.alb & 1 & 0 & Solution & 30.10 & 8 &  0.00 &  0.00\\
instance n=50 86.alb & 1 & 0 & Solution & 30.15 & 7 &  0.00 &  0.00\\
instance n=50 87.alb & 1 & 0 & Solution & 30.10 & 8 &  0.00 &  0.00\\
instance n=50 88.alb & 1 & 0 & Solution & 30.10 & 8 &  0.00 &  0.00\\
instance n=50 89.alb & 1 & 0 & Solution & 30.09 & 7 &  0.00 &  0.00\\
instance n=50 9.alb & 1 & 0 & Solution & 30.12 & 9 &  0.00 &  0.00\\
instance n=50 90.alb & 1 & 0 & Solution & 30.11 & 8 &  0.00 &  0.00\\
instance n=50 91.alb & 1 & 0 & Solution & 30.11 & 7 &  0.00 &  0.00\\
instance n=50 92.alb & 1 & 0 & Solution & 30.14 & 7 &  0.00 &  0.00\\
instance n=50 93.alb & 1 & 0 & Solution & 30.10 & 7 &  0.00 &  0.00\\
instance n=50 94.alb & 1 & 0 & Solution & 30.10 & 7 &  0.00 &  0.00\\
instance n=50 95.alb & 1 & 0 & Solution & 30.10 & 7 &  0.00 &  0.00\\
instance n=50 96.alb & 1 & 0 & Solution & 30.11 & 7 &  0.00 &  0.00\\
instance n=50 97.alb & 1 & 0 & Solution & 30.10 & 7 &  0.00 &  0.00\\
instance n=50 98.alb & 1 & 0 & Solution & 30.16 & 8 &  0.00 &  0.00\\
instance n=50 99.alb & 1 & 0 & Solution & 30.10 & 7 &  0.00 &  0.00\\
\end{longtable}



\section{Results for MiniZinc/Chuffed}

\begin{longtable}{lrrlrrrr}
\caption{Results for SALBP-1 Problems (Chuffed) (1029 Instances)}\\\toprule
Name & \shortstack{Nr\\Jobs} & \shortstack{Nr\\Machines} & Status & Time & Makespan & Bound & \shortstack{Gap\\Percent}\\ \midrule
\endhead
\bottomrule
\endfoot
instance n=100 1.alb & 1 & 0 & Solution & 120.13 & 78 &  0.00 &  0.00\\
instance n=100 10.alb & 1 & 0 & Solution & 120.13 & 56 &  0.00 &  0.00\\
instance n=100 100.alb & 1 & 0 & Solution & 120.14 & 65 &  0.00 &  0.00\\
instance n=100 101.alb & 1 & 0 & Solution & 120.13 & 70 &  0.00 &  0.00\\
instance n=100 102.alb & 1 & 0 & Solution & 120.13 & 15 &  0.00 &  0.00\\
instance n=100 103.alb & 1 & 0 & Solution & 120.13 & 14 &  0.00 &  0.00\\
instance n=100 104.alb & 1 & 0 & Solution & 120.13 & 83 &  0.00 &  0.00\\
instance n=100 105.alb & 1 & 0 & Solution & 120.11 & 13 &  0.00 &  0.00\\
instance n=100 106.alb & 1 & 0 & Solution & 120.12 & 14 &  0.00 &  0.00\\
instance n=100 107.alb & 1 & 0 & Solution & 120.12 & 14 &  0.00 &  0.00\\
instance n=100 108.alb & 1 & 0 & Solution & 120.14 & 15 &  0.00 &  0.00\\
instance n=100 109.alb & 1 & 0 & Solution & 120.12 & 92 &  0.00 &  0.00\\
instance n=100 11.alb & 1 & 0 & Solution & 120.13 & 88 &  0.00 &  0.00\\
instance n=100 110.alb & 1 & 0 & Solution & 120.13 & 84 &  0.00 &  0.00\\
instance n=100 111.alb & 1 & 0 & Solution & 120.13 & 98 &  0.00 &  0.00\\
instance n=100 112.alb & 1 & 0 & Solution & 120.14 & 14 &  0.00 &  0.00\\
instance n=100 113.alb & 1 & 0 & Solution & 120.14 & 49 &  0.00 &  0.00\\
instance n=100 114.alb & 1 & 0 & Solution & 120.13 & 14 &  0.00 &  0.00\\
instance n=100 115.alb & 1 & 0 & Solution & 120.13 & 17 &  0.00 &  0.00\\
instance n=100 116.alb & 1 & 0 & Solution & 120.14 & 71 &  0.00 &  0.00\\
instance n=100 117.alb & 1 & 0 & Solution & 120.12 & 78 &  0.00 &  0.00\\
instance n=100 118.alb & 1 & 0 & Solution & 120.12 & 15 &  0.00 &  0.00\\
instance n=100 119.alb & 1 & 0 & Solution & 120.13 & 90 &  0.00 &  0.00\\
instance n=100 12.alb & 1 & 0 & Solution & 120.13 & 79 &  0.00 &  0.00\\
instance n=100 120.alb & 1 & 0 & Solution & 120.13 & 14 &  0.00 &  0.00\\
instance n=100 121.alb & 1 & 0 & Solution & 120.13 & 15 &  0.00 &  0.00\\
instance n=100 122.alb & 1 & 0 & Solution & 120.12 & 19 &  0.00 &  0.00\\
instance n=100 123.alb & 1 & 0 & Solution & 120.12 & 71 &  0.00 &  0.00\\
instance n=100 124.alb & 1 & 0 & Solution & 120.13 & 16 &  0.00 &  0.00\\
instance n=100 125.alb & 1 & 0 & Solution & 120.13 & 14 &  0.00 &  0.00\\
instance n=100 126.alb & 1 & 0 & Solution & 120.14 & 63 &  0.00 &  0.00\\
instance n=100 127.alb & 1 & 0 & Solution & 120.14 & 53 &  0.00 &  0.00\\
instance n=100 128.alb & 1 & 0 & Solution & 120.12 & 83 &  0.00 &  0.00\\
instance n=100 129.alb & 1 & 0 & Solution & 120.13 & 55 &  0.00 &  0.00\\
instance n=100 13.alb & 1 & 0 & Solution & 120.13 & 84 &  0.00 &  0.00\\
instance n=100 130.alb & 1 & 0 & Solution & 120.13 & 56 &  0.00 &  0.00\\
instance n=100 131.alb & 1 & 0 & Solution & 120.13 & 71 &  0.00 &  0.00\\
instance n=100 132.alb & 1 & 0 & Solution & 120.13 & 77 &  0.00 &  0.00\\
instance n=100 133.alb & 1 & 0 & Solution & 120.12 & 56 &  0.00 &  0.00\\
instance n=100 134.alb & 1 & 0 & Solution & 120.12 & 56 &  0.00 &  0.00\\
instance n=100 135.alb & 1 & 0 & Solution & 120.12 & 57 &  0.00 &  0.00\\
instance n=100 136.alb & 1 & 0 & Solution & 120.13 & 76 &  0.00 &  0.00\\
instance n=100 137.alb & 1 & 0 & Solution & 120.13 & 66 &  0.00 &  0.00\\
instance n=100 138.alb & 1 & 0 & Solution & 120.13 & 76 &  0.00 &  0.00\\
instance n=100 139.alb & 1 & 0 & Solution & 120.12 & 84 &  0.00 &  0.00\\
instance n=100 14.alb & 1 & 0 & Solution & 120.12 & 66 &  0.00 &  0.00\\
instance n=100 140.alb & 1 & 0 & Solution & 120.13 & 69 &  0.00 &  0.00\\
instance n=100 141.alb & 1 & 0 & Solution & 120.13 & 53 &  0.00 &  0.00\\
instance n=100 142.alb & 1 & 0 & Solution & 120.12 & 91 &  0.00 &  0.00\\
instance n=100 143.alb & 1 & 0 & Solution & 120.11 & 64 &  0.00 &  0.00\\
instance n=100 144.alb & 1 & 0 & Solution & 120.12 & 76 &  0.00 &  0.00\\
instance n=100 145.alb & 1 & 0 & Solution & 120.11 & 82 &  0.00 &  0.00\\
instance n=100 146.alb & 1 & 0 & Solution & 120.12 & 53 &  0.00 &  0.00\\
instance n=100 147.alb & 1 & 0 & Solution & 120.12 & 71 &  0.00 &  0.00\\
instance n=100 148.alb & 1 & 0 & Solution & 120.13 & 80 &  0.00 &  0.00\\
instance n=100 149.alb & 1 & 0 & Solution & 120.13 & 76 &  0.00 &  0.00\\
instance n=100 15.alb & 1 & 0 & Solution & 120.13 & 63 &  0.00 &  0.00\\
instance n=100 150.alb & 1 & 0 & Solution & 120.12 & 59 &  0.00 &  0.00\\
instance n=100 151.alb & 1 & 0 & Solution & 120.13 & 36 &  0.00 &  0.00\\
instance n=100 152.alb & 1 & 0 & Solution & 120.13 & 75 &  0.00 &  0.00\\
instance n=100 153.alb & 1 & 0 & Solution & 120.13 & 21 &  0.00 &  0.00\\
instance n=100 154.alb & 1 & 0 & Solution & 120.12 & 76 &  0.00 &  0.00\\
instance n=100 155.alb & 1 & 0 & Solution & 120.13 & 98 &  0.00 &  0.00\\
instance n=100 156.alb & 1 & 0 & Solution & 120.12 & 69 &  0.00 &  0.00\\
instance n=100 157.alb & 1 & 0 & Solution & 120.13 & 42 &  0.00 &  0.00\\
instance n=100 158.alb & 1 & 0 & Solution & 120.13 & 86 &  0.00 &  0.00\\
instance n=100 159.alb & 1 & 0 & Solution & 120.12 & 78 &  0.00 &  0.00\\
instance n=100 16.alb & 1 & 0 & Solution & 120.12 & 91 &  0.00 &  0.00\\
instance n=100 160.alb & 1 & 0 & Solution & 120.11 & 89 &  0.00 &  0.00\\
instance n=100 161.alb & 1 & 0 & Solution & 120.12 & 56 &  0.00 &  0.00\\
instance n=100 162.alb & 1 & 0 & Solution & 120.14 & 32 &  0.00 &  0.00\\
instance n=100 163.alb & 1 & 0 & Solution & 120.13 & 76 &  0.00 &  0.00\\
instance n=100 164.alb & 1 & 0 & Solution & 120.12 & 53 &  0.00 &  0.00\\
instance n=100 165.alb & 1 & 0 & Solution & 120.13 & 70 &  0.00 &  0.00\\
instance n=100 166.alb & 1 & 0 & Solution & 120.13 & 32 &  0.00 &  0.00\\
instance n=100 167.alb & 1 & 0 & Solution & 120.12 & 51 &  0.00 &  0.00\\
instance n=100 168.alb & 1 & 0 & Solution & 120.11 & 75 &  0.00 &  0.00\\
instance n=100 169.alb & 1 & 0 & Solution & 120.14 & 94 &  0.00 &  0.00\\
instance n=100 17.alb & 1 & 0 & Solution & 120.13 & 68 &  0.00 &  0.00\\
instance n=100 170.alb & 1 & 0 & Solution & 120.13 & 38 &  0.00 &  0.00\\
instance n=100 171.alb & 1 & 0 & Solution & 120.13 & 25 &  0.00 &  0.00\\
instance n=100 172.alb & 1 & 0 & Solution & 120.12 & 91 &  0.00 &  0.00\\
instance n=100 173.alb & 1 & 0 & Solution & 120.12 & 91 &  0.00 &  0.00\\
instance n=100 174.alb & 1 & 0 & Solution & 120.12 & 47 &  0.00 &  0.00\\
instance n=100 175.alb & 1 & 0 & Solution & 120.13 & 71 &  0.00 &  0.00\\
instance n=100 176.alb & 1 & 0 & Solution & 120.13 & 80 &  0.00 &  0.00\\
instance n=100 177.alb & 1 & 0 & Solution & 120.12 & 95 &  0.00 &  0.00\\
instance n=100 178.alb & 1 & 0 & Solution & 120.12 & 81 &  0.00 &  0.00\\
instance n=100 179.alb & 1 & 0 & Solution & 120.11 & 74 &  0.00 &  0.00\\
instance n=100 18.alb & 1 & 0 & Solution & 120.13 & 95 &  0.00 &  0.00\\
instance n=100 180.alb & 1 & 0 & Solution & 120.13 & 91 &  0.00 &  0.00\\
instance n=100 181.alb & 1 & 0 & Solution & 120.13 & 60 &  0.00 &  0.00\\
instance n=100 182.alb & 1 & 0 & Solution & 120.12 & 47 &  0.00 &  0.00\\
instance n=100 183.alb & 1 & 0 & Solution & 120.13 & 72 &  0.00 &  0.00\\
instance n=100 184.alb & 1 & 0 & Solution & 120.11 & 96 &  0.00 &  0.00\\
instance n=100 185.alb & 1 & 0 & Solution & 120.13 & 54 &  0.00 &  0.00\\
instance n=100 186.alb & 1 & 0 & Solution & 120.14 & 64 &  0.00 &  0.00\\
instance n=100 187.alb & 1 & 0 & Solution & 120.12 & 48 &  0.00 &  0.00\\
instance n=100 188.alb & 1 & 0 & Solution & 120.13 & 69 &  0.00 &  0.00\\
instance n=100 189.alb & 1 & 0 & Solution & 120.13 & 85 &  0.00 &  0.00\\
instance n=100 19.alb & 1 & 0 & Solution & 120.12 & 91 &  0.00 &  0.00\\
instance n=100 190.alb & 1 & 0 & Solution & 120.13 & 89 &  0.00 &  0.00\\
instance n=100 191.alb & 1 & 0 & Solution & 120.13 & 78 &  0.00 &  0.00\\
instance n=100 192.alb & 1 & 0 & Solution & 120.13 & 78 &  0.00 &  0.00\\
instance n=100 193.alb & 1 & 0 & Solution & 120.13 & 98 &  0.00 &  0.00\\
instance n=100 194.alb & 1 & 0 & Solution & 120.12 & 80 &  0.00 &  0.00\\
instance n=100 195.alb & 1 & 0 & Solution & 120.12 & 85 &  0.00 &  0.00\\
instance n=100 196.alb & 1 & 0 & Solution & 120.12 & 97 &  0.00 &  0.00\\
instance n=100 197.alb & 1 & 0 & Solution & 120.16 & 24 &  0.00 &  0.00\\
instance n=100 198.alb & 1 & 0 & Solution & 120.13 & 79 &  0.00 &  0.00\\
instance n=100 199.alb & 1 & 0 & Solution & 120.14 & 19 &  0.00 &  0.00\\
instance n=100 2.alb & 1 & 0 & Solution & 120.12 & 51 &  0.00 &  0.00\\
instance n=100 20.alb & 1 & 0 & Solution & 120.12 & 76 &  0.00 &  0.00\\
instance n=100 200.alb & 1 & 0 & Solution & 120.11 & 96 &  0.00 &  0.00\\
instance n=100 201.alb & 1 & 0 & Solution & 120.12 & 84 &  0.00 &  0.00\\
instance n=100 202.alb & 1 & 0 & Solution & 120.12 & 92 &  0.00 &  0.00\\
instance n=100 203.alb & 1 & 0 & Solution & 120.13 & 64 &  0.00 &  0.00\\
instance n=100 204.alb & 1 & 0 & Solution & 120.14 & 80 &  0.00 &  0.00\\
instance n=100 205.alb & 1 & 0 & Solution & 120.12 & 91 &  0.00 &  0.00\\
instance n=100 206.alb & 1 & 0 & Solution & 120.11 & 57 &  0.00 &  0.00\\
instance n=100 207.alb & 1 & 0 & Solution & 120.13 & 70 &  0.00 &  0.00\\
instance n=100 208.alb & 1 & 0 & Solution & 120.13 & 87 &  0.00 &  0.00\\
instance n=100 209.alb & 1 & 0 & Solution & 120.13 & 78 &  0.00 &  0.00\\
instance n=100 21.alb & 1 & 0 & Solution & 120.12 & 79 &  0.00 &  0.00\\
instance n=100 210.alb & 1 & 0 & Solution & 120.12 & 65 &  0.00 &  0.00\\
instance n=100 211.alb & 1 & 0 & Solution & 120.11 & 81 &  0.00 &  0.00\\
instance n=100 212.alb & 1 & 0 & Solution & 120.13 & 55 &  0.00 &  0.00\\
instance n=100 213.alb & 1 & 0 & Solution & 120.13 & 96 &  0.00 &  0.00\\
instance n=100 214.alb & 1 & 0 & Solution & 120.13 & 87 &  0.00 &  0.00\\
instance n=100 215.alb & 1 & 0 & Solution & 120.12 & 76 &  0.00 &  0.00\\
instance n=100 216.alb & 1 & 0 & Solution & 120.13 & 94 &  0.00 &  0.00\\
instance n=100 217.alb & 1 & 0 & Solution & 120.12 & 84 &  0.00 &  0.00\\
instance n=100 218.alb & 1 & 0 & Solution & 120.13 & 57 &  0.00 &  0.00\\
instance n=100 219.alb & 1 & 0 & Solution & 120.13 & 88 &  0.00 &  0.00\\
instance n=100 22.alb & 1 & 0 & Solution & 120.12 & 48 &  0.00 &  0.00\\
instance n=100 220.alb & 1 & 0 & Solution & 120.14 & 87 &  0.00 &  0.00\\
instance n=100 221.alb & 1 & 0 & Solution & 120.13 & 88 &  0.00 &  0.00\\
instance n=100 222.alb & 1 & 0 & Solution & 120.12 & 66 &  0.00 &  0.00\\
instance n=100 223.alb & 1 & 0 & Solution & 120.12 & 64 &  0.00 &  0.00\\
instance n=100 224.alb & 1 & 0 & Solution & 120.13 & 63 &  0.00 &  0.00\\
instance n=100 225.alb & 1 & 0 & Solution & 120.13 & 58 &  0.00 &  0.00\\
instance n=100 226.alb & 1 & 0 & Solution & 120.14 & 68 &  0.00 &  0.00\\
instance n=100 227.alb & 1 & 0 & Solution & 120.13 & 27 &  0.00 &  0.00\\
instance n=100 228.alb & 1 & 0 & Solution & 120.12 & 22 &  0.00 &  0.00\\
instance n=100 229.alb & 1 & 0 & Solution & 120.13 & 24 &  0.00 &  0.00\\
instance n=100 23.alb & 1 & 0 & Solution & 120.13 & 98 &  0.00 &  0.00\\
instance n=100 230.alb & 1 & 0 & Solution & 120.14 & 38 &  0.00 &  0.00\\
instance n=100 231.alb & 1 & 0 & Solution & 120.13 & 23 &  0.00 &  0.00\\
instance n=100 232.alb & 1 & 0 & Solution & 120.13 & 35 &  0.00 &  0.00\\
instance n=100 233.alb & 1 & 0 & Solution & 120.12 & 36 &  0.00 &  0.00\\
instance n=100 234.alb & 1 & 0 & Solution & 120.12 & 23 &  0.00 &  0.00\\
instance n=100 235.alb & 1 & 0 & Solution & 120.13 & 26 &  0.00 &  0.00\\
instance n=100 236.alb & 1 & 0 & Solution & 120.13 & 23 &  0.00 &  0.00\\
instance n=100 237.alb & 1 & 0 & Solution & 120.12 & 23 &  0.00 &  0.00\\
instance n=100 238.alb & 1 & 0 & Solution & 120.13 & 49 &  0.00 &  0.00\\
instance n=100 239.alb & 1 & 0 & Solution & 120.13 & 92 &  0.00 &  0.00\\
instance n=100 24.alb & 1 & 0 & Solution & 120.12 & 69 &  0.00 &  0.00\\
instance n=100 240.alb & 1 & 0 & Solution & 120.13 & 22 &  0.00 &  0.00\\
instance n=100 241.alb & 1 & 0 & Solution & 120.13 & 67 &  0.00 &  0.00\\
instance n=100 242.alb & 1 & 0 & Solution & 120.13 & 87 &  0.00 &  0.00\\
instance n=100 243.alb & 1 & 0 & Solution & 120.13 & 28 &  0.00 &  0.00\\
instance n=100 244.alb & 1 & 0 & Solution & 120.13 & 51 &  0.00 &  0.00\\
instance n=100 245.alb & 1 & 0 & Solution & 120.12 & 24 &  0.00 &  0.00\\
instance n=100 246.alb & 1 & 0 & Solution & 120.12 & 26 &  0.00 &  0.00\\
instance n=100 247.alb & 1 & 0 & Solution & 120.14 & 24 &  0.00 &  0.00\\
instance n=100 248.alb & 1 & 0 & Solution & 120.13 & 20 &  0.00 &  0.00\\
instance n=100 249.alb & 1 & 0 & Solution & 120.13 & 69 &  0.00 &  0.00\\
instance n=100 25.alb & 1 & 0 & Solution & 120.12 & 95 &  0.00 &  0.00\\
instance n=100 250.alb & 1 & 0 & Solution & 120.12 & 24 &  0.00 &  0.00\\
instance n=100 251.alb & 1 & 0 & Solution & 120.13 & 15 &  0.00 &  0.00\\
instance n=100 252.alb & 1 & 0 & Solution & 120.13 & 14 &  0.00 &  0.00\\
instance n=100 253.alb & 1 & 0 & Solution & 120.12 & 93 &  0.00 &  0.00\\
instance n=100 254.alb & 1 & 0 & Solution & 120.13 & 14 &  0.00 &  0.00\\
instance n=100 255.alb & 1 & 0 & Solution & 120.13 & 14 &  0.00 &  0.00\\
instance n=100 256.alb & 1 & 0 & Solution & 120.12 & 15 &  0.00 &  0.00\\
instance n=100 257.alb & 1 & 0 & Solution & 120.13 & 13 &  0.00 &  0.00\\
instance n=100 258.alb & 1 & 0 & Solution & 120.13 & 15 &  0.00 &  0.00\\
instance n=100 259.alb & 1 & 0 & Solution & 120.14 & 35 &  0.00 &  0.00\\
instance n=100 26.alb & 1 & 0 & Solution & 120.13 & 81 &  0.00 &  0.00\\
instance n=100 260.alb & 1 & 0 & Solution & 120.13 & 15 &  0.00 &  0.00\\
instance n=100 261.alb & 1 & 0 & Solution & 120.12 & 49 &  0.00 &  0.00\\
instance n=100 262.alb & 1 & 0 & Solution & 120.14 & 14 &  0.00 &  0.00\\
instance n=100 263.alb & 1 & 0 & Solution & 120.12 & 14 &  0.00 &  0.00\\
instance n=100 264.alb & 1 & 0 & Solution & 120.12 & 69 &  0.00 &  0.00\\
instance n=100 265.alb & 1 & 0 & Solution & 120.13 & 96 &  0.00 &  0.00\\
instance n=100 266.alb & 1 & 0 & Solution & 120.13 & 13 &  0.00 &  0.00\\
instance n=100 267.alb & 1 & 0 & Solution & 120.12 & 13 &  0.00 &  0.00\\
instance n=100 268.alb & 1 & 0 & Solution & 120.13 & 15 &  0.00 &  0.00\\
instance n=100 269.alb & 1 & 0 & Solution & 120.13 & 15 &  0.00 &  0.00\\
instance n=100 27.alb & 1 & 0 & Solution & 120.13 & 56 &  0.00 &  0.00\\
instance n=100 270.alb & 1 & 0 & Solution & 120.13 & 13 &  0.00 &  0.00\\
instance n=100 271.alb & 1 & 0 & Solution & 120.13 & 95 &  0.00 &  0.00\\
instance n=100 272.alb & 1 & 0 & Solution & 120.12 & 14 &  0.00 &  0.00\\
instance n=100 273.alb & 1 & 0 & Solution & 120.13 & 13 &  0.00 &  0.00\\
instance n=100 274.alb & 1 & 0 & Solution & 120.14 & 14 &  0.00 &  0.00\\
instance n=100 275.alb & 1 & 0 & Solution & 120.14 & 13 &  0.00 &  0.00\\
instance n=100 276.alb & 1 & 0 & Solution & 120.12 & 63 &  0.00 &  0.00\\
instance n=100 277.alb & 1 & 0 & Solution & 120.13 & 78 &  0.00 &  0.00\\
instance n=100 278.alb & 1 & 0 & Solution & 120.12 & 61 &  0.00 &  0.00\\
instance n=100 279.alb & 1 & 0 & Solution & 120.12 & 58 &  0.00 &  0.00\\
instance n=100 28.alb & 1 & 0 & Solution & 120.14 & 74 &  0.00 &  0.00\\
instance n=100 280.alb & 1 & 0 & Solution & 120.13 & 68 &  0.00 &  0.00\\
instance n=100 281.alb & 1 & 0 & Solution & 120.13 & 83 &  0.00 &  0.00\\
instance n=100 282.alb & 1 & 0 & Solution & 120.12 & 90 &  0.00 &  0.00\\
instance n=100 283.alb & 1 & 0 & Solution & 120.12 & 56 &  0.00 &  0.00\\
instance n=100 284.alb & 1 & 0 & Solution & 120.13 & 59 &  0.00 &  0.00\\
instance n=100 285.alb & 1 & 0 & Solution & 120.14 & 57 &  0.00 &  0.00\\
instance n=100 286.alb & 1 & 0 & Solution & 120.13 & 84 &  0.00 &  0.00\\
instance n=100 287.alb & 1 & 0 & Solution & 120.14 & 55 &  0.00 &  0.00\\
instance n=100 288.alb & 1 & 0 & Solution & 120.13 & 58 &  0.00 &  0.00\\
instance n=100 289.alb & 1 & 0 & Solution & 120.11 & 64 &  0.00 &  0.00\\
instance n=100 29.alb & 1 & 0 & Solution & 120.12 & 83 &  0.00 &  0.00\\
instance n=20 1.alb & 1 & 0 & Optimal &  0.51 & 3 &  0.00 &  0.00\\
instance n=20 10.alb & 1 & 0 & Optimal &  0.23 & 3 &  0.00 &  0.00\\
instance n=20 100.alb & 1 & 0 & Optimal &  0.26 & 11 &  0.00 &  0.00\\
instance n=20 101.alb & 1 & 0 & Optimal &  0.32 & 13 &  0.00 &  0.00\\
instance n=20 102.alb & 1 & 0 & Optimal &  0.26 & 13 &  0.00 &  0.00\\
instance n=20 103.alb & 1 & 0 & Optimal &  0.29 & 12 &  0.00 &  0.00\\
instance n=20 104.alb & 1 & 0 & Optimal &  0.25 & 11 &  0.00 &  0.00\\
instance n=20 105.alb & 1 & 0 & Optimal &  0.26 & 12 &  0.00 &  0.00\\
instance n=20 106.alb & 1 & 0 & Optimal &  0.25 & 10 &  0.00 &  0.00\\
instance n=20 107.alb & 1 & 0 & Optimal &  0.33 & 14 &  0.00 &  0.00\\
instance n=20 108.alb & 1 & 0 & Optimal &  0.29 & 15 &  0.00 &  0.00\\
instance n=20 109.alb & 1 & 0 & Optimal &  0.24 & 12 &  0.00 &  0.00\\
instance n=20 11.alb & 1 & 0 & Optimal &  0.23 & 3 &  0.00 &  0.00\\
instance n=20 110.alb & 1 & 0 & Optimal &  0.26 & 11 &  0.00 &  0.00\\
instance n=20 111.alb & 1 & 0 & Optimal &  0.25 & 13 &  0.00 &  0.00\\
instance n=20 112.alb & 1 & 0 & Optimal &  0.24 & 11 &  0.00 &  0.00\\
instance n=20 113.alb & 1 & 0 & Optimal &  0.26 & 12 &  0.00 &  0.00\\
instance n=20 114.alb & 1 & 0 & Optimal &  0.26 & 13 &  0.00 &  0.00\\
instance n=20 115.alb & 1 & 0 & Optimal &  0.24 & 11 &  0.00 &  0.00\\
instance n=20 116.alb & 1 & 0 & Optimal &  0.23 & 5 &  0.00 &  0.00\\
instance n=20 117.alb & 1 & 0 & Optimal &  0.24 & 5 &  0.00 &  0.00\\
instance n=20 118.alb & 1 & 0 & Optimal &  0.24 & 5 &  0.00 &  0.00\\
instance n=20 119.alb & 1 & 0 & Optimal &  0.22 & 6 &  0.00 &  0.00\\
instance n=20 12.alb & 1 & 0 & Optimal &  0.24 & 3 &  0.00 &  0.00\\
instance n=20 120.alb & 1 & 0 & Optimal &  0.24 & 6 &  0.00 &  0.00\\
instance n=20 121.alb & 1 & 0 & Optimal &  0.23 & 5 &  0.00 &  0.00\\
instance n=20 122.alb & 1 & 0 & Optimal &  0.24 & 6 &  0.00 &  0.00\\
instance n=20 123.alb & 1 & 0 & Optimal &  0.26 & 5 &  0.00 &  0.00\\
instance n=20 124.alb & 1 & 0 & Optimal &  0.25 & 5 &  0.00 &  0.00\\
instance n=20 125.alb & 1 & 0 & Optimal &  0.24 & 5 &  0.00 &  0.00\\
instance n=20 126.alb & 1 & 0 & Optimal &  0.23 & 5 &  0.00 &  0.00\\
instance n=20 127.alb & 1 & 0 & Optimal &  0.25 & 4 &  0.00 &  0.00\\
instance n=20 128.alb & 1 & 0 & Optimal &  0.24 & 5 &  0.00 &  0.00\\
instance n=20 129.alb & 1 & 0 & Optimal &  0.24 & 5 &  0.00 &  0.00\\
instance n=20 13.alb & 1 & 0 & Optimal &  0.23 & 3 &  0.00 &  0.00\\
instance n=20 130.alb & 1 & 0 & Optimal &  0.23 & 6 &  0.00 &  0.00\\
instance n=20 131.alb & 1 & 0 & Optimal &  0.23 & 7 &  0.00 &  0.00\\
instance n=20 132.alb & 1 & 0 & Optimal &  0.24 & 4 &  0.00 &  0.00\\
instance n=20 133.alb & 1 & 0 & Optimal &  0.24 & 5 &  0.00 &  0.00\\
instance n=20 134.alb & 1 & 0 & Optimal &  0.24 & 6 &  0.00 &  0.00\\
instance n=20 135.alb & 1 & 0 & Optimal &  0.24 & 6 &  0.00 &  0.00\\
instance n=20 136.alb & 1 & 0 & Optimal &  0.25 & 6 &  0.00 &  0.00\\
instance n=20 137.alb & 1 & 0 & Optimal &  0.24 & 5 &  0.00 &  0.00\\
instance n=20 138.alb & 1 & 0 & Optimal &  0.24 & 5 &  0.00 &  0.00\\
instance n=20 139.alb & 1 & 0 & Optimal &  0.23 & 5 &  0.00 &  0.00\\
instance n=20 14.alb & 1 & 0 & Optimal &  0.25 & 3 &  0.00 &  0.00\\
instance n=20 140.alb & 1 & 0 & Optimal &  0.25 & 5 &  0.00 &  0.00\\
instance n=20 141.alb & 1 & 0 & Optimal &  0.23 & 3 &  0.00 &  0.00\\
instance n=20 142.alb & 1 & 0 & Optimal &  0.25 & 3 &  0.00 &  0.00\\
instance n=20 143.alb & 1 & 0 & Optimal &  0.22 & 3 &  0.00 &  0.00\\
instance n=20 144.alb & 1 & 0 & Optimal &  0.25 & 4 &  0.00 &  0.00\\
instance n=20 145.alb & 1 & 0 & Optimal &  0.24 & 3 &  0.00 &  0.00\\
instance n=20 146.alb & 1 & 0 & Optimal &  0.24 & 3 &  0.00 &  0.00\\
instance n=20 147.alb & 1 & 0 & Optimal &  0.24 & 3 &  0.00 &  0.00\\
instance n=20 148.alb & 1 & 0 & Optimal &  0.23 & 3 &  0.00 &  0.00\\
instance n=20 149.alb & 1 & 0 & Optimal &  0.23 & 3 &  0.00 &  0.00\\
instance n=20 15.alb & 1 & 0 & Optimal &  0.23 & 3 &  0.00 &  0.00\\
instance n=20 150.alb & 1 & 0 & Optimal &  0.23 & 3 &  0.00 &  0.00\\
instance n=20 151.alb & 1 & 0 & Optimal &  0.23 & 3 &  0.00 &  0.00\\
instance n=20 152.alb & 1 & 0 & Optimal &  0.25 & 3 &  0.00 &  0.00\\
instance n=20 153.alb & 1 & 0 & Optimal &  0.25 & 3 &  0.00 &  0.00\\
instance n=20 154.alb & 1 & 0 & Optimal &  0.23 & 3 &  0.00 &  0.00\\
instance n=20 155.alb & 1 & 0 & Optimal &  0.23 & 3 &  0.00 &  0.00\\
instance n=20 156.alb & 1 & 0 & Optimal &  0.24 & 3 &  0.00 &  0.00\\
instance n=20 157.alb & 1 & 0 & Optimal &  0.23 & 3 &  0.00 &  0.00\\
instance n=20 158.alb & 1 & 0 & Optimal &  0.24 & 3 &  0.00 &  0.00\\
instance n=20 159.alb & 1 & 0 & Optimal &  0.24 & 3 &  0.00 &  0.00\\
instance n=20 16.alb & 1 & 0 & Optimal &  0.30 & 12 &  0.00 &  0.00\\
instance n=20 160.alb & 1 & 0 & Optimal &  0.23 & 3 &  0.00 &  0.00\\
instance n=20 161.alb & 1 & 0 & Optimal &  0.22 & 3 &  0.00 &  0.00\\
instance n=20 162.alb & 1 & 0 & Optimal &  0.24 & 3 &  0.00 &  0.00\\
instance n=20 163.alb & 1 & 0 & Optimal &  0.24 & 3 &  0.00 &  0.00\\
instance n=20 164.alb & 1 & 0 & Optimal &  0.24 & 4 &  0.00 &  0.00\\
instance n=20 165.alb & 1 & 0 & Optimal &  0.24 & 3 &  0.00 &  0.00\\
instance n=20 166.alb & 1 & 0 & Optimal &  0.37 & 12 &  0.00 &  0.00\\
instance n=20 167.alb & 1 & 0 & Optimal &  0.59 & 11 &  0.00 &  0.00\\
instance n=20 168.alb & 1 & 0 & Optimal &  0.27 & 10 &  0.00 &  0.00\\
instance n=20 169.alb & 1 & 0 & Optimal &  0.31 & 11 &  0.00 &  0.00\\
instance n=20 17.alb & 1 & 0 & Optimal &  0.27 & 10 &  0.00 &  0.00\\
instance n=20 170.alb & 1 & 0 & Optimal &  0.32 & 11 &  0.00 &  0.00\\
instance n=20 171.alb & 1 & 0 & Optimal &  1.97 & 13 &  0.00 &  0.00\\
instance n=20 172.alb & 1 & 0 & Optimal &  0.40 & 11 &  0.00 &  0.00\\
instance n=20 173.alb & 1 & 0 & Optimal &  0.33 & 11 &  0.00 &  0.00\\
instance n=20 174.alb & 1 & 0 & Optimal &  0.33 & 12 &  0.00 &  0.00\\
instance n=20 175.alb & 1 & 0 & Optimal &  0.26 & 10 &  0.00 &  0.00\\
instance n=20 176.alb & 1 & 0 & Optimal &  0.34 & 11 &  0.00 &  0.00\\
instance n=20 177.alb & 1 & 0 & Optimal &  0.49 & 10 &  0.00 &  0.00\\
instance n=20 178.alb & 1 & 0 & Optimal &  0.31 & 11 &  0.00 &  0.00\\
instance n=20 179.alb & 1 & 0 & Optimal &  0.31 & 11 &  0.00 &  0.00\\
instance n=20 18.alb & 1 & 0 & Optimal &  0.26 & 11 &  0.00 &  0.00\\
instance n=20 180.alb & 1 & 0 & Optimal &  0.45 & 13 &  0.00 &  0.00\\
instance n=20 181.alb & 1 & 0 & Optimal &  0.25 & 11 &  0.00 &  0.00\\
instance n=20 182.alb & 1 & 0 & Optimal &  0.85 & 11 &  0.00 &  0.00\\
instance n=20 183.alb & 1 & 0 & Optimal &  0.64 & 13 &  0.00 &  0.00\\
instance n=20 184.alb & 1 & 0 & Optimal &  0.33 & 12 &  0.00 &  0.00\\
instance n=20 185.alb & 1 & 0 & Optimal &  0.45 & 15 &  0.00 &  0.00\\
instance n=20 186.alb & 1 & 0 & Optimal &  0.56 & 14 &  0.00 &  0.00\\
instance n=20 187.alb & 1 & 0 & Optimal &  0.23 & 10 &  0.00 &  0.00\\
instance n=20 188.alb & 1 & 0 & Optimal &  0.30 & 11 &  0.00 &  0.00\\
instance n=20 189.alb & 1 & 0 & Optimal &  0.30 & 13 &  0.00 &  0.00\\
instance n=20 19.alb & 1 & 0 & Optimal &  0.50 & 14 &  0.00 &  0.00\\
instance n=20 190.alb & 1 & 0 & Optimal &  0.84 & 15 &  0.00 &  0.00\\
instance n=20 191.alb & 1 & 0 & Optimal &  0.23 & 4 &  0.00 &  0.00\\
instance n=20 192.alb & 1 & 0 & Optimal &  0.24 & 5 &  0.00 &  0.00\\
instance n=20 193.alb & 1 & 0 & Optimal &  0.23 & 5 &  0.00 &  0.00\\
instance n=20 194.alb & 1 & 0 & Optimal &  0.26 & 6 &  0.00 &  0.00\\
instance n=20 195.alb & 1 & 0 & Optimal &  0.30 & 6 &  0.00 &  0.00\\
instance n=20 196.alb & 1 & 0 & Optimal &  0.26 & 5 &  0.00 &  0.00\\
instance n=20 197.alb & 1 & 0 & Optimal &  0.22 & 4 &  0.00 &  0.00\\
instance n=20 198.alb & 1 & 0 & Optimal &  0.25 & 6 &  0.00 &  0.00\\
instance n=20 199.alb & 1 & 0 & Optimal &  0.23 & 5 &  0.00 &  0.00\\
instance n=20 2.alb & 1 & 0 & Optimal &  0.23 & 3 &  0.00 &  0.00\\
instance n=20 20.alb & 1 & 0 & Optimal &  0.37 & 11 &  0.00 &  0.00\\
instance n=20 200.alb & 1 & 0 & Optimal &  0.25 & 6 &  0.00 &  0.00\\
instance n=20 201.alb & 1 & 0 & Optimal &  0.35 & 6 &  0.00 &  0.00\\
instance n=20 202.alb & 1 & 0 & Optimal &  0.24 & 4 &  0.00 &  0.00\\
instance n=20 203.alb & 1 & 0 & Optimal &  0.24 & 4 &  0.00 &  0.00\\
instance n=20 204.alb & 1 & 0 & Optimal &  0.25 & 5 &  0.00 &  0.00\\
instance n=20 205.alb & 1 & 0 & Optimal &  0.29 & 6 &  0.00 &  0.00\\
instance n=20 206.alb & 1 & 0 & Optimal &  0.24 & 5 &  0.00 &  0.00\\
instance n=20 207.alb & 1 & 0 & Optimal &  0.29 & 6 &  0.00 &  0.00\\
instance n=20 208.alb & 1 & 0 & Optimal &  0.24 & 5 &  0.00 &  0.00\\
instance n=20 209.alb & 1 & 0 & Optimal &  0.24 & 4 &  0.00 &  0.00\\
instance n=20 21.alb & 1 & 0 & Optimal &  0.31 & 14 &  0.00 &  0.00\\
instance n=20 210.alb & 1 & 0 & Optimal &  0.25 & 5 &  0.00 &  0.00\\
instance n=20 211.alb & 1 & 0 & Optimal &  0.26 & 5 &  0.00 &  0.00\\
instance n=20 212.alb & 1 & 0 & Optimal &  0.26 & 5 &  0.00 &  0.00\\
instance n=20 213.alb & 1 & 0 & Optimal &  0.27 & 5 &  0.00 &  0.00\\
instance n=20 214.alb & 1 & 0 & Optimal &  0.23 & 5 &  0.00 &  0.00\\
instance n=20 215.alb & 1 & 0 & Optimal &  0.26 & 5 &  0.00 &  0.00\\
instance n=20 216.alb & 1 & 0 & Optimal &  0.22 & 3 &  0.00 &  0.00\\
instance n=20 217.alb & 1 & 0 & Optimal &  0.24 & 4 &  0.00 &  0.00\\
instance n=20 218.alb & 1 & 0 & Optimal &  0.23 & 3 &  0.00 &  0.00\\
instance n=20 219.alb & 1 & 0 & Optimal &  0.22 & 3 &  0.00 &  0.00\\
instance n=20 22.alb & 1 & 0 & Optimal &  0.28 & 12 &  0.00 &  0.00\\
instance n=20 220.alb & 1 & 0 & Optimal &  0.23 & 3 &  0.00 &  0.00\\
instance n=20 221.alb & 1 & 0 & Optimal &  0.25 & 3 &  0.00 &  0.00\\
instance n=20 222.alb & 1 & 0 & Optimal &  0.24 & 3 &  0.00 &  0.00\\
instance n=20 223.alb & 1 & 0 & Optimal &  0.24 & 3 &  0.00 &  0.00\\
instance n=20 224.alb & 1 & 0 & Optimal &  0.23 & 3 &  0.00 &  0.00\\
instance n=20 225.alb & 1 & 0 & Optimal &  0.23 & 3 &  0.00 &  0.00\\
instance n=20 226.alb & 1 & 0 & Optimal &  0.24 & 3 &  0.00 &  0.00\\
instance n=20 227.alb & 1 & 0 & Optimal &  0.23 & 3 &  0.00 &  0.00\\
instance n=20 228.alb & 1 & 0 & Optimal &  0.24 & 2 &  0.00 &  0.00\\
instance n=20 229.alb & 1 & 0 & Optimal &  0.24 & 3 &  0.00 &  0.00\\
instance n=20 23.alb & 1 & 0 & Optimal &  0.68 & 13 &  0.00 &  0.00\\
instance n=20 230.alb & 1 & 0 & Optimal &  0.23 & 3 &  0.00 &  0.00\\
instance n=20 231.alb & 1 & 0 & Optimal &  0.24 & 3 &  0.00 &  0.00\\
instance n=20 232.alb & 1 & 0 & Optimal &  0.23 & 3 &  0.00 &  0.00\\
instance n=20 233.alb & 1 & 0 & Optimal &  0.25 & 3 &  0.00 &  0.00\\
instance n=20 234.alb & 1 & 0 & Optimal &  0.23 & 3 &  0.00 &  0.00\\
instance n=20 235.alb & 1 & 0 & Optimal &  0.25 & 3 &  0.00 &  0.00\\
instance n=20 236.alb & 1 & 0 & Optimal &  0.24 & 3 &  0.00 &  0.00\\
instance n=20 237.alb & 1 & 0 & Optimal &  0.23 & 3 &  0.00 &  0.00\\
instance n=20 238.alb & 1 & 0 & Optimal &  0.23 & 3 &  0.00 &  0.00\\
instance n=20 239.alb & 1 & 0 & Optimal &  0.24 & 3 &  0.00 &  0.00\\
instance n=20 24.alb & 1 & 0 & Optimal &  0.30 & 11 &  0.00 &  0.00\\
instance n=20 240.alb & 1 & 0 & Optimal &  0.23 & 3 &  0.00 &  0.00\\
instance n=20 241.alb & 1 & 0 & Optimal &  0.23 & 13 &  0.00 &  0.00\\
instance n=20 242.alb & 1 & 0 & Optimal &  0.24 & 12 &  0.00 &  0.00\\
instance n=20 243.alb & 1 & 0 & Optimal &  0.26 & 10 &  0.00 &  0.00\\
instance n=20 244.alb & 1 & 0 & Optimal &  0.24 & 11 &  0.00 &  0.00\\
instance n=20 245.alb & 1 & 0 & Optimal &  0.23 & 13 &  0.00 &  0.00\\
instance n=20 246.alb & 1 & 0 & Optimal &  0.29 & 13 &  0.00 &  0.00\\
instance n=20 247.alb & 1 & 0 & Optimal &  0.24 & 11 &  0.00 &  0.00\\
instance n=20 248.alb & 1 & 0 & Optimal &  0.25 & 11 &  0.00 &  0.00\\
instance n=20 249.alb & 1 & 0 & Optimal &  0.27 & 13 &  0.00 &  0.00\\
instance n=20 25.alb & 1 & 0 & Optimal &  0.28 & 11 &  0.00 &  0.00\\
instance n=20 250.alb & 1 & 0 & Optimal &  0.25 & 10 &  0.00 &  0.00\\
instance n=20 251.alb & 1 & 0 & Optimal &  0.24 & 12 &  0.00 &  0.00\\
instance n=20 252.alb & 1 & 0 & Optimal &  0.24 & 11 &  0.00 &  0.00\\
instance n=20 253.alb & 1 & 0 & Optimal &  0.25 & 13 &  0.00 &  0.00\\
instance n=20 254.alb & 1 & 0 & Optimal &  0.24 & 12 &  0.00 &  0.00\\
instance n=20 255.alb & 1 & 0 & Optimal &  0.26 & 13 &  0.00 &  0.00\\
instance n=20 256.alb & 1 & 0 & Optimal &  0.25 & 14 &  0.00 &  0.00\\
instance n=20 257.alb & 1 & 0 & Optimal &  0.24 & 10 &  0.00 &  0.00\\
instance n=20 258.alb & 1 & 0 & Optimal &  0.24 & 13 &  0.00 &  0.00\\
instance n=20 259.alb & 1 & 0 & Optimal &  0.25 & 13 &  0.00 &  0.00\\
instance n=20 26.alb & 1 & 0 & Optimal &  0.28 & 12 &  0.00 &  0.00\\
instance n=20 260.alb & 1 & 0 & Optimal &  0.27 & 12 &  0.00 &  0.00\\
instance n=20 261.alb & 1 & 0 & Optimal &  0.24 & 12 &  0.00 &  0.00\\
instance n=20 262.alb & 1 & 0 & Optimal &  0.24 & 11 &  0.00 &  0.00\\
instance n=20 263.alb & 1 & 0 & Optimal &  0.24 & 12 &  0.00 &  0.00\\
instance n=20 264.alb & 1 & 0 & Optimal &  0.26 & 12 &  0.00 &  0.00\\
instance n=20 265.alb & 1 & 0 & Optimal &  0.24 & 12 &  0.00 &  0.00\\
instance n=20 266.alb & 1 & 0 & Optimal &  0.23 & 5 &  0.00 &  0.00\\
instance n=20 267.alb & 1 & 0 & Optimal &  0.24 & 6 &  0.00 &  0.00\\
instance n=20 268.alb & 1 & 0 & Optimal &  0.25 & 6 &  0.00 &  0.00\\
instance n=20 269.alb & 1 & 0 & Optimal &  0.25 & 7 &  0.00 &  0.00\\
instance n=20 27.alb & 1 & 0 & Optimal &  0.40 & 13 &  0.00 &  0.00\\
instance n=20 270.alb & 1 & 0 & Optimal &  0.24 & 7 &  0.00 &  0.00\\
instance n=20 271.alb & 1 & 0 & Optimal &  0.23 & 6 &  0.00 &  0.00\\
instance n=20 272.alb & 1 & 0 & Optimal &  0.25 & 5 &  0.00 &  0.00\\
instance n=20 273.alb & 1 & 0 & Optimal &  0.24 & 5 &  0.00 &  0.00\\
instance n=20 274.alb & 1 & 0 & Optimal &  0.23 & 6 &  0.00 &  0.00\\
instance n=20 275.alb & 1 & 0 & Optimal &  0.24 & 5 &  0.00 &  0.00\\
instance n=20 276.alb & 1 & 0 & Optimal &  0.23 & 4 &  0.00 &  0.00\\
instance n=20 277.alb & 1 & 0 & Optimal &  0.25 & 4 &  0.00 &  0.00\\
instance n=20 278.alb & 1 & 0 & Optimal &  0.24 & 6 &  0.00 &  0.00\\
instance n=20 279.alb & 1 & 0 & Optimal &  0.24 & 6 &  0.00 &  0.00\\
instance n=20 28.alb & 1 & 0 & Optimal &  0.45 & 12 &  0.00 &  0.00\\
instance n=20 280.alb & 1 & 0 & Optimal &  0.23 & 5 &  0.00 &  0.00\\
instance n=20 281.alb & 1 & 0 & Optimal &  0.25 & 4 &  0.00 &  0.00\\
instance n=20 282.alb & 1 & 0 & Optimal &  0.24 & 4 &  0.00 &  0.00\\
instance n=20 283.alb & 1 & 0 & Optimal &  0.25 & 5 &  0.00 &  0.00\\
instance n=20 284.alb & 1 & 0 & Optimal &  0.24 & 5 &  0.00 &  0.00\\
instance n=20 285.alb & 1 & 0 & Optimal &  0.24 & 5 &  0.00 &  0.00\\
instance n=20 286.alb & 1 & 0 & Optimal &  0.25 & 5 &  0.00 &  0.00\\
instance n=20 287.alb & 1 & 0 & Optimal &  0.23 & 5 &  0.00 &  0.00\\
instance n=20 288.alb & 1 & 0 & Optimal &  0.25 & 6 &  0.00 &  0.00\\
instance n=20 289.alb & 1 & 0 & Optimal &  0.24 & 5 &  0.00 &  0.00\\
instance n=20 29.alb & 1 & 0 & Optimal &  0.95 & 10 &  0.00 &  0.00\\
instance n=20 290.alb & 1 & 0 & Optimal &  0.24 & 5 &  0.00 &  0.00\\
instance n=20 291.alb & 1 & 0 & Optimal &  0.23 & 3 &  0.00 &  0.00\\
instance n=20 292.alb & 1 & 0 & Optimal &  0.23 & 3 &  0.00 &  0.00\\
instance n=20 293.alb & 1 & 0 & Optimal &  0.23 & 3 &  0.00 &  0.00\\
instance n=20 294.alb & 1 & 0 & Optimal &  0.24 & 3 &  0.00 &  0.00\\
instance n=20 295.alb & 1 & 0 & Optimal &  0.24 & 3 &  0.00 &  0.00\\
instance n=20 296.alb & 1 & 0 & Optimal &  0.23 & 3 &  0.00 &  0.00\\
instance n=20 297.alb & 1 & 0 & Optimal &  0.24 & 3 &  0.00 &  0.00\\
instance n=20 298.alb & 1 & 0 & Optimal &  0.22 & 3 &  0.00 &  0.00\\
instance n=20 299.alb & 1 & 0 & Optimal &  0.24 & 3 &  0.00 &  0.00\\
instance n=20 3.alb & 1 & 0 & Optimal &  0.25 & 3 &  0.00 &  0.00\\
instance n=20 30.alb & 1 & 0 & Optimal &  0.57 & 16 &  0.00 &  0.00\\
instance n=20 300.alb & 1 & 0 & Optimal &  0.25 & 4 &  0.00 &  0.00\\
instance n=20 301.alb & 1 & 0 & Optimal &  0.24 & 3 &  0.00 &  0.00\\
instance n=20 302.alb & 1 & 0 & Optimal &  0.23 & 3 &  0.00 &  0.00\\
instance n=20 303.alb & 1 & 0 & Optimal &  0.23 & 3 &  0.00 &  0.00\\
instance n=20 304.alb & 1 & 0 & Optimal &  0.25 & 3 &  0.00 &  0.00\\
instance n=20 305.alb & 1 & 0 & Optimal &  0.25 & 3 &  0.00 &  0.00\\
instance n=20 306.alb & 1 & 0 & Optimal &  0.23 & 3 &  0.00 &  0.00\\
instance n=20 307.alb & 1 & 0 & Optimal &  0.24 & 3 &  0.00 &  0.00\\
instance n=20 308.alb & 1 & 0 & Optimal &  0.23 & 3 &  0.00 &  0.00\\
instance n=20 309.alb & 1 & 0 & Optimal &  0.23 & 3 &  0.00 &  0.00\\
instance n=20 31.alb & 1 & 0 & Optimal &  0.27 & 12 &  0.00 &  0.00\\
instance n=20 310.alb & 1 & 0 & Optimal &  0.24 & 3 &  0.00 &  0.00\\
instance n=20 311.alb & 1 & 0 & Optimal &  0.23 & 3 &  0.00 &  0.00\\
instance n=20 312.alb & 1 & 0 & Optimal &  0.23 & 4 &  0.00 &  0.00\\
instance n=20 313.alb & 1 & 0 & Optimal &  0.24 & 3 &  0.00 &  0.00\\
instance n=20 314.alb & 1 & 0 & Optimal &  0.25 & 3 &  0.00 &  0.00\\
instance n=20 315.alb & 1 & 0 & Optimal &  0.24 & 3 &  0.00 &  0.00\\
instance n=20 316.alb & 1 & 0 & Optimal &  1.11 & 10 &  0.00 &  0.00\\
instance n=20 317.alb & 1 & 0 & Optimal &  0.90 & 10 &  0.00 &  0.00\\
instance n=20 318.alb & 1 & 0 & Optimal &  0.25 & 10 &  0.00 &  0.00\\
instance n=20 319.alb & 1 & 0 & Optimal &  0.37 & 14 &  0.00 &  0.00\\
instance n=20 32.alb & 1 & 0 & Optimal &  0.77 & 13 &  0.00 &  0.00\\
instance n=20 320.alb & 1 & 0 & Optimal &  0.27 & 12 &  0.00 &  0.00\\
instance n=20 321.alb & 1 & 0 & Optimal &  2.38 & 14 &  0.00 &  0.00\\
instance n=20 322.alb & 1 & 0 & Optimal &  0.42 & 12 &  0.00 &  0.00\\
instance n=20 323.alb & 1 & 0 & Optimal &  0.37 & 13 &  0.00 &  0.00\\
instance n=20 324.alb & 1 & 0 & Optimal &  0.53 & 9 &  0.00 &  0.00\\
instance n=20 325.alb & 1 & 0 & Optimal &  0.65 & 14 &  0.00 &  0.00\\
instance n=20 326.alb & 1 & 0 & Optimal &  0.52 & 14 &  0.00 &  0.00\\
instance n=20 327.alb & 1 & 0 & Optimal &  0.43 & 13 &  0.00 &  0.00\\
instance n=20 328.alb & 1 & 0 & Optimal &  0.45 & 13 &  0.00 &  0.00\\
instance n=20 329.alb & 1 & 0 & Optimal &  0.31 & 10 &  0.00 &  0.00\\
instance n=20 33.alb & 1 & 0 & Optimal &  0.28 & 11 &  0.00 &  0.00\\
instance n=20 330.alb & 1 & 0 & Optimal &  0.43 & 12 &  0.00 &  0.00\\
instance n=20 331.alb & 1 & 0 & Optimal &  1.50 & 13 &  0.00 &  0.00\\
instance n=20 332.alb & 1 & 0 & Optimal &  0.40 & 13 &  0.00 &  0.00\\
instance n=20 333.alb & 1 & 0 & Optimal &  0.35 & 11 &  0.00 &  0.00\\
instance n=20 334.alb & 1 & 0 & Optimal &  0.28 & 10 &  0.00 &  0.00\\
instance n=20 335.alb & 1 & 0 & Optimal &  1.94 & 14 &  0.00 &  0.00\\
instance n=20 336.alb & 1 & 0 & Optimal &  0.27 & 11 &  0.00 &  0.00\\
instance n=20 337.alb & 1 & 0 & Optimal &  0.31 & 10 &  0.00 &  0.00\\
instance n=20 338.alb & 1 & 0 & Optimal &  0.44 & 14 &  0.00 &  0.00\\
instance n=20 339.alb & 1 & 0 & Optimal &  0.45 & 13 &  0.00 &  0.00\\
instance n=20 34.alb & 1 & 0 & Optimal &  0.29 & 12 &  0.00 &  0.00\\
instance n=20 340.alb & 1 & 0 & Optimal &  0.27 & 11 &  0.00 &  0.00\\
instance n=20 341.alb & 1 & 0 & Optimal &  0.28 & 6 &  0.00 &  0.00\\
instance n=20 342.alb & 1 & 0 & Optimal &  0.25 & 6 &  0.00 &  0.00\\
instance n=20 343.alb & 1 & 0 & Optimal &  0.49 & 6 &  0.00 &  0.00\\
instance n=20 344.alb & 1 & 0 & Optimal &  0.25 & 6 &  0.00 &  0.00\\
instance n=20 345.alb & 1 & 0 & Optimal &  0.24 & 4 &  0.00 &  0.00\\
instance n=20 346.alb & 1 & 0 & Optimal &  0.24 & 5 &  0.00 &  0.00\\
instance n=20 347.alb & 1 & 0 & Optimal &  0.44 & 6 &  0.00 &  0.00\\
instance n=20 348.alb & 1 & 0 & Optimal &  0.24 & 5 &  0.00 &  0.00\\
instance n=20 349.alb & 1 & 0 & Optimal &  0.25 & 5 &  0.00 &  0.00\\
instance n=20 35.alb & 1 & 0 & Optimal &  0.24 & 12 &  0.00 &  0.00\\
instance n=20 350.alb & 1 & 0 & Optimal &  0.23 & 5 &  0.00 &  0.00\\
instance n=20 351.alb & 1 & 0 & Optimal &  0.25 & 5 &  0.00 &  0.00\\
instance n=20 352.alb & 1 & 0 & Optimal &  0.24 & 4 &  0.00 &  0.00\\
instance n=20 353.alb & 1 & 0 & Optimal &  0.24 & 6 &  0.00 &  0.00\\
instance n=20 354.alb & 1 & 0 & Optimal &  0.30 & 6 &  0.00 &  0.00\\
instance n=20 355.alb & 1 & 0 & Optimal &  0.25 & 5 &  0.00 &  0.00\\
instance n=20 356.alb & 1 & 0 & Optimal &  0.23 & 5 &  0.00 &  0.00\\
instance n=20 357.alb & 1 & 0 & Optimal &  0.24 & 5 &  0.00 &  0.00\\
instance n=20 358.alb & 1 & 0 & Optimal &  0.23 & 4 &  0.00 &  0.00\\
instance n=20 359.alb & 1 & 0 & Optimal &  0.26 & 4 &  0.00 &  0.00\\
instance n=20 36.alb & 1 & 0 & Optimal &  0.28 & 13 &  0.00 &  0.00\\
instance n=20 360.alb & 1 & 0 & Optimal &  0.28 & 6 &  0.00 &  0.00\\
instance n=20 361.alb & 1 & 0 & Optimal &  0.26 & 5 &  0.00 &  0.00\\
instance n=20 362.alb & 1 & 0 & Optimal &  0.25 & 5 &  0.00 &  0.00\\
instance n=20 363.alb & 1 & 0 & Optimal &  0.70 & 7 &  0.00 &  0.00\\
instance n=20 364.alb & 1 & 0 & Optimal &  0.22 & 4 &  0.00 &  0.00\\
instance n=20 365.alb & 1 & 0 & Optimal &  0.25 & 5 &  0.00 &  0.00\\
instance n=20 366.alb & 1 & 0 & Optimal &  0.24 & 3 &  0.00 &  0.00\\
instance n=20 367.alb & 1 & 0 & Optimal &  0.24 & 3 &  0.00 &  0.00\\
instance n=20 368.alb & 1 & 0 & Optimal &  0.23 & 3 &  0.00 &  0.00\\
instance n=20 369.alb & 1 & 0 & Optimal &  0.24 & 3 &  0.00 &  0.00\\
instance n=20 37.alb & 1 & 0 & Optimal &  0.32 & 12 &  0.00 &  0.00\\
instance n=20 370.alb & 1 & 0 & Optimal &  0.23 & 3 &  0.00 &  0.00\\
instance n=20 371.alb & 1 & 0 & Optimal &  0.23 & 3 &  0.00 &  0.00\\
instance n=20 372.alb & 1 & 0 & Optimal &  0.23 & 3 &  0.00 &  0.00\\
instance n=20 373.alb & 1 & 0 & Optimal &  0.25 & 3 &  0.00 &  0.00\\
instance n=20 374.alb & 1 & 0 & Optimal &  0.23 & 3 &  0.00 &  0.00\\
instance n=20 375.alb & 1 & 0 & Optimal &  0.23 & 3 &  0.00 &  0.00\\
instance n=20 376.alb & 1 & 0 & Optimal &  0.24 & 3 &  0.00 &  0.00\\
instance n=20 377.alb & 1 & 0 & Optimal &  0.23 & 3 &  0.00 &  0.00\\
instance n=20 378.alb & 1 & 0 & Optimal &  0.24 & 3 &  0.00 &  0.00\\
instance n=20 379.alb & 1 & 0 & Optimal &  0.22 & 4 &  0.00 &  0.00\\
instance n=20 38.alb & 1 & 0 & Optimal &  0.27 & 12 &  0.00 &  0.00\\
instance n=20 380.alb & 1 & 0 & Optimal &  0.22 & 3 &  0.00 &  0.00\\
instance n=20 381.alb & 1 & 0 & Optimal &  0.23 & 3 &  0.00 &  0.00\\
instance n=20 382.alb & 1 & 0 & Optimal &  0.24 & 4 &  0.00 &  0.00\\
instance n=20 383.alb & 1 & 0 & Optimal &  0.24 & 3 &  0.00 &  0.00\\
instance n=20 384.alb & 1 & 0 & Optimal &  0.24 & 3 &  0.00 &  0.00\\
instance n=20 385.alb & 1 & 0 & Optimal &  0.23 & 3 &  0.00 &  0.00\\
instance n=20 386.alb & 1 & 0 & Optimal &  0.25 & 3 &  0.00 &  0.00\\
instance n=20 387.alb & 1 & 0 & Optimal &  0.24 & 3 &  0.00 &  0.00\\
instance n=20 388.alb & 1 & 0 & Optimal &  0.24 & 3 &  0.00 &  0.00\\
instance n=20 389.alb & 1 & 0 & Optimal &  0.24 & 3 &  0.00 &  0.00\\
instance n=20 39.alb & 1 & 0 & Optimal &  0.25 & 13 &  0.00 &  0.00\\
instance n=20 390.alb & 1 & 0 & Optimal &  0.24 & 3 &  0.00 &  0.00\\
instance n=20 391.alb & 1 & 0 & Optimal &  0.25 & 11 &  0.00 &  0.00\\
instance n=20 392.alb & 1 & 0 & Optimal &  0.27 & 14 &  0.00 &  0.00\\
instance n=20 393.alb & 1 & 0 & Optimal &  0.24 & 11 &  0.00 &  0.00\\
instance n=20 394.alb & 1 & 0 & Optimal &  0.24 & 12 &  0.00 &  0.00\\
instance n=20 395.alb & 1 & 0 & Optimal &  0.25 & 12 &  0.00 &  0.00\\
instance n=20 396.alb & 1 & 0 & Optimal &  0.25 & 13 &  0.00 &  0.00\\
instance n=20 397.alb & 1 & 0 & Optimal &  0.25 & 10 &  0.00 &  0.00\\
instance n=20 398.alb & 1 & 0 & Optimal &  0.23 & 11 &  0.00 &  0.00\\
instance n=20 399.alb & 1 & 0 & Optimal &  0.26 & 13 &  0.00 &  0.00\\
instance n=20 4.alb & 1 & 0 & Optimal &  0.24 & 3 &  0.00 &  0.00\\
instance n=20 40.alb & 1 & 0 & Optimal &  0.39 & 12 &  0.00 &  0.00\\
instance n=20 400.alb & 1 & 0 & Optimal &  0.25 & 12 &  0.00 &  0.00\\
instance n=20 401.alb & 1 & 0 & Optimal &  0.26 & 12 &  0.00 &  0.00\\
instance n=20 402.alb & 1 & 0 & Optimal &  0.24 & 12 &  0.00 &  0.00\\
instance n=20 403.alb & 1 & 0 & Optimal &  0.25 & 12 &  0.00 &  0.00\\
instance n=20 404.alb & 1 & 0 & Optimal &  0.26 & 10 &  0.00 &  0.00\\
instance n=20 405.alb & 1 & 0 & Optimal &  0.25 & 12 &  0.00 &  0.00\\
instance n=20 406.alb & 1 & 0 & Optimal &  0.25 & 14 &  0.00 &  0.00\\
instance n=20 407.alb & 1 & 0 & Optimal &  0.25 & 10 &  0.00 &  0.00\\
instance n=20 408.alb & 1 & 0 & Optimal &  0.25 & 14 &  0.00 &  0.00\\
instance n=20 409.alb & 1 & 0 & Optimal &  0.24 & 12 &  0.00 &  0.00\\
instance n=20 41.alb & 1 & 0 & Optimal &  0.29 & 6 &  0.00 &  0.00\\
instance n=20 410.alb & 1 & 0 & Optimal &  0.25 & 11 &  0.00 &  0.00\\
instance n=20 411.alb & 1 & 0 & Optimal &  0.26 & 15 &  0.00 &  0.00\\
instance n=20 412.alb & 1 & 0 & Optimal &  0.25 & 11 &  0.00 &  0.00\\
instance n=20 413.alb & 1 & 0 & Optimal &  0.24 & 10 &  0.00 &  0.00\\
instance n=20 414.alb & 1 & 0 & Optimal &  0.26 & 12 &  0.00 &  0.00\\
instance n=20 415.alb & 1 & 0 & Optimal &  0.25 & 10 &  0.00 &  0.00\\
instance n=20 416.alb & 1 & 0 & Optimal &  0.25 & 6 &  0.00 &  0.00\\
instance n=20 417.alb & 1 & 0 & Optimal &  0.24 & 5 &  0.00 &  0.00\\
instance n=20 418.alb & 1 & 0 & Optimal &  0.24 & 6 &  0.00 &  0.00\\
instance n=20 419.alb & 1 & 0 & Optimal &  0.24 & 4 &  0.00 &  0.00\\
instance n=20 42.alb & 1 & 0 & Optimal &  0.24 & 5 &  0.00 &  0.00\\
instance n=20 420.alb & 1 & 0 & Optimal &  0.24 & 5 &  0.00 &  0.00\\
instance n=20 421.alb & 1 & 0 & Optimal &  0.24 & 6 &  0.00 &  0.00\\
instance n=20 422.alb & 1 & 0 & Optimal &  0.24 & 4 &  0.00 &  0.00\\
instance n=20 423.alb & 1 & 0 & Optimal &  0.25 & 6 &  0.00 &  0.00\\
instance n=20 424.alb & 1 & 0 & Optimal &  0.24 & 5 &  0.00 &  0.00\\
instance n=20 425.alb & 1 & 0 & Optimal &  0.25 & 6 &  0.00 &  0.00\\
instance n=20 426.alb & 1 & 0 & Optimal &  0.23 & 5 &  0.00 &  0.00\\
instance n=20 427.alb & 1 & 0 & Optimal &  0.23 & 6 &  0.00 &  0.00\\
instance n=20 428.alb & 1 & 0 & Optimal &  0.23 & 5 &  0.00 &  0.00\\
instance n=20 429.alb & 1 & 0 & Optimal &  0.24 & 4 &  0.00 &  0.00\\
instance n=20 43.alb & 1 & 0 & Optimal &  0.25 & 5 &  0.00 &  0.00\\
instance n=20 430.alb & 1 & 0 & Optimal &  0.24 & 5 &  0.00 &  0.00\\
instance n=20 431.alb & 1 & 0 & Optimal &  0.23 & 6 &  0.00 &  0.00\\
instance n=20 432.alb & 1 & 0 & Optimal &  0.25 & 5 &  0.00 &  0.00\\
instance n=20 433.alb & 1 & 0 & Optimal &  0.24 & 5 &  0.00 &  0.00\\
instance n=20 434.alb & 1 & 0 & Optimal &  0.24 & 5 &  0.00 &  0.00\\
instance n=20 435.alb & 1 & 0 & Optimal &  0.24 & 7 &  0.00 &  0.00\\
instance n=20 436.alb & 1 & 0 & Optimal &  0.23 & 5 &  0.00 &  0.00\\
instance n=20 437.alb & 1 & 0 & Optimal &  0.24 & 5 &  0.00 &  0.00\\
instance n=20 438.alb & 1 & 0 & Optimal &  0.24 & 6 &  0.00 &  0.00\\
instance n=20 439.alb & 1 & 0 & Optimal &  0.24 & 5 &  0.00 &  0.00\\
instance n=20 44.alb & 1 & 0 & Optimal &  0.24 & 5 &  0.00 &  0.00\\
instance n=20 440.alb & 1 & 0 & Optimal &  0.24 & 5 &  0.00 &  0.00\\
instance n=20 441.alb & 1 & 0 & Optimal &  0.24 & 3 &  0.00 &  0.00\\
instance n=20 442.alb & 1 & 0 & Optimal &  0.23 & 3 &  0.00 &  0.00\\
instance n=20 443.alb & 1 & 0 & Optimal &  0.24 & 3 &  0.00 &  0.00\\
instance n=20 444.alb & 1 & 0 & Optimal &  0.23 & 3 &  0.00 &  0.00\\
instance n=20 445.alb & 1 & 0 & Optimal &  0.25 & 3 &  0.00 &  0.00\\
instance n=20 446.alb & 1 & 0 & Optimal &  0.24 & 3 &  0.00 &  0.00\\
instance n=20 447.alb & 1 & 0 & Optimal &  0.25 & 3 &  0.00 &  0.00\\
instance n=20 448.alb & 1 & 0 & Optimal &  0.24 & 3 &  0.00 &  0.00\\
instance n=20 449.alb & 1 & 0 & Optimal &  0.23 & 3 &  0.00 &  0.00\\
instance n=20 45.alb & 1 & 0 & Optimal &  0.25 & 6 &  0.00 &  0.00\\
instance n=20 450.alb & 1 & 0 & Optimal &  0.24 & 3 &  0.00 &  0.00\\
instance n=20 451.alb & 1 & 0 & Optimal &  0.25 & 3 &  0.00 &  0.00\\
instance n=20 452.alb & 1 & 0 & Optimal &  0.23 & 3 &  0.00 &  0.00\\
instance n=20 453.alb & 1 & 0 & Optimal &  0.24 & 3 &  0.00 &  0.00\\
instance n=20 454.alb & 1 & 0 & Optimal &  0.23 & 3 &  0.00 &  0.00\\
instance n=20 455.alb & 1 & 0 & Optimal &  0.24 & 3 &  0.00 &  0.00\\
instance n=20 456.alb & 1 & 0 & Optimal &  0.23 & 4 &  0.00 &  0.00\\
instance n=20 457.alb & 1 & 0 & Optimal &  0.24 & 3 &  0.00 &  0.00\\
instance n=20 458.alb & 1 & 0 & Optimal &  0.24 & 3 &  0.00 &  0.00\\
instance n=20 459.alb & 1 & 0 & Optimal &  0.24 & 3 &  0.00 &  0.00\\
instance n=20 46.alb & 1 & 0 & Optimal &  0.25 & 4 &  0.00 &  0.00\\
instance n=20 460.alb & 1 & 0 & Optimal &  0.24 & 3 &  0.00 &  0.00\\
instance n=20 461.alb & 1 & 0 & Optimal &  0.26 & 3 &  0.00 &  0.00\\
instance n=20 462.alb & 1 & 0 & Optimal &  0.23 & 3 &  0.00 &  0.00\\
instance n=20 463.alb & 1 & 0 & Optimal &  0.24 & 3 &  0.00 &  0.00\\
instance n=20 464.alb & 1 & 0 & Optimal &  0.25 & 3 &  0.00 &  0.00\\
instance n=20 465.alb & 1 & 0 & Optimal &  0.24 & 3 &  0.00 &  0.00\\
instance n=20 466.alb & 1 & 0 & Optimal &  0.24 & 13 &  0.00 &  0.00\\
instance n=20 467.alb & 1 & 0 & Optimal &  0.24 & 14 &  0.00 &  0.00\\
instance n=20 468.alb & 1 & 0 & Optimal &  0.25 & 13 &  0.00 &  0.00\\
instance n=20 469.alb & 1 & 0 & Optimal &  0.23 & 14 &  0.00 &  0.00\\
instance n=20 47.alb & 1 & 0 & Optimal &  0.24 & 4 &  0.00 &  0.00\\
instance n=20 470.alb & 1 & 0 & Optimal &  0.23 & 12 &  0.00 &  0.00\\
instance n=20 471.alb & 1 & 0 & Optimal &  0.24 & 12 &  0.00 &  0.00\\
instance n=20 472.alb & 1 & 0 & Optimal &  0.23 & 13 &  0.00 &  0.00\\
instance n=20 473.alb & 1 & 0 & Optimal &  0.24 & 10 &  0.00 &  0.00\\
instance n=20 474.alb & 1 & 0 & Optimal &  0.25 & 14 &  0.00 &  0.00\\
instance n=20 475.alb & 1 & 0 & Optimal &  0.23 & 11 &  0.00 &  0.00\\
instance n=20 476.alb & 1 & 0 & Optimal &  0.23 & 11 &  0.00 &  0.00\\
instance n=20 477.alb & 1 & 0 & Optimal &  0.23 & 11 &  0.00 &  0.00\\
instance n=20 478.alb & 1 & 0 & Optimal &  0.25 & 12 &  0.00 &  0.00\\
instance n=20 479.alb & 1 & 0 & Optimal &  0.23 & 13 &  0.00 &  0.00\\
instance n=20 48.alb & 1 & 0 & Optimal &  0.24 & 5 &  0.00 &  0.00\\
instance n=20 480.alb & 1 & 0 & Optimal &  0.24 & 13 &  0.00 &  0.00\\
instance n=20 481.alb & 1 & 0 & Optimal &  0.24 & 13 &  0.00 &  0.00\\
instance n=20 482.alb & 1 & 0 & Optimal &  0.23 & 13 &  0.00 &  0.00\\
instance n=20 483.alb & 1 & 0 & Optimal &  0.25 & 12 &  0.00 &  0.00\\
instance n=20 484.alb & 1 & 0 & Optimal &  0.25 & 13 &  0.00 &  0.00\\
instance n=20 485.alb & 1 & 0 & Optimal &  0.24 & 15 &  0.00 &  0.00\\
instance n=20 486.alb & 1 & 0 & Optimal &  0.23 & 11 &  0.00 &  0.00\\
instance n=20 487.alb & 1 & 0 & Optimal &  0.24 & 12 &  0.00 &  0.00\\
instance n=20 488.alb & 1 & 0 & Optimal &  0.25 & 15 &  0.00 &  0.00\\
instance n=20 489.alb & 1 & 0 & Optimal &  0.25 & 12 &  0.00 &  0.00\\
instance n=20 49.alb & 1 & 0 & Optimal &  0.23 & 4 &  0.00 &  0.00\\
instance n=20 490.alb & 1 & 0 & Optimal &  0.25 & 12 &  0.00 &  0.00\\
instance n=20 491.alb & 1 & 0 & Optimal &  0.24 & 6 &  0.00 &  0.00\\
instance n=20 492.alb & 1 & 0 & Optimal &  0.23 & 5 &  0.00 &  0.00\\
instance n=20 493.alb & 1 & 0 & Optimal &  0.25 & 5 &  0.00 &  0.00\\
instance n=20 494.alb & 1 & 0 & Optimal &  0.23 & 6 &  0.00 &  0.00\\
instance n=20 495.alb & 1 & 0 & Optimal &  0.23 & 6 &  0.00 &  0.00\\
instance n=20 496.alb & 1 & 0 & Optimal &  0.25 & 5 &  0.00 &  0.00\\
instance n=20 497.alb & 1 & 0 & Optimal &  0.26 & 6 &  0.00 &  0.00\\
instance n=20 498.alb & 1 & 0 & Optimal &  0.24 & 6 &  0.00 &  0.00\\
instance n=20 499.alb & 1 & 0 & Optimal &  0.25 & 5 &  0.00 &  0.00\\
instance n=20 5.alb & 1 & 0 & Optimal &  0.25 & 3 &  0.00 &  0.00\\
instance n=20 50.alb & 1 & 0 & Optimal &  0.24 & 4 &  0.00 &  0.00\\
instance n=20 500.alb & 1 & 0 & Optimal &  0.23 & 8 &  0.00 &  0.00\\
instance n=20 501.alb & 1 & 0 & Optimal &  0.25 & 5 &  0.00 &  0.00\\
instance n=20 502.alb & 1 & 0 & Optimal &  0.26 & 4 &  0.00 &  0.00\\
instance n=20 503.alb & 1 & 0 & Optimal &  0.23 & 6 &  0.00 &  0.00\\
instance n=20 504.alb & 1 & 0 & Optimal &  0.24 & 6 &  0.00 &  0.00\\
instance n=20 505.alb & 1 & 0 & Optimal &  0.23 & 6 &  0.00 &  0.00\\
instance n=20 506.alb & 1 & 0 & Optimal &  0.24 & 5 &  0.00 &  0.00\\
instance n=20 507.alb & 1 & 0 & Optimal &  0.24 & 5 &  0.00 &  0.00\\
instance n=20 508.alb & 1 & 0 & Optimal &  0.23 & 5 &  0.00 &  0.00\\
instance n=20 509.alb & 1 & 0 & Optimal &  0.25 & 4 &  0.00 &  0.00\\
instance n=20 51.alb & 1 & 0 & Optimal &  0.22 & 4 &  0.00 &  0.00\\
instance n=20 510.alb & 1 & 0 & Optimal &  0.24 & 5 &  0.00 &  0.00\\
instance n=20 511.alb & 1 & 0 & Optimal &  0.24 & 5 &  0.00 &  0.00\\
instance n=20 512.alb & 1 & 0 & Optimal &  0.25 & 5 &  0.00 &  0.00\\
instance n=20 513.alb & 1 & 0 & Optimal &  0.25 & 5 &  0.00 &  0.00\\
instance n=20 514.alb & 1 & 0 & Optimal &  0.24 & 5 &  0.00 &  0.00\\
instance n=20 515.alb & 1 & 0 & Optimal &  0.25 & 6 &  0.00 &  0.00\\
instance n=20 516.alb & 1 & 0 & Optimal &  0.23 & 3 &  0.00 &  0.00\\
instance n=20 517.alb & 1 & 0 & Optimal &  0.23 & 3 &  0.00 &  0.00\\
instance n=20 518.alb & 1 & 0 & Optimal &  0.23 & 3 &  0.00 &  0.00\\
instance n=20 519.alb & 1 & 0 & Optimal &  0.26 & 3 &  0.00 &  0.00\\
instance n=20 52.alb & 1 & 0 & Optimal &  0.23 & 4 &  0.00 &  0.00\\
instance n=20 520.alb & 1 & 0 & Optimal &  0.24 & 3 &  0.00 &  0.00\\
instance n=20 521.alb & 1 & 0 & Optimal &  0.24 & 3 &  0.00 &  0.00\\
instance n=20 522.alb & 1 & 0 & Optimal &  0.23 & 3 &  0.00 &  0.00\\
instance n=20 523.alb & 1 & 0 & Optimal &  0.24 & 3 &  0.00 &  0.00\\
instance n=20 524.alb & 1 & 0 & Optimal &  0.23 & 3 &  0.00 &  0.00\\
instance n=20 525.alb & 1 & 0 & Optimal &  0.25 & 3 &  0.00 &  0.00\\
instance n=20 53.alb & 1 & 0 & Optimal &  0.23 & 5 &  0.00 &  0.00\\
instance n=20 54.alb & 1 & 0 & Optimal &  0.23 & 5 &  0.00 &  0.00\\
instance n=20 55.alb & 1 & 0 & Optimal &  0.25 & 5 &  0.00 &  0.00\\
instance n=20 56.alb & 1 & 0 & Optimal &  0.24 & 4 &  0.00 &  0.00\\
instance n=20 57.alb & 1 & 0 & Optimal &  0.23 & 4 &  0.00 &  0.00\\
instance n=20 58.alb & 1 & 0 & Optimal &  0.25 & 5 &  0.00 &  0.00\\
instance n=20 59.alb & 1 & 0 & Optimal &  0.26 & 4 &  0.00 &  0.00\\
instance n=20 6.alb & 1 & 0 & Optimal &  0.24 & 3 &  0.00 &  0.00\\
instance n=20 60.alb & 1 & 0 & Optimal &  0.58 & 6 &  0.00 &  0.00\\
instance n=20 61.alb & 1 & 0 & Optimal &  0.26 & 7 &  0.00 &  0.00\\
instance n=20 62.alb & 1 & 0 & Optimal &  0.25 & 5 &  0.00 &  0.00\\
instance n=20 63.alb & 1 & 0 & Optimal &  0.25 & 5 &  0.00 &  0.00\\
instance n=20 64.alb & 1 & 0 & Optimal &  0.26 & 5 &  0.00 &  0.00\\
instance n=20 65.alb & 1 & 0 & Optimal &  0.24 & 5 &  0.00 &  0.00\\
instance n=20 66.alb & 1 & 0 & Optimal &  0.24 & 3 &  0.00 &  0.00\\
instance n=20 67.alb & 1 & 0 & Optimal &  0.24 & 3 &  0.00 &  0.00\\
instance n=20 68.alb & 1 & 0 & Optimal &  0.23 & 3 &  0.00 &  0.00\\
instance n=20 69.alb & 1 & 0 & Optimal &  0.24 & 2 &  0.00 &  0.00\\
instance n=20 7.alb & 1 & 0 & Optimal &  0.24 & 3 &  0.00 &  0.00\\
instance n=20 70.alb & 1 & 0 & Optimal &  0.23 & 3 &  0.00 &  0.00\\
instance n=20 71.alb & 1 & 0 & Optimal &  0.24 & 3 &  0.00 &  0.00\\
instance n=20 72.alb & 1 & 0 & Optimal &  0.24 & 3 &  0.00 &  0.00\\
instance n=20 73.alb & 1 & 0 & Optimal &  0.24 & 2 &  0.00 &  0.00\\
instance n=20 74.alb & 1 & 0 & Optimal &  0.25 & 3 &  0.00 &  0.00\\
instance n=20 75.alb & 1 & 0 & Optimal &  0.23 & 3 &  0.00 &  0.00\\
instance n=20 76.alb & 1 & 0 & Optimal &  0.24 & 3 &  0.00 &  0.00\\
instance n=20 77.alb & 1 & 0 & Optimal &  0.24 & 3 &  0.00 &  0.00\\
instance n=20 78.alb & 1 & 0 & Optimal &  0.23 & 3 &  0.00 &  0.00\\
instance n=20 79.alb & 1 & 0 & Optimal &  0.24 & 3 &  0.00 &  0.00\\
instance n=20 8.alb & 1 & 0 & Optimal &  0.23 & 3 &  0.00 &  0.00\\
instance n=20 80.alb & 1 & 0 & Optimal &  0.24 & 3 &  0.00 &  0.00\\
instance n=20 81.alb & 1 & 0 & Optimal &  0.23 & 3 &  0.00 &  0.00\\
instance n=20 82.alb & 1 & 0 & Optimal &  0.25 & 4 &  0.00 &  0.00\\
instance n=20 83.alb & 1 & 0 & Optimal &  0.24 & 3 &  0.00 &  0.00\\
instance n=20 84.alb & 1 & 0 & Optimal &  0.25 & 3 &  0.00 &  0.00\\
instance n=20 85.alb & 1 & 0 & Optimal &  0.24 & 3 &  0.00 &  0.00\\
instance n=20 86.alb & 1 & 0 & Optimal &  0.23 & 3 &  0.00 &  0.00\\
instance n=20 87.alb & 1 & 0 & Optimal &  0.23 & 3 &  0.00 &  0.00\\
instance n=20 88.alb & 1 & 0 & Optimal &  0.23 & 3 &  0.00 &  0.00\\
instance n=20 89.alb & 1 & 0 & Optimal &  0.24 & 3 &  0.00 &  0.00\\
instance n=20 9.alb & 1 & 0 & Optimal &  0.25 & 3 &  0.00 &  0.00\\
instance n=20 90.alb & 1 & 0 & Optimal &  0.24 & 3 &  0.00 &  0.00\\
instance n=20 91.alb & 1 & 0 & Optimal &  0.24 & 11 &  0.00 &  0.00\\
instance n=20 92.alb & 1 & 0 & Optimal &  0.23 & 11 &  0.00 &  0.00\\
instance n=20 93.alb & 1 & 0 & Optimal &  0.27 & 13 &  0.00 &  0.00\\
instance n=20 94.alb & 1 & 0 & Optimal &  0.25 & 10 &  0.00 &  0.00\\
instance n=20 95.alb & 1 & 0 & Optimal &  0.25 & 12 &  0.00 &  0.00\\
instance n=20 96.alb & 1 & 0 & Optimal &  0.25 & 10 &  0.00 &  0.00\\
instance n=20 97.alb & 1 & 0 & Optimal &  0.30 & 15 &  0.00 &  0.00\\
instance n=20 98.alb & 1 & 0 & Optimal &  0.26 & 13 &  0.00 &  0.00\\
instance n=20 99.alb & 1 & 0 & Optimal &  0.27 & 12 &  0.00 &  0.00\\
instance n=50 1.alb & 1 & 0 & Solution & 30.05 & 8 &  0.00 &  0.00\\
instance n=50 10.alb & 1 & 0 & Solution & 30.05 & 7 &  0.00 &  0.00\\
instance n=50 100.alb & 1 & 0 & Optimal &  1.34 & 7 &  0.00 &  0.00\\
instance n=50 101.alb & 1 & 0 & Optimal & 15.64 & 30 &  0.00 &  0.00\\
instance n=50 102.alb & 1 & 0 & Optimal & 14.72 & 32 &  0.00 &  0.00\\
instance n=50 103.alb & 1 & 0 & Optimal & 29.99 & 29 &  0.00 &  0.00\\
instance n=50 104.alb & 1 & 0 & Optimal &  1.86 & 27 &  0.00 &  0.00\\
instance n=50 105.alb & 1 & 0 & Solution & 30.07 & 24 &  0.00 &  0.00\\
instance n=50 106.alb & 1 & 0 & Optimal & 10.20 & 28 &  0.00 &  0.00\\
instance n=50 107.alb & 1 & 0 & Optimal &  5.48 & 28 &  0.00 &  0.00\\
instance n=50 108.alb & 1 & 0 & Optimal & 25.74 & 30 &  0.00 &  0.00\\
instance n=50 109.alb & 1 & 0 & Optimal &  5.94 & 30 &  0.00 &  0.00\\
instance n=50 11.alb & 1 & 0 & Solution & 30.07 & 7 &  0.00 &  0.00\\
instance n=50 110.alb & 1 & 0 & Solution & 30.07 & 27 &  0.00 &  0.00\\
instance n=50 111.alb & 1 & 0 & Optimal &  2.37 & 28 &  0.00 &  0.00\\
instance n=50 112.alb & 1 & 0 & Optimal &  2.83 & 27 &  0.00 &  0.00\\
instance n=50 113.alb & 1 & 0 & Solution & 30.06 & 28 &  0.00 &  0.00\\
instance n=50 114.alb & 1 & 0 & Optimal &  5.26 & 27 &  0.00 &  0.00\\
instance n=50 115.alb & 1 & 0 & Solution & 30.07 & 29 &  0.00 &  0.00\\
instance n=50 116.alb & 1 & 0 & Optimal & 14.07 & 32 &  0.00 &  0.00\\
instance n=50 117.alb & 1 & 0 & Optimal &  9.98 & 27 &  0.00 &  0.00\\
instance n=50 118.alb & 1 & 0 & Optimal &  2.58 & 29 &  0.00 &  0.00\\
instance n=50 119.alb & 1 & 0 & Optimal &  6.11 & 25 &  0.00 &  0.00\\
instance n=50 12.alb & 1 & 0 & Solution & 30.06 & 7 &  0.00 &  0.00\\
instance n=50 120.alb & 1 & 0 & Optimal &  3.50 & 27 &  0.00 &  0.00\\
instance n=50 121.alb & 1 & 0 & Optimal &  6.45 & 32 &  0.00 &  0.00\\
instance n=50 122.alb & 1 & 0 & Optimal & 11.70 & 29 &  0.00 &  0.00\\
instance n=50 123.alb & 1 & 0 & Optimal & 12.53 & 32 &  0.00 &  0.00\\
instance n=50 124.alb & 1 & 0 & Optimal &  8.08 & 29 &  0.00 &  0.00\\
instance n=50 125.alb & 1 & 0 & Solution & 30.05 & 33 &  0.00 &  0.00\\
instance n=50 126.alb & 1 & 0 & Optimal & 10.18 & 12 &  0.00 &  0.00\\
instance n=50 127.alb & 1 & 0 & Optimal &  3.28 & 14 &  0.00 &  0.00\\
instance n=50 128.alb & 1 & 0 & Optimal &  4.33 & 12 &  0.00 &  0.00\\
instance n=50 129.alb & 1 & 0 & Optimal & 11.79 & 13 &  0.00 &  0.00\\
instance n=50 13.alb & 1 & 0 & Solution & 30.06 & 6 &  0.00 &  0.00\\
instance n=50 130.alb & 1 & 0 & Optimal &  6.71 & 13 &  0.00 &  0.00\\
instance n=50 131.alb & 1 & 0 & Optimal &  3.67 & 12 &  0.00 &  0.00\\
instance n=50 132.alb & 1 & 0 & Optimal & 13.49 & 12 &  0.00 &  0.00\\
instance n=50 133.alb & 1 & 0 & Optimal &  4.76 & 12 &  0.00 &  0.00\\
instance n=50 134.alb & 1 & 0 & Optimal &  2.63 & 14 &  0.00 &  0.00\\
instance n=50 135.alb & 1 & 0 & Optimal &  5.17 & 13 &  0.00 &  0.00\\
instance n=50 136.alb & 1 & 0 & Optimal &  5.89 & 11 &  0.00 &  0.00\\
instance n=50 137.alb & 1 & 0 & Optimal &  6.50 & 11 &  0.00 &  0.00\\
instance n=50 138.alb & 1 & 0 & Optimal &  4.26 & 12 &  0.00 &  0.00\\
instance n=50 139.alb & 1 & 0 & Optimal & 11.13 & 11 &  0.00 &  0.00\\
instance n=50 14.alb & 1 & 0 & Solution & 30.05 & 7 &  0.00 &  0.00\\
instance n=50 140.alb & 1 & 0 & Optimal &  1.44 & 12 &  0.00 &  0.00\\
instance n=50 141.alb & 1 & 0 & Optimal &  2.72 & 13 &  0.00 &  0.00\\
instance n=50 142.alb & 1 & 0 & Optimal &  4.07 & 11 &  0.00 &  0.00\\
instance n=50 143.alb & 1 & 0 & Optimal &  0.56 & 12 &  0.00 &  0.00\\
instance n=50 144.alb & 1 & 0 & Optimal &  1.24 & 13 &  0.00 &  0.00\\
instance n=50 145.alb & 1 & 0 & Optimal &  1.21 & 10 &  0.00 &  0.00\\
instance n=50 146.alb & 1 & 0 & Optimal &  1.48 & 13 &  0.00 &  0.00\\
instance n=50 147.alb & 1 & 0 & Optimal &  8.57 & 13 &  0.00 &  0.00\\
instance n=50 148.alb & 1 & 0 & Optimal &  4.22 & 10 &  0.00 &  0.00\\
instance n=50 149.alb & 1 & 0 & Optimal &  1.32 & 12 &  0.00 &  0.00\\
instance n=50 15.alb & 1 & 0 & Solution & 30.05 & 8 &  0.00 &  0.00\\
instance n=50 150.alb & 1 & 0 & Optimal &  1.38 & 11 &  0.00 &  0.00\\
instance n=50 151.alb & 1 & 0 & Solution & 30.05 & 7 &  0.00 &  0.00\\
instance n=50 152.alb & 1 & 0 & Solution & 30.05 & 7 &  0.00 &  0.00\\
instance n=50 153.alb & 1 & 0 & Solution & 30.06 & 8 &  0.00 &  0.00\\
instance n=50 154.alb & 1 & 0 & Solution & 30.05 & 8 &  0.00 &  0.00\\
instance n=50 155.alb & 1 & 0 & Solution & 30.06 & 7 &  0.00 &  0.00\\
instance n=50 156.alb & 1 & 0 & Solution & 30.06 & 7 &  0.00 &  0.00\\
instance n=50 157.alb & 1 & 0 & Solution & 30.06 & 8 &  0.00 &  0.00\\
instance n=50 158.alb & 1 & 0 & Solution & 30.06 & 46 &  0.00 &  0.00\\
instance n=50 159.alb & 1 & 0 & Solution & 30.05 & 7 &  0.00 &  0.00\\
instance n=50 16.alb & 1 & 0 & Solution & 30.05 & 8 &  0.00 &  0.00\\
instance n=50 160.alb & 1 & 0 & Solution & 30.05 & 8 &  0.00 &  0.00\\
instance n=50 161.alb & 1 & 0 & Solution & 30.05 & 7 &  0.00 &  0.00\\
instance n=50 162.alb & 1 & 0 & Solution & 30.06 & 8 &  0.00 &  0.00\\
instance n=50 163.alb & 1 & 0 & Solution & 30.06 & 7 &  0.00 &  0.00\\
instance n=50 164.alb & 1 & 0 & Solution & 30.06 & 7 &  0.00 &  0.00\\
instance n=50 165.alb & 1 & 0 & Solution & 30.06 & 8 &  0.00 &  0.00\\
instance n=50 166.alb & 1 & 0 & Solution & 30.05 & 8 &  0.00 &  0.00\\
instance n=50 167.alb & 1 & 0 & Solution & 30.07 & 8 &  0.00 &  0.00\\
instance n=50 168.alb & 1 & 0 & Solution & 30.06 & 8 &  0.00 &  0.00\\
instance n=50 169.alb & 1 & 0 & Solution & 30.06 & 8 &  0.00 &  0.00\\
instance n=50 17.alb & 1 & 0 & Solution & 30.07 & 7 &  0.00 &  0.00\\
instance n=50 170.alb & 1 & 0 & Solution & 30.06 & 7 &  0.00 &  0.00\\
instance n=50 171.alb & 1 & 0 & Solution & 30.05 & 8 &  0.00 &  0.00\\
instance n=50 172.alb & 1 & 0 & Solution & 30.06 & 7 &  0.00 &  0.00\\
instance n=50 173.alb & 1 & 0 & Solution & 30.06 & 8 &  0.00 &  0.00\\
instance n=50 174.alb & 1 & 0 & Solution & 30.07 & 7 &  0.00 &  0.00\\
instance n=50 175.alb & 1 & 0 & Solution & 30.06 & 7 &  0.00 &  0.00\\
instance n=50 176.alb & 1 & 0 & Solution & 30.06 & 28 &  0.00 &  0.00\\
instance n=50 177.alb & 1 & 0 & Solution & 30.07 & 28 &  0.00 &  0.00\\
instance n=50 178.alb & 1 & 0 & Solution & 30.06 & 28 &  0.00 &  0.00\\
instance n=50 179.alb & 1 & 0 & Solution & 30.06 & 28 &  0.00 &  0.00\\
instance n=50 18.alb & 1 & 0 & Solution & 30.06 & 7 &  0.00 &  0.00\\
instance n=50 180.alb & 1 & 0 & Solution & 30.07 & 26 &  0.00 &  0.00\\
instance n=50 181.alb & 1 & 0 & Solution & 30.06 & 31 &  0.00 &  0.00\\
instance n=50 182.alb & 1 & 0 & Solution & 30.05 & 27 &  0.00 &  0.00\\
instance n=50 183.alb & 1 & 0 & Solution & 30.06 & 28 &  0.00 &  0.00\\
instance n=50 184.alb & 1 & 0 & Solution & 30.05 & 40 &  0.00 &  0.00\\
instance n=50 185.alb & 1 & 0 & Solution & 30.06 & 26 &  0.00 &  0.00\\
instance n=50 186.alb & 1 & 0 & Solution & 30.07 & 27 &  0.00 &  0.00\\
instance n=50 187.alb & 1 & 0 & Solution & 30.06 & 26 &  0.00 &  0.00\\
instance n=50 188.alb & 1 & 0 & Solution & 30.06 & 25 &  0.00 &  0.00\\
instance n=50 189.alb & 1 & 0 & Solution & 30.06 & 28 &  0.00 &  0.00\\
instance n=50 19.alb & 1 & 0 & Solution & 30.06 & 8 &  0.00 &  0.00\\
instance n=50 190.alb & 1 & 0 & Solution & 30.06 & 31 &  0.00 &  0.00\\
instance n=50 191.alb & 1 & 0 & Solution & 30.06 & 30 &  0.00 &  0.00\\
instance n=50 192.alb & 1 & 0 & Solution & 30.06 & 28 &  0.00 &  0.00\\
instance n=50 193.alb & 1 & 0 & Solution & 30.05 & 29 &  0.00 &  0.00\\
instance n=50 194.alb & 1 & 0 & Solution & 30.06 & 39 &  0.00 &  0.00\\
instance n=50 195.alb & 1 & 0 & Solution & 30.06 & 28 &  0.00 &  0.00\\
instance n=50 196.alb & 1 & 0 & Solution & 30.07 & 28 &  0.00 &  0.00\\
instance n=50 197.alb & 1 & 0 & Solution & 30.07 & 29 &  0.00 &  0.00\\
instance n=50 198.alb & 1 & 0 & Solution & 30.06 & 28 &  0.00 &  0.00\\
instance n=50 199.alb & 1 & 0 & Solution & 30.06 & 29 &  0.00 &  0.00\\
instance n=50 2.alb & 1 & 0 & Solution & 30.06 & 6 &  0.00 &  0.00\\
instance n=50 20.alb & 1 & 0 & Solution & 30.06 & 8 &  0.00 &  0.00\\
instance n=50 200.alb & 1 & 0 & Solution & 30.07 & 37 &  0.00 &  0.00\\
instance n=50 201.alb & 1 & 0 & Solution & 30.05 & 13 &  0.00 &  0.00\\
instance n=50 202.alb & 1 & 0 & Solution & 30.06 & 9 &  0.00 &  0.00\\
instance n=50 203.alb & 1 & 0 & Solution & 30.06 & 11 &  0.00 &  0.00\\
instance n=50 204.alb & 1 & 0 & Solution & 30.06 & 11 &  0.00 &  0.00\\
instance n=50 205.alb & 1 & 0 & Solution & 30.07 & 13 &  0.00 &  0.00\\
instance n=50 206.alb & 1 & 0 & Solution & 30.06 & 12 &  0.00 &  0.00\\
instance n=50 207.alb & 1 & 0 & Solution & 30.07 & 10 &  0.00 &  0.00\\
instance n=50 208.alb & 1 & 0 & Solution & 30.05 & 50 &  0.00 &  0.00\\
instance n=50 209.alb & 1 & 0 & Solution & 30.06 & 11 &  0.00 &  0.00\\
instance n=50 21.alb & 1 & 0 & Solution & 30.07 & 6 &  0.00 &  0.00\\
instance n=50 210.alb & 1 & 0 & Solution & 30.06 & 13 &  0.00 &  0.00\\
instance n=50 211.alb & 1 & 0 & Solution & 30.06 & 12 &  0.00 &  0.00\\
instance n=50 212.alb & 1 & 0 & Solution & 30.05 & 10 &  0.00 &  0.00\\
instance n=50 213.alb & 1 & 0 & Solution & 30.06 & 13 &  0.00 &  0.00\\
instance n=50 214.alb & 1 & 0 & Solution & 30.07 & 11 &  0.00 &  0.00\\
instance n=50 215.alb & 1 & 0 & Solution & 30.06 & 11 &  0.00 &  0.00\\
instance n=50 216.alb & 1 & 0 & Solution & 30.06 & 12 &  0.00 &  0.00\\
instance n=50 217.alb & 1 & 0 & Solution & 30.07 & 13 &  0.00 &  0.00\\
instance n=50 218.alb & 1 & 0 & Solution & 30.06 & 12 &  0.00 &  0.00\\
instance n=50 219.alb & 1 & 0 & Solution & 30.06 & 11 &  0.00 &  0.00\\
instance n=50 22.alb & 1 & 0 & Solution & 30.06 & 44 &  0.00 &  0.00\\
instance n=50 220.alb & 1 & 0 & Solution & 30.07 & 11 &  0.00 &  0.00\\
instance n=50 221.alb & 1 & 0 & Solution & 30.06 & 11 &  0.00 &  0.00\\
instance n=50 222.alb & 1 & 0 & Solution & 30.06 & 14 &  0.00 &  0.00\\
instance n=50 223.alb & 1 & 0 & Solution & 30.04 & 11 &  0.00 &  0.00\\
instance n=50 224.alb & 1 & 0 & Solution & 30.07 & 11 &  0.00 &  0.00\\
instance n=50 225.alb & 1 & 0 & Solution & 30.07 & 12 &  0.00 &  0.00\\
instance n=50 226.alb & 1 & 0 & Optimal &  0.77 & 7 &  0.00 &  0.00\\
instance n=50 227.alb & 1 & 0 & Optimal &  0.64 & 6 &  0.00 &  0.00\\
instance n=50 228.alb & 1 & 0 & Optimal &  1.39 & 6 &  0.00 &  0.00\\
instance n=50 229.alb & 1 & 0 & Optimal &  0.34 & 6 &  0.00 &  0.00\\
instance n=50 23.alb & 1 & 0 & Solution & 30.07 & 7 &  0.00 &  0.00\\
instance n=50 230.alb & 1 & 0 & Optimal &  0.92 & 7 &  0.00 &  0.00\\
instance n=50 231.alb & 1 & 0 & Optimal &  0.54 & 7 &  0.00 &  0.00\\
instance n=50 232.alb & 1 & 0 & Optimal &  0.62 & 7 &  0.00 &  0.00\\
instance n=50 233.alb & 1 & 0 & Optimal &  0.38 & 6 &  0.00 &  0.00\\
instance n=50 234.alb & 1 & 0 & Optimal &  1.46 & 8 &  0.00 &  0.00\\
instance n=50 235.alb & 1 & 0 & Optimal &  0.88 & 7 &  0.00 &  0.00\\
instance n=50 236.alb & 1 & 0 & Optimal &  1.04 & 7 &  0.00 &  0.00\\
instance n=50 237.alb & 1 & 0 & Optimal &  1.20 & 8 &  0.00 &  0.00\\
instance n=50 238.alb & 1 & 0 & Optimal &  0.91 & 7 &  0.00 &  0.00\\
instance n=50 239.alb & 1 & 0 & Optimal &  0.56 & 7 &  0.00 &  0.00\\
instance n=50 24.alb & 1 & 0 & Solution & 30.06 & 7 &  0.00 &  0.00\\
instance n=50 240.alb & 1 & 0 & Optimal &  0.66 & 7 &  0.00 &  0.00\\
instance n=50 241.alb & 1 & 0 & Optimal &  3.00 & 7 &  0.00 &  0.00\\
instance n=50 242.alb & 1 & 0 & Optimal &  0.58 & 8 &  0.00 &  0.00\\
instance n=50 243.alb & 1 & 0 & Optimal &  1.77 & 7 &  0.00 &  0.00\\
instance n=50 244.alb & 1 & 0 & Optimal &  0.39 & 7 &  0.00 &  0.00\\
instance n=50 245.alb & 1 & 0 & Optimal &  1.08 & 7 &  0.00 &  0.00\\
instance n=50 246.alb & 1 & 0 & Optimal &  1.37 & 8 &  0.00 &  0.00\\
instance n=50 247.alb & 1 & 0 & Optimal &  0.62 & 7 &  0.00 &  0.00\\
instance n=50 248.alb & 1 & 0 & Optimal &  0.71 & 7 &  0.00 &  0.00\\
instance n=50 249.alb & 1 & 0 & Optimal &  0.73 & 7 &  0.00 &  0.00\\
instance n=50 25.alb & 1 & 0 & Solution & 30.07 & 6 &  0.00 &  0.00\\
instance n=50 250.alb & 1 & 0 & Optimal &  0.82 & 7 &  0.00 &  0.00\\
instance n=50 251.alb & 1 & 0 & Optimal &  4.64 & 27 &  0.00 &  0.00\\
instance n=50 252.alb & 1 & 0 & Optimal &  4.19 & 32 &  0.00 &  0.00\\
instance n=50 253.alb & 1 & 0 & Optimal &  7.88 & 28 &  0.00 &  0.00\\
instance n=50 254.alb & 1 & 0 & Optimal & 11.32 & 30 &  0.00 &  0.00\\
instance n=50 255.alb & 1 & 0 & Optimal &  4.35 & 29 &  0.00 &  0.00\\
instance n=50 256.alb & 1 & 0 & Optimal & 25.77 & 30 &  0.00 &  0.00\\
instance n=50 257.alb & 1 & 0 & Optimal &  7.59 & 33 &  0.00 &  0.00\\
instance n=50 258.alb & 1 & 0 & Optimal &  6.89 & 28 &  0.00 &  0.00\\
instance n=50 259.alb & 1 & 0 & Optimal &  4.14 & 31 &  0.00 &  0.00\\
instance n=50 26.alb & 1 & 0 & Solution & 30.04 & 27 &  0.00 &  0.00\\
instance n=50 260.alb & 1 & 0 & Optimal &  2.99 & 29 &  0.00 &  0.00\\
instance n=50 261.alb & 1 & 0 & Optimal &  1.98 & 28 &  0.00 &  0.00\\
instance n=50 262.alb & 1 & 0 & Optimal &  1.69 & 31 &  0.00 &  0.00\\
instance n=50 263.alb & 1 & 0 & Optimal &  2.75 & 29 &  0.00 &  0.00\\
instance n=50 264.alb & 1 & 0 & Optimal & 12.04 & 27 &  0.00 &  0.00\\
instance n=50 265.alb & 1 & 0 & Optimal &  2.55 & 27 &  0.00 &  0.00\\
instance n=50 266.alb & 1 & 0 & Optimal & 15.62 & 29 &  0.00 &  0.00\\
instance n=50 267.alb & 1 & 0 & Optimal &  5.11 & 28 &  0.00 &  0.00\\
instance n=50 268.alb & 1 & 0 & Optimal &  9.37 & 29 &  0.00 &  0.00\\
instance n=50 269.alb & 1 & 0 & Optimal & 19.23 & 26 &  0.00 &  0.00\\
instance n=50 27.alb & 1 & 0 & Solution & 30.06 & 30 &  0.00 &  0.00\\
instance n=50 270.alb & 1 & 0 & Optimal &  9.09 & 28 &  0.00 &  0.00\\
instance n=50 271.alb & 1 & 0 & Optimal &  4.22 & 31 &  0.00 &  0.00\\
instance n=50 272.alb & 1 & 0 & Optimal &  1.99 & 27 &  0.00 &  0.00\\
instance n=50 273.alb & 1 & 0 & Optimal & 16.36 & 27 &  0.00 &  0.00\\
instance n=50 274.alb & 1 & 0 & Optimal &  2.11 & 29 &  0.00 &  0.00\\
instance n=50 275.alb & 1 & 0 & Optimal &  5.10 & 27 &  0.00 &  0.00\\
instance n=50 276.alb & 1 & 0 & Optimal &  0.67 & 12 &  0.00 &  0.00\\
instance n=50 277.alb & 1 & 0 & Optimal &  0.63 & 13 &  0.00 &  0.00\\
instance n=50 278.alb & 1 & 0 & Optimal &  1.56 & 12 &  0.00 &  0.00\\
instance n=50 279.alb & 1 & 0 & Optimal &  6.73 & 11 &  0.00 &  0.00\\
instance n=50 28.alb & 1 & 0 & Solution & 30.05 & 28 &  0.00 &  0.00\\
instance n=50 280.alb & 1 & 0 & Optimal &  0.80 & 13 &  0.00 &  0.00\\
instance n=50 281.alb & 1 & 0 & Optimal &  0.63 & 11 &  0.00 &  0.00\\
instance n=50 282.alb & 1 & 0 & Optimal &  5.70 & 12 &  0.00 &  0.00\\
instance n=50 283.alb & 1 & 0 & Optimal &  1.74 & 12 &  0.00 &  0.00\\
instance n=50 284.alb & 1 & 0 & Optimal &  1.73 & 11 &  0.00 &  0.00\\
instance n=50 285.alb & 1 & 0 & Optimal &  0.93 & 13 &  0.00 &  0.00\\
instance n=50 286.alb & 1 & 0 & Optimal &  1.43 & 11 &  0.00 &  0.00\\
instance n=50 287.alb & 1 & 0 & Optimal &  2.51 & 12 &  0.00 &  0.00\\
instance n=50 288.alb & 1 & 0 & Optimal &  0.98 & 10 &  0.00 &  0.00\\
instance n=50 289.alb & 1 & 0 & Optimal &  1.34 & 11 &  0.00 &  0.00\\
instance n=50 29.alb & 1 & 0 & Solution & 30.07 & 29 &  0.00 &  0.00\\
instance n=50 290.alb & 1 & 0 & Optimal &  1.61 & 14 &  0.00 &  0.00\\
instance n=50 291.alb & 1 & 0 & Optimal &  1.26 & 12 &  0.00 &  0.00\\
instance n=50 292.alb & 1 & 0 & Optimal &  1.52 & 13 &  0.00 &  0.00\\
instance n=50 293.alb & 1 & 0 & Optimal &  0.92 & 12 &  0.00 &  0.00\\
instance n=50 294.alb & 1 & 0 & Optimal &  1.20 & 13 &  0.00 &  0.00\\
instance n=50 295.alb & 1 & 0 & Optimal &  9.01 & 16 &  0.00 &  0.00\\
instance n=50 296.alb & 1 & 0 & Solution & 30.06 & 13 &  0.00 &  0.00\\
instance n=50 297.alb & 1 & 0 & Optimal &  4.57 & 13 &  0.00 &  0.00\\
instance n=50 298.alb & 1 & 0 & Optimal &  1.05 & 11 &  0.00 &  0.00\\
instance n=50 299.alb & 1 & 0 & Optimal &  1.79 & 12 &  0.00 &  0.00\\
instance n=50 3.alb & 1 & 0 & Solution & 30.06 & 8 &  0.00 &  0.00\\
instance n=50 30.alb & 1 & 0 & Solution & 30.06 & 27 &  0.00 &  0.00\\
instance n=50 300.alb & 1 & 0 & Optimal &  0.71 & 12 &  0.00 &  0.00\\
instance n=50 301.alb & 1 & 0 & Solution & 30.05 & 6 &  0.00 &  0.00\\
instance n=50 302.alb & 1 & 0 & Solution & 30.05 & 7 &  0.00 &  0.00\\
instance n=50 303.alb & 1 & 0 & Solution & 30.06 & 8 &  0.00 &  0.00\\
instance n=50 304.alb & 1 & 0 & Solution & 30.04 & 7 &  0.00 &  0.00\\
instance n=50 305.alb & 1 & 0 & Solution & 30.05 & 8 &  0.00 &  0.00\\
instance n=50 306.alb & 1 & 0 & Solution & 30.05 & 7 &  0.00 &  0.00\\
instance n=50 307.alb & 1 & 0 & Solution & 30.07 & 7 &  0.00 &  0.00\\
instance n=50 308.alb & 1 & 0 & Solution & 30.07 & 8 &  0.00 &  0.00\\
instance n=50 309.alb & 1 & 0 & Solution & 30.06 & 29 &  0.00 &  0.00\\
instance n=50 31.alb & 1 & 0 & Solution & 30.05 & 28 &  0.00 &  0.00\\
instance n=50 310.alb & 1 & 0 & Solution & 30.05 & 8 &  0.00 &  0.00\\
instance n=50 311.alb & 1 & 0 & Solution & 30.04 & 8 &  0.00 &  0.00\\
instance n=50 312.alb & 1 & 0 & Solution & 30.05 & 6 &  0.00 &  0.00\\
instance n=50 313.alb & 1 & 0 & Solution & 30.05 & 8 &  0.00 &  0.00\\
instance n=50 314.alb & 1 & 0 & Solution & 30.06 & 22 &  0.00 &  0.00\\
instance n=50 315.alb & 1 & 0 & Solution & 30.06 & 8 &  0.00 &  0.00\\
instance n=50 316.alb & 1 & 0 & Solution & 30.06 & 8 &  0.00 &  0.00\\
instance n=50 317.alb & 1 & 0 & Solution & 30.05 & 6 &  0.00 &  0.00\\
instance n=50 318.alb & 1 & 0 & Solution & 30.06 & 8 &  0.00 &  0.00\\
instance n=50 319.alb & 1 & 0 & Solution & 30.05 & 7 &  0.00 &  0.00\\
instance n=50 32.alb & 1 & 0 & Solution & 30.06 & 26 &  0.00 &  0.00\\
instance n=50 320.alb & 1 & 0 & Solution & 30.06 & 8 &  0.00 &  0.00\\
instance n=50 321.alb & 1 & 0 & Solution & 30.06 & 6 &  0.00 &  0.00\\
instance n=50 322.alb & 1 & 0 & Solution & 30.06 & 7 &  0.00 &  0.00\\
instance n=50 323.alb & 1 & 0 & Solution & 30.06 & 7 &  0.00 &  0.00\\
instance n=50 324.alb & 1 & 0 & Solution & 30.06 & 7 &  0.00 &  0.00\\
instance n=50 325.alb & 1 & 0 & Solution & 30.06 & 7 &  0.00 &  0.00\\
instance n=50 326.alb & 1 & 0 & Solution & 30.06 & 33 &  0.00 &  0.00\\
instance n=50 327.alb & 1 & 0 & Solution & 30.06 & 28 &  0.00 &  0.00\\
instance n=50 328.alb & 1 & 0 & Solution & 30.06 & 32 &  0.00 &  0.00\\
instance n=50 329.alb & 1 & 0 & Solution & 30.06 & 25 &  0.00 &  0.00\\
instance n=50 33.alb & 1 & 0 & Solution & 30.07 & 25 &  0.00 &  0.00\\
instance n=50 330.alb & 1 & 0 & Solution & 30.06 & 30 &  0.00 &  0.00\\
instance n=50 331.alb & 1 & 0 & Solution & 30.06 & 40 &  0.00 &  0.00\\
instance n=50 332.alb & 1 & 0 & Solution & 30.07 & 25 &  0.00 &  0.00\\
instance n=50 333.alb & 1 & 0 & Solution & 30.06 & 28 &  0.00 &  0.00\\
instance n=50 334.alb & 1 & 0 & Solution & 30.05 & 29 &  0.00 &  0.00\\
instance n=50 335.alb & 1 & 0 & Solution & 30.05 & 27 &  0.00 &  0.00\\
instance n=50 336.alb & 1 & 0 & Solution & 30.06 & 26 &  0.00 &  0.00\\
instance n=50 337.alb & 1 & 0 & Solution & 30.06 & 26 &  0.00 &  0.00\\
instance n=50 338.alb & 1 & 0 & Solution & 30.06 & 36 &  0.00 &  0.00\\
instance n=50 339.alb & 1 & 0 & Solution & 30.06 & 29 &  0.00 &  0.00\\
instance n=50 34.alb & 1 & 0 & Solution & 30.06 & 30 &  0.00 &  0.00\\
instance n=50 340.alb & 1 & 0 & Solution & 30.06 & 32 &  0.00 &  0.00\\
instance n=50 341.alb & 1 & 0 & Solution & 30.06 & 27 &  0.00 &  0.00\\
instance n=50 342.alb & 1 & 0 & Solution & 30.06 & 29 &  0.00 &  0.00\\
instance n=50 343.alb & 1 & 0 & Solution & 30.06 & 28 &  0.00 &  0.00\\
instance n=50 344.alb & 1 & 0 & Solution & 30.05 & 30 &  0.00 &  0.00\\
instance n=50 345.alb & 1 & 0 & Solution & 30.05 & 29 &  0.00 &  0.00\\
instance n=50 346.alb & 1 & 0 & Solution & 30.06 & 27 &  0.00 &  0.00\\
instance n=50 347.alb & 1 & 0 & Solution & 30.06 & 31 &  0.00 &  0.00\\
instance n=50 348.alb & 1 & 0 & Solution & 30.06 & 30 &  0.00 &  0.00\\
instance n=50 349.alb & 1 & 0 & Solution & 30.06 & 29 &  0.00 &  0.00\\
instance n=50 35.alb & 1 & 0 & Solution & 30.08 & 32 &  0.00 &  0.00\\
instance n=50 350.alb & 1 & 0 & Solution & 30.07 & 32 &  0.00 &  0.00\\
instance n=50 351.alb & 1 & 0 & Solution & 30.07 & 12 &  0.00 &  0.00\\
instance n=50 352.alb & 1 & 0 & Solution & 30.06 & 10 &  0.00 &  0.00\\
instance n=50 353.alb & 1 & 0 & Solution & 30.06 & 13 &  0.00 &  0.00\\
instance n=50 354.alb & 1 & 0 & Solution & 30.06 & 14 &  0.00 &  0.00\\
instance n=50 355.alb & 1 & 0 & Solution & 30.06 & 11 &  0.00 &  0.00\\
instance n=50 356.alb & 1 & 0 & Solution & 30.04 & 15 &  0.00 &  0.00\\
instance n=50 357.alb & 1 & 0 & Solution & 30.07 & 12 &  0.00 &  0.00\\
instance n=50 358.alb & 1 & 0 & Solution & 30.05 & 11 &  0.00 &  0.00\\
instance n=50 359.alb & 1 & 0 & Solution & 30.10 & 10 &  0.00 &  0.00\\
instance n=50 36.alb & 1 & 0 & Solution & 30.06 & 31 &  0.00 &  0.00\\
instance n=50 360.alb & 1 & 0 & Solution & 30.06 & 12 &  0.00 &  0.00\\
instance n=50 361.alb & 1 & 0 & Solution & 30.07 & 11 &  0.00 &  0.00\\
\end{longtable}



\section{Results for MiniZinc/CPSat}

\begin{longtable}{lrrlrrrr}
\caption{Results for SALBP-1 Problems (MiniZinc/CPSat) (1050 Instances)}\\\toprule
Name & \shortstack{Nr\\Jobs} & \shortstack{Nr\\Machines} & Status & Time & Makespan & Bound & \shortstack{Gap\\Percent}\\ \midrule
\endhead
\bottomrule
\endfoot
instance n=20 1.alb & 1 & 0 & Optimal &  0.29 & 3 &  0.00 &  0.00\\
instance n=20 10.alb & 1 & 0 & Optimal &  0.27 & 3 &  0.00 &  0.00\\
instance n=20 100.alb & 1 & 0 & Optimal &  0.30 & 11 &  0.00 &  0.00\\
instance n=20 101.alb & 1 & 0 & Optimal &  0.40 & 13 &  0.00 &  0.00\\
instance n=20 102.alb & 1 & 0 & Optimal &  0.28 & 13 &  0.00 &  0.00\\
instance n=20 103.alb & 1 & 0 & Optimal &  0.33 & 12 &  0.00 &  0.00\\
instance n=20 104.alb & 1 & 0 & Optimal &  0.28 & 11 &  0.00 &  0.00\\
instance n=20 105.alb & 1 & 0 & Optimal &  0.26 & 12 &  0.00 &  0.00\\
instance n=20 106.alb & 1 & 0 & Optimal &  0.29 & 10 &  0.00 &  0.00\\
instance n=20 107.alb & 1 & 0 & Optimal &  0.28 & 14 &  0.00 &  0.00\\
instance n=20 108.alb & 1 & 0 & Optimal &  0.28 & 15 &  0.00 &  0.00\\
instance n=20 109.alb & 1 & 0 & Optimal &  0.28 & 12 &  0.00 &  0.00\\
instance n=20 11.alb & 1 & 0 & Optimal &  0.26 & 3 &  0.00 &  0.00\\
instance n=20 110.alb & 1 & 0 & Optimal &  0.29 & 11 &  0.00 &  0.00\\
instance n=20 111.alb & 1 & 0 & Optimal &  0.30 & 13 &  0.00 &  0.00\\
instance n=20 112.alb & 1 & 0 & Optimal &  0.28 & 11 &  0.00 &  0.00\\
instance n=20 113.alb & 1 & 0 & Optimal &  0.27 & 12 &  0.00 &  0.00\\
instance n=20 114.alb & 1 & 0 & Optimal &  0.30 & 13 &  0.00 &  0.00\\
instance n=20 115.alb & 1 & 0 & Optimal &  0.27 & 11 &  0.00 &  0.00\\
instance n=20 116.alb & 1 & 0 & Optimal &  0.28 & 5 &  0.00 &  0.00\\
instance n=20 117.alb & 1 & 0 & Optimal &  0.27 & 5 &  0.00 &  0.00\\
instance n=20 118.alb & 1 & 0 & Optimal &  0.28 & 5 &  0.00 &  0.00\\
instance n=20 119.alb & 1 & 0 & Optimal &  0.27 & 6 &  0.00 &  0.00\\
instance n=20 12.alb & 1 & 0 & Optimal &  0.29 & 3 &  0.00 &  0.00\\
instance n=20 120.alb & 1 & 0 & Optimal &  0.27 & 6 &  0.00 &  0.00\\
instance n=20 121.alb & 1 & 0 & Optimal &  0.26 & 5 &  0.00 &  0.00\\
instance n=20 122.alb & 1 & 0 & Optimal &  0.29 & 6 &  0.00 &  0.00\\
instance n=20 123.alb & 1 & 0 & Optimal &  0.27 & 5 &  0.00 &  0.00\\
instance n=20 124.alb & 1 & 0 & Optimal &  0.27 & 5 &  0.00 &  0.00\\
instance n=20 125.alb & 1 & 0 & Optimal &  0.27 & 5 &  0.00 &  0.00\\
instance n=20 126.alb & 1 & 0 & Optimal &  0.27 & 5 &  0.00 &  0.00\\
instance n=20 127.alb & 1 & 0 & Optimal &  0.29 & 4 &  0.00 &  0.00\\
instance n=20 128.alb & 1 & 0 & Optimal &  0.27 & 5 &  0.00 &  0.00\\
instance n=20 129.alb & 1 & 0 & Optimal &  0.28 & 5 &  0.00 &  0.00\\
instance n=20 13.alb & 1 & 0 & Optimal &  0.28 & 3 &  0.00 &  0.00\\
instance n=20 130.alb & 1 & 0 & Optimal &  0.28 & 6 &  0.00 &  0.00\\
instance n=20 131.alb & 1 & 0 & Optimal &  0.32 & 7 &  0.00 &  0.00\\
instance n=20 132.alb & 1 & 0 & Optimal &  0.27 & 4 &  0.00 &  0.00\\
instance n=20 133.alb & 1 & 0 & Optimal &  0.25 & 5 &  0.00 &  0.00\\
instance n=20 134.alb & 1 & 0 & Optimal &  0.30 & 6 &  0.00 &  0.00\\
instance n=20 135.alb & 1 & 0 & Optimal &  0.30 & 6 &  0.00 &  0.00\\
instance n=20 136.alb & 1 & 0 & Optimal &  0.27 & 6 &  0.00 &  0.00\\
instance n=20 137.alb & 1 & 0 & Optimal &  0.28 & 5 &  0.00 &  0.00\\
instance n=20 138.alb & 1 & 0 & Optimal &  0.27 & 5 &  0.00 &  0.00\\
instance n=20 139.alb & 1 & 0 & Optimal &  0.28 & 5 &  0.00 &  0.00\\
instance n=20 14.alb & 1 & 0 & Optimal &  0.29 & 3 &  0.00 &  0.00\\
instance n=20 140.alb & 1 & 0 & Optimal &  0.27 & 5 &  0.00 &  0.00\\
instance n=20 141.alb & 1 & 0 & Optimal &  0.27 & 3 &  0.00 &  0.00\\
instance n=20 142.alb & 1 & 0 & Optimal &  0.29 & 3 &  0.00 &  0.00\\
instance n=20 143.alb & 1 & 0 & Optimal &  0.27 & 3 &  0.00 &  0.00\\
instance n=20 144.alb & 1 & 0 & Optimal &  0.27 & 4 &  0.00 &  0.00\\
instance n=20 145.alb & 1 & 0 & Optimal &  0.30 & 3 &  0.00 &  0.00\\
instance n=20 146.alb & 1 & 0 & Optimal &  0.27 & 3 &  0.00 &  0.00\\
instance n=20 147.alb & 1 & 0 & Optimal &  0.27 & 3 &  0.00 &  0.00\\
instance n=20 148.alb & 1 & 0 & Optimal &  0.27 & 3 &  0.00 &  0.00\\
instance n=20 149.alb & 1 & 0 & Optimal &  0.28 & 3 &  0.00 &  0.00\\
instance n=20 15.alb & 1 & 0 & Optimal &  0.27 & 3 &  0.00 &  0.00\\
instance n=20 150.alb & 1 & 0 & Optimal &  0.29 & 3 &  0.00 &  0.00\\
instance n=20 151.alb & 1 & 0 & Optimal &  0.28 & 3 &  0.00 &  0.00\\
instance n=20 152.alb & 1 & 0 & Optimal &  0.44 & 3 &  0.00 &  0.00\\
instance n=20 153.alb & 1 & 0 & Optimal &  0.28 & 3 &  0.00 &  0.00\\
instance n=20 154.alb & 1 & 0 & Optimal &  0.27 & 3 &  0.00 &  0.00\\
instance n=20 155.alb & 1 & 0 & Optimal &  0.27 & 3 &  0.00 &  0.00\\
instance n=20 156.alb & 1 & 0 & Optimal &  0.29 & 3 &  0.00 &  0.00\\
instance n=20 157.alb & 1 & 0 & Optimal &  0.27 & 3 &  0.00 &  0.00\\
instance n=20 158.alb & 1 & 0 & Optimal &  0.29 & 3 &  0.00 &  0.00\\
instance n=20 159.alb & 1 & 0 & Optimal &  0.30 & 3 &  0.00 &  0.00\\
instance n=20 16.alb & 1 & 0 & Optimal &  0.29 & 12 &  0.00 &  0.00\\
instance n=20 160.alb & 1 & 0 & Optimal &  0.27 & 3 &  0.00 &  0.00\\
instance n=20 161.alb & 1 & 0 & Optimal &  0.32 & 3 &  0.00 &  0.00\\
instance n=20 162.alb & 1 & 0 & Optimal &  0.27 & 3 &  0.00 &  0.00\\
instance n=20 163.alb & 1 & 0 & Optimal &  0.26 & 3 &  0.00 &  0.00\\
instance n=20 164.alb & 1 & 0 & Optimal &  0.46 & 4 &  0.00 &  0.00\\
instance n=20 165.alb & 1 & 0 & Optimal &  0.27 & 3 &  0.00 &  0.00\\
instance n=20 166.alb & 1 & 0 & Optimal &  0.27 & 12 &  0.00 &  0.00\\
instance n=20 167.alb & 1 & 0 & Optimal &  0.29 & 11 &  0.00 &  0.00\\
instance n=20 168.alb & 1 & 0 & Optimal &  0.27 & 10 &  0.00 &  0.00\\
instance n=20 169.alb & 1 & 0 & Optimal &  0.29 & 11 &  0.00 &  0.00\\
instance n=20 17.alb & 1 & 0 & Optimal &  0.32 & 10 &  0.00 &  0.00\\
instance n=20 170.alb & 1 & 0 & Optimal &  0.30 & 11 &  0.00 &  0.00\\
instance n=20 171.alb & 1 & 0 & Optimal &  0.30 & 13 &  0.00 &  0.00\\
instance n=20 172.alb & 1 & 0 & Optimal &  0.29 & 11 &  0.00 &  0.00\\
instance n=20 173.alb & 1 & 0 & Optimal &  0.29 & 11 &  0.00 &  0.00\\
instance n=20 174.alb & 1 & 0 & Optimal &  0.27 & 12 &  0.00 &  0.00\\
instance n=20 175.alb & 1 & 0 & Optimal &  0.30 & 10 &  0.00 &  0.00\\
instance n=20 176.alb & 1 & 0 & Optimal &  0.29 & 11 &  0.00 &  0.00\\
instance n=20 177.alb & 1 & 0 & Optimal &  0.58 & 10 &  0.00 &  0.00\\
instance n=20 178.alb & 1 & 0 & Optimal &  0.28 & 11 &  0.00 &  0.00\\
instance n=20 179.alb & 1 & 0 & Optimal &  0.27 & 11 &  0.00 &  0.00\\
instance n=20 18.alb & 1 & 0 & Optimal &  0.29 & 11 &  0.00 &  0.00\\
instance n=20 180.alb & 1 & 0 & Optimal &  0.27 & 13 &  0.00 &  0.00\\
instance n=20 181.alb & 1 & 0 & Optimal &  0.27 & 11 &  0.00 &  0.00\\
instance n=20 182.alb & 1 & 0 & Optimal &  0.27 & 11 &  0.00 &  0.00\\
instance n=20 183.alb & 1 & 0 & Optimal &  0.29 & 13 &  0.00 &  0.00\\
instance n=20 184.alb & 1 & 0 & Optimal &  0.27 & 12 &  0.00 &  0.00\\
instance n=20 185.alb & 1 & 0 & Optimal &  0.27 & 15 &  0.00 &  0.00\\
instance n=20 186.alb & 1 & 0 & Optimal &  0.82 & 14 &  0.00 &  0.00\\
instance n=20 187.alb & 1 & 0 & Optimal &  0.33 & 10 &  0.00 &  0.00\\
instance n=20 188.alb & 1 & 0 & Optimal &  0.29 & 11 &  0.00 &  0.00\\
instance n=20 189.alb & 1 & 0 & Optimal &  0.29 & 13 &  0.00 &  0.00\\
instance n=20 19.alb & 1 & 0 & Optimal &  0.32 & 14 &  0.00 &  0.00\\
instance n=20 190.alb & 1 & 0 & Optimal &  0.32 & 15 &  0.00 &  0.00\\
instance n=20 191.alb & 1 & 0 & Optimal &  0.28 & 4 &  0.00 &  0.00\\
instance n=20 192.alb & 1 & 0 & Optimal &  0.29 & 5 &  0.00 &  0.00\\
instance n=20 193.alb & 1 & 0 & Optimal &  0.28 & 5 &  0.00 &  0.00\\
instance n=20 194.alb & 1 & 0 & Optimal &  0.27 & 6 &  0.00 &  0.00\\
instance n=20 195.alb & 1 & 0 & Optimal &  0.31 & 6 &  0.00 &  0.00\\
instance n=20 196.alb & 1 & 0 & Optimal &  0.32 & 5 &  0.00 &  0.00\\
instance n=20 197.alb & 1 & 0 & Optimal &  0.47 & 4 &  0.00 &  0.00\\
instance n=20 198.alb & 1 & 0 & Optimal &  0.32 & 6 &  0.00 &  0.00\\
instance n=20 199.alb & 1 & 0 & Optimal &  0.29 & 5 &  0.00 &  0.00\\
instance n=20 2.alb & 1 & 0 & Optimal &  0.34 & 3 &  0.00 &  0.00\\
instance n=20 20.alb & 1 & 0 & Optimal &  0.29 & 11 &  0.00 &  0.00\\
instance n=20 200.alb & 1 & 0 & Optimal &  0.28 & 6 &  0.00 &  0.00\\
instance n=20 201.alb & 1 & 0 & Optimal &  0.30 & 6 &  0.00 &  0.00\\
instance n=20 202.alb & 1 & 0 & Optimal &  0.29 & 4 &  0.00 &  0.00\\
instance n=20 203.alb & 1 & 0 & Optimal &  0.26 & 4 &  0.00 &  0.00\\
instance n=20 204.alb & 1 & 0 & Optimal &  0.31 & 5 &  0.00 &  0.00\\
instance n=20 205.alb & 1 & 0 & Optimal &  0.28 & 6 &  0.00 &  0.00\\
instance n=20 206.alb & 1 & 0 & Optimal &  0.29 & 5 &  0.00 &  0.00\\
instance n=20 207.alb & 1 & 0 & Optimal &  0.29 & 6 &  0.00 &  0.00\\
instance n=20 208.alb & 1 & 0 & Optimal &  0.28 & 5 &  0.00 &  0.00\\
instance n=20 209.alb & 1 & 0 & Optimal &  0.27 & 4 &  0.00 &  0.00\\
instance n=20 21.alb & 1 & 0 & Optimal &  0.32 & 14 &  0.00 &  0.00\\
instance n=20 210.alb & 1 & 0 & Optimal &  0.28 & 5 &  0.00 &  0.00\\
instance n=20 211.alb & 1 & 0 & Optimal &  0.28 & 5 &  0.00 &  0.00\\
instance n=20 212.alb & 1 & 0 & Optimal &  0.29 & 5 &  0.00 &  0.00\\
instance n=20 213.alb & 1 & 0 & Optimal &  0.26 & 5 &  0.00 &  0.00\\
instance n=20 214.alb & 1 & 0 & Optimal &  0.29 & 5 &  0.00 &  0.00\\
instance n=20 215.alb & 1 & 0 & Optimal &  0.27 & 5 &  0.00 &  0.00\\
instance n=20 216.alb & 1 & 0 & Optimal &  0.27 & 3 &  0.00 &  0.00\\
instance n=20 217.alb & 1 & 0 & Optimal &  0.28 & 4 &  0.00 &  0.00\\
instance n=20 218.alb & 1 & 0 & Optimal &  0.30 & 3 &  0.00 &  0.00\\
instance n=20 219.alb & 1 & 0 & Optimal &  0.29 & 3 &  0.00 &  0.00\\
instance n=20 22.alb & 1 & 0 & Optimal &  0.30 & 12 &  0.00 &  0.00\\
instance n=20 220.alb & 1 & 0 & Optimal &  0.31 & 3 &  0.00 &  0.00\\
instance n=20 221.alb & 1 & 0 & Optimal &  0.27 & 3 &  0.00 &  0.00\\
instance n=20 222.alb & 1 & 0 & Optimal &  0.27 & 3 &  0.00 &  0.00\\
instance n=20 223.alb & 1 & 0 & Optimal &  0.29 & 3 &  0.00 &  0.00\\
instance n=20 224.alb & 1 & 0 & Optimal &  0.27 & 3 &  0.00 &  0.00\\
instance n=20 225.alb & 1 & 0 & Optimal &  0.27 & 3 &  0.00 &  0.00\\
instance n=20 226.alb & 1 & 0 & Optimal &  0.27 & 3 &  0.00 &  0.00\\
instance n=20 227.alb & 1 & 0 & Optimal &  0.28 & 3 &  0.00 &  0.00\\
instance n=20 228.alb & 1 & 0 & Optimal &  0.26 & 2 &  0.00 &  0.00\\
instance n=20 229.alb & 1 & 0 & Optimal &  0.29 & 3 &  0.00 &  0.00\\
instance n=20 23.alb & 1 & 0 & Optimal &  0.30 & 13 &  0.00 &  0.00\\
instance n=20 230.alb & 1 & 0 & Optimal &  0.27 & 3 &  0.00 &  0.00\\
instance n=20 231.alb & 1 & 0 & Optimal &  0.41 & 3 &  0.00 &  0.00\\
instance n=20 232.alb & 1 & 0 & Optimal &  0.27 & 3 &  0.00 &  0.00\\
instance n=20 233.alb & 1 & 0 & Optimal &  0.27 & 3 &  0.00 &  0.00\\
instance n=20 234.alb & 1 & 0 & Optimal &  0.30 & 3 &  0.00 &  0.00\\
instance n=20 235.alb & 1 & 0 & Optimal &  0.27 & 3 &  0.00 &  0.00\\
instance n=20 236.alb & 1 & 0 & Optimal &  0.27 & 3 &  0.00 &  0.00\\
instance n=20 237.alb & 1 & 0 & Optimal &  0.44 & 3 &  0.00 &  0.00\\
instance n=20 238.alb & 1 & 0 & Optimal &  0.28 & 3 &  0.00 &  0.00\\
instance n=20 239.alb & 1 & 0 & Optimal &  0.26 & 3 &  0.00 &  0.00\\
instance n=20 24.alb & 1 & 0 & Optimal &  0.29 & 11 &  0.00 &  0.00\\
instance n=20 240.alb & 1 & 0 & Optimal &  0.26 & 3 &  0.00 &  0.00\\
instance n=20 241.alb & 1 & 0 & Optimal &  0.29 & 13 &  0.00 &  0.00\\
instance n=20 242.alb & 1 & 0 & Optimal &  0.29 & 12 &  0.00 &  0.00\\
instance n=20 243.alb & 1 & 0 & Optimal &  0.29 & 10 &  0.00 &  0.00\\
instance n=20 244.alb & 1 & 0 & Optimal &  0.29 & 11 &  0.00 &  0.00\\
instance n=20 245.alb & 1 & 0 & Optimal &  0.30 & 13 &  0.00 &  0.00\\
instance n=20 246.alb & 1 & 0 & Optimal &  0.27 & 13 &  0.00 &  0.00\\
instance n=20 247.alb & 1 & 0 & Optimal &  0.29 & 11 &  0.00 &  0.00\\
instance n=20 248.alb & 1 & 0 & Optimal &  0.29 & 11 &  0.00 &  0.00\\
instance n=20 249.alb & 1 & 0 & Optimal &  0.27 & 13 &  0.00 &  0.00\\
instance n=20 25.alb & 1 & 0 & Optimal &  0.35 & 11 &  0.00 &  0.00\\
instance n=20 250.alb & 1 & 0 & Optimal &  0.29 & 10 &  0.00 &  0.00\\
instance n=20 251.alb & 1 & 0 & Optimal &  0.30 & 12 &  0.00 &  0.00\\
instance n=20 252.alb & 1 & 0 & Optimal &  0.29 & 11 &  0.00 &  0.00\\
instance n=20 253.alb & 1 & 0 & Optimal &  0.29 & 13 &  0.00 &  0.00\\
instance n=20 254.alb & 1 & 0 & Optimal &  0.29 & 12 &  0.00 &  0.00\\
instance n=20 255.alb & 1 & 0 & Optimal &  0.28 & 13 &  0.00 &  0.00\\
instance n=20 256.alb & 1 & 0 & Optimal &  0.30 & 14 &  0.00 &  0.00\\
instance n=20 257.alb & 1 & 0 & Optimal &  0.28 & 10 &  0.00 &  0.00\\
instance n=20 258.alb & 1 & 0 & Optimal &  0.30 & 13 &  0.00 &  0.00\\
instance n=20 259.alb & 1 & 0 & Optimal &  0.29 & 13 &  0.00 &  0.00\\
instance n=20 26.alb & 1 & 0 & Optimal &  0.27 & 12 &  0.00 &  0.00\\
instance n=20 260.alb & 1 & 0 & Optimal &  0.27 & 12 &  0.00 &  0.00\\
instance n=20 261.alb & 1 & 0 & Optimal &  0.30 & 12 &  0.00 &  0.00\\
instance n=20 262.alb & 1 & 0 & Optimal &  0.28 & 11 &  0.00 &  0.00\\
instance n=20 263.alb & 1 & 0 & Optimal &  0.28 & 12 &  0.00 &  0.00\\
instance n=20 264.alb & 1 & 0 & Optimal &  0.34 & 12 &  0.00 &  0.00\\
instance n=20 265.alb & 1 & 0 & Optimal &  0.30 & 12 &  0.00 &  0.00\\
instance n=20 266.alb & 1 & 0 & Optimal &  0.27 & 5 &  0.00 &  0.00\\
instance n=20 267.alb & 1 & 0 & Optimal &  0.30 & 6 &  0.00 &  0.00\\
instance n=20 268.alb & 1 & 0 & Optimal &  0.28 & 6 &  0.00 &  0.00\\
instance n=20 269.alb & 1 & 0 & Optimal &  0.30 & 7 &  0.00 &  0.00\\
instance n=20 27.alb & 1 & 0 & Optimal &  0.30 & 13 &  0.00 &  0.00\\
instance n=20 270.alb & 1 & 0 & Optimal &  0.27 & 7 &  0.00 &  0.00\\
instance n=20 271.alb & 1 & 0 & Optimal &  0.29 & 6 &  0.00 &  0.00\\
instance n=20 272.alb & 1 & 0 & Optimal &  0.27 & 5 &  0.00 &  0.00\\
instance n=20 273.alb & 1 & 0 & Optimal &  0.28 & 5 &  0.00 &  0.00\\
instance n=20 274.alb & 1 & 0 & Optimal &  0.28 & 6 &  0.00 &  0.00\\
instance n=20 275.alb & 1 & 0 & Optimal &  0.28 & 5 &  0.00 &  0.00\\
instance n=20 276.alb & 1 & 0 & Optimal &  0.30 & 4 &  0.00 &  0.00\\
instance n=20 277.alb & 1 & 0 & Optimal &  0.29 & 4 &  0.00 &  0.00\\
instance n=20 278.alb & 1 & 0 & Optimal &  0.30 & 6 &  0.00 &  0.00\\
instance n=20 279.alb & 1 & 0 & Optimal &  0.29 & 6 &  0.00 &  0.00\\
instance n=20 28.alb & 1 & 0 & Optimal &  0.29 & 12 &  0.00 &  0.00\\
instance n=20 280.alb & 1 & 0 & Optimal &  0.33 & 5 &  0.00 &  0.00\\
instance n=20 281.alb & 1 & 0 & Optimal &  0.29 & 4 &  0.00 &  0.00\\
instance n=20 282.alb & 1 & 0 & Optimal &  0.28 & 4 &  0.00 &  0.00\\
instance n=20 283.alb & 1 & 0 & Optimal &  0.30 & 5 &  0.00 &  0.00\\
instance n=20 284.alb & 1 & 0 & Optimal &  0.30 & 5 &  0.00 &  0.00\\
instance n=20 285.alb & 1 & 0 & Optimal &  0.28 & 5 &  0.00 &  0.00\\
instance n=20 286.alb & 1 & 0 & Optimal &  0.30 & 5 &  0.00 &  0.00\\
instance n=20 287.alb & 1 & 0 & Optimal &  0.30 & 5 &  0.00 &  0.00\\
instance n=20 288.alb & 1 & 0 & Optimal &  0.44 & 6 &  0.00 &  0.00\\
instance n=20 289.alb & 1 & 0 & Optimal &  0.31 & 5 &  0.00 &  0.00\\
instance n=20 29.alb & 1 & 0 & Optimal &  0.28 & 10 &  0.00 &  0.00\\
instance n=20 290.alb & 1 & 0 & Optimal &  0.28 & 5 &  0.00 &  0.00\\
instance n=20 291.alb & 1 & 0 & Optimal &  0.30 & 3 &  0.00 &  0.00\\
instance n=20 292.alb & 1 & 0 & Optimal &  0.28 & 3 &  0.00 &  0.00\\
instance n=20 293.alb & 1 & 0 & Optimal &  0.28 & 3 &  0.00 &  0.00\\
instance n=20 294.alb & 1 & 0 & Optimal &  0.29 & 3 &  0.00 &  0.00\\
instance n=20 295.alb & 1 & 0 & Optimal &  0.28 & 3 &  0.00 &  0.00\\
instance n=20 296.alb & 1 & 0 & Optimal &  0.28 & 3 &  0.00 &  0.00\\
instance n=20 297.alb & 1 & 0 & Optimal &  0.30 & 3 &  0.00 &  0.00\\
instance n=20 298.alb & 1 & 0 & Optimal &  0.28 & 3 &  0.00 &  0.00\\
instance n=20 299.alb & 1 & 0 & Optimal &  0.30 & 3 &  0.00 &  0.00\\
instance n=20 3.alb & 1 & 0 & Optimal &  0.30 & 3 &  0.00 &  0.00\\
instance n=20 30.alb & 1 & 0 & Optimal &  0.29 & 16 &  0.00 &  0.00\\
instance n=20 300.alb & 1 & 0 & Optimal &  0.29 & 4 &  0.00 &  0.00\\
instance n=20 301.alb & 1 & 0 & Optimal &  0.29 & 3 &  0.00 &  0.00\\
instance n=20 302.alb & 1 & 0 & Optimal &  0.29 & 3 &  0.00 &  0.00\\
instance n=20 303.alb & 1 & 0 & Optimal &  0.27 & 3 &  0.00 &  0.00\\
instance n=20 304.alb & 1 & 0 & Optimal &  0.29 & 3 &  0.00 &  0.00\\
instance n=20 305.alb & 1 & 0 & Optimal &  0.27 & 3 &  0.00 &  0.00\\
instance n=20 306.alb & 1 & 0 & Optimal &  0.27 & 3 &  0.00 &  0.00\\
instance n=20 307.alb & 1 & 0 & Optimal &  0.30 & 3 &  0.00 &  0.00\\
instance n=20 308.alb & 1 & 0 & Optimal &  0.27 & 3 &  0.00 &  0.00\\
instance n=20 309.alb & 1 & 0 & Optimal &  0.29 & 3 &  0.00 &  0.00\\
instance n=20 31.alb & 1 & 0 & Optimal &  0.30 & 12 &  0.00 &  0.00\\
instance n=20 310.alb & 1 & 0 & Optimal &  0.28 & 3 &  0.00 &  0.00\\
instance n=20 311.alb & 1 & 0 & Optimal &  0.27 & 3 &  0.00 &  0.00\\
instance n=20 312.alb & 1 & 0 & Optimal &  0.28 & 4 &  0.00 &  0.00\\
instance n=20 313.alb & 1 & 0 & Optimal &  0.29 & 3 &  0.00 &  0.00\\
instance n=20 314.alb & 1 & 0 & Optimal &  0.27 & 3 &  0.00 &  0.00\\
instance n=20 315.alb & 1 & 0 & Optimal &  0.29 & 3 &  0.00 &  0.00\\
instance n=20 316.alb & 1 & 0 & Optimal &  0.31 & 10 &  0.00 &  0.00\\
instance n=20 317.alb & 1 & 0 & Optimal &  0.30 & 10 &  0.00 &  0.00\\
instance n=20 318.alb & 1 & 0 & Optimal &  0.30 & 10 &  0.00 &  0.00\\
instance n=20 319.alb & 1 & 0 & Optimal &  0.29 & 14 &  0.00 &  0.00\\
instance n=20 32.alb & 1 & 0 & Optimal &  0.30 & 13 &  0.00 &  0.00\\
instance n=20 320.alb & 1 & 0 & Optimal &  0.29 & 12 &  0.00 &  0.00\\
instance n=20 321.alb & 1 & 0 & Optimal &  0.53 & 14 &  0.00 &  0.00\\
instance n=20 322.alb & 1 & 0 & Optimal &  0.27 & 12 &  0.00 &  0.00\\
instance n=20 323.alb & 1 & 0 & Optimal &  0.28 & 13 &  0.00 &  0.00\\
instance n=20 324.alb & 1 & 0 & Optimal &  0.29 & 9 &  0.00 &  0.00\\
instance n=20 325.alb & 1 & 0 & Optimal &  0.29 & 14 &  0.00 &  0.00\\
instance n=20 326.alb & 1 & 0 & Optimal &  0.52 & 14 &  0.00 &  0.00\\
instance n=20 327.alb & 1 & 0 & Optimal &  1.83 & 13 &  0.00 &  0.00\\
instance n=20 328.alb & 1 & 0 & Optimal &  0.28 & 13 &  0.00 &  0.00\\
instance n=20 329.alb & 1 & 0 & Optimal &  0.27 & 10 &  0.00 &  0.00\\
instance n=20 33.alb & 1 & 0 & Optimal &  0.29 & 11 &  0.00 &  0.00\\
instance n=20 330.alb & 1 & 0 & Optimal &  0.30 & 12 &  0.00 &  0.00\\
instance n=20 331.alb & 1 & 0 & Optimal &  0.32 & 13 &  0.00 &  0.00\\
instance n=20 332.alb & 1 & 0 & Optimal &  0.27 & 13 &  0.00 &  0.00\\
instance n=20 333.alb & 1 & 0 & Optimal &  0.30 & 11 &  0.00 &  0.00\\
instance n=20 334.alb & 1 & 0 & Optimal &  0.30 & 10 &  0.00 &  0.00\\
instance n=20 335.alb & 1 & 0 & Optimal &  0.27 & 14 &  0.00 &  0.00\\
instance n=20 336.alb & 1 & 0 & Optimal &  0.29 & 11 &  0.00 &  0.00\\
instance n=20 337.alb & 1 & 0 & Optimal &  0.29 & 10 &  0.00 &  0.00\\
instance n=20 338.alb & 1 & 0 & Optimal &  0.27 & 14 &  0.00 &  0.00\\
instance n=20 339.alb & 1 & 0 & Optimal &  0.29 & 13 &  0.00 &  0.00\\
instance n=20 34.alb & 1 & 0 & Optimal &  0.30 & 12 &  0.00 &  0.00\\
instance n=20 340.alb & 1 & 0 & Optimal &  0.32 & 11 &  0.00 &  0.00\\
instance n=20 341.alb & 1 & 0 & Optimal &  0.31 & 6 &  0.00 &  0.00\\
instance n=20 342.alb & 1 & 0 & Optimal &  0.30 & 6 &  0.00 &  0.00\\
instance n=20 343.alb & 1 & 0 & Optimal &  0.27 & 6 &  0.00 &  0.00\\
instance n=20 344.alb & 1 & 0 & Optimal &  0.31 & 6 &  0.00 &  0.00\\
instance n=20 345.alb & 1 & 0 & Optimal &  0.27 & 4 &  0.00 &  0.00\\
instance n=20 346.alb & 1 & 0 & Optimal &  0.27 & 5 &  0.00 &  0.00\\
instance n=20 347.alb & 1 & 0 & Optimal &  0.27 & 6 &  0.00 &  0.00\\
instance n=20 348.alb & 1 & 0 & Optimal &  0.27 & 5 &  0.00 &  0.00\\
instance n=20 349.alb & 1 & 0 & Optimal &  0.27 & 5 &  0.00 &  0.00\\
instance n=20 35.alb & 1 & 0 & Optimal &  0.29 & 12 &  0.00 &  0.00\\
instance n=20 350.alb & 1 & 0 & Optimal &  0.28 & 5 &  0.00 &  0.00\\
instance n=20 351.alb & 1 & 0 & Optimal &  0.27 & 5 &  0.00 &  0.00\\
instance n=20 352.alb & 1 & 0 & Optimal &  0.29 & 4 &  0.00 &  0.00\\
instance n=20 353.alb & 1 & 0 & Optimal &  0.30 & 6 &  0.00 &  0.00\\
instance n=20 354.alb & 1 & 0 & Optimal &  0.29 & 6 &  0.00 &  0.00\\
instance n=20 355.alb & 1 & 0 & Optimal &  0.29 & 5 &  0.00 &  0.00\\
instance n=20 356.alb & 1 & 0 & Optimal &  0.27 & 5 &  0.00 &  0.00\\
instance n=20 357.alb & 1 & 0 & Optimal &  0.27 & 5 &  0.00 &  0.00\\
instance n=20 358.alb & 1 & 0 & Optimal &  0.29 & 4 &  0.00 &  0.00\\
instance n=20 359.alb & 1 & 0 & Optimal &  0.27 & 4 &  0.00 &  0.00\\
instance n=20 36.alb & 1 & 0 & Optimal &  0.29 & 13 &  0.00 &  0.00\\
instance n=20 360.alb & 1 & 0 & Optimal &  0.29 & 6 &  0.00 &  0.00\\
instance n=20 361.alb & 1 & 0 & Optimal &  0.27 & 5 &  0.00 &  0.00\\
instance n=20 362.alb & 1 & 0 & Optimal &  0.27 & 5 &  0.00 &  0.00\\
instance n=20 363.alb & 1 & 0 & Optimal &  0.27 & 7 &  0.00 &  0.00\\
instance n=20 364.alb & 1 & 0 & Optimal &  0.28 & 4 &  0.00 &  0.00\\
instance n=20 365.alb & 1 & 0 & Optimal &  0.29 & 5 &  0.00 &  0.00\\
instance n=20 366.alb & 1 & 0 & Optimal &  0.26 & 3 &  0.00 &  0.00\\
instance n=20 367.alb & 1 & 0 & Optimal &  0.30 & 3 &  0.00 &  0.00\\
instance n=20 368.alb & 1 & 0 & Optimal &  0.29 & 3 &  0.00 &  0.00\\
instance n=20 369.alb & 1 & 0 & Optimal &  0.29 & 3 &  0.00 &  0.00\\
instance n=20 37.alb & 1 & 0 & Optimal &  0.29 & 12 &  0.00 &  0.00\\
instance n=20 370.alb & 1 & 0 & Optimal &  0.27 & 3 &  0.00 &  0.00\\
instance n=20 371.alb & 1 & 0 & Optimal &  0.28 & 3 &  0.00 &  0.00\\
instance n=20 372.alb & 1 & 0 & Optimal &  0.30 & 3 &  0.00 &  0.00\\
instance n=20 373.alb & 1 & 0 & Optimal &  0.27 & 3 &  0.00 &  0.00\\
instance n=20 374.alb & 1 & 0 & Optimal &  0.29 & 3 &  0.00 &  0.00\\
instance n=20 375.alb & 1 & 0 & Optimal &  0.30 & 3 &  0.00 &  0.00\\
instance n=20 376.alb & 1 & 0 & Optimal &  0.27 & 3 &  0.00 &  0.00\\
instance n=20 377.alb & 1 & 0 & Optimal &  0.27 & 3 &  0.00 &  0.00\\
instance n=20 378.alb & 1 & 0 & Optimal &  0.30 & 3 &  0.00 &  0.00\\
instance n=20 379.alb & 1 & 0 & Optimal &  0.29 & 4 &  0.00 &  0.00\\
instance n=20 38.alb & 1 & 0 & Optimal &  0.27 & 12 &  0.00 &  0.00\\
instance n=20 380.alb & 1 & 0 & Optimal &  0.29 & 3 &  0.00 &  0.00\\
instance n=20 381.alb & 1 & 0 & Optimal &  0.28 & 3 &  0.00 &  0.00\\
instance n=20 382.alb & 1 & 0 & Optimal &  0.29 & 4 &  0.00 &  0.00\\
instance n=20 383.alb & 1 & 0 & Optimal &  0.28 & 3 &  0.00 &  0.00\\
instance n=20 384.alb & 1 & 0 & Optimal &  0.28 & 3 &  0.00 &  0.00\\
instance n=20 385.alb & 1 & 0 & Optimal &  0.26 & 3 &  0.00 &  0.00\\
instance n=20 386.alb & 1 & 0 & Optimal &  0.30 & 3 &  0.00 &  0.00\\
instance n=20 387.alb & 1 & 0 & Optimal &  0.28 & 3 &  0.00 &  0.00\\
instance n=20 388.alb & 1 & 0 & Optimal &  0.29 & 3 &  0.00 &  0.00\\
instance n=20 389.alb & 1 & 0 & Optimal &  0.30 & 3 &  0.00 &  0.00\\
instance n=20 39.alb & 1 & 0 & Optimal &  0.28 & 13 &  0.00 &  0.00\\
instance n=20 390.alb & 1 & 0 & Optimal &  0.27 & 3 &  0.00 &  0.00\\
instance n=20 391.alb & 1 & 0 & Optimal &  0.30 & 11 &  0.00 &  0.00\\
instance n=20 392.alb & 1 & 0 & Optimal &  0.30 & 14 &  0.00 &  0.00\\
instance n=20 393.alb & 1 & 0 & Optimal &  0.29 & 11 &  0.00 &  0.00\\
instance n=20 394.alb & 1 & 0 & Optimal &  0.35 & 12 &  0.00 &  0.00\\
instance n=20 395.alb & 1 & 0 & Optimal &  0.28 & 12 &  0.00 &  0.00\\
instance n=20 396.alb & 1 & 0 & Optimal &  0.29 & 13 &  0.00 &  0.00\\
instance n=20 397.alb & 1 & 0 & Optimal &  0.33 & 10 &  0.00 &  0.00\\
instance n=20 398.alb & 1 & 0 & Optimal &  0.28 & 11 &  0.00 &  0.00\\
instance n=20 399.alb & 1 & 0 & Optimal &  0.27 & 13 &  0.00 &  0.00\\
instance n=20 4.alb & 1 & 0 & Optimal &  0.30 & 3 &  0.00 &  0.00\\
instance n=20 40.alb & 1 & 0 & Optimal &  0.30 & 12 &  0.00 &  0.00\\
instance n=20 400.alb & 1 & 0 & Optimal &  0.29 & 12 &  0.00 &  0.00\\
instance n=20 401.alb & 1 & 0 & Optimal &  0.31 & 12 &  0.00 &  0.00\\
instance n=20 402.alb & 1 & 0 & Optimal &  0.29 & 12 &  0.00 &  0.00\\
instance n=20 403.alb & 1 & 0 & Optimal &  0.29 & 12 &  0.00 &  0.00\\
instance n=20 404.alb & 1 & 0 & Optimal &  0.31 & 10 &  0.00 &  0.00\\
instance n=20 405.alb & 1 & 0 & Optimal &  0.30 & 12 &  0.00 &  0.00\\
instance n=20 406.alb & 1 & 0 & Optimal &  0.25 & 14 &  0.00 &  0.00\\
instance n=20 407.alb & 1 & 0 & Optimal &  0.30 & 10 &  0.00 &  0.00\\
instance n=20 408.alb & 1 & 0 & Optimal &  0.29 & 14 &  0.00 &  0.00\\
instance n=20 409.alb & 1 & 0 & Optimal &  0.29 & 12 &  0.00 &  0.00\\
instance n=20 41.alb & 1 & 0 & Optimal &  0.29 & 6 &  0.00 &  0.00\\
instance n=20 410.alb & 1 & 0 & Optimal &  0.30 & 11 &  0.00 &  0.00\\
instance n=20 411.alb & 1 & 0 & Optimal &  0.30 & 15 &  0.00 &  0.00\\
instance n=20 412.alb & 1 & 0 & Optimal &  0.30 & 11 &  0.00 &  0.00\\
instance n=20 413.alb & 1 & 0 & Optimal &  0.28 & 10 &  0.00 &  0.00\\
instance n=20 414.alb & 1 & 0 & Optimal &  0.32 & 12 &  0.00 &  0.00\\
instance n=20 415.alb & 1 & 0 & Optimal &  0.30 & 10 &  0.00 &  0.00\\
instance n=20 416.alb & 1 & 0 & Optimal &  0.29 & 6 &  0.00 &  0.00\\
instance n=20 417.alb & 1 & 0 & Optimal &  0.25 & 5 &  0.00 &  0.00\\
instance n=20 418.alb & 1 & 0 & Optimal &  0.29 & 6 &  0.00 &  0.00\\
instance n=20 419.alb & 1 & 0 & Optimal &  0.29 & 4 &  0.00 &  0.00\\
instance n=20 42.alb & 1 & 0 & Optimal &  0.27 & 5 &  0.00 &  0.00\\
instance n=20 420.alb & 1 & 0 & Optimal &  0.29 & 5 &  0.00 &  0.00\\
instance n=20 421.alb & 1 & 0 & Optimal &  0.30 & 6 &  0.00 &  0.00\\
instance n=20 422.alb & 1 & 0 & Optimal &  0.27 & 4 &  0.00 &  0.00\\
instance n=20 423.alb & 1 & 0 & Optimal &  0.29 & 6 &  0.00 &  0.00\\
instance n=20 424.alb & 1 & 0 & Optimal &  0.28 & 5 &  0.00 &  0.00\\
instance n=20 425.alb & 1 & 0 & Optimal &  0.30 & 6 &  0.00 &  0.00\\
instance n=20 426.alb & 1 & 0 & Optimal &  0.30 & 5 &  0.00 &  0.00\\
instance n=20 427.alb & 1 & 0 & Optimal &  0.27 & 6 &  0.00 &  0.00\\
instance n=20 428.alb & 1 & 0 & Optimal &  0.27 & 5 &  0.00 &  0.00\\
instance n=20 429.alb & 1 & 0 & Optimal &  0.29 & 4 &  0.00 &  0.00\\
instance n=20 43.alb & 1 & 0 & Optimal &  0.29 & 5 &  0.00 &  0.00\\
instance n=20 430.alb & 1 & 0 & Optimal &  0.27 & 5 &  0.00 &  0.00\\
instance n=20 431.alb & 1 & 0 & Optimal &  0.29 & 6 &  0.00 &  0.00\\
instance n=20 432.alb & 1 & 0 & Optimal &  0.29 & 5 &  0.00 &  0.00\\
instance n=20 433.alb & 1 & 0 & Optimal &  0.27 & 5 &  0.00 &  0.00\\
instance n=20 434.alb & 1 & 0 & Optimal &  0.27 & 5 &  0.00 &  0.00\\
instance n=20 435.alb & 1 & 0 & Optimal &  0.28 & 7 &  0.00 &  0.00\\
instance n=20 436.alb & 1 & 0 & Optimal &  0.27 & 5 &  0.00 &  0.00\\
instance n=20 437.alb & 1 & 0 & Optimal &  0.46 & 5 &  0.00 &  0.00\\
instance n=20 438.alb & 1 & 0 & Optimal &  0.27 & 6 &  0.00 &  0.00\\
instance n=20 439.alb & 1 & 0 & Optimal &  0.27 & 5 &  0.00 &  0.00\\
instance n=20 44.alb & 1 & 0 & Optimal &  0.28 & 5 &  0.00 &  0.00\\
instance n=20 440.alb & 1 & 0 & Optimal &  0.28 & 5 &  0.00 &  0.00\\
instance n=20 441.alb & 1 & 0 & Optimal &  0.29 & 3 &  0.00 &  0.00\\
instance n=20 442.alb & 1 & 0 & Optimal &  0.29 & 3 &  0.00 &  0.00\\
instance n=20 443.alb & 1 & 0 & Optimal &  0.29 & 3 &  0.00 &  0.00\\
instance n=20 444.alb & 1 & 0 & Optimal &  0.29 & 3 &  0.00 &  0.00\\
instance n=20 445.alb & 1 & 0 & Optimal &  0.28 & 3 &  0.00 &  0.00\\
instance n=20 446.alb & 1 & 0 & Optimal &  0.27 & 3 &  0.00 &  0.00\\
instance n=20 447.alb & 1 & 0 & Optimal &  0.27 & 3 &  0.00 &  0.00\\
instance n=20 448.alb & 1 & 0 & Optimal &  0.27 & 3 &  0.00 &  0.00\\
instance n=20 449.alb & 1 & 0 & Optimal &  0.28 & 3 &  0.00 &  0.00\\
instance n=20 45.alb & 1 & 0 & Optimal &  0.27 & 6 &  0.00 &  0.00\\
instance n=20 450.alb & 1 & 0 & Optimal &  0.27 & 3 &  0.00 &  0.00\\
instance n=20 451.alb & 1 & 0 & Optimal &  0.30 & 3 &  0.00 &  0.00\\
instance n=20 452.alb & 1 & 0 & Optimal &  0.27 & 3 &  0.00 &  0.00\\
instance n=20 453.alb & 1 & 0 & Optimal &  0.27 & 3 &  0.00 &  0.00\\
instance n=20 454.alb & 1 & 0 & Optimal &  0.30 & 3 &  0.00 &  0.00\\
instance n=20 455.alb & 1 & 0 & Optimal &  0.26 & 3 &  0.00 &  0.00\\
instance n=20 456.alb & 1 & 0 & Optimal &  0.27 & 4 &  0.00 &  0.00\\
instance n=20 457.alb & 1 & 0 & Optimal &  0.29 & 3 &  0.00 &  0.00\\
instance n=20 458.alb & 1 & 0 & Optimal &  0.28 & 3 &  0.00 &  0.00\\
instance n=20 459.alb & 1 & 0 & Optimal &  0.27 & 3 &  0.00 &  0.00\\
instance n=20 46.alb & 1 & 0 & Optimal &  0.27 & 4 &  0.00 &  0.00\\
instance n=20 460.alb & 1 & 0 & Optimal &  0.26 & 3 &  0.00 &  0.00\\
instance n=20 461.alb & 1 & 0 & Optimal &  0.26 & 3 &  0.00 &  0.00\\
instance n=20 462.alb & 1 & 0 & Optimal &  0.29 & 3 &  0.00 &  0.00\\
instance n=20 463.alb & 1 & 0 & Optimal &  0.32 & 3 &  0.00 &  0.00\\
instance n=20 464.alb & 1 & 0 & Optimal &  0.26 & 3 &  0.00 &  0.00\\
instance n=20 465.alb & 1 & 0 & Optimal &  0.27 & 3 &  0.00 &  0.00\\
instance n=20 466.alb & 1 & 0 & Optimal &  0.27 & 13 &  0.00 &  0.00\\
instance n=20 467.alb & 1 & 0 & Optimal &  0.26 & 14 &  0.00 &  0.00\\
instance n=20 468.alb & 1 & 0 & Optimal &  0.27 & 13 &  0.00 &  0.00\\
instance n=20 469.alb & 1 & 0 & Optimal &  0.30 & 14 &  0.00 &  0.00\\
instance n=20 47.alb & 1 & 0 & Optimal &  0.26 & 4 &  0.00 &  0.00\\
instance n=20 470.alb & 1 & 0 & Optimal &  0.27 & 12 &  0.00 &  0.00\\
instance n=20 471.alb & 1 & 0 & Optimal &  0.30 & 12 &  0.00 &  0.00\\
instance n=20 472.alb & 1 & 0 & Optimal &  0.27 & 13 &  0.00 &  0.00\\
instance n=20 473.alb & 1 & 0 & Optimal &  0.27 & 10 &  0.00 &  0.00\\
instance n=20 474.alb & 1 & 0 & Optimal &  0.29 & 14 &  0.00 &  0.00\\
instance n=20 475.alb & 1 & 0 & Optimal &  0.27 & 11 &  0.00 &  0.00\\
instance n=20 476.alb & 1 & 0 & Optimal &  0.27 & 11 &  0.00 &  0.00\\
instance n=20 477.alb & 1 & 0 & Optimal &  0.29 & 11 &  0.00 &  0.00\\
instance n=20 478.alb & 1 & 0 & Optimal &  0.27 & 12 &  0.00 &  0.00\\
instance n=20 479.alb & 1 & 0 & Optimal &  0.27 & 13 &  0.00 &  0.00\\
instance n=20 48.alb & 1 & 0 & Optimal &  0.27 & 5 &  0.00 &  0.00\\
instance n=20 480.alb & 1 & 0 & Optimal &  0.28 & 13 &  0.00 &  0.00\\
instance n=20 481.alb & 1 & 0 & Optimal &  0.27 & 13 &  0.00 &  0.00\\
instance n=20 482.alb & 1 & 0 & Optimal &  0.27 & 13 &  0.00 &  0.00\\
instance n=20 483.alb & 1 & 0 & Optimal &  0.28 & 12 &  0.00 &  0.00\\
instance n=20 484.alb & 1 & 0 & Optimal &  0.27 & 13 &  0.00 &  0.00\\
instance n=20 485.alb & 1 & 0 & Optimal &  0.27 & 15 &  0.00 &  0.00\\
instance n=20 486.alb & 1 & 0 & Optimal &  0.27 & 11 &  0.00 &  0.00\\
instance n=20 487.alb & 1 & 0 & Optimal &  0.27 & 12 &  0.00 &  0.00\\
instance n=20 488.alb & 1 & 0 & Optimal &  0.27 & 15 &  0.00 &  0.00\\
instance n=20 489.alb & 1 & 0 & Optimal &  0.29 & 12 &  0.00 &  0.00\\
instance n=20 49.alb & 1 & 0 & Optimal &  0.26 & 4 &  0.00 &  0.00\\
instance n=20 490.alb & 1 & 0 & Optimal &  0.27 & 12 &  0.00 &  0.00\\
instance n=20 491.alb & 1 & 0 & Optimal &  0.27 & 6 &  0.00 &  0.00\\
instance n=20 492.alb & 1 & 0 & Optimal &  0.27 & 5 &  0.00 &  0.00\\
instance n=20 493.alb & 1 & 0 & Optimal &  0.27 & 5 &  0.00 &  0.00\\
instance n=20 494.alb & 1 & 0 & Optimal &  0.27 & 6 &  0.00 &  0.00\\
instance n=20 495.alb & 1 & 0 & Optimal &  0.26 & 6 &  0.00 &  0.00\\
instance n=20 496.alb & 1 & 0 & Optimal &  0.27 & 5 &  0.00 &  0.00\\
instance n=20 497.alb & 1 & 0 & Optimal &  0.29 & 6 &  0.00 &  0.00\\
instance n=20 498.alb & 1 & 0 & Optimal &  0.27 & 6 &  0.00 &  0.00\\
instance n=20 499.alb & 1 & 0 & Optimal &  0.27 & 5 &  0.00 &  0.00\\
instance n=20 5.alb & 1 & 0 & Optimal &  0.30 & 3 &  0.00 &  0.00\\
instance n=20 50.alb & 1 & 0 & Optimal &  0.27 & 4 &  0.00 &  0.00\\
instance n=20 500.alb & 1 & 0 & Optimal &  0.26 & 8 &  0.00 &  0.00\\
instance n=20 501.alb & 1 & 0 & Optimal &  0.27 & 5 &  0.00 &  0.00\\
instance n=20 502.alb & 1 & 0 & Optimal &  0.28 & 4 &  0.00 &  0.00\\
instance n=20 503.alb & 1 & 0 & Optimal &  0.26 & 6 &  0.00 &  0.00\\
instance n=20 504.alb & 1 & 0 & Optimal &  0.27 & 6 &  0.00 &  0.00\\
instance n=20 505.alb & 1 & 0 & Optimal &  0.30 & 6 &  0.00 &  0.00\\
instance n=20 506.alb & 1 & 0 & Optimal &  0.28 & 5 &  0.00 &  0.00\\
instance n=20 507.alb & 1 & 0 & Optimal &  0.28 & 5 &  0.00 &  0.00\\
instance n=20 508.alb & 1 & 0 & Optimal &  0.27 & 5 &  0.00 &  0.00\\
instance n=20 509.alb & 1 & 0 & Optimal &  0.26 & 4 &  0.00 &  0.00\\
instance n=20 51.alb & 1 & 0 & Optimal &  0.29 & 4 &  0.00 &  0.00\\
instance n=20 510.alb & 1 & 0 & Optimal &  0.29 & 5 &  0.00 &  0.00\\
instance n=20 511.alb & 1 & 0 & Optimal &  0.27 & 5 &  0.00 &  0.00\\
instance n=20 512.alb & 1 & 0 & Optimal &  0.27 & 5 &  0.00 &  0.00\\
instance n=20 513.alb & 1 & 0 & Optimal &  0.30 & 5 &  0.00 &  0.00\\
instance n=20 514.alb & 1 & 0 & Optimal &  0.27 & 5 &  0.00 &  0.00\\
instance n=20 515.alb & 1 & 0 & Optimal &  0.27 & 6 &  0.00 &  0.00\\
instance n=20 516.alb & 1 & 0 & Optimal &  0.27 & 3 &  0.00 &  0.00\\
instance n=20 517.alb & 1 & 0 & Optimal &  0.26 & 3 &  0.00 &  0.00\\
instance n=20 518.alb & 1 & 0 & Optimal &  0.27 & 3 &  0.00 &  0.00\\
instance n=20 519.alb & 1 & 0 & Optimal &  0.46 & 3 &  0.00 &  0.00\\
instance n=20 52.alb & 1 & 0 & Optimal &  0.27 & 4 &  0.00 &  0.00\\
instance n=20 520.alb & 1 & 0 & Optimal &  0.29 & 3 &  0.00 &  0.00\\
instance n=20 521.alb & 1 & 0 & Optimal &  0.30 & 3 &  0.00 &  0.00\\
instance n=20 522.alb & 1 & 0 & Optimal &  0.28 & 3 &  0.00 &  0.00\\
instance n=20 523.alb & 1 & 0 & Optimal &  0.26 & 3 &  0.00 &  0.00\\
instance n=20 524.alb & 1 & 0 & Optimal &  0.30 & 3 &  0.00 &  0.00\\
instance n=20 525.alb & 1 & 0 & Optimal &  0.26 & 3 &  0.00 &  0.00\\
instance n=20 53.alb & 1 & 0 & Optimal &  0.27 & 5 &  0.00 &  0.00\\
instance n=20 54.alb & 1 & 0 & Optimal &  0.30 & 5 &  0.00 &  0.00\\
instance n=20 55.alb & 1 & 0 & Optimal &  0.27 & 5 &  0.00 &  0.00\\
instance n=20 56.alb & 1 & 0 & Optimal &  0.27 & 4 &  0.00 &  0.00\\
instance n=20 57.alb & 1 & 0 & Optimal &  0.27 & 4 &  0.00 &  0.00\\
instance n=20 58.alb & 1 & 0 & Optimal &  0.27 & 5 &  0.00 &  0.00\\
instance n=20 59.alb & 1 & 0 & Optimal &  0.26 & 4 &  0.00 &  0.00\\
instance n=20 6.alb & 1 & 0 & Optimal &  0.27 & 3 &  0.00 &  0.00\\
instance n=20 60.alb & 1 & 0 & Optimal &  0.28 & 6 &  0.00 &  0.00\\
instance n=20 61.alb & 1 & 0 & Optimal &  0.27 & 7 &  0.00 &  0.00\\
instance n=20 62.alb & 1 & 0 & Optimal &  0.27 & 5 &  0.00 &  0.00\\
instance n=20 63.alb & 1 & 0 & Optimal &  0.28 & 5 &  0.00 &  0.00\\
instance n=20 64.alb & 1 & 0 & Optimal &  0.26 & 5 &  0.00 &  0.00\\
instance n=20 65.alb & 1 & 0 & Optimal &  0.27 & 5 &  0.00 &  0.00\\
instance n=20 66.alb & 1 & 0 & Optimal &  0.29 & 3 &  0.00 &  0.00\\
instance n=20 67.alb & 1 & 0 & Optimal &  0.27 & 3 &  0.00 &  0.00\\
instance n=20 68.alb & 1 & 0 & Optimal &  0.27 & 3 &  0.00 &  0.00\\
instance n=20 69.alb & 1 & 0 & Optimal &  0.29 & 2 &  0.00 &  0.00\\
instance n=20 7.alb & 1 & 0 & Optimal &  0.27 & 3 &  0.00 &  0.00\\
instance n=20 70.alb & 1 & 0 & Optimal &  0.26 & 3 &  0.00 &  0.00\\
instance n=20 71.alb & 1 & 0 & Optimal &  0.29 & 3 &  0.00 &  0.00\\
instance n=20 72.alb & 1 & 0 & Optimal &  0.27 & 3 &  0.00 &  0.00\\
instance n=20 73.alb & 1 & 0 & Optimal &  0.27 & 2 &  0.00 &  0.00\\
instance n=20 74.alb & 1 & 0 & Optimal &  0.27 & 3 &  0.00 &  0.00\\
instance n=20 75.alb & 1 & 0 & Optimal &  0.28 & 3 &  0.00 &  0.00\\
instance n=20 76.alb & 1 & 0 & Optimal &  0.27 & 3 &  0.00 &  0.00\\
instance n=20 77.alb & 1 & 0 & Optimal &  0.27 & 3 &  0.00 &  0.00\\
instance n=20 78.alb & 1 & 0 & Optimal &  0.29 & 3 &  0.00 &  0.00\\
instance n=20 79.alb & 1 & 0 & Optimal &  0.27 & 3 &  0.00 &  0.00\\
instance n=20 8.alb & 1 & 0 & Optimal &  0.44 & 3 &  0.00 &  0.00\\
instance n=20 80.alb & 1 & 0 & Optimal &  0.29 & 3 &  0.00 &  0.00\\
instance n=20 81.alb & 1 & 0 & Optimal &  0.26 & 3 &  0.00 &  0.00\\
instance n=20 82.alb & 1 & 0 & Optimal &  0.27 & 4 &  0.00 &  0.00\\
instance n=20 83.alb & 1 & 0 & Optimal &  0.27 & 3 &  0.00 &  0.00\\
instance n=20 84.alb & 1 & 0 & Optimal &  0.27 & 3 &  0.00 &  0.00\\
instance n=20 85.alb & 1 & 0 & Optimal &  0.27 & 3 &  0.00 &  0.00\\
instance n=20 86.alb & 1 & 0 & Optimal &  0.29 & 3 &  0.00 &  0.00\\
instance n=20 87.alb & 1 & 0 & Optimal &  0.25 & 3 &  0.00 &  0.00\\
instance n=20 88.alb & 1 & 0 & Optimal &  0.27 & 3 &  0.00 &  0.00\\
instance n=20 89.alb & 1 & 0 & Optimal &  0.28 & 3 &  0.00 &  0.00\\
instance n=20 9.alb & 1 & 0 & Optimal &  0.26 & 3 &  0.00 &  0.00\\
instance n=20 90.alb & 1 & 0 & Optimal &  0.27 & 3 &  0.00 &  0.00\\
instance n=20 91.alb & 1 & 0 & Optimal &  0.29 & 11 &  0.00 &  0.00\\
instance n=20 92.alb & 1 & 0 & Optimal &  0.26 & 11 &  0.00 &  0.00\\
instance n=20 93.alb & 1 & 0 & Optimal &  0.27 & 13 &  0.00 &  0.00\\
instance n=20 94.alb & 1 & 0 & Optimal &  0.29 & 10 &  0.00 &  0.00\\
instance n=20 95.alb & 1 & 0 & Optimal &  0.27 & 12 &  0.00 &  0.00\\
instance n=20 96.alb & 1 & 0 & Optimal &  0.27 & 10 &  0.00 &  0.00\\
instance n=20 97.alb & 1 & 0 & Optimal &  0.32 & 15 &  0.00 &  0.00\\
instance n=20 98.alb & 1 & 0 & Optimal &  0.27 & 13 &  0.00 &  0.00\\
instance n=20 99.alb & 1 & 0 & Optimal &  0.30 & 12 &  0.00 &  0.00\\
instance n=50 1.alb & 1 & 0 & Optimal &  0.29 & 8 &  0.00 &  0.00\\
instance n=50 10.alb & 1 & 0 & Optimal &  0.27 & 7 &  0.00 &  0.00\\
instance n=50 100.alb & 1 & 0 & Optimal &  0.26 & 7 &  0.00 &  0.00\\
instance n=50 101.alb & 1 & 0 & Optimal & 13.72 & 30 &  0.00 &  0.00\\
instance n=50 102.alb & 1 & 0 & Optimal & 41.12 & 32 &  0.00 &  0.00\\
instance n=50 103.alb & 1 & 0 & Optimal &  0.37 & 29 &  0.00 &  0.00\\
instance n=50 104.alb & 1 & 0 & Optimal &  1.05 & 27 &  0.00 &  0.00\\
instance n=50 105.alb & 1 & 0 & Optimal & 20.12 & 24 &  0.00 &  0.00\\
instance n=50 106.alb & 1 & 0 & Optimal &  8.64 & 28 &  0.00 &  0.00\\
instance n=50 107.alb & 1 & 0 & Optimal &  1.97 & 28 &  0.00 &  0.00\\
instance n=50 108.alb & 1 & 0 & Optimal &  0.55 & 30 &  0.00 &  0.00\\
instance n=50 109.alb & 1 & 0 & Optimal &  0.36 & 30 &  0.00 &  0.00\\
instance n=50 11.alb & 1 & 0 & Optimal &  0.29 & 7 &  0.00 &  0.00\\
instance n=50 110.alb & 1 & 0 & Optimal &  0.64 & 26 &  0.00 &  0.00\\
instance n=50 111.alb & 1 & 0 & Optimal &  0.40 & 28 &  0.00 &  0.00\\
instance n=50 112.alb & 1 & 0 & Optimal &  1.45 & 27 &  0.00 &  0.00\\
instance n=50 113.alb & 1 & 0 & Optimal &  7.24 & 28 &  0.00 &  0.00\\
instance n=50 114.alb & 1 & 0 & Optimal &  0.97 & 27 &  0.00 &  0.00\\
instance n=50 116.alb & 1 & 0 & Optimal &  0.63 & 32 &  0.00 &  0.00\\
instance n=50 117.alb & 1 & 0 & Optimal & 13.43 & 27 &  0.00 &  0.00\\
instance n=50 118.alb & 1 & 0 & Optimal &  0.37 & 29 &  0.00 &  0.00\\
instance n=50 119.alb & 1 & 0 & Optimal &  0.40 & 25 &  0.00 &  0.00\\
instance n=50 12.alb & 1 & 0 & Optimal &  0.27 & 6 &  0.00 &  0.00\\
instance n=50 120.alb & 1 & 0 & Optimal &  0.39 & 27 &  0.00 &  0.00\\
instance n=50 121.alb & 1 & 0 & Optimal &  8.68 & 32 &  0.00 &  0.00\\
instance n=50 122.alb & 1 & 0 & Optimal & 23.93 & 29 &  0.00 &  0.00\\
instance n=50 123.alb & 1 & 0 & Optimal &  0.47 & 32 &  0.00 &  0.00\\
instance n=50 124.alb & 1 & 0 & Optimal &  1.41 & 29 &  0.00 &  0.00\\
instance n=50 125.alb & 1 & 0 & Optimal &  0.33 & 33 &  0.00 &  0.00\\
instance n=50 126.alb & 1 & 0 & Optimal &  0.28 & 12 &  0.00 &  0.00\\
instance n=50 127.alb & 1 & 0 & Optimal &  0.31 & 14 &  0.00 &  0.00\\
instance n=50 128.alb & 1 & 0 & Optimal &  0.34 & 12 &  0.00 &  0.00\\
instance n=50 129.alb & 1 & 0 & Optimal &  0.29 & 13 &  0.00 &  0.00\\
instance n=50 13.alb & 1 & 0 & Optimal &  0.28 & 6 &  0.00 &  0.00\\
instance n=50 130.alb & 1 & 0 & Optimal &  0.30 & 13 &  0.00 &  0.00\\
instance n=50 131.alb & 1 & 0 & Optimal &  0.28 & 12 &  0.00 &  0.00\\
instance n=50 132.alb & 1 & 0 & Optimal &  0.54 & 12 &  0.00 &  0.00\\
instance n=50 133.alb & 1 & 0 & Optimal &  0.31 & 12 &  0.00 &  0.00\\
instance n=50 134.alb & 1 & 0 & Optimal &  0.33 & 14 &  0.00 &  0.00\\
instance n=50 135.alb & 1 & 0 & Optimal &  0.35 & 13 &  0.00 &  0.00\\
instance n=50 136.alb & 1 & 0 & Optimal &  0.27 & 11 &  0.00 &  0.00\\
instance n=50 137.alb & 1 & 0 & Optimal &  0.28 & 11 &  0.00 &  0.00\\
instance n=50 138.alb & 1 & 0 & Optimal &  0.28 & 12 &  0.00 &  0.00\\
instance n=50 139.alb & 1 & 0 & Optimal &  6.70 & 11 &  0.00 &  0.00\\
instance n=50 14.alb & 1 & 0 & Optimal &  0.30 & 7 &  0.00 &  0.00\\
instance n=50 140.alb & 1 & 0 & Optimal &  0.30 & 12 &  0.00 &  0.00\\
instance n=50 141.alb & 1 & 0 & Optimal &  0.32 & 13 &  0.00 &  0.00\\
instance n=50 142.alb & 1 & 0 & Optimal &  0.29 & 11 &  0.00 &  0.00\\
instance n=50 143.alb & 1 & 0 & Optimal &  0.27 & 12 &  0.00 &  0.00\\
instance n=50 144.alb & 1 & 0 & Optimal &  0.30 & 13 &  0.00 &  0.00\\
instance n=50 145.alb & 1 & 0 & Optimal &  0.26 & 10 &  0.00 &  0.00\\
instance n=50 146.alb & 1 & 0 & Optimal &  0.28 & 13 &  0.00 &  0.00\\
instance n=50 147.alb & 1 & 0 & Optimal &  0.32 & 13 &  0.00 &  0.00\\
instance n=50 148.alb & 1 & 0 & Optimal &  0.29 & 10 &  0.00 &  0.00\\
instance n=50 149.alb & 1 & 0 & Optimal &  0.31 & 12 &  0.00 &  0.00\\
instance n=50 15.alb & 1 & 0 & Optimal &  0.28 & 8 &  0.00 &  0.00\\
instance n=50 150.alb & 1 & 0 & Optimal &  0.28 & 11 &  0.00 &  0.00\\
instance n=50 151.alb & 1 & 0 & Optimal &  0.28 & 7 &  0.00 &  0.00\\
instance n=50 152.alb & 1 & 0 & Optimal &  0.29 & 7 &  0.00 &  0.00\\
instance n=50 153.alb & 1 & 0 & Optimal &  1.86 & 7 &  0.00 &  0.00\\
instance n=50 154.alb & 1 & 0 & Optimal &  0.29 & 8 &  0.00 &  0.00\\
instance n=50 155.alb & 1 & 0 & Optimal &  0.27 & 7 &  0.00 &  0.00\\
instance n=50 156.alb & 1 & 0 & Optimal &  0.27 & 7 &  0.00 &  0.00\\
instance n=50 157.alb & 1 & 0 & Optimal &  0.46 & 8 &  0.00 &  0.00\\
instance n=50 158.alb & 1 & 0 & Optimal &  0.28 & 7 &  0.00 &  0.00\\
instance n=50 159.alb & 1 & 0 & Optimal &  0.29 & 7 &  0.00 &  0.00\\
instance n=50 16.alb & 1 & 0 & Optimal &  0.30 & 8 &  0.00 &  0.00\\
instance n=50 160.alb & 1 & 0 & Optimal &  0.27 & 8 &  0.00 &  0.00\\
instance n=50 161.alb & 1 & 0 & Optimal &  0.27 & 7 &  0.00 &  0.00\\
instance n=50 162.alb & 1 & 0 & Optimal &  0.28 & 8 &  0.00 &  0.00\\
instance n=50 163.alb & 1 & 0 & Optimal &  0.26 & 7 &  0.00 &  0.00\\
instance n=50 164.alb & 1 & 0 & Optimal &  0.28 & 7 &  0.00 &  0.00\\
instance n=50 165.alb & 1 & 0 & Optimal &  0.29 & 8 &  0.00 &  0.00\\
instance n=50 166.alb & 1 & 0 & Optimal &  0.28 & 8 &  0.00 &  0.00\\
instance n=50 167.alb & 1 & 0 & Optimal &  0.86 & 7 &  0.00 &  0.00\\
instance n=50 168.alb & 1 & 0 & Optimal &  0.42 & 8 &  0.00 &  0.00\\
instance n=50 169.alb & 1 & 0 & Optimal &  0.29 & 8 &  0.00 &  0.00\\
instance n=50 17.alb & 1 & 0 & Optimal &  0.31 & 7 &  0.00 &  0.00\\
instance n=50 170.alb & 1 & 0 & Optimal &  0.34 & 7 &  0.00 &  0.00\\
instance n=50 171.alb & 1 & 0 & Optimal &  0.28 & 8 &  0.00 &  0.00\\
instance n=50 172.alb & 1 & 0 & Optimal &  0.26 & 7 &  0.00 &  0.00\\
instance n=50 173.alb & 1 & 0 & Optimal &  0.30 & 7 &  0.00 &  0.00\\
instance n=50 174.alb & 1 & 0 & Optimal &  0.27 & 7 &  0.00 &  0.00\\
instance n=50 175.alb & 1 & 0 & Optimal &  0.27 & 7 &  0.00 &  0.00\\
instance n=50 176.alb & 1 & 0 & Optimal &  1.21 & 27 &  0.00 &  0.00\\
instance n=50 179.alb & 1 & 0 & Optimal & 19.92 & 26 &  0.00 &  0.00\\
instance n=50 18.alb & 1 & 0 & Optimal &  0.30 & 7 &  0.00 &  0.00\\
instance n=50 180.alb & 1 & 0 & Optimal &  1.18 & 26 &  0.00 &  0.00\\
instance n=50 181.alb & 1 & 0 & Optimal &  7.20 & 29 &  0.00 &  0.00\\
instance n=50 183.alb & 1 & 0 & Optimal &  0.92 & 28 &  0.00 &  0.00\\
instance n=50 184.alb & 1 & 0 & Optimal &  0.29 & 38 &  0.00 &  0.00\\
instance n=50 185.alb & 1 & 0 & Optimal &  0.94 & 26 &  0.00 &  0.00\\
instance n=50 186.alb & 1 & 0 & Optimal &  0.85 & 26 &  0.00 &  0.00\\
instance n=50 19.alb & 1 & 0 & Optimal &  0.27 & 8 &  0.00 &  0.00\\
instance n=50 190.alb & 1 & 0 & Optimal &  0.33 & 30 &  0.00 &  0.00\\
instance n=50 191.alb & 1 & 0 & Optimal & 16.20 & 27 &  0.00 &  0.00\\
instance n=50 192.alb & 1 & 0 & Optimal &  3.71 & 27 &  0.00 &  0.00\\
instance n=50 193.alb & 1 & 0 & Optimal & 70.48 & 28 &  0.00 &  0.00\\
instance n=50 194.alb & 1 & 0 & Optimal &  1.88 & 28 &  0.00 &  0.00\\
instance n=50 195.alb & 1 & 0 & Optimal &  3.02 & 28 &  0.00 &  0.00\\
instance n=50 196.alb & 1 & 0 & Optimal & 26.04 & 27 &  0.00 &  0.00\\
instance n=50 197.alb & 1 & 0 & Optimal &  0.61 & 28 &  0.00 &  0.00\\
instance n=50 198.alb & 1 & 0 & Optimal &  0.30 & 28 &  0.00 &  0.00\\
instance n=50 199.alb & 1 & 0 & Optimal &  0.36 & 29 &  0.00 &  0.00\\
instance n=50 2.alb & 1 & 0 & Optimal &  0.26 & 6 &  0.00 &  0.00\\
instance n=50 20.alb & 1 & 0 & Optimal &  0.27 & 8 &  0.00 &  0.00\\
instance n=50 201.alb & 1 & 0 & Optimal &  0.27 & 13 &  0.00 &  0.00\\
instance n=50 202.alb & 1 & 0 & Optimal &  0.28 & 9 &  0.00 &  0.00\\
instance n=50 203.alb & 1 & 0 & Optimal &  0.29 & 11 &  0.00 &  0.00\\
instance n=50 204.alb & 1 & 0 & Optimal &  0.51 & 10 &  0.00 &  0.00\\
instance n=50 205.alb & 1 & 0 & Optimal &  0.27 & 13 &  0.00 &  0.00\\
instance n=50 206.alb & 1 & 0 & Optimal & 51.58 & 11 &  0.00 &  0.00\\
instance n=50 207.alb & 1 & 0 & Optimal &  0.27 & 10 &  0.00 &  0.00\\
instance n=50 208.alb & 1 & 0 & Optimal &  0.30 & 13 &  0.00 &  0.00\\
instance n=50 209.alb & 1 & 0 & Optimal &  0.28 & 11 &  0.00 &  0.00\\
instance n=50 21.alb & 1 & 0 & Optimal &  0.30 & 6 &  0.00 &  0.00\\
instance n=50 210.alb & 1 & 0 & Optimal &  0.27 & 13 &  0.00 &  0.00\\
instance n=50 211.alb & 1 & 0 & Optimal &  0.27 & 12 &  0.00 &  0.00\\
instance n=50 212.alb & 1 & 0 & Optimal &  0.43 & 10 &  0.00 &  0.00\\
instance n=50 213.alb & 1 & 0 & Optimal &  0.29 & 13 &  0.00 &  0.00\\
instance n=50 214.alb & 1 & 0 & Optimal &  0.26 & 11 &  0.00 &  0.00\\
instance n=50 215.alb & 1 & 0 & Optimal &  0.30 & 11 &  0.00 &  0.00\\
instance n=50 216.alb & 1 & 0 & Optimal &  0.28 & 12 &  0.00 &  0.00\\
instance n=50 217.alb & 1 & 0 & Optimal &  1.00 & 13 &  0.00 &  0.00\\
instance n=50 218.alb & 1 & 0 & Optimal &  0.26 & 12 &  0.00 &  0.00\\
instance n=50 219.alb & 1 & 0 & Optimal &  0.27 & 11 &  0.00 &  0.00\\
instance n=50 22.alb & 1 & 0 & Optimal &  0.27 & 7 &  0.00 &  0.00\\
instance n=50 220.alb & 1 & 0 & Optimal &  0.27 & 11 &  0.00 &  0.00\\
instance n=50 221.alb & 1 & 0 & Optimal &  0.38 & 11 &  0.00 &  0.00\\
instance n=50 222.alb & 1 & 0 & Optimal &  0.28 & 14 &  0.00 &  0.00\\
instance n=50 223.alb & 1 & 0 & Optimal &  0.35 & 11 &  0.00 &  0.00\\
instance n=50 224.alb & 1 & 0 & Optimal &  0.28 & 11 &  0.00 &  0.00\\
instance n=50 225.alb & 1 & 0 & Optimal &  0.26 & 12 &  0.00 &  0.00\\
instance n=50 226.alb & 1 & 0 & Optimal &  0.27 & 7 &  0.00 &  0.00\\
instance n=50 227.alb & 1 & 0 & Optimal &  0.28 & 6 &  0.00 &  0.00\\
instance n=50 228.alb & 1 & 0 & Optimal &  0.26 & 6 &  0.00 &  0.00\\
instance n=50 229.alb & 1 & 0 & Optimal &  0.41 & 6 &  0.00 &  0.00\\
instance n=50 23.alb & 1 & 0 & Optimal &  0.26 & 7 &  0.00 &  0.00\\
instance n=50 230.alb & 1 & 0 & Optimal &  0.26 & 7 &  0.00 &  0.00\\
instance n=50 231.alb & 1 & 0 & Optimal &  0.27 & 7 &  0.00 &  0.00\\
instance n=50 232.alb & 1 & 0 & Optimal &  0.31 & 7 &  0.00 &  0.00\\
instance n=50 233.alb & 1 & 0 & Optimal &  0.25 & 6 &  0.00 &  0.00\\
instance n=50 234.alb & 1 & 0 & Optimal &  0.27 & 8 &  0.00 &  0.00\\
instance n=50 235.alb & 1 & 0 & Optimal &  0.29 & 7 &  0.00 &  0.00\\
instance n=50 236.alb & 1 & 0 & Optimal &  0.29 & 7 &  0.00 &  0.00\\
instance n=50 237.alb & 1 & 0 & Optimal &  0.28 & 8 &  0.00 &  0.00\\
instance n=50 238.alb & 1 & 0 & Optimal &  0.26 & 7 &  0.00 &  0.00\\
instance n=50 239.alb & 1 & 0 & Optimal &  0.27 & 7 &  0.00 &  0.00\\
instance n=50 24.alb & 1 & 0 & Optimal &  0.27 & 7 &  0.00 &  0.00\\
instance n=50 240.alb & 1 & 0 & Optimal &  0.27 & 7 &  0.00 &  0.00\\
instance n=50 241.alb & 1 & 0 & Optimal &  0.26 & 7 &  0.00 &  0.00\\
instance n=50 242.alb & 1 & 0 & Optimal &  0.25 & 8 &  0.00 &  0.00\\
instance n=50 243.alb & 1 & 0 & Optimal &  0.28 & 7 &  0.00 &  0.00\\
instance n=50 244.alb & 1 & 0 & Optimal &  0.28 & 7 &  0.00 &  0.00\\
instance n=50 245.alb & 1 & 0 & Optimal &  0.29 & 7 &  0.00 &  0.00\\
instance n=50 246.alb & 1 & 0 & Optimal &  0.44 & 8 &  0.00 &  0.00\\
instance n=50 247.alb & 1 & 0 & Optimal &  0.27 & 7 &  0.00 &  0.00\\
instance n=50 248.alb & 1 & 0 & Optimal &  0.27 & 7 &  0.00 &  0.00\\
instance n=50 249.alb & 1 & 0 & Optimal &  0.32 & 7 &  0.00 &  0.00\\
instance n=50 25.alb & 1 & 0 & Optimal &  0.26 & 6 &  0.00 &  0.00\\
instance n=50 250.alb & 1 & 0 & Optimal &  0.29 & 7 &  0.00 &  0.00\\
instance n=50 251.alb & 1 & 0 & Optimal &  2.71 & 27 &  0.00 &  0.00\\
instance n=50 252.alb & 1 & 0 & Optimal &  4.31 & 32 &  0.00 &  0.00\\
instance n=50 253.alb & 1 & 0 & Optimal &  3.78 & 28 &  0.00 &  0.00\\
instance n=50 254.alb & 1 & 0 & Optimal &  0.32 & 30 &  0.00 &  0.00\\
instance n=50 255.alb & 1 & 0 & Optimal &  0.80 & 29 &  0.00 &  0.00\\
instance n=50 256.alb & 1 & 0 & Optimal &  0.40 & 30 &  0.00 &  0.00\\
instance n=50 257.alb & 1 & 0 & Optimal &  4.81 & 33 &  0.00 &  0.00\\
instance n=50 258.alb & 1 & 0 & Optimal &  5.00 & 28 &  0.00 &  0.00\\
instance n=50 259.alb & 1 & 0 & Optimal &  3.91 & 31 &  0.00 &  0.00\\
instance n=50 26.alb & 1 & 0 & Optimal &  0.39 & 27 &  0.00 &  0.00\\
instance n=50 260.alb & 1 & 0 & Optimal &  0.62 & 29 &  0.00 &  0.00\\
instance n=50 261.alb & 1 & 0 & Optimal &  2.08 & 28 &  0.00 &  0.00\\
instance n=50 262.alb & 1 & 0 & Optimal &  0.94 & 31 &  0.00 &  0.00\\
instance n=50 263.alb & 1 & 0 & Optimal &  1.27 & 29 &  0.00 &  0.00\\
instance n=50 264.alb & 1 & 0 & Optimal &  2.65 & 27 &  0.00 &  0.00\\
instance n=50 265.alb & 1 & 0 & Optimal &  0.91 & 27 &  0.00 &  0.00\\
instance n=50 266.alb & 1 & 0 & Optimal &  5.48 & 29 &  0.00 &  0.00\\
instance n=50 267.alb & 1 & 0 & Optimal &  3.86 & 28 &  0.00 &  0.00\\
instance n=50 268.alb & 1 & 0 & Optimal &  5.26 & 29 &  0.00 &  0.00\\
instance n=50 269.alb & 1 & 0 & Optimal &  1.13 & 26 &  0.00 &  0.00\\
instance n=50 27.alb & 1 & 0 & Optimal &  0.37 & 30 &  0.00 &  0.00\\
instance n=50 270.alb & 1 & 0 & Optimal &  0.38 & 28 &  0.00 &  0.00\\
instance n=50 271.alb & 1 & 0 & Optimal &  3.38 & 31 &  0.00 &  0.00\\
instance n=50 272.alb & 1 & 0 & Optimal &  2.06 & 27 &  0.00 &  0.00\\
instance n=50 273.alb & 1 & 0 & Optimal &  5.74 & 27 &  0.00 &  0.00\\
instance n=50 274.alb & 1 & 0 & Optimal &  0.33 & 29 &  0.00 &  0.00\\
instance n=50 275.alb & 1 & 0 & Optimal &  2.43 & 27 &  0.00 &  0.00\\
instance n=50 276.alb & 1 & 0 & Optimal &  0.32 & 12 &  0.00 &  0.00\\
instance n=50 277.alb & 1 & 0 & Optimal &  0.28 & 13 &  0.00 &  0.00\\
instance n=50 278.alb & 1 & 0 & Optimal &  0.30 & 12 &  0.00 &  0.00\\
instance n=50 279.alb & 1 & 0 & Optimal &  0.33 & 11 &  0.00 &  0.00\\
instance n=50 28.alb & 1 & 0 & Optimal &  1.70 & 28 &  0.00 &  0.00\\
instance n=50 280.alb & 1 & 0 & Optimal &  0.32 & 13 &  0.00 &  0.00\\
instance n=50 281.alb & 1 & 0 & Optimal &  0.31 & 11 &  0.00 &  0.00\\
instance n=50 282.alb & 1 & 0 & Optimal &  3.82 & 12 &  0.00 &  0.00\\
instance n=50 283.alb & 1 & 0 & Optimal &  0.34 & 12 &  0.00 &  0.00\\
instance n=50 284.alb & 1 & 0 & Optimal &  0.28 & 11 &  0.00 &  0.00\\
instance n=50 285.alb & 1 & 0 & Optimal &  0.31 & 13 &  0.00 &  0.00\\
instance n=50 286.alb & 1 & 0 & Optimal &  0.34 & 11 &  0.00 &  0.00\\
instance n=50 287.alb & 1 & 0 & Optimal &  0.47 & 12 &  0.00 &  0.00\\
instance n=50 288.alb & 1 & 0 & Optimal &  0.36 & 10 &  0.00 &  0.00\\
instance n=50 289.alb & 1 & 0 & Optimal &  0.38 & 11 &  0.00 &  0.00\\
instance n=50 29.alb & 1 & 0 & Optimal &  0.28 & 29 &  0.00 &  0.00\\
instance n=50 290.alb & 1 & 0 & Optimal &  0.31 & 14 &  0.00 &  0.00\\
instance n=50 291.alb & 1 & 0 & Optimal &  0.27 & 12 &  0.00 &  0.00\\
instance n=50 292.alb & 1 & 0 & Optimal &  0.28 & 13 &  0.00 &  0.00\\
instance n=50 293.alb & 1 & 0 & Optimal &  0.30 & 12 &  0.00 &  0.00\\
instance n=50 294.alb & 1 & 0 & Optimal &  0.29 & 13 &  0.00 &  0.00\\
instance n=50 295.alb & 1 & 0 & Optimal &  0.30 & 16 &  0.00 &  0.00\\
instance n=50 296.alb & 1 & 0 & Optimal &  0.29 & 13 &  0.00 &  0.00\\
instance n=50 297.alb & 1 & 0 & Optimal &  0.28 & 13 &  0.00 &  0.00\\
instance n=50 298.alb & 1 & 0 & Optimal &  0.28 & 11 &  0.00 &  0.00\\
instance n=50 299.alb & 1 & 0 & Optimal &  1.22 & 12 &  0.00 &  0.00\\
instance n=50 3.alb & 1 & 0 & Optimal &  0.25 & 8 &  0.00 &  0.00\\
instance n=50 30.alb & 1 & 0 & Optimal & 21.20 & 26 &  0.00 &  0.00\\
instance n=50 300.alb & 1 & 0 & Optimal &  0.29 & 12 &  0.00 &  0.00\\
instance n=50 301.alb & 1 & 0 & Optimal &  0.27 & 6 &  0.00 &  0.00\\
instance n=50 302.alb & 1 & 0 & Optimal &  0.39 & 7 &  0.00 &  0.00\\
instance n=50 303.alb & 1 & 0 & Optimal &  0.27 & 8 &  0.00 &  0.00\\
instance n=50 304.alb & 1 & 0 & Optimal &  0.28 & 7 &  0.00 &  0.00\\
instance n=50 305.alb & 1 & 0 & Optimal &  0.26 & 8 &  0.00 &  0.00\\
instance n=50 306.alb & 1 & 0 & Optimal &  0.27 & 7 &  0.00 &  0.00\\
instance n=50 307.alb & 1 & 0 & Optimal &  0.26 & 7 &  0.00 &  0.00\\
instance n=50 308.alb & 1 & 0 & Optimal &  0.27 & 8 &  0.00 &  0.00\\
instance n=50 309.alb & 1 & 0 & Optimal &  0.31 & 7 &  0.00 &  0.00\\
instance n=50 310.alb & 1 & 0 & Optimal &  0.28 & 8 &  0.00 &  0.00\\
instance n=50 311.alb & 1 & 0 & Optimal &  0.27 & 8 &  0.00 &  0.00\\
instance n=50 312.alb & 1 & 0 & Optimal &  0.32 & 6 &  0.00 &  0.00\\
instance n=50 313.alb & 1 & 0 & Optimal &  0.28 & 8 &  0.00 &  0.00\\
instance n=50 314.alb & 1 & 0 & Optimal &  0.27 & 7 &  0.00 &  0.00\\
instance n=50 315.alb & 1 & 0 & Optimal &  0.27 & 8 &  0.00 &  0.00\\
instance n=50 316.alb & 1 & 0 & Optimal &  0.27 & 8 &  0.00 &  0.00\\
instance n=50 317.alb & 1 & 0 & Optimal &  0.26 & 6 &  0.00 &  0.00\\
instance n=50 318.alb & 1 & 0 & Optimal &  0.25 & 8 &  0.00 &  0.00\\
instance n=50 319.alb & 1 & 0 & Optimal &  0.28 & 7 &  0.00 &  0.00\\
instance n=50 32.alb & 1 & 0 & Optimal &  0.90 & 25 &  0.00 &  0.00\\
instance n=50 320.alb & 1 & 0 & Optimal &  0.25 & 8 &  0.00 &  0.00\\
instance n=50 321.alb & 1 & 0 & Optimal &  0.27 & 6 &  0.00 &  0.00\\
instance n=50 322.alb & 1 & 0 & Optimal &  0.29 & 7 &  0.00 &  0.00\\
instance n=50 323.alb & 1 & 0 & Optimal &  0.25 & 7 &  0.00 &  0.00\\
instance n=50 324.alb & 1 & 0 & Optimal &  0.27 & 7 &  0.00 &  0.00\\
instance n=50 325.alb & 1 & 0 & Optimal &  0.27 & 7 &  0.00 &  0.00\\
instance n=50 326.alb & 1 & 0 & Optimal &  0.35 & 33 &  0.00 &  0.00\\
instance n=50 327.alb & 1 & 0 & Optimal &  0.29 & 28 &  0.00 &  0.00\\
instance n=50 328.alb & 1 & 0 & Optimal &  0.29 & 32 &  0.00 &  0.00\\
instance n=50 330.alb & 1 & 0 & Optimal &  0.27 & 29 &  0.00 &  0.00\\
instance n=50 331.alb & 1 & 0 & Optimal &  0.83 & 29 &  0.00 &  0.00\\
instance n=50 333.alb & 1 & 0 & Optimal &  0.51 & 28 &  0.00 &  0.00\\
instance n=50 334.alb & 1 & 0 & Optimal &  0.28 & 29 &  0.00 &  0.00\\
instance n=50 335.alb & 1 & 0 & Optimal &  0.40 & 27 &  0.00 &  0.00\\
instance n=50 337.alb & 1 & 0 & Optimal &  0.49 & 26 &  0.00 &  0.00\\
instance n=50 338.alb & 1 & 0 & Optimal & 11.11 & 26 &  0.00 &  0.00\\
instance n=50 339.alb & 1 & 0 & Optimal &  0.35 & 27 &  0.00 &  0.00\\
instance n=50 34.alb & 1 & 0 & Optimal &  0.45 & 30 &  0.00 &  0.00\\
instance n=50 341.alb & 1 & 0 & Optimal &  0.33 & 27 &  0.00 &  0.00\\
instance n=50 343.alb & 1 & 0 & Optimal &  0.35 & 27 &  0.00 &  0.00\\
instance n=50 344.alb & 1 & 0 & Optimal &  0.79 & 30 &  0.00 &  0.00\\
instance n=50 345.alb & 1 & 0 & Optimal &  0.74 & 29 &  0.00 &  0.00\\
instance n=50 346.alb & 1 & 0 & Optimal &  0.34 & 27 &  0.00 &  0.00\\
instance n=50 347.alb & 1 & 0 & Optimal &  7.23 & 25 &  0.00 &  0.00\\
instance n=50 348.alb & 1 & 0 & Optimal &  0.30 & 30 &  0.00 &  0.00\\
instance n=50 349.alb & 1 & 0 & Optimal &  1.58 & 28 &  0.00 &  0.00\\
instance n=50 35.alb & 1 & 0 & Optimal &  1.44 & 31 &  0.00 &  0.00\\
instance n=50 351.alb & 1 & 0 & Optimal &  0.29 & 12 &  0.00 &  0.00\\
instance n=50 352.alb & 1 & 0 & Optimal &  1.33 & 10 &  0.00 &  0.00\\
instance n=50 353.alb & 1 & 0 & Optimal &  0.30 & 13 &  0.00 &  0.00\\
instance n=50 354.alb & 1 & 0 & Optimal & 27.87 & 13 &  0.00 &  0.00\\
instance n=50 355.alb & 1 & 0 & Optimal &  0.28 & 11 &  0.00 &  0.00\\
instance n=50 356.alb & 1 & 0 & Optimal &  0.27 & 15 &  0.00 &  0.00\\
instance n=50 357.alb & 1 & 0 & Optimal &  0.27 & 12 &  0.00 &  0.00\\
instance n=50 358.alb & 1 & 0 & Optimal &  0.30 & 11 &  0.00 &  0.00\\
instance n=50 359.alb & 1 & 0 & Optimal &  0.27 & 10 &  0.00 &  0.00\\
instance n=50 36.alb & 1 & 0 & Optimal &  0.34 & 31 &  0.00 &  0.00\\
instance n=50 360.alb & 1 & 0 & Optimal &  0.32 & 12 &  0.00 &  0.00\\
instance n=50 361.alb & 1 & 0 & Optimal &  0.27 & 11 &  0.00 &  0.00\\
instance n=50 362.alb & 1 & 0 & Optimal &  0.27 & 10 &  0.00 &  0.00\\
instance n=50 364.alb & 1 & 0 & Optimal &  0.30 & 13 &  0.00 &  0.00\\
instance n=50 365.alb & 1 & 0 & Optimal &  0.28 & 11 &  0.00 &  0.00\\
instance n=50 366.alb & 1 & 0 & Optimal &  0.28 & 13 &  0.00 &  0.00\\
instance n=50 367.alb & 1 & 0 & Optimal &  0.31 & 12 &  0.00 &  0.00\\
instance n=50 368.alb & 1 & 0 & Optimal &  0.30 & 12 &  0.00 &  0.00\\
instance n=50 369.alb & 1 & 0 & Optimal &  0.29 & 12 &  0.00 &  0.00\\
instance n=50 370.alb & 1 & 0 & Optimal &  0.29 & 12 &  0.00 &  0.00\\
instance n=50 371.alb & 1 & 0 & Optimal &  0.84 & 11 &  0.00 &  0.00\\
instance n=50 372.alb & 1 & 0 & Optimal &  1.94 & 10 &  0.00 &  0.00\\
instance n=50 373.alb & 1 & 0 & Optimal &  0.30 & 12 &  0.00 &  0.00\\
instance n=50 374.alb & 1 & 0 & Optimal &  0.26 & 11 &  0.00 &  0.00\\
instance n=50 375.alb & 1 & 0 & Optimal &  0.31 & 13 &  0.00 &  0.00\\
instance n=50 376.alb & 1 & 0 & Optimal &  0.30 & 7 &  0.00 &  0.00\\
instance n=50 377.alb & 1 & 0 & Optimal &  0.27 & 7 &  0.00 &  0.00\\
instance n=50 378.alb & 1 & 0 & Optimal &  0.27 & 8 &  0.00 &  0.00\\
instance n=50 379.alb & 1 & 0 & Optimal &  0.28 & 7 &  0.00 &  0.00\\
instance n=50 38.alb & 1 & 0 & Optimal &  0.47 & 31 &  0.00 &  0.00\\
instance n=50 380.alb & 1 & 0 & Optimal &  0.28 & 7 &  0.00 &  0.00\\
instance n=50 381.alb & 1 & 0 & Optimal &  0.28 & 8 &  0.00 &  0.00\\
instance n=50 382.alb & 1 & 0 & Optimal &  0.26 & 6 &  0.00 &  0.00\\
instance n=50 383.alb & 1 & 0 & Optimal &  0.27 & 7 &  0.00 &  0.00\\
instance n=50 384.alb & 1 & 0 & Optimal &  0.29 & 8 &  0.00 &  0.00\\
instance n=50 385.alb & 1 & 0 & Optimal &  0.28 & 7 &  0.00 &  0.00\\
instance n=50 386.alb & 1 & 0 & Optimal &  0.26 & 7 &  0.00 &  0.00\\
instance n=50 387.alb & 1 & 0 & Optimal &  0.29 & 8 &  0.00 &  0.00\\
instance n=50 388.alb & 1 & 0 & Optimal &  0.41 & 7 &  0.00 &  0.00\\
instance n=50 389.alb & 1 & 0 & Optimal &  0.27 & 8 &  0.00 &  0.00\\
instance n=50 390.alb & 1 & 0 & Optimal &  0.30 & 7 &  0.00 &  0.00\\
instance n=50 391.alb & 1 & 0 & Optimal &  0.43 & 7 &  0.00 &  0.00\\
instance n=50 392.alb & 1 & 0 & Optimal &  0.27 & 8 &  0.00 &  0.00\\
instance n=50 393.alb & 1 & 0 & Optimal &  0.28 & 7 &  0.00 &  0.00\\
instance n=50 394.alb & 1 & 0 & Optimal &  0.27 & 8 &  0.00 &  0.00\\
instance n=50 395.alb & 1 & 0 & Optimal &  0.30 & 7 &  0.00 &  0.00\\
instance n=50 396.alb & 1 & 0 & Optimal &  0.26 & 8 &  0.00 &  0.00\\
instance n=50 397.alb & 1 & 0 & Optimal &  0.29 & 7 &  0.00 &  0.00\\
instance n=50 398.alb & 1 & 0 & Optimal &  0.32 & 6 &  0.00 &  0.00\\
instance n=50 399.alb & 1 & 0 & Optimal &  0.65 & 7 &  0.00 &  0.00\\
instance n=50 4.alb & 1 & 0 & Optimal &  0.29 & 7 &  0.00 &  0.00\\
instance n=50 40.alb & 1 & 0 & Optimal &  1.81 & 26 &  0.00 &  0.00\\
instance n=50 400.alb & 1 & 0 & Optimal &  0.27 & 8 &  0.00 &  0.00\\
instance n=50 401.alb & 1 & 0 & Optimal & 57.99 & 28 &  0.00 &  0.00\\
instance n=50 402.alb & 1 & 0 & Optimal &  0.59 & 27 &  0.00 &  0.00\\
instance n=50 403.alb & 1 & 0 & Optimal &  3.11 & 34 &  0.00 &  0.00\\
instance n=50 404.alb & 1 & 0 & Optimal &  3.16 & 31 &  0.00 &  0.00\\
instance n=50 405.alb & 1 & 0 & Optimal &  2.26 & 27 &  0.00 &  0.00\\
instance n=50 406.alb & 1 & 0 & Optimal &  0.75 & 32 &  0.00 &  0.00\\
instance n=50 407.alb & 1 & 0 & Optimal &  9.11 & 29 &  0.00 &  0.00\\
instance n=50 408.alb & 1 & 0 & Optimal &  0.35 & 26 &  0.00 &  0.00\\
instance n=50 409.alb & 1 & 0 & Optimal &  6.88 & 33 &  0.00 &  0.00\\
instance n=50 41.alb & 1 & 0 & Optimal & 17.22 & 25 &  0.00 &  0.00\\
instance n=50 410.alb & 1 & 0 & Optimal &  0.49 & 28 &  0.00 &  0.00\\
instance n=50 411.alb & 1 & 0 & Optimal &  0.36 & 29 &  0.00 &  0.00\\
instance n=50 412.alb & 1 & 0 & Optimal &  0.37 & 26 &  0.00 &  0.00\\
instance n=50 413.alb & 1 & 0 & Optimal &  0.39 & 30 &  0.00 &  0.00\\
instance n=50 414.alb & 1 & 0 & Optimal & 24.18 & 27 &  0.00 &  0.00\\
instance n=50 415.alb & 1 & 0 & Optimal &  0.38 & 28 &  0.00 &  0.00\\
instance n=50 416.alb & 1 & 0 & Optimal &  0.46 & 27 &  0.00 &  0.00\\
instance n=50 417.alb & 1 & 0 & Optimal & 53.93 & 30 &  0.00 &  0.00\\
instance n=50 418.alb & 1 & 0 & Optimal &  1.38 & 27 &  0.00 &  0.00\\
instance n=50 419.alb & 1 & 0 & Optimal & 11.25 & 33 &  0.00 &  0.00\\
instance n=50 420.alb & 1 & 0 & Optimal & 11.48 & 28 &  0.00 &  0.00\\
instance n=50 421.alb & 1 & 0 & Optimal &  2.20 & 34 &  0.00 &  0.00\\
instance n=50 422.alb & 1 & 0 & Optimal &  4.10 & 29 &  0.00 &  0.00\\
instance n=50 423.alb & 1 & 0 & Optimal &  0.43 & 29 &  0.00 &  0.00\\
instance n=50 424.alb & 1 & 0 & Optimal &  1.05 & 27 &  0.00 &  0.00\\
instance n=50 425.alb & 1 & 0 & Optimal &  4.65 & 34 &  0.00 &  0.00\\
instance n=50 426.alb & 1 & 0 & Optimal &  0.35 & 11 &  0.00 &  0.00\\
instance n=50 427.alb & 1 & 0 & Optimal &  0.27 & 12 &  0.00 &  0.00\\
instance n=50 428.alb & 1 & 0 & Optimal &  0.29 & 13 &  0.00 &  0.00\\
instance n=50 429.alb & 1 & 0 & Optimal &  0.29 & 11 &  0.00 &  0.00\\
instance n=50 43.alb & 1 & 0 & Optimal &  0.70 & 25 &  0.00 &  0.00\\
instance n=50 430.alb & 1 & 0 & Optimal &  1.10 & 14 &  0.00 &  0.00\\
instance n=50 431.alb & 1 & 0 & Optimal &  0.30 & 11 &  0.00 &  0.00\\
instance n=50 432.alb & 1 & 0 & Optimal &  0.37 & 12 &  0.00 &  0.00\\
instance n=50 433.alb & 1 & 0 & Optimal &  0.29 & 12 &  0.00 &  0.00\\
instance n=50 434.alb & 1 & 0 & Optimal &  0.28 & 11 &  0.00 &  0.00\\
instance n=50 435.alb & 1 & 0 & Optimal &  0.26 & 11 &  0.00 &  0.00\\
instance n=50 436.alb & 1 & 0 & Optimal &  0.31 & 11 &  0.00 &  0.00\\
instance n=50 437.alb & 1 & 0 & Optimal &  5.39 & 12 &  0.00 &  0.00\\
instance n=50 438.alb & 1 & 0 & Optimal &  1.71 & 10 &  0.00 &  0.00\\
instance n=50 439.alb & 1 & 0 & Optimal &  1.25 & 12 &  0.00 &  0.00\\
instance n=50 440.alb & 1 & 0 & Optimal &  1.33 & 13 &  0.00 &  0.00\\
instance n=50 441.alb & 1 & 0 & Optimal &  0.32 & 11 &  0.00 &  0.00\\
instance n=50 442.alb & 1 & 0 & Optimal &  0.31 & 12 &  0.00 &  0.00\\
instance n=50 443.alb & 1 & 0 & Optimal &  0.32 & 11 &  0.00 &  0.00\\
instance n=50 444.alb & 1 & 0 & Optimal &  0.29 & 12 &  0.00 &  0.00\\
instance n=50 445.alb & 1 & 0 & Optimal &  0.30 & 12 &  0.00 &  0.00\\
instance n=50 446.alb & 1 & 0 & Optimal &  0.30 & 12 &  0.00 &  0.00\\
instance n=50 447.alb & 1 & 0 & Optimal &  0.31 & 13 &  0.00 &  0.00\\
instance n=50 448.alb & 1 & 0 & Optimal &  0.87 & 12 &  0.00 &  0.00\\
instance n=50 449.alb & 1 & 0 & Optimal &  0.32 & 11 &  0.00 &  0.00\\
instance n=50 450.alb & 1 & 0 & Optimal &  0.29 & 11 &  0.00 &  0.00\\
instance n=50 451.alb & 1 & 0 & Optimal &  0.27 & 8 &  0.00 &  0.00\\
instance n=50 452.alb & 1 & 0 & Optimal &  0.29 & 8 &  0.00 &  0.00\\
instance n=50 453.alb & 1 & 0 & Optimal &  0.27 & 7 &  0.00 &  0.00\\
instance n=50 454.alb & 1 & 0 & Optimal &  0.26 & 8 &  0.00 &  0.00\\
instance n=50 455.alb & 1 & 0 & Optimal &  0.28 & 6 &  0.00 &  0.00\\
instance n=50 456.alb & 1 & 0 & Optimal &  0.29 & 8 &  0.00 &  0.00\\
instance n=50 457.alb & 1 & 0 & Optimal &  0.27 & 8 &  0.00 &  0.00\\
instance n=50 458.alb & 1 & 0 & Optimal &  0.28 & 7 &  0.00 &  0.00\\
instance n=50 459.alb & 1 & 0 & Optimal &  0.29 & 7 &  0.00 &  0.00\\
instance n=50 46.alb & 1 & 0 & Optimal &  0.46 & 28 &  0.00 &  0.00\\
instance n=50 460.alb & 1 & 0 & Optimal &  0.29 & 7 &  0.00 &  0.00\\
instance n=50 461.alb & 1 & 0 & Optimal &  0.27 & 6 &  0.00 &  0.00\\
instance n=50 462.alb & 1 & 0 & Optimal &  0.28 & 7 &  0.00 &  0.00\\
instance n=50 463.alb & 1 & 0 & Optimal &  0.44 & 8 &  0.00 &  0.00\\
instance n=50 464.alb & 1 & 0 & Optimal &  0.28 & 6 &  0.00 &  0.00\\
instance n=50 465.alb & 1 & 0 & Optimal &  0.27 & 8 &  0.00 &  0.00\\
instance n=50 466.alb & 1 & 0 & Optimal &  0.29 & 7 &  0.00 &  0.00\\
instance n=50 467.alb & 1 & 0 & Optimal &  0.29 & 9 &  0.00 &  0.00\\
instance n=50 468.alb & 1 & 0 & Optimal &  0.27 & 7 &  0.00 &  0.00\\
instance n=50 469.alb & 1 & 0 & Optimal &  0.30 & 8 &  0.00 &  0.00\\
instance n=50 47.alb & 1 & 0 & Optimal &  0.66 & 28 &  0.00 &  0.00\\
instance n=50 470.alb & 1 & 0 & Optimal &  0.29 & 8 &  0.00 &  0.00\\
instance n=50 471.alb & 1 & 0 & Optimal &  0.27 & 7 &  0.00 &  0.00\\
instance n=50 472.alb & 1 & 0 & Optimal &  0.27 & 8 &  0.00 &  0.00\\
instance n=50 473.alb & 1 & 0 & Optimal &  0.29 & 7 &  0.00 &  0.00\\
instance n=50 474.alb & 1 & 0 & Optimal &  0.28 & 7 &  0.00 &  0.00\\
instance n=50 475.alb & 1 & 0 & Optimal &  0.28 & 6 &  0.00 &  0.00\\
instance n=50 476.alb & 1 & 0 & Optimal &  0.32 & 28 &  0.00 &  0.00\\
instance n=50 477.alb & 1 & 0 & Optimal &  0.31 & 29 &  0.00 &  0.00\\
instance n=50 478.alb & 1 & 0 & Optimal &  0.34 & 32 &  0.00 &  0.00\\
instance n=50 479.alb & 1 & 0 & Optimal &  0.31 & 28 &  0.00 &  0.00\\
instance n=50 48.alb & 1 & 0 & Optimal &  1.37 & 27 &  0.00 &  0.00\\
instance n=50 480.alb & 1 & 0 & Optimal &  0.29 & 34 &  0.00 &  0.00\\
instance n=50 481.alb & 1 & 0 & Optimal &  0.30 & 28 &  0.00 &  0.00\\
instance n=50 482.alb & 1 & 0 & Optimal &  0.30 & 27 &  0.00 &  0.00\\
instance n=50 483.alb & 1 & 0 & Optimal &  0.36 & 30 &  0.00 &  0.00\\
instance n=50 484.alb & 1 & 0 & Optimal &  0.28 & 32 &  0.00 &  0.00\\
instance n=50 485.alb & 1 & 0 & Optimal &  0.31 & 31 &  0.00 &  0.00\\
instance n=50 486.alb & 1 & 0 & Optimal &  0.33 & 32 &  0.00 &  0.00\\
instance n=50 487.alb & 1 & 0 & Optimal &  0.39 & 31 &  0.00 &  0.00\\
instance n=50 488.alb & 1 & 0 & Optimal &  0.30 & 31 &  0.00 &  0.00\\
instance n=50 489.alb & 1 & 0 & Optimal &  0.32 & 35 &  0.00 &  0.00\\
instance n=50 49.alb & 1 & 0 & Optimal &  0.40 & 25 &  0.00 &  0.00\\
instance n=50 490.alb & 1 & 0 & Optimal &  0.29 & 29 &  0.00 &  0.00\\
instance n=50 491.alb & 1 & 0 & Optimal &  0.30 & 35 &  0.00 &  0.00\\
instance n=50 492.alb & 1 & 0 & Optimal &  0.28 & 29 &  0.00 &  0.00\\
instance n=50 493.alb & 1 & 0 & Optimal &  0.35 & 30 &  0.00 &  0.00\\
instance n=50 494.alb & 1 & 0 & Optimal &  0.30 & 32 &  0.00 &  0.00\\
instance n=50 495.alb & 1 & 0 & Optimal &  0.30 & 34 &  0.00 &  0.00\\
instance n=50 496.alb & 1 & 0 & Optimal &  0.31 & 29 &  0.00 &  0.00\\
instance n=50 497.alb & 1 & 0 & Optimal &  0.34 & 30 &  0.00 &  0.00\\
instance n=50 498.alb & 1 & 0 & Optimal &  0.31 & 30 &  0.00 &  0.00\\
instance n=50 499.alb & 1 & 0 & Optimal &  0.34 & 33 &  0.00 &  0.00\\
instance n=50 5.alb & 1 & 0 & Optimal &  0.26 & 7 &  0.00 &  0.00\\
instance n=50 500.alb & 1 & 0 & Optimal &  0.33 & 34 &  0.00 &  0.00\\
instance n=50 501.alb & 1 & 0 & Optimal &  0.27 & 12 &  0.00 &  0.00\\
instance n=50 502.alb & 1 & 0 & Optimal &  0.28 & 10 &  0.00 &  0.00\\
instance n=50 503.alb & 1 & 0 & Optimal &  0.28 & 13 &  0.00 &  0.00\\
instance n=50 504.alb & 1 & 0 & Optimal &  0.26 & 11 &  0.00 &  0.00\\
instance n=50 505.alb & 1 & 0 & Optimal &  0.29 & 12 &  0.00 &  0.00\\
instance n=50 506.alb & 1 & 0 & Optimal &  0.28 & 11 &  0.00 &  0.00\\
instance n=50 507.alb & 1 & 0 & Optimal &  0.28 & 13 &  0.00 &  0.00\\
instance n=50 508.alb & 1 & 0 & Optimal &  0.28 & 14 &  0.00 &  0.00\\
instance n=50 509.alb & 1 & 0 & Optimal &  0.28 & 13 &  0.00 &  0.00\\
instance n=50 51.alb & 1 & 0 & Optimal &  0.28 & 12 &  0.00 &  0.00\\
instance n=50 510.alb & 1 & 0 & Optimal &  0.29 & 11 &  0.00 &  0.00\\
instance n=50 511.alb & 1 & 0 & Optimal &  0.28 & 13 &  0.00 &  0.00\\
instance n=50 512.alb & 1 & 0 & Optimal &  0.31 & 13 &  0.00 &  0.00\\
instance n=50 513.alb & 1 & 0 & Optimal &  0.31 & 12 &  0.00 &  0.00\\
instance n=50 514.alb & 1 & 0 & Optimal &  0.30 & 12 &  0.00 &  0.00\\
instance n=50 515.alb & 1 & 0 & Optimal &  0.27 & 11 &  0.00 &  0.00\\
instance n=50 516.alb & 1 & 0 & Optimal &  0.29 & 13 &  0.00 &  0.00\\
instance n=50 517.alb & 1 & 0 & Optimal &  0.29 & 14 &  0.00 &  0.00\\
instance n=50 518.alb & 1 & 0 & Optimal &  0.28 & 11 &  0.00 &  0.00\\
instance n=50 519.alb & 1 & 0 & Optimal &  0.31 & 12 &  0.00 &  0.00\\
instance n=50 52.alb & 1 & 0 & Optimal &  0.28 & 11 &  0.00 &  0.00\\
instance n=50 520.alb & 1 & 0 & Optimal &  0.28 & 11 &  0.00 &  0.00\\
instance n=50 521.alb & 1 & 0 & Optimal &  0.28 & 10 &  0.00 &  0.00\\
instance n=50 522.alb & 1 & 0 & Optimal &  0.28 & 11 &  0.00 &  0.00\\
instance n=50 523.alb & 1 & 0 & Optimal &  0.27 & 11 &  0.00 &  0.00\\
instance n=50 524.alb & 1 & 0 & Optimal &  0.27 & 14 &  0.00 &  0.00\\
instance n=50 525.alb & 1 & 0 & Optimal &  0.28 & 11 &  0.00 &  0.00\\
instance n=50 54.alb & 1 & 0 & Optimal &  0.30 & 11 &  0.00 &  0.00\\
instance n=50 55.alb & 1 & 0 & Optimal &  0.28 & 13 &  0.00 &  0.00\\
instance n=50 56.alb & 1 & 0 & Optimal &  0.30 & 11 &  0.00 &  0.00\\
instance n=50 57.alb & 1 & 0 & Optimal &  0.28 & 13 &  0.00 &  0.00\\
instance n=50 58.alb & 1 & 0 & Optimal &  0.30 & 11 &  0.00 &  0.00\\
instance n=50 59.alb & 1 & 0 & Optimal &  0.29 & 11 &  0.00 &  0.00\\
instance n=50 6.alb & 1 & 0 & Optimal &  0.28 & 6 &  0.00 &  0.00\\
instance n=50 60.alb & 1 & 0 & Optimal &  0.29 & 12 &  0.00 &  0.00\\
instance n=50 61.alb & 1 & 0 & Optimal &  0.31 & 13 &  0.00 &  0.00\\
instance n=50 62.alb & 1 & 0 & Optimal &  0.28 & 13 &  0.00 &  0.00\\
instance n=50 63.alb & 1 & 0 & Optimal &  0.27 & 12 &  0.00 &  0.00\\
instance n=50 64.alb & 1 & 0 & Optimal &  0.28 & 13 &  0.00 &  0.00\\
instance n=50 65.alb & 1 & 0 & Optimal &  0.27 & 12 &  0.00 &  0.00\\
instance n=50 66.alb & 1 & 0 & Optimal &  0.31 & 12 &  0.00 &  0.00\\
instance n=50 67.alb & 1 & 0 & Optimal &  0.33 & 12 &  0.00 &  0.00\\
instance n=50 68.alb & 1 & 0 & Optimal &  0.29 & 12 &  0.00 &  0.00\\
instance n=50 69.alb & 1 & 0 & Optimal &  0.29 & 12 &  0.00 &  0.00\\
instance n=50 7.alb & 1 & 0 & Optimal &  0.29 & 7 &  0.00 &  0.00\\
instance n=50 70.alb & 1 & 0 & Optimal &  0.31 & 10 &  0.00 &  0.00\\
instance n=50 71.alb & 1 & 0 & Optimal &  0.28 & 13 &  0.00 &  0.00\\
instance n=50 72.alb & 1 & 0 & Optimal &  0.29 & 11 &  0.00 &  0.00\\
instance n=50 73.alb & 1 & 0 & Optimal &  0.28 & 11 &  0.00 &  0.00\\
instance n=50 74.alb & 1 & 0 & Optimal &  0.27 & 12 &  0.00 &  0.00\\
instance n=50 75.alb & 1 & 0 & Optimal &  0.40 & 11 &  0.00 &  0.00\\
instance n=50 76.alb & 1 & 0 & Optimal &  0.28 & 7 &  0.00 &  0.00\\
instance n=50 77.alb & 1 & 0 & Optimal &  0.27 & 7 &  0.00 &  0.00\\
instance n=50 78.alb & 1 & 0 & Optimal &  0.29 & 7 &  0.00 &  0.00\\
instance n=50 79.alb & 1 & 0 & Optimal &  0.32 & 8 &  0.00 &  0.00\\
instance n=50 8.alb & 1 & 0 & Optimal &  0.29 & 7 &  0.00 &  0.00\\
instance n=50 80.alb & 1 & 0 & Optimal &  0.31 & 7 &  0.00 &  0.00\\
instance n=50 81.alb & 1 & 0 & Optimal &  0.28 & 7 &  0.00 &  0.00\\
instance n=50 82.alb & 1 & 0 & Optimal &  0.30 & 6 &  0.00 &  0.00\\
instance n=50 83.alb & 1 & 0 & Optimal &  0.31 & 8 &  0.00 &  0.00\\
instance n=50 84.alb & 1 & 0 & Optimal &  0.27 & 7 &  0.00 &  0.00\\
instance n=50 85.alb & 1 & 0 & Optimal &  0.27 & 8 &  0.00 &  0.00\\
instance n=50 86.alb & 1 & 0 & Optimal &  0.32 & 7 &  0.00 &  0.00\\
instance n=50 87.alb & 1 & 0 & Optimal &  0.29 & 8 &  0.00 &  0.00\\
instance n=50 88.alb & 1 & 0 & Optimal &  0.30 & 8 &  0.00 &  0.00\\
instance n=50 89.alb & 1 & 0 & Optimal &  0.29 & 7 &  0.00 &  0.00\\
instance n=50 9.alb & 1 & 0 & Optimal &  0.29 & 9 &  0.00 &  0.00\\
instance n=50 90.alb & 1 & 0 & Optimal &  0.35 & 7 &  0.00 &  0.00\\
instance n=50 91.alb & 1 & 0 & Optimal &  0.30 & 7 &  0.00 &  0.00\\
instance n=50 92.alb & 1 & 0 & Optimal &  0.26 & 7 &  0.00 &  0.00\\
instance n=50 93.alb & 1 & 0 & Optimal &  0.28 & 7 &  0.00 &  0.00\\
instance n=50 94.alb & 1 & 0 & Optimal &  0.27 & 7 &  0.00 &  0.00\\
instance n=50 95.alb & 1 & 0 & Optimal &  0.27 & 7 &  0.00 &  0.00\\
instance n=50 96.alb & 1 & 0 & Optimal &  0.27 & 7 &  0.00 &  0.00\\
instance n=50 97.alb & 1 & 0 & Optimal &  0.30 & 7 &  0.00 &  0.00\\
instance n=50 98.alb & 1 & 0 & Optimal &  0.40 & 8 &  0.00 &  0.00\\
instance n=50 99.alb & 1 & 0 & Optimal &  0.29 & 7 &  0.00 &  0.00\\
\end{longtable}



\section{Alternative Model}

\subsection{CPO}

\begin{longtable}{lrrlrrrr}
\caption{Results for SALBP-1 Problems Alternative (CPO) (2100 Instances)}\\\toprule
Name & \shortstack{Nr\\Jobs} & \shortstack{Nr\\Machines} & Status & Time & Makespan & Bound & \shortstack{Gap\\Percent}\\ \midrule
\endhead
\bottomrule
\endfoot
instance n=1000 1.alb & 1 & 1 & Solution & 120.20 & 136 & 135.00 &  0.74\\
instance n=1000 10.alb & 1 & 1 & Solution & 120.07 & 140 & 140.00 &  0.00\\
instance n=1000 100.alb & 1 & 1 & Solution & 120.05 & 138 & 137.00 &  0.72\\
instance n=1000 101.alb & 1 & 1 & Solution & 120.13 & 552 & 505.00 &  8.51\\
instance n=1000 102.alb & 1 & 1 & Solution & 120.25 & 549 & 503.00 &  8.38\\
instance n=1000 103.alb & 1 & 1 & Solution & 120.11 & 556 & 503.00 &  9.53\\
instance n=1000 104.alb & 1 & 1 & Solution & 120.24 & 544 & 504.00 &  7.35\\
instance n=1000 105.alb & 1 & 1 & Solution & 120.13 & 536 & 499.00 &  6.90\\
instance n=1000 106.alb & 1 & 1 & Solution & 120.13 & 545 & 499.00 &  8.44\\
instance n=1000 107.alb & 1 & 1 & Solution & 120.05 & 531 & 496.00 &  6.59\\
instance n=1000 108.alb & 1 & 1 & Solution & 120.08 & 538 & 498.00 &  7.43\\
instance n=1000 109.alb & 1 & 1 & Solution & 120.05 & 541 & 500.00 &  7.58\\
instance n=1000 11.alb & 1 & 1 & Solution & 120.06 & 135 & 134.00 &  0.74\\
instance n=1000 110.alb & 1 & 1 & Solution & 120.12 & 548 & 501.00 &  8.58\\
instance n=1000 111.alb & 1 & 1 & Solution & 120.08 & 536 & 500.00 &  6.72\\
instance n=1000 112.alb & 1 & 1 & Solution & 120.15 & 542 & 499.00 &  7.93\\
instance n=1000 113.alb & 1 & 1 & Solution & 120.13 & 533 & 495.00 &  7.13\\
instance n=1000 114.alb & 1 & 1 & Solution & 120.12 & 539 & 502.00 &  6.86\\
instance n=1000 115.alb & 1 & 1 & Solution & 120.08 & 533 & 498.00 &  6.57\\
instance n=1000 116.alb & 1 & 1 & Solution & 120.07 & 536 & 496.00 &  7.46\\
instance n=1000 117.alb & 1 & 1 & Solution & 120.05 & 542 & 500.00 &  7.75\\
instance n=1000 118.alb & 1 & 1 & Solution & 120.04 & 556 & 509.00 &  8.45\\
instance n=1000 119.alb & 1 & 1 & Solution & 120.15 & 525 & 496.00 &  5.52\\
instance n=1000 12.alb & 1 & 1 & Solution & 120.04 & 134 & 134.00 &  0.00\\
instance n=1000 120.alb & 1 & 1 & Solution & 120.03 & 541 & 502.00 &  7.21\\
instance n=1000 121.alb & 1 & 1 & Solution & 120.10 & 533 & 496.00 &  6.94\\
instance n=1000 122.alb & 1 & 1 & Solution & 120.10 & 524 & 493.00 &  5.92\\
instance n=1000 123.alb & 1 & 1 & Solution & 120.05 & 545 & 504.00 &  7.52\\
instance n=1000 124.alb & 1 & 1 & Solution & 120.04 & 533 & 498.00 &  6.57\\
instance n=1000 125.alb & 1 & 1 & Solution & 120.04 & 537 & 499.00 &  7.08\\
instance n=1000 126.alb & 1 & 1 & Solution & 120.08 & 231 & 228.00 &  1.30\\
instance n=1000 127.alb & 1 & 1 & Solution & 120.08 & 222 & 221.00 &  0.45\\
instance n=1000 128.alb & 1 & 1 & Solution & 120.08 & 224 & 222.00 &  0.89\\
instance n=1000 129.alb & 1 & 1 & Solution & 120.08 & 224 & 223.00 &  0.45\\
instance n=1000 13.alb & 1 & 1 & Solution & 120.06 & 132 & 131.00 &  0.76\\
instance n=1000 130.alb & 1 & 1 & Solution & 120.08 & 223 & 221.00 &  0.90\\
instance n=1000 131.alb & 1 & 1 & Solution & 120.09 & 222 & 220.00 &  0.90\\
instance n=1000 132.alb & 1 & 1 & Solution & 120.06 & 216 & 214.00 &  0.93\\
instance n=1000 133.alb & 1 & 1 & Solution & 120.11 & 228 & 226.00 &  0.88\\
instance n=1000 134.alb & 1 & 1 & Solution & 120.10 & 217 & 215.00 &  0.92\\
instance n=1000 135.alb & 1 & 1 & Solution & 120.07 & 227 & 225.00 &  0.88\\
instance n=1000 136.alb & 1 & 1 & Solution & 120.08 & 230 & 228.00 &  0.87\\
instance n=1000 137.alb & 1 & 1 & Solution & 120.11 & 215 & 213.00 &  0.93\\
instance n=1000 138.alb & 1 & 1 & Solution & 120.10 & 223 & 221.00 &  0.90\\
instance n=1000 139.alb & 1 & 1 & Solution & 120.08 & 226 & 224.00 &  0.88\\
instance n=1000 14.alb & 1 & 1 & Solution & 120.05 & 137 & 136.00 &  0.73\\
instance n=1000 140.alb & 1 & 1 & Solution & 120.09 & 228 & 226.00 &  0.88\\
instance n=1000 141.alb & 1 & 1 & Solution & 120.09 & 217 & 215.00 &  0.92\\
instance n=1000 142.alb & 1 & 1 & Solution & 120.04 & 222 & 220.00 &  0.90\\
instance n=1000 143.alb & 1 & 1 & Solution & 120.04 & 215 & 213.00 &  0.93\\
instance n=1000 144.alb & 1 & 1 & Solution & 120.07 & 219 & 217.00 &  0.91\\
instance n=1000 145.alb & 1 & 1 & Solution & 120.08 & 222 & 220.00 &  0.90\\
instance n=1000 146.alb & 1 & 1 & Solution & 120.08 & 221 & 219.00 &  0.90\\
instance n=1000 147.alb & 1 & 1 & Solution & 120.05 & 232 & 229.00 &  1.29\\
instance n=1000 148.alb & 1 & 1 & Solution & 120.09 & 221 & 219.00 &  0.90\\
instance n=1000 149.alb & 1 & 1 & Solution & 120.08 & 239 & 237.00 &  0.84\\
instance n=1000 15.alb & 1 & 1 & Solution & 120.04 & 136 & 136.00 &  0.00\\
instance n=1000 150.alb & 1 & 1 & Solution & 120.06 & 224 & 222.00 &  0.89\\
instance n=1000 151.alb & 1 & 1 & Solution & 120.04 & 139 & 138.00 &  0.72\\
instance n=1000 152.alb & 1 & 1 & Solution & 120.06 & 137 & 136.00 &  0.73\\
instance n=1000 153.alb & 1 & 1 & Solution & 120.05 & 138 & 137.00 &  0.72\\
instance n=1000 154.alb & 1 & 1 & Solution & 120.06 & 140 & 140.00 &  0.00\\
instance n=1000 155.alb & 1 & 1 & Solution & 120.08 & 140 & 139.00 &  0.71\\
instance n=1000 156.alb & 1 & 1 & Solution & 120.08 & 142 & 141.00 &  0.70\\
instance n=1000 157.alb & 1 & 1 & Solution & 120.09 & 141 & 140.00 &  0.71\\
instance n=1000 158.alb & 1 & 1 & Solution & 120.03 & 136 & 136.00 &  0.00\\
instance n=1000 159.alb & 1 & 1 & Solution & 120.06 & 139 & 138.00 &  0.72\\
instance n=1000 16.alb & 1 & 1 & Solution & 120.06 & 137 & 137.00 &  0.00\\
instance n=1000 160.alb & 1 & 1 & Solution & 120.07 & 139 & 138.00 &  0.72\\
instance n=1000 161.alb & 1 & 1 & Solution & 120.06 & 133 & 133.00 &  0.00\\
instance n=1000 162.alb & 1 & 1 & Solution & 120.06 & 136 & 136.00 &  0.00\\
instance n=1000 163.alb & 1 & 1 & Solution & 120.04 & 140 & 139.00 &  0.71\\
instance n=1000 164.alb & 1 & 1 & Solution & 120.03 & 142 & 141.00 &  0.70\\
instance n=1000 165.alb & 1 & 1 & Solution & 120.06 & 136 & 135.00 &  0.74\\
instance n=1000 166.alb & 1 & 1 & Solution & 120.03 & 140 & 139.00 &  0.71\\
instance n=1000 167.alb & 1 & 1 & Solution & 120.04 & 140 & 139.00 &  0.71\\
instance n=1000 168.alb & 1 & 1 & Solution & 120.06 & 139 & 138.00 &  0.72\\
instance n=1000 169.alb & 1 & 1 & Solution & 120.02 & 135 & 134.00 &  0.74\\
instance n=1000 17.alb & 1 & 1 & Solution & 120.06 & 135 & 135.00 &  0.00\\
instance n=1000 170.alb & 1 & 1 & Solution & 120.07 & 135 & 134.00 &  0.74\\
instance n=1000 171.alb & 1 & 1 & Solution & 120.08 & 138 & 137.00 &  0.72\\
instance n=1000 172.alb & 1 & 1 & Solution & 120.04 & 136 & 135.00 &  0.74\\
instance n=1000 173.alb & 1 & 1 & Solution & 120.06 & 136 & 135.00 &  0.74\\
instance n=1000 174.alb & 1 & 1 & Solution & 120.07 & 137 & 136.00 &  0.73\\
instance n=1000 175.alb & 1 & 1 & Solution & 120.06 & 139 & 138.00 &  0.72\\
instance n=1000 176.alb & 1 & 1 & Solution & 120.11 & 538 & 500.00 &  7.06\\
instance n=1000 177.alb & 1 & 1 & Solution & 120.04 & 532 & 499.00 &  6.20\\
instance n=1000 178.alb & 1 & 1 & Solution & 120.10 & 553 & 506.00 &  8.50\\
instance n=1000 179.alb & 1 & 1 & Solution & 120.06 & 544 & 505.00 &  7.17\\
instance n=1000 18.alb & 1 & 1 & Solution & 120.05 & 134 & 134.00 &  0.00\\
instance n=1000 180.alb & 1 & 1 & Solution & 120.05 & 554 & 503.00 &  9.21\\
instance n=1000 181.alb & 1 & 1 & Solution & 120.02 & 549 & 505.00 &  8.01\\
instance n=1000 182.alb & 1 & 1 & Solution & 120.10 & 543 & 502.00 &  7.55\\
instance n=1000 183.alb & 1 & 1 & Solution & 120.08 & 539 & 500.00 &  7.24\\
instance n=1000 184.alb & 1 & 1 & Solution & 120.07 & 546 & 502.00 &  8.06\\
instance n=1000 185.alb & 1 & 1 & Solution & 120.08 & 545 & 503.00 &  7.71\\
instance n=1000 186.alb & 1 & 1 & Solution & 120.12 & 536 & 500.00 &  6.72\\
instance n=1000 187.alb & 1 & 1 & Solution & 120.07 & 553 & 505.00 &  8.68\\
instance n=1000 188.alb & 1 & 1 & Solution & 120.05 & 538 & 498.00 &  7.43\\
instance n=1000 189.alb & 1 & 1 & Solution & 120.05 & 533 & 498.00 &  6.57\\
instance n=1000 19.alb & 1 & 1 & Solution & 120.07 & 137 & 137.00 &  0.00\\
instance n=1000 190.alb & 1 & 1 & Solution & 120.12 & 543 & 501.00 &  7.73\\
instance n=1000 191.alb & 1 & 1 & Solution & 120.11 & 542 & 501.00 &  7.56\\
instance n=1000 192.alb & 1 & 1 & Solution & 120.10 & 539 & 501.00 &  7.05\\
instance n=1000 193.alb & 1 & 1 & Solution & 120.10 & 542 & 503.00 &  7.20\\
instance n=1000 194.alb & 1 & 1 & Solution & 120.09 & 540 & 502.00 &  7.04\\
instance n=1000 195.alb & 1 & 1 & Solution & 120.13 & 553 & 502.00 &  9.22\\
instance n=1000 196.alb & 1 & 1 & Solution & 120.09 & 545 & 500.00 &  8.26\\
instance n=1000 197.alb & 1 & 1 & Solution & 120.06 & 523 & 496.00 &  5.16\\
instance n=1000 198.alb & 1 & 1 & Solution & 120.11 & 548 & 503.00 &  8.21\\
instance n=1000 199.alb & 1 & 1 & Solution & 120.04 & 527 & 495.00 &  6.07\\
instance n=1000 2.alb & 1 & 1 & Solution & 120.04 & 138 & 137.00 &  0.72\\
instance n=1000 20.alb & 1 & 1 & Solution & 120.01 & 138 & 138.00 &  0.00\\
instance n=1000 200.alb & 1 & 1 & Solution & 120.04 & 529 & 498.00 &  5.86\\
instance n=1000 201.alb & 1 & 1 & Solution & 120.01 & 231 & 229.00 &  0.87\\
instance n=1000 202.alb & 1 & 1 & Solution & 120.04 & 228 & 225.00 &  1.32\\
instance n=1000 203.alb & 1 & 1 & Solution & 120.02 & 232 & 229.00 &  1.29\\
instance n=1000 204.alb & 1 & 1 & Solution & 120.06 & 231 & 228.00 &  1.30\\
instance n=1000 205.alb & 1 & 1 & Solution & 120.04 & 231 & 229.00 &  0.87\\
instance n=1000 206.alb & 1 & 1 & Solution & 120.06 & 231 & 229.00 &  0.87\\
instance n=1000 207.alb & 1 & 1 & Solution & 120.08 & 232 & 230.00 &  0.86\\
instance n=1000 208.alb & 1 & 1 & Solution & 120.04 & 232 & 229.00 &  1.29\\
instance n=1000 209.alb & 1 & 1 & Solution & 120.01 & 230 & 228.00 &  0.87\\
instance n=1000 21.alb & 1 & 1 & Solution & 120.04 & 138 & 138.00 &  0.00\\
instance n=1000 210.alb & 1 & 1 & Solution & 120.07 & 226 & 224.00 &  0.88\\
instance n=1000 211.alb & 1 & 1 & Solution & 120.07 & 221 & 219.00 &  0.90\\
instance n=1000 212.alb & 1 & 1 & Solution & 120.05 & 219 & 217.00 &  0.91\\
instance n=1000 213.alb & 1 & 1 & Solution & 120.07 & 236 & 233.00 &  1.27\\
instance n=1000 214.alb & 1 & 1 & Solution & 120.03 & 227 & 225.00 &  0.88\\
instance n=1000 215.alb & 1 & 1 & Solution & 120.08 & 225 & 223.00 &  0.89\\
instance n=1000 216.alb & 1 & 1 & Solution & 120.07 & 222 & 221.00 &  0.45\\
instance n=1000 217.alb & 1 & 1 & Solution & 120.08 & 227 & 225.00 &  0.88\\
instance n=1000 218.alb & 1 & 1 & Solution & 120.05 & 221 & 219.00 &  0.90\\
instance n=1000 219.alb & 1 & 1 & Solution & 120.02 & 234 & 232.00 &  0.85\\
instance n=1000 22.alb & 1 & 1 & Solution & 120.08 & 138 & 137.00 &  0.72\\
instance n=1000 220.alb & 1 & 1 & Solution & 120.06 & 227 & 225.00 &  0.88\\
instance n=1000 221.alb & 1 & 1 & Solution & 120.05 & 233 & 231.00 &  0.86\\
instance n=1000 222.alb & 1 & 1 & Solution & 120.09 & 224 & 221.00 &  1.34\\
instance n=1000 223.alb & 1 & 1 & Solution & 120.10 & 223 & 221.00 &  0.90\\
instance n=1000 224.alb & 1 & 1 & Solution & 120.08 & 229 & 226.00 &  1.31\\
instance n=1000 225.alb & 1 & 1 & Solution & 120.04 & 231 & 229.00 &  0.87\\
instance n=1000 226.alb & 1 & 1 & Solution & 120.08 & 137 & 136.00 &  0.73\\
instance n=1000 227.alb & 1 & 1 & Solution & 120.07 & 139 & 138.00 &  0.72\\
instance n=1000 228.alb & 1 & 1 & Solution & 120.09 & 134 & 133.00 &  0.75\\
instance n=1000 229.alb & 1 & 1 & Solution & 120.09 & 135 & 134.00 &  0.74\\
instance n=1000 23.alb & 1 & 1 & Solution & 120.03 & 136 & 136.00 &  0.00\\
instance n=1000 230.alb & 1 & 1 & Solution & 120.06 & 132 & 131.00 &  0.76\\
instance n=1000 231.alb & 1 & 1 & Solution & 120.11 & 139 & 138.00 &  0.72\\
instance n=1000 232.alb & 1 & 1 & Solution & 120.09 & 134 & 133.00 &  0.75\\
instance n=1000 233.alb & 1 & 1 & Solution & 120.07 & 136 & 135.00 &  0.74\\
instance n=1000 234.alb & 1 & 1 & Solution & 120.03 & 138 & 137.00 &  0.72\\
instance n=1000 235.alb & 1 & 1 & Solution & 120.08 & 134 & 133.00 &  0.75\\
instance n=1000 236.alb & 1 & 1 & Solution & 120.13 & 137 & 136.00 &  0.73\\
instance n=1000 237.alb & 1 & 1 & Solution & 120.10 & 139 & 138.00 &  0.72\\
instance n=1000 238.alb & 1 & 1 & Solution & 120.05 & 139 & 138.00 &  0.72\\
instance n=1000 239.alb & 1 & 1 & Solution & 120.06 & 136 & 135.00 &  0.74\\
instance n=1000 24.alb & 1 & 1 & Solution & 120.07 & 140 & 140.00 &  0.00\\
instance n=1000 240.alb & 1 & 1 & Solution & 120.02 & 136 & 135.00 &  0.74\\
instance n=1000 241.alb & 1 & 1 & Solution & 120.03 & 139 & 138.00 &  0.72\\
instance n=1000 242.alb & 1 & 1 & Solution & 120.05 & 136 & 135.00 &  0.74\\
instance n=1000 243.alb & 1 & 1 & Solution & 120.08 & 138 & 137.00 &  0.72\\
instance n=1000 244.alb & 1 & 1 & Solution & 120.08 & 138 & 137.00 &  0.72\\
instance n=1000 245.alb & 1 & 1 & Solution & 120.10 & 136 & 135.00 &  0.74\\
instance n=1000 246.alb & 1 & 1 & Solution & 120.04 & 136 & 135.00 &  0.74\\
instance n=1000 247.alb & 1 & 1 & Solution & 120.05 & 139 & 138.00 &  0.72\\
instance n=1000 248.alb & 1 & 1 & Solution & 120.05 & 140 & 138.00 &  1.43\\
instance n=1000 249.alb & 1 & 1 & Solution & 120.09 & 139 & 138.00 &  0.72\\
instance n=1000 25.alb & 1 & 1 & Solution & 120.04 & 136 & 136.00 &  0.00\\
instance n=1000 250.alb & 1 & 1 & Solution & 120.10 & 141 & 140.00 &  0.71\\
instance n=1000 251.alb & 1 & 1 & Solution & 120.07 & 558 & 502.00 & 10.04\\
instance n=1000 252.alb & 1 & 1 & Solution & 120.05 & 560 & 501.00 & 10.54\\
instance n=1000 253.alb & 1 & 1 & Solution & 120.04 & 555 & 502.00 &  9.55\\
instance n=1000 254.alb & 1 & 1 & Solution & 120.12 & 550 & 501.00 &  8.91\\
instance n=1000 255.alb & 1 & 1 & Solution & 120.11 & 547 & 498.00 &  8.96\\
instance n=1000 256.alb & 1 & 1 & Solution & 120.14 & 542 & 495.00 &  8.67\\
instance n=1000 257.alb & 1 & 1 & Solution & 120.04 & 559 & 502.00 & 10.20\\
instance n=1000 258.alb & 1 & 1 & Solution & 120.13 & 556 & 497.00 & 10.61\\
instance n=1000 259.alb & 1 & 1 & Solution & 120.18 & 545 & 496.00 &  8.99\\
instance n=1000 26.alb & 1 & 1 & Solution & 120.10 & 541 & 502.00 &  7.21\\
instance n=1000 260.alb & 1 & 1 & Solution & 120.09 & 547 & 495.00 &  9.51\\
instance n=1000 261.alb & 1 & 1 & Solution & 120.10 & 553 & 501.00 &  9.40\\
instance n=1000 262.alb & 1 & 1 & Solution & 120.14 & 534 & 495.00 &  7.30\\
instance n=1000 263.alb & 1 & 1 & Solution & 120.07 & 553 & 499.00 &  9.76\\
instance n=1000 264.alb & 1 & 1 & Solution & 120.12 & 546 & 499.00 &  8.61\\
instance n=1000 265.alb & 1 & 1 & Solution & 120.07 & 570 & 506.00 & 11.23\\
instance n=1000 266.alb & 1 & 1 & Solution & 120.12 & 554 & 500.00 &  9.75\\
instance n=1000 267.alb & 1 & 1 & Solution & 120.17 & 560 & 506.00 &  9.64\\
instance n=1000 268.alb & 1 & 1 & Solution & 120.05 & 544 & 497.00 &  8.64\\
instance n=1000 269.alb & 1 & 1 & Solution & 120.17 & 549 & 500.00 &  8.93\\
instance n=1000 27.alb & 1 & 1 & Solution & 120.12 & 542 & 502.00 &  7.38\\
instance n=1000 270.alb & 1 & 1 & Solution & 120.05 & 578 & 508.00 & 12.11\\
instance n=1000 271.alb & 1 & 1 & Solution & 120.10 & 543 & 497.00 &  8.47\\
instance n=1000 272.alb & 1 & 1 & Solution & 120.05 & 558 & 502.00 & 10.04\\
instance n=1000 273.alb & 1 & 1 & Solution & 120.10 & 552 & 500.00 &  9.42\\
instance n=1000 274.alb & 1 & 1 & Solution & 120.08 & 554 & 496.00 & 10.47\\
instance n=1000 275.alb & 1 & 1 & Solution & 120.06 & 563 & 504.00 & 10.48\\
instance n=1000 276.alb & 1 & 1 & Solution & 120.11 & 220 & 217.00 &  1.36\\
instance n=1000 277.alb & 1 & 1 & Solution & 120.11 & 228 & 225.00 &  1.32\\
instance n=1000 278.alb & 1 & 1 & Solution & 120.11 & 224 & 220.00 &  1.79\\
instance n=1000 279.alb & 1 & 1 & Solution & 120.04 & 218 & 215.00 &  1.38\\
instance n=1000 28.alb & 1 & 1 & Solution & 120.13 & 526 & 497.00 &  5.51\\
instance n=1000 280.alb & 1 & 1 & Solution & 120.13 & 229 & 226.00 &  1.31\\
instance n=1000 281.alb & 1 & 1 & Solution & 120.15 & 223 & 219.00 &  1.79\\
instance n=1000 282.alb & 1 & 1 & Solution & 120.14 & 217 & 214.00 &  1.38\\
instance n=1000 283.alb & 1 & 1 & Solution & 120.05 & 227 & 224.00 &  1.32\\
instance n=1000 284.alb & 1 & 1 & Solution & 120.16 & 220 & 217.00 &  1.36\\
instance n=1000 285.alb & 1 & 1 & Solution & 120.09 & 225 & 221.00 &  1.78\\
instance n=1000 286.alb & 1 & 1 & Solution & 120.17 & 225 & 221.00 &  1.78\\
instance n=1000 287.alb & 1 & 1 & Solution & 120.06 & 227 & 224.00 &  1.32\\
instance n=1000 288.alb & 1 & 1 & Solution & 120.14 & 222 & 219.00 &  1.35\\
instance n=1000 289.alb & 1 & 1 & Solution & 120.06 & 224 & 220.00 &  1.79\\
instance n=1000 29.alb & 1 & 1 & Solution & 120.10 & 530 & 498.00 &  6.04\\
instance n=1000 290.alb & 1 & 1 & Solution & 120.04 & 225 & 222.00 &  1.33\\
instance n=1000 291.alb & 1 & 1 & Solution & 120.05 & 228 & 225.00 &  1.32\\
instance n=1000 292.alb & 1 & 1 & Solution & 120.04 & 229 & 226.00 &  1.31\\
instance n=1000 293.alb & 1 & 1 & Solution & 120.15 & 228 & 225.00 &  1.32\\
instance n=1000 294.alb & 1 & 1 & Solution & 120.11 & 233 & 230.00 &  1.29\\
instance n=1000 295.alb & 1 & 1 & Solution & 120.12 & 230 & 227.00 &  1.30\\
instance n=1000 296.alb & 1 & 1 & Solution & 120.16 & 210 & 208.00 &  0.95\\
instance n=1000 297.alb & 1 & 1 & Solution & 120.17 & 219 & 217.00 &  0.91\\
instance n=1000 298.alb & 1 & 1 & Solution & 120.09 & 218 & 214.00 &  1.83\\
instance n=1000 299.alb & 1 & 1 & Solution & 120.09 & 229 & 226.00 &  1.31\\
instance n=1000 3.alb & 1 & 1 & Solution & 120.05 & 137 & 136.00 &  0.73\\
instance n=1000 30.alb & 1 & 1 & Solution & 120.06 & 546 & 506.00 &  7.33\\
instance n=1000 300.alb & 1 & 1 & Solution & 120.12 & 232 & 228.00 &  1.72\\
instance n=1000 301.alb & 1 & 1 & Solution & 120.10 & 137 & 137.00 &  0.00\\
instance n=1000 302.alb & 1 & 1 & Solution & 120.06 & 139 & 139.00 &  0.00\\
instance n=1000 303.alb & 1 & 1 & Solution & 120.09 & 139 & 138.00 &  0.72\\
instance n=1000 304.alb & 1 & 1 & Solution & 120.10 & 137 & 136.00 &  0.73\\
instance n=1000 305.alb & 1 & 1 & Solution & 120.13 & 140 & 140.00 &  0.00\\
instance n=1000 306.alb & 1 & 1 & Solution & 120.07 & 135 & 135.00 &  0.00\\
instance n=1000 307.alb & 1 & 1 & Solution & 120.03 & 136 & 136.00 &  0.00\\
instance n=1000 308.alb & 1 & 1 & Solution & 120.05 & 138 & 137.00 &  0.72\\
instance n=1000 309.alb & 1 & 1 & Solution & 120.14 & 135 & 135.00 &  0.00\\
instance n=1000 31.alb & 1 & 1 & Solution & 120.07 & 539 & 506.00 &  6.12\\
instance n=1000 310.alb & 1 & 1 & Solution & 120.07 & 142 & 141.00 &  0.70\\
instance n=1000 311.alb & 1 & 1 & Solution & 120.06 & 140 & 139.00 &  0.71\\
instance n=1000 312.alb & 1 & 1 & Solution & 120.11 & 135 & 135.00 &  0.00\\
instance n=1000 313.alb & 1 & 1 & Solution & 120.07 & 138 & 138.00 &  0.00\\
instance n=1000 314.alb & 1 & 1 & Solution & 120.14 & 142 & 142.00 &  0.00\\
instance n=1000 315.alb & 1 & 1 & Solution & 120.10 & 137 & 136.00 &  0.73\\
instance n=1000 316.alb & 1 & 1 & Solution & 120.07 & 138 & 137.00 &  0.72\\
instance n=1000 317.alb & 1 & 1 & Solution & 120.12 & 137 & 136.00 &  0.73\\
instance n=1000 318.alb & 1 & 1 & Solution & 120.08 & 138 & 138.00 &  0.00\\
instance n=1000 319.alb & 1 & 1 & Solution & 120.13 & 141 & 140.00 &  0.71\\
instance n=1000 32.alb & 1 & 1 & Solution & 120.10 & 527 & 502.00 &  4.74\\
instance n=1000 320.alb & 1 & 1 & Solution & 120.09 & 141 & 141.00 &  0.00\\
instance n=1000 321.alb & 1 & 1 & Solution & 120.08 & 140 & 140.00 &  0.00\\
instance n=1000 322.alb & 1 & 1 & Solution & 120.09 & 139 & 139.00 &  0.00\\
instance n=1000 323.alb & 1 & 1 & Solution & 120.09 & 138 & 138.00 &  0.00\\
instance n=1000 324.alb & 1 & 1 & Solution & 120.10 & 141 & 140.00 &  0.71\\
instance n=1000 325.alb & 1 & 1 & Solution & 120.05 & 139 & 138.00 &  0.72\\
instance n=1000 326.alb & 1 & 1 & Solution & 120.04 & 529 & 496.00 &  6.24\\
instance n=1000 327.alb & 1 & 1 & Solution & 120.05 & 535 & 503.00 &  5.98\\
instance n=1000 328.alb & 1 & 1 & Solution & 120.05 & 526 & 500.00 &  4.94\\
instance n=1000 329.alb & 1 & 1 & Solution & 120.09 & 533 & 502.00 &  5.82\\
instance n=1000 33.alb & 1 & 1 & Solution & 120.16 & 528 & 501.00 &  5.11\\
instance n=1000 330.alb & 1 & 1 & Solution & 120.17 & 525 & 498.00 &  5.14\\
instance n=1000 331.alb & 1 & 1 & Solution & 120.16 & 527 & 498.00 &  5.50\\
instance n=1000 332.alb & 1 & 1 & Solution & 120.04 & 522 & 495.00 &  5.17\\
instance n=1000 333.alb & 1 & 1 & Solution & 120.12 & 541 & 499.00 &  7.76\\
instance n=1000 334.alb & 1 & 1 & Solution & 120.18 & 521 & 498.00 &  4.41\\
instance n=1000 335.alb & 1 & 1 & Solution & 120.22 & 531 & 496.00 &  6.59\\
instance n=1000 336.alb & 1 & 1 & Solution & 120.08 & 523 & 497.00 &  4.97\\
instance n=1000 337.alb & 1 & 1 & Solution & 120.17 & 537 & 501.00 &  6.70\\
instance n=1000 338.alb & 1 & 1 & Solution & 120.06 & 535 & 502.00 &  6.17\\
instance n=1000 339.alb & 1 & 1 & Solution & 120.04 & 539 & 500.00 &  7.24\\
instance n=1000 34.alb & 1 & 1 & Solution & 120.04 & 555 & 507.00 &  8.65\\
instance n=1000 340.alb & 1 & 1 & Solution & 120.05 & 551 & 505.00 &  8.35\\
instance n=1000 341.alb & 1 & 1 & Solution & 120.16 & 539 & 503.00 &  6.68\\
instance n=1000 342.alb & 1 & 1 & Solution & 120.16 & 534 & 500.00 &  6.37\\
instance n=1000 343.alb & 1 & 1 & Solution & 120.11 & 538 & 500.00 &  7.06\\
instance n=1000 344.alb & 1 & 1 & Solution & 120.06 & 531 & 500.00 &  5.84\\
instance n=1000 345.alb & 1 & 1 & Solution & 120.17 & 535 & 502.00 &  6.17\\
instance n=1000 346.alb & 1 & 1 & Solution & 120.15 & 530 & 501.00 &  5.47\\
instance n=1000 347.alb & 1 & 1 & Solution & 120.05 & 533 & 498.00 &  6.57\\
instance n=1000 348.alb & 1 & 1 & Solution & 120.09 & 556 & 506.00 &  8.99\\
instance n=1000 349.alb & 1 & 1 & Solution & 120.10 & 539 & 503.00 &  6.68\\
instance n=1000 35.alb & 1 & 1 & Solution & 120.15 & 528 & 501.00 &  5.11\\
instance n=1000 350.alb & 1 & 1 & Solution & 120.11 & 524 & 496.00 &  5.34\\
instance n=1000 351.alb & 1 & 1 & Solution & 120.15 & 229 & 227.00 &  0.87\\
instance n=1000 352.alb & 1 & 1 & Solution & 120.14 & 229 & 227.00 &  0.87\\
instance n=1000 353.alb & 1 & 1 & Solution & 120.12 & 219 & 217.00 &  0.91\\
instance n=1000 354.alb & 1 & 1 & Solution & 120.14 & 224 & 222.00 &  0.89\\
instance n=1000 355.alb & 1 & 1 & Solution & 120.05 & 222 & 220.00 &  0.90\\
instance n=1000 356.alb & 1 & 1 & Solution & 120.10 & 228 & 226.00 &  0.88\\
instance n=1000 357.alb & 1 & 1 & Solution & 120.12 & 215 & 213.00 &  0.93\\
instance n=1000 358.alb & 1 & 1 & Solution & 120.07 & 221 & 219.00 &  0.90\\
instance n=1000 359.alb & 1 & 1 & Solution & 120.14 & 224 & 222.00 &  0.89\\
instance n=1000 36.alb & 1 & 1 & Solution & 120.08 & 524 & 497.00 &  5.15\\
instance n=1000 360.alb & 1 & 1 & Solution & 120.17 & 231 & 229.00 &  0.87\\
instance n=1000 361.alb & 1 & 1 & Solution & 120.08 & 217 & 215.00 &  0.92\\
instance n=1000 362.alb & 1 & 1 & Solution & 120.10 & 224 & 223.00 &  0.45\\
instance n=1000 363.alb & 1 & 1 & Solution & 120.05 & 217 & 215.00 &  0.92\\
instance n=1000 364.alb & 1 & 1 & Solution & 120.06 & 223 & 221.00 &  0.90\\
instance n=1000 365.alb & 1 & 1 & Solution & 120.12 & 229 & 227.00 &  0.87\\
instance n=1000 366.alb & 1 & 1 & Solution & 120.11 & 230 & 228.00 &  0.87\\
instance n=1000 367.alb & 1 & 1 & Solution & 120.05 & 229 & 227.00 &  0.87\\
instance n=1000 368.alb & 1 & 1 & Solution & 120.14 & 228 & 226.00 &  0.88\\
instance n=1000 369.alb & 1 & 1 & Solution & 120.12 & 222 & 220.00 &  0.90\\
instance n=1000 37.alb & 1 & 1 & Solution & 120.07 & 550 & 506.00 &  8.00\\
instance n=1000 370.alb & 1 & 1 & Solution & 120.09 & 225 & 223.00 &  0.89\\
instance n=1000 371.alb & 1 & 1 & Solution & 120.16 & 221 & 220.00 &  0.45\\
instance n=1000 372.alb & 1 & 1 & Solution & 120.14 & 232 & 230.00 &  0.86\\
instance n=1000 373.alb & 1 & 1 & Solution & 120.09 & 220 & 219.00 &  0.45\\
instance n=1000 374.alb & 1 & 1 & Solution & 120.12 & 220 & 219.00 &  0.45\\
instance n=1000 375.alb & 1 & 1 & Solution & 120.06 & 229 & 227.00 &  0.87\\
instance n=1000 376.alb & 1 & 1 & Solution & 120.06 & 133 & 132.00 &  0.75\\
instance n=1000 377.alb & 1 & 1 & Solution & 120.05 & 137 & 137.00 &  0.00\\
instance n=1000 378.alb & 1 & 1 & Solution & 120.03 & 135 & 134.00 &  0.74\\
instance n=1000 379.alb & 1 & 1 & Solution & 120.09 & 138 & 137.00 &  0.72\\
instance n=1000 38.alb & 1 & 1 & Solution & 120.24 & 545 & 504.00 &  7.52\\
instance n=1000 380.alb & 1 & 1 & Solution & 120.09 & 135 & 134.00 &  0.74\\
instance n=1000 381.alb & 1 & 1 & Solution & 120.05 & 138 & 138.00 &  0.00\\
instance n=1000 382.alb & 1 & 1 & Solution & 120.22 & 132 & 131.00 &  0.76\\
instance n=1000 383.alb & 1 & 1 & Solution & 120.12 & 139 & 138.00 &  0.72\\
instance n=1000 384.alb & 1 & 1 & Solution & 120.15 & 140 & 139.00 &  0.71\\
instance n=1000 385.alb & 1 & 1 & Solution & 120.11 & 136 & 135.00 &  0.74\\
instance n=1000 386.alb & 1 & 1 & Solution & 120.11 & 140 & 139.00 &  0.71\\
instance n=1000 387.alb & 1 & 1 & Solution & 120.05 & 138 & 137.00 &  0.72\\
instance n=1000 388.alb & 1 & 1 & Solution & 120.06 & 137 & 137.00 &  0.00\\
instance n=1000 389.alb & 1 & 1 & Solution & 120.06 & 137 & 136.00 &  0.73\\
instance n=1000 39.alb & 1 & 1 & Solution & 120.20 & 545 & 507.00 &  6.97\\
instance n=1000 390.alb & 1 & 1 & Solution & 120.15 & 137 & 136.00 &  0.73\\
instance n=1000 391.alb & 1 & 1 & Solution & 120.10 & 136 & 135.00 &  0.74\\
instance n=1000 392.alb & 1 & 1 & Solution & 120.20 & 137 & 136.00 &  0.73\\
instance n=1000 393.alb & 1 & 1 & Solution & 120.07 & 137 & 136.00 &  0.73\\
instance n=1000 394.alb & 1 & 1 & Solution & 120.05 & 140 & 138.00 &  1.43\\
instance n=1000 395.alb & 1 & 1 & Solution & 120.12 & 140 & 139.00 &  0.71\\
instance n=1000 396.alb & 1 & 1 & Solution & 120.05 & 137 & 136.00 &  0.73\\
instance n=1000 397.alb & 1 & 1 & Solution & 120.09 & 141 & 140.00 &  0.71\\
instance n=1000 398.alb & 1 & 1 & Solution & 120.10 & 135 & 134.00 &  0.74\\
instance n=1000 399.alb & 1 & 1 & Solution & 120.10 & 140 & 139.00 &  0.71\\
instance n=1000 4.alb & 1 & 1 & Solution & 120.10 & 139 & 138.00 &  0.72\\
instance n=1000 40.alb & 1 & 1 & Solution & 120.12 & 519 & 496.00 &  4.43\\
instance n=1000 400.alb & 1 & 1 & Solution & 120.08 & 141 & 140.00 &  0.71\\
instance n=1000 401.alb & 1 & 1 & Solution & 120.24 & 539 & 497.00 &  7.79\\
instance n=1000 402.alb & 1 & 1 & Solution & 120.03 & 552 & 500.00 &  9.42\\
instance n=1000 403.alb & 1 & 1 & Solution & 120.15 & 549 & 500.00 &  8.93\\
instance n=1000 404.alb & 1 & 1 & Solution & 120.13 & 543 & 500.00 &  7.92\\
instance n=1000 405.alb & 1 & 1 & Solution & 120.25 & 559 & 501.00 & 10.38\\
instance n=1000 406.alb & 1 & 1 & Solution & 120.11 & 533 & 495.00 &  7.13\\
instance n=1000 407.alb & 1 & 1 & Solution & 120.06 & 548 & 498.00 &  9.12\\
instance n=1000 408.alb & 1 & 1 & Solution & 120.16 & 552 & 501.00 &  9.24\\
instance n=1000 409.alb & 1 & 1 & Solution & 120.25 & 548 & 504.00 &  8.03\\
instance n=1000 41.alb & 1 & 1 & Solution & 120.04 & 527 & 500.00 &  5.12\\
instance n=1000 410.alb & 1 & 1 & Solution & 120.11 & 569 & 505.00 & 11.25\\
instance n=1000 411.alb & 1 & 1 & Solution & 120.19 & 546 & 498.00 &  8.79\\
instance n=1000 412.alb & 1 & 1 & Solution & 120.11 & 550 & 499.00 &  9.27\\
instance n=1000 413.alb & 1 & 1 & Solution & 120.07 & 549 & 503.00 &  8.38\\
instance n=1000 414.alb & 1 & 1 & Solution & 120.08 & 547 & 501.00 &  8.41\\
instance n=1000 415.alb & 1 & 1 & Solution & 120.08 & 545 & 501.00 &  8.07\\
instance n=1000 416.alb & 1 & 1 & Solution & 120.23 & 550 & 502.00 &  8.73\\
instance n=1000 417.alb & 1 & 1 & Solution & 120.08 & 580 & 512.00 & 11.72\\
instance n=1000 418.alb & 1 & 1 & Solution & 120.08 & 549 & 501.00 &  8.74\\
instance n=1000 419.alb & 1 & 1 & Solution & 120.22 & 574 & 510.00 & 11.15\\
instance n=1000 42.alb & 1 & 1 & Solution & 120.06 & 518 & 497.00 &  4.05\\
instance n=1000 420.alb & 1 & 1 & Solution & 120.25 & 553 & 501.00 &  9.40\\
instance n=1000 421.alb & 1 & 1 & Solution & 120.11 & 545 & 499.00 &  8.44\\
instance n=1000 422.alb & 1 & 1 & Solution & 120.07 & 543 & 495.00 &  8.84\\
instance n=1000 423.alb & 1 & 1 & Solution & 120.14 & 559 & 500.00 & 10.55\\
instance n=1000 424.alb & 1 & 1 & Solution & 120.09 & 535 & 495.00 &  7.48\\
instance n=1000 425.alb & 1 & 1 & Solution & 120.08 & 559 & 504.00 &  9.84\\
instance n=1000 426.alb & 1 & 1 & Solution & 120.22 & 227 & 224.00 &  1.32\\
instance n=1000 427.alb & 1 & 1 & Solution & 120.04 & 232 & 229.00 &  1.29\\
instance n=1000 428.alb & 1 & 1 & Solution & 120.12 & 226 & 224.00 &  0.88\\
instance n=1000 429.alb & 1 & 1 & Solution & 120.15 & 238 & 235.00 &  1.26\\
instance n=1000 43.alb & 1 & 1 & Solution & 120.23 & 524 & 496.00 &  5.34\\
instance n=1000 430.alb & 1 & 1 & Solution & 120.12 & 222 & 220.00 &  0.90\\
instance n=1000 431.alb & 1 & 1 & Solution & 120.20 & 232 & 230.00 &  0.86\\
instance n=1000 432.alb & 1 & 1 & Solution & 120.15 & 230 & 227.00 &  1.30\\
instance n=1000 433.alb & 1 & 1 & Solution & 120.08 & 232 & 229.00 &  1.29\\
instance n=1000 434.alb & 1 & 1 & Solution & 120.08 & 214 & 212.00 &  0.93\\
instance n=1000 435.alb & 1 & 1 & Solution & 120.25 & 229 & 227.00 &  0.87\\
instance n=1000 436.alb & 1 & 1 & Solution & 120.05 & 230 & 226.00 &  1.74\\
instance n=1000 437.alb & 1 & 1 & Solution & 120.32 & 224 & 222.00 &  0.89\\
instance n=1000 438.alb & 1 & 1 & Solution & 120.05 & 223 & 221.00 &  0.90\\
instance n=1000 439.alb & 1 & 1 & Solution & 120.07 & 227 & 225.00 &  0.88\\
instance n=1000 44.alb & 1 & 1 & Solution & 120.11 & 543 & 502.00 &  7.55\\
instance n=1000 440.alb & 1 & 1 & Solution & 120.15 & 228 & 225.00 &  1.32\\
instance n=1000 441.alb & 1 & 1 & Solution & 120.07 & 224 & 221.00 &  1.34\\
instance n=1000 442.alb & 1 & 1 & Solution & 120.08 & 233 & 230.00 &  1.29\\
instance n=1000 443.alb & 1 & 1 & Solution & 120.18 & 220 & 217.00 &  1.36\\
instance n=1000 444.alb & 1 & 1 & Solution & 120.10 & 225 & 222.00 &  1.33\\
instance n=1000 445.alb & 1 & 1 & Solution & 120.15 & 233 & 229.00 &  1.72\\
instance n=1000 446.alb & 1 & 1 & Solution & 120.09 & 230 & 228.00 &  0.87\\
instance n=1000 447.alb & 1 & 1 & Solution & 120.19 & 224 & 221.00 &  1.34\\
instance n=1000 448.alb & 1 & 1 & Solution & 120.08 & 224 & 222.00 &  0.89\\
instance n=1000 449.alb & 1 & 1 & Solution & 120.05 & 236 & 232.00 &  1.69\\
instance n=1000 45.alb & 1 & 1 & Solution & 120.05 & 509 & 492.00 &  3.34\\
instance n=1000 450.alb & 1 & 1 & Solution & 120.09 & 222 & 220.00 &  0.90\\
instance n=1000 451.alb & 1 & 1 & Solution & 120.12 & 138 & 136.00 &  1.45\\
instance n=1000 452.alb & 1 & 1 & Solution & 120.04 & 133 & 132.00 &  0.75\\
instance n=1000 453.alb & 1 & 1 & Solution & 120.11 & 140 & 138.00 &  1.43\\
instance n=1000 454.alb & 1 & 1 & Solution & 120.10 & 141 & 139.00 &  1.42\\
instance n=1000 455.alb & 1 & 1 & Solution & 120.06 & 138 & 136.00 &  1.45\\
instance n=1000 456.alb & 1 & 1 & Solution & 120.10 & 137 & 135.00 &  1.46\\
instance n=1000 457.alb & 1 & 1 & Solution & 120.09 & 139 & 137.00 &  1.44\\
instance n=1000 458.alb & 1 & 1 & Solution & 120.09 & 136 & 135.00 &  0.74\\
instance n=1000 459.alb & 1 & 1 & Solution & 120.11 & 139 & 137.00 &  1.44\\
instance n=1000 46.alb & 1 & 1 & Solution & 120.10 & 526 & 498.00 &  5.32\\
instance n=1000 460.alb & 1 & 1 & Solution & 120.08 & 139 & 138.00 &  0.72\\
instance n=1000 461.alb & 1 & 1 & Solution & 120.12 & 138 & 137.00 &  0.72\\
instance n=1000 462.alb & 1 & 1 & Solution & 120.11 & 138 & 136.00 &  1.45\\
instance n=1000 463.alb & 1 & 1 & Solution & 120.10 & 138 & 136.00 &  1.45\\
instance n=1000 464.alb & 1 & 1 & Solution & 120.06 & 140 & 138.00 &  1.43\\
instance n=1000 465.alb & 1 & 1 & Solution & 120.11 & 140 & 138.00 &  1.43\\
instance n=1000 466.alb & 1 & 1 & Solution & 120.06 & 135 & 133.00 &  1.48\\
instance n=1000 467.alb & 1 & 1 & Solution & 120.10 & 139 & 138.00 &  0.72\\
instance n=1000 468.alb & 1 & 1 & Solution & 120.06 & 138 & 137.00 &  0.72\\
instance n=1000 469.alb & 1 & 1 & Solution & 120.09 & 139 & 137.00 &  1.44\\
instance n=1000 47.alb & 1 & 1 & Solution & 120.16 & 526 & 499.00 &  5.13\\
instance n=1000 470.alb & 1 & 1 & Solution & 120.05 & 136 & 135.00 &  0.74\\
instance n=1000 471.alb & 1 & 1 & Solution & 120.10 & 137 & 135.00 &  1.46\\
instance n=1000 472.alb & 1 & 1 & Solution & 120.08 & 142 & 140.00 &  1.41\\
instance n=1000 473.alb & 1 & 1 & Solution & 120.07 & 137 & 135.00 &  1.46\\
instance n=1000 474.alb & 1 & 1 & Solution & 120.06 & 138 & 136.00 &  1.45\\
instance n=1000 475.alb & 1 & 1 & Solution & 120.11 & 138 & 136.00 &  1.45\\
instance n=1000 476.alb & 1 & 1 & Solution & 120.05 & 575 & 503.00 & 12.52\\
instance n=1000 477.alb & 1 & 1 & Solution & 120.12 & 582 & 507.00 & 12.89\\
instance n=1000 478.alb & 1 & 1 & Unknown & 120.09 & - & - & -\\
instance n=1000 479.alb & 1 & 1 & Solution & 120.04 & 573 & 503.00 & 12.22\\
instance n=1000 48.alb & 1 & 1 & Solution & 120.06 & 553 & 508.00 &  8.14\\
instance n=1000 480.alb & 1 & 1 & Solution & 120.10 & 566 & 498.00 & 12.01\\
instance n=1000 481.alb & 1 & 1 & Solution & 120.06 & 579 & 504.00 & 12.95\\
instance n=1000 482.alb & 1 & 1 & Solution & 120.10 & 595 & 505.00 & 15.13\\
instance n=1000 483.alb & 1 & 1 & Solution & 120.05 & 565 & 499.00 & 11.68\\
instance n=1000 484.alb & 1 & 1 & Solution & 120.06 & 591 & 508.00 & 14.04\\
instance n=1000 485.alb & 1 & 1 & Solution & 120.05 & 578 & 505.00 & 12.63\\
instance n=1000 486.alb & 1 & 1 & Solution & 120.11 & 569 & 500.00 & 12.13\\
instance n=1000 487.alb & 1 & 1 & Solution & 120.05 & 579 & 502.00 & 13.30\\
instance n=1000 488.alb & 1 & 1 & Solution & 120.12 & 571 & 502.00 & 12.08\\
instance n=1000 489.alb & 1 & 1 & Solution & 120.11 & 564 & 498.00 & 11.70\\
instance n=1000 49.alb & 1 & 1 & Solution & 120.09 & 529 & 500.00 &  5.48\\
instance n=1000 490.alb & 1 & 1 & Solution & 120.18 & 573 & 501.00 & 12.57\\
instance n=1000 491.alb & 1 & 1 & Solution & 120.06 & 566 & 500.00 & 11.66\\
instance n=1000 492.alb & 1 & 1 & Solution & 120.19 & 585 & 509.00 & 12.99\\
instance n=1000 493.alb & 1 & 1 & Solution & 120.08 & 556 & 495.00 & 10.97\\
instance n=1000 494.alb & 1 & 1 & Solution & 120.06 & 571 & 500.00 & 12.43\\
instance n=1000 495.alb & 1 & 1 & Solution & 120.13 & 587 & 507.00 & 13.63\\
instance n=1000 496.alb & 1 & 1 & Solution & 120.10 & 559 & 495.00 & 11.45\\
instance n=1000 497.alb & 1 & 1 & Solution & 120.12 & 561 & 499.00 & 11.05\\
instance n=1000 498.alb & 1 & 1 & Solution & 120.18 & 581 & 506.00 & 12.91\\
instance n=1000 499.alb & 1 & 1 & Solution & 120.17 & 565 & 499.00 & 11.68\\
instance n=1000 5.alb & 1 & 1 & Solution & 120.18 & 136 & 135.00 &  0.74\\
instance n=1000 50.alb & 1 & 1 & Solution & 120.04 & 512 & 493.00 &  3.71\\
instance n=1000 500.alb & 1 & 1 & Solution & 120.14 & 568 & 503.00 & 11.44\\
instance n=1000 501.alb & 1 & 1 & Solution & 120.13 & 232 & 227.00 &  2.16\\
instance n=1000 502.alb & 1 & 1 & Solution & 120.10 & 229 & 224.00 &  2.18\\
instance n=1000 503.alb & 1 & 1 & Solution & 120.14 & 230 & 224.00 &  2.61\\
instance n=1000 504.alb & 1 & 1 & Solution & 120.06 & 233 & 227.00 &  2.58\\
instance n=1000 505.alb & 1 & 1 & Solution & 120.05 & 218 & 213.00 &  2.29\\
instance n=1000 506.alb & 1 & 1 & Solution & 120.10 & 228 & 223.00 &  2.19\\
instance n=1000 507.alb & 1 & 1 & Solution & 120.15 & 225 & 220.00 &  2.22\\
instance n=1000 508.alb & 1 & 1 & Solution & 120.06 & 222 & 219.00 &  1.35\\
instance n=1000 509.alb & 1 & 1 & Solution & 120.09 & 230 & 225.00 &  2.17\\
instance n=1000 51.alb & 1 & 1 & Solution & 120.09 & 228 & 226.00 &  0.88\\
instance n=1000 510.alb & 1 & 1 & Solution & 120.07 & 232 & 226.00 &  2.59\\
instance n=1000 511.alb & 1 & 1 & Solution & 120.16 & 235 & 230.00 &  2.13\\
instance n=1000 512.alb & 1 & 1 & Solution & 120.10 & 224 & 219.00 &  2.23\\
instance n=1000 513.alb & 1 & 1 & Solution & 120.11 & 224 & 219.00 &  2.23\\
instance n=1000 514.alb & 1 & 1 & Solution & 120.08 & 232 & 226.00 &  2.59\\
instance n=1000 515.alb & 1 & 1 & Solution & 120.18 & 226 & 221.00 &  2.21\\
instance n=1000 516.alb & 1 & 1 & Solution & 120.07 & 234 & 229.00 &  2.14\\
instance n=1000 517.alb & 1 & 1 & Solution & 120.06 & 226 & 221.00 &  2.21\\
instance n=1000 518.alb & 1 & 1 & Solution & 120.12 & 224 & 220.00 &  1.79\\
instance n=1000 519.alb & 1 & 1 & Solution & 120.16 & 226 & 221.00 &  2.21\\
instance n=1000 52.alb & 1 & 1 & Solution & 120.08 & 230 & 228.00 &  0.87\\
instance n=1000 520.alb & 1 & 1 & Solution & 120.06 & 231 & 226.00 &  2.16\\
instance n=1000 521.alb & 1 & 1 & Solution & 120.20 & 235 & 229.00 &  2.55\\
instance n=1000 522.alb & 1 & 1 & Solution & 120.08 & 220 & 215.00 &  2.27\\
instance n=1000 523.alb & 1 & 1 & Solution & 120.07 & 225 & 220.00 &  2.22\\
instance n=1000 524.alb & 1 & 1 & Solution & 120.05 & 231 & 225.00 &  2.60\\
instance n=1000 525.alb & 1 & 1 & Solution & 120.08 & 225 & 221.00 &  1.78\\
instance n=1000 53.alb & 1 & 1 & Solution & 120.07 & 228 & 227.00 &  0.44\\
instance n=1000 54.alb & 1 & 1 & Solution & 120.05 & 221 & 219.00 &  0.90\\
instance n=1000 55.alb & 1 & 1 & Solution & 120.08 & 218 & 217.00 &  0.46\\
instance n=1000 56.alb & 1 & 1 & Solution & 120.10 & 229 & 228.00 &  0.44\\
instance n=1000 57.alb & 1 & 1 & Solution & 120.08 & 225 & 224.00 &  0.44\\
instance n=1000 58.alb & 1 & 1 & Solution & 120.07 & 225 & 224.00 &  0.44\\
instance n=1000 59.alb & 1 & 1 & Solution & 120.08 & 224 & 223.00 &  0.45\\
instance n=1000 6.alb & 1 & 1 & Solution & 120.08 & 142 & 141.00 &  0.70\\
instance n=1000 60.alb & 1 & 1 & Solution & 120.08 & 232 & 230.00 &  0.86\\
instance n=1000 61.alb & 1 & 1 & Solution & 120.08 & 231 & 229.00 &  0.87\\
instance n=1000 62.alb & 1 & 1 & Solution & 120.06 & 224 & 223.00 &  0.45\\
instance n=1000 63.alb & 1 & 1 & Solution & 120.05 & 228 & 227.00 &  0.44\\
instance n=1000 64.alb & 1 & 1 & Solution & 120.07 & 231 & 229.00 &  0.87\\
instance n=1000 65.alb & 1 & 1 & Solution & 120.06 & 226 & 225.00 &  0.44\\
instance n=1000 66.alb & 1 & 1 & Solution & 120.04 & 229 & 227.00 &  0.87\\
instance n=1000 67.alb & 1 & 1 & Solution & 120.08 & 224 & 223.00 &  0.45\\
instance n=1000 68.alb & 1 & 1 & Solution & 120.07 & 228 & 226.00 &  0.88\\
instance n=1000 69.alb & 1 & 1 & Solution & 120.09 & 225 & 224.00 &  0.44\\
instance n=1000 7.alb & 1 & 1 & Solution & 120.08 & 137 & 136.00 &  0.73\\
instance n=1000 70.alb & 1 & 1 & Solution & 120.07 & 230 & 228.00 &  0.87\\
instance n=1000 71.alb & 1 & 1 & Solution & 120.07 & 231 & 230.00 &  0.43\\
instance n=1000 72.alb & 1 & 1 & Solution & 120.06 & 223 & 222.00 &  0.45\\
instance n=1000 73.alb & 1 & 1 & Solution & 120.06 & 222 & 221.00 &  0.45\\
instance n=1000 74.alb & 1 & 1 & Solution & 120.07 & 228 & 227.00 &  0.44\\
instance n=1000 75.alb & 1 & 1 & Solution & 120.08 & 229 & 227.00 &  0.87\\
instance n=1000 76.alb & 1 & 1 & Solution & 120.08 & 137 & 136.00 &  0.73\\
instance n=1000 77.alb & 1 & 1 & Solution & 120.07 & 136 & 136.00 &  0.00\\
instance n=1000 78.alb & 1 & 1 & Solution & 120.05 & 139 & 138.00 &  0.72\\
instance n=1000 79.alb & 1 & 1 & Solution & 120.08 & 142 & 142.00 &  0.00\\
instance n=1000 8.alb & 1 & 1 & Solution & 120.05 & 139 & 138.00 &  0.72\\
instance n=1000 80.alb & 1 & 1 & Solution & 120.08 & 141 & 140.00 &  0.71\\
instance n=1000 81.alb & 1 & 1 & Solution & 120.05 & 137 & 136.00 &  0.73\\
instance n=1000 82.alb & 1 & 1 & Solution & 120.10 & 136 & 136.00 &  0.00\\
instance n=1000 83.alb & 1 & 1 & Solution & 120.05 & 140 & 140.00 &  0.00\\
instance n=1000 84.alb & 1 & 1 & Solution & 120.07 & 135 & 135.00 &  0.00\\
instance n=1000 85.alb & 1 & 1 & Solution & 120.05 & 137 & 136.00 &  0.73\\
instance n=1000 86.alb & 1 & 1 & Solution & 120.06 & 139 & 138.00 &  0.72\\
instance n=1000 87.alb & 1 & 1 & Solution & 120.08 & 141 & 140.00 &  0.71\\
instance n=1000 88.alb & 1 & 1 & Solution & 120.06 & 141 & 140.00 &  0.71\\
instance n=1000 89.alb & 1 & 1 & Solution & 120.07 & 141 & 140.00 &  0.71\\
instance n=1000 9.alb & 1 & 1 & Solution & 120.07 & 135 & 134.00 &  0.74\\
instance n=1000 90.alb & 1 & 1 & Solution & 120.07 & 138 & 138.00 &  0.00\\
instance n=1000 91.alb & 1 & 1 & Solution & 120.07 & 141 & 141.00 &  0.00\\
instance n=1000 92.alb & 1 & 1 & Solution & 120.07 & 136 & 136.00 &  0.00\\
instance n=1000 93.alb & 1 & 1 & Solution & 120.07 & 137 & 137.00 &  0.00\\
instance n=1000 94.alb & 1 & 1 & Solution & 120.06 & 138 & 137.00 &  0.72\\
instance n=1000 95.alb & 1 & 1 & Solution & 120.09 & 136 & 136.00 &  0.00\\
instance n=1000 96.alb & 1 & 1 & Solution & 120.05 & 138 & 137.00 &  0.72\\
instance n=1000 97.alb & 1 & 1 & Solution & 120.07 & 139 & 138.00 &  0.72\\
instance n=1000 98.alb & 1 & 1 & Solution & 120.07 & 136 & 136.00 &  0.00\\
instance n=1000 99.alb & 1 & 1 & Solution & 120.04 & 137 & 136.00 &  0.73\\
instance n=100 1.alb & 1 & 1 & Solution & 120.12 & 23 & 23.00 &  0.00\\
instance n=100 10.alb & 1 & 1 & Solution & 120.02 & 22 & 22.00 &  0.00\\
instance n=100 100.alb & 1 & 1 & Solution & 120.02 & 25 & 25.00 &  0.00\\
instance n=100 101.alb & 1 & 1 & Solution & 120.02 & 15 & 15.00 &  0.00\\
instance n=100 102.alb & 1 & 1 & Optimal &  4.76 & 14 & 14.00 &  0.00\\
instance n=100 103.alb & 1 & 1 & Optimal &  8.81 & 14 & 14.00 &  0.00\\
instance n=100 104.alb & 1 & 1 & Optimal & 77.15 & 14 & 14.00 &  0.00\\
instance n=100 105.alb & 1 & 1 & Optimal & 28.61 & 13 & 13.00 &  0.00\\
instance n=100 106.alb & 1 & 1 & Optimal &  1.74 & 14 & 14.00 &  0.00\\
instance n=100 107.alb & 1 & 1 & Optimal & 26.66 & 14 & 14.00 &  0.00\\
instance n=100 108.alb & 1 & 1 & Solution & 120.02 & 14 & 14.00 &  0.00\\
instance n=100 109.alb & 1 & 1 & Optimal & 69.08 & 15 & 15.00 &  0.00\\
instance n=100 11.alb & 1 & 1 & Solution & 120.02 & 24 & 24.00 &  0.00\\
instance n=100 110.alb & 1 & 1 & Optimal & 18.47 & 13 & 13.00 &  0.00\\
instance n=100 111.alb & 1 & 1 & Solution & 120.02 & 16 & 16.00 &  0.00\\
instance n=100 112.alb & 1 & 1 & Optimal & 21.63 & 13 & 13.00 &  0.00\\
instance n=100 113.alb & 1 & 1 & Optimal &  6.45 & 14 & 14.00 &  0.00\\
instance n=100 114.alb & 1 & 1 & Optimal & 22.13 & 13 & 13.00 &  0.00\\
instance n=100 115.alb & 1 & 1 & Optimal & 51.29 & 14 & 14.00 &  0.00\\
instance n=100 116.alb & 1 & 1 & Optimal & 107.09 & 16 & 16.00 &  0.00\\
instance n=100 117.alb & 1 & 1 & Optimal & 90.37 & 15 & 15.00 &  0.00\\
instance n=100 118.alb & 1 & 1 & Optimal & 53.38 & 15 & 15.00 &  0.00\\
instance n=100 119.alb & 1 & 1 & Optimal & 63.43 & 14 & 14.00 &  0.00\\
instance n=100 12.alb & 1 & 1 & Solution & 120.02 & 25 & 25.00 &  0.00\\
instance n=100 120.alb & 1 & 1 & Optimal & 29.64 & 14 & 14.00 &  0.00\\
instance n=100 121.alb & 1 & 1 & Optimal & 82.84 & 15 & 15.00 &  0.00\\
instance n=100 122.alb & 1 & 1 & Optimal & 14.12 & 13 & 13.00 &  0.00\\
instance n=100 123.alb & 1 & 1 & Optimal & 64.76 & 15 & 15.00 &  0.00\\
instance n=100 124.alb & 1 & 1 & Optimal & 55.04 & 15 & 15.00 &  0.00\\
instance n=100 125.alb & 1 & 1 & Optimal & 38.80 & 14 & 14.00 &  0.00\\
instance n=100 126.alb & 1 & 1 & Solution & 120.01 & 51 & 49.00 &  3.92\\
instance n=100 127.alb & 1 & 1 & Solution & 120.02 & 52 & 49.00 &  5.77\\
instance n=100 128.alb & 1 & 1 & Solution & 120.02 & 57 & 52.00 &  8.77\\
instance n=100 129.alb & 1 & 1 & Solution & 120.02 & 55 & 50.00 &  9.09\\
instance n=100 13.alb & 1 & 1 & Solution & 120.02 & 24 & 24.00 &  0.00\\
instance n=100 130.alb & 1 & 1 & Solution & 120.02 & 55 & 51.00 &  7.27\\
instance n=100 131.alb & 1 & 1 & Solution & 120.02 & 52 & 50.00 &  3.85\\
instance n=100 132.alb & 1 & 1 & Solution & 120.02 & 57 & 52.00 &  8.77\\
instance n=100 133.alb & 1 & 1 & Solution & 120.02 & 55 & 51.00 &  7.27\\
instance n=100 134.alb & 1 & 1 & Solution & 120.02 & 55 & 51.00 &  7.27\\
instance n=100 135.alb & 1 & 1 & Solution & 120.02 & 55 & 51.00 &  7.27\\
instance n=100 136.alb & 1 & 1 & Solution & 120.02 & 53 & 49.00 &  7.55\\
instance n=100 137.alb & 1 & 1 & Solution & 120.02 & 53 & 50.00 &  5.66\\
instance n=100 138.alb & 1 & 1 & Solution & 120.01 & 57 & 52.00 &  8.77\\
instance n=100 139.alb & 1 & 1 & Solution & 120.02 & 51 & 49.00 &  3.92\\
instance n=100 14.alb & 1 & 1 & Solution & 120.02 & 20 & 20.00 &  0.00\\
instance n=100 140.alb & 1 & 1 & Solution & 120.02 & 55 & 51.00 &  7.27\\
instance n=100 141.alb & 1 & 1 & Solution & 120.01 & 50 & 49.00 &  2.00\\
instance n=100 142.alb & 1 & 1 & Solution & 120.02 & 55 & 50.00 &  9.09\\
instance n=100 143.alb & 1 & 1 & Solution & 120.02 & 53 & 50.00 &  5.66\\
instance n=100 144.alb & 1 & 1 & Solution & 120.01 & 49 & 47.00 &  4.08\\
instance n=100 145.alb & 1 & 1 & Solution & 120.02 & 56 & 51.00 &  8.93\\
instance n=100 146.alb & 1 & 1 & Solution & 120.02 & 53 & 50.00 &  5.66\\
instance n=100 147.alb & 1 & 1 & Solution & 120.02 & 59 & 52.00 & 11.86\\
instance n=100 148.alb & 1 & 1 & Solution & 120.02 & 53 & 50.00 &  5.66\\
instance n=100 149.alb & 1 & 1 & Solution & 120.02 & 55 & 51.00 &  7.27\\
instance n=100 15.alb & 1 & 1 & Solution & 120.02 & 24 & 24.00 &  0.00\\
instance n=100 150.alb & 1 & 1 & Solution & 120.02 & 57 & 51.00 & 10.53\\
instance n=100 151.alb & 1 & 1 & Solution & 120.02 & 22 & 21.00 &  4.55\\
instance n=100 152.alb & 1 & 1 & Solution & 120.03 & 22 & 22.00 &  0.00\\
instance n=100 153.alb & 1 & 1 & Solution & 120.02 & 21 & 21.00 &  0.00\\
instance n=100 154.alb & 1 & 1 & Solution & 120.02 & 25 & 25.00 &  0.00\\
instance n=100 155.alb & 1 & 1 & Solution & 120.02 & 22 & 22.00 &  0.00\\
instance n=100 156.alb & 1 & 1 & Solution & 120.02 & 23 & 23.00 &  0.00\\
instance n=100 157.alb & 1 & 1 & Solution & 120.02 & 26 & 26.00 &  0.00\\
instance n=100 158.alb & 1 & 1 & Solution & 120.02 & 23 & 23.00 &  0.00\\
instance n=100 159.alb & 1 & 1 & Solution & 120.02 & 19 & 19.00 &  0.00\\
instance n=100 16.alb & 1 & 1 & Solution & 120.02 & 23 & 23.00 &  0.00\\
instance n=100 160.alb & 1 & 1 & Solution & 120.02 & 22 & 22.00 &  0.00\\
instance n=100 161.alb & 1 & 1 & Solution & 120.02 & 22 & 22.00 &  0.00\\
instance n=100 162.alb & 1 & 1 & Solution & 120.02 & 22 & 22.00 &  0.00\\
instance n=100 163.alb & 1 & 1 & Solution & 120.02 & 25 & 25.00 &  0.00\\
instance n=100 164.alb & 1 & 1 & Solution & 120.01 & 23 & 23.00 &  0.00\\
instance n=100 165.alb & 1 & 1 & Solution & 120.02 & 25 & 24.00 &  4.00\\
instance n=100 166.alb & 1 & 1 & Solution & 120.02 & 24 & 24.00 &  0.00\\
instance n=100 167.alb & 1 & 1 & Solution & 120.02 & 22 & 22.00 &  0.00\\
instance n=100 168.alb & 1 & 1 & Solution & 120.02 & 21 & 21.00 &  0.00\\
instance n=100 169.alb & 1 & 1 & Solution & 120.02 & 21 & 21.00 &  0.00\\
instance n=100 17.alb & 1 & 1 & Solution & 120.03 & 22 & 21.00 &  4.55\\
instance n=100 170.alb & 1 & 1 & Solution & 120.02 & 24 & 24.00 &  0.00\\
instance n=100 171.alb & 1 & 1 & Solution & 120.02 & 24 & 24.00 &  0.00\\
instance n=100 172.alb & 1 & 1 & Solution & 120.02 & 24 & 24.00 &  0.00\\
instance n=100 173.alb & 1 & 1 & Solution & 120.02 & 24 & 24.00 &  0.00\\
instance n=100 174.alb & 1 & 1 & Solution & 120.02 & 22 & 22.00 &  0.00\\
instance n=100 175.alb & 1 & 1 & Solution & 120.02 & 27 & 26.00 &  3.70\\
instance n=100 176.alb & 1 & 1 & Optimal & 31.45 & 13 & 13.00 &  0.00\\
instance n=100 177.alb & 1 & 1 & Solution & 120.01 & 14 & 14.00 &  0.00\\
instance n=100 178.alb & 1 & 1 & Solution & 120.01 & 15 & 15.00 &  0.00\\
instance n=100 179.alb & 1 & 1 & Solution & 120.02 & 15 & 15.00 &  0.00\\
instance n=100 18.alb & 1 & 1 & Solution & 120.02 & 20 & 19.00 &  5.00\\
instance n=100 180.alb & 1 & 1 & Solution & 120.02 & 15 & 15.00 &  0.00\\
instance n=100 181.alb & 1 & 1 & Optimal & 61.01 & 13 & 13.00 &  0.00\\
instance n=100 182.alb & 1 & 1 & Optimal & 112.91 & 15 & 15.00 &  0.00\\
instance n=100 183.alb & 1 & 1 & Solution & 120.02 & 14 & 14.00 &  0.00\\
instance n=100 184.alb & 1 & 1 & Optimal & 48.71 & 14 & 14.00 &  0.00\\
instance n=100 185.alb & 1 & 1 & Optimal & 11.55 & 15 & 15.00 &  0.00\\
instance n=100 186.alb & 1 & 1 & Solution & 120.02 & 14 & 14.00 &  0.00\\
instance n=100 187.alb & 1 & 1 & Optimal & 32.37 & 13 & 13.00 &  0.00\\
instance n=100 188.alb & 1 & 1 & Solution & 120.02 & 16 & 16.00 &  0.00\\
instance n=100 189.alb & 1 & 1 & Solution & 120.01 & 14 & 14.00 &  0.00\\
instance n=100 19.alb & 1 & 1 & Solution & 120.03 & 23 & 23.00 &  0.00\\
instance n=100 190.alb & 1 & 1 & Optimal & 69.51 & 13 & 13.00 &  0.00\\
instance n=100 191.alb & 1 & 1 & Optimal & 100.25 & 14 & 14.00 &  0.00\\
instance n=100 192.alb & 1 & 1 & Optimal & 77.04 & 13 & 13.00 &  0.00\\
instance n=100 193.alb & 1 & 1 & Solution & 120.02 & 15 & 15.00 &  0.00\\
instance n=100 194.alb & 1 & 1 & Solution & 120.01 & 15 & 15.00 &  0.00\\
instance n=100 195.alb & 1 & 1 & Optimal & 71.97 & 15 & 15.00 &  0.00\\
instance n=100 196.alb & 1 & 1 & Solution & 120.02 & 15 & 15.00 &  0.00\\
instance n=100 197.alb & 1 & 1 & Solution & 120.02 & 15 & 15.00 &  0.00\\
instance n=100 198.alb & 1 & 1 & Solution & 120.01 & 13 & 13.00 &  0.00\\
instance n=100 199.alb & 1 & 1 & Optimal & 72.67 & 14 & 14.00 &  0.00\\
instance n=100 2.alb & 1 & 1 & Solution & 120.02 & 21 & 21.00 &  0.00\\
instance n=100 20.alb & 1 & 1 & Solution & 120.02 & 21 & 21.00 &  0.00\\
instance n=100 200.alb & 1 & 1 & Solution & 120.01 & 15 & 15.00 &  0.00\\
instance n=100 201.alb & 1 & 1 & Solution & 120.02 & 52 & 50.00 &  3.85\\
instance n=100 202.alb & 1 & 1 & Solution & 120.02 & 61 & 52.00 & 14.75\\
instance n=100 203.alb & 1 & 1 & Solution & 120.02 & 52 & 49.00 &  5.77\\
instance n=100 204.alb & 1 & 1 & Solution & 120.03 & 50 & 48.00 &  4.00\\
instance n=100 205.alb & 1 & 1 & Solution & 120.02 & 56 & 51.00 &  8.93\\
instance n=100 206.alb & 1 & 1 & Solution & 120.02 & 51 & 49.00 &  3.92\\
instance n=100 207.alb & 1 & 1 & Solution & 120.02 & 51 & 49.00 &  3.92\\
instance n=100 208.alb & 1 & 1 & Solution & 120.02 & 56 & 51.00 &  8.93\\
instance n=100 209.alb & 1 & 1 & Solution & 120.01 & 54 & 51.00 &  5.56\\
instance n=100 21.alb & 1 & 1 & Solution & 120.02 & 21 & 21.00 &  0.00\\
instance n=100 210.alb & 1 & 1 & Solution & 120.02 & 52 & 49.00 &  5.77\\
instance n=100 211.alb & 1 & 1 & Solution & 120.02 & 51 & 49.00 &  3.92\\
instance n=100 212.alb & 1 & 1 & Solution & 120.02 & 52 & 50.00 &  3.85\\
instance n=100 213.alb & 1 & 1 & Solution & 120.02 & 52 & 50.00 &  3.85\\
instance n=100 214.alb & 1 & 1 & Solution & 120.02 & 54 & 50.00 &  7.41\\
instance n=100 215.alb & 1 & 1 & Solution & 120.02 & 49 & 47.00 &  4.08\\
instance n=100 216.alb & 1 & 1 & Solution & 120.02 & 53 & 50.00 &  5.66\\
instance n=100 217.alb & 1 & 1 & Solution & 120.02 & 52 & 49.00 &  5.77\\
instance n=100 218.alb & 1 & 1 & Solution & 120.02 & 53 & 50.00 &  5.66\\
instance n=100 219.alb & 1 & 1 & Solution & 120.02 & 51 & 49.00 &  3.92\\
instance n=100 22.alb & 1 & 1 & Solution & 120.03 & 24 & 24.00 &  0.00\\
instance n=100 220.alb & 1 & 1 & Solution & 120.02 & 53 & 50.00 &  5.66\\
instance n=100 221.alb & 1 & 1 & Solution & 120.02 & 57 & 51.00 & 10.53\\
instance n=100 222.alb & 1 & 1 & Solution & 120.03 & 53 & 50.00 &  5.66\\
instance n=100 223.alb & 1 & 1 & Solution & 120.02 & 51 & 49.00 &  3.92\\
instance n=100 224.alb & 1 & 1 & Solution & 120.02 & 55 & 51.00 &  7.27\\
instance n=100 225.alb & 1 & 1 & Solution & 120.02 & 53 & 50.00 &  5.66\\
instance n=100 226.alb & 1 & 1 & Solution & 120.02 & 25 & 24.00 &  4.00\\
instance n=100 227.alb & 1 & 1 & Solution & 120.02 & 27 & 26.00 &  3.70\\
instance n=100 228.alb & 1 & 1 & Solution & 120.02 & 22 & 22.00 &  0.00\\
instance n=100 229.alb & 1 & 1 & Solution & 120.02 & 24 & 24.00 &  0.00\\
instance n=100 23.alb & 1 & 1 & Solution & 120.02 & 24 & 24.00 &  0.00\\
instance n=100 230.alb & 1 & 1 & Solution & 120.02 & 23 & 23.00 &  0.00\\
instance n=100 231.alb & 1 & 1 & Solution & 120.03 & 22 & 22.00 &  0.00\\
instance n=100 232.alb & 1 & 1 & Solution & 120.02 & 22 & 22.00 &  0.00\\
instance n=100 233.alb & 1 & 1 & Solution & 120.02 & 23 & 22.00 &  4.35\\
instance n=100 234.alb & 1 & 1 & Solution & 120.02 & 23 & 23.00 &  0.00\\
instance n=100 235.alb & 1 & 1 & Solution & 120.01 & 26 & 26.00 &  0.00\\
instance n=100 236.alb & 1 & 1 & Solution & 120.02 & 23 & 22.00 &  4.35\\
instance n=100 237.alb & 1 & 1 & Solution & 120.02 & 23 & 23.00 &  0.00\\
instance n=100 238.alb & 1 & 1 & Solution & 120.02 & 23 & 23.00 &  0.00\\
instance n=100 239.alb & 1 & 1 & Solution & 120.02 & 21 & 21.00 &  0.00\\
instance n=100 24.alb & 1 & 1 & Solution & 120.02 & 24 & 24.00 &  0.00\\
instance n=100 240.alb & 1 & 1 & Solution & 120.02 & 22 & 22.00 &  0.00\\
instance n=100 241.alb & 1 & 1 & Solution & 120.01 & 22 & 22.00 &  0.00\\
instance n=100 242.alb & 1 & 1 & Solution & 120.02 & 23 & 23.00 &  0.00\\
instance n=100 243.alb & 1 & 1 & Solution & 120.02 & 23 & 23.00 &  0.00\\
instance n=100 244.alb & 1 & 1 & Solution & 120.02 & 21 & 21.00 &  0.00\\
instance n=100 245.alb & 1 & 1 & Solution & 120.02 & 24 & 23.00 &  4.17\\
instance n=100 246.alb & 1 & 1 & Solution & 120.02 & 26 & 26.00 &  0.00\\
instance n=100 247.alb & 1 & 1 & Solution & 120.02 & 22 & 22.00 &  0.00\\
instance n=100 248.alb & 1 & 1 & Solution & 120.02 & 19 & 19.00 &  0.00\\
instance n=100 249.alb & 1 & 1 & Solution & 120.01 & 21 & 21.00 &  0.00\\
instance n=100 25.alb & 1 & 1 & Solution & 120.02 & 22 & 22.00 &  0.00\\
instance n=100 250.alb & 1 & 1 & Solution & 120.02 & 24 & 24.00 &  0.00\\
instance n=100 251.alb & 1 & 1 & Optimal & 78.03 & 15 & 15.00 &  0.00\\
instance n=100 252.alb & 1 & 1 & Optimal & 36.22 & 14 & 14.00 &  0.00\\
instance n=100 253.alb & 1 & 1 & Optimal & 45.76 & 14 & 14.00 &  0.00\\
instance n=100 254.alb & 1 & 1 & Optimal & 87.69 & 14 & 14.00 &  0.00\\
instance n=100 255.alb & 1 & 1 & Optimal &  0.52 & 14 & 14.00 &  0.00\\
instance n=100 256.alb & 1 & 1 & Optimal & 19.10 & 15 & 15.00 &  0.00\\
instance n=100 257.alb & 1 & 1 & Optimal & 11.64 & 12 & 12.00 &  0.00\\
instance n=100 258.alb & 1 & 1 & Optimal & 97.17 & 14 & 14.00 &  0.00\\
instance n=100 259.alb & 1 & 1 & Optimal & 73.31 & 15 & 15.00 &  0.00\\
instance n=100 26.alb & 1 & 1 & Optimal & 61.17 & 14 & 14.00 &  0.00\\
instance n=100 260.alb & 1 & 1 & Solution & 120.02 & 15 & 15.00 &  0.00\\
instance n=100 261.alb & 1 & 1 & Optimal &  4.94 & 14 & 14.00 &  0.00\\
instance n=100 262.alb & 1 & 1 & Optimal & 23.72 & 14 & 14.00 &  0.00\\
instance n=100 263.alb & 1 & 1 & Optimal &  9.24 & 14 & 14.00 &  0.00\\
instance n=100 264.alb & 1 & 1 & Solution & 120.02 & 15 & 15.00 &  0.00\\
instance n=100 265.alb & 1 & 1 & Optimal &  7.88 & 14 & 14.00 &  0.00\\
instance n=100 266.alb & 1 & 1 & Optimal & 85.09 & 13 & 13.00 &  0.00\\
instance n=100 267.alb & 1 & 1 & Optimal & 85.29 & 13 & 13.00 &  0.00\\
instance n=100 268.alb & 1 & 1 & Optimal & 10.41 & 15 & 15.00 &  0.00\\
instance n=100 269.alb & 1 & 1 & Optimal & 51.02 & 15 & 15.00 &  0.00\\
instance n=100 27.alb & 1 & 1 & Optimal & 41.61 & 13 & 13.00 &  0.00\\
instance n=100 270.alb & 1 & 1 & Optimal & 27.54 & 13 & 13.00 &  0.00\\
instance n=100 271.alb & 1 & 1 & Optimal & 26.61 & 13 & 13.00 &  0.00\\
instance n=100 272.alb & 1 & 1 & Optimal & 62.54 & 14 & 14.00 &  0.00\\
instance n=100 273.alb & 1 & 1 & Optimal & 58.55 & 13 & 13.00 &  0.00\\
instance n=100 274.alb & 1 & 1 & Optimal & 35.91 & 13 & 13.00 &  0.00\\
instance n=100 275.alb & 1 & 1 & Optimal & 50.14 & 13 & 13.00 &  0.00\\
instance n=100 276.alb & 1 & 1 & Solution & 120.02 & 60 & 51.00 & 15.00\\
instance n=100 277.alb & 1 & 1 & Solution & 120.02 & 57 & 51.00 & 10.53\\
instance n=100 278.alb & 1 & 1 & Solution & 120.02 & 57 & 51.00 & 10.53\\
instance n=100 279.alb & 1 & 1 & Solution & 120.02 & 53 & 50.00 &  5.66\\
instance n=100 28.alb & 1 & 1 & Optimal & 80.31 & 14 & 14.00 &  0.00\\
instance n=100 280.alb & 1 & 1 & Solution & 120.02 & 55 & 50.00 &  9.09\\
instance n=100 281.alb & 1 & 1 & Solution & 120.02 & 62 & 53.00 & 14.52\\
instance n=100 282.alb & 1 & 1 & Solution & 120.02 & 60 & 52.00 & 13.33\\
instance n=100 283.alb & 1 & 1 & Solution & 120.02 & 55 & 50.00 &  9.09\\
instance n=100 284.alb & 1 & 1 & Solution & 120.02 & 55 & 50.00 &  9.09\\
instance n=100 285.alb & 1 & 1 & Solution & 120.02 & 54 & 50.00 &  7.41\\
instance n=100 286.alb & 1 & 1 & Solution & 120.02 & 56 & 51.00 &  8.93\\
instance n=100 287.alb & 1 & 1 & Solution & 120.02 & 54 & 50.00 &  7.41\\
instance n=100 288.alb & 1 & 1 & Solution & 120.02 & 56 & 51.00 &  8.93\\
instance n=100 289.alb & 1 & 1 & Solution & 120.02 & 62 & 51.00 & 17.74\\
instance n=100 29.alb & 1 & 1 & Optimal & 54.51 & 14 & 14.00 &  0.00\\
instance n=100 290.alb & 1 & 1 & Solution & 120.02 & 54 & 50.00 &  7.41\\
instance n=100 291.alb & 1 & 1 & Solution & 120.03 & 53 & 49.00 &  7.55\\
instance n=100 292.alb & 1 & 1 & Solution & 120.02 & 59 & 51.00 & 13.56\\
instance n=100 293.alb & 1 & 1 & Solution & 120.02 & 52 & 49.00 &  5.77\\
instance n=100 294.alb & 1 & 1 & Solution & 120.02 & 57 & 51.00 & 10.53\\
instance n=100 295.alb & 1 & 1 & Solution & 120.02 & 57 & 51.00 & 10.53\\
instance n=100 296.alb & 1 & 1 & Solution & 120.01 & 55 & 50.00 &  9.09\\
instance n=100 297.alb & 1 & 1 & Solution & 120.01 & 59 & 51.00 & 13.56\\
instance n=100 298.alb & 1 & 1 & Solution & 120.02 & 58 & 52.00 & 10.34\\
instance n=100 299.alb & 1 & 1 & Solution & 120.02 & 54 & 50.00 &  7.41\\
instance n=100 3.alb & 1 & 1 & Solution & 120.02 & 20 & 20.00 &  0.00\\
instance n=100 30.alb & 1 & 1 & Solution & 120.02 & 15 & 15.00 &  0.00\\
instance n=100 300.alb & 1 & 1 & Solution & 120.03 & 54 & 49.00 &  9.26\\
instance n=100 301.alb & 1 & 1 & Solution & 120.02 & 23 & 23.00 &  0.00\\
instance n=100 302.alb & 1 & 1 & Solution & 120.03 & 24 & 24.00 &  0.00\\
instance n=100 303.alb & 1 & 1 & Solution & 120.02 & 24 & 24.00 &  0.00\\
instance n=100 304.alb & 1 & 1 & Solution & 120.02 & 21 & 21.00 &  0.00\\
instance n=100 305.alb & 1 & 1 & Solution & 120.02 & 22 & 22.00 &  0.00\\
instance n=100 306.alb & 1 & 1 & Solution & 120.02 & 24 & 24.00 &  0.00\\
instance n=100 307.alb & 1 & 1 & Solution & 120.03 & 24 & 23.00 &  4.17\\
instance n=100 308.alb & 1 & 1 & Solution & 120.02 & 20 & 20.00 &  0.00\\
instance n=100 309.alb & 1 & 1 & Solution & 120.02 & 22 & 21.00 &  4.55\\
instance n=100 31.alb & 1 & 1 & Solution & 120.02 & 14 & 14.00 &  0.00\\
instance n=100 310.alb & 1 & 1 & Solution & 120.02 & 23 & 23.00 &  0.00\\
instance n=100 311.alb & 1 & 1 & Solution & 120.02 & 21 & 21.00 &  0.00\\
instance n=100 312.alb & 1 & 1 & Solution & 120.02 & 22 & 22.00 &  0.00\\
instance n=100 313.alb & 1 & 1 & Solution & 120.02 & 23 & 23.00 &  0.00\\
instance n=100 314.alb & 1 & 1 & Solution & 120.02 & 19 & 19.00 &  0.00\\
instance n=100 315.alb & 1 & 1 & Solution & 120.02 & 22 & 22.00 &  0.00\\
instance n=100 316.alb & 1 & 1 & Solution & 120.02 & 24 & 24.00 &  0.00\\
instance n=100 317.alb & 1 & 1 & Solution & 120.02 & 26 & 26.00 &  0.00\\
instance n=100 318.alb & 1 & 1 & Solution & 120.02 & 21 & 21.00 &  0.00\\
instance n=100 319.alb & 1 & 1 & Solution & 120.03 & 23 & 23.00 &  0.00\\
instance n=100 32.alb & 1 & 1 & Optimal & 45.28 & 14 & 14.00 &  0.00\\
instance n=100 320.alb & 1 & 1 & Solution & 120.02 & 22 & 22.00 &  0.00\\
instance n=100 321.alb & 1 & 1 & Solution & 120.02 & 26 & 26.00 &  0.00\\
instance n=100 322.alb & 1 & 1 & Solution & 120.02 & 23 & 23.00 &  0.00\\
instance n=100 323.alb & 1 & 1 & Solution & 120.02 & 24 & 24.00 &  0.00\\
instance n=100 324.alb & 1 & 1 & Solution & 120.02 & 23 & 23.00 &  0.00\\
instance n=100 325.alb & 1 & 1 & Solution & 120.02 & 25 & 25.00 &  0.00\\
instance n=100 326.alb & 1 & 1 & Optimal & 28.14 & 13 & 13.00 &  0.00\\
instance n=100 327.alb & 1 & 1 & Optimal & 85.13 & 14 & 14.00 &  0.00\\
instance n=100 328.alb & 1 & 1 & Optimal & 15.38 & 14 & 14.00 &  0.00\\
instance n=100 329.alb & 1 & 1 & Optimal & 98.27 & 14 & 14.00 &  0.00\\
instance n=100 33.alb & 1 & 1 & Solution & 120.01 & 15 & 15.00 &  0.00\\
instance n=100 330.alb & 1 & 1 & Optimal & 53.13 & 14 & 14.00 &  0.00\\
instance n=100 331.alb & 1 & 1 & Optimal & 71.96 & 14 & 14.00 &  0.00\\
instance n=100 332.alb & 1 & 1 & Optimal & 43.83 & 14 & 14.00 &  0.00\\
instance n=100 333.alb & 1 & 1 & Optimal & 51.59 & 15 & 15.00 &  0.00\\
instance n=100 334.alb & 1 & 1 & Optimal & 39.37 & 14 & 14.00 &  0.00\\
instance n=100 335.alb & 1 & 1 & Optimal &  9.22 & 13 & 13.00 &  0.00\\
instance n=100 336.alb & 1 & 1 & Optimal & 72.34 & 15 & 15.00 &  0.00\\
instance n=100 337.alb & 1 & 1 & Solution & 120.02 & 13 & 13.00 &  0.00\\
instance n=100 338.alb & 1 & 1 & Optimal & 112.11 & 14 & 14.00 &  0.00\\
instance n=100 339.alb & 1 & 1 & Optimal & 58.33 & 14 & 14.00 &  0.00\\
instance n=100 34.alb & 1 & 1 & Solution & 120.02 & 15 & 15.00 &  0.00\\
instance n=100 340.alb & 1 & 1 & Optimal & 57.69 & 14 & 14.00 &  0.00\\
instance n=100 341.alb & 1 & 1 & Solution & 120.02 & 16 & 16.00 &  0.00\\
instance n=100 342.alb & 1 & 1 & Optimal & 83.24 & 14 & 14.00 &  0.00\\
instance n=100 343.alb & 1 & 1 & Solution & 120.02 & 16 & 16.00 &  0.00\\
instance n=100 344.alb & 1 & 1 & Solution & 120.02 & 15 & 15.00 &  0.00\\
instance n=100 345.alb & 1 & 1 & Optimal & 66.40 & 14 & 14.00 &  0.00\\
instance n=100 346.alb & 1 & 1 & Optimal & 110.41 & 14 & 14.00 &  0.00\\
instance n=100 347.alb & 1 & 1 & Optimal & 92.50 & 14 & 14.00 &  0.00\\
instance n=100 348.alb & 1 & 1 & Optimal & 27.87 & 14 & 14.00 &  0.00\\
instance n=100 349.alb & 1 & 1 & Optimal & 29.09 & 13 & 13.00 &  0.00\\
instance n=100 35.alb & 1 & 1 & Optimal & 88.58 & 15 & 15.00 &  0.00\\
instance n=100 350.alb & 1 & 1 & Optimal & 89.57 & 14 & 14.00 &  0.00\\
instance n=100 351.alb & 1 & 1 & Solution & 120.02 & 59 & 52.00 & 11.86\\
instance n=100 352.alb & 1 & 1 & Solution & 120.02 & 63 & 52.00 & 17.46\\
instance n=100 353.alb & 1 & 1 & Solution & 120.02 & 50 & 49.00 &  2.00\\
instance n=100 354.alb & 1 & 1 & Solution & 120.02 & 52 & 49.00 &  5.77\\
instance n=100 355.alb & 1 & 1 & Solution & 120.03 & 54 & 51.00 &  5.56\\
instance n=100 356.alb & 1 & 1 & Solution & 120.02 & 59 & 53.00 & 10.17\\
instance n=100 357.alb & 1 & 1 & Solution & 120.02 & 53 & 50.00 &  5.66\\
instance n=100 358.alb & 1 & 1 & Solution & 120.03 & 52 & 50.00 &  3.85\\
instance n=100 359.alb & 1 & 1 & Solution & 120.03 & 53 & 50.00 &  5.66\\
instance n=100 36.alb & 1 & 1 & Solution & 120.02 & 14 & 14.00 &  0.00\\
instance n=100 360.alb & 1 & 1 & Solution & 120.02 & 54 & 51.00 &  5.56\\
instance n=100 361.alb & 1 & 1 & Solution & 120.02 & 51 & 49.00 &  3.92\\
instance n=100 362.alb & 1 & 1 & Solution & 120.02 & 57 & 51.00 & 10.53\\
instance n=100 363.alb & 1 & 1 & Solution & 120.02 & 52 & 50.00 &  3.85\\
instance n=100 364.alb & 1 & 1 & Solution & 120.02 & 52 & 50.00 &  3.85\\
instance n=100 365.alb & 1 & 1 & Solution & 120.02 & 52 & 50.00 &  3.85\\
instance n=100 366.alb & 1 & 1 & Solution & 120.02 & 61 & 53.00 & 13.11\\
instance n=100 367.alb & 1 & 1 & Solution & 120.02 & 55 & 51.00 &  7.27\\
instance n=100 368.alb & 1 & 1 & Solution & 120.02 & 59 & 52.00 & 11.86\\
instance n=100 369.alb & 1 & 1 & Solution & 120.02 & 51 & 49.00 &  3.92\\
instance n=100 37.alb & 1 & 1 & Optimal & 64.67 & 14 & 14.00 &  0.00\\
instance n=100 370.alb & 1 & 1 & Solution & 120.02 & 56 & 52.00 &  7.14\\
instance n=100 371.alb & 1 & 1 & Solution & 120.02 & 53 & 50.00 &  5.66\\
instance n=100 372.alb & 1 & 1 & Solution & 120.02 & 48 & 47.00 &  2.08\\
instance n=100 373.alb & 1 & 1 & Solution & 120.02 & 51 & 49.00 &  3.92\\
instance n=100 374.alb & 1 & 1 & Solution & 120.02 & 51 & 50.00 &  1.96\\
instance n=100 375.alb & 1 & 1 & Solution & 120.02 & 57 & 52.00 &  8.77\\
instance n=100 376.alb & 1 & 1 & Solution & 120.02 & 23 & 23.00 &  0.00\\
instance n=100 377.alb & 1 & 1 & Solution & 120.02 & 20 & 20.00 &  0.00\\
instance n=100 378.alb & 1 & 1 & Solution & 120.02 & 22 & 22.00 &  0.00\\
instance n=100 379.alb & 1 & 1 & Solution & 120.03 & 23 & 23.00 &  0.00\\
instance n=100 38.alb & 1 & 1 & Solution & 120.02 & 14 & 14.00 &  0.00\\
instance n=100 380.alb & 1 & 1 & Solution & 120.02 & 22 & 22.00 &  0.00\\
instance n=100 381.alb & 1 & 1 & Solution & 120.02 & 24 & 24.00 &  0.00\\
instance n=100 382.alb & 1 & 1 & Solution & 120.02 & 25 & 25.00 &  0.00\\
instance n=100 383.alb & 1 & 1 & Solution & 120.02 & 25 & 25.00 &  0.00\\
instance n=100 384.alb & 1 & 1 & Solution & 120.02 & 25 & 25.00 &  0.00\\
instance n=100 385.alb & 1 & 1 & Solution & 120.02 & 22 & 22.00 &  0.00\\
instance n=100 386.alb & 1 & 1 & Solution & 120.02 & 23 & 23.00 &  0.00\\
instance n=100 387.alb & 1 & 1 & Solution & 120.02 & 22 & 22.00 &  0.00\\
instance n=100 388.alb & 1 & 1 & Solution & 120.02 & 25 & 25.00 &  0.00\\
instance n=100 389.alb & 1 & 1 & Solution & 120.01 & 23 & 23.00 &  0.00\\
instance n=100 39.alb & 1 & 1 & Optimal & 26.16 & 14 & 14.00 &  0.00\\
instance n=100 390.alb & 1 & 1 & Solution & 120.02 & 22 & 22.00 &  0.00\\
instance n=100 391.alb & 1 & 1 & Solution & 120.02 & 20 & 20.00 &  0.00\\
instance n=100 392.alb & 1 & 1 & Solution & 120.02 & 22 & 22.00 &  0.00\\
instance n=100 393.alb & 1 & 1 & Solution & 120.02 & 23 & 23.00 &  0.00\\
instance n=100 394.alb & 1 & 1 & Solution & 120.02 & 22 & 22.00 &  0.00\\
instance n=100 395.alb & 1 & 1 & Solution & 120.02 & 24 & 24.00 &  0.00\\
instance n=100 396.alb & 1 & 1 & Solution & 120.02 & 20 & 20.00 &  0.00\\
instance n=100 397.alb & 1 & 1 & Solution & 120.01 & 26 & 25.00 &  3.85\\
instance n=100 398.alb & 1 & 1 & Solution & 120.02 & 25 & 24.00 &  4.00\\
instance n=100 399.alb & 1 & 1 & Solution & 120.02 & 23 & 23.00 &  0.00\\
instance n=100 4.alb & 1 & 1 & Solution & 120.02 & 24 & 24.00 &  0.00\\
instance n=100 40.alb & 1 & 1 & Optimal & 68.59 & 14 & 14.00 &  0.00\\
instance n=100 400.alb & 1 & 1 & Solution & 120.02 & 24 & 24.00 &  0.00\\
instance n=100 401.alb & 1 & 1 & Solution & 120.02 & 15 & 15.00 &  0.00\\
instance n=100 402.alb & 1 & 1 & Optimal & 80.91 & 15 & 15.00 &  0.00\\
instance n=100 403.alb & 1 & 1 & Solution & 120.01 & 14 & 14.00 &  0.00\\
instance n=100 404.alb & 1 & 1 & Optimal &  9.37 & 15 & 15.00 &  0.00\\
instance n=100 405.alb & 1 & 1 & Optimal & 36.55 & 13 & 13.00 &  0.00\\
instance n=100 406.alb & 1 & 1 & Optimal & 24.64 & 14 & 14.00 &  0.00\\
instance n=100 407.alb & 1 & 1 & Optimal & 105.38 & 15 & 15.00 &  0.00\\
instance n=100 408.alb & 1 & 1 & Optimal & 13.98 & 14 & 14.00 &  0.00\\
instance n=100 409.alb & 1 & 1 & Optimal & 53.47 & 15 & 15.00 &  0.00\\
instance n=100 41.alb & 1 & 1 & Optimal & 33.89 & 13 & 13.00 &  0.00\\
instance n=100 410.alb & 1 & 1 & Optimal & 67.35 & 14 & 14.00 &  0.00\\
instance n=100 411.alb & 1 & 1 & Optimal & 53.70 & 14 & 14.00 &  0.00\\
instance n=100 412.alb & 1 & 1 & Solution & 120.02 & 14 & 14.00 &  0.00\\
instance n=100 413.alb & 1 & 1 & Optimal & 79.43 & 14 & 14.00 &  0.00\\
instance n=100 414.alb & 1 & 1 & Optimal & 34.87 & 14 & 14.00 &  0.00\\
instance n=100 415.alb & 1 & 1 & Optimal & 46.96 & 13 & 13.00 &  0.00\\
instance n=100 416.alb & 1 & 1 & Optimal & 53.61 & 14 & 14.00 &  0.00\\
instance n=100 417.alb & 1 & 1 & Optimal & 62.59 & 15 & 15.00 &  0.00\\
instance n=100 418.alb & 1 & 1 & Optimal & 80.34 & 16 & 16.00 &  0.00\\
instance n=100 419.alb & 1 & 1 & Optimal & 41.20 & 14 & 14.00 &  0.00\\
instance n=100 42.alb & 1 & 1 & Optimal & 79.57 & 14 & 14.00 &  0.00\\
instance n=100 420.alb & 1 & 1 & Optimal & 21.39 & 14 & 14.00 &  0.00\\
instance n=100 421.alb & 1 & 1 & Optimal & 53.13 & 14 & 14.00 &  0.00\\
instance n=100 422.alb & 1 & 1 & Optimal & 59.02 & 15 & 15.00 &  0.00\\
instance n=100 423.alb & 1 & 1 & Optimal & 45.27 & 14 & 14.00 &  0.00\\
instance n=100 424.alb & 1 & 1 & Optimal & 31.47 & 14 & 14.00 &  0.00\\
instance n=100 425.alb & 1 & 1 & Optimal & 33.15 & 15 & 15.00 &  0.00\\
instance n=100 426.alb & 1 & 1 & Solution & 120.03 & 60 & 53.00 & 11.67\\
instance n=100 427.alb & 1 & 1 & Solution & 120.02 & 56 & 50.00 & 10.71\\
instance n=100 428.alb & 1 & 1 & Solution & 120.02 & 54 & 51.00 &  5.56\\
instance n=100 429.alb & 1 & 1 & Solution & 120.13 & 58 & 52.00 & 10.34\\
instance n=100 43.alb & 1 & 1 & Optimal & 39.34 & 14 & 14.00 &  0.00\\
instance n=100 430.alb & 1 & 1 & Solution & 120.03 & 54 & 50.00 &  7.41\\
instance n=100 431.alb & 1 & 1 & Solution & 120.02 & 54 & 50.00 &  7.41\\
instance n=100 432.alb & 1 & 1 & Solution & 120.03 & 56 & 50.00 & 10.71\\
instance n=100 433.alb & 1 & 1 & Solution & 120.03 & 52 & 49.00 &  5.77\\
instance n=100 434.alb & 1 & 1 & Solution & 120.13 & 56 & 51.00 &  8.93\\
instance n=100 435.alb & 1 & 1 & Solution & 120.03 & 56 & 50.00 & 10.71\\
instance n=100 436.alb & 1 & 1 & Solution & 120.04 & 52 & 48.00 &  7.69\\
instance n=100 437.alb & 1 & 1 & Solution & 120.02 & 53 & 50.00 &  5.66\\
instance n=100 438.alb & 1 & 1 & Solution & 120.03 & 55 & 51.00 &  7.27\\
instance n=100 439.alb & 1 & 1 & Solution & 120.04 & 55 & 51.00 &  7.27\\
instance n=100 44.alb & 1 & 1 & Optimal & 43.40 & 14 & 14.00 &  0.00\\
instance n=100 440.alb & 1 & 1 & Solution & 120.03 & 53 & 49.00 &  7.55\\
instance n=100 441.alb & 1 & 1 & Solution & 120.04 & 52 & 49.00 &  5.77\\
instance n=100 442.alb & 1 & 1 & Solution & 120.04 & 52 & 48.00 &  7.69\\
instance n=100 443.alb & 1 & 1 & Solution & 120.03 & 56 & 50.00 & 10.71\\
instance n=100 444.alb & 1 & 1 & Solution & 120.03 & 53 & 50.00 &  5.66\\
instance n=100 445.alb & 1 & 1 & Solution & 120.02 & 55 & 51.00 &  7.27\\
instance n=100 446.alb & 1 & 1 & Solution & 120.01 & 56 & 51.00 &  8.93\\
instance n=100 447.alb & 1 & 1 & Solution & 120.02 & 54 & 50.00 &  7.41\\
instance n=100 448.alb & 1 & 1 & Solution & 120.02 & 55 & 51.00 &  7.27\\
instance n=100 449.alb & 1 & 1 & Solution & 120.03 & 55 & 50.00 &  9.09\\
instance n=100 45.alb & 1 & 1 & Optimal & 76.40 & 14 & 14.00 &  0.00\\
instance n=100 450.alb & 1 & 1 & Solution & 120.03 & 54 & 50.00 &  7.41\\
instance n=100 451.alb & 1 & 1 & Optimal &  5.18 & 26 & 26.00 &  0.00\\
instance n=100 452.alb & 1 & 1 & Optimal & 13.35 & 22 & 22.00 &  0.00\\
instance n=100 453.alb & 1 & 1 & Optimal &  6.11 & 24 & 24.00 &  0.00\\
instance n=100 454.alb & 1 & 1 & Optimal &  5.62 & 23 & 23.00 &  0.00\\
instance n=100 455.alb & 1 & 1 & Optimal &  6.14 & 23 & 23.00 &  0.00\\
instance n=100 456.alb & 1 & 1 & Optimal &  8.80 & 26 & 26.00 &  0.00\\
instance n=100 457.alb & 1 & 1 & Optimal &  2.88 & 23 & 23.00 &  0.00\\
instance n=100 458.alb & 1 & 1 & Optimal &  4.74 & 24 & 24.00 &  0.00\\
instance n=100 459.alb & 1 & 1 & Optimal & 11.04 & 23 & 23.00 &  0.00\\
instance n=100 46.alb & 1 & 1 & Optimal & 108.02 & 14 & 14.00 &  0.00\\
instance n=100 460.alb & 1 & 1 & Optimal &  4.08 & 23 & 23.00 &  0.00\\
instance n=100 461.alb & 1 & 1 & Optimal &  8.17 & 23 & 23.00 &  0.00\\
instance n=100 462.alb & 1 & 1 & Optimal &  7.10 & 23 & 23.00 &  0.00\\
instance n=100 463.alb & 1 & 1 & Optimal &  7.23 & 26 & 26.00 &  0.00\\
instance n=100 464.alb & 1 & 1 & Optimal &  6.02 & 25 & 25.00 &  0.00\\
instance n=100 465.alb & 1 & 1 & Optimal &  6.46 & 22 & 22.00 &  0.00\\
instance n=100 466.alb & 1 & 1 & Optimal &  6.55 & 26 & 26.00 &  0.00\\
instance n=100 467.alb & 1 & 1 & Optimal & 12.28 & 21 & 21.00 &  0.00\\
instance n=100 468.alb & 1 & 1 & Optimal &  6.74 & 25 & 25.00 &  0.00\\
instance n=100 469.alb & 1 & 1 & Optimal &  6.88 & 22 & 22.00 &  0.00\\
instance n=100 47.alb & 1 & 1 & Optimal & 22.13 & 14 & 14.00 &  0.00\\
instance n=100 470.alb & 1 & 1 & Optimal &  7.75 & 26 & 26.00 &  0.00\\
instance n=100 471.alb & 1 & 1 & Optimal & 10.60 & 26 & 26.00 &  0.00\\
instance n=100 472.alb & 1 & 1 & Optimal &  7.66 & 23 & 23.00 &  0.00\\
instance n=100 473.alb & 1 & 1 & Optimal & 46.76 & 28 & 28.00 &  0.00\\
instance n=100 474.alb & 1 & 1 & Optimal &  4.27 & 23 & 23.00 &  0.00\\
instance n=100 475.alb & 1 & 1 & Optimal &  7.30 & 24 & 24.00 &  0.00\\
instance n=100 476.alb & 1 & 1 & Optimal &  3.93 & 14 & 14.00 &  0.00\\
instance n=100 477.alb & 1 & 1 & Optimal &  4.69 & 14 & 14.00 &  0.00\\
instance n=100 478.alb & 1 & 1 & Optimal &  2.67 & 14 & 14.00 &  0.00\\
instance n=100 479.alb & 1 & 1 & Optimal &  1.51 & 16 & 16.00 &  0.00\\
instance n=100 48.alb & 1 & 1 & Solution & 120.03 & 15 & 15.00 &  0.00\\
instance n=100 480.alb & 1 & 1 & Optimal &  1.38 & 15 & 15.00 &  0.00\\
instance n=100 481.alb & 1 & 1 & Optimal &  4.43 & 15 & 15.00 &  0.00\\
instance n=100 482.alb & 1 & 1 & Optimal &  1.06 & 15 & 15.00 &  0.00\\
instance n=100 483.alb & 1 & 1 & Optimal &  2.35 & 14 & 14.00 &  0.00\\
instance n=100 484.alb & 1 & 1 & Optimal &  4.86 & 14 & 14.00 &  0.00\\
instance n=100 485.alb & 1 & 1 & Optimal &  0.91 & 16 & 16.00 &  0.00\\
instance n=100 486.alb & 1 & 1 & Optimal &  3.86 & 15 & 15.00 &  0.00\\
instance n=100 487.alb & 1 & 1 & Optimal &  1.36 & 15 & 15.00 &  0.00\\
instance n=100 488.alb & 1 & 1 & Optimal &  2.37 & 16 & 16.00 &  0.00\\
instance n=100 489.alb & 1 & 1 & Optimal &  2.25 & 13 & 13.00 &  0.00\\
instance n=100 49.alb & 1 & 1 & Solution & 120.02 & 14 & 14.00 &  0.00\\
instance n=100 490.alb & 1 & 1 & Optimal &  1.35 & 15 & 15.00 &  0.00\\
instance n=100 491.alb & 1 & 1 & Optimal &  1.74 & 16 & 16.00 &  0.00\\
instance n=100 492.alb & 1 & 1 & Optimal &  2.08 & 14 & 14.00 &  0.00\\
instance n=100 493.alb & 1 & 1 & Optimal &  3.44 & 14 & 14.00 &  0.00\\
instance n=100 494.alb & 1 & 1 & Optimal &  3.43 & 14 & 14.00 &  0.00\\
instance n=100 495.alb & 1 & 1 & Optimal &  3.04 & 15 & 15.00 &  0.00\\
instance n=100 496.alb & 1 & 1 & Optimal &  2.85 & 14 & 14.00 &  0.00\\
instance n=100 497.alb & 1 & 1 & Optimal &  2.21 & 13 & 13.00 &  0.00\\
instance n=100 498.alb & 1 & 1 & Optimal &  3.08 & 14 & 14.00 &  0.00\\
instance n=100 499.alb & 1 & 1 & Optimal &  4.93 & 14 & 14.00 &  0.00\\
instance n=100 5.alb & 1 & 1 & Solution & 120.12 & 22 & 22.00 &  0.00\\
instance n=100 50.alb & 1 & 1 & Optimal & 40.19 & 14 & 14.00 &  0.00\\
instance n=100 500.alb & 1 & 1 & Optimal &  4.52 & 14 & 14.00 &  0.00\\
instance n=100 501.alb & 1 & 1 & Solution & 120.02 & 63 & 55.00 & 12.70\\
instance n=100 502.alb & 1 & 1 & Solution & 120.02 & 64 & 55.00 & 14.06\\
instance n=100 503.alb & 1 & 1 & Solution & 120.02 & 60 & 53.00 & 11.67\\
instance n=100 504.alb & 1 & 1 & Solution & 120.01 & 60 & 54.00 & 10.00\\
instance n=100 505.alb & 1 & 1 & Solution & 120.02 & 61 & 52.00 & 14.75\\
instance n=100 506.alb & 1 & 1 & Solution & 120.01 & 58 & 53.00 &  8.62\\
instance n=100 507.alb & 1 & 1 & Solution & 120.02 & 59 & 52.00 & 11.86\\
instance n=100 508.alb & 1 & 1 & Solution & 120.02 & 56 & 53.00 &  5.36\\
instance n=100 509.alb & 1 & 1 & Solution & 120.00 & 57 & 53.00 &  7.02\\
instance n=100 51.alb & 1 & 1 & Solution & 120.03 & 50 & 48.00 &  4.00\\
instance n=100 510.alb & 1 & 1 & Solution & 120.02 & 58 & 53.00 &  8.62\\
instance n=100 511.alb & 1 & 1 & Solution & 120.02 & 59 & 54.00 &  8.47\\
instance n=100 512.alb & 1 & 1 & Solution & 120.03 & 60 & 54.00 & 10.00\\
instance n=100 513.alb & 1 & 1 & Solution & 120.03 & 62 & 52.00 & 16.13\\
instance n=100 514.alb & 1 & 1 & Solution & 120.02 & 58 & 52.00 & 10.34\\
instance n=100 515.alb & 1 & 1 & Solution & 120.03 & 61 & 53.00 & 13.11\\
instance n=100 516.alb & 1 & 1 & Solution & 120.03 & 70 & 56.00 & 20.00\\
instance n=100 517.alb & 1 & 1 & Solution & 120.02 & 62 & 54.00 & 12.90\\
instance n=100 518.alb & 1 & 1 & Solution & 120.03 & 57 & 51.00 & 10.53\\
instance n=100 519.alb & 1 & 1 & Solution & 120.01 & 61 & 54.00 & 11.48\\
instance n=100 52.alb & 1 & 1 & Solution & 120.02 & 52 & 50.00 &  3.85\\
instance n=100 520.alb & 1 & 1 & Solution & 120.03 & 60 & 53.00 & 11.67\\
instance n=100 521.alb & 1 & 1 & Solution & 120.02 & 70 & 57.00 & 18.57\\
instance n=100 522.alb & 1 & 1 & Solution & 120.03 & 59 & 53.00 & 10.17\\
instance n=100 523.alb & 1 & 1 & Solution & 120.03 & 55 & 51.00 &  7.27\\
instance n=100 524.alb & 1 & 1 & Solution & 120.02 & 59 & 52.00 & 11.86\\
instance n=100 525.alb & 1 & 1 & Solution & 120.04 & 62 & 52.00 & 16.13\\
instance n=100 53.alb & 1 & 1 & Solution & 120.02 & 52 & 50.00 &  3.85\\
instance n=100 54.alb & 1 & 1 & Solution & 120.02 & 51 & 49.00 &  3.92\\
instance n=100 55.alb & 1 & 1 & Solution & 120.03 & 53 & 50.00 &  5.66\\
instance n=100 56.alb & 1 & 1 & Solution & 120.02 & 52 & 50.00 &  3.85\\
instance n=100 57.alb & 1 & 1 & Solution & 120.03 & 54 & 51.00 &  5.56\\
instance n=100 58.alb & 1 & 1 & Solution & 120.02 & 57 & 52.00 &  8.77\\
instance n=100 59.alb & 1 & 1 & Solution & 120.03 & 57 & 51.00 & 10.53\\
instance n=100 6.alb & 1 & 1 & Solution & 120.03 & 22 & 22.00 &  0.00\\
instance n=100 60.alb & 1 & 1 & Solution & 120.02 & 53 & 51.00 &  3.77\\
instance n=100 61.alb & 1 & 1 & Solution & 120.02 & 55 & 51.00 &  7.27\\
instance n=100 62.alb & 1 & 1 & Solution & 1254.58 & 52 & 49.00 &  5.77\\
instance n=100 63.alb & 1 & 1 & Solution & 120.05 & 61 & 52.00 & 14.75\\
instance n=100 64.alb & 1 & 1 & Solution & 120.01 & 56 & 51.00 &  8.93\\
instance n=100 65.alb & 1 & 1 & Solution & 120.02 & 62 & 53.00 & 14.52\\
instance n=100 66.alb & 1 & 1 & Solution & 120.02 & 51 & 49.00 &  3.92\\
instance n=100 67.alb & 1 & 1 & Solution & 120.02 & 55 & 51.00 &  7.27\\
instance n=100 68.alb & 1 & 1 & Solution & 120.03 & 57 & 49.00 & 14.04\\
instance n=100 69.alb & 1 & 1 & Solution & 120.03 & 53 & 51.00 &  3.77\\
instance n=100 7.alb & 1 & 1 & Solution & 120.03 & 26 & 26.00 &  0.00\\
instance n=100 70.alb & 1 & 1 & Solution & 120.02 & 52 & 50.00 &  3.85\\
instance n=100 71.alb & 1 & 1 & Solution & 120.02 & 53 & 50.00 &  5.66\\
instance n=100 72.alb & 1 & 1 & Solution & 120.02 & 53 & 50.00 &  5.66\\
instance n=100 73.alb & 1 & 1 & Solution & 120.03 & 55 & 52.00 &  5.45\\
instance n=100 74.alb & 1 & 1 & Solution & 120.02 & 51 & 49.00 &  3.92\\
instance n=100 75.alb & 1 & 1 & Solution & 120.03 & 54 & 51.00 &  5.56\\
instance n=100 76.alb & 1 & 1 & Solution & 120.02 & 23 & 23.00 &  0.00\\
instance n=100 77.alb & 1 & 1 & Solution & 120.03 & 20 & 20.00 &  0.00\\
instance n=100 78.alb & 1 & 1 & Solution & 120.02 & 21 & 21.00 &  0.00\\
instance n=100 79.alb & 1 & 1 & Solution & 120.02 & 21 & 21.00 &  0.00\\
instance n=100 8.alb & 1 & 1 & Solution & 120.03 & 24 & 24.00 &  0.00\\
instance n=100 80.alb & 1 & 1 & Solution & 120.03 & 22 & 22.00 &  0.00\\
instance n=100 81.alb & 1 & 1 & Solution & 120.02 & 20 & 20.00 &  0.00\\
instance n=100 82.alb & 1 & 1 & Solution & 120.03 & 21 & 21.00 &  0.00\\
instance n=100 83.alb & 1 & 1 & Solution & 120.02 & 22 & 22.00 &  0.00\\
instance n=100 84.alb & 1 & 1 & Solution & 120.01 & 27 & 26.00 &  3.70\\
instance n=100 85.alb & 1 & 1 & Solution & 120.02 & 25 & 24.00 &  4.00\\
instance n=100 86.alb & 1 & 1 & Solution & 120.02 & 23 & 23.00 &  0.00\\
instance n=100 87.alb & 1 & 1 & Solution & 120.03 & 22 & 22.00 &  0.00\\
instance n=100 88.alb & 1 & 1 & Solution & 120.02 & 23 & 23.00 &  0.00\\
instance n=100 89.alb & 1 & 1 & Solution & 120.02 & 24 & 24.00 &  0.00\\
instance n=100 9.alb & 1 & 1 & Solution & 120.02 & 23 & 23.00 &  0.00\\
instance n=100 90.alb & 1 & 1 & Solution & 120.02 & 21 & 20.00 &  4.76\\
instance n=100 91.alb & 1 & 1 & Solution & 120.02 & 25 & 25.00 &  0.00\\
instance n=100 92.alb & 1 & 1 & Solution & 120.03 & 24 & 24.00 &  0.00\\
instance n=100 93.alb & 1 & 1 & Solution & 120.03 & 27 & 27.00 &  0.00\\
instance n=100 94.alb & 1 & 1 & Solution & 120.02 & 22 & 22.00 &  0.00\\
instance n=100 95.alb & 1 & 1 & Solution & 120.02 & 21 & 21.00 &  0.00\\
instance n=100 96.alb & 1 & 1 & Solution & 120.03 & 21 & 21.00 &  0.00\\
instance n=100 97.alb & 1 & 1 & Solution & 120.02 & 22 & 22.00 &  0.00\\
instance n=100 98.alb & 1 & 1 & Solution & 120.02 & 22 & 22.00 &  0.00\\
instance n=100 99.alb & 1 & 1 & Solution & 120.03 & 22 & 22.00 &  0.00\\
instance n=20 1.alb & 1 & 1 & Optimal &  0.45 & 3 &  3.00 &  0.00\\
instance n=20 10.alb & 1 & 1 & Optimal &  0.16 & 3 &  3.00 &  0.00\\
instance n=20 100.alb & 1 & 1 & Optimal &  0.89 & 11 & 11.00 &  0.00\\
instance n=20 101.alb & 1 & 1 & Optimal &  1.81 & 13 & 13.00 &  0.00\\
instance n=20 102.alb & 1 & 1 & Optimal &  0.71 & 13 & 13.00 &  0.00\\
instance n=20 103.alb & 1 & 1 & Optimal &  0.24 & 12 & 12.00 &  0.00\\
instance n=20 104.alb & 1 & 1 & Optimal &  0.28 & 11 & 11.00 &  0.00\\
instance n=20 105.alb & 1 & 1 & Optimal &  0.46 & 12 & 12.00 &  0.00\\
instance n=20 106.alb & 1 & 1 & Optimal &  0.13 & 10 & 10.00 &  0.00\\
instance n=20 107.alb & 1 & 1 & Optimal &  1.69 & 14 & 14.00 &  0.00\\
instance n=20 108.alb & 1 & 1 & Optimal &  4.38 & 15 & 15.00 &  0.00\\
instance n=20 109.alb & 1 & 1 & Optimal &  0.38 & 12 & 12.00 &  0.00\\
instance n=20 11.alb & 1 & 1 & Optimal &  0.02 & 3 &  3.00 &  0.00\\
instance n=20 110.alb & 1 & 1 & Optimal &  0.29 & 11 & 11.00 &  0.00\\
instance n=20 111.alb & 1 & 1 & Optimal &  0.53 & 13 & 13.00 &  0.00\\
instance n=20 112.alb & 1 & 1 & Optimal &  0.17 & 11 & 11.00 &  0.00\\
instance n=20 113.alb & 1 & 1 & Optimal &  0.61 & 12 & 12.00 &  0.00\\
instance n=20 114.alb & 1 & 1 & Optimal &  0.66 & 13 & 13.00 &  0.00\\
instance n=20 115.alb & 1 & 1 & Optimal &  0.34 & 11 & 11.00 &  0.00\\
instance n=20 116.alb & 1 & 1 & Optimal &  0.13 & 5 &  5.00 &  0.00\\
instance n=20 117.alb & 1 & 1 & Optimal &  0.11 & 5 &  5.00 &  0.00\\
instance n=20 118.alb & 1 & 1 & Optimal &  0.13 & 5 &  5.00 &  0.00\\
instance n=20 119.alb & 1 & 1 & Optimal &  0.12 & 6 &  6.00 &  0.00\\
instance n=20 12.alb & 1 & 1 & Optimal &  0.16 & 3 &  3.00 &  0.00\\
instance n=20 120.alb & 1 & 1 & Optimal &  0.15 & 6 &  6.00 &  0.00\\
instance n=20 121.alb & 1 & 1 & Optimal &  0.27 & 5 &  5.00 &  0.00\\
instance n=20 122.alb & 1 & 1 & Optimal &  0.11 & 6 &  6.00 &  0.00\\
instance n=20 123.alb & 1 & 1 & Optimal &  0.13 & 5 &  5.00 &  0.00\\
instance n=20 124.alb & 1 & 1 & Optimal &  0.06 & 5 &  5.00 &  0.00\\
instance n=20 125.alb & 1 & 1 & Optimal &  0.14 & 5 &  5.00 &  0.00\\
instance n=20 126.alb & 1 & 1 & Optimal &  0.06 & 5 &  5.00 &  0.00\\
instance n=20 127.alb & 1 & 1 & Optimal &  0.13 & 4 &  4.00 &  0.00\\
instance n=20 128.alb & 1 & 1 & Optimal &  0.11 & 5 &  5.00 &  0.00\\
instance n=20 129.alb & 1 & 1 & Optimal &  0.13 & 5 &  5.00 &  0.00\\
instance n=20 13.alb & 1 & 1 & Optimal &  0.64 & 3 &  3.00 &  0.00\\
instance n=20 130.alb & 1 & 1 & Optimal &  0.22 & 6 &  6.00 &  0.00\\
instance n=20 131.alb & 1 & 1 & Optimal &  0.13 & 7 &  7.00 &  0.00\\
instance n=20 132.alb & 1 & 1 & Optimal &  0.16 & 4 &  4.00 &  0.00\\
instance n=20 133.alb & 1 & 1 & Optimal &  0.16 & 5 &  5.00 &  0.00\\
instance n=20 134.alb & 1 & 1 & Optimal &  0.11 & 6 &  6.00 &  0.00\\
instance n=20 135.alb & 1 & 1 & Optimal &  0.03 & 6 &  6.00 &  0.00\\
instance n=20 136.alb & 1 & 1 & Optimal &  0.13 & 6 &  6.00 &  0.00\\
instance n=20 137.alb & 1 & 1 & Optimal &  0.11 & 5 &  5.00 &  0.00\\
instance n=20 138.alb & 1 & 1 & Optimal &  0.11 & 5 &  5.00 &  0.00\\
instance n=20 139.alb & 1 & 1 & Optimal &  0.16 & 5 &  5.00 &  0.00\\
instance n=20 14.alb & 1 & 1 & Optimal &  0.47 & 3 &  3.00 &  0.00\\
instance n=20 140.alb & 1 & 1 & Optimal &  0.12 & 5 &  5.00 &  0.00\\
instance n=20 141.alb & 1 & 1 & Optimal &  0.40 & 3 &  3.00 &  0.00\\
instance n=20 142.alb & 1 & 1 & Optimal &  0.39 & 3 &  3.00 &  0.00\\
instance n=20 143.alb & 1 & 1 & Optimal &  0.14 & 3 &  3.00 &  0.00\\
instance n=20 144.alb & 1 & 1 & Optimal &  0.27 & 4 &  4.00 &  0.00\\
instance n=20 145.alb & 1 & 1 & Optimal &  0.43 & 3 &  3.00 &  0.00\\
instance n=20 146.alb & 1 & 1 & Optimal &  0.43 & 3 &  3.00 &  0.00\\
instance n=20 147.alb & 1 & 1 & Optimal &  0.39 & 3 &  3.00 &  0.00\\
instance n=20 148.alb & 1 & 1 & Optimal &  0.52 & 3 &  3.00 &  0.00\\
instance n=20 149.alb & 1 & 1 & Optimal &  0.47 & 3 &  3.00 &  0.00\\
instance n=20 15.alb & 1 & 1 & Optimal &  0.02 & 3 &  3.00 &  0.00\\
instance n=20 150.alb & 1 & 1 & Optimal &  0.50 & 3 &  3.00 &  0.00\\
instance n=20 151.alb & 1 & 1 & Optimal &  0.44 & 3 &  3.00 &  0.00\\
instance n=20 152.alb & 1 & 1 & Optimal &  0.11 & 3 &  3.00 &  0.00\\
instance n=20 153.alb & 1 & 1 & Optimal &  0.41 & 3 &  3.00 &  0.00\\
instance n=20 154.alb & 1 & 1 & Optimal &  0.05 & 3 &  3.00 &  0.00\\
instance n=20 155.alb & 1 & 1 & Optimal &  0.50 & 3 &  3.00 &  0.00\\
instance n=20 156.alb & 1 & 1 & Optimal &  0.11 & 3 &  3.00 &  0.00\\
instance n=20 157.alb & 1 & 1 & Optimal &  0.16 & 3 &  3.00 &  0.00\\
instance n=20 158.alb & 1 & 1 & Optimal &  0.14 & 3 &  3.00 &  0.00\\
instance n=20 159.alb & 1 & 1 & Optimal &  0.24 & 3 &  3.00 &  0.00\\
instance n=20 16.alb & 1 & 1 & Optimal &  0.99 & 12 & 12.00 &  0.00\\
instance n=20 160.alb & 1 & 1 & Optimal &  0.48 & 3 &  3.00 &  0.00\\
instance n=20 161.alb & 1 & 1 & Optimal &  0.53 & 3 &  3.00 &  0.00\\
instance n=20 162.alb & 1 & 1 & Optimal &  0.35 & 3 &  3.00 &  0.00\\
instance n=20 163.alb & 1 & 1 & Optimal &  0.16 & 3 &  3.00 &  0.00\\
instance n=20 164.alb & 1 & 1 & Optimal &  0.24 & 4 &  4.00 &  0.00\\
instance n=20 165.alb & 1 & 1 & Optimal &  0.03 & 3 &  3.00 &  0.00\\
instance n=20 166.alb & 1 & 1 & Optimal &  3.09 & 12 & 12.00 &  0.00\\
instance n=20 167.alb & 1 & 1 & Optimal &  0.58 & 11 & 11.00 &  0.00\\
instance n=20 168.alb & 1 & 1 & Optimal &  0.64 & 10 & 10.00 &  0.00\\
instance n=20 169.alb & 1 & 1 & Optimal &  1.84 & 11 & 11.00 &  0.00\\
instance n=20 17.alb & 1 & 1 & Optimal &  0.93 & 10 & 10.00 &  0.00\\
instance n=20 170.alb & 1 & 1 & Optimal &  3.20 & 11 & 11.00 &  0.00\\
instance n=20 171.alb & 1 & 1 & Optimal & 32.91 & 13 & 13.00 &  0.00\\
instance n=20 172.alb & 1 & 1 & Optimal &  0.61 & 11 & 11.00 &  0.00\\
instance n=20 173.alb & 1 & 1 & Optimal &  1.10 & 11 & 11.00 &  0.00\\
instance n=20 174.alb & 1 & 1 & Optimal &  0.93 & 12 & 12.00 &  0.00\\
instance n=20 175.alb & 1 & 1 & Optimal &  0.24 & 10 & 10.00 &  0.00\\
instance n=20 176.alb & 1 & 1 & Optimal &  0.47 & 11 & 11.00 &  0.00\\
instance n=20 177.alb & 1 & 1 & Optimal &  0.64 & 10 & 10.00 &  0.00\\
instance n=20 178.alb & 1 & 1 & Optimal &  0.90 & 11 & 11.00 &  0.00\\
instance n=20 179.alb & 1 & 1 & Optimal &  0.94 & 11 & 11.00 &  0.00\\
instance n=20 18.alb & 1 & 1 & Optimal &  1.34 & 11 & 11.00 &  0.00\\
instance n=20 180.alb & 1 & 1 & Optimal & 25.37 & 13 & 13.00 &  0.00\\
instance n=20 181.alb & 1 & 1 & Optimal &  0.49 & 11 & 11.00 &  0.00\\
instance n=20 182.alb & 1 & 1 & Optimal &  0.92 & 11 & 11.00 &  0.00\\
instance n=20 183.alb & 1 & 1 & Optimal & 23.41 & 13 & 13.00 &  0.00\\
instance n=20 184.alb & 1 & 1 & Optimal &  1.58 & 12 & 12.00 &  0.00\\
instance n=20 185.alb & 1 & 1 & Optimal & 61.73 & 15 & 15.00 &  0.00\\
instance n=20 186.alb & 1 & 1 & Optimal & 13.46 & 14 & 14.00 &  0.00\\
instance n=20 187.alb & 1 & 1 & Optimal &  0.35 & 10 & 10.00 &  0.00\\
instance n=20 188.alb & 1 & 1 & Optimal &  1.13 & 11 & 11.00 &  0.00\\
instance n=20 189.alb & 1 & 1 & Optimal &  5.05 & 13 & 13.00 &  0.00\\
instance n=20 19.alb & 1 & 1 & Optimal & 43.23 & 14 & 14.00 &  0.00\\
instance n=20 190.alb & 1 & 1 & Solution & 120.01 & 15 & 12.00 & 20.00\\
instance n=20 191.alb & 1 & 1 & Optimal &  0.52 & 4 &  4.00 &  0.00\\
instance n=20 192.alb & 1 & 1 & Optimal &  0.17 & 5 &  5.00 &  0.00\\
instance n=20 193.alb & 1 & 1 & Optimal &  0.35 & 5 &  5.00 &  0.00\\
instance n=20 194.alb & 1 & 1 & Optimal &  0.22 & 6 &  6.00 &  0.00\\
instance n=20 195.alb & 1 & 1 & Optimal &  0.02 & 6 &  6.00 &  0.00\\
instance n=20 196.alb & 1 & 1 & Optimal &  0.24 & 5 &  5.00 &  0.00\\
instance n=20 197.alb & 1 & 1 & Optimal &  0.27 & 4 &  4.00 &  0.00\\
instance n=20 198.alb & 1 & 1 & Optimal &  0.16 & 6 &  6.00 &  0.00\\
instance n=20 199.alb & 1 & 1 & Optimal &  0.28 & 5 &  5.00 &  0.00\\
instance n=20 2.alb & 1 & 1 & Optimal &  0.27 & 3 &  3.00 &  0.00\\
instance n=20 20.alb & 1 & 1 & Optimal &  1.16 & 11 & 11.00 &  0.00\\
instance n=20 200.alb & 1 & 1 & Optimal &  0.16 & 6 &  6.00 &  0.00\\
instance n=20 201.alb & 1 & 1 & Optimal &  0.27 & 6 &  6.00 &  0.00\\
instance n=20 202.alb & 1 & 1 & Optimal &  0.35 & 4 &  4.00 &  0.00\\
instance n=20 203.alb & 1 & 1 & Optimal &  0.25 & 4 &  4.00 &  0.00\\
instance n=20 204.alb & 1 & 1 & Optimal &  0.42 & 5 &  5.00 &  0.00\\
instance n=20 205.alb & 1 & 1 & Optimal &  0.28 & 6 &  6.00 &  0.00\\
instance n=20 206.alb & 1 & 1 & Optimal &  0.16 & 5 &  5.00 &  0.00\\
instance n=20 207.alb & 1 & 1 & Optimal &  0.28 & 6 &  6.00 &  0.00\\
instance n=20 208.alb & 1 & 1 & Optimal &  0.45 & 5 &  5.00 &  0.00\\
instance n=20 209.alb & 1 & 1 & Optimal &  0.31 & 4 &  4.00 &  0.00\\
instance n=20 21.alb & 1 & 1 & Optimal &  4.09 & 14 & 14.00 &  0.00\\
instance n=20 210.alb & 1 & 1 & Optimal &  0.29 & 5 &  5.00 &  0.00\\
instance n=20 211.alb & 1 & 1 & Optimal &  0.11 & 5 &  5.00 &  0.00\\
instance n=20 212.alb & 1 & 1 & Optimal &  0.17 & 5 &  5.00 &  0.00\\
instance n=20 213.alb & 1 & 1 & Optimal &  0.13 & 5 &  5.00 &  0.00\\
instance n=20 214.alb & 1 & 1 & Optimal &  0.42 & 5 &  5.00 &  0.00\\
instance n=20 215.alb & 1 & 1 & Optimal &  0.39 & 5 &  5.00 &  0.00\\
instance n=20 216.alb & 1 & 1 & Optimal &  0.16 & 3 &  3.00 &  0.00\\
instance n=20 217.alb & 1 & 1 & Optimal &  0.11 & 4 &  4.00 &  0.00\\
instance n=20 218.alb & 1 & 1 & Optimal &  0.11 & 3 &  3.00 &  0.00\\
instance n=20 219.alb & 1 & 1 & Optimal &  0.06 & 3 &  3.00 &  0.00\\
instance n=20 22.alb & 1 & 1 & Optimal &  0.72 & 12 & 12.00 &  0.00\\
instance n=20 220.alb & 1 & 1 & Optimal &  0.11 & 3 &  3.00 &  0.00\\
instance n=20 221.alb & 1 & 1 & Optimal &  0.06 & 3 &  3.00 &  0.00\\
instance n=20 222.alb & 1 & 1 & Optimal &  0.06 & 3 &  3.00 &  0.00\\
instance n=20 223.alb & 1 & 1 & Optimal &  0.12 & 3 &  3.00 &  0.00\\
instance n=20 224.alb & 1 & 1 & Optimal &  0.10 & 3 &  3.00 &  0.00\\
instance n=20 225.alb & 1 & 1 & Optimal &  0.12 & 3 &  3.00 &  0.00\\
instance n=20 226.alb & 1 & 1 & Optimal &  0.33 & 3 &  3.00 &  0.00\\
instance n=20 227.alb & 1 & 1 & Optimal &  0.20 & 3 &  3.00 &  0.00\\
instance n=20 228.alb & 1 & 1 & Optimal &  0.06 & 2 &  2.00 &  0.00\\
instance n=20 229.alb & 1 & 1 & Optimal &  0.23 & 3 &  3.00 &  0.00\\
instance n=20 23.alb & 1 & 1 & Optimal & 25.65 & 13 & 13.00 &  0.00\\
instance n=20 230.alb & 1 & 1 & Optimal &  0.10 & 3 &  3.00 &  0.00\\
instance n=20 231.alb & 1 & 1 & Optimal &  0.11 & 3 &  3.00 &  0.00\\
instance n=20 232.alb & 1 & 1 & Optimal &  0.11 & 3 &  3.00 &  0.00\\
instance n=20 233.alb & 1 & 1 & Optimal &  0.11 & 3 &  3.00 &  0.00\\
instance n=20 234.alb & 1 & 1 & Optimal &  0.05 & 3 &  3.00 &  0.00\\
instance n=20 235.alb & 1 & 1 & Optimal &  0.16 & 3 &  3.00 &  0.00\\
instance n=20 236.alb & 1 & 1 & Optimal &  0.07 & 3 &  3.00 &  0.00\\
instance n=20 237.alb & 1 & 1 & Optimal &  0.06 & 3 &  3.00 &  0.00\\
instance n=20 238.alb & 1 & 1 & Optimal &  0.11 & 3 &  3.00 &  0.00\\
instance n=20 239.alb & 1 & 1 & Optimal &  0.11 & 3 &  3.00 &  0.00\\
instance n=20 24.alb & 1 & 1 & Optimal &  4.01 & 11 & 11.00 &  0.00\\
instance n=20 240.alb & 1 & 1 & Optimal &  0.11 & 3 &  3.00 &  0.00\\
instance n=20 241.alb & 1 & 1 & Optimal &  0.36 & 13 & 13.00 &  0.00\\
instance n=20 242.alb & 1 & 1 & Optimal &  0.13 & 12 & 12.00 &  0.00\\
instance n=20 243.alb & 1 & 1 & Optimal &  0.23 & 10 & 10.00 &  0.00\\
instance n=20 244.alb & 1 & 1 & Optimal &  0.11 & 11 & 11.00 &  0.00\\
instance n=20 245.alb & 1 & 1 & Optimal &  0.33 & 13 & 13.00 &  0.00\\
instance n=20 246.alb & 1 & 1 & Optimal &  0.61 & 13 & 13.00 &  0.00\\
instance n=20 247.alb & 1 & 1 & Optimal &  0.28 & 11 & 11.00 &  0.00\\
instance n=20 248.alb & 1 & 1 & Optimal &  0.19 & 11 & 11.00 &  0.00\\
instance n=20 249.alb & 1 & 1 & Optimal &  0.64 & 13 & 13.00 &  0.00\\
instance n=20 25.alb & 1 & 1 & Optimal &  0.15 & 11 & 11.00 &  0.00\\
instance n=20 250.alb & 1 & 1 & Optimal &  0.21 & 10 & 10.00 &  0.00\\
instance n=20 251.alb & 1 & 1 & Optimal &  0.24 & 12 & 12.00 &  0.00\\
instance n=20 252.alb & 1 & 1 & Optimal &  0.27 & 11 & 11.00 &  0.00\\
instance n=20 253.alb & 1 & 1 & Optimal &  0.41 & 13 & 13.00 &  0.00\\
instance n=20 254.alb & 1 & 1 & Optimal &  0.22 & 12 & 12.00 &  0.00\\
instance n=20 255.alb & 1 & 1 & Optimal &  0.52 & 13 & 13.00 &  0.00\\
instance n=20 256.alb & 1 & 1 & Optimal &  0.35 & 14 & 14.00 &  0.00\\
instance n=20 257.alb & 1 & 1 & Optimal &  0.11 & 10 & 10.00 &  0.00\\
instance n=20 258.alb & 1 & 1 & Optimal &  0.27 & 13 & 13.00 &  0.00\\
instance n=20 259.alb & 1 & 1 & Optimal &  0.43 & 13 & 13.00 &  0.00\\
instance n=20 26.alb & 1 & 1 & Optimal &  1.13 & 12 & 12.00 &  0.00\\
instance n=20 260.alb & 1 & 1 & Optimal &  0.60 & 12 & 12.00 &  0.00\\
instance n=20 261.alb & 1 & 1 & Optimal &  0.16 & 12 & 12.00 &  0.00\\
instance n=20 262.alb & 1 & 1 & Optimal &  0.11 & 11 & 11.00 &  0.00\\
instance n=20 263.alb & 1 & 1 & Optimal &  0.31 & 12 & 12.00 &  0.00\\
instance n=20 264.alb & 1 & 1 & Optimal &  0.38 & 12 & 12.00 &  0.00\\
instance n=20 265.alb & 1 & 1 & Optimal &  0.24 & 12 & 12.00 &  0.00\\
instance n=20 266.alb & 1 & 1 & Optimal &  0.11 & 5 &  5.00 &  0.00\\
instance n=20 267.alb & 1 & 1 & Optimal &  0.11 & 6 &  6.00 &  0.00\\
instance n=20 268.alb & 1 & 1 & Optimal &  0.06 & 6 &  6.00 &  0.00\\
instance n=20 269.alb & 1 & 1 & Optimal &  0.19 & 7 &  7.00 &  0.00\\
instance n=20 27.alb & 1 & 1 & Optimal &  8.22 & 13 & 13.00 &  0.00\\
instance n=20 270.alb & 1 & 1 & Optimal &  0.10 & 7 &  7.00 &  0.00\\
instance n=20 271.alb & 1 & 1 & Optimal &  0.07 & 6 &  6.00 &  0.00\\
instance n=20 272.alb & 1 & 1 & Optimal &  0.16 & 5 &  5.00 &  0.00\\
instance n=20 273.alb & 1 & 1 & Optimal &  0.06 & 5 &  5.00 &  0.00\\
instance n=20 274.alb & 1 & 1 & Optimal &  0.13 & 6 &  6.00 &  0.00\\
instance n=20 275.alb & 1 & 1 & Optimal &  0.11 & 5 &  5.00 &  0.00\\
instance n=20 276.alb & 1 & 1 & Optimal &  0.12 & 4 &  4.00 &  0.00\\
instance n=20 277.alb & 1 & 1 & Optimal &  0.11 & 4 &  4.00 &  0.00\\
instance n=20 278.alb & 1 & 1 & Optimal &  0.14 & 6 &  6.00 &  0.00\\
instance n=20 279.alb & 1 & 1 & Optimal &  0.11 & 6 &  6.00 &  0.00\\
instance n=20 28.alb & 1 & 1 & Optimal &  2.78 & 12 & 12.00 &  0.00\\
instance n=20 280.alb & 1 & 1 & Optimal &  0.11 & 5 &  5.00 &  0.00\\
instance n=20 281.alb & 1 & 1 & Optimal &  0.17 & 4 &  4.00 &  0.00\\
instance n=20 282.alb & 1 & 1 & Optimal &  0.06 & 4 &  4.00 &  0.00\\
instance n=20 283.alb & 1 & 1 & Optimal &  0.08 & 5 &  5.00 &  0.00\\
instance n=20 284.alb & 1 & 1 & Optimal &  0.14 & 5 &  5.00 &  0.00\\
instance n=20 285.alb & 1 & 1 & Optimal &  0.20 & 5 &  5.00 &  0.00\\
instance n=20 286.alb & 1 & 1 & Optimal &  0.12 & 5 &  5.00 &  0.00\\
instance n=20 287.alb & 1 & 1 & Optimal &  0.13 & 5 &  5.00 &  0.00\\
instance n=20 288.alb & 1 & 1 & Optimal &  0.06 & 6 &  6.00 &  0.00\\
instance n=20 289.alb & 1 & 1 & Optimal &  0.11 & 5 &  5.00 &  0.00\\
instance n=20 29.alb & 1 & 1 & Optimal &  0.39 & 10 & 10.00 &  0.00\\
instance n=20 290.alb & 1 & 1 & Optimal &  0.06 & 5 &  5.00 &  0.00\\
instance n=20 291.alb & 1 & 1 & Optimal &  0.30 & 3 &  3.00 &  0.00\\
instance n=20 292.alb & 1 & 1 & Optimal &  0.02 & 3 &  3.00 &  0.00\\
instance n=20 293.alb & 1 & 1 & Optimal &  0.02 & 3 &  3.00 &  0.00\\
instance n=20 294.alb & 1 & 1 & Optimal &  0.03 & 3 &  3.00 &  0.00\\
instance n=20 295.alb & 1 & 1 & Optimal &  0.39 & 3 &  3.00 &  0.00\\
instance n=20 296.alb & 1 & 1 & Optimal &  0.11 & 3 &  3.00 &  0.00\\
instance n=20 297.alb & 1 & 1 & Optimal &  0.27 & 3 &  3.00 &  0.00\\
instance n=20 298.alb & 1 & 1 & Optimal &  0.03 & 3 &  3.00 &  0.00\\
instance n=20 299.alb & 1 & 1 & Optimal &  0.11 & 3 &  3.00 &  0.00\\
instance n=20 3.alb & 1 & 1 & Optimal &  0.02 & 3 &  3.00 &  0.00\\
instance n=20 30.alb & 1 & 1 & Optimal & 86.26 & 16 & 16.00 &  0.00\\
instance n=20 300.alb & 1 & 1 & Optimal &  0.42 & 4 &  4.00 &  0.00\\
instance n=20 301.alb & 1 & 1 & Optimal &  0.02 & 3 &  3.00 &  0.00\\
instance n=20 302.alb & 1 & 1 & Optimal &  0.16 & 3 &  3.00 &  0.00\\
instance n=20 303.alb & 1 & 1 & Optimal &  0.03 & 3 &  3.00 &  0.00\\
instance n=20 304.alb & 1 & 1 & Optimal &  0.17 & 3 &  3.00 &  0.00\\
instance n=20 305.alb & 1 & 1 & Optimal &  0.03 & 3 &  3.00 &  0.00\\
instance n=20 306.alb & 1 & 1 & Optimal &  0.61 & 3 &  3.00 &  0.00\\
instance n=20 307.alb & 1 & 1 & Optimal &  0.17 & 3 &  3.00 &  0.00\\
instance n=20 308.alb & 1 & 1 & Optimal &  0.02 & 3 &  3.00 &  0.00\\
instance n=20 309.alb & 1 & 1 & Optimal &  0.55 & 3 &  3.00 &  0.00\\
instance n=20 31.alb & 1 & 1 & Optimal &  2.94 & 12 & 12.00 &  0.00\\
instance n=20 310.alb & 1 & 1 & Optimal &  0.10 & 3 &  3.00 &  0.00\\
instance n=20 311.alb & 1 & 1 & Optimal &  0.03 & 3 &  3.00 &  0.00\\
instance n=20 312.alb & 1 & 1 & Optimal &  0.30 & 4 &  4.00 &  0.00\\
instance n=20 313.alb & 1 & 1 & Optimal &  0.28 & 3 &  3.00 &  0.00\\
instance n=20 314.alb & 1 & 1 & Optimal &  0.49 & 3 &  3.00 &  0.00\\
instance n=20 315.alb & 1 & 1 & Optimal &  0.05 & 3 &  3.00 &  0.00\\
instance n=20 316.alb & 1 & 1 & Optimal &  6.82 & 10 & 10.00 &  0.00\\
instance n=20 317.alb & 1 & 1 & Optimal &  2.08 & 10 & 10.00 &  0.00\\
instance n=20 318.alb & 1 & 1 & Optimal &  0.35 & 10 & 10.00 &  0.00\\
instance n=20 319.alb & 1 & 1 & Optimal &  6.90 & 14 & 14.00 &  0.00\\
instance n=20 32.alb & 1 & 1 & Optimal & 17.57 & 13 & 13.00 &  0.00\\
instance n=20 320.alb & 1 & 1 & Optimal &  1.18 & 12 & 12.00 &  0.00\\
instance n=20 321.alb & 1 & 1 & Solution & 120.01 & 14 & 11.00 & 21.43\\
instance n=20 322.alb & 1 & 1 & Optimal & 13.16 & 12 & 12.00 &  0.00\\
instance n=20 323.alb & 1 & 1 & Optimal &  7.89 & 13 & 13.00 &  0.00\\
instance n=20 324.alb & 1 & 1 & Optimal &  0.88 & 9 &  9.00 &  0.00\\
instance n=20 325.alb & 1 & 1 & Optimal & 69.16 & 14 & 14.00 &  0.00\\
instance n=20 326.alb & 1 & 1 & Optimal & 53.63 & 14 & 14.00 &  0.00\\
instance n=20 327.alb & 1 & 1 & Optimal & 12.95 & 13 & 13.00 &  0.00\\
instance n=20 328.alb & 1 & 1 & Optimal & 15.76 & 13 & 13.00 &  0.00\\
instance n=20 329.alb & 1 & 1 & Optimal &  1.15 & 10 & 10.00 &  0.00\\
instance n=20 33.alb & 1 & 1 & Optimal &  0.80 & 11 & 11.00 &  0.00\\
instance n=20 330.alb & 1 & 1 & Optimal & 88.63 & 12 & 12.00 &  0.00\\
instance n=20 331.alb & 1 & 1 & Solution & 120.00 & 13 & 12.00 &  7.69\\
instance n=20 332.alb & 1 & 1 & Optimal & 11.42 & 13 & 13.00 &  0.00\\
instance n=20 333.alb & 1 & 1 & Optimal &  1.63 & 11 & 11.00 &  0.00\\
instance n=20 334.alb & 1 & 1 & Optimal &  0.74 & 10 & 10.00 &  0.00\\
instance n=20 335.alb & 1 & 1 & Solution & 120.00 & 14 & 11.00 & 21.43\\
instance n=20 336.alb & 1 & 1 & Optimal &  1.01 & 11 & 11.00 &  0.00\\
instance n=20 337.alb & 1 & 1 & Optimal &  0.22 & 10 & 10.00 &  0.00\\
instance n=20 338.alb & 1 & 1 & Optimal & 14.53 & 14 & 14.00 &  0.00\\
instance n=20 339.alb & 1 & 1 & Optimal & 21.75 & 13 & 13.00 &  0.00\\
instance n=20 34.alb & 1 & 1 & Optimal &  2.44 & 12 & 12.00 &  0.00\\
instance n=20 340.alb & 1 & 1 & Optimal &  0.70 & 11 & 11.00 &  0.00\\
instance n=20 341.alb & 1 & 1 & Optimal &  0.09 & 6 &  6.00 &  0.00\\
instance n=20 342.alb & 1 & 1 & Optimal &  0.38 & 6 &  6.00 &  0.00\\
instance n=20 343.alb & 1 & 1 & Optimal &  0.24 & 6 &  6.00 &  0.00\\
instance n=20 344.alb & 1 & 1 & Optimal &  0.39 & 6 &  6.00 &  0.00\\
instance n=20 345.alb & 1 & 1 & Optimal &  0.57 & 4 &  4.00 &  0.00\\
instance n=20 346.alb & 1 & 1 & Optimal &  0.28 & 5 &  5.00 &  0.00\\
instance n=20 347.alb & 1 & 1 & Optimal &  0.41 & 6 &  6.00 &  0.00\\
instance n=20 348.alb & 1 & 1 & Optimal &  0.22 & 5 &  5.00 &  0.00\\
instance n=20 349.alb & 1 & 1 & Optimal &  0.51 & 5 &  5.00 &  0.00\\
instance n=20 35.alb & 1 & 1 & Optimal &  1.59 & 12 & 12.00 &  0.00\\
instance n=20 350.alb & 1 & 1 & Optimal &  0.46 & 5 &  5.00 &  0.00\\
instance n=20 351.alb & 1 & 1 & Optimal &  0.18 & 5 &  5.00 &  0.00\\
instance n=20 352.alb & 1 & 1 & Optimal &  0.58 & 4 &  4.00 &  0.00\\
instance n=20 353.alb & 1 & 1 & Optimal &  0.27 & 6 &  6.00 &  0.00\\
instance n=20 354.alb & 1 & 1 & Optimal &  0.31 & 6 &  6.00 &  0.00\\
instance n=20 355.alb & 1 & 1 & Optimal &  0.24 & 5 &  5.00 &  0.00\\
instance n=20 356.alb & 1 & 1 & Optimal &  0.32 & 5 &  5.00 &  0.00\\
instance n=20 357.alb & 1 & 1 & Optimal &  0.72 & 5 &  5.00 &  0.00\\
instance n=20 358.alb & 1 & 1 & Optimal &  0.27 & 4 &  4.00 &  0.00\\
instance n=20 359.alb & 1 & 1 & Optimal &  0.32 & 4 &  4.00 &  0.00\\
instance n=20 36.alb & 1 & 1 & Optimal &  1.81 & 13 & 13.00 &  0.00\\
instance n=20 360.alb & 1 & 1 & Optimal &  0.38 & 6 &  6.00 &  0.00\\
instance n=20 361.alb & 1 & 1 & Optimal &  0.44 & 5 &  5.00 &  0.00\\
instance n=20 362.alb & 1 & 1 & Optimal &  0.32 & 5 &  5.00 &  0.00\\
instance n=20 363.alb & 1 & 1 & Optimal &  0.41 & 7 &  7.00 &  0.00\\
instance n=20 364.alb & 1 & 1 & Optimal &  0.31 & 4 &  4.00 &  0.00\\
instance n=20 365.alb & 1 & 1 & Optimal &  0.44 & 5 &  5.00 &  0.00\\
instance n=20 366.alb & 1 & 1 & Optimal &  0.17 & 3 &  3.00 &  0.00\\
instance n=20 367.alb & 1 & 1 & Optimal &  0.11 & 3 &  3.00 &  0.00\\
instance n=20 368.alb & 1 & 1 & Optimal &  0.11 & 3 &  3.00 &  0.00\\
instance n=20 369.alb & 1 & 1 & Optimal &  0.25 & 3 &  3.00 &  0.00\\
instance n=20 37.alb & 1 & 1 & Optimal &  3.26 & 12 & 12.00 &  0.00\\
instance n=20 370.alb & 1 & 1 & Optimal &  0.03 & 3 &  3.00 &  0.00\\
instance n=20 371.alb & 1 & 1 & Optimal &  0.17 & 3 &  3.00 &  0.00\\
instance n=20 372.alb & 1 & 1 & Optimal &  0.16 & 3 &  3.00 &  0.00\\
instance n=20 373.alb & 1 & 1 & Optimal &  0.10 & 3 &  3.00 &  0.00\\
instance n=20 374.alb & 1 & 1 & Optimal &  0.05 & 3 &  3.00 &  0.00\\
instance n=20 375.alb & 1 & 1 & Optimal &  0.11 & 3 &  3.00 &  0.00\\
instance n=20 376.alb & 1 & 1 & Optimal &  0.17 & 3 &  3.00 &  0.00\\
instance n=20 377.alb & 1 & 1 & Optimal &  0.16 & 3 &  3.00 &  0.00\\
instance n=20 378.alb & 1 & 1 & Optimal &  0.02 & 3 &  3.00 &  0.00\\
instance n=20 379.alb & 1 & 1 & Optimal &  0.06 & 4 &  4.00 &  0.00\\
instance n=20 38.alb & 1 & 1 & Optimal &  0.53 & 12 & 12.00 &  0.00\\
instance n=20 380.alb & 1 & 1 & Optimal &  0.19 & 3 &  3.00 &  0.00\\
instance n=20 381.alb & 1 & 1 & Optimal &  0.16 & 3 &  3.00 &  0.00\\
instance n=20 382.alb & 1 & 1 & Optimal &  0.17 & 4 &  4.00 &  0.00\\
instance n=20 383.alb & 1 & 1 & Optimal &  0.10 & 3 &  3.00 &  0.00\\
instance n=20 384.alb & 1 & 1 & Optimal &  0.21 & 3 &  3.00 &  0.00\\
instance n=20 385.alb & 1 & 1 & Optimal &  0.22 & 3 &  3.00 &  0.00\\
instance n=20 386.alb & 1 & 1 & Optimal &  0.16 & 3 &  3.00 &  0.00\\
instance n=20 387.alb & 1 & 1 & Optimal &  0.26 & 3 &  3.00 &  0.00\\
instance n=20 388.alb & 1 & 1 & Optimal &  0.05 & 3 &  3.00 &  0.00\\
instance n=20 389.alb & 1 & 1 & Optimal &  0.13 & 3 &  3.00 &  0.00\\
instance n=20 39.alb & 1 & 1 & Optimal &  5.75 & 13 & 13.00 &  0.00\\
instance n=20 390.alb & 1 & 1 & Optimal &  0.11 & 3 &  3.00 &  0.00\\
instance n=20 391.alb & 1 & 1 & Optimal &  0.27 & 11 & 11.00 &  0.00\\
instance n=20 392.alb & 1 & 1 & Optimal &  0.39 & 14 & 14.00 &  0.00\\
instance n=20 393.alb & 1 & 1 & Optimal &  0.24 & 11 & 11.00 &  0.00\\
instance n=20 394.alb & 1 & 1 & Optimal &  0.32 & 12 & 12.00 &  0.00\\
instance n=20 395.alb & 1 & 1 & Optimal &  0.22 & 12 & 12.00 &  0.00\\
instance n=20 396.alb & 1 & 1 & Optimal &  0.33 & 13 & 13.00 &  0.00\\
instance n=20 397.alb & 1 & 1 & Optimal &  0.24 & 10 & 10.00 &  0.00\\
instance n=20 398.alb & 1 & 1 & Optimal &  0.16 & 11 & 11.00 &  0.00\\
instance n=20 399.alb & 1 & 1 & Optimal &  0.42 & 13 & 13.00 &  0.00\\
instance n=20 4.alb & 1 & 1 & Optimal &  0.10 & 3 &  3.00 &  0.00\\
instance n=20 40.alb & 1 & 1 & Optimal &  1.02 & 12 & 12.00 &  0.00\\
instance n=20 400.alb & 1 & 1 & Optimal &  0.46 & 12 & 12.00 &  0.00\\
instance n=20 401.alb & 1 & 1 & Optimal &  0.25 & 12 & 12.00 &  0.00\\
instance n=20 402.alb & 1 & 1 & Optimal &  0.30 & 12 & 12.00 &  0.00\\
instance n=20 403.alb & 1 & 1 & Optimal &  0.21 & 12 & 12.00 &  0.00\\
instance n=20 404.alb & 1 & 1 & Optimal &  0.16 & 10 & 10.00 &  0.00\\
instance n=20 405.alb & 1 & 1 & Optimal &  0.17 & 12 & 12.00 &  0.00\\
instance n=20 406.alb & 1 & 1 & Optimal &  1.42 & 14 & 14.00 &  0.00\\
instance n=20 407.alb & 1 & 1 & Optimal &  0.18 & 10 & 10.00 &  0.00\\
instance n=20 408.alb & 1 & 1 & Optimal &  1.24 & 14 & 14.00 &  0.00\\
instance n=20 409.alb & 1 & 1 & Optimal &  0.31 & 12 & 12.00 &  0.00\\
instance n=20 41.alb & 1 & 1 & Optimal &  0.24 & 6 &  6.00 &  0.00\\
instance n=20 410.alb & 1 & 1 & Optimal &  0.30 & 11 & 11.00 &  0.00\\
instance n=20 411.alb & 1 & 1 & Optimal &  1.37 & 15 & 15.00 &  0.00\\
instance n=20 412.alb & 1 & 1 & Optimal &  0.24 & 11 & 11.00 &  0.00\\
instance n=20 413.alb & 1 & 1 & Optimal &  0.19 & 10 & 10.00 &  0.00\\
instance n=20 414.alb & 1 & 1 & Optimal &  0.39 & 12 & 12.00 &  0.00\\
instance n=20 415.alb & 1 & 1 & Optimal &  0.17 & 10 & 10.00 &  0.00\\
instance n=20 416.alb & 1 & 1 & Optimal &  0.06 & 6 &  6.00 &  0.00\\
instance n=20 417.alb & 1 & 1 & Optimal &  0.14 & 5 &  5.00 &  0.00\\
instance n=20 418.alb & 1 & 1 & Optimal &  0.04 & 6 &  6.00 &  0.00\\
instance n=20 419.alb & 1 & 1 & Optimal &  0.16 & 4 &  4.00 &  0.00\\
instance n=20 42.alb & 1 & 1 & Optimal &  0.36 & 5 &  5.00 &  0.00\\
instance n=20 420.alb & 1 & 1 & Optimal &  0.13 & 5 &  5.00 &  0.00\\
instance n=20 421.alb & 1 & 1 & Optimal &  0.14 & 6 &  6.00 &  0.00\\
instance n=20 422.alb & 1 & 1 & Optimal &  0.12 & 4 &  4.00 &  0.00\\
instance n=20 423.alb & 1 & 1 & Optimal &  0.13 & 6 &  6.00 &  0.00\\
instance n=20 424.alb & 1 & 1 & Optimal &  0.15 & 5 &  5.00 &  0.00\\
instance n=20 425.alb & 1 & 1 & Optimal &  0.06 & 6 &  6.00 &  0.00\\
instance n=20 426.alb & 1 & 1 & Optimal &  0.14 & 5 &  5.00 &  0.00\\
instance n=20 427.alb & 1 & 1 & Optimal &  0.13 & 6 &  6.00 &  0.00\\
instance n=20 428.alb & 1 & 1 & Optimal &  0.13 & 5 &  5.00 &  0.00\\
instance n=20 429.alb & 1 & 1 & Optimal &  0.13 & 4 &  4.00 &  0.00\\
instance n=20 43.alb & 1 & 1 & Optimal &  0.25 & 5 &  5.00 &  0.00\\
instance n=20 430.alb & 1 & 1 & Optimal &  0.13 & 5 &  5.00 &  0.00\\
instance n=20 431.alb & 1 & 1 & Optimal &  0.06 & 6 &  6.00 &  0.00\\
instance n=20 432.alb & 1 & 1 & Optimal &  0.07 & 5 &  5.00 &  0.00\\
instance n=20 433.alb & 1 & 1 & Optimal &  0.14 & 5 &  5.00 &  0.00\\
instance n=20 434.alb & 1 & 1 & Optimal &  0.13 & 5 &  5.00 &  0.00\\
instance n=20 435.alb & 1 & 1 & Optimal &  0.18 & 7 &  7.00 &  0.00\\
instance n=20 436.alb & 1 & 1 & Optimal &  0.11 & 5 &  5.00 &  0.00\\
instance n=20 437.alb & 1 & 1 & Optimal &  0.13 & 5 &  5.00 &  0.00\\
instance n=20 438.alb & 1 & 1 & Optimal &  0.13 & 6 &  6.00 &  0.00\\
instance n=20 439.alb & 1 & 1 & Optimal &  0.14 & 5 &  5.00 &  0.00\\
instance n=20 44.alb & 1 & 1 & Optimal &  0.36 & 5 &  5.00 &  0.00\\
instance n=20 440.alb & 1 & 1 & Optimal &  0.08 & 5 &  5.00 &  0.00\\
instance n=20 441.alb & 1 & 1 & Optimal &  0.03 & 3 &  3.00 &  0.00\\
instance n=20 442.alb & 1 & 1 & Optimal &  0.06 & 3 &  3.00 &  0.00\\
instance n=20 443.alb & 1 & 1 & Optimal &  0.08 & 3 &  3.00 &  0.00\\
instance n=20 444.alb & 1 & 1 & Optimal &  0.07 & 3 &  3.00 &  0.00\\
instance n=20 445.alb & 1 & 1 & Optimal &  0.06 & 3 &  3.00 &  0.00\\
instance n=20 446.alb & 1 & 1 & Optimal &  0.06 & 3 &  3.00 &  0.00\\
instance n=20 447.alb & 1 & 1 & Optimal &  0.07 & 3 &  3.00 &  0.00\\
instance n=20 448.alb & 1 & 1 & Optimal &  0.06 & 3 &  3.00 &  0.00\\
instance n=20 449.alb & 1 & 1 & Optimal &  0.06 & 3 &  3.00 &  0.00\\
instance n=20 45.alb & 1 & 1 & Optimal &  0.17 & 6 &  6.00 &  0.00\\
instance n=20 450.alb & 1 & 1 & Optimal &  0.06 & 3 &  3.00 &  0.00\\
instance n=20 451.alb & 1 & 1 & Optimal &  0.05 & 3 &  3.00 &  0.00\\
instance n=20 452.alb & 1 & 1 & Optimal &  0.05 & 3 &  3.00 &  0.00\\
instance n=20 453.alb & 1 & 1 & Optimal &  0.06 & 3 &  3.00 &  0.00\\
instance n=20 454.alb & 1 & 1 & Optimal &  0.05 & 3 &  3.00 &  0.00\\
instance n=20 455.alb & 1 & 1 & Optimal &  0.06 & 3 &  3.00 &  0.00\\
instance n=20 456.alb & 1 & 1 & Optimal &  0.06 & 4 &  4.00 &  0.00\\
instance n=20 457.alb & 1 & 1 & Optimal &  0.06 & 3 &  3.00 &  0.00\\
instance n=20 458.alb & 1 & 1 & Optimal &  0.06 & 3 &  3.00 &  0.00\\
instance n=20 459.alb & 1 & 1 & Optimal &  0.06 & 3 &  3.00 &  0.00\\
instance n=20 46.alb & 1 & 1 & Optimal &  0.30 & 4 &  4.00 &  0.00\\
instance n=20 460.alb & 1 & 1 & Optimal &  0.06 & 3 &  3.00 &  0.00\\
instance n=20 461.alb & 1 & 1 & Optimal &  0.06 & 3 &  3.00 &  0.00\\
instance n=20 462.alb & 1 & 1 & Optimal &  0.06 & 3 &  3.00 &  0.00\\
instance n=20 463.alb & 1 & 1 & Optimal &  0.07 & 3 &  3.00 &  0.00\\
instance n=20 464.alb & 1 & 1 & Optimal &  0.05 & 3 &  3.00 &  0.00\\
instance n=20 465.alb & 1 & 1 & Optimal &  0.06 & 3 &  3.00 &  0.00\\
instance n=20 466.alb & 1 & 1 & Optimal &  0.07 & 13 & 13.00 &  0.00\\
instance n=20 467.alb & 1 & 1 & Optimal &  0.13 & 14 & 14.00 &  0.00\\
instance n=20 468.alb & 1 & 1 & Optimal &  0.13 & 13 & 13.00 &  0.00\\
instance n=20 469.alb & 1 & 1 & Optimal &  0.12 & 14 & 14.00 &  0.00\\
instance n=20 47.alb & 1 & 1 & Optimal &  0.24 & 4 &  4.00 &  0.00\\
instance n=20 470.alb & 1 & 1 & Optimal &  0.14 & 12 & 12.00 &  0.00\\
instance n=20 471.alb & 1 & 1 & Optimal &  0.06 & 12 & 12.00 &  0.00\\
instance n=20 472.alb & 1 & 1 & Optimal &  0.13 & 13 & 13.00 &  0.00\\
instance n=20 473.alb & 1 & 1 & Optimal &  0.09 & 10 & 10.00 &  0.00\\
instance n=20 474.alb & 1 & 1 & Optimal &  0.13 & 14 & 14.00 &  0.00\\
instance n=20 475.alb & 1 & 1 & Optimal &  0.06 & 11 & 11.00 &  0.00\\
instance n=20 476.alb & 1 & 1 & Optimal &  0.13 & 11 & 11.00 &  0.00\\
instance n=20 477.alb & 1 & 1 & Optimal &  0.06 & 11 & 11.00 &  0.00\\
instance n=20 478.alb & 1 & 1 & Optimal &  0.08 & 12 & 12.00 &  0.00\\
instance n=20 479.alb & 1 & 1 & Optimal &  0.07 & 13 & 13.00 &  0.00\\
instance n=20 48.alb & 1 & 1 & Optimal &  0.27 & 5 &  5.00 &  0.00\\
instance n=20 480.alb & 1 & 1 & Optimal &  0.14 & 13 & 13.00 &  0.00\\
instance n=20 481.alb & 1 & 1 & Optimal &  0.19 & 13 & 13.00 &  0.00\\
instance n=20 482.alb & 1 & 1 & Optimal &  0.06 & 13 & 13.00 &  0.00\\
instance n=20 483.alb & 1 & 1 & Optimal &  0.06 & 12 & 12.00 &  0.00\\
instance n=20 484.alb & 1 & 1 & Optimal &  0.13 & 13 & 13.00 &  0.00\\
instance n=20 485.alb & 1 & 1 & Optimal &  0.17 & 15 & 15.00 &  0.00\\
instance n=20 486.alb & 1 & 1 & Optimal &  0.08 & 11 & 11.00 &  0.00\\
instance n=20 487.alb & 1 & 1 & Optimal &  0.06 & 12 & 12.00 &  0.00\\
instance n=20 488.alb & 1 & 1 & Optimal &  0.16 & 15 & 15.00 &  0.00\\
instance n=20 489.alb & 1 & 1 & Optimal &  0.06 & 12 & 12.00 &  0.00\\
instance n=20 49.alb & 1 & 1 & Optimal &  0.30 & 4 &  4.00 &  0.00\\
instance n=20 490.alb & 1 & 1 & Optimal &  0.10 & 12 & 12.00 &  0.00\\
instance n=20 491.alb & 1 & 1 & Optimal &  0.05 & 6 &  6.00 &  0.00\\
instance n=20 492.alb & 1 & 1 & Optimal &  0.07 & 5 &  5.00 &  0.00\\
instance n=20 493.alb & 1 & 1 & Optimal &  0.01 & 5 &  5.00 &  0.00\\
instance n=20 494.alb & 1 & 1 & Optimal &  0.08 & 6 &  6.00 &  0.00\\
instance n=20 495.alb & 1 & 1 & Optimal &  0.05 & 6 &  6.00 &  0.00\\
instance n=20 496.alb & 1 & 1 & Optimal &  0.06 & 5 &  5.00 &  0.00\\
instance n=20 497.alb & 1 & 1 & Optimal &  0.06 & 6 &  6.00 &  0.00\\
instance n=20 498.alb & 1 & 1 & Optimal &  0.06 & 6 &  6.00 &  0.00\\
instance n=20 499.alb & 1 & 1 & Optimal &  0.07 & 5 &  5.00 &  0.00\\
instance n=20 5.alb & 1 & 1 & Optimal &  0.11 & 3 &  3.00 &  0.00\\
instance n=20 50.alb & 1 & 1 & Optimal &  0.39 & 4 &  4.00 &  0.00\\
instance n=20 500.alb & 1 & 1 & Optimal &  0.06 & 8 &  8.00 &  0.00\\
instance n=20 501.alb & 1 & 1 & Optimal &  0.13 & 5 &  5.00 &  0.00\\
instance n=20 502.alb & 1 & 1 & Optimal &  0.06 & 4 &  4.00 &  0.00\\
instance n=20 503.alb & 1 & 1 & Optimal &  0.06 & 6 &  6.00 &  0.00\\
instance n=20 504.alb & 1 & 1 & Optimal &  0.07 & 6 &  6.00 &  0.00\\
instance n=20 505.alb & 1 & 1 & Optimal &  0.05 & 6 &  6.00 &  0.00\\
instance n=20 506.alb & 1 & 1 & Optimal &  0.06 & 5 &  5.00 &  0.00\\
instance n=20 507.alb & 1 & 1 & Optimal &  0.03 & 5 &  5.00 &  0.00\\
instance n=20 508.alb & 1 & 1 & Optimal &  0.06 & 5 &  5.00 &  0.00\\
instance n=20 509.alb & 1 & 1 & Optimal &  0.02 & 4 &  4.00 &  0.00\\
instance n=20 51.alb & 1 & 1 & Optimal &  0.30 & 4 &  4.00 &  0.00\\
instance n=20 510.alb & 1 & 1 & Optimal &  0.06 & 5 &  5.00 &  0.00\\
instance n=20 511.alb & 1 & 1 & Optimal &  0.06 & 5 &  5.00 &  0.00\\
instance n=20 512.alb & 1 & 1 & Optimal &  0.08 & 5 &  5.00 &  0.00\\
instance n=20 513.alb & 1 & 1 & Optimal &  0.05 & 5 &  5.00 &  0.00\\
instance n=20 514.alb & 1 & 1 & Optimal &  0.06 & 5 &  5.00 &  0.00\\
instance n=20 515.alb & 1 & 1 & Optimal &  0.06 & 6 &  6.00 &  0.00\\
instance n=20 516.alb & 1 & 1 & Optimal &  0.36 & 3 &  3.00 &  0.00\\
instance n=20 517.alb & 1 & 1 & Optimal &  0.35 & 3 &  3.00 &  0.00\\
instance n=20 518.alb & 1 & 1 & Optimal &  0.55 & 3 &  3.00 &  0.00\\
instance n=20 519.alb & 1 & 1 & Optimal &  0.38 & 3 &  3.00 &  0.00\\
instance n=20 52.alb & 1 & 1 & Optimal &  0.37 & 4 &  4.00 &  0.00\\
instance n=20 520.alb & 1 & 1 & Optimal &  0.33 & 3 &  3.00 &  0.00\\
instance n=20 521.alb & 1 & 1 & Optimal &  0.35 & 3 &  3.00 &  0.00\\
instance n=20 522.alb & 1 & 1 & Optimal &  0.44 & 3 &  3.00 &  0.00\\
instance n=20 523.alb & 1 & 1 & Optimal &  0.49 & 3 &  3.00 &  0.00\\
instance n=20 524.alb & 1 & 1 & Optimal &  0.41 & 3 &  3.00 &  0.00\\
instance n=20 525.alb & 1 & 1 & Optimal &  0.44 & 3 &  3.00 &  0.00\\
instance n=20 53.alb & 1 & 1 & Optimal &  0.24 & 5 &  5.00 &  0.00\\
instance n=20 54.alb & 1 & 1 & Optimal &  0.13 & 5 &  5.00 &  0.00\\
instance n=20 55.alb & 1 & 1 & Optimal &  0.39 & 5 &  5.00 &  0.00\\
instance n=20 56.alb & 1 & 1 & Optimal &  0.39 & 4 &  4.00 &  0.00\\
instance n=20 57.alb & 1 & 1 & Optimal &  0.39 & 4 &  4.00 &  0.00\\
instance n=20 58.alb & 1 & 1 & Optimal &  0.25 & 5 &  5.00 &  0.00\\
instance n=20 59.alb & 1 & 1 & Optimal &  0.58 & 4 &  4.00 &  0.00\\
instance n=20 6.alb & 1 & 1 & Optimal &  0.05 & 3 &  3.00 &  0.00\\
instance n=20 60.alb & 1 & 1 & Optimal &  0.50 & 6 &  6.00 &  0.00\\
instance n=20 61.alb & 1 & 1 & Optimal &  0.35 & 7 &  7.00 &  0.00\\
instance n=20 62.alb & 1 & 1 & Optimal &  0.24 & 5 &  5.00 &  0.00\\
instance n=20 63.alb & 1 & 1 & Optimal &  0.57 & 5 &  5.00 &  0.00\\
instance n=20 64.alb & 1 & 1 & Optimal &  0.29 & 5 &  5.00 &  0.00\\
instance n=20 65.alb & 1 & 1 & Optimal &  0.25 & 5 &  5.00 &  0.00\\
instance n=20 66.alb & 1 & 1 & Optimal &  0.16 & 3 &  3.00 &  0.00\\
instance n=20 67.alb & 1 & 1 & Optimal &  0.10 & 3 &  3.00 &  0.00\\
instance n=20 68.alb & 1 & 1 & Optimal &  0.07 & 3 &  3.00 &  0.00\\
instance n=20 69.alb & 1 & 1 & Optimal &  0.06 & 2 &  2.00 &  0.00\\
instance n=20 7.alb & 1 & 1 & Optimal &  0.11 & 3 &  3.00 &  0.00\\
instance n=20 70.alb & 1 & 1 & Optimal &  0.11 & 3 &  3.00 &  0.00\\
instance n=20 71.alb & 1 & 1 & Optimal &  0.22 & 3 &  3.00 &  0.00\\
instance n=20 72.alb & 1 & 1 & Optimal &  0.11 & 3 &  3.00 &  0.00\\
instance n=20 73.alb & 1 & 1 & Optimal &  0.11 & 2 &  2.00 &  0.00\\
instance n=20 74.alb & 1 & 1 & Optimal &  0.30 & 3 &  3.00 &  0.00\\
instance n=20 75.alb & 1 & 1 & Optimal &  0.11 & 3 &  3.00 &  0.00\\
instance n=20 76.alb & 1 & 1 & Optimal &  0.12 & 3 &  3.00 &  0.00\\
instance n=20 77.alb & 1 & 1 & Optimal &  0.14 & 3 &  3.00 &  0.00\\
instance n=20 78.alb & 1 & 1 & Optimal &  0.13 & 3 &  3.00 &  0.00\\
instance n=20 79.alb & 1 & 1 & Optimal &  0.12 & 3 &  3.00 &  0.00\\
instance n=20 8.alb & 1 & 1 & Optimal &  0.16 & 3 &  3.00 &  0.00\\
instance n=20 80.alb & 1 & 1 & Optimal &  0.11 & 3 &  3.00 &  0.00\\
instance n=20 81.alb & 1 & 1 & Optimal &  0.21 & 3 &  3.00 &  0.00\\
instance n=20 82.alb & 1 & 1 & Optimal &  0.22 & 4 &  4.00 &  0.00\\
instance n=20 83.alb & 1 & 1 & Optimal &  0.10 & 3 &  3.00 &  0.00\\
instance n=20 84.alb & 1 & 1 & Optimal &  0.23 & 3 &  3.00 &  0.00\\
instance n=20 85.alb & 1 & 1 & Optimal &  0.28 & 3 &  3.00 &  0.00\\
instance n=20 86.alb & 1 & 1 & Optimal &  0.16 & 3 &  3.00 &  0.00\\
instance n=20 87.alb & 1 & 1 & Optimal &  0.33 & 3 &  3.00 &  0.00\\
instance n=20 88.alb & 1 & 1 & Optimal &  0.21 & 3 &  3.00 &  0.00\\
instance n=20 89.alb & 1 & 1 & Optimal &  0.20 & 3 &  3.00 &  0.00\\
instance n=20 9.alb & 1 & 1 & Optimal &  0.77 & 3 &  3.00 &  0.00\\
instance n=20 90.alb & 1 & 1 & Optimal &  0.25 & 3 &  3.00 &  0.00\\
instance n=20 91.alb & 1 & 1 & Optimal &  0.14 & 11 & 11.00 &  0.00\\
instance n=20 92.alb & 1 & 1 & Optimal &  0.24 & 11 & 11.00 &  0.00\\
instance n=20 93.alb & 1 & 1 & Optimal &  1.07 & 13 & 13.00 &  0.00\\
instance n=20 94.alb & 1 & 1 & Optimal &  0.22 & 10 & 10.00 &  0.00\\
instance n=20 95.alb & 1 & 1 & Optimal &  0.22 & 12 & 12.00 &  0.00\\
instance n=20 96.alb & 1 & 1 & Optimal &  0.22 & 10 & 10.00 &  0.00\\
instance n=20 97.alb & 1 & 1 & Optimal &  5.29 & 15 & 15.00 &  0.00\\
instance n=20 98.alb & 1 & 1 & Optimal &  0.99 & 13 & 13.00 &  0.00\\
instance n=20 99.alb & 1 & 1 & Optimal &  0.43 & 12 & 12.00 &  0.00\\
instance n=50 1.alb & 1 & 1 & Optimal &  2.41 & 8 &  8.00 &  0.00\\
instance n=50 10.alb & 1 & 1 & Optimal &  1.81 & 7 &  7.00 &  0.00\\
instance n=50 100.alb & 1 & 1 & Optimal &  2.36 & 7 &  7.00 &  0.00\\
instance n=50 101.alb & 1 & 1 & Solution & 120.02 & 30 & 26.00 & 13.33\\
instance n=50 102.alb & 1 & 1 & Solution & 120.02 & 32 & 27.00 & 15.63\\
instance n=50 103.alb & 1 & 1 & Solution & 120.02 & 29 & 25.00 & 13.79\\
instance n=50 104.alb & 1 & 1 & Solution & 120.01 & 27 & 25.00 &  7.41\\
instance n=50 105.alb & 1 & 1 & Solution & 120.01 & 24 & 23.00 &  4.17\\
instance n=50 106.alb & 1 & 1 & Solution & 120.02 & 28 & 26.00 &  7.14\\
instance n=50 107.alb & 1 & 1 & Solution & 120.02 & 28 & 27.00 &  3.57\\
instance n=50 108.alb & 1 & 1 & Solution & 120.02 & 30 & 26.00 & 13.33\\
instance n=50 109.alb & 1 & 1 & Solution & 120.01 & 30 & 25.00 & 16.67\\
instance n=50 11.alb & 1 & 1 & Optimal &  3.30 & 7 &  7.00 &  0.00\\
instance n=50 110.alb & 1 & 1 & Solution & 120.01 & 26 & 25.00 &  3.85\\
instance n=50 111.alb & 1 & 1 & Solution & 120.01 & 28 & 26.00 &  7.14\\
instance n=50 112.alb & 1 & 1 & Solution & 120.02 & 27 & 25.00 &  7.41\\
instance n=50 113.alb & 1 & 1 & Solution & 120.01 & 28 & 26.00 &  7.14\\
instance n=50 114.alb & 1 & 1 & Solution & 120.01 & 27 & 26.00 &  3.70\\
instance n=50 115.alb & 1 & 1 & Solution & 120.02 & 28 & 25.00 & 10.71\\
instance n=50 116.alb & 1 & 1 & Solution & 120.01 & 32 & 26.00 & 18.75\\
instance n=50 117.alb & 1 & 1 & Solution & 120.01 & 27 & 25.00 &  7.41\\
instance n=50 118.alb & 1 & 1 & Solution & 120.01 & 29 & 26.00 & 10.34\\
instance n=50 119.alb & 1 & 1 & Solution & 120.01 & 25 & 25.00 &  0.00\\
instance n=50 12.alb & 1 & 1 & Optimal &  0.10 & 6 &  6.00 &  0.00\\
instance n=50 120.alb & 1 & 1 & Solution & 120.01 & 27 & 26.00 &  3.70\\
instance n=50 121.alb & 1 & 1 & Solution & 120.02 & 32 & 27.00 & 15.63\\
instance n=50 122.alb & 1 & 1 & Solution & 120.01 & 29 & 26.00 & 10.34\\
instance n=50 123.alb & 1 & 1 & Solution & 120.02 & 32 & 26.00 & 18.75\\
instance n=50 124.alb & 1 & 1 & Solution & 120.01 & 29 & 26.00 & 10.34\\
instance n=50 125.alb & 1 & 1 & Solution & 120.01 & 33 & 27.00 & 18.18\\
instance n=50 126.alb & 1 & 1 & Optimal &  1.24 & 12 & 12.00 &  0.00\\
instance n=50 127.alb & 1 & 1 & Optimal & 12.49 & 14 & 14.00 &  0.00\\
instance n=50 128.alb & 1 & 1 & Optimal & 13.27 & 12 & 12.00 &  0.00\\
instance n=50 129.alb & 1 & 1 & Optimal &  3.23 & 13 & 13.00 &  0.00\\
instance n=50 13.alb & 1 & 1 & Optimal &  0.69 & 6 &  6.00 &  0.00\\
instance n=50 130.alb & 1 & 1 & Optimal &  7.38 & 13 & 13.00 &  0.00\\
instance n=50 131.alb & 1 & 1 & Optimal & 24.71 & 12 & 12.00 &  0.00\\
instance n=50 132.alb & 1 & 1 & Optimal & 19.58 & 12 & 12.00 &  0.00\\
instance n=50 133.alb & 1 & 1 & Optimal &  5.01 & 12 & 12.00 &  0.00\\
instance n=50 134.alb & 1 & 1 & Optimal & 26.48 & 14 & 14.00 &  0.00\\
instance n=50 135.alb & 1 & 1 & Optimal &  8.21 & 13 & 13.00 &  0.00\\
instance n=50 136.alb & 1 & 1 & Optimal &  6.46 & 11 & 11.00 &  0.00\\
instance n=50 137.alb & 1 & 1 & Optimal &  8.85 & 11 & 11.00 &  0.00\\
instance n=50 138.alb & 1 & 1 & Optimal & 21.31 & 12 & 12.00 &  0.00\\
instance n=50 139.alb & 1 & 1 & Optimal & 32.22 & 11 & 11.00 &  0.00\\
instance n=50 14.alb & 1 & 1 & Optimal &  4.93 & 7 &  7.00 &  0.00\\
instance n=50 140.alb & 1 & 1 & Optimal &  3.80 & 12 & 12.00 &  0.00\\
instance n=50 141.alb & 1 & 1 & Optimal &  7.59 & 13 & 13.00 &  0.00\\
instance n=50 142.alb & 1 & 1 & Optimal & 13.14 & 11 & 11.00 &  0.00\\
instance n=50 143.alb & 1 & 1 & Optimal &  5.27 & 12 & 12.00 &  0.00\\
instance n=50 144.alb & 1 & 1 & Optimal &  1.27 & 13 & 13.00 &  0.00\\
instance n=50 145.alb & 1 & 1 & Optimal &  4.29 & 10 & 10.00 &  0.00\\
instance n=50 146.alb & 1 & 1 & Optimal &  4.94 & 13 & 13.00 &  0.00\\
instance n=50 147.alb & 1 & 1 & Optimal &  1.14 & 13 & 13.00 &  0.00\\
instance n=50 148.alb & 1 & 1 & Optimal &  2.88 & 10 & 10.00 &  0.00\\
instance n=50 149.alb & 1 & 1 & Optimal &  3.41 & 12 & 12.00 &  0.00\\
instance n=50 15.alb & 1 & 1 & Optimal &  6.67 & 8 &  8.00 &  0.00\\
instance n=50 150.alb & 1 & 1 & Optimal &  1.78 & 11 & 11.00 &  0.00\\
instance n=50 151.alb & 1 & 1 & Optimal &  1.74 & 7 &  7.00 &  0.00\\
instance n=50 152.alb & 1 & 1 & Optimal &  3.61 & 7 &  7.00 &  0.00\\
instance n=50 153.alb & 1 & 1 & Optimal &  0.95 & 7 &  7.00 &  0.00\\
instance n=50 154.alb & 1 & 1 & Optimal &  4.31 & 8 &  8.00 &  0.00\\
instance n=50 155.alb & 1 & 1 & Optimal &  2.69 & 7 &  7.00 &  0.00\\
instance n=50 156.alb & 1 & 1 & Optimal &  3.08 & 7 &  7.00 &  0.00\\
instance n=50 157.alb & 1 & 1 & Optimal &  1.61 & 8 &  8.00 &  0.00\\
instance n=50 158.alb & 1 & 1 & Optimal &  1.13 & 7 &  7.00 &  0.00\\
instance n=50 159.alb & 1 & 1 & Optimal &  3.88 & 7 &  7.00 &  0.00\\
instance n=50 16.alb & 1 & 1 & Optimal &  2.19 & 8 &  8.00 &  0.00\\
instance n=50 160.alb & 1 & 1 & Optimal &  5.78 & 8 &  8.00 &  0.00\\
instance n=50 161.alb & 1 & 1 & Optimal &  1.17 & 7 &  7.00 &  0.00\\
instance n=50 162.alb & 1 & 1 & Optimal &  3.60 & 8 &  8.00 &  0.00\\
instance n=50 163.alb & 1 & 1 & Optimal &  3.51 & 7 &  7.00 &  0.00\\
instance n=50 164.alb & 1 & 1 & Optimal &  2.32 & 7 &  7.00 &  0.00\\
instance n=50 165.alb & 1 & 1 & Optimal &  6.20 & 8 &  8.00 &  0.00\\
instance n=50 166.alb & 1 & 1 & Optimal &  5.12 & 8 &  8.00 &  0.00\\
instance n=50 167.alb & 1 & 1 & Optimal &  4.73 & 7 &  7.00 &  0.00\\
instance n=50 168.alb & 1 & 1 & Optimal & 11.73 & 8 &  8.00 &  0.00\\
instance n=50 169.alb & 1 & 1 & Optimal &  9.66 & 8 &  8.00 &  0.00\\
instance n=50 17.alb & 1 & 1 & Optimal &  1.30 & 7 &  7.00 &  0.00\\
instance n=50 170.alb & 1 & 1 & Optimal &  4.51 & 7 &  7.00 &  0.00\\
instance n=50 171.alb & 1 & 1 & Optimal &  2.69 & 8 &  8.00 &  0.00\\
instance n=50 172.alb & 1 & 1 & Optimal &  0.59 & 7 &  7.00 &  0.00\\
instance n=50 173.alb & 1 & 1 & Optimal &  2.49 & 7 &  7.00 &  0.00\\
instance n=50 174.alb & 1 & 1 & Optimal &  2.45 & 7 &  7.00 &  0.00\\
instance n=50 175.alb & 1 & 1 & Optimal &  1.97 & 7 &  7.00 &  0.00\\
instance n=50 176.alb & 1 & 1 & Solution & 120.01 & 27 & 25.00 &  7.41\\
instance n=50 177.alb & 1 & 1 & Solution & 120.01 & 28 & 26.00 &  7.14\\
instance n=50 178.alb & 1 & 1 & Solution & 120.01 & 28 & 25.00 & 10.71\\
instance n=50 179.alb & 1 & 1 & Solution & 120.01 & 26 & 25.00 &  3.85\\
instance n=50 18.alb & 1 & 1 & Optimal &  3.40 & 7 &  7.00 &  0.00\\
instance n=50 180.alb & 1 & 1 & Solution & 120.02 & 26 & 25.00 &  3.85\\
instance n=50 181.alb & 1 & 1 & Solution & 120.01 & 29 & 26.00 & 10.34\\
instance n=50 182.alb & 1 & 1 & Solution & 120.01 & 26 & 25.00 &  3.85\\
instance n=50 183.alb & 1 & 1 & Solution & 120.01 & 29 & 26.00 & 10.34\\
instance n=50 184.alb & 1 & 1 & Solution & 120.01 & 38 & 28.00 & 26.32\\
instance n=50 185.alb & 1 & 1 & Solution & 120.01 & 26 & 25.00 &  3.85\\
instance n=50 186.alb & 1 & 1 & Solution & 120.01 & 26 & 25.00 &  3.85\\
instance n=50 187.alb & 1 & 1 & Solution & 120.01 & 26 & 25.00 &  3.85\\
instance n=50 188.alb & 1 & 1 & Solution & 120.01 & 25 & 24.00 &  4.00\\
instance n=50 189.alb & 1 & 1 & Solution & 120.01 & 26 & 25.00 &  3.85\\
instance n=50 19.alb & 1 & 1 & Optimal &  4.08 & 8 &  8.00 &  0.00\\
instance n=50 190.alb & 1 & 1 & Solution & 120.01 & 30 & 26.00 & 13.33\\
instance n=50 191.alb & 1 & 1 & Solution & 120.01 & 27 & 26.00 &  3.70\\
instance n=50 192.alb & 1 & 1 & Solution & 120.01 & 27 & 25.00 &  7.41\\
instance n=50 193.alb & 1 & 1 & Solution & 120.01 & 28 & 26.00 &  7.14\\
instance n=50 194.alb & 1 & 1 & Solution & 120.01 & 28 & 26.00 &  7.14\\
instance n=50 195.alb & 1 & 1 & Solution & 120.01 & 28 & 26.00 &  7.14\\
instance n=50 196.alb & 1 & 1 & Solution & 120.01 & 27 & 26.00 &  3.70\\
instance n=50 197.alb & 1 & 1 & Solution & 120.01 & 28 & 26.00 &  7.14\\
instance n=50 198.alb & 1 & 1 & Solution & 120.01 & 28 & 25.00 & 10.71\\
instance n=50 199.alb & 1 & 1 & Solution & 120.02 & 29 & 26.00 & 10.34\\
instance n=50 2.alb & 1 & 1 & Optimal &  0.08 & 6 &  6.00 &  0.00\\
instance n=50 20.alb & 1 & 1 & Optimal &  1.57 & 8 &  8.00 &  0.00\\
instance n=50 200.alb & 1 & 1 & Solution & 120.00 & 25 & 24.00 &  4.00\\
instance n=50 201.alb & 1 & 1 & Optimal & 54.74 & 13 & 13.00 &  0.00\\
instance n=50 202.alb & 1 & 1 & Optimal & 24.05 & 9 &  9.00 &  0.00\\
instance n=50 203.alb & 1 & 1 & Optimal & 23.16 & 11 & 11.00 &  0.00\\
instance n=50 204.alb & 1 & 1 & Optimal & 18.76 & 10 & 10.00 &  0.00\\
instance n=50 205.alb & 1 & 1 & Optimal & 48.36 & 13 & 13.00 &  0.00\\
instance n=50 206.alb & 1 & 1 & Solution & 120.01 & 11 & 11.00 &  0.00\\
instance n=50 207.alb & 1 & 1 & Optimal & 31.80 & 10 & 10.00 &  0.00\\
instance n=50 208.alb & 1 & 1 & Solution & 120.01 & 13 & 13.00 &  0.00\\
instance n=50 209.alb & 1 & 1 & Optimal & 13.39 & 11 & 11.00 &  0.00\\
instance n=50 21.alb & 1 & 1 & Optimal &  1.19 & 6 &  6.00 &  0.00\\
instance n=50 210.alb & 1 & 1 & Solution & 120.01 & 13 & 13.00 &  0.00\\
instance n=50 211.alb & 1 & 1 & Optimal & 17.75 & 12 & 12.00 &  0.00\\
instance n=50 212.alb & 1 & 1 & Optimal & 43.15 & 10 & 10.00 &  0.00\\
instance n=50 213.alb & 1 & 1 & Solution & 120.01 & 13 & 13.00 &  0.00\\
instance n=50 214.alb & 1 & 1 & Optimal & 56.73 & 11 & 11.00 &  0.00\\
instance n=50 215.alb & 1 & 1 & Optimal &  8.51 & 11 & 11.00 &  0.00\\
instance n=50 216.alb & 1 & 1 & Solution & 120.01 & 12 & 12.00 &  0.00\\
instance n=50 217.alb & 1 & 1 & Solution & 120.01 & 13 & 13.00 &  0.00\\
instance n=50 218.alb & 1 & 1 & Optimal & 59.05 & 12 & 12.00 &  0.00\\
instance n=50 219.alb & 1 & 1 & Optimal & 28.68 & 11 & 11.00 &  0.00\\
instance n=50 22.alb & 1 & 1 & Optimal &  2.35 & 7 &  7.00 &  0.00\\
instance n=50 220.alb & 1 & 1 & Optimal & 20.63 & 11 & 11.00 &  0.00\\
instance n=50 221.alb & 1 & 1 & Optimal & 27.74 & 11 & 11.00 &  0.00\\
instance n=50 222.alb & 1 & 1 & Optimal & 62.70 & 14 & 14.00 &  0.00\\
instance n=50 223.alb & 1 & 1 & Optimal & 48.00 & 11 & 11.00 &  0.00\\
instance n=50 224.alb & 1 & 1 & Optimal & 31.62 & 11 & 11.00 &  0.00\\
instance n=50 225.alb & 1 & 1 & Optimal &  2.05 & 12 & 12.00 &  0.00\\
instance n=50 226.alb & 1 & 1 & Optimal &  0.69 & 7 &  7.00 &  0.00\\
instance n=50 227.alb & 1 & 1 & Optimal &  0.43 & 6 &  6.00 &  0.00\\
instance n=50 228.alb & 1 & 1 & Optimal &  0.72 & 6 &  6.00 &  0.00\\
instance n=50 229.alb & 1 & 1 & Optimal &  1.41 & 6 &  6.00 &  0.00\\
instance n=50 23.alb & 1 & 1 & Optimal &  6.05 & 7 &  7.00 &  0.00\\
instance n=50 230.alb & 1 & 1 & Optimal &  0.95 & 7 &  7.00 &  0.00\\
instance n=50 231.alb & 1 & 1 & Optimal &  0.56 & 7 &  7.00 &  0.00\\
instance n=50 232.alb & 1 & 1 & Optimal &  1.46 & 7 &  7.00 &  0.00\\
instance n=50 233.alb & 1 & 1 & Optimal &  0.99 & 6 &  6.00 &  0.00\\
instance n=50 234.alb & 1 & 1 & Optimal &  0.49 & 8 &  8.00 &  0.00\\
instance n=50 235.alb & 1 & 1 & Optimal &  0.49 & 7 &  7.00 &  0.00\\
instance n=50 236.alb & 1 & 1 & Optimal &  1.90 & 7 &  7.00 &  0.00\\
instance n=50 237.alb & 1 & 1 & Optimal &  1.83 & 8 &  8.00 &  0.00\\
instance n=50 238.alb & 1 & 1 & Optimal &  0.64 & 7 &  7.00 &  0.00\\
instance n=50 239.alb & 1 & 1 & Optimal &  2.17 & 7 &  7.00 &  0.00\\
instance n=50 24.alb & 1 & 1 & Optimal &  2.18 & 7 &  7.00 &  0.00\\
instance n=50 240.alb & 1 & 1 & Optimal &  1.93 & 7 &  7.00 &  0.00\\
instance n=50 241.alb & 1 & 1 & Optimal &  1.46 & 7 &  7.00 &  0.00\\
instance n=50 242.alb & 1 & 1 & Optimal &  1.10 & 8 &  8.00 &  0.00\\
instance n=50 243.alb & 1 & 1 & Optimal &  1.09 & 7 &  7.00 &  0.00\\
instance n=50 244.alb & 1 & 1 & Optimal &  0.76 & 7 &  7.00 &  0.00\\
instance n=50 245.alb & 1 & 1 & Optimal &  1.98 & 7 &  7.00 &  0.00\\
instance n=50 246.alb & 1 & 1 & Optimal &  1.70 & 8 &  8.00 &  0.00\\
instance n=50 247.alb & 1 & 1 & Optimal &  0.83 & 7 &  7.00 &  0.00\\
instance n=50 248.alb & 1 & 1 & Optimal &  1.82 & 7 &  7.00 &  0.00\\
instance n=50 249.alb & 1 & 1 & Optimal &  2.12 & 7 &  7.00 &  0.00\\
instance n=50 25.alb & 1 & 1 & Optimal &  1.07 & 6 &  6.00 &  0.00\\
instance n=50 250.alb & 1 & 1 & Optimal &  1.10 & 7 &  7.00 &  0.00\\
instance n=50 251.alb & 1 & 1 & Solution & 120.11 & 27 & 25.00 &  7.41\\
instance n=50 252.alb & 1 & 1 & Solution & 120.01 & 32 & 27.00 & 15.63\\
instance n=50 253.alb & 1 & 1 & Solution & 120.01 & 28 & 26.00 &  7.14\\
instance n=50 254.alb & 1 & 1 & Solution & 120.02 & 30 & 26.00 & 13.33\\
instance n=50 255.alb & 1 & 1 & Solution & 120.01 & 29 & 25.00 & 13.79\\
instance n=50 256.alb & 1 & 1 & Solution & 120.01 & 30 & 27.00 & 10.00\\
instance n=50 257.alb & 1 & 1 & Solution & 120.01 & 33 & 27.00 & 18.18\\
instance n=50 258.alb & 1 & 1 & Solution & 120.01 & 28 & 25.00 & 10.71\\
instance n=50 259.alb & 1 & 1 & Solution & 120.01 & 31 & 26.00 & 16.13\\
instance n=50 26.alb & 1 & 1 & Solution & 120.01 & 27 & 25.00 &  7.41\\
instance n=50 260.alb & 1 & 1 & Solution & 120.01 & 29 & 26.00 & 10.34\\
instance n=50 261.alb & 1 & 1 & Solution & 120.01 & 28 & 25.00 & 10.71\\
instance n=50 262.alb & 1 & 1 & Solution & 120.01 & 31 & 26.00 & 16.13\\
instance n=50 263.alb & 1 & 1 & Solution & 120.01 & 30 & 26.00 & 13.33\\
instance n=50 264.alb & 1 & 1 & Solution & 120.01 & 27 & 25.00 &  7.41\\
instance n=50 265.alb & 1 & 1 & Solution & 120.01 & 27 & 25.00 &  7.41\\
instance n=50 266.alb & 1 & 1 & Optimal & 25.58 & 29 & 29.00 &  0.00\\
instance n=50 267.alb & 1 & 1 & Solution & 120.01 & 28 & 26.00 &  7.14\\
instance n=50 268.alb & 1 & 1 & Solution & 120.01 & 29 & 26.00 & 10.34\\
instance n=50 269.alb & 1 & 1 & Solution & 120.01 & 26 & 25.00 &  3.85\\
instance n=50 27.alb & 1 & 1 & Solution & 120.01 & 30 & 27.00 & 10.00\\
instance n=50 270.alb & 1 & 1 & Solution & 120.01 & 28 & 26.00 &  7.14\\
instance n=50 271.alb & 1 & 1 & Solution & 120.01 & 31 & 26.00 & 16.13\\
instance n=50 272.alb & 1 & 1 & Solution & 120.01 & 27 & 25.00 &  7.41\\
instance n=50 273.alb & 1 & 1 & Optimal & 21.79 & 27 & 27.00 &  0.00\\
instance n=50 274.alb & 1 & 1 & Solution & 120.01 & 29 & 26.00 & 10.34\\
instance n=50 275.alb & 1 & 1 & Solution & 120.01 & 27 & 26.00 &  3.70\\
instance n=50 276.alb & 1 & 1 & Optimal &  1.13 & 12 & 12.00 &  0.00\\
instance n=50 277.alb & 1 & 1 & Optimal &  0.98 & 13 & 13.00 &  0.00\\
instance n=50 278.alb & 1 & 1 & Optimal &  2.34 & 12 & 12.00 &  0.00\\
instance n=50 279.alb & 1 & 1 & Optimal &  4.97 & 11 & 11.00 &  0.00\\
instance n=50 28.alb & 1 & 1 & Solution & 120.01 & 28 & 26.00 &  7.14\\
instance n=50 280.alb & 1 & 1 & Optimal &  6.12 & 13 & 13.00 &  0.00\\
instance n=50 281.alb & 1 & 1 & Optimal &  1.49 & 11 & 11.00 &  0.00\\
instance n=50 282.alb & 1 & 1 & Optimal &  1.27 & 12 & 12.00 &  0.00\\
instance n=50 283.alb & 1 & 1 & Optimal &  2.01 & 12 & 12.00 &  0.00\\
instance n=50 284.alb & 1 & 1 & Optimal &  1.10 & 11 & 11.00 &  0.00\\
instance n=50 285.alb & 1 & 1 & Optimal &  1.07 & 13 & 13.00 &  0.00\\
instance n=50 286.alb & 1 & 1 & Optimal &  1.53 & 11 & 11.00 &  0.00\\
instance n=50 287.alb & 1 & 1 & Optimal &  5.08 & 12 & 12.00 &  0.00\\
instance n=50 288.alb & 1 & 1 & Optimal &  1.16 & 10 & 10.00 &  0.00\\
instance n=50 289.alb & 1 & 1 & Optimal &  5.95 & 11 & 11.00 &  0.00\\
instance n=50 29.alb & 1 & 1 & Solution & 120.01 & 29 & 25.00 & 13.79\\
instance n=50 290.alb & 1 & 1 & Optimal &  5.65 & 14 & 14.00 &  0.00\\
instance n=50 291.alb & 1 & 1 & Optimal &  2.26 & 12 & 12.00 &  0.00\\
instance n=50 292.alb & 1 & 1 & Optimal &  3.77 & 13 & 13.00 &  0.00\\
instance n=50 293.alb & 1 & 1 & Optimal &  2.23 & 12 & 12.00 &  0.00\\
instance n=50 294.alb & 1 & 1 & Optimal & 11.37 & 13 & 13.00 &  0.00\\
instance n=50 295.alb & 1 & 1 & Optimal &  4.51 & 16 & 16.00 &  0.00\\
instance n=50 296.alb & 1 & 1 & Optimal &  2.58 & 13 & 13.00 &  0.00\\
instance n=50 297.alb & 1 & 1 & Optimal &  1.94 & 13 & 13.00 &  0.00\\
instance n=50 298.alb & 1 & 1 & Optimal &  3.32 & 11 & 11.00 &  0.00\\
instance n=50 299.alb & 1 & 1 & Optimal &  1.34 & 12 & 12.00 &  0.00\\
instance n=50 3.alb & 1 & 1 & Optimal &  1.93 & 8 &  8.00 &  0.00\\
instance n=50 30.alb & 1 & 1 & Solution & 120.01 & 26 & 25.00 &  3.85\\
instance n=50 300.alb & 1 & 1 & Optimal &  1.31 & 12 & 12.00 &  0.00\\
instance n=50 301.alb & 1 & 1 & Optimal &  0.34 & 6 &  6.00 &  0.00\\
instance n=50 302.alb & 1 & 1 & Optimal &  0.91 & 7 &  7.00 &  0.00\\
instance n=50 303.alb & 1 & 1 & Optimal &  1.73 & 8 &  8.00 &  0.00\\
instance n=50 304.alb & 1 & 1 & Optimal &  2.48 & 7 &  7.00 &  0.00\\
instance n=50 305.alb & 1 & 1 & Optimal &  2.14 & 8 &  8.00 &  0.00\\
instance n=50 306.alb & 1 & 1 & Optimal &  2.44 & 7 &  7.00 &  0.00\\
instance n=50 307.alb & 1 & 1 & Optimal &  0.73 & 7 &  7.00 &  0.00\\
instance n=50 308.alb & 1 & 1 & Optimal &  2.41 & 8 &  8.00 &  0.00\\
instance n=50 309.alb & 1 & 1 & Optimal &  3.16 & 7 &  7.00 &  0.00\\
instance n=50 31.alb & 1 & 1 & Solution & 120.01 & 28 & 25.00 & 10.71\\
instance n=50 310.alb & 1 & 1 & Optimal &  4.62 & 8 &  8.00 &  0.00\\
instance n=50 311.alb & 1 & 1 & Optimal &  6.30 & 8 &  8.00 &  0.00\\
instance n=50 312.alb & 1 & 1 & Optimal &  0.36 & 6 &  6.00 &  0.00\\
instance n=50 313.alb & 1 & 1 & Optimal &  2.95 & 8 &  8.00 &  0.00\\
instance n=50 314.alb & 1 & 1 & Optimal &  0.73 & 7 &  7.00 &  0.00\\
instance n=50 315.alb & 1 & 1 & Optimal &  2.45 & 8 &  8.00 &  0.00\\
instance n=50 316.alb & 1 & 1 & Optimal &  2.57 & 8 &  8.00 &  0.00\\
instance n=50 317.alb & 1 & 1 & Optimal &  1.09 & 6 &  6.00 &  0.00\\
instance n=50 318.alb & 1 & 1 & Optimal &  5.75 & 8 &  8.00 &  0.00\\
instance n=50 319.alb & 1 & 1 & Optimal &  2.87 & 7 &  7.00 &  0.00\\
instance n=50 32.alb & 1 & 1 & Solution & 120.01 & 25 & 25.00 &  0.00\\
instance n=50 320.alb & 1 & 1 & Optimal &  4.95 & 8 &  8.00 &  0.00\\
instance n=50 321.alb & 1 & 1 & Optimal &  1.33 & 6 &  6.00 &  0.00\\
instance n=50 322.alb & 1 & 1 & Optimal &  0.97 & 7 &  7.00 &  0.00\\
instance n=50 323.alb & 1 & 1 & Optimal &  1.74 & 7 &  7.00 &  0.00\\
instance n=50 324.alb & 1 & 1 & Optimal &  1.44 & 7 &  7.00 &  0.00\\
instance n=50 325.alb & 1 & 1 & Optimal &  0.86 & 7 &  7.00 &  0.00\\
instance n=50 326.alb & 1 & 1 & Solution & 120.01 & 33 & 27.00 & 18.18\\
instance n=50 327.alb & 1 & 1 & Solution & 120.01 & 28 & 25.00 & 10.71\\
instance n=50 328.alb & 1 & 1 & Solution & 120.01 & 32 & 27.00 & 15.63\\
instance n=50 329.alb & 1 & 1 & Solution & 120.01 & 24 & 24.00 &  0.00\\
instance n=50 33.alb & 1 & 1 & Solution & 120.01 & 25 & 24.00 &  4.00\\
instance n=50 330.alb & 1 & 1 & Solution & 120.01 & 29 & 25.00 & 13.79\\
instance n=50 331.alb & 1 & 1 & Solution & 120.01 & 29 & 26.00 & 10.34\\
instance n=50 332.alb & 1 & 1 & Solution & 120.01 & 25 & 24.00 &  4.00\\
instance n=50 333.alb & 1 & 1 & Solution & 120.01 & 28 & 25.00 & 10.71\\
instance n=50 334.alb & 1 & 1 & Solution & 120.01 & 29 & 25.00 & 13.79\\
instance n=50 335.alb & 1 & 1 & Solution & 120.02 & 27 & 26.00 &  3.70\\
instance n=50 336.alb & 1 & 1 & Solution & 120.01 & 26 & 25.00 &  3.85\\
instance n=50 337.alb & 1 & 1 & Solution & 120.01 & 26 & 25.00 &  3.85\\
instance n=50 338.alb & 1 & 1 & Solution & 120.01 & 26 & 25.00 &  3.85\\
instance n=50 339.alb & 1 & 1 & Solution & 120.01 & 27 & 26.00 &  3.70\\
instance n=50 34.alb & 1 & 1 & Solution & 120.01 & 30 & 26.00 & 13.33\\
instance n=50 340.alb & 1 & 1 & Solution & 120.01 & 28 & 25.00 & 10.71\\
instance n=50 341.alb & 1 & 1 & Solution & 120.01 & 27 & 25.00 &  7.41\\
instance n=50 342.alb & 1 & 1 & Solution & 120.01 & 28 & 26.00 &  7.14\\
instance n=50 343.alb & 1 & 1 & Solution & 120.01 & 27 & 25.00 &  7.41\\
instance n=50 344.alb & 1 & 1 & Solution & 120.01 & 30 & 26.00 & 13.33\\
instance n=50 345.alb & 1 & 1 & Solution & 120.01 & 29 & 26.00 & 10.34\\
instance n=50 346.alb & 1 & 1 & Solution & 120.01 & 27 & 25.00 &  7.41\\
instance n=50 347.alb & 1 & 1 & Solution & 120.01 & 25 & 24.00 &  4.00\\
instance n=50 348.alb & 1 & 1 & Solution & 120.01 & 30 & 25.00 & 16.67\\
instance n=50 349.alb & 1 & 1 & Solution & 120.01 & 28 & 26.00 &  7.14\\
instance n=50 35.alb & 1 & 1 & Solution & 120.01 & 32 & 27.00 & 15.63\\
instance n=50 350.alb & 1 & 1 & Solution & 120.01 & 24 & 23.00 &  4.17\\
instance n=50 351.alb & 1 & 1 & Optimal & 103.80 & 12 & 12.00 &  0.00\\
instance n=50 352.alb & 1 & 1 & Optimal & 25.37 & 10 & 10.00 &  0.00\\
instance n=50 353.alb & 1 & 1 & Solution & 120.01 & 13 & 13.00 &  0.00\\
instance n=50 354.alb & 1 & 1 & Solution & 120.01 & 13 & 13.00 &  0.00\\
instance n=50 355.alb & 1 & 1 & Optimal & 11.73 & 11 & 11.00 &  0.00\\
instance n=50 356.alb & 1 & 1 & Optimal & 96.73 & 15 & 15.00 &  0.00\\
instance n=50 357.alb & 1 & 1 & Solution & 120.01 & 12 & 12.00 &  0.00\\
instance n=50 358.alb & 1 & 1 & Optimal &  1.78 & 11 & 11.00 &  0.00\\
instance n=50 359.alb & 1 & 1 & Optimal &  8.90 & 10 & 10.00 &  0.00\\
instance n=50 36.alb & 1 & 1 & Solution & 120.01 & 31 & 26.00 & 16.13\\
instance n=50 360.alb & 1 & 1 & Optimal & 81.66 & 12 & 12.00 &  0.00\\
instance n=50 361.alb & 1 & 1 & Optimal & 29.94 & 11 & 11.00 &  0.00\\
instance n=50 362.alb & 1 & 1 & Optimal & 36.11 & 10 & 10.00 &  0.00\\
instance n=50 363.alb & 1 & 1 & Optimal & 10.94 & 11 & 11.00 &  0.00\\
instance n=50 364.alb & 1 & 1 & Solution & 120.01 & 13 & 13.00 &  0.00\\
instance n=50 365.alb & 1 & 1 & Optimal &  8.09 & 11 & 11.00 &  0.00\\
instance n=50 366.alb & 1 & 1 & Solution & 120.01 & 13 & 13.00 &  0.00\\
instance n=50 367.alb & 1 & 1 & Solution & 120.01 & 12 & 12.00 &  0.00\\
instance n=50 368.alb & 1 & 1 & Optimal & 63.88 & 12 & 12.00 &  0.00\\
instance n=50 369.alb & 1 & 1 & Solution & 120.01 & 12 & 12.00 &  0.00\\
instance n=50 37.alb & 1 & 1 & Solution & 120.01 & 32 & 27.00 & 15.63\\
instance n=50 370.alb & 1 & 1 & Optimal & 26.02 & 12 & 12.00 &  0.00\\
instance n=50 371.alb & 1 & 1 & Optimal & 72.33 & 11 & 11.00 &  0.00\\
instance n=50 372.alb & 1 & 1 & Optimal & 69.83 & 10 & 10.00 &  0.00\\
instance n=50 373.alb & 1 & 1 & Solution & 120.01 & 12 & 12.00 &  0.00\\
instance n=50 374.alb & 1 & 1 & Optimal & 13.56 & 11 & 11.00 &  0.00\\
instance n=50 375.alb & 1 & 1 & Solution & 120.01 & 13 & 13.00 &  0.00\\
instance n=50 376.alb & 1 & 1 & Optimal &  2.32 & 7 &  7.00 &  0.00\\
instance n=50 377.alb & 1 & 1 & Optimal &  3.96 & 7 &  7.00 &  0.00\\
instance n=50 378.alb & 1 & 1 & Optimal &  1.41 & 8 &  8.00 &  0.00\\
instance n=50 379.alb & 1 & 1 & Optimal &  1.26 & 7 &  7.00 &  0.00\\
instance n=50 38.alb & 1 & 1 & Solution & 120.01 & 31 & 27.00 & 12.90\\
instance n=50 380.alb & 1 & 1 & Optimal &  2.90 & 7 &  7.00 &  0.00\\
instance n=50 381.alb & 1 & 1 & Optimal &  1.60 & 8 &  8.00 &  0.00\\
instance n=50 382.alb & 1 & 1 & Optimal &  1.70 & 6 &  6.00 &  0.00\\
instance n=50 383.alb & 1 & 1 & Optimal &  1.73 & 7 &  7.00 &  0.00\\
instance n=50 384.alb & 1 & 1 & Optimal &  2.31 & 8 &  8.00 &  0.00\\
instance n=50 385.alb & 1 & 1 & Optimal &  3.04 & 7 &  7.00 &  0.00\\
instance n=50 386.alb & 1 & 1 & Optimal &  2.36 & 7 &  7.00 &  0.00\\
instance n=50 387.alb & 1 & 1 & Optimal &  2.46 & 8 &  8.00 &  0.00\\
instance n=50 388.alb & 1 & 1 & Optimal &  1.89 & 7 &  7.00 &  0.00\\
instance n=50 389.alb & 1 & 1 & Optimal &  1.10 & 8 &  8.00 &  0.00\\
instance n=50 39.alb & 1 & 1 & Solution & 120.01 & 29 & 26.00 & 10.34\\
instance n=50 390.alb & 1 & 1 & Optimal &  3.18 & 7 &  7.00 &  0.00\\
instance n=50 391.alb & 1 & 1 & Optimal &  3.67 & 7 &  7.00 &  0.00\\
instance n=50 392.alb & 1 & 1 & Optimal &  1.90 & 8 &  8.00 &  0.00\\
instance n=50 393.alb & 1 & 1 & Optimal &  2.30 & 7 &  7.00 &  0.00\\
instance n=50 394.alb & 1 & 1 & Optimal &  2.32 & 8 &  8.00 &  0.00\\
instance n=50 395.alb & 1 & 1 & Optimal &  3.09 & 7 &  7.00 &  0.00\\
instance n=50 396.alb & 1 & 1 & Optimal &  1.89 & 8 &  8.00 &  0.00\\
instance n=50 397.alb & 1 & 1 & Optimal &  1.04 & 7 &  7.00 &  0.00\\
instance n=50 398.alb & 1 & 1 & Optimal &  1.15 & 6 &  6.00 &  0.00\\
instance n=50 399.alb & 1 & 1 & Optimal &  0.85 & 7 &  7.00 &  0.00\\
instance n=50 4.alb & 1 & 1 & Optimal &  2.80 & 7 &  7.00 &  0.00\\
instance n=50 40.alb & 1 & 1 & Solution & 120.01 & 26 & 25.00 &  3.85\\
instance n=50 400.alb & 1 & 1 & Optimal &  1.31 & 8 &  8.00 &  0.00\\
instance n=50 401.alb & 1 & 1 & Solution & 120.01 & 28 & 25.00 & 10.71\\
instance n=50 402.alb & 1 & 1 & Solution & 120.02 & 27 & 25.00 &  7.41\\
instance n=50 403.alb & 1 & 1 & Solution & 120.01 & 34 & 27.00 & 20.59\\
instance n=50 404.alb & 1 & 1 & Solution & 120.01 & 31 & 26.00 & 16.13\\
instance n=50 405.alb & 1 & 1 & Solution & 120.01 & 27 & 25.00 &  7.41\\
instance n=50 406.alb & 1 & 1 & Solution & 120.01 & 32 & 27.00 & 15.63\\
instance n=50 407.alb & 1 & 1 & Solution & 120.01 & 29 & 26.00 & 10.34\\
instance n=50 408.alb & 1 & 1 & Solution & 120.01 & 26 & 25.00 &  3.85\\
instance n=50 409.alb & 1 & 1 & Solution & 120.01 & 33 & 26.00 & 21.21\\
instance n=50 41.alb & 1 & 1 & Solution & 120.02 & 26 & 24.00 &  7.69\\
instance n=50 410.alb & 1 & 1 & Solution & 120.02 & 28 & 26.00 &  7.14\\
instance n=50 411.alb & 1 & 1 & Solution & 120.02 & 29 & 26.00 & 10.34\\
instance n=50 412.alb & 1 & 1 & Solution & 120.01 & 26 & 25.00 &  3.85\\
instance n=50 413.alb & 1 & 1 & Solution & 120.02 & 30 & 26.00 & 13.33\\
instance n=50 414.alb & 1 & 1 & Solution & 120.02 & 27 & 25.00 &  7.41\\
instance n=50 415.alb & 1 & 1 & Solution & 120.01 & 28 & 25.00 & 10.71\\
instance n=50 416.alb & 1 & 1 & Solution & 120.02 & 27 & 25.00 &  7.41\\
instance n=50 417.alb & 1 & 1 & Solution & 120.02 & 30 & 27.00 & 10.00\\
instance n=50 418.alb & 1 & 1 & Solution & 120.01 & 27 & 25.00 &  7.41\\
instance n=50 419.alb & 1 & 1 & Solution & 120.02 & 33 & 27.00 & 18.18\\
instance n=50 42.alb & 1 & 1 & Solution & 120.01 & 24 & 23.00 &  4.17\\
instance n=50 420.alb & 1 & 1 & Solution & 120.01 & 28 & 25.00 & 10.71\\
instance n=50 421.alb & 1 & 1 & Solution & 120.02 & 34 & 27.00 & 20.59\\
instance n=50 422.alb & 1 & 1 & Solution & 120.01 & 29 & 25.00 & 13.79\\
instance n=50 423.alb & 1 & 1 & Solution & 120.01 & 29 & 26.00 & 10.34\\
instance n=50 424.alb & 1 & 1 & Solution & 120.01 & 27 & 25.00 &  7.41\\
instance n=50 425.alb & 1 & 1 & Solution & 120.01 & 34 & 28.00 & 17.65\\
instance n=50 426.alb & 1 & 1 & Optimal &  5.34 & 11 & 11.00 &  0.00\\
instance n=50 427.alb & 1 & 1 & Optimal &  2.23 & 12 & 12.00 &  0.00\\
instance n=50 428.alb & 1 & 1 & Optimal & 12.89 & 13 & 13.00 &  0.00\\
instance n=50 429.alb & 1 & 1 & Optimal &  3.39 & 11 & 11.00 &  0.00\\
instance n=50 43.alb & 1 & 1 & Solution & 120.01 & 25 & 24.00 &  4.00\\
instance n=50 430.alb & 1 & 1 & Optimal & 24.16 & 14 & 14.00 &  0.00\\
instance n=50 431.alb & 1 & 1 & Optimal &  0.64 & 11 & 11.00 &  0.00\\
instance n=50 432.alb & 1 & 1 & Optimal &  6.40 & 12 & 12.00 &  0.00\\
instance n=50 433.alb & 1 & 1 & Optimal &  7.08 & 12 & 12.00 &  0.00\\
instance n=50 434.alb & 1 & 1 & Optimal &  3.09 & 11 & 11.00 &  0.00\\
instance n=50 435.alb & 1 & 1 & Optimal &  9.83 & 11 & 11.00 &  0.00\\
instance n=50 436.alb & 1 & 1 & Optimal &  3.12 & 11 & 11.00 &  0.00\\
instance n=50 437.alb & 1 & 1 & Optimal & 10.49 & 12 & 12.00 &  0.00\\
instance n=50 438.alb & 1 & 1 & Optimal &  7.92 & 10 & 10.00 &  0.00\\
instance n=50 439.alb & 1 & 1 & Optimal &  8.94 & 12 & 12.00 &  0.00\\
instance n=50 44.alb & 1 & 1 & Solution & 120.01 & 25 & 24.00 &  4.00\\
instance n=50 440.alb & 1 & 1 & Optimal &  2.79 & 13 & 13.00 &  0.00\\
instance n=50 441.alb & 1 & 1 & Optimal &  1.94 & 11 & 11.00 &  0.00\\
instance n=50 442.alb & 1 & 1 & Optimal &  4.21 & 12 & 12.00 &  0.00\\
instance n=50 443.alb & 1 & 1 & Optimal &  7.56 & 11 & 11.00 &  0.00\\
instance n=50 444.alb & 1 & 1 & Optimal &  7.65 & 12 & 12.00 &  0.00\\
instance n=50 445.alb & 1 & 1 & Optimal &  7.35 & 12 & 12.00 &  0.00\\
instance n=50 446.alb & 1 & 1 & Optimal &  5.38 & 12 & 12.00 &  0.00\\
instance n=50 447.alb & 1 & 1 & Optimal & 19.23 & 13 & 13.00 &  0.00\\
instance n=50 448.alb & 1 & 1 & Optimal & 10.60 & 12 & 12.00 &  0.00\\
instance n=50 449.alb & 1 & 1 & Optimal &  6.58 & 11 & 11.00 &  0.00\\
instance n=50 45.alb & 1 & 1 & Solution & 120.01 & 25 & 24.00 &  4.00\\
instance n=50 450.alb & 1 & 1 & Optimal &  7.07 & 11 & 11.00 &  0.00\\
instance n=50 451.alb & 1 & 1 & Optimal &  0.14 & 8 &  8.00 &  0.00\\
instance n=50 452.alb & 1 & 1 & Optimal &  0.23 & 8 &  8.00 &  0.00\\
instance n=50 453.alb & 1 & 1 & Optimal &  0.25 & 7 &  7.00 &  0.00\\
instance n=50 454.alb & 1 & 1 & Optimal &  0.21 & 8 &  8.00 &  0.00\\
instance n=50 455.alb & 1 & 1 & Optimal &  0.16 & 6 &  6.00 &  0.00\\
instance n=50 456.alb & 1 & 1 & Optimal &  0.26 & 8 &  8.00 &  0.00\\
instance n=50 457.alb & 1 & 1 & Optimal &  0.18 & 8 &  8.00 &  0.00\\
instance n=50 458.alb & 1 & 1 & Optimal &  0.11 & 7 &  7.00 &  0.00\\
instance n=50 459.alb & 1 & 1 & Optimal &  0.21 & 7 &  7.00 &  0.00\\
instance n=50 46.alb & 1 & 1 & Solution & 120.01 & 28 & 26.00 &  7.14\\
instance n=50 460.alb & 1 & 1 & Optimal &  0.20 & 7 &  7.00 &  0.00\\
instance n=50 461.alb & 1 & 1 & Optimal &  0.21 & 6 &  6.00 &  0.00\\
instance n=50 462.alb & 1 & 1 & Optimal &  0.11 & 7 &  7.00 &  0.00\\
instance n=50 463.alb & 1 & 1 & Optimal &  0.11 & 8 &  8.00 &  0.00\\
instance n=50 464.alb & 1 & 1 & Optimal &  0.45 & 6 &  6.00 &  0.00\\
instance n=50 465.alb & 1 & 1 & Optimal &  0.22 & 8 &  8.00 &  0.00\\
instance n=50 466.alb & 1 & 1 & Optimal &  0.15 & 7 &  7.00 &  0.00\\
instance n=50 467.alb & 1 & 1 & Optimal &  0.10 & 9 &  9.00 &  0.00\\
instance n=50 468.alb & 1 & 1 & Optimal &  0.21 & 7 &  7.00 &  0.00\\
instance n=50 469.alb & 1 & 1 & Optimal &  0.32 & 8 &  8.00 &  0.00\\
instance n=50 47.alb & 1 & 1 & Solution & 120.02 & 28 & 26.00 &  7.14\\
instance n=50 470.alb & 1 & 1 & Optimal &  0.14 & 8 &  8.00 &  0.00\\
instance n=50 471.alb & 1 & 1 & Optimal &  0.13 & 7 &  7.00 &  0.00\\
instance n=50 472.alb & 1 & 1 & Optimal &  0.10 & 8 &  8.00 &  0.00\\
instance n=50 473.alb & 1 & 1 & Optimal &  0.31 & 7 &  7.00 &  0.00\\
instance n=50 474.alb & 1 & 1 & Optimal &  0.11 & 7 &  7.00 &  0.00\\
instance n=50 475.alb & 1 & 1 & Optimal &  0.10 & 6 &  6.00 &  0.00\\
instance n=50 476.alb & 1 & 1 & Optimal &  0.97 & 28 & 28.00 &  0.00\\
instance n=50 477.alb & 1 & 1 & Optimal &  1.48 & 29 & 29.00 &  0.00\\
instance n=50 478.alb & 1 & 1 & Optimal &  2.65 & 32 & 32.00 &  0.00\\
instance n=50 479.alb & 1 & 1 & Optimal &  0.76 & 28 & 28.00 &  0.00\\
instance n=50 48.alb & 1 & 1 & Solution & 120.01 & 27 & 26.00 &  3.70\\
instance n=50 480.alb & 1 & 1 & Optimal &  1.18 & 34 & 34.00 &  0.00\\
instance n=50 481.alb & 1 & 1 & Optimal &  0.56 & 28 & 28.00 &  0.00\\
instance n=50 482.alb & 1 & 1 & Optimal &  0.69 & 27 & 27.00 &  0.00\\
instance n=50 483.alb & 1 & 1 & Optimal &  2.07 & 30 & 30.00 &  0.00\\
instance n=50 484.alb & 1 & 1 & Optimal &  1.07 & 32 & 32.00 &  0.00\\
instance n=50 485.alb & 1 & 1 & Optimal &  1.15 & 31 & 31.00 &  0.00\\
instance n=50 486.alb & 1 & 1 & Optimal &  1.08 & 32 & 32.00 &  0.00\\
instance n=50 487.alb & 1 & 1 & Optimal &  1.15 & 31 & 31.00 &  0.00\\
instance n=50 488.alb & 1 & 1 & Optimal &  2.81 & 31 & 31.00 &  0.00\\
instance n=50 489.alb & 1 & 1 & Optimal &  4.65 & 35 & 35.00 &  0.00\\
instance n=50 49.alb & 1 & 1 & Solution & 120.01 & 25 & 24.00 &  4.00\\
instance n=50 490.alb & 1 & 1 & Optimal &  0.95 & 29 & 29.00 &  0.00\\
instance n=50 491.alb & 1 & 1 & Optimal & 22.31 & 35 & 35.00 &  0.00\\
instance n=50 492.alb & 1 & 1 & Optimal &  2.25 & 29 & 29.00 &  0.00\\
instance n=50 493.alb & 1 & 1 & Optimal &  1.24 & 30 & 30.00 &  0.00\\
instance n=50 494.alb & 1 & 1 & Optimal &  3.34 & 32 & 32.00 &  0.00\\
instance n=50 495.alb & 1 & 1 & Optimal &  0.94 & 34 & 34.00 &  0.00\\
instance n=50 496.alb & 1 & 1 & Optimal &  1.17 & 29 & 29.00 &  0.00\\
instance n=50 497.alb & 1 & 1 & Optimal &  0.77 & 30 & 30.00 &  0.00\\
instance n=50 498.alb & 1 & 1 & Optimal &  1.17 & 30 & 30.00 &  0.00\\
instance n=50 499.alb & 1 & 1 & Optimal &  2.23 & 33 & 33.00 &  0.00\\
instance n=50 5.alb & 1 & 1 & Optimal &  1.51 & 7 &  7.00 &  0.00\\
instance n=50 50.alb & 1 & 1 & Solution & 120.01 & 27 & 25.00 &  7.41\\
instance n=50 500.alb & 1 & 1 & Optimal &  2.44 & 34 & 34.00 &  0.00\\
instance n=50 501.alb & 1 & 1 & Optimal &  0.34 & 12 & 12.00 &  0.00\\
instance n=50 502.alb & 1 & 1 & Optimal &  0.10 & 10 & 10.00 &  0.00\\
instance n=50 503.alb & 1 & 1 & Optimal &  0.26 & 13 & 13.00 &  0.00\\
instance n=50 504.alb & 1 & 1 & Optimal &  0.26 & 11 & 11.00 &  0.00\\
instance n=50 505.alb & 1 & 1 & Optimal &  0.31 & 12 & 12.00 &  0.00\\
instance n=50 506.alb & 1 & 1 & Optimal &  0.37 & 11 & 11.00 &  0.00\\
instance n=50 507.alb & 1 & 1 & Optimal &  0.18 & 13 & 13.00 &  0.00\\
instance n=50 508.alb & 1 & 1 & Optimal &  0.22 & 14 & 14.00 &  0.00\\
instance n=50 509.alb & 1 & 1 & Optimal &  0.22 & 13 & 13.00 &  0.00\\
instance n=50 51.alb & 1 & 1 & Optimal & 13.69 & 12 & 12.00 &  0.00\\
instance n=50 510.alb & 1 & 1 & Optimal &  0.10 & 11 & 11.00 &  0.00\\
instance n=50 511.alb & 1 & 1 & Optimal &  0.15 & 13 & 13.00 &  0.00\\
instance n=50 512.alb & 1 & 1 & Optimal &  0.31 & 13 & 13.00 &  0.00\\
instance n=50 513.alb & 1 & 1 & Optimal &  0.32 & 12 & 12.00 &  0.00\\
instance n=50 514.alb & 1 & 1 & Optimal &  0.20 & 12 & 12.00 &  0.00\\
instance n=50 515.alb & 1 & 1 & Optimal &  0.29 & 11 & 11.00 &  0.00\\
instance n=50 516.alb & 1 & 1 & Optimal &  0.28 & 13 & 13.00 &  0.00\\
instance n=50 517.alb & 1 & 1 & Optimal &  0.22 & 14 & 14.00 &  0.00\\
instance n=50 518.alb & 1 & 1 & Optimal &  0.49 & 11 & 11.00 &  0.00\\
instance n=50 519.alb & 1 & 1 & Optimal &  0.21 & 12 & 12.00 &  0.00\\
instance n=50 52.alb & 1 & 1 & Optimal & 71.48 & 11 & 11.00 &  0.00\\
instance n=50 520.alb & 1 & 1 & Optimal &  0.44 & 11 & 11.00 &  0.00\\
instance n=50 521.alb & 1 & 1 & Optimal &  0.29 & 10 & 10.00 &  0.00\\
instance n=50 522.alb & 1 & 1 & Optimal &  0.26 & 11 & 11.00 &  0.00\\
instance n=50 523.alb & 1 & 1 & Optimal &  0.27 & 11 & 11.00 &  0.00\\
instance n=50 524.alb & 1 & 1 & Optimal &  0.37 & 14 & 14.00 &  0.00\\
instance n=50 525.alb & 1 & 1 & Optimal &  0.15 & 11 & 11.00 &  0.00\\
instance n=50 53.alb & 1 & 1 & Solution & 120.01 & 12 & 12.00 &  0.00\\
instance n=50 54.alb & 1 & 1 & Optimal &  7.36 & 11 & 11.00 &  0.00\\
instance n=50 55.alb & 1 & 1 & Optimal & 35.50 & 13 & 13.00 &  0.00\\
instance n=50 56.alb & 1 & 1 & Optimal & 43.81 & 11 & 11.00 &  0.00\\
instance n=50 57.alb & 1 & 1 & Solution & 120.01 & 13 & 13.00 &  0.00\\
instance n=50 58.alb & 1 & 1 & Solution & 120.01 & 11 & 11.00 &  0.00\\
instance n=50 59.alb & 1 & 1 & Solution & 120.01 & 11 & 11.00 &  0.00\\
instance n=50 6.alb & 1 & 1 & Optimal &  1.23 & 6 &  6.00 &  0.00\\
instance n=50 60.alb & 1 & 1 & Optimal & 102.19 & 12 & 12.00 &  0.00\\
instance n=50 61.alb & 1 & 1 & Solution & 120.01 & 13 & 13.00 &  0.00\\
instance n=50 62.alb & 1 & 1 & Solution & 120.01 & 13 & 13.00 &  0.00\\
instance n=50 63.alb & 1 & 1 & Optimal & 115.14 & 12 & 12.00 &  0.00\\
instance n=50 64.alb & 1 & 1 & Optimal & 36.74 & 13 & 13.00 &  0.00\\
instance n=50 65.alb & 1 & 1 & Solution & 120.01 & 12 & 12.00 &  0.00\\
instance n=50 66.alb & 1 & 1 & Optimal & 84.17 & 12 & 12.00 &  0.00\\
instance n=50 67.alb & 1 & 1 & Solution & 120.01 & 12 & 12.00 &  0.00\\
instance n=50 68.alb & 1 & 1 & Optimal & 17.78 & 12 & 12.00 &  0.00\\
instance n=50 69.alb & 1 & 1 & Optimal & 92.32 & 12 & 12.00 &  0.00\\
instance n=50 7.alb & 1 & 1 & Optimal &  3.10 & 7 &  7.00 &  0.00\\
instance n=50 70.alb & 1 & 1 & Optimal & 36.21 & 10 & 10.00 &  0.00\\
instance n=50 71.alb & 1 & 1 & Solution & 120.01 & 13 & 13.00 &  0.00\\
instance n=50 72.alb & 1 & 1 & Optimal & 56.59 & 11 & 11.00 &  0.00\\
instance n=50 73.alb & 1 & 1 & Optimal & 100.54 & 11 & 11.00 &  0.00\\
instance n=50 74.alb & 1 & 1 & Solution & 120.01 & 12 & 12.00 &  0.00\\
instance n=50 75.alb & 1 & 1 & Optimal & 66.28 & 11 & 11.00 &  0.00\\
instance n=50 76.alb & 1 & 1 & Optimal &  2.43 & 7 &  7.00 &  0.00\\
instance n=50 77.alb & 1 & 1 & Optimal &  1.07 & 7 &  7.00 &  0.00\\
instance n=50 78.alb & 1 & 1 & Optimal &  1.90 & 7 &  7.00 &  0.00\\
instance n=50 79.alb & 1 & 1 & Optimal &  5.66 & 8 &  8.00 &  0.00\\
instance n=50 8.alb & 1 & 1 & Optimal &  3.64 & 7 &  7.00 &  0.00\\
instance n=50 80.alb & 1 & 1 & Optimal &  0.90 & 7 &  7.00 &  0.00\\
instance n=50 81.alb & 1 & 1 & Optimal &  1.85 & 7 &  7.00 &  0.00\\
instance n=50 82.alb & 1 & 1 & Optimal &  1.05 & 6 &  6.00 &  0.00\\
instance n=50 83.alb & 1 & 1 & Optimal &  8.47 & 8 &  8.00 &  0.00\\
instance n=50 84.alb & 1 & 1 & Optimal &  0.89 & 7 &  7.00 &  0.00\\
instance n=50 85.alb & 1 & 1 & Optimal &  2.09 & 8 &  8.00 &  0.00\\
instance n=50 86.alb & 1 & 1 & Optimal &  2.04 & 7 &  7.00 &  0.00\\
instance n=50 87.alb & 1 & 1 & Optimal &  5.79 & 8 &  8.00 &  0.00\\
instance n=50 88.alb & 1 & 1 & Optimal &  0.48 & 8 &  8.00 &  0.00\\
instance n=50 89.alb & 1 & 1 & Optimal &  1.47 & 7 &  7.00 &  0.00\\
instance n=50 9.alb & 1 & 1 & Optimal &  3.14 & 9 &  9.00 &  0.00\\
instance n=50 90.alb & 1 & 1 & Optimal &  1.41 & 7 &  7.00 &  0.00\\
instance n=50 91.alb & 1 & 1 & Optimal &  1.86 & 7 &  7.00 &  0.00\\
instance n=50 92.alb & 1 & 1 & Optimal &  4.71 & 7 &  7.00 &  0.00\\
instance n=50 93.alb & 1 & 1 & Optimal &  1.48 & 7 &  7.00 &  0.00\\
instance n=50 94.alb & 1 & 1 & Optimal &  3.18 & 7 &  7.00 &  0.00\\
instance n=50 95.alb & 1 & 1 & Optimal &  1.29 & 7 &  7.00 &  0.00\\
instance n=50 96.alb & 1 & 1 & Optimal &  1.03 & 7 &  7.00 &  0.00\\
instance n=50 97.alb & 1 & 1 & Optimal &  1.40 & 7 &  7.00 &  0.00\\
instance n=50 98.alb & 1 & 1 & Optimal &  3.38 & 8 &  8.00 &  0.00\\
instance n=50 99.alb & 1 & 1 & Optimal &  2.28 & 7 &  7.00 &  0.00\\
\end{longtable}



\subsection{CPSat}

\begin{longtable}{lrrlrrrr}
\caption{Results for SALBP-1 Problems Alternative (CPSat) (1367 Instances)}\\\toprule
Name & \shortstack{Nr\\Jobs} & \shortstack{Nr\\Machines} & Status & Time & Makespan & Bound & \shortstack{Gap\\Percent}\\ \midrule
\endhead
\bottomrule
\endfoot
instance n=100 1.alb & 1 & 1 & Solution & 30.05 & 24 & 23.00 &  4.17\\
instance n=100 10.alb & 1 & 1 & Solution & 30.06 & 22 & 22.00 &  0.00\\
instance n=100 100.alb & 1 & 1 & Solution & 30.05 & 26 & 25.00 &  3.85\\
instance n=100 101.alb & 1 & 1 & Solution & 30.05 & 15 & 15.00 &  0.00\\
instance n=100 102.alb & 1 & 1 & Solution & 30.04 & 14 & 14.00 &  0.00\\
instance n=100 103.alb & 1 & 1 & Solution & 30.06 & 14 & 14.00 &  0.00\\
instance n=100 104.alb & 1 & 1 & Solution & 30.05 & 14 & 14.00 &  0.00\\
instance n=100 105.alb & 1 & 1 & Solution & 30.05 & 13 & 13.00 &  0.00\\
instance n=100 106.alb & 1 & 1 & Solution & 30.04 & 14 & 14.00 &  0.00\\
instance n=100 107.alb & 1 & 1 & Solution & 30.04 & 14 & 14.00 &  0.00\\
instance n=100 108.alb & 1 & 1 & Solution & 30.05 & 15 & 14.00 &  6.67\\
instance n=100 109.alb & 1 & 1 & Solution & 30.05 & 15 & 15.00 &  0.00\\
instance n=100 11.alb & 1 & 1 & Solution & 30.04 & 24 & 24.00 &  0.00\\
instance n=100 110.alb & 1 & 1 & Solution & 30.06 & 13 & 13.00 &  0.00\\
instance n=100 111.alb & 1 & 1 & Solution & 30.06 & 16 & 16.00 &  0.00\\
instance n=100 112.alb & 1 & 1 & Solution & 30.04 & 13 & 13.00 &  0.00\\
instance n=100 113.alb & 1 & 1 & Solution & 30.04 & 14 & 14.00 &  0.00\\
instance n=100 114.alb & 1 & 1 & Solution & 30.02 & 13 & 13.00 &  0.00\\
instance n=100 115.alb & 1 & 1 & Solution & 30.05 & 14 & 14.00 &  0.00\\
instance n=100 116.alb & 1 & 1 & Solution & 30.03 & 16 & 16.00 &  0.00\\
instance n=100 117.alb & 1 & 1 & Solution & 30.04 & 16 & 15.00 &  6.25\\
instance n=100 118.alb & 1 & 1 & Solution & 30.06 & 15 & 15.00 &  0.00\\
instance n=100 119.alb & 1 & 1 & Solution & 30.06 & 14 & 14.00 &  0.00\\
instance n=100 12.alb & 1 & 1 & Solution & 30.04 & 25 & 25.00 &  0.00\\
instance n=100 120.alb & 1 & 1 & Solution & 30.04 & 14 & 14.00 &  0.00\\
instance n=100 121.alb & 1 & 1 & Solution & 30.10 & 15 & 15.00 &  0.00\\
instance n=100 122.alb & 1 & 1 & Solution & 30.04 & 13 & 13.00 &  0.00\\
instance n=100 123.alb & 1 & 1 & Solution & 30.04 & 15 & 15.00 &  0.00\\
instance n=100 124.alb & 1 & 1 & Solution & 30.05 & 15 & 15.00 &  0.00\\
instance n=100 125.alb & 1 & 1 & Solution & 30.05 & 14 & 14.00 &  0.00\\
instance n=100 126.alb & 1 & 1 & Solution & 30.04 & 52 & 49.00 &  5.77\\
instance n=100 127.alb & 1 & 1 & Solution & 30.04 & 54 & 49.00 &  9.26\\
instance n=100 128.alb & 1 & 1 & Solution & 30.04 & 58 & 52.00 & 10.34\\
instance n=100 129.alb & 1 & 1 & Solution & 30.05 & 56 & 50.00 & 10.71\\
instance n=100 13.alb & 1 & 1 & Solution & 30.04 & 24 & 24.00 &  0.00\\
instance n=100 130.alb & 1 & 1 & Solution & 30.04 & 56 & 51.00 &  8.93\\
instance n=100 131.alb & 1 & 1 & Solution & 30.05 & 55 & 50.00 &  9.09\\
instance n=100 132.alb & 1 & 1 & Solution & 30.03 & 59 & 52.00 & 11.86\\
instance n=100 133.alb & 1 & 1 & Solution & 30.04 & 57 & 51.00 & 10.53\\
instance n=100 134.alb & 1 & 1 & Solution & 30.06 & 56 & 51.00 &  8.93\\
instance n=100 135.alb & 1 & 1 & Solution & 30.03 & 58 & 51.00 & 12.07\\
instance n=100 136.alb & 1 & 1 & Solution & 30.04 & 58 & 49.00 & 15.52\\
instance n=100 137.alb & 1 & 1 & Solution & 30.05 & 57 & 50.00 & 12.28\\
instance n=100 138.alb & 1 & 1 & Solution & 30.02 & 59 & 52.00 & 11.86\\
instance n=100 139.alb & 1 & 1 & Solution & 30.03 & 52 & 49.00 &  5.77\\
instance n=100 14.alb & 1 & 1 & Solution & 30.04 & 20 & 20.00 &  0.00\\
instance n=100 140.alb & 1 & 1 & Solution & 30.04 & 56 & 51.00 &  8.93\\
instance n=100 141.alb & 1 & 1 & Solution & 30.04 & 52 & 49.00 &  5.77\\
instance n=100 142.alb & 1 & 1 & Solution & 30.04 & 55 & 50.00 &  9.09\\
instance n=100 143.alb & 1 & 1 & Solution & 30.04 & 55 & 50.00 &  9.09\\
instance n=100 144.alb & 1 & 1 & Solution & 30.04 & 49 & 47.00 &  4.08\\
instance n=100 145.alb & 1 & 1 & Solution & 30.03 & 59 & 51.00 & 13.56\\
instance n=100 146.alb & 1 & 1 & Solution & 30.04 & 53 & 50.00 &  5.66\\
instance n=100 147.alb & 1 & 1 & Solution & 30.03 & 61 & 52.00 & 14.75\\
instance n=100 148.alb & 1 & 1 & Solution & 30.04 & 55 & 50.00 &  9.09\\
instance n=100 149.alb & 1 & 1 & Solution & 30.05 & 56 & 51.00 &  8.93\\
instance n=100 15.alb & 1 & 1 & Solution & 30.05 & 24 & 24.00 &  0.00\\
instance n=100 150.alb & 1 & 1 & Solution & 30.06 & 58 & 51.00 & 12.07\\
instance n=100 151.alb & 1 & 1 & Solution & 30.05 & 22 & 21.00 &  4.55\\
instance n=100 152.alb & 1 & 1 & Solution & 30.05 & 22 & 22.00 &  0.00\\
instance n=100 153.alb & 1 & 1 & Solution & 30.06 & 21 & 21.00 &  0.00\\
instance n=100 154.alb & 1 & 1 & Solution & 30.04 & 25 & 25.00 &  0.00\\
instance n=100 155.alb & 1 & 1 & Solution & 30.06 & 22 & 22.00 &  0.00\\
instance n=100 156.alb & 1 & 1 & Solution & 30.06 & 23 & 23.00 &  0.00\\
instance n=100 157.alb & 1 & 1 & Solution & 30.05 & 26 & 26.00 &  0.00\\
instance n=100 158.alb & 1 & 1 & Solution & 30.05 & 23 & 23.00 &  0.00\\
instance n=100 159.alb & 1 & 1 & Solution & 30.03 & 19 & 19.00 &  0.00\\
instance n=100 16.alb & 1 & 1 & Solution & 30.03 & 23 & 23.00 &  0.00\\
instance n=100 160.alb & 1 & 1 & Solution & 30.06 & 22 & 22.00 &  0.00\\
instance n=100 161.alb & 1 & 1 & Solution & 30.05 & 23 & 22.00 &  4.35\\
instance n=100 162.alb & 1 & 1 & Solution & 30.03 & 23 & 22.00 &  4.35\\
instance n=100 163.alb & 1 & 1 & Solution & 30.06 & 25 & 25.00 &  0.00\\
instance n=100 164.alb & 1 & 1 & Solution & 30.03 & 23 & 23.00 &  0.00\\
instance n=100 165.alb & 1 & 1 & Solution & 30.03 & 25 & 24.00 &  4.00\\
instance n=100 166.alb & 1 & 1 & Solution & 30.05 & 24 & 24.00 &  0.00\\
instance n=100 167.alb & 1 & 1 & Solution & 30.14 & 22 & 22.00 &  0.00\\
instance n=100 168.alb & 1 & 1 & Solution & 30.07 & 22 & 21.00 &  4.55\\
instance n=100 169.alb & 1 & 1 & Solution & 30.04 & 21 & 21.00 &  0.00\\
instance n=100 17.alb & 1 & 1 & Solution & 30.10 & 22 & 21.00 &  4.55\\
instance n=100 170.alb & 1 & 1 & Solution & 30.03 & 25 & 24.00 &  4.00\\
instance n=100 171.alb & 1 & 1 & Solution & 30.04 & 25 & 24.00 &  4.00\\
instance n=100 172.alb & 1 & 1 & Solution & 30.06 & 24 & 24.00 &  0.00\\
instance n=100 173.alb & 1 & 1 & Solution & 30.06 & 25 & 24.00 &  4.00\\
instance n=100 174.alb & 1 & 1 & Solution & 30.05 & 23 & 22.00 &  4.35\\
instance n=100 175.alb & 1 & 1 & Solution & 30.03 & 27 & 26.00 &  3.70\\
instance n=100 176.alb & 1 & 1 & Solution & 30.05 & 13 & 13.00 &  0.00\\
instance n=100 177.alb & 1 & 1 & Solution & 30.04 & 14 & 14.00 &  0.00\\
instance n=100 178.alb & 1 & 1 & Solution & 30.04 & 15 & 15.00 &  0.00\\
instance n=100 179.alb & 1 & 1 & Solution & 30.06 & 15 & 15.00 &  0.00\\
instance n=100 18.alb & 1 & 1 & Solution & 30.04 & 20 & 19.00 &  5.00\\
instance n=100 180.alb & 1 & 1 & Solution & 30.05 & 15 & 15.00 &  0.00\\
instance n=100 181.alb & 1 & 1 & Solution & 30.05 & 13 & 13.00 &  0.00\\
instance n=100 182.alb & 1 & 1 & Solution & 30.05 & 15 & 15.00 &  0.00\\
instance n=100 183.alb & 1 & 1 & Solution & 30.05 & 14 & 14.00 &  0.00\\
instance n=100 184.alb & 1 & 1 & Solution & 30.04 & 14 & 14.00 &  0.00\\
instance n=100 185.alb & 1 & 1 & Solution & 30.06 & 15 & 15.00 &  0.00\\
instance n=100 186.alb & 1 & 1 & Solution & 30.06 & 14 & 14.00 &  0.00\\
instance n=100 187.alb & 1 & 1 & Solution & 30.08 & 14 & 13.00 &  7.14\\
instance n=100 188.alb & 1 & 1 & Solution & 30.06 & 16 & 16.00 &  0.00\\
instance n=100 189.alb & 1 & 1 & Solution & 30.05 & 14 & 14.00 &  0.00\\
instance n=100 19.alb & 1 & 1 & Solution & 30.04 & 23 & 23.00 &  0.00\\
instance n=100 190.alb & 1 & 1 & Solution & 30.03 & 13 & 13.00 &  0.00\\
instance n=100 191.alb & 1 & 1 & Solution & 30.05 & 14 & 14.00 &  0.00\\
instance n=100 192.alb & 1 & 1 & Solution & 30.05 & 13 & 13.00 &  0.00\\
instance n=100 193.alb & 1 & 1 & Solution & 30.06 & 15 & 15.00 &  0.00\\
instance n=100 194.alb & 1 & 1 & Solution & 30.04 & 15 & 15.00 &  0.00\\
instance n=100 195.alb & 1 & 1 & Solution & 30.06 & 15 & 15.00 &  0.00\\
instance n=100 196.alb & 1 & 1 & Solution & 30.05 & 15 & 15.00 &  0.00\\
instance n=100 197.alb & 1 & 1 & Solution & 30.07 & 15 & 15.00 &  0.00\\
instance n=100 198.alb & 1 & 1 & Solution & 30.06 & 13 & 13.00 &  0.00\\
instance n=100 199.alb & 1 & 1 & Solution & 30.06 & 14 & 14.00 &  0.00\\
instance n=100 2.alb & 1 & 1 & Solution & 30.05 & 21 & 21.00 &  0.00\\
instance n=100 20.alb & 1 & 1 & Solution & 30.05 & 21 & 21.00 &  0.00\\
instance n=100 200.alb & 1 & 1 & Solution & 30.06 & 15 & 15.00 &  0.00\\
instance n=100 201.alb & 1 & 1 & Solution & 30.04 & 55 & 50.00 &  9.09\\
instance n=100 202.alb & 1 & 1 & Solution & 30.04 & 65 & 52.00 & 20.00\\
instance n=100 203.alb & 1 & 1 & Solution & 30.05 & 53 & 49.00 &  7.55\\
instance n=100 204.alb & 1 & 1 & Solution & 30.05 & 51 & 48.00 &  5.88\\
instance n=100 205.alb & 1 & 1 & Solution & 30.05 & 59 & 51.00 & 13.56\\
instance n=100 206.alb & 1 & 1 & Solution & 30.03 & 54 & 49.00 &  9.26\\
instance n=100 207.alb & 1 & 1 & Solution & 30.06 & 53 & 49.00 &  7.55\\
instance n=100 208.alb & 1 & 1 & Solution & 30.04 & 59 & 51.00 & 13.56\\
instance n=100 209.alb & 1 & 1 & Solution & 30.07 & 58 & 51.00 & 12.07\\
instance n=100 21.alb & 1 & 1 & Solution & 30.06 & 21 & 21.00 &  0.00\\
instance n=100 210.alb & 1 & 1 & Solution & 30.05 & 53 & 49.00 &  7.55\\
instance n=100 211.alb & 1 & 1 & Solution & 30.05 & 53 & 49.00 &  7.55\\
instance n=100 212.alb & 1 & 1 & Solution & 30.06 & 54 & 50.00 &  7.41\\
instance n=100 213.alb & 1 & 1 & Solution & 30.03 & 55 & 50.00 &  9.09\\
instance n=100 214.alb & 1 & 1 & Solution & 30.04 & 56 & 50.00 & 10.71\\
instance n=100 215.alb & 1 & 1 & Solution & 30.05 & 51 & 47.00 &  7.84\\
instance n=100 216.alb & 1 & 1 & Solution & 30.05 & 55 & 50.00 &  9.09\\
instance n=100 217.alb & 1 & 1 & Solution & 30.05 & 55 & 49.00 & 10.91\\
instance n=100 218.alb & 1 & 1 & Solution & 30.07 & 55 & 50.00 &  9.09\\
instance n=100 219.alb & 1 & 1 & Solution & 30.04 & 54 & 49.00 &  9.26\\
instance n=100 22.alb & 1 & 1 & Solution & 30.08 & 25 & 24.00 &  4.00\\
instance n=100 220.alb & 1 & 1 & Solution & 30.03 & 55 & 50.00 &  9.09\\
instance n=100 221.alb & 1 & 1 & Solution & 30.04 & 60 & 51.00 & 15.00\\
instance n=100 222.alb & 1 & 1 & Solution & 30.05 & 55 & 50.00 &  9.09\\
instance n=100 223.alb & 1 & 1 & Solution & 30.05 & 53 & 49.00 &  7.55\\
instance n=100 224.alb & 1 & 1 & Solution & 30.04 & 57 & 51.00 & 10.53\\
instance n=100 225.alb & 1 & 1 & Solution & 30.06 & 55 & 50.00 &  9.09\\
instance n=100 226.alb & 1 & 1 & Solution & 30.03 & 25 & 24.00 &  4.00\\
instance n=100 227.alb & 1 & 1 & Solution & 30.04 & 27 & 26.00 &  3.70\\
instance n=100 228.alb & 1 & 1 & Solution & 30.05 & 23 & 22.00 &  4.35\\
instance n=100 229.alb & 1 & 1 & Solution & 30.06 & 24 & 24.00 &  0.00\\
instance n=100 23.alb & 1 & 1 & Solution & 30.06 & 24 & 24.00 &  0.00\\
instance n=100 230.alb & 1 & 1 & Solution & 30.03 & 24 & 23.00 &  4.17\\
instance n=100 231.alb & 1 & 1 & Solution & 30.03 & 22 & 22.00 &  0.00\\
instance n=100 232.alb & 1 & 1 & Solution & 30.04 & 22 & 22.00 &  0.00\\
instance n=100 233.alb & 1 & 1 & Solution & 30.04 & 23 & 22.00 &  4.35\\
instance n=100 234.alb & 1 & 1 & Solution & 30.03 & 23 & 23.00 &  0.00\\
instance n=100 235.alb & 1 & 1 & Solution & 30.03 & 26 & 26.00 &  0.00\\
instance n=100 236.alb & 1 & 1 & Solution & 30.06 & 23 & 22.00 &  4.35\\
instance n=100 237.alb & 1 & 1 & Solution & 30.06 & 23 & 23.00 &  0.00\\
instance n=100 238.alb & 1 & 1 & Solution & 30.05 & 23 & 23.00 &  0.00\\
instance n=100 239.alb & 1 & 1 & Solution & 30.03 & 21 & 21.00 &  0.00\\
instance n=100 24.alb & 1 & 1 & Solution & 30.06 & 24 & 24.00 &  0.00\\
instance n=100 240.alb & 1 & 1 & Solution & 30.04 & 22 & 22.00 &  0.00\\
instance n=100 241.alb & 1 & 1 & Solution & 30.04 & 23 & 22.00 &  4.35\\
instance n=100 242.alb & 1 & 1 & Solution & 30.05 & 24 & 23.00 &  4.17\\
instance n=100 243.alb & 1 & 1 & Solution & 30.04 & 24 & 23.00 &  4.17\\
instance n=100 244.alb & 1 & 1 & Solution & 30.04 & 21 & 21.00 &  0.00\\
instance n=100 245.alb & 1 & 1 & Solution & 30.05 & 24 & 23.00 &  4.17\\
instance n=100 246.alb & 1 & 1 & Solution & 30.03 & 27 & 26.00 &  3.70\\
instance n=100 247.alb & 1 & 1 & Solution & 30.05 & 22 & 22.00 &  0.00\\
instance n=100 248.alb & 1 & 1 & Solution & 30.06 & 19 & 19.00 &  0.00\\
instance n=100 249.alb & 1 & 1 & Solution & 30.07 & 21 & 21.00 &  0.00\\
instance n=100 25.alb & 1 & 1 & Solution & 30.05 & 22 & 22.00 &  0.00\\
instance n=100 250.alb & 1 & 1 & Solution & 30.03 & 24 & 24.00 &  0.00\\
instance n=100 251.alb & 1 & 1 & Solution & 30.06 & 15 & 15.00 &  0.00\\
instance n=100 252.alb & 1 & 1 & Solution & 30.05 & 14 & 14.00 &  0.00\\
instance n=100 253.alb & 1 & 1 & Solution & 30.06 & 14 & 14.00 &  0.00\\
instance n=100 254.alb & 1 & 1 & Solution & 30.07 & 14 & 14.00 &  0.00\\
instance n=100 255.alb & 1 & 1 & Optimal & 30.01 & 14 & 14.00 &  0.00\\
instance n=100 256.alb & 1 & 1 & Solution & 30.06 & 15 & 15.00 &  0.00\\
instance n=100 257.alb & 1 & 1 & Solution & 30.05 & 12 & 12.00 &  0.00\\
instance n=100 258.alb & 1 & 1 & Solution & 30.07 & 14 & 14.00 &  0.00\\
instance n=100 259.alb & 1 & 1 & Solution & 30.04 & 15 & 15.00 &  0.00\\
instance n=100 26.alb & 1 & 1 & Solution & 30.11 & 14 & 14.00 &  0.00\\
instance n=100 260.alb & 1 & 1 & Solution & 30.04 & 15 & 15.00 &  0.00\\
instance n=100 261.alb & 1 & 1 & Solution & 30.06 & 14 & 14.00 &  0.00\\
instance n=100 262.alb & 1 & 1 & Solution & 30.03 & 14 & 14.00 &  0.00\\
instance n=100 263.alb & 1 & 1 & Solution & 30.05 & 14 & 14.00 &  0.00\\
instance n=100 264.alb & 1 & 1 & Solution & 30.05 & 15 & 15.00 &  0.00\\
instance n=100 265.alb & 1 & 1 & Solution & 30.05 & 14 & 14.00 &  0.00\\
instance n=100 266.alb & 1 & 1 & Solution & 30.05 & 13 & 13.00 &  0.00\\
instance n=100 267.alb & 1 & 1 & Solution & 30.05 & 13 & 13.00 &  0.00\\
instance n=100 268.alb & 1 & 1 & Solution & 30.04 & 15 & 15.00 &  0.00\\
instance n=100 269.alb & 1 & 1 & Solution & 30.04 & 15 & 15.00 &  0.00\\
instance n=100 27.alb & 1 & 1 & Solution & 30.04 & 13 & 13.00 &  0.00\\
instance n=100 270.alb & 1 & 1 & Solution & 30.05 & 13 & 13.00 &  0.00\\
instance n=100 271.alb & 1 & 1 & Solution & 30.05 & 14 & 13.00 &  7.14\\
instance n=100 272.alb & 1 & 1 & Solution & 30.06 & 14 & 14.00 &  0.00\\
instance n=100 273.alb & 1 & 1 & Solution & 30.05 & 13 & 13.00 &  0.00\\
instance n=100 274.alb & 1 & 1 & Solution & 30.04 & 13 & 13.00 &  0.00\\
instance n=100 275.alb & 1 & 1 & Solution & 30.04 & 13 & 13.00 &  0.00\\
instance n=100 276.alb & 1 & 1 & Solution & 30.05 & 62 & 52.00 & 16.13\\
instance n=100 277.alb & 1 & 1 & Solution & 30.04 & 61 & 51.00 & 16.39\\
instance n=100 278.alb & 1 & 1 & Solution & 30.05 & 59 & 52.00 & 11.86\\
instance n=100 279.alb & 1 & 1 & Solution & 30.03 & 56 & 51.00 &  8.93\\
instance n=100 28.alb & 1 & 1 & Solution & 30.04 & 14 & 14.00 &  0.00\\
instance n=100 280.alb & 1 & 1 & Solution & 30.04 & 57 & 50.00 & 12.28\\
instance n=100 281.alb & 1 & 1 & Solution & 30.04 & 64 & 53.00 & 17.19\\
instance n=100 282.alb & 1 & 1 & Solution & 30.04 & 63 & 52.00 & 17.46\\
instance n=100 283.alb & 1 & 1 & Solution & 30.04 & 57 & 50.00 & 12.28\\
instance n=100 284.alb & 1 & 1 & Solution & 30.04 & 57 & 51.00 & 10.53\\
instance n=100 285.alb & 1 & 1 & Solution & 30.03 & 57 & 50.00 & 12.28\\
instance n=100 286.alb & 1 & 1 & Solution & 30.04 & 59 & 51.00 & 13.56\\
instance n=100 287.alb & 1 & 1 & Solution & 30.04 & 56 & 50.00 & 10.71\\
instance n=100 288.alb & 1 & 1 & Solution & 30.02 & 59 & 51.00 & 13.56\\
instance n=100 289.alb & 1 & 1 & Solution & 30.04 & 63 & 52.00 & 17.46\\
instance n=100 29.alb & 1 & 1 & Solution & 30.06 & 14 & 14.00 &  0.00\\
instance n=100 290.alb & 1 & 1 & Solution & 30.06 & 57 & 51.00 & 10.53\\
instance n=100 291.alb & 1 & 1 & Solution & 30.04 & 54 & 49.00 &  9.26\\
instance n=100 292.alb & 1 & 1 & Solution & 30.04 & 60 & 51.00 & 15.00\\
instance n=100 293.alb & 1 & 1 & Solution & 30.05 & 55 & 49.00 & 10.91\\
instance n=100 294.alb & 1 & 1 & Solution & 30.04 & 59 & 51.00 & 13.56\\
instance n=100 295.alb & 1 & 1 & Solution & 30.03 & 59 & 51.00 & 13.56\\
instance n=100 296.alb & 1 & 1 & Solution & 30.04 & 58 & 50.00 & 13.79\\
instance n=100 297.alb & 1 & 1 & Solution & 30.04 & 61 & 51.00 & 16.39\\
instance n=100 298.alb & 1 & 1 & Solution & 30.06 & 62 & 52.00 & 16.13\\
instance n=100 299.alb & 1 & 1 & Solution & 30.03 & 57 & 50.00 & 12.28\\
instance n=100 3.alb & 1 & 1 & Solution & 30.05 & 20 & 20.00 &  0.00\\
instance n=100 30.alb & 1 & 1 & Solution & 30.11 & 15 & 15.00 &  0.00\\
instance n=100 300.alb & 1 & 1 & Solution & 30.03 & 56 & 49.00 & 12.50\\
instance n=100 301.alb & 1 & 1 & Solution & 30.07 & 23 & 23.00 &  0.00\\
instance n=100 302.alb & 1 & 1 & Solution & 30.04 & 24 & 24.00 &  0.00\\
instance n=100 303.alb & 1 & 1 & Solution & 30.09 & 24 & 24.00 &  0.00\\
instance n=100 304.alb & 1 & 1 & Solution & 30.03 & 21 & 21.00 &  0.00\\
instance n=100 305.alb & 1 & 1 & Solution & 30.05 & 22 & 22.00 &  0.00\\
instance n=100 306.alb & 1 & 1 & Solution & 30.05 & 24 & 24.00 &  0.00\\
instance n=100 307.alb & 1 & 1 & Solution & 30.05 & 24 & 23.00 &  4.17\\
instance n=100 308.alb & 1 & 1 & Solution & 30.03 & 21 & 20.00 &  4.76\\
instance n=100 309.alb & 1 & 1 & Solution & 30.06 & 22 & 21.00 &  4.55\\
instance n=100 31.alb & 1 & 1 & Solution & 30.05 & 14 & 14.00 &  0.00\\
instance n=100 310.alb & 1 & 1 & Solution & 30.04 & 23 & 23.00 &  0.00\\
instance n=100 311.alb & 1 & 1 & Solution & 30.04 & 21 & 21.00 &  0.00\\
instance n=100 312.alb & 1 & 1 & Solution & 30.07 & 22 & 22.00 &  0.00\\
instance n=100 313.alb & 1 & 1 & Solution & 30.05 & 23 & 23.00 &  0.00\\
instance n=100 314.alb & 1 & 1 & Solution & 30.05 & 19 & 19.00 &  0.00\\
instance n=100 315.alb & 1 & 1 & Solution & 30.07 & 23 & 22.00 &  4.35\\
instance n=100 316.alb & 1 & 1 & Solution & 30.02 & 24 & 24.00 &  0.00\\
instance n=100 317.alb & 1 & 1 & Solution & 30.04 & 26 & 26.00 &  0.00\\
instance n=100 318.alb & 1 & 1 & Solution & 30.06 & 21 & 21.00 &  0.00\\
instance n=100 319.alb & 1 & 1 & Solution & 30.05 & 23 & 23.00 &  0.00\\
instance n=100 32.alb & 1 & 1 & Solution & 30.06 & 14 & 14.00 &  0.00\\
instance n=100 320.alb & 1 & 1 & Solution & 30.04 & 22 & 22.00 &  0.00\\
instance n=100 321.alb & 1 & 1 & Solution & 30.06 & 26 & 26.00 &  0.00\\
instance n=100 322.alb & 1 & 1 & Solution & 30.14 & 24 & 23.00 &  4.17\\
instance n=100 323.alb & 1 & 1 & Solution & 30.07 & 24 & 24.00 &  0.00\\
instance n=100 324.alb & 1 & 1 & Solution & 30.05 & 23 & 23.00 &  0.00\\
instance n=100 325.alb & 1 & 1 & Solution & 30.06 & 26 & 25.00 &  3.85\\
instance n=100 326.alb & 1 & 1 & Solution & 30.06 & 13 & 13.00 &  0.00\\
instance n=100 327.alb & 1 & 1 & Solution & 30.03 & 14 & 14.00 &  0.00\\
instance n=100 328.alb & 1 & 1 & Solution & 30.05 & 15 & 14.00 &  6.67\\
instance n=100 329.alb & 1 & 1 & Solution & 30.06 & 14 & 14.00 &  0.00\\
instance n=100 33.alb & 1 & 1 & Solution & 30.04 & 15 & 15.00 &  0.00\\
instance n=100 330.alb & 1 & 1 & Solution & 30.05 & 15 & 14.00 &  6.67\\
instance n=100 331.alb & 1 & 1 & Solution & 30.05 & 14 & 14.00 &  0.00\\
instance n=100 332.alb & 1 & 1 & Solution & 30.07 & 14 & 14.00 &  0.00\\
instance n=100 333.alb & 1 & 1 & Solution & 30.06 & 15 & 15.00 &  0.00\\
instance n=100 334.alb & 1 & 1 & Solution & 30.06 & 14 & 14.00 &  0.00\\
instance n=100 335.alb & 1 & 1 & Solution & 30.03 & 13 & 13.00 &  0.00\\
instance n=100 336.alb & 1 & 1 & Solution & 30.05 & 15 & 15.00 &  0.00\\
instance n=100 337.alb & 1 & 1 & Solution & 30.07 & 13 & 13.00 &  0.00\\
instance n=100 338.alb & 1 & 1 & Solution & 30.07 & 15 & 14.00 &  6.67\\
instance n=100 339.alb & 1 & 1 & Solution & 30.08 & 14 & 14.00 &  0.00\\
instance n=100 34.alb & 1 & 1 & Solution & 30.07 & 15 & 15.00 &  0.00\\
instance n=100 340.alb & 1 & 1 & Solution & 30.12 & 14 & 14.00 &  0.00\\
instance n=100 341.alb & 1 & 1 & Solution & 30.06 & 16 & 16.00 &  0.00\\
instance n=100 342.alb & 1 & 1 & Solution & 30.04 & 14 & 14.00 &  0.00\\
instance n=100 343.alb & 1 & 1 & Solution & 30.06 & 16 & 16.00 &  0.00\\
instance n=100 344.alb & 1 & 1 & Solution & 30.07 & 15 & 15.00 &  0.00\\
instance n=100 345.alb & 1 & 1 & Solution & 30.07 & 14 & 14.00 &  0.00\\
instance n=100 346.alb & 1 & 1 & Solution & 30.06 & 14 & 14.00 &  0.00\\
instance n=100 347.alb & 1 & 1 & Solution & 30.05 & 14 & 14.00 &  0.00\\
instance n=100 348.alb & 1 & 1 & Solution & 30.08 & 14 & 14.00 &  0.00\\
instance n=100 349.alb & 1 & 1 & Solution & 30.05 & 13 & 13.00 &  0.00\\
instance n=100 35.alb & 1 & 1 & Solution & 30.05 & 15 & 15.00 &  0.00\\
instance n=100 350.alb & 1 & 1 & Solution & 30.05 & 14 & 14.00 &  0.00\\
instance n=100 351.alb & 1 & 1 & Solution & 30.06 & 60 & 52.00 & 13.33\\
instance n=100 352.alb & 1 & 1 & Solution & 30.06 & 64 & 52.00 & 18.75\\
instance n=100 353.alb & 1 & 1 & Solution & 30.05 & 53 & 49.00 &  7.55\\
instance n=100 354.alb & 1 & 1 & Solution & 30.03 & 53 & 49.00 &  7.55\\
instance n=100 355.alb & 1 & 1 & Solution & 30.05 & 56 & 51.00 &  8.93\\
instance n=100 356.alb & 1 & 1 & Solution & 30.05 & 61 & 53.00 & 13.11\\
instance n=100 357.alb & 1 & 1 & Solution & 30.06 & 54 & 50.00 &  7.41\\
instance n=100 358.alb & 1 & 1 & Solution & 30.05 & 53 & 50.00 &  5.66\\
instance n=100 359.alb & 1 & 1 & Solution & 30.05 & 54 & 50.00 &  7.41\\
instance n=100 36.alb & 1 & 1 & Solution & 30.04 & 15 & 14.00 &  6.67\\
instance n=100 360.alb & 1 & 1 & Solution & 30.05 & 57 & 51.00 & 10.53\\
instance n=100 361.alb & 1 & 1 & Solution & 30.04 & 52 & 49.00 &  5.77\\
instance n=100 362.alb & 1 & 1 & Solution & 30.06 & 59 & 51.00 & 13.56\\
instance n=100 363.alb & 1 & 1 & Solution & 30.04 & 53 & 50.00 &  5.66\\
instance n=100 364.alb & 1 & 1 & Solution & 30.06 & 54 & 50.00 &  7.41\\
instance n=100 365.alb & 1 & 1 & Solution & 30.05 & 56 & 50.00 & 10.71\\
instance n=100 366.alb & 1 & 1 & Solution & 30.03 & 62 & 53.00 & 14.52\\
instance n=100 367.alb & 1 & 1 & Solution & 30.03 & 58 & 51.00 & 12.07\\
instance n=100 368.alb & 1 & 1 & Solution & 30.04 & 61 & 52.00 & 14.75\\
instance n=100 369.alb & 1 & 1 & Solution & 30.04 & 53 & 49.00 &  7.55\\
instance n=100 37.alb & 1 & 1 & Solution & 30.06 & 14 & 14.00 &  0.00\\
instance n=100 370.alb & 1 & 1 & Solution & 30.05 & 59 & 52.00 & 11.86\\
instance n=100 371.alb & 1 & 1 & Solution & 30.04 & 55 & 50.00 &  9.09\\
instance n=100 372.alb & 1 & 1 & Solution & 30.04 & 50 & 47.00 &  6.00\\
instance n=100 373.alb & 1 & 1 & Solution & 30.05 & 52 & 49.00 &  5.77\\
instance n=100 374.alb & 1 & 1 & Solution & 30.05 & 53 & 50.00 &  5.66\\
instance n=100 375.alb & 1 & 1 & Solution & 30.03 & 61 & 52.00 & 14.75\\
instance n=100 376.alb & 1 & 1 & Solution & 30.04 & 23 & 23.00 &  0.00\\
instance n=100 377.alb & 1 & 1 & Solution & 30.05 & 21 & 20.00 &  4.76\\
instance n=100 378.alb & 1 & 1 & Solution & 30.03 & 22 & 22.00 &  0.00\\
instance n=100 379.alb & 1 & 1 & Solution & 30.05 & 24 & 23.00 &  4.17\\
instance n=100 38.alb & 1 & 1 & Solution & 30.04 & 14 & 14.00 &  0.00\\
instance n=100 380.alb & 1 & 1 & Solution & 30.04 & 23 & 22.00 &  4.35\\
instance n=100 381.alb & 1 & 1 & Solution & 30.04 & 24 & 24.00 &  0.00\\
instance n=100 382.alb & 1 & 1 & Solution & 30.03 & 25 & 25.00 &  0.00\\
instance n=100 383.alb & 1 & 1 & Solution & 30.03 & 25 & 25.00 &  0.00\\
instance n=100 384.alb & 1 & 1 & Solution & 30.05 & 25 & 25.00 &  0.00\\
instance n=100 385.alb & 1 & 1 & Solution & 30.04 & 22 & 22.00 &  0.00\\
instance n=100 386.alb & 1 & 1 & Solution & 30.05 & 24 & 23.00 &  4.17\\
instance n=100 387.alb & 1 & 1 & Solution & 30.04 & 22 & 22.00 &  0.00\\
instance n=100 388.alb & 1 & 1 & Solution & 30.04 & 26 & 25.00 &  3.85\\
instance n=100 389.alb & 1 & 1 & Solution & 30.07 & 23 & 23.00 &  0.00\\
instance n=100 39.alb & 1 & 1 & Solution & 30.05 & 14 & 14.00 &  0.00\\
instance n=100 390.alb & 1 & 1 & Solution & 30.06 & 23 & 22.00 &  4.35\\
instance n=100 391.alb & 1 & 1 & Solution & 30.04 & 20 & 20.00 &  0.00\\
instance n=100 392.alb & 1 & 1 & Solution & 30.05 & 22 & 22.00 &  0.00\\
instance n=100 393.alb & 1 & 1 & Solution & 30.04 & 24 & 23.00 &  4.17\\
instance n=100 394.alb & 1 & 1 & Solution & 30.05 & 22 & 22.00 &  0.00\\
instance n=100 395.alb & 1 & 1 & Solution & 30.05 & 24 & 24.00 &  0.00\\
instance n=100 396.alb & 1 & 1 & Solution & 30.05 & 20 & 20.00 &  0.00\\
instance n=100 397.alb & 1 & 1 & Solution & 30.05 & 26 & 25.00 &  3.85\\
instance n=100 398.alb & 1 & 1 & Solution & 30.04 & 25 & 24.00 &  4.00\\
instance n=100 399.alb & 1 & 1 & Solution & 30.04 & 23 & 23.00 &  0.00\\
instance n=100 4.alb & 1 & 1 & Solution & 30.09 & 24 & 24.00 &  0.00\\
instance n=100 40.alb & 1 & 1 & Solution & 30.05 & 14 & 14.00 &  0.00\\
instance n=100 400.alb & 1 & 1 & Solution & 30.04 & 24 & 24.00 &  0.00\\
instance n=100 401.alb & 1 & 1 & Solution & 30.05 & 15 & 15.00 &  0.00\\
instance n=100 402.alb & 1 & 1 & Solution & 30.04 & 15 & 15.00 &  0.00\\
instance n=100 403.alb & 1 & 1 & Solution & 30.03 & 14 & 14.00 &  0.00\\
instance n=100 404.alb & 1 & 1 & Solution & 30.06 & 15 & 15.00 &  0.00\\
instance n=100 405.alb & 1 & 1 & Solution & 30.05 & 13 & 13.00 &  0.00\\
instance n=100 406.alb & 1 & 1 & Solution & 30.04 & 14 & 14.00 &  0.00\\
instance n=100 407.alb & 1 & 1 & Solution & 30.04 & 15 & 15.00 &  0.00\\
instance n=100 408.alb & 1 & 1 & Solution & 30.04 & 14 & 14.00 &  0.00\\
instance n=100 409.alb & 1 & 1 & Solution & 30.04 & 15 & 15.00 &  0.00\\
instance n=100 41.alb & 1 & 1 & Solution & 30.06 & 13 & 13.00 &  0.00\\
instance n=100 410.alb & 1 & 1 & Solution & 30.04 & 14 & 14.00 &  0.00\\
instance n=100 411.alb & 1 & 1 & Solution & 30.04 & 14 & 14.00 &  0.00\\
instance n=100 412.alb & 1 & 1 & Solution & 30.04 & 14 & 14.00 &  0.00\\
instance n=100 413.alb & 1 & 1 & Solution & 30.04 & 14 & 14.00 &  0.00\\
instance n=100 414.alb & 1 & 1 & Solution & 30.04 & 15 & 14.00 &  6.67\\
instance n=100 415.alb & 1 & 1 & Solution & 30.03 & 13 & 13.00 &  0.00\\
instance n=100 416.alb & 1 & 1 & Solution & 30.05 & 14 & 14.00 &  0.00\\
instance n=100 417.alb & 1 & 1 & Solution & 30.03 & 15 & 15.00 &  0.00\\
instance n=100 418.alb & 1 & 1 & Solution & 30.05 & 16 & 16.00 &  0.00\\
instance n=100 419.alb & 1 & 1 & Solution & 30.05 & 14 & 14.00 &  0.00\\
instance n=100 42.alb & 1 & 1 & Solution & 30.05 & 14 & 14.00 &  0.00\\
instance n=100 420.alb & 1 & 1 & Solution & 30.04 & 14 & 14.00 &  0.00\\
instance n=100 421.alb & 1 & 1 & Solution & 30.04 & 14 & 14.00 &  0.00\\
instance n=100 422.alb & 1 & 1 & Solution & 30.03 & 15 & 15.00 &  0.00\\
instance n=100 423.alb & 1 & 1 & Solution & 30.05 & 14 & 14.00 &  0.00\\
instance n=100 424.alb & 1 & 1 & Solution & 30.03 & 14 & 14.00 &  0.00\\
instance n=100 425.alb & 1 & 1 & Solution & 30.04 & 15 & 15.00 &  0.00\\
instance n=100 426.alb & 1 & 1 & Solution & 30.04 & 62 & 53.00 & 14.52\\
instance n=100 427.alb & 1 & 1 & Solution & 30.04 & 57 & 50.00 & 12.28\\
instance n=100 428.alb & 1 & 1 & Solution & 30.04 & 56 & 50.00 & 10.71\\
instance n=100 429.alb & 1 & 1 & Solution & 30.04 & 60 & 52.00 & 13.33\\
instance n=100 43.alb & 1 & 1 & Solution & 30.07 & 14 & 14.00 &  0.00\\
instance n=100 430.alb & 1 & 1 & Solution & 30.05 & 57 & 50.00 & 12.28\\
instance n=100 431.alb & 1 & 1 & Solution & 30.04 & 54 & 50.00 &  7.41\\
instance n=100 432.alb & 1 & 1 & Solution & 30.06 & 58 & 51.00 & 12.07\\
instance n=100 433.alb & 1 & 1 & Solution & 30.04 & 54 & 49.00 &  9.26\\
instance n=100 434.alb & 1 & 1 & Solution & 30.04 & 59 & 51.00 & 13.56\\
instance n=100 435.alb & 1 & 1 & Solution & 30.04 & 58 & 51.00 & 12.07\\
instance n=100 436.alb & 1 & 1 & Solution & 30.04 & 53 & 49.00 &  7.55\\
instance n=100 437.alb & 1 & 1 & Solution & 30.03 & 54 & 50.00 &  7.41\\
instance n=100 438.alb & 1 & 1 & Solution & 30.06 & 57 & 51.00 & 10.53\\
instance n=100 439.alb & 1 & 1 & Solution & 30.03 & 57 & 51.00 & 10.53\\
instance n=100 44.alb & 1 & 1 & Solution & 30.05 & 14 & 14.00 &  0.00\\
instance n=100 440.alb & 1 & 1 & Solution & 30.03 & 55 & 49.00 & 10.91\\
instance n=100 441.alb & 1 & 1 & Solution & 30.03 & 54 & 50.00 &  7.41\\
instance n=100 442.alb & 1 & 1 & Solution & 30.03 & 55 & 48.00 & 12.73\\
instance n=100 443.alb & 1 & 1 & Solution & 30.04 & 57 & 50.00 & 12.28\\
instance n=100 444.alb & 1 & 1 & Solution & 30.06 & 54 & 50.00 &  7.41\\
instance n=100 445.alb & 1 & 1 & Solution & 30.07 & 57 & 51.00 & 10.53\\
instance n=100 446.alb & 1 & 1 & Solution & 30.02 & 59 & 52.00 & 11.86\\
instance n=100 447.alb & 1 & 1 & Solution & 30.06 & 56 & 50.00 & 10.71\\
instance n=100 448.alb & 1 & 1 & Solution & 30.06 & 57 & 51.00 & 10.53\\
instance n=100 449.alb & 1 & 1 & Solution & 30.06 & 56 & 50.00 & 10.71\\
instance n=100 45.alb & 1 & 1 & Solution & 30.06 & 14 & 14.00 &  0.00\\
instance n=100 450.alb & 1 & 1 & Solution & 30.03 & 55 & 51.00 &  7.27\\
instance n=100 451.alb & 1 & 1 & Solution & 30.04 & 26 & 26.00 &  0.00\\
instance n=100 452.alb & 1 & 1 & Solution & 30.03 & 22 & 22.00 &  0.00\\
instance n=100 453.alb & 1 & 1 & Solution & 30.03 & 24 & 24.00 &  0.00\\
instance n=100 454.alb & 1 & 1 & Solution & 30.04 & 23 & 23.00 &  0.00\\
instance n=100 455.alb & 1 & 1 & Solution & 30.03 & 23 & 23.00 &  0.00\\
instance n=100 456.alb & 1 & 1 & Solution & 30.02 & 26 & 26.00 &  0.00\\
instance n=100 457.alb & 1 & 1 & Solution & 30.04 & 23 & 23.00 &  0.00\\
instance n=100 458.alb & 1 & 1 & Solution & 30.04 & 24 & 24.00 &  0.00\\
instance n=100 459.alb & 1 & 1 & Solution & 30.03 & 23 & 23.00 &  0.00\\
instance n=100 46.alb & 1 & 1 & Solution & 30.07 & 14 & 14.00 &  0.00\\
instance n=100 460.alb & 1 & 1 & Solution & 30.02 & 23 & 23.00 &  0.00\\
instance n=100 461.alb & 1 & 1 & Solution & 30.03 & 23 & 23.00 &  0.00\\
instance n=100 462.alb & 1 & 1 & Solution & 30.03 & 23 & 23.00 &  0.00\\
instance n=100 463.alb & 1 & 1 & Solution & 30.04 & 26 & 26.00 &  0.00\\
instance n=100 464.alb & 1 & 1 & Solution & 30.04 & 25 & 25.00 &  0.00\\
instance n=100 465.alb & 1 & 1 & Solution & 30.02 & 22 & 22.00 &  0.00\\
instance n=100 466.alb & 1 & 1 & Solution & 30.03 & 26 & 26.00 &  0.00\\
instance n=100 467.alb & 1 & 1 & Solution & 30.03 & 21 & 20.00 &  4.76\\
instance n=100 468.alb & 1 & 1 & Solution & 30.05 & 25 & 25.00 &  0.00\\
instance n=100 469.alb & 1 & 1 & Solution & 30.02 & 22 & 22.00 &  0.00\\
instance n=100 47.alb & 1 & 1 & Solution & 30.05 & 14 & 14.00 &  0.00\\
instance n=100 470.alb & 1 & 1 & Solution & 30.03 & 26 & 25.00 &  3.85\\
instance n=100 471.alb & 1 & 1 & Solution & 30.03 & 26 & 25.00 &  3.85\\
instance n=100 472.alb & 1 & 1 & Solution & 30.03 & 23 & 23.00 &  0.00\\
instance n=100 473.alb & 1 & 1 & Solution & 30.04 & 28 & 28.00 &  0.00\\
instance n=100 474.alb & 1 & 1 & Solution & 30.05 & 23 & 23.00 &  0.00\\
instance n=100 475.alb & 1 & 1 & Solution & 30.03 & 24 & 23.00 &  4.17\\
instance n=100 476.alb & 1 & 1 & Solution & 30.03 & 14 & 14.00 &  0.00\\
instance n=100 477.alb & 1 & 1 & Solution & 30.04 & 14 & 14.00 &  0.00\\
instance n=100 478.alb & 1 & 1 & Solution & 30.05 & 14 & 14.00 &  0.00\\
instance n=100 479.alb & 1 & 1 & Optimal & 20.30 & 16 & 16.00 &  0.00\\
instance n=100 48.alb & 1 & 1 & Solution & 30.06 & 15 & 15.00 &  0.00\\
instance n=100 480.alb & 1 & 1 & Optimal & 15.06 & 15 & 15.00 &  0.00\\
instance n=100 481.alb & 1 & 1 & Solution & 30.03 & 15 & 15.00 &  0.00\\
instance n=100 482.alb & 1 & 1 & Optimal & 30.01 & 15 & 15.00 &  0.00\\
instance n=100 483.alb & 1 & 1 & Solution & 30.04 & 14 & 14.00 &  0.00\\
instance n=100 484.alb & 1 & 1 & Optimal & 30.04 & 14 & 14.00 &  0.00\\
instance n=100 485.alb & 1 & 1 & Optimal & 30.01 & 16 & 16.00 &  0.00\\
instance n=100 486.alb & 1 & 1 & Solution & 30.04 & 15 & 15.00 &  0.00\\
instance n=100 487.alb & 1 & 1 & Optimal & 30.01 & 15 & 15.00 &  0.00\\
instance n=100 488.alb & 1 & 1 & Solution & 30.04 & 16 & 16.00 &  0.00\\
instance n=100 489.alb & 1 & 1 & Solution & 30.04 & 13 & 13.00 &  0.00\\
instance n=100 49.alb & 1 & 1 & Solution & 30.07 & 14 & 14.00 &  0.00\\
instance n=100 490.alb & 1 & 1 & Optimal & 30.01 & 15 & 15.00 &  0.00\\
instance n=100 491.alb & 1 & 1 & Optimal & 30.02 & 16 & 16.00 &  0.00\\
instance n=100 492.alb & 1 & 1 & Optimal & 30.03 & 14 & 14.00 &  0.00\\
instance n=100 493.alb & 1 & 1 & Solution & 30.03 & 14 & 14.00 &  0.00\\
instance n=100 494.alb & 1 & 1 & Solution & 30.04 & 14 & 14.00 &  0.00\\
instance n=100 495.alb & 1 & 1 & Solution & 30.03 & 15 & 15.00 &  0.00\\
instance n=100 496.alb & 1 & 1 & Solution & 30.03 & 14 & 14.00 &  0.00\\
instance n=100 497.alb & 1 & 1 & Solution & 30.05 & 13 & 13.00 &  0.00\\
instance n=100 498.alb & 1 & 1 & Solution & 30.05 & 14 & 14.00 &  0.00\\
instance n=100 499.alb & 1 & 1 & Solution & 30.03 & 14 & 14.00 &  0.00\\
instance n=100 5.alb & 1 & 1 & Solution & 30.04 & 22 & 22.00 &  0.00\\
instance n=100 50.alb & 1 & 1 & Solution & 30.03 & 14 & 14.00 &  0.00\\
instance n=100 500.alb & 1 & 1 & Solution & 30.03 & 14 & 14.00 &  0.00\\
instance n=100 501.alb & 1 & 1 & Solution & 30.02 & 63 & 56.00 & 11.11\\
instance n=100 502.alb & 1 & 1 & Solution & 30.04 & 66 & 55.00 & 16.67\\
instance n=100 503.alb & 1 & 1 & Solution & 30.02 & 62 & 53.00 & 14.52\\
instance n=100 504.alb & 1 & 1 & Solution & 30.03 & 60 & 52.00 & 13.33\\
instance n=100 505.alb & 1 & 1 & Solution & 30.03 & 62 & 52.00 & 16.13\\
instance n=100 506.alb & 1 & 1 & Solution & 30.02 & 61 & 52.00 & 14.75\\
instance n=100 507.alb & 1 & 1 & Solution & 30.04 & 60 & 53.00 & 11.67\\
instance n=100 508.alb & 1 & 1 & Solution & 30.04 & 56 & 53.00 &  5.36\\
instance n=100 509.alb & 1 & 1 & Solution & 30.03 & 59 & 52.00 & 11.86\\
instance n=100 51.alb & 1 & 1 & Solution & 30.05 & 51 & 48.00 &  5.88\\
instance n=100 510.alb & 1 & 1 & Solution & 30.02 & 58 & 53.00 &  8.62\\
instance n=100 511.alb & 1 & 1 & Solution & 30.03 & 60 & 53.00 & 11.67\\
instance n=100 512.alb & 1 & 1 & Solution & 30.02 & 62 & 54.00 & 12.90\\
instance n=100 513.alb & 1 & 1 & Solution & 30.04 & 63 & 51.00 & 19.05\\
instance n=100 514.alb & 1 & 1 & Solution & 30.04 & 59 & 52.00 & 11.86\\
instance n=100 515.alb & 1 & 1 & Solution & 30.03 & 63 & 53.00 & 15.87\\
instance n=100 516.alb & 1 & 1 & Solution & 30.03 & 70 & 56.00 & 20.00\\
instance n=100 517.alb & 1 & 1 & Solution & 30.04 & 62 & 54.00 & 12.90\\
instance n=100 518.alb & 1 & 1 & Solution & 30.02 & 59 & 51.00 & 13.56\\
instance n=100 519.alb & 1 & 1 & Solution & 30.04 & 63 & 54.00 & 14.29\\
instance n=100 52.alb & 1 & 1 & Solution & 30.05 & 54 & 50.00 &  7.41\\
instance n=100 520.alb & 1 & 1 & Solution & 30.03 & 61 & 53.00 & 13.11\\
instance n=100 521.alb & 1 & 1 & Solution & 30.04 & 70 & 57.00 & 18.57\\
instance n=100 522.alb & 1 & 1 & Solution & 30.03 & 60 & 51.00 & 15.00\\
instance n=100 523.alb & 1 & 1 & Solution & 30.03 & 56 & 50.00 & 10.71\\
instance n=100 524.alb & 1 & 1 & Solution & 30.03 & 59 & 52.00 & 11.86\\
instance n=100 525.alb & 1 & 1 & Solution & 30.04 & 62 & 52.00 & 16.13\\
instance n=100 53.alb & 1 & 1 & Solution & 30.05 & 52 & 50.00 &  3.85\\
instance n=100 54.alb & 1 & 1 & Solution & 30.05 & 52 & 49.00 &  5.77\\
instance n=100 55.alb & 1 & 1 & Solution & 30.05 & 56 & 50.00 & 10.71\\
instance n=100 56.alb & 1 & 1 & Solution & 30.06 & 53 & 50.00 &  5.66\\
instance n=100 57.alb & 1 & 1 & Solution & 30.04 & 58 & 51.00 & 12.07\\
instance n=100 58.alb & 1 & 1 & Solution & 30.05 & 58 & 52.00 & 10.34\\
instance n=100 59.alb & 1 & 1 & Solution & 30.05 & 58 & 51.00 & 12.07\\
instance n=100 6.alb & 1 & 1 & Solution & 30.05 & 22 & 22.00 &  0.00\\
instance n=100 60.alb & 1 & 1 & Solution & 30.04 & 55 & 51.00 &  7.27\\
instance n=100 61.alb & 1 & 1 & Solution & 30.04 & 57 & 51.00 & 10.53\\
instance n=100 62.alb & 1 & 1 & Solution & 30.07 & 54 & 49.00 &  9.26\\
instance n=100 63.alb & 1 & 1 & Solution & 30.05 & 61 & 52.00 & 14.75\\
instance n=100 64.alb & 1 & 1 & Solution & 30.06 & 57 & 51.00 & 10.53\\
instance n=100 65.alb & 1 & 1 & Solution & 30.06 & 62 & 53.00 & 14.52\\
instance n=100 66.alb & 1 & 1 & Solution & 30.05 & 52 & 49.00 &  5.77\\
instance n=100 67.alb & 1 & 1 & Solution & 30.07 & 56 & 51.00 &  8.93\\
instance n=100 68.alb & 1 & 1 & Solution & 30.04 & 57 & 49.00 & 14.04\\
instance n=100 69.alb & 1 & 1 & Solution & 30.05 & 55 & 51.00 &  7.27\\
instance n=100 7.alb & 1 & 1 & Solution & 30.04 & 26 & 26.00 &  0.00\\
instance n=100 70.alb & 1 & 1 & Solution & 30.05 & 56 & 50.00 & 10.71\\
instance n=100 71.alb & 1 & 1 & Solution & 30.07 & 54 & 50.00 &  7.41\\
instance n=100 72.alb & 1 & 1 & Solution & 30.04 & 56 & 50.00 & 10.71\\
instance n=100 73.alb & 1 & 1 & Solution & 30.06 & 58 & 52.00 & 10.34\\
instance n=100 74.alb & 1 & 1 & Solution & 30.04 & 52 & 49.00 &  5.77\\
instance n=100 75.alb & 1 & 1 & Solution & 30.04 & 56 & 51.00 &  8.93\\
instance n=100 76.alb & 1 & 1 & Solution & 30.04 & 23 & 23.00 &  0.00\\
instance n=100 77.alb & 1 & 1 & Solution & 30.05 & 20 & 20.00 &  0.00\\
instance n=100 78.alb & 1 & 1 & Solution & 30.05 & 21 & 21.00 &  0.00\\
instance n=100 79.alb & 1 & 1 & Solution & 30.05 & 21 & 21.00 &  0.00\\
instance n=100 8.alb & 1 & 1 & Solution & 30.03 & 24 & 24.00 &  0.00\\
instance n=100 80.alb & 1 & 1 & Solution & 30.05 & 23 & 22.00 &  4.35\\
instance n=100 81.alb & 1 & 1 & Solution & 30.05 & 20 & 20.00 &  0.00\\
instance n=100 82.alb & 1 & 1 & Solution & 30.03 & 21 & 21.00 &  0.00\\
instance n=100 83.alb & 1 & 1 & Solution & 30.04 & 22 & 22.00 &  0.00\\
instance n=100 84.alb & 1 & 1 & Solution & 30.03 & 27 & 26.00 &  3.70\\
instance n=100 85.alb & 1 & 1 & Solution & 30.05 & 25 & 24.00 &  4.00\\
instance n=100 86.alb & 1 & 1 & Solution & 30.05 & 23 & 23.00 &  0.00\\
instance n=100 87.alb & 1 & 1 & Solution & 30.04 & 22 & 22.00 &  0.00\\
instance n=100 88.alb & 1 & 1 & Solution & 30.06 & 24 & 23.00 &  4.17\\
instance n=100 89.alb & 1 & 1 & Solution & 30.06 & 24 & 24.00 &  0.00\\
instance n=100 9.alb & 1 & 1 & Solution & 30.04 & 24 & 23.00 &  4.17\\
instance n=100 90.alb & 1 & 1 & Solution & 30.06 & 21 & 20.00 &  4.76\\
instance n=100 91.alb & 1 & 1 & Solution & 30.05 & 25 & 25.00 &  0.00\\
instance n=100 92.alb & 1 & 1 & Solution & 30.05 & 24 & 24.00 &  0.00\\
instance n=100 93.alb & 1 & 1 & Solution & 30.03 & 28 & 27.00 &  3.57\\
instance n=100 94.alb & 1 & 1 & Solution & 30.03 & 22 & 22.00 &  0.00\\
instance n=100 95.alb & 1 & 1 & Solution & 30.04 & 21 & 21.00 &  0.00\\
instance n=100 96.alb & 1 & 1 & Solution & 30.06 & 21 & 21.00 &  0.00\\
instance n=100 97.alb & 1 & 1 & Solution & 30.06 & 22 & 22.00 &  0.00\\
instance n=100 98.alb & 1 & 1 & Solution & 30.03 & 22 & 22.00 &  0.00\\
instance n=100 99.alb & 1 & 1 & Solution & 30.05 & 22 & 22.00 &  0.00\\
instance n=20 1.alb & 1 & 1 & Optimal &  1.27 & 3 &  3.00 &  0.00\\
instance n=20 10.alb & 1 & 1 & Optimal &  0.15 & 3 &  3.00 &  0.00\\
instance n=20 100.alb & 1 & 1 & Optimal & 30.02 & 11 & 11.00 &  0.00\\
instance n=20 101.alb & 1 & 1 & Optimal & 30.02 & 13 & 13.00 &  0.00\\
instance n=20 102.alb & 1 & 1 & Optimal & 30.02 & 13 & 13.00 &  0.00\\
instance n=20 103.alb & 1 & 1 & Optimal & 30.02 & 12 & 12.00 &  0.00\\
instance n=20 104.alb & 1 & 1 & Optimal & 30.01 & 11 & 11.00 &  0.00\\
instance n=20 105.alb & 1 & 1 & Optimal & 30.01 & 12 & 12.00 &  0.00\\
instance n=20 106.alb & 1 & 1 & Optimal & 30.01 & 10 & 10.00 &  0.00\\
instance n=20 107.alb & 1 & 1 & Optimal & 30.01 & 14 & 14.00 &  0.00\\
instance n=20 108.alb & 1 & 1 & Optimal & 30.01 & 15 & 15.00 &  0.00\\
instance n=20 109.alb & 1 & 1 & Optimal & 30.01 & 12 & 12.00 &  0.00\\
instance n=20 11.alb & 1 & 1 & Optimal &  0.34 & 3 &  3.00 &  0.00\\
instance n=20 110.alb & 1 & 1 & Optimal &  6.20 & 11 & 11.00 &  0.00\\
instance n=20 111.alb & 1 & 1 & Optimal & 30.01 & 13 & 13.00 &  0.00\\
instance n=20 112.alb & 1 & 1 & Optimal & 30.01 & 11 & 11.00 &  0.00\\
instance n=20 113.alb & 1 & 1 & Optimal & 30.01 & 12 & 12.00 &  0.00\\
instance n=20 114.alb & 1 & 1 & Optimal & 30.01 & 13 & 13.00 &  0.00\\
instance n=20 115.alb & 1 & 1 & Optimal & 30.01 & 11 & 11.00 &  0.00\\
instance n=20 116.alb & 1 & 1 & Optimal &  0.31 & 5 &  5.00 &  0.00\\
instance n=20 117.alb & 1 & 1 & Optimal &  0.24 & 5 &  5.00 &  0.00\\
instance n=20 118.alb & 1 & 1 & Optimal &  0.23 & 5 &  5.00 &  0.00\\
instance n=20 119.alb & 1 & 1 & Optimal &  0.43 & 6 &  6.00 &  0.00\\
instance n=20 12.alb & 1 & 1 & Optimal & 30.01 & 3 &  3.00 &  0.00\\
instance n=20 120.alb & 1 & 1 & Optimal &  0.73 & 6 &  6.00 &  0.00\\
instance n=20 121.alb & 1 & 1 & Optimal &  2.01 & 5 &  5.00 &  0.00\\
instance n=20 122.alb & 1 & 1 & Optimal & 30.01 & 6 &  6.00 &  0.00\\
instance n=20 123.alb & 1 & 1 & Optimal &  0.88 & 5 &  5.00 &  0.00\\
instance n=20 124.alb & 1 & 1 & Optimal &  0.30 & 5 &  5.00 &  0.00\\
instance n=20 125.alb & 1 & 1 & Optimal &  0.12 & 5 &  5.00 &  0.00\\
instance n=20 126.alb & 1 & 1 & Optimal &  0.11 & 5 &  5.00 &  0.00\\
instance n=20 127.alb & 1 & 1 & Optimal & 30.01 & 4 &  4.00 &  0.00\\
instance n=20 128.alb & 1 & 1 & Optimal & 30.02 & 5 &  5.00 &  0.00\\
instance n=20 129.alb & 1 & 1 & Optimal & 30.01 & 5 &  5.00 &  0.00\\
instance n=20 13.alb & 1 & 1 & Optimal & 30.01 & 3 &  3.00 &  0.00\\
instance n=20 130.alb & 1 & 1 & Optimal & 30.01 & 6 &  6.00 &  0.00\\
instance n=20 131.alb & 1 & 1 & Optimal & 30.01 & 7 &  7.00 &  0.00\\
instance n=20 132.alb & 1 & 1 & Optimal & 30.01 & 4 &  4.00 &  0.00\\
instance n=20 133.alb & 1 & 1 & Optimal &  0.48 & 5 &  5.00 &  0.00\\
instance n=20 134.alb & 1 & 1 & Optimal &  4.45 & 6 &  6.00 &  0.00\\
instance n=20 135.alb & 1 & 1 & Optimal &  0.09 & 6 &  6.00 &  0.00\\
instance n=20 136.alb & 1 & 1 & Optimal & 21.37 & 6 &  6.00 &  0.00\\
instance n=20 137.alb & 1 & 1 & Optimal & 30.01 & 5 &  5.00 &  0.00\\
instance n=20 138.alb & 1 & 1 & Optimal &  0.51 & 5 &  5.00 &  0.00\\
instance n=20 139.alb & 1 & 1 & Optimal & 30.01 & 5 &  5.00 &  0.00\\
instance n=20 14.alb & 1 & 1 & Optimal & 30.02 & 3 &  3.00 &  0.00\\
instance n=20 140.alb & 1 & 1 & Optimal & 21.60 & 5 &  5.00 &  0.00\\
instance n=20 141.alb & 1 & 1 & Optimal &  0.98 & 3 &  3.00 &  0.00\\
instance n=20 142.alb & 1 & 1 & Optimal &  3.51 & 3 &  3.00 &  0.00\\
instance n=20 143.alb & 1 & 1 & Optimal &  3.23 & 3 &  3.00 &  0.00\\
instance n=20 144.alb & 1 & 1 & Optimal & 30.01 & 4 &  4.00 &  0.00\\
instance n=20 145.alb & 1 & 1 & Optimal &  3.10 & 3 &  3.00 &  0.00\\
instance n=20 146.alb & 1 & 1 & Optimal & 25.18 & 3 &  3.00 &  0.00\\
instance n=20 147.alb & 1 & 1 & Optimal & 30.01 & 3 &  3.00 &  0.00\\
instance n=20 148.alb & 1 & 1 & Optimal &  2.50 & 3 &  3.00 &  0.00\\
instance n=20 149.alb & 1 & 1 & Optimal &  3.15 & 3 &  3.00 &  0.00\\
instance n=20 15.alb & 1 & 1 & Optimal &  0.13 & 3 &  3.00 &  0.00\\
instance n=20 150.alb & 1 & 1 & Optimal & 30.01 & 3 &  3.00 &  0.00\\
instance n=20 151.alb & 1 & 1 & Optimal &  1.73 & 3 &  3.00 &  0.00\\
instance n=20 152.alb & 1 & 1 & Optimal &  0.23 & 3 &  3.00 &  0.00\\
instance n=20 153.alb & 1 & 1 & Optimal &  2.39 & 3 &  3.00 &  0.00\\
instance n=20 154.alb & 1 & 1 & Optimal &  0.54 & 3 &  3.00 &  0.00\\
instance n=20 155.alb & 1 & 1 & Optimal &  1.31 & 3 &  3.00 &  0.00\\
instance n=20 156.alb & 1 & 1 & Optimal &  0.24 & 3 &  3.00 &  0.00\\
instance n=20 157.alb & 1 & 1 & Optimal &  1.03 & 3 &  3.00 &  0.00\\
instance n=20 158.alb & 1 & 1 & Optimal &  0.59 & 3 &  3.00 &  0.00\\
instance n=20 159.alb & 1 & 1 & Optimal & 30.02 & 3 &  3.00 &  0.00\\
instance n=20 16.alb & 1 & 1 & Optimal & 30.04 & 12 & 12.00 &  0.00\\
instance n=20 160.alb & 1 & 1 & Optimal & 15.95 & 3 &  3.00 &  0.00\\
instance n=20 161.alb & 1 & 1 & Optimal & 30.02 & 3 &  3.00 &  0.00\\
instance n=20 162.alb & 1 & 1 & Optimal &  1.51 & 3 &  3.00 &  0.00\\
instance n=20 163.alb & 1 & 1 & Optimal &  0.29 & 3 &  3.00 &  0.00\\
instance n=20 164.alb & 1 & 1 & Optimal & 30.02 & 4 &  4.00 &  0.00\\
instance n=20 165.alb & 1 & 1 & Optimal &  0.89 & 3 &  3.00 &  0.00\\
instance n=20 166.alb & 1 & 1 & Solution & 30.06 & 12 & 11.00 &  8.33\\
instance n=20 167.alb & 1 & 1 & Optimal & 30.02 & 11 & 11.00 &  0.00\\
instance n=20 168.alb & 1 & 1 & Optimal & 30.02 & 10 & 10.00 &  0.00\\
instance n=20 169.alb & 1 & 1 & Optimal & 30.02 & 11 & 11.00 &  0.00\\
instance n=20 17.alb & 1 & 1 & Optimal & 30.01 & 10 & 10.00 &  0.00\\
instance n=20 170.alb & 1 & 1 & Solution & 30.07 & 11 & 11.00 &  0.00\\
instance n=20 171.alb & 1 & 1 & Solution & 30.06 & 13 & 11.00 & 15.38\\
instance n=20 172.alb & 1 & 1 & Optimal & 30.02 & 11 & 11.00 &  0.00\\
instance n=20 173.alb & 1 & 1 & Optimal & 30.01 & 11 & 11.00 &  0.00\\
instance n=20 174.alb & 1 & 1 & Optimal & 30.02 & 12 & 12.00 &  0.00\\
instance n=20 175.alb & 1 & 1 & Optimal & 30.00 & 10 & 10.00 &  0.00\\
instance n=20 176.alb & 1 & 1 & Optimal & 30.01 & 11 & 11.00 &  0.00\\
instance n=20 177.alb & 1 & 1 & Optimal & 30.00 & 10 & 10.00 &  0.00\\
instance n=20 178.alb & 1 & 1 & Optimal & 30.01 & 11 & 11.00 &  0.00\\
instance n=20 179.alb & 1 & 1 & Optimal & 30.01 & 11 & 11.00 &  0.00\\
instance n=20 18.alb & 1 & 1 & Optimal & 30.02 & 11 & 11.00 &  0.00\\
instance n=20 180.alb & 1 & 1 & Solution & 30.03 & 13 & 12.00 &  7.69\\
instance n=20 181.alb & 1 & 1 & Optimal & 30.01 & 11 & 11.00 &  0.00\\
instance n=20 182.alb & 1 & 1 & Optimal & 30.00 & 11 & 11.00 &  0.00\\
instance n=20 183.alb & 1 & 1 & Solution & 30.07 & 13 & 11.00 & 15.38\\
instance n=20 184.alb & 1 & 1 & Optimal & 30.02 & 12 & 12.00 &  0.00\\
instance n=20 185.alb & 1 & 1 & Solution & 30.05 & 15 & 15.00 &  0.00\\
instance n=20 186.alb & 1 & 1 & Solution & 30.04 & 14 & 13.00 &  7.14\\
instance n=20 187.alb & 1 & 1 & Optimal & 30.01 & 10 & 10.00 &  0.00\\
instance n=20 188.alb & 1 & 1 & Optimal & 30.01 & 11 & 11.00 &  0.00\\
instance n=20 189.alb & 1 & 1 & Optimal & 30.03 & 13 & 13.00 &  0.00\\
instance n=20 19.alb & 1 & 1 & Solution & 30.18 & 14 & 13.00 &  7.14\\
instance n=20 190.alb & 1 & 1 & Solution & 30.05 & 15 & 13.00 & 13.33\\
instance n=20 191.alb & 1 & 1 & Optimal &  1.62 & 4 &  4.00 &  0.00\\
instance n=20 192.alb & 1 & 1 & Optimal & 30.01 & 5 &  5.00 &  0.00\\
instance n=20 193.alb & 1 & 1 & Optimal &  1.93 & 5 &  5.00 &  0.00\\
instance n=20 194.alb & 1 & 1 & Optimal &  0.60 & 6 &  6.00 &  0.00\\
instance n=20 195.alb & 1 & 1 & Optimal &  0.12 & 6 &  6.00 &  0.00\\
instance n=20 196.alb & 1 & 1 & Optimal &  0.25 & 5 &  5.00 &  0.00\\
instance n=20 197.alb & 1 & 1 & Optimal &  9.03 & 4 &  4.00 &  0.00\\
instance n=20 198.alb & 1 & 1 & Optimal & 30.01 & 6 &  6.00 &  0.00\\
instance n=20 199.alb & 1 & 1 & Optimal & 30.01 & 5 &  5.00 &  0.00\\
instance n=20 2.alb & 1 & 1 & Optimal & 30.02 & 3 &  3.00 &  0.00\\
instance n=20 20.alb & 1 & 1 & Optimal & 30.02 & 11 & 11.00 &  0.00\\
instance n=20 200.alb & 1 & 1 & Optimal & 30.03 & 6 &  6.00 &  0.00\\
instance n=20 201.alb & 1 & 1 & Optimal & 30.01 & 6 &  6.00 &  0.00\\
instance n=20 202.alb & 1 & 1 & Optimal &  0.98 & 4 &  4.00 &  0.00\\
instance n=20 203.alb & 1 & 1 & Optimal & 27.03 & 4 &  4.00 &  0.00\\
instance n=20 204.alb & 1 & 1 & Optimal & 30.02 & 5 &  5.00 &  0.00\\
instance n=20 205.alb & 1 & 1 & Optimal &  0.52 & 6 &  6.00 &  0.00\\
instance n=20 206.alb & 1 & 1 & Optimal & 30.01 & 5 &  5.00 &  0.00\\
instance n=20 207.alb & 1 & 1 & Optimal & 30.01 & 6 &  6.00 &  0.00\\
instance n=20 208.alb & 1 & 1 & Optimal & 30.01 & 5 &  5.00 &  0.00\\
instance n=20 209.alb & 1 & 1 & Optimal &  1.31 & 4 &  4.00 &  0.00\\
instance n=20 21.alb & 1 & 1 & Optimal & 30.01 & 14 & 14.00 &  0.00\\
instance n=20 210.alb & 1 & 1 & Optimal &  9.87 & 5 &  5.00 &  0.00\\
instance n=20 211.alb & 1 & 1 & Optimal & 30.01 & 5 &  5.00 &  0.00\\
instance n=20 212.alb & 1 & 1 & Optimal & 30.03 & 5 &  5.00 &  0.00\\
instance n=20 213.alb & 1 & 1 & Optimal & 30.02 & 5 &  5.00 &  0.00\\
instance n=20 214.alb & 1 & 1 & Optimal & 30.00 & 5 &  5.00 &  0.00\\
instance n=20 215.alb & 1 & 1 & Optimal & 30.01 & 5 &  5.00 &  0.00\\
instance n=20 216.alb & 1 & 1 & Optimal & 30.01 & 3 &  3.00 &  0.00\\
instance n=20 217.alb & 1 & 1 & Optimal &  0.20 & 4 &  4.00 &  0.00\\
instance n=20 218.alb & 1 & 1 & Optimal & 30.01 & 3 &  3.00 &  0.00\\
instance n=20 219.alb & 1 & 1 & Optimal &  2.23 & 3 &  3.00 &  0.00\\
instance n=20 22.alb & 1 & 1 & Optimal & 30.02 & 12 & 12.00 &  0.00\\
instance n=20 220.alb & 1 & 1 & Optimal & 21.55 & 3 &  3.00 &  0.00\\
instance n=20 221.alb & 1 & 1 & Optimal &  0.10 & 3 &  3.00 &  0.00\\
instance n=20 222.alb & 1 & 1 & Optimal & 24.81 & 3 &  3.00 &  0.00\\
instance n=20 223.alb & 1 & 1 & Optimal &  1.09 & 3 &  3.00 &  0.00\\
instance n=20 224.alb & 1 & 1 & Optimal &  0.19 & 3 &  3.00 &  0.00\\
instance n=20 225.alb & 1 & 1 & Optimal & 10.72 & 3 &  3.00 &  0.00\\
instance n=20 226.alb & 1 & 1 & Optimal &  0.43 & 3 &  3.00 &  0.00\\
instance n=20 227.alb & 1 & 1 & Optimal &  1.12 & 3 &  3.00 &  0.00\\
instance n=20 228.alb & 1 & 1 & Optimal &  0.10 & 2 &  2.00 &  0.00\\
instance n=20 229.alb & 1 & 1 & Optimal &  0.61 & 3 &  3.00 &  0.00\\
instance n=20 23.alb & 1 & 1 & Solution & 30.08 & 13 & 11.00 & 15.38\\
instance n=20 230.alb & 1 & 1 & Optimal &  1.10 & 3 &  3.00 &  0.00\\
instance n=20 231.alb & 1 & 1 & Optimal &  0.82 & 3 &  3.00 &  0.00\\
instance n=20 232.alb & 1 & 1 & Optimal &  0.45 & 3 &  3.00 &  0.00\\
instance n=20 233.alb & 1 & 1 & Optimal &  0.14 & 3 &  3.00 &  0.00\\
instance n=20 234.alb & 1 & 1 & Optimal &  0.17 & 3 &  3.00 &  0.00\\
instance n=20 235.alb & 1 & 1 & Optimal &  6.98 & 3 &  3.00 &  0.00\\
instance n=20 236.alb & 1 & 1 & Optimal &  4.29 & 3 &  3.00 &  0.00\\
instance n=20 237.alb & 1 & 1 & Optimal &  0.10 & 3 &  3.00 &  0.00\\
instance n=20 238.alb & 1 & 1 & Optimal &  3.79 & 3 &  3.00 &  0.00\\
instance n=20 239.alb & 1 & 1 & Optimal &  3.39 & 3 &  3.00 &  0.00\\
instance n=20 24.alb & 1 & 1 & Optimal & 30.03 & 11 & 11.00 &  0.00\\
instance n=20 240.alb & 1 & 1 & Optimal & 30.01 & 3 &  3.00 &  0.00\\
instance n=20 241.alb & 1 & 1 & Optimal & 30.01 & 13 & 13.00 &  0.00\\
instance n=20 242.alb & 1 & 1 & Optimal & 30.01 & 12 & 12.00 &  0.00\\
instance n=20 243.alb & 1 & 1 & Optimal & 30.01 & 10 & 10.00 &  0.00\\
instance n=20 244.alb & 1 & 1 & Optimal &  7.09 & 11 & 11.00 &  0.00\\
instance n=20 245.alb & 1 & 1 & Optimal & 30.01 & 13 & 13.00 &  0.00\\
instance n=20 246.alb & 1 & 1 & Optimal & 30.03 & 13 & 13.00 &  0.00\\
instance n=20 247.alb & 1 & 1 & Optimal &  9.11 & 11 & 11.00 &  0.00\\
instance n=20 248.alb & 1 & 1 & Optimal & 30.01 & 11 & 11.00 &  0.00\\
instance n=20 249.alb & 1 & 1 & Optimal & 30.01 & 13 & 13.00 &  0.00\\
instance n=20 25.alb & 1 & 1 & Optimal & 30.01 & 11 & 11.00 &  0.00\\
instance n=20 250.alb & 1 & 1 & Optimal & 30.01 & 10 & 10.00 &  0.00\\
instance n=20 251.alb & 1 & 1 & Optimal & 30.01 & 12 & 12.00 &  0.00\\
instance n=20 252.alb & 1 & 1 & Optimal & 30.01 & 11 & 11.00 &  0.00\\
instance n=20 253.alb & 1 & 1 & Optimal & 30.03 & 13 & 13.00 &  0.00\\
instance n=20 254.alb & 1 & 1 & Optimal & 30.02 & 12 & 12.00 &  0.00\\
instance n=20 255.alb & 1 & 1 & Optimal & 30.01 & 13 & 13.00 &  0.00\\
instance n=20 256.alb & 1 & 1 & Optimal & 30.02 & 14 & 14.00 &  0.00\\
instance n=20 257.alb & 1 & 1 & Optimal & 30.02 & 10 & 10.00 &  0.00\\
instance n=20 258.alb & 1 & 1 & Optimal & 30.02 & 13 & 13.00 &  0.00\\
instance n=20 259.alb & 1 & 1 & Optimal & 30.01 & 13 & 13.00 &  0.00\\
instance n=20 26.alb & 1 & 1 & Optimal & 30.02 & 12 & 12.00 &  0.00\\
instance n=20 260.alb & 1 & 1 & Optimal & 30.03 & 12 & 12.00 &  0.00\\
instance n=20 261.alb & 1 & 1 & Optimal & 30.00 & 12 & 12.00 &  0.00\\
instance n=20 262.alb & 1 & 1 & Optimal & 30.01 & 11 & 11.00 &  0.00\\
instance n=20 263.alb & 1 & 1 & Optimal & 30.00 & 12 & 12.00 &  0.00\\
instance n=20 264.alb & 1 & 1 & Optimal & 30.01 & 12 & 12.00 &  0.00\\
instance n=20 265.alb & 1 & 1 & Optimal & 30.01 & 12 & 12.00 &  0.00\\
instance n=20 266.alb & 1 & 1 & Optimal &  0.12 & 5 &  5.00 &  0.00\\
instance n=20 267.alb & 1 & 1 & Optimal & 30.02 & 6 &  6.00 &  0.00\\
instance n=20 268.alb & 1 & 1 & Optimal &  4.54 & 6 &  6.00 &  0.00\\
instance n=20 269.alb & 1 & 1 & Optimal & 30.01 & 7 &  7.00 &  0.00\\
instance n=20 27.alb & 1 & 1 & Solution & 30.04 & 13 & 12.00 &  7.69\\
instance n=20 270.alb & 1 & 1 & Optimal & 30.03 & 7 &  7.00 &  0.00\\
instance n=20 271.alb & 1 & 1 & Optimal &  0.07 & 6 &  6.00 &  0.00\\
instance n=20 272.alb & 1 & 1 & Optimal &  0.11 & 5 &  5.00 &  0.00\\
instance n=20 273.alb & 1 & 1 & Optimal &  0.10 & 5 &  5.00 &  0.00\\
instance n=20 274.alb & 1 & 1 & Optimal & 30.02 & 6 &  6.00 &  0.00\\
instance n=20 275.alb & 1 & 1 & Optimal & 30.02 & 5 &  5.00 &  0.00\\
instance n=20 276.alb & 1 & 1 & Optimal & 20.71 & 4 &  4.00 &  0.00\\
instance n=20 277.alb & 1 & 1 & Optimal &  3.59 & 4 &  4.00 &  0.00\\
instance n=20 278.alb & 1 & 1 & Optimal &  0.12 & 6 &  6.00 &  0.00\\
instance n=20 279.alb & 1 & 1 & Optimal &  0.13 & 6 &  6.00 &  0.00\\
instance n=20 28.alb & 1 & 1 & Optimal & 30.01 & 12 & 12.00 &  0.00\\
instance n=20 280.alb & 1 & 1 & Optimal &  1.24 & 5 &  5.00 &  0.00\\
instance n=20 281.alb & 1 & 1 & Optimal &  0.31 & 4 &  4.00 &  0.00\\
instance n=20 282.alb & 1 & 1 & Optimal &  0.20 & 4 &  4.00 &  0.00\\
instance n=20 283.alb & 1 & 1 & Optimal & 30.01 & 5 &  5.00 &  0.00\\
instance n=20 284.alb & 1 & 1 & Optimal & 30.03 & 5 &  5.00 &  0.00\\
instance n=20 285.alb & 1 & 1 & Optimal & 14.03 & 5 &  5.00 &  0.00\\
instance n=20 286.alb & 1 & 1 & Optimal & 30.01 & 5 &  5.00 &  0.00\\
instance n=20 287.alb & 1 & 1 & Optimal &  1.50 & 5 &  5.00 &  0.00\\
instance n=20 288.alb & 1 & 1 & Optimal & 25.38 & 6 &  6.00 &  0.00\\
instance n=20 289.alb & 1 & 1 & Optimal & 30.01 & 5 &  5.00 &  0.00\\
instance n=20 29.alb & 1 & 1 & Optimal & 30.01 & 10 & 10.00 &  0.00\\
instance n=20 290.alb & 1 & 1 & Optimal &  4.51 & 5 &  5.00 &  0.00\\
instance n=20 291.alb & 1 & 1 & Optimal &  1.04 & 3 &  3.00 &  0.00\\
instance n=20 292.alb & 1 & 1 & Optimal &  1.04 & 3 &  3.00 &  0.00\\
instance n=20 293.alb & 1 & 1 & Optimal &  0.52 & 3 &  3.00 &  0.00\\
instance n=20 294.alb & 1 & 1 & Optimal &  0.11 & 3 &  3.00 &  0.00\\
instance n=20 295.alb & 1 & 1 & Optimal &  2.53 & 3 &  3.00 &  0.00\\
instance n=20 296.alb & 1 & 1 & Optimal &  2.28 & 3 &  3.00 &  0.00\\
instance n=20 297.alb & 1 & 1 & Optimal & 30.01 & 3 &  3.00 &  0.00\\
instance n=20 298.alb & 1 & 1 & Optimal &  0.22 & 3 &  3.00 &  0.00\\
instance n=20 299.alb & 1 & 1 & Optimal &  8.46 & 3 &  3.00 &  0.00\\
instance n=20 3.alb & 1 & 1 & Optimal &  0.96 & 3 &  3.00 &  0.00\\
instance n=20 30.alb & 1 & 1 & Solution & 30.07 & 16 & 14.00 & 12.50\\
instance n=20 300.alb & 1 & 1 & Optimal & 30.02 & 4 &  4.00 &  0.00\\
instance n=20 301.alb & 1 & 1 & Optimal &  0.35 & 3 &  3.00 &  0.00\\
instance n=20 302.alb & 1 & 1 & Optimal &  1.88 & 3 &  3.00 &  0.00\\
instance n=20 303.alb & 1 & 1 & Optimal &  0.59 & 3 &  3.00 &  0.00\\
instance n=20 304.alb & 1 & 1 & Optimal &  0.63 & 3 &  3.00 &  0.00\\
instance n=20 305.alb & 1 & 1 & Optimal &  0.09 & 3 &  3.00 &  0.00\\
instance n=20 306.alb & 1 & 1 & Optimal &  2.53 & 3 &  3.00 &  0.00\\
instance n=20 307.alb & 1 & 1 & Optimal &  0.21 & 3 &  3.00 &  0.00\\
instance n=20 308.alb & 1 & 1 & Optimal &  0.26 & 3 &  3.00 &  0.00\\
instance n=20 309.alb & 1 & 1 & Optimal & 30.01 & 3 &  3.00 &  0.00\\
instance n=20 31.alb & 1 & 1 & Optimal & 30.02 & 12 & 12.00 &  0.00\\
instance n=20 310.alb & 1 & 1 & Optimal &  0.36 & 3 &  3.00 &  0.00\\
instance n=20 311.alb & 1 & 1 & Optimal &  0.14 & 3 &  3.00 &  0.00\\
instance n=20 312.alb & 1 & 1 & Optimal & 30.02 & 4 &  4.00 &  0.00\\
instance n=20 313.alb & 1 & 1 & Optimal & 23.64 & 3 &  3.00 &  0.00\\
instance n=20 314.alb & 1 & 1 & Optimal &  1.93 & 3 &  3.00 &  0.00\\
instance n=20 315.alb & 1 & 1 & Optimal &  4.15 & 3 &  3.00 &  0.00\\
instance n=20 316.alb & 1 & 1 & Solution & 30.09 & 10 & 10.00 &  0.00\\
instance n=20 317.alb & 1 & 1 & Optimal & 30.02 & 10 & 10.00 &  0.00\\
instance n=20 318.alb & 1 & 1 & Optimal & 30.02 & 10 & 10.00 &  0.00\\
instance n=20 319.alb & 1 & 1 & Solution & 30.06 & 14 & 13.00 &  7.14\\
instance n=20 32.alb & 1 & 1 & Solution & 30.06 & 13 & 12.00 &  7.69\\
instance n=20 320.alb & 1 & 1 & Optimal & 30.02 & 12 & 12.00 &  0.00\\
instance n=20 321.alb & 1 & 1 & Solution & 30.03 & 14 & 11.00 & 21.43\\
instance n=20 322.alb & 1 & 1 & Solution & 30.03 & 12 & 11.00 &  8.33\\
instance n=20 323.alb & 1 & 1 & Solution & 30.05 & 13 & 12.00 &  7.69\\
instance n=20 324.alb & 1 & 1 & Optimal & 30.03 & 9 &  9.00 &  0.00\\
instance n=20 325.alb & 1 & 1 & Solution & 30.20 & 14 & 12.00 & 14.29\\
instance n=20 326.alb & 1 & 1 & Solution & 30.04 & 14 & 12.00 & 14.29\\
instance n=20 327.alb & 1 & 1 & Solution & 30.06 & 13 & 11.00 & 15.38\\
instance n=20 328.alb & 1 & 1 & Solution & 30.06 & 13 & 11.00 & 15.38\\
instance n=20 329.alb & 1 & 1 & Optimal & 30.02 & 10 & 10.00 &  0.00\\
instance n=20 33.alb & 1 & 1 & Optimal & 30.01 & 11 & 11.00 &  0.00\\
instance n=20 330.alb & 1 & 1 & Solution & 30.04 & 12 & 11.00 &  8.33\\
instance n=20 331.alb & 1 & 1 & Solution & 30.05 & 13 & 12.00 &  7.69\\
instance n=20 332.alb & 1 & 1 & Solution & 30.04 & 13 & 12.00 &  7.69\\
instance n=20 333.alb & 1 & 1 & Optimal & 30.02 & 11 & 11.00 &  0.00\\
instance n=20 334.alb & 1 & 1 & Optimal & 30.01 & 10 & 10.00 &  0.00\\
instance n=20 335.alb & 1 & 1 & Solution & 30.05 & 14 & 11.00 & 21.43\\
instance n=20 336.alb & 1 & 1 & Optimal & 30.01 & 11 & 11.00 &  0.00\\
instance n=20 337.alb & 1 & 1 & Optimal & 30.01 & 10 & 10.00 &  0.00\\
instance n=20 338.alb & 1 & 1 & Optimal & 30.03 & 14 & 14.00 &  0.00\\
instance n=20 339.alb & 1 & 1 & Solution & 30.04 & 13 & 11.00 & 15.38\\
instance n=20 34.alb & 1 & 1 & Optimal & 30.01 & 12 & 12.00 &  0.00\\
instance n=20 340.alb & 1 & 1 & Optimal & 30.02 & 11 & 11.00 &  0.00\\
instance n=20 341.alb & 1 & 1 & Optimal & 30.00 & 6 &  6.00 &  0.00\\
instance n=20 342.alb & 1 & 1 & Optimal & 30.01 & 6 &  6.00 &  0.00\\
instance n=20 343.alb & 1 & 1 & Optimal &  8.37 & 6 &  6.00 &  0.00\\
instance n=20 344.alb & 1 & 1 & Optimal & 30.01 & 6 &  6.00 &  0.00\\
instance n=20 345.alb & 1 & 1 & Optimal & 30.01 & 4 &  4.00 &  0.00\\
instance n=20 346.alb & 1 & 1 & Optimal &  5.31 & 5 &  5.00 &  0.00\\
instance n=20 347.alb & 1 & 1 & Optimal & 30.01 & 6 &  6.00 &  0.00\\
instance n=20 348.alb & 1 & 1 & Optimal &  7.51 & 5 &  5.00 &  0.00\\
instance n=20 349.alb & 1 & 1 & Optimal & 30.01 & 5 &  5.00 &  0.00\\
instance n=20 35.alb & 1 & 1 & Optimal & 30.01 & 12 & 12.00 &  0.00\\
instance n=20 350.alb & 1 & 1 & Optimal &  4.40 & 5 &  5.00 &  0.00\\
instance n=20 351.alb & 1 & 1 & Optimal &  0.12 & 5 &  5.00 &  0.00\\
instance n=20 352.alb & 1 & 1 & Optimal & 30.01 & 4 &  4.00 &  0.00\\
instance n=20 353.alb & 1 & 1 & Optimal & 30.01 & 6 &  6.00 &  0.00\\
instance n=20 354.alb & 1 & 1 & Optimal & 30.01 & 6 &  6.00 &  0.00\\
instance n=20 355.alb & 1 & 1 & Optimal & 30.01 & 5 &  5.00 &  0.00\\
instance n=20 356.alb & 1 & 1 & Optimal & 30.01 & 5 &  5.00 &  0.00\\
instance n=20 357.alb & 1 & 1 & Optimal & 10.73 & 5 &  5.00 &  0.00\\
instance n=20 358.alb & 1 & 1 & Optimal & 30.01 & 4 &  4.00 &  0.00\\
instance n=20 359.alb & 1 & 1 & Optimal & 11.45 & 4 &  4.00 &  0.00\\
instance n=20 36.alb & 1 & 1 & Optimal & 30.02 & 13 & 13.00 &  0.00\\
instance n=20 360.alb & 1 & 1 & Optimal & 30.01 & 6 &  6.00 &  0.00\\
instance n=20 361.alb & 1 & 1 & Optimal & 30.01 & 5 &  5.00 &  0.00\\
instance n=20 362.alb & 1 & 1 & Optimal &  3.28 & 5 &  5.00 &  0.00\\
instance n=20 363.alb & 1 & 1 & Optimal & 30.01 & 7 &  7.00 &  0.00\\
instance n=20 364.alb & 1 & 1 & Optimal & 30.01 & 4 &  4.00 &  0.00\\
instance n=20 365.alb & 1 & 1 & Optimal &  9.54 & 5 &  5.00 &  0.00\\
instance n=20 366.alb & 1 & 1 & Optimal &  4.01 & 3 &  3.00 &  0.00\\
instance n=20 367.alb & 1 & 1 & Optimal & 16.45 & 3 &  3.00 &  0.00\\
instance n=20 368.alb & 1 & 1 & Optimal &  5.64 & 3 &  3.00 &  0.00\\
instance n=20 369.alb & 1 & 1 & Optimal & 30.01 & 3 &  3.00 &  0.00\\
instance n=20 37.alb & 1 & 1 & Solution & 30.05 & 12 & 12.00 &  0.00\\
instance n=20 370.alb & 1 & 1 & Optimal &  0.07 & 3 &  3.00 &  0.00\\
instance n=20 371.alb & 1 & 1 & Optimal & 18.45 & 3 &  3.00 &  0.00\\
instance n=20 372.alb & 1 & 1 & Optimal &  0.23 & 3 &  3.00 &  0.00\\
instance n=20 373.alb & 1 & 1 & Optimal &  0.20 & 3 &  3.00 &  0.00\\
instance n=20 374.alb & 1 & 1 & Optimal &  0.98 & 3 &  3.00 &  0.00\\
instance n=20 375.alb & 1 & 1 & Optimal &  0.29 & 3 &  3.00 &  0.00\\
instance n=20 376.alb & 1 & 1 & Optimal &  2.16 & 3 &  3.00 &  0.00\\
instance n=20 377.alb & 1 & 1 & Optimal &  0.31 & 3 &  3.00 &  0.00\\
instance n=20 378.alb & 1 & 1 & Optimal &  0.07 & 3 &  3.00 &  0.00\\
instance n=20 379.alb & 1 & 1 & Optimal & 30.01 & 4 &  4.00 &  0.00\\
instance n=20 38.alb & 1 & 1 & Optimal & 30.01 & 12 & 12.00 &  0.00\\
instance n=20 380.alb & 1 & 1 & Optimal &  4.54 & 3 &  3.00 &  0.00\\
instance n=20 381.alb & 1 & 1 & Optimal & 10.04 & 3 &  3.00 &  0.00\\
instance n=20 382.alb & 1 & 1 & Optimal &  2.84 & 4 &  4.00 &  0.00\\
instance n=20 383.alb & 1 & 1 & Optimal & 30.01 & 3 &  3.00 &  0.00\\
instance n=20 384.alb & 1 & 1 & Optimal & 13.68 & 3 &  3.00 &  0.00\\
instance n=20 385.alb & 1 & 1 & Optimal & 30.01 & 3 &  3.00 &  0.00\\
instance n=20 386.alb & 1 & 1 & Optimal & 29.28 & 3 &  3.00 &  0.00\\
instance n=20 387.alb & 1 & 1 & Optimal & 30.01 & 3 &  3.00 &  0.00\\
instance n=20 388.alb & 1 & 1 & Optimal &  5.18 & 3 &  3.00 &  0.00\\
instance n=20 389.alb & 1 & 1 & Optimal &  6.90 & 3 &  3.00 &  0.00\\
instance n=20 39.alb & 1 & 1 & Optimal & 30.03 & 13 & 13.00 &  0.00\\
instance n=20 390.alb & 1 & 1 & Optimal & 30.01 & 3 &  3.00 &  0.00\\
instance n=20 391.alb & 1 & 1 & Optimal & 30.02 & 11 & 11.00 &  0.00\\
instance n=20 392.alb & 1 & 1 & Optimal & 30.01 & 14 & 14.00 &  0.00\\
instance n=20 393.alb & 1 & 1 & Optimal & 30.01 & 11 & 11.00 &  0.00\\
instance n=20 394.alb & 1 & 1 & Optimal & 30.01 & 12 & 12.00 &  0.00\\
instance n=20 395.alb & 1 & 1 & Optimal & 30.03 & 12 & 12.00 &  0.00\\
instance n=20 396.alb & 1 & 1 & Optimal & 30.01 & 13 & 13.00 &  0.00\\
instance n=20 397.alb & 1 & 1 & Optimal & 25.45 & 10 & 10.00 &  0.00\\
instance n=20 398.alb & 1 & 1 & Optimal & 30.03 & 11 & 11.00 &  0.00\\
instance n=20 399.alb & 1 & 1 & Optimal & 30.01 & 13 & 13.00 &  0.00\\
instance n=20 4.alb & 1 & 1 & Optimal &  0.28 & 3 &  3.00 &  0.00\\
instance n=20 40.alb & 1 & 1 & Optimal & 30.02 & 12 & 12.00 &  0.00\\
instance n=20 400.alb & 1 & 1 & Optimal & 30.01 & 12 & 12.00 &  0.00\\
instance n=20 401.alb & 1 & 1 & Optimal & 30.01 & 12 & 12.00 &  0.00\\
instance n=20 402.alb & 1 & 1 & Optimal & 30.01 & 12 & 12.00 &  0.00\\
instance n=20 403.alb & 1 & 1 & Optimal & 30.02 & 12 & 12.00 &  0.00\\
instance n=20 404.alb & 1 & 1 & Optimal & 17.25 & 10 & 10.00 &  0.00\\
instance n=20 405.alb & 1 & 1 & Optimal & 30.01 & 12 & 12.00 &  0.00\\
instance n=20 406.alb & 1 & 1 & Optimal & 30.01 & 14 & 14.00 &  0.00\\
instance n=20 407.alb & 1 & 1 & Optimal & 30.01 & 10 & 10.00 &  0.00\\
instance n=20 408.alb & 1 & 1 & Optimal & 30.03 & 14 & 14.00 &  0.00\\
instance n=20 409.alb & 1 & 1 & Optimal & 30.01 & 12 & 12.00 &  0.00\\
instance n=20 41.alb & 1 & 1 & Optimal & 30.01 & 6 &  6.00 &  0.00\\
instance n=20 410.alb & 1 & 1 & Optimal & 30.01 & 11 & 11.00 &  0.00\\
instance n=20 411.alb & 1 & 1 & Optimal & 30.01 & 15 & 15.00 &  0.00\\
instance n=20 412.alb & 1 & 1 & Optimal & 30.01 & 11 & 11.00 &  0.00\\
instance n=20 413.alb & 1 & 1 & Optimal & 30.01 & 10 & 10.00 &  0.00\\
instance n=20 414.alb & 1 & 1 & Optimal & 30.01 & 12 & 12.00 &  0.00\\
instance n=20 415.alb & 1 & 1 & Optimal & 30.03 & 10 & 10.00 &  0.00\\
instance n=20 416.alb & 1 & 1 & Optimal & 30.02 & 6 &  6.00 &  0.00\\
instance n=20 417.alb & 1 & 1 & Optimal & 30.01 & 5 &  5.00 &  0.00\\
instance n=20 418.alb & 1 & 1 & Optimal &  0.24 & 6 &  6.00 &  0.00\\
instance n=20 419.alb & 1 & 1 & Optimal & 30.02 & 4 &  4.00 &  0.00\\
instance n=20 42.alb & 1 & 1 & Optimal &  0.97 & 5 &  5.00 &  0.00\\
instance n=20 420.alb & 1 & 1 & Optimal & 30.01 & 5 &  5.00 &  0.00\\
instance n=20 421.alb & 1 & 1 & Optimal & 30.01 & 6 &  6.00 &  0.00\\
instance n=20 422.alb & 1 & 1 & Optimal & 30.01 & 4 &  4.00 &  0.00\\
instance n=20 423.alb & 1 & 1 & Optimal & 30.01 & 6 &  6.00 &  0.00\\
instance n=20 424.alb & 1 & 1 & Optimal &  0.21 & 5 &  5.00 &  0.00\\
instance n=20 425.alb & 1 & 1 & Optimal & 30.02 & 6 &  6.00 &  0.00\\
instance n=20 426.alb & 1 & 1 & Optimal &  0.78 & 5 &  5.00 &  0.00\\
instance n=20 427.alb & 1 & 1 & Optimal &  5.78 & 6 &  6.00 &  0.00\\
instance n=20 428.alb & 1 & 1 & Optimal &  0.12 & 5 &  5.00 &  0.00\\
instance n=20 429.alb & 1 & 1 & Optimal &  3.09 & 4 &  4.00 &  0.00\\
instance n=20 43.alb & 1 & 1 & Optimal & 30.01 & 5 &  5.00 &  0.00\\
instance n=20 430.alb & 1 & 1 & Optimal & 30.02 & 5 &  5.00 &  0.00\\
instance n=20 431.alb & 1 & 1 & Optimal &  0.20 & 6 &  6.00 &  0.00\\
instance n=20 432.alb & 1 & 1 & Optimal & 30.02 & 5 &  5.00 &  0.00\\
instance n=20 433.alb & 1 & 1 & Optimal & 30.03 & 5 &  5.00 &  0.00\\
instance n=20 434.alb & 1 & 1 & Optimal & 30.01 & 5 &  5.00 &  0.00\\
instance n=20 435.alb & 1 & 1 & Optimal &  0.43 & 7 &  7.00 &  0.00\\
instance n=20 436.alb & 1 & 1 & Optimal &  1.15 & 5 &  5.00 &  0.00\\
instance n=20 437.alb & 1 & 1 & Optimal &  0.43 & 5 &  5.00 &  0.00\\
instance n=20 438.alb & 1 & 1 & Optimal & 30.01 & 6 &  6.00 &  0.00\\
instance n=20 439.alb & 1 & 1 & Optimal & 23.89 & 5 &  5.00 &  0.00\\
instance n=20 44.alb & 1 & 1 & Optimal & 30.01 & 5 &  5.00 &  0.00\\
instance n=20 440.alb & 1 & 1 & Optimal &  0.09 & 5 &  5.00 &  0.00\\
instance n=20 441.alb & 1 & 1 & Optimal &  0.11 & 3 &  3.00 &  0.00\\
instance n=20 442.alb & 1 & 1 & Optimal &  0.10 & 3 &  3.00 &  0.00\\
instance n=20 443.alb & 1 & 1 & Optimal &  0.11 & 3 &  3.00 &  0.00\\
instance n=20 444.alb & 1 & 1 & Optimal &  0.10 & 3 &  3.00 &  0.00\\
instance n=20 445.alb & 1 & 1 & Optimal &  0.11 & 3 &  3.00 &  0.00\\
instance n=20 446.alb & 1 & 1 & Optimal &  0.11 & 3 &  3.00 &  0.00\\
instance n=20 447.alb & 1 & 1 & Optimal &  0.11 & 3 &  3.00 &  0.00\\
instance n=20 448.alb & 1 & 1 & Optimal &  0.13 & 3 &  3.00 &  0.00\\
instance n=20 449.alb & 1 & 1 & Optimal &  0.11 & 3 &  3.00 &  0.00\\
instance n=20 45.alb & 1 & 1 & Optimal &  2.00 & 6 &  6.00 &  0.00\\
instance n=20 450.alb & 1 & 1 & Optimal &  0.11 & 3 &  3.00 &  0.00\\
instance n=20 451.alb & 1 & 1 & Optimal &  0.43 & 3 &  3.00 &  0.00\\
instance n=20 452.alb & 1 & 1 & Optimal &  0.11 & 3 &  3.00 &  0.00\\
instance n=20 453.alb & 1 & 1 & Optimal &  0.11 & 3 &  3.00 &  0.00\\
instance n=20 454.alb & 1 & 1 & Optimal &  0.11 & 3 &  3.00 &  0.00\\
instance n=20 455.alb & 1 & 1 & Optimal &  0.11 & 3 &  3.00 &  0.00\\
instance n=20 456.alb & 1 & 1 & Optimal &  0.10 & 4 &  4.00 &  0.00\\
instance n=20 457.alb & 1 & 1 & Optimal &  0.12 & 3 &  3.00 &  0.00\\
instance n=20 458.alb & 1 & 1 & Optimal &  0.11 & 3 &  3.00 &  0.00\\
instance n=20 459.alb & 1 & 1 & Optimal &  0.11 & 3 &  3.00 &  0.00\\
instance n=20 46.alb & 1 & 1 & Optimal & 15.33 & 4 &  4.00 &  0.00\\
instance n=20 460.alb & 1 & 1 & Optimal &  0.11 & 3 &  3.00 &  0.00\\
instance n=20 461.alb & 1 & 1 & Optimal &  0.11 & 3 &  3.00 &  0.00\\
instance n=20 462.alb & 1 & 1 & Optimal &  0.14 & 3 &  3.00 &  0.00\\
instance n=20 463.alb & 1 & 1 & Optimal &  0.11 & 3 &  3.00 &  0.00\\
instance n=20 464.alb & 1 & 1 & Optimal &  0.11 & 3 &  3.00 &  0.00\\
instance n=20 465.alb & 1 & 1 & Optimal &  0.11 & 3 &  3.00 &  0.00\\
instance n=20 466.alb & 1 & 1 & Optimal &  0.15 & 13 & 13.00 &  0.00\\
instance n=20 467.alb & 1 & 1 & Optimal & 30.02 & 14 & 14.00 &  0.00\\
instance n=20 468.alb & 1 & 1 & Optimal &  5.15 & 13 & 13.00 &  0.00\\
instance n=20 469.alb & 1 & 1 & Optimal & 30.02 & 14 & 14.00 &  0.00\\
instance n=20 47.alb & 1 & 1 & Optimal & 30.01 & 4 &  4.00 &  0.00\\
instance n=20 470.alb & 1 & 1 & Optimal & 30.01 & 12 & 12.00 &  0.00\\
instance n=20 471.alb & 1 & 1 & Optimal & 14.20 & 12 & 12.00 &  0.00\\
instance n=20 472.alb & 1 & 1 & Optimal & 30.03 & 13 & 13.00 &  0.00\\
instance n=20 473.alb & 1 & 1 & Optimal &  0.12 & 10 & 10.00 &  0.00\\
instance n=20 474.alb & 1 & 1 & Optimal & 30.01 & 14 & 14.00 &  0.00\\
instance n=20 475.alb & 1 & 1 & Optimal & 30.01 & 11 & 11.00 &  0.00\\
instance n=20 476.alb & 1 & 1 & Optimal & 30.01 & 11 & 11.00 &  0.00\\
instance n=20 477.alb & 1 & 1 & Optimal & 30.01 & 11 & 11.00 &  0.00\\
instance n=20 478.alb & 1 & 1 & Optimal & 30.01 & 12 & 12.00 &  0.00\\
instance n=20 479.alb & 1 & 1 & Optimal & 30.02 & 13 & 13.00 &  0.00\\
instance n=20 48.alb & 1 & 1 & Optimal &  8.68 & 5 &  5.00 &  0.00\\
instance n=20 480.alb & 1 & 1 & Optimal & 30.01 & 13 & 13.00 &  0.00\\
instance n=20 481.alb & 1 & 1 & Optimal & 30.02 & 13 & 13.00 &  0.00\\
instance n=20 482.alb & 1 & 1 & Optimal & 30.02 & 13 & 13.00 &  0.00\\
instance n=20 483.alb & 1 & 1 & Optimal & 30.01 & 12 & 12.00 &  0.00\\
instance n=20 484.alb & 1 & 1 & Optimal & 30.01 & 13 & 13.00 &  0.00\\
instance n=20 485.alb & 1 & 1 & Optimal & 30.01 & 15 & 15.00 &  0.00\\
instance n=20 486.alb & 1 & 1 & Optimal & 30.01 & 11 & 11.00 &  0.00\\
instance n=20 487.alb & 1 & 1 & Optimal & 30.01 & 12 & 12.00 &  0.00\\
instance n=20 488.alb & 1 & 1 & Optimal & 19.31 & 15 & 15.00 &  0.00\\
instance n=20 489.alb & 1 & 1 & Optimal &  5.34 & 12 & 12.00 &  0.00\\
instance n=20 49.alb & 1 & 1 & Optimal & 30.01 & 4 &  4.00 &  0.00\\
instance n=20 490.alb & 1 & 1 & Optimal & 30.01 & 12 & 12.00 &  0.00\\
instance n=20 491.alb & 1 & 1 & Optimal &  0.09 & 6 &  6.00 &  0.00\\
instance n=20 492.alb & 1 & 1 & Optimal &  0.09 & 5 &  5.00 &  0.00\\
instance n=20 493.alb & 1 & 1 & Optimal &  0.10 & 5 &  5.00 &  0.00\\
instance n=20 494.alb & 1 & 1 & Optimal &  0.09 & 6 &  6.00 &  0.00\\
instance n=20 495.alb & 1 & 1 & Optimal &  0.10 & 6 &  6.00 &  0.00\\
instance n=20 496.alb & 1 & 1 & Optimal &  0.11 & 5 &  5.00 &  0.00\\
instance n=20 497.alb & 1 & 1 & Optimal &  0.08 & 6 &  6.00 &  0.00\\
instance n=20 498.alb & 1 & 1 & Optimal &  0.09 & 6 &  6.00 &  0.00\\
instance n=20 499.alb & 1 & 1 & Optimal &  6.90 & 5 &  5.00 &  0.00\\
instance n=20 5.alb & 1 & 1 & Optimal &  0.07 & 3 &  3.00 &  0.00\\
instance n=20 50.alb & 1 & 1 & Optimal & 30.02 & 4 &  4.00 &  0.00\\
instance n=20 500.alb & 1 & 1 & Optimal &  0.09 & 8 &  8.00 &  0.00\\
instance n=20 501.alb & 1 & 1 & Optimal &  0.12 & 5 &  5.00 &  0.00\\
instance n=20 502.alb & 1 & 1 & Optimal &  0.11 & 4 &  4.00 &  0.00\\
instance n=20 503.alb & 1 & 1 & Optimal &  0.43 & 6 &  6.00 &  0.00\\
instance n=20 504.alb & 1 & 1 & Optimal &  0.92 & 6 &  6.00 &  0.00\\
instance n=20 505.alb & 1 & 1 & Optimal &  0.09 & 6 &  6.00 &  0.00\\
instance n=20 506.alb & 1 & 1 & Optimal &  0.11 & 5 &  5.00 &  0.00\\
instance n=20 507.alb & 1 & 1 & Optimal &  0.09 & 5 &  5.00 &  0.00\\
instance n=20 508.alb & 1 & 1 & Optimal &  0.09 & 5 &  5.00 &  0.00\\
instance n=20 509.alb & 1 & 1 & Optimal &  0.09 & 4 &  4.00 &  0.00\\
instance n=20 51.alb & 1 & 1 & Optimal & 30.02 & 4 &  4.00 &  0.00\\
instance n=20 510.alb & 1 & 1 & Optimal &  0.10 & 5 &  5.00 &  0.00\\
instance n=20 511.alb & 1 & 1 & Optimal &  0.09 & 5 &  5.00 &  0.00\\
instance n=20 512.alb & 1 & 1 & Optimal &  0.11 & 5 &  5.00 &  0.00\\
instance n=20 513.alb & 1 & 1 & Optimal &  0.09 & 5 &  5.00 &  0.00\\
instance n=20 514.alb & 1 & 1 & Optimal &  0.08 & 5 &  5.00 &  0.00\\
instance n=20 515.alb & 1 & 1 & Optimal &  0.11 & 6 &  6.00 &  0.00\\
instance n=20 516.alb & 1 & 1 & Optimal &  1.15 & 3 &  3.00 &  0.00\\
instance n=20 517.alb & 1 & 1 & Optimal & 26.24 & 3 &  3.00 &  0.00\\
instance n=20 518.alb & 1 & 1 & Optimal & 30.01 & 3 &  3.00 &  0.00\\
instance n=20 519.alb & 1 & 1 & Optimal & 22.25 & 3 &  3.00 &  0.00\\
instance n=20 52.alb & 1 & 1 & Optimal & 30.02 & 4 &  4.00 &  0.00\\
instance n=20 520.alb & 1 & 1 & Optimal & 30.01 & 3 &  3.00 &  0.00\\
instance n=20 521.alb & 1 & 1 & Optimal & 30.01 & 3 &  3.00 &  0.00\\
instance n=20 522.alb & 1 & 1 & Optimal &  2.16 & 3 &  3.00 &  0.00\\
instance n=20 523.alb & 1 & 1 & Optimal &  5.27 & 3 &  3.00 &  0.00\\
instance n=20 524.alb & 1 & 1 & Optimal &  8.22 & 3 &  3.00 &  0.00\\
instance n=20 525.alb & 1 & 1 & Optimal &  0.53 & 3 &  3.00 &  0.00\\
instance n=20 53.alb & 1 & 1 & Optimal & 30.01 & 5 &  5.00 &  0.00\\
instance n=20 54.alb & 1 & 1 & Optimal & 11.85 & 5 &  5.00 &  0.00\\
instance n=20 55.alb & 1 & 1 & Optimal & 30.01 & 5 &  5.00 &  0.00\\
instance n=20 56.alb & 1 & 1 & Optimal & 30.01 & 4 &  4.00 &  0.00\\
instance n=20 57.alb & 1 & 1 & Optimal & 30.01 & 4 &  4.00 &  0.00\\
instance n=20 58.alb & 1 & 1 & Optimal &  0.75 & 5 &  5.00 &  0.00\\
instance n=20 59.alb & 1 & 1 & Optimal &  6.90 & 4 &  4.00 &  0.00\\
instance n=20 6.alb & 1 & 1 & Optimal &  2.76 & 3 &  3.00 &  0.00\\
instance n=20 60.alb & 1 & 1 & Optimal & 30.01 & 6 &  6.00 &  0.00\\
instance n=20 61.alb & 1 & 1 & Optimal & 30.01 & 7 &  7.00 &  0.00\\
instance n=20 62.alb & 1 & 1 & Optimal &  0.70 & 5 &  5.00 &  0.00\\
instance n=20 63.alb & 1 & 1 & Optimal & 30.03 & 5 &  5.00 &  0.00\\
instance n=20 64.alb & 1 & 1 & Optimal &  1.64 & 5 &  5.00 &  0.00\\
instance n=20 65.alb & 1 & 1 & Optimal & 30.01 & 5 &  5.00 &  0.00\\
instance n=20 66.alb & 1 & 1 & Optimal &  0.71 & 3 &  3.00 &  0.00\\
instance n=20 67.alb & 1 & 1 & Optimal &  1.74 & 3 &  3.00 &  0.00\\
instance n=20 68.alb & 1 & 1 & Optimal &  0.81 & 3 &  3.00 &  0.00\\
instance n=20 69.alb & 1 & 1 & Optimal &  0.07 & 2 &  2.00 &  0.00\\
instance n=20 7.alb & 1 & 1 & Optimal & 30.02 & 3 &  3.00 &  0.00\\
instance n=20 70.alb & 1 & 1 & Optimal & 30.00 & 3 &  3.00 &  0.00\\
instance n=20 71.alb & 1 & 1 & Optimal &  0.42 & 3 &  3.00 &  0.00\\
instance n=20 72.alb & 1 & 1 & Optimal &  2.04 & 3 &  3.00 &  0.00\\
instance n=20 73.alb & 1 & 1 & Optimal &  0.36 & 2 &  2.00 &  0.00\\
instance n=20 74.alb & 1 & 1 & Optimal & 30.01 & 3 &  3.00 &  0.00\\
instance n=20 75.alb & 1 & 1 & Optimal &  0.34 & 3 &  3.00 &  0.00\\
instance n=20 76.alb & 1 & 1 & Optimal &  8.91 & 3 &  3.00 &  0.00\\
instance n=20 77.alb & 1 & 1 & Optimal & 11.56 & 3 &  3.00 &  0.00\\
instance n=20 78.alb & 1 & 1 & Optimal &  0.22 & 3 &  3.00 &  0.00\\
instance n=20 79.alb & 1 & 1 & Optimal &  0.25 & 3 &  3.00 &  0.00\\
instance n=20 8.alb & 1 & 1 & Optimal &  2.39 & 3 &  3.00 &  0.00\\
instance n=20 80.alb & 1 & 1 & Optimal &  3.51 & 3 &  3.00 &  0.00\\
instance n=20 81.alb & 1 & 1 & Optimal &  0.13 & 3 &  3.00 &  0.00\\
instance n=20 82.alb & 1 & 1 & Optimal & 11.66 & 4 &  4.00 &  0.00\\
instance n=20 83.alb & 1 & 1 & Optimal &  0.10 & 3 &  3.00 &  0.00\\
instance n=20 84.alb & 1 & 1 & Optimal &  0.12 & 3 &  3.00 &  0.00\\
instance n=20 85.alb & 1 & 1 & Optimal & 30.01 & 3 &  3.00 &  0.00\\
instance n=20 86.alb & 1 & 1 & Optimal & 18.29 & 3 &  3.00 &  0.00\\
instance n=20 87.alb & 1 & 1 & Optimal &  0.35 & 3 &  3.00 &  0.00\\
instance n=20 88.alb & 1 & 1 & Optimal &  8.43 & 3 &  3.00 &  0.00\\
instance n=20 89.alb & 1 & 1 & Optimal &  0.71 & 3 &  3.00 &  0.00\\
instance n=20 9.alb & 1 & 1 & Optimal &  0.73 & 3 &  3.00 &  0.00\\
instance n=20 90.alb & 1 & 1 & Optimal & 30.02 & 3 &  3.00 &  0.00\\
instance n=20 91.alb & 1 & 1 & Optimal & 30.01 & 11 & 11.00 &  0.00\\
instance n=20 92.alb & 1 & 1 & Optimal & 30.04 & 11 & 11.00 &  0.00\\
instance n=20 93.alb & 1 & 1 & Optimal & 30.02 & 13 & 13.00 &  0.00\\
instance n=20 94.alb & 1 & 1 & Optimal & 30.01 & 10 & 10.00 &  0.00\\
instance n=20 95.alb & 1 & 1 & Optimal & 30.01 & 12 & 12.00 &  0.00\\
instance n=20 96.alb & 1 & 1 & Optimal & 30.01 & 10 & 10.00 &  0.00\\
instance n=20 97.alb & 1 & 1 & Optimal & 30.01 & 15 & 15.00 &  0.00\\
instance n=20 98.alb & 1 & 1 & Optimal & 30.02 & 13 & 13.00 &  0.00\\
instance n=20 99.alb & 1 & 1 & Optimal & 30.02 & 12 & 12.00 &  0.00\\
instance n=50 1.alb & 1 & 1 & Solution & 30.05 & 8 &  8.00 &  0.00\\
instance n=50 10.alb & 1 & 1 & Optimal & 13.56 & 7 &  7.00 &  0.00\\
instance n=50 100.alb & 1 & 1 & Solution & 30.03 & 7 &  7.00 &  0.00\\
instance n=50 101.alb & 1 & 1 & Solution & 30.04 & 30 & 26.00 & 13.33\\
instance n=50 102.alb & 1 & 1 & Solution & 30.03 & 32 & 27.00 & 15.63\\
instance n=50 103.alb & 1 & 1 & Solution & 30.04 & 29 & 25.00 & 13.79\\
instance n=50 104.alb & 1 & 1 & Solution & 30.03 & 28 & 25.00 & 10.71\\
instance n=50 105.alb & 1 & 1 & Solution & 30.04 & 24 & 23.00 &  4.17\\
instance n=50 106.alb & 1 & 1 & Solution & 30.03 & 28 & 26.00 &  7.14\\
instance n=50 107.alb & 1 & 1 & Solution & 30.04 & 28 & 26.00 &  7.14\\
instance n=50 108.alb & 1 & 1 & Solution & 30.05 & 31 & 26.00 & 16.13\\
instance n=50 109.alb & 1 & 1 & Solution & 30.02 & 30 & 25.00 & 16.67\\
instance n=50 11.alb & 1 & 1 & Solution & 30.05 & 7 &  7.00 &  0.00\\
instance n=50 110.alb & 1 & 1 & Solution & 30.06 & 26 & 25.00 &  3.85\\
instance n=50 111.alb & 1 & 1 & Solution & 30.04 & 28 & 26.00 &  7.14\\
instance n=50 112.alb & 1 & 1 & Solution & 30.04 & 27 & 25.00 &  7.41\\
instance n=50 113.alb & 1 & 1 & Solution & 30.04 & 28 & 26.00 &  7.14\\
instance n=50 114.alb & 1 & 1 & Solution & 30.07 & 28 & 25.00 & 10.71\\
instance n=50 115.alb & 1 & 1 & Solution & 30.04 & 29 & 25.00 & 13.79\\
instance n=50 116.alb & 1 & 1 & Solution & 30.06 & 32 & 27.00 & 15.63\\
instance n=50 117.alb & 1 & 1 & Solution & 30.05 & 27 & 25.00 &  7.41\\
instance n=50 118.alb & 1 & 1 & Solution & 30.03 & 29 & 27.00 &  6.90\\
instance n=50 119.alb & 1 & 1 & Solution & 30.04 & 25 & 25.00 &  0.00\\
instance n=50 12.alb & 1 & 1 & Optimal &  1.83 & 6 &  6.00 &  0.00\\
instance n=50 120.alb & 1 & 1 & Solution & 30.03 & 28 & 26.00 &  7.14\\
instance n=50 121.alb & 1 & 1 & Solution & 30.06 & 32 & 27.00 & 15.63\\
instance n=50 122.alb & 1 & 1 & Solution & 30.04 & 30 & 26.00 & 13.33\\
instance n=50 123.alb & 1 & 1 & Solution & 30.05 & 32 & 26.00 & 18.75\\
instance n=50 124.alb & 1 & 1 & Solution & 30.03 & 29 & 26.00 & 10.34\\
instance n=50 125.alb & 1 & 1 & Solution & 30.03 & 33 & 26.00 & 21.21\\
instance n=50 126.alb & 1 & 1 & Optimal & 30.01 & 12 & 12.00 &  0.00\\
instance n=50 127.alb & 1 & 1 & Solution & 30.03 & 14 & 14.00 &  0.00\\
instance n=50 128.alb & 1 & 1 & Solution & 30.05 & 12 & 12.00 &  0.00\\
instance n=50 129.alb & 1 & 1 & Solution & 30.06 & 13 & 13.00 &  0.00\\
instance n=50 13.alb & 1 & 1 & Optimal & 21.59 & 6 &  6.00 &  0.00\\
instance n=50 130.alb & 1 & 1 & Solution & 30.03 & 13 & 13.00 &  0.00\\
instance n=50 131.alb & 1 & 1 & Solution & 30.04 & 12 & 12.00 &  0.00\\
instance n=50 132.alb & 1 & 1 & Solution & 30.04 & 12 & 12.00 &  0.00\\
instance n=50 133.alb & 1 & 1 & Solution & 30.04 & 12 & 12.00 &  0.00\\
instance n=50 134.alb & 1 & 1 & Solution & 30.05 & 14 & 14.00 &  0.00\\
instance n=50 135.alb & 1 & 1 & Solution & 30.03 & 13 & 13.00 &  0.00\\
instance n=50 136.alb & 1 & 1 & Solution & 30.06 & 11 & 11.00 &  0.00\\
instance n=50 137.alb & 1 & 1 & Solution & 30.05 & 11 & 11.00 &  0.00\\
instance n=50 138.alb & 1 & 1 & Solution & 30.04 & 12 & 12.00 &  0.00\\
instance n=50 139.alb & 1 & 1 & Solution & 30.05 & 12 & 11.00 &  8.33\\
instance n=50 14.alb & 1 & 1 & Solution & 30.06 & 7 &  7.00 &  0.00\\
instance n=50 140.alb & 1 & 1 & Solution & 30.04 & 12 & 12.00 &  0.00\\
instance n=50 141.alb & 1 & 1 & Solution & 30.04 & 13 & 13.00 &  0.00\\
instance n=50 142.alb & 1 & 1 & Solution & 30.04 & 11 & 11.00 &  0.00\\
instance n=50 143.alb & 1 & 1 & Solution & 30.04 & 12 & 12.00 &  0.00\\
instance n=50 144.alb & 1 & 1 & Solution & 30.03 & 13 & 13.00 &  0.00\\
instance n=50 145.alb & 1 & 1 & Solution & 30.03 & 10 & 10.00 &  0.00\\
instance n=50 146.alb & 1 & 1 & Solution & 30.03 & 13 & 13.00 &  0.00\\
instance n=50 147.alb & 1 & 1 & Solution & 30.04 & 13 & 13.00 &  0.00\\
instance n=50 148.alb & 1 & 1 & Solution & 30.06 & 10 & 10.00 &  0.00\\
instance n=50 149.alb & 1 & 1 & Solution & 30.06 & 12 & 12.00 &  0.00\\
instance n=50 15.alb & 1 & 1 & Solution & 30.03 & 8 &  8.00 &  0.00\\
instance n=50 150.alb & 1 & 1 & Solution & 30.03 & 11 & 11.00 &  0.00\\
instance n=50 151.alb & 1 & 1 & Solution & 30.06 & 7 &  7.00 &  0.00\\
instance n=50 152.alb & 1 & 1 & Solution & 30.04 & 7 &  7.00 &  0.00\\
instance n=50 153.alb & 1 & 1 & Solution & 30.03 & 7 &  7.00 &  0.00\\
instance n=50 154.alb & 1 & 1 & Solution & 30.05 & 8 &  8.00 &  0.00\\
instance n=50 155.alb & 1 & 1 & Solution & 30.05 & 7 &  7.00 &  0.00\\
instance n=50 156.alb & 1 & 1 & Solution & 30.05 & 7 &  7.00 &  0.00\\
instance n=50 157.alb & 1 & 1 & Optimal & 30.02 & 8 &  8.00 &  0.00\\
instance n=50 158.alb & 1 & 1 & Solution & 30.06 & 7 &  7.00 &  0.00\\
instance n=50 159.alb & 1 & 1 & Solution & 30.07 & 7 &  7.00 &  0.00\\
instance n=50 16.alb & 1 & 1 & Solution & 30.04 & 8 &  8.00 &  0.00\\
instance n=50 160.alb & 1 & 1 & Solution & 30.04 & 8 &  8.00 &  0.00\\
instance n=50 161.alb & 1 & 1 & Optimal & 30.02 & 7 &  7.00 &  0.00\\
instance n=50 162.alb & 1 & 1 & Solution & 30.04 & 8 &  8.00 &  0.00\\
instance n=50 163.alb & 1 & 1 & Solution & 30.06 & 7 &  7.00 &  0.00\\
instance n=50 164.alb & 1 & 1 & Solution & 30.05 & 7 &  7.00 &  0.00\\
instance n=50 165.alb & 1 & 1 & Solution & 30.04 & 8 &  8.00 &  0.00\\
instance n=50 166.alb & 1 & 1 & Solution & 30.06 & 8 &  8.00 &  0.00\\
instance n=50 167.alb & 1 & 1 & Solution & 30.03 & 7 &  7.00 &  0.00\\
instance n=50 168.alb & 1 & 1 & Solution & 30.04 & 8 &  8.00 &  0.00\\
instance n=50 169.alb & 1 & 1 & Solution & 30.05 & 8 &  8.00 &  0.00\\
instance n=50 17.alb & 1 & 1 & Optimal & 23.10 & 7 &  7.00 &  0.00\\
instance n=50 170.alb & 1 & 1 & Solution & 30.05 & 7 &  7.00 &  0.00\\
instance n=50 171.alb & 1 & 1 & Solution & 30.05 & 8 &  8.00 &  0.00\\
instance n=50 172.alb & 1 & 1 & Optimal & 30.02 & 7 &  7.00 &  0.00\\
instance n=50 173.alb & 1 & 1 & Solution & 30.04 & 7 &  7.00 &  0.00\\
instance n=50 174.alb & 1 & 1 & Solution & 30.06 & 7 &  7.00 &  0.00\\
instance n=50 175.alb & 1 & 1 & Solution & 30.06 & 7 &  7.00 &  0.00\\
instance n=50 176.alb & 1 & 1 & Solution & 30.07 & 27 & 25.00 &  7.41\\
instance n=50 177.alb & 1 & 1 & Solution & 30.05 & 28 & 26.00 &  7.14\\
instance n=50 178.alb & 1 & 1 & Solution & 30.06 & 28 & 25.00 & 10.71\\
instance n=50 179.alb & 1 & 1 & Solution & 30.05 & 27 & 25.00 &  7.41\\
instance n=50 18.alb & 1 & 1 & Solution & 30.07 & 7 &  7.00 &  0.00\\
instance n=50 180.alb & 1 & 1 & Solution & 30.03 & 26 & 25.00 &  3.85\\
instance n=50 181.alb & 1 & 1 & Solution & 30.06 & 30 & 26.00 & 13.33\\
instance n=50 182.alb & 1 & 1 & Solution & 30.07 & 27 & 25.00 &  7.41\\
instance n=50 183.alb & 1 & 1 & Solution & 30.07 & 29 & 26.00 & 10.34\\
instance n=50 184.alb & 1 & 1 & Solution & 30.05 & 38 & 28.00 & 26.32\\
instance n=50 185.alb & 1 & 1 & Solution & 30.04 & 27 & 25.00 &  7.41\\
instance n=50 186.alb & 1 & 1 & Solution & 30.05 & 28 & 25.00 & 10.71\\
instance n=50 187.alb & 1 & 1 & Solution & 30.07 & 26 & 25.00 &  3.85\\
instance n=50 188.alb & 1 & 1 & Solution & 30.04 & 25 & 24.00 &  4.00\\
instance n=50 189.alb & 1 & 1 & Solution & 30.04 & 26 & 25.00 &  3.85\\
instance n=50 19.alb & 1 & 1 & Solution & 30.05 & 8 &  8.00 &  0.00\\
instance n=50 190.alb & 1 & 1 & Solution & 30.05 & 30 & 26.00 & 13.33\\
instance n=50 191.alb & 1 & 1 & Solution & 30.04 & 28 & 26.00 &  7.14\\
instance n=50 192.alb & 1 & 1 & Solution & 30.04 & 27 & 26.00 &  3.70\\
instance n=50 193.alb & 1 & 1 & Solution & 30.03 & 29 & 26.00 & 10.34\\
instance n=50 194.alb & 1 & 1 & Solution & 30.04 & 28 & 26.00 &  7.14\\
instance n=50 195.alb & 1 & 1 & Solution & 30.04 & 28 & 26.00 &  7.14\\
instance n=50 196.alb & 1 & 1 & Solution & 30.04 & 29 & 26.00 & 10.34\\
instance n=50 197.alb & 1 & 1 & Solution & 30.05 & 28 & 26.00 &  7.14\\
instance n=50 198.alb & 1 & 1 & Solution & 30.07 & 28 & 25.00 & 10.71\\
instance n=50 199.alb & 1 & 1 & Solution & 30.06 & 29 & 26.00 & 10.34\\
instance n=50 2.alb & 1 & 1 & Optimal & 30.02 & 6 &  6.00 &  0.00\\
instance n=50 20.alb & 1 & 1 & Solution & 30.04 & 8 &  8.00 &  0.00\\
instance n=50 200.alb & 1 & 1 & Solution & 30.04 & 25 & 24.00 &  4.00\\
instance n=50 201.alb & 1 & 1 & Solution & 30.07 & 13 & 13.00 &  0.00\\
instance n=50 202.alb & 1 & 1 & Solution & 30.08 & 9 &  9.00 &  0.00\\
instance n=50 203.alb & 1 & 1 & Solution & 30.07 & 11 & 11.00 &  0.00\\
instance n=50 204.alb & 1 & 1 & Solution & 30.04 & 10 & 10.00 &  0.00\\
instance n=50 205.alb & 1 & 1 & Solution & 30.05 & 13 & 13.00 &  0.00\\
instance n=50 206.alb & 1 & 1 & Solution & 30.04 & 12 & 11.00 &  8.33\\
instance n=50 207.alb & 1 & 1 & Solution & 30.05 & 10 & 10.00 &  0.00\\
instance n=50 208.alb & 1 & 1 & Solution & 30.05 & 13 & 13.00 &  0.00\\
instance n=50 209.alb & 1 & 1 & Solution & 30.05 & 11 & 11.00 &  0.00\\
instance n=50 21.alb & 1 & 1 & Optimal & 30.02 & 6 &  6.00 &  0.00\\
instance n=50 210.alb & 1 & 1 & Solution & 30.02 & 13 & 13.00 &  0.00\\
instance n=50 211.alb & 1 & 1 & Solution & 30.07 & 12 & 12.00 &  0.00\\
instance n=50 212.alb & 1 & 1 & Solution & 30.06 & 10 & 10.00 &  0.00\\
instance n=50 213.alb & 1 & 1 & Solution & 30.03 & 13 & 13.00 &  0.00\\
instance n=50 214.alb & 1 & 1 & Solution & 30.04 & 11 & 11.00 &  0.00\\
instance n=50 215.alb & 1 & 1 & Solution & 30.04 & 11 & 11.00 &  0.00\\
instance n=50 216.alb & 1 & 1 & Solution & 30.03 & 12 & 12.00 &  0.00\\
instance n=50 217.alb & 1 & 1 & Solution & 30.04 & 13 & 13.00 &  0.00\\
instance n=50 218.alb & 1 & 1 & Solution & 30.05 & 12 & 12.00 &  0.00\\
instance n=50 219.alb & 1 & 1 & Solution & 30.04 & 11 & 11.00 &  0.00\\
instance n=50 22.alb & 1 & 1 & Solution & 30.05 & 7 &  7.00 &  0.00\\
instance n=50 220.alb & 1 & 1 & Solution & 30.04 & 11 & 11.00 &  0.00\\
instance n=50 221.alb & 1 & 1 & Solution & 30.04 & 11 & 11.00 &  0.00\\
instance n=50 222.alb & 1 & 1 & Solution & 30.06 & 14 & 14.00 &  0.00\\
instance n=50 223.alb & 1 & 1 & Solution & 30.03 & 11 & 11.00 &  0.00\\
instance n=50 224.alb & 1 & 1 & Solution & 30.07 & 11 & 11.00 &  0.00\\
instance n=50 225.alb & 1 & 1 & Optimal & 30.02 & 12 & 12.00 &  0.00\\
instance n=50 226.alb & 1 & 1 & Optimal & 11.79 & 7 &  7.00 &  0.00\\
instance n=50 227.alb & 1 & 1 & Optimal & 30.00 & 6 &  6.00 &  0.00\\
instance n=50 228.alb & 1 & 1 & Optimal & 10.45 & 6 &  6.00 &  0.00\\
instance n=50 229.alb & 1 & 1 & Solution & 30.06 & 6 &  6.00 &  0.00\\
instance n=50 23.alb & 1 & 1 & Solution & 30.08 & 7 &  7.00 &  0.00\\
instance n=50 230.alb & 1 & 1 & Optimal & 30.02 & 7 &  7.00 &  0.00\\
instance n=50 231.alb & 1 & 1 & Optimal & 30.01 & 7 &  7.00 &  0.00\\
instance n=50 232.alb & 1 & 1 & Solution & 30.04 & 7 &  7.00 &  0.00\\
instance n=50 233.alb & 1 & 1 & Optimal & 30.02 & 6 &  6.00 &  0.00\\
instance n=50 234.alb & 1 & 1 & Optimal & 14.43 & 8 &  8.00 &  0.00\\
instance n=50 235.alb & 1 & 1 & Optimal & 23.91 & 7 &  7.00 &  0.00\\
instance n=50 236.alb & 1 & 1 & Solution & 30.06 & 7 &  7.00 &  0.00\\
instance n=50 237.alb & 1 & 1 & Solution & 30.05 & 8 &  8.00 &  0.00\\
instance n=50 238.alb & 1 & 1 & Optimal & 30.01 & 7 &  7.00 &  0.00\\
instance n=50 239.alb & 1 & 1 & Solution & 30.04 & 7 &  7.00 &  0.00\\
instance n=50 24.alb & 1 & 1 & Optimal & 30.02 & 7 &  7.00 &  0.00\\
instance n=50 240.alb & 1 & 1 & Solution & 30.04 & 7 &  7.00 &  0.00\\
instance n=50 241.alb & 1 & 1 & Solution & 30.07 & 7 &  7.00 &  0.00\\
instance n=50 242.alb & 1 & 1 & Optimal & 30.03 & 8 &  8.00 &  0.00\\
instance n=50 243.alb & 1 & 1 & Optimal & 30.01 & 7 &  7.00 &  0.00\\
instance n=50 244.alb & 1 & 1 & Optimal & 30.01 & 7 &  7.00 &  0.00\\
instance n=50 245.alb & 1 & 1 & Optimal & 30.01 & 7 &  7.00 &  0.00\\
instance n=50 246.alb & 1 & 1 & Solution & 30.05 & 8 &  8.00 &  0.00\\
instance n=50 247.alb & 1 & 1 & Optimal & 30.01 & 7 &  7.00 &  0.00\\
instance n=50 248.alb & 1 & 1 & Solution & 30.04 & 7 &  7.00 &  0.00\\
instance n=50 249.alb & 1 & 1 & Solution & 30.05 & 7 &  7.00 &  0.00\\
instance n=50 25.alb & 1 & 1 & Optimal &  6.31 & 6 &  6.00 &  0.00\\
instance n=50 250.alb & 1 & 1 & Optimal & 26.33 & 7 &  7.00 &  0.00\\
instance n=50 251.alb & 1 & 1 & Solution & 30.04 & 28 & 25.00 & 10.71\\
instance n=50 252.alb & 1 & 1 & Solution & 30.09 & 32 & 27.00 & 15.63\\
instance n=50 253.alb & 1 & 1 & Solution & 30.07 & 29 & 26.00 & 10.34\\
instance n=50 254.alb & 1 & 1 & Solution & 30.03 & 30 & 26.00 & 13.33\\
instance n=50 255.alb & 1 & 1 & Solution & 30.03 & 30 & 26.00 & 13.33\\
instance n=50 256.alb & 1 & 1 & Solution & 30.05 & 31 & 27.00 & 12.90\\
instance n=50 257.alb & 1 & 1 & Solution & 30.05 & 33 & 28.00 & 15.15\\
instance n=50 258.alb & 1 & 1 & Solution & 30.03 & 28 & 25.00 & 10.71\\
instance n=50 259.alb & 1 & 1 & Solution & 30.04 & 31 & 26.00 & 16.13\\
instance n=50 26.alb & 1 & 1 & Solution & 30.05 & 27 & 25.00 &  7.41\\
instance n=50 260.alb & 1 & 1 & Solution & 30.04 & 29 & 26.00 & 10.34\\
instance n=50 261.alb & 1 & 1 & Solution & 30.03 & 28 & 25.00 & 10.71\\
instance n=50 262.alb & 1 & 1 & Solution & 30.06 & 31 & 26.00 & 16.13\\
instance n=50 263.alb & 1 & 1 & Solution & 30.05 & 31 & 26.00 & 16.13\\
instance n=50 264.alb & 1 & 1 & Solution & 30.05 & 28 & 25.00 & 10.71\\
instance n=50 265.alb & 1 & 1 & Solution & 30.03 & 27 & 25.00 &  7.41\\
instance n=50 266.alb & 1 & 1 & Solution & 30.02 & 30 & 26.00 & 13.33\\
instance n=50 267.alb & 1 & 1 & Solution & 30.03 & 28 & 26.00 &  7.14\\
instance n=50 268.alb & 1 & 1 & Solution & 30.04 & 29 & 26.00 & 10.34\\
instance n=50 269.alb & 1 & 1 & Solution & 30.03 & 26 & 25.00 &  3.85\\
instance n=50 27.alb & 1 & 1 & Solution & 30.05 & 30 & 27.00 & 10.00\\
instance n=50 270.alb & 1 & 1 & Solution & 30.03 & 28 & 26.00 &  7.14\\
instance n=50 271.alb & 1 & 1 & Solution & 30.04 & 31 & 26.00 & 16.13\\
instance n=50 272.alb & 1 & 1 & Solution & 30.04 & 27 & 25.00 &  7.41\\
instance n=50 273.alb & 1 & 1 & Solution & 30.03 & 28 & 25.00 & 10.71\\
instance n=50 274.alb & 1 & 1 & Solution & 30.04 & 30 & 26.00 & 13.33\\
instance n=50 275.alb & 1 & 1 & Solution & 30.03 & 28 & 26.00 &  7.14\\
instance n=50 276.alb & 1 & 1 & Solution & 30.04 & 12 & 12.00 &  0.00\\
instance n=50 277.alb & 1 & 1 & Optimal & 30.01 & 13 & 13.00 &  0.00\\
instance n=50 278.alb & 1 & 1 & Solution & 30.04 & 12 & 12.00 &  0.00\\
instance n=50 279.alb & 1 & 1 & Solution & 30.05 & 11 & 11.00 &  0.00\\
instance n=50 28.alb & 1 & 1 & Solution & 30.07 & 28 & 26.00 &  7.14\\
instance n=50 280.alb & 1 & 1 & Solution & 30.06 & 13 & 13.00 &  0.00\\
instance n=50 281.alb & 1 & 1 & Optimal & 30.02 & 11 & 11.00 &  0.00\\
instance n=50 282.alb & 1 & 1 & Solution & 30.04 & 12 & 11.00 &  8.33\\
instance n=50 283.alb & 1 & 1 & Solution & 30.05 & 12 & 12.00 &  0.00\\
instance n=50 284.alb & 1 & 1 & Optimal & 30.01 & 11 & 11.00 &  0.00\\
instance n=50 285.alb & 1 & 1 & Optimal & 30.01 & 13 & 13.00 &  0.00\\
instance n=50 286.alb & 1 & 1 & Solution & 30.05 & 11 & 11.00 &  0.00\\
instance n=50 287.alb & 1 & 1 & Solution & 30.05 & 12 & 12.00 &  0.00\\
instance n=50 288.alb & 1 & 1 & Optimal & 30.02 & 10 & 10.00 &  0.00\\
instance n=50 289.alb & 1 & 1 & Solution & 30.04 & 11 & 11.00 &  0.00\\
instance n=50 29.alb & 1 & 1 & Solution & 30.07 & 29 & 25.00 & 13.79\\
instance n=50 290.alb & 1 & 1 & Solution & 30.05 & 14 & 14.00 &  0.00\\
instance n=50 291.alb & 1 & 1 & Solution & 30.04 & 12 & 12.00 &  0.00\\
instance n=50 292.alb & 1 & 1 & Solution & 30.02 & 13 & 13.00 &  0.00\\
instance n=50 293.alb & 1 & 1 & Solution & 30.03 & 12 & 12.00 &  0.00\\
instance n=50 294.alb & 1 & 1 & Solution & 30.03 & 13 & 13.00 &  0.00\\
instance n=50 295.alb & 1 & 1 & Solution & 30.04 & 16 & 16.00 &  0.00\\
instance n=50 296.alb & 1 & 1 & Solution & 30.02 & 13 & 13.00 &  0.00\\
instance n=50 297.alb & 1 & 1 & Solution & 30.06 & 13 & 13.00 &  0.00\\
instance n=50 298.alb & 1 & 1 & Solution & 30.05 & 11 & 11.00 &  0.00\\
instance n=50 299.alb & 1 & 1 & Solution & 30.04 & 12 & 11.00 &  8.33\\
instance n=50 3.alb & 1 & 1 & Solution & 30.06 & 8 &  8.00 &  0.00\\
instance n=50 30.alb & 1 & 1 & Solution & 30.05 & 28 & 25.00 & 10.71\\
instance n=50 300.alb & 1 & 1 & Optimal & 30.01 & 12 & 12.00 &  0.00\\
instance n=50 301.alb & 1 & 1 & Optimal &  5.64 & 6 &  6.00 &  0.00\\
instance n=50 302.alb & 1 & 1 & Solution & 30.05 & 7 &  7.00 &  0.00\\
instance n=50 303.alb & 1 & 1 & Solution & 30.03 & 8 &  8.00 &  0.00\\
instance n=50 304.alb & 1 & 1 & Optimal & 11.04 & 7 &  7.00 &  0.00\\
instance n=50 305.alb & 1 & 1 & Solution & 30.04 & 8 &  8.00 &  0.00\\
instance n=50 306.alb & 1 & 1 & Optimal & 30.01 & 7 &  7.00 &  0.00\\
instance n=50 307.alb & 1 & 1 & Solution & 30.05 & 7 &  7.00 &  0.00\\
instance n=50 308.alb & 1 & 1 & Solution & 30.07 & 8 &  8.00 &  0.00\\
instance n=50 309.alb & 1 & 1 & Optimal & 29.06 & 7 &  7.00 &  0.00\\
instance n=50 31.alb & 1 & 1 & Solution & 30.06 & 28 & 25.00 & 10.71\\
instance n=50 310.alb & 1 & 1 & Solution & 30.04 & 8 &  8.00 &  0.00\\
instance n=50 311.alb & 1 & 1 & Solution & 30.05 & 8 &  8.00 &  0.00\\
instance n=50 312.alb & 1 & 1 & Optimal &  5.25 & 6 &  6.00 &  0.00\\
instance n=50 313.alb & 1 & 1 & Solution & 30.06 & 8 &  8.00 &  0.00\\
instance n=50 314.alb & 1 & 1 & Optimal & 11.91 & 7 &  7.00 &  0.00\\
instance n=50 315.alb & 1 & 1 & Solution & 30.03 & 8 &  8.00 &  0.00\\
instance n=50 316.alb & 1 & 1 & Solution & 30.04 & 8 &  8.00 &  0.00\\
instance n=50 317.alb & 1 & 1 & Optimal & 20.23 & 6 &  6.00 &  0.00\\
instance n=50 32.alb & 1 & 1 & Solution & 30.06 & 26 & 25.00 &  3.85\\
instance n=50 33.alb & 1 & 1 & Solution & 30.06 & 25 & 24.00 &  4.00\\
instance n=50 34.alb & 1 & 1 & Solution & 30.05 & 30 & 26.00 & 13.33\\
instance n=50 35.alb & 1 & 1 & Solution & 30.07 & 33 & 27.00 & 18.18\\
instance n=50 36.alb & 1 & 1 & Solution & 30.04 & 31 & 26.00 & 16.13\\
instance n=50 37.alb & 1 & 1 & Solution & 30.06 & 32 & 27.00 & 15.63\\
instance n=50 38.alb & 1 & 1 & Solution & 30.04 & 31 & 27.00 & 12.90\\
instance n=50 39.alb & 1 & 1 & Solution & 30.08 & 29 & 26.00 & 10.34\\
instance n=50 4.alb & 1 & 1 & Optimal & 30.02 & 7 &  7.00 &  0.00\\
instance n=50 40.alb & 1 & 1 & Solution & 30.05 & 27 & 25.00 &  7.41\\
instance n=50 41.alb & 1 & 1 & Solution & 30.04 & 26 & 25.00 &  3.85\\
instance n=50 42.alb & 1 & 1 & Solution & 30.05 & 24 & 23.00 &  4.17\\
instance n=50 43.alb & 1 & 1 & Solution & 30.03 & 26 & 25.00 &  3.85\\
instance n=50 44.alb & 1 & 1 & Solution & 30.05 & 25 & 24.00 &  4.00\\
instance n=50 45.alb & 1 & 1 & Solution & 30.03 & 25 & 24.00 &  4.00\\
instance n=50 46.alb & 1 & 1 & Solution & 30.05 & 29 & 26.00 & 10.34\\
instance n=50 47.alb & 1 & 1 & Solution & 30.07 & 28 & 26.00 &  7.14\\
instance n=50 48.alb & 1 & 1 & Solution & 30.05 & 28 & 26.00 &  7.14\\
instance n=50 49.alb & 1 & 1 & Solution & 30.03 & 25 & 24.00 &  4.00\\
instance n=50 5.alb & 1 & 1 & Solution & 30.07 & 7 &  7.00 &  0.00\\
instance n=50 50.alb & 1 & 1 & Solution & 30.03 & 27 & 25.00 &  7.41\\
instance n=50 51.alb & 1 & 1 & Solution & 30.07 & 12 & 12.00 &  0.00\\
instance n=50 52.alb & 1 & 1 & Solution & 30.04 & 11 & 11.00 &  0.00\\
instance n=50 53.alb & 1 & 1 & Solution & 30.05 & 13 & 12.00 &  7.69\\
instance n=50 54.alb & 1 & 1 & Solution & 30.04 & 11 & 11.00 &  0.00\\
instance n=50 55.alb & 1 & 1 & Solution & 30.09 & 13 & 13.00 &  0.00\\
instance n=50 56.alb & 1 & 1 & Solution & 30.06 & 11 & 11.00 &  0.00\\
instance n=50 57.alb & 1 & 1 & Solution & 30.03 & 13 & 13.00 &  0.00\\
instance n=50 58.alb & 1 & 1 & Solution & 30.06 & 11 & 11.00 &  0.00\\
instance n=50 59.alb & 1 & 1 & Solution & 30.03 & 11 & 11.00 &  0.00\\
instance n=50 6.alb & 1 & 1 & Optimal & 18.59 & 6 &  6.00 &  0.00\\
instance n=50 60.alb & 1 & 1 & Solution & 30.05 & 12 & 12.00 &  0.00\\
instance n=50 61.alb & 1 & 1 & Solution & 30.05 & 13 & 13.00 &  0.00\\
instance n=50 62.alb & 1 & 1 & Solution & 30.05 & 13 & 13.00 &  0.00\\
instance n=50 63.alb & 1 & 1 & Solution & 30.04 & 12 & 12.00 &  0.00\\
instance n=50 64.alb & 1 & 1 & Solution & 30.06 & 13 & 13.00 &  0.00\\
instance n=50 65.alb & 1 & 1 & Solution & 30.06 & 12 & 12.00 &  0.00\\
instance n=50 66.alb & 1 & 1 & Solution & 30.05 & 12 & 12.00 &  0.00\\
instance n=50 67.alb & 1 & 1 & Solution & 30.05 & 12 & 12.00 &  0.00\\
instance n=50 68.alb & 1 & 1 & Solution & 30.04 & 12 & 12.00 &  0.00\\
instance n=50 69.alb & 1 & 1 & Solution & 30.02 & 12 & 12.00 &  0.00\\
instance n=50 7.alb & 1 & 1 & Solution & 30.08 & 7 &  7.00 &  0.00\\
instance n=50 70.alb & 1 & 1 & Solution & 30.05 & 10 & 10.00 &  0.00\\
instance n=50 71.alb & 1 & 1 & Solution & 30.06 & 13 & 13.00 &  0.00\\
instance n=50 72.alb & 1 & 1 & Solution & 30.04 & 11 & 11.00 &  0.00\\
instance n=50 73.alb & 1 & 1 & Solution & 30.04 & 11 & 11.00 &  0.00\\
instance n=50 74.alb & 1 & 1 & Solution & 30.06 & 12 & 12.00 &  0.00\\
instance n=50 75.alb & 1 & 1 & Solution & 30.05 & 11 & 11.00 &  0.00\\
instance n=50 76.alb & 1 & 1 & Optimal & 30.02 & 7 &  7.00 &  0.00\\
instance n=50 77.alb & 1 & 1 & Optimal & 30.02 & 7 &  7.00 &  0.00\\
instance n=50 78.alb & 1 & 1 & Optimal & 30.01 & 7 &  7.00 &  0.00\\
instance n=50 79.alb & 1 & 1 & Solution & 30.04 & 8 &  8.00 &  0.00\\
instance n=50 8.alb & 1 & 1 & Optimal & 17.93 & 7 &  7.00 &  0.00\\
instance n=50 80.alb & 1 & 1 & Optimal & 30.01 & 7 &  7.00 &  0.00\\
instance n=50 81.alb & 1 & 1 & Solution & 30.04 & 7 &  7.00 &  0.00\\
instance n=50 82.alb & 1 & 1 & Optimal & 30.01 & 6 &  6.00 &  0.00\\
instance n=50 83.alb & 1 & 1 & Solution & 30.04 & 8 &  8.00 &  0.00\\
instance n=50 84.alb & 1 & 1 & Optimal & 30.01 & 7 &  7.00 &  0.00\\
instance n=50 85.alb & 1 & 1 & Optimal & 30.01 & 8 &  8.00 &  0.00\\
instance n=50 86.alb & 1 & 1 & Solution & 30.05 & 7 &  7.00 &  0.00\\
instance n=50 87.alb & 1 & 1 & Solution & 30.02 & 8 &  8.00 &  0.00\\
instance n=50 88.alb & 1 & 1 & Optimal & 30.02 & 8 &  8.00 &  0.00\\
instance n=50 89.alb & 1 & 1 & Optimal & 30.01 & 7 &  7.00 &  0.00\\
instance n=50 9.alb & 1 & 1 & Solution & 30.05 & 9 &  9.00 &  0.00\\
instance n=50 90.alb & 1 & 1 & Solution & 30.04 & 7 &  7.00 &  0.00\\
instance n=50 91.alb & 1 & 1 & Solution & 30.05 & 7 &  7.00 &  0.00\\
instance n=50 92.alb & 1 & 1 & Solution & 30.06 & 7 &  7.00 &  0.00\\
instance n=50 93.alb & 1 & 1 & Solution & 30.04 & 7 &  7.00 &  0.00\\
instance n=50 94.alb & 1 & 1 & Solution & 30.05 & 7 &  7.00 &  0.00\\
instance n=50 95.alb & 1 & 1 & Optimal & 30.01 & 7 &  7.00 &  0.00\\
instance n=50 96.alb & 1 & 1 & Optimal & 30.01 & 7 &  7.00 &  0.00\\
instance n=50 97.alb & 1 & 1 & Optimal & 30.02 & 7 &  7.00 &  0.00\\
instance n=50 98.alb & 1 & 1 & Solution & 30.03 & 8 &  8.00 &  0.00\\
instance n=50 99.alb & 1 & 1 & Solution & 30.04 & 7 &  7.00 &  0.00\\
\end{longtable}



\subsection{Chuffed}

\begin{longtable}{lrrlrrrr}
\caption{Results for SALBP-1 Problems Alternative (Chuffed) (525 Instances)}\\\toprule
Name & \shortstack{Nr\\Jobs} & \shortstack{Nr\\Machines} & Status & Time & Makespan & Bound & \shortstack{Gap\\Percent}\\ \midrule
\endhead
\bottomrule
\endfoot
instance n=20 1.alb & 1 & 1 & Solution & 120.25 & 11 &  0.00 & 100.00\\
instance n=20 10.alb & 1 & 1 & Solution & 120.65 & 11 &  0.00 & 100.00\\
instance n=20 100.alb & 1 & 1 & Optimal & 21.66 & 11 &  0.00 & 100.00\\
instance n=20 101.alb & 1 & 1 & Optimal & 21.66 & 13 &  0.00 & 100.00\\
instance n=20 102.alb & 1 & 1 & Optimal & 15.92 & 13 &  0.00 & 100.00\\
instance n=20 103.alb & 1 & 1 & Optimal & 14.58 & 12 &  0.00 & 100.00\\
instance n=20 104.alb & 1 & 1 & Optimal & 38.11 & 11 &  0.00 & 100.00\\
instance n=20 105.alb & 1 & 1 & Optimal & 22.04 & 12 &  0.00 & 100.00\\
instance n=20 106.alb & 1 & 1 & Optimal & 39.82 & 10 &  0.00 & 100.00\\
instance n=20 107.alb & 1 & 1 & Optimal & 11.78 & 14 &  0.00 & 100.00\\
instance n=20 108.alb & 1 & 1 & Optimal & 12.38 & 15 &  0.00 & 100.00\\
instance n=20 109.alb & 1 & 1 & Optimal & 17.36 & 12 &  0.00 & 100.00\\
instance n=20 11.alb & 1 & 1 & Solution & 121.03 & 12 &  0.00 & 100.00\\
instance n=20 110.alb & 1 & 1 & Optimal & 30.37 & 11 &  0.00 & 100.00\\
instance n=20 111.alb & 1 & 1 & Optimal & 12.21 & 13 &  0.00 & 100.00\\
instance n=20 112.alb & 1 & 1 & Optimal & 29.89 & 11 &  0.00 & 100.00\\
instance n=20 113.alb & 1 & 1 & Optimal & 28.28 & 12 &  0.00 & 100.00\\
instance n=20 114.alb & 1 & 1 & Optimal & 17.37 & 13 &  0.00 & 100.00\\
instance n=20 115.alb & 1 & 1 & Optimal & 61.28 & 11 &  0.00 & 100.00\\
instance n=20 116.alb & 1 & 1 & Solution & 120.32 & 11 &  0.00 & 100.00\\
instance n=20 117.alb & 1 & 1 & Solution & 120.42 & 10 &  0.00 & 100.00\\
instance n=20 118.alb & 1 & 1 & Solution & 120.41 & 9 &  0.00 & 100.00\\
instance n=20 119.alb & 1 & 1 & Solution & 121.01 & 9 &  0.00 & 100.00\\
instance n=20 12.alb & 1 & 1 & Solution & 120.40 & 11 &  0.00 & 100.00\\
instance n=20 120.alb & 1 & 1 & Solution & 120.99 & 9 &  0.00 & 100.00\\
instance n=20 121.alb & 1 & 1 & Solution & 120.25 & 9 &  0.00 & 100.00\\
instance n=20 122.alb & 1 & 1 & Solution & 120.37 & 10 &  0.00 & 100.00\\
instance n=20 123.alb & 1 & 1 & Solution & 121.02 & 10 &  0.00 & 100.00\\
instance n=20 124.alb & 1 & 1 & Solution & 120.15 & 10 &  0.00 & 100.00\\
instance n=20 125.alb & 1 & 1 & Solution & 121.01 & 11 &  0.00 & 100.00\\
instance n=20 126.alb & 1 & 1 & Solution & 120.11 & 8 &  0.00 & 100.00\\
instance n=20 127.alb & 1 & 1 & Solution & 121.02 & 10 &  0.00 & 100.00\\
instance n=20 128.alb & 1 & 1 & Unknown & 121010.00 & - & - & -\\
instance n=20 129.alb & 1 & 1 & Solution & 121.01 & 10 &  0.00 & 100.00\\
instance n=20 13.alb & 1 & 1 & Solution & 120.22 & 11 &  0.00 & 100.00\\
instance n=20 130.alb & 1 & 1 & Solution & 120.62 & 7 &  0.00 & 100.00\\
instance n=20 131.alb & 1 & 1 & Optimal & 43.43 & 7 &  0.00 & 100.00\\
instance n=20 132.alb & 1 & 1 & Solution & 121.02 & 10 &  0.00 & 100.00\\
instance n=20 133.alb & 1 & 1 & Solution & 121.03 & 10 &  0.00 & 100.00\\
instance n=20 134.alb & 1 & 1 & Solution & 120.24 & 7 &  0.00 & 100.00\\
instance n=20 135.alb & 1 & 1 & Solution & 121.02 & 8 &  0.00 & 100.00\\
instance n=20 136.alb & 1 & 1 & Solution & 120.91 & 9 &  0.00 & 100.00\\
instance n=20 137.alb & 1 & 1 & Solution & 120.30 & 6 &  0.00 & 100.00\\
instance n=20 138.alb & 1 & 1 & Solution & 120.63 & 9 &  0.00 & 100.00\\
instance n=20 139.alb & 1 & 1 & Solution & 121.02 & 9 &  0.00 & 100.00\\
instance n=20 14.alb & 1 & 1 & Solution & 120.76 & 11 &  0.00 & 100.00\\
instance n=20 140.alb & 1 & 1 & Solution & 121.02 & 10 &  0.00 & 100.00\\
instance n=20 141.alb & 1 & 1 & Solution & 121.00 & 12 &  0.00 & 100.00\\
instance n=20 142.alb & 1 & 1 & Solution & 120.64 & 12 &  0.00 & 100.00\\
instance n=20 143.alb & 1 & 1 & Solution & 121.01 & 12 &  0.00 & 100.00\\
instance n=20 144.alb & 1 & 1 & Solution & 121.02 & 11 &  0.00 & 100.00\\
instance n=20 145.alb & 1 & 1 & Solution & 121.01 & 11 &  0.00 & 100.00\\
instance n=20 146.alb & 1 & 1 & Solution & 121.02 & 11 &  0.00 & 100.00\\
instance n=20 147.alb & 1 & 1 & Solution & 120.96 & 12 &  0.00 & 100.00\\
instance n=20 148.alb & 1 & 1 & Solution & 120.71 & 12 &  0.00 & 100.00\\
instance n=20 149.alb & 1 & 1 & Solution & 120.82 & 12 &  0.00 & 100.00\\
instance n=20 15.alb & 1 & 1 & Solution & 121.01 & 12 &  0.00 & 100.00\\
instance n=20 150.alb & 1 & 1 & Solution & 121.00 & 12 &  0.00 & 100.00\\
instance n=20 151.alb & 1 & 1 & Solution & 120.47 & 11 &  0.00 & 100.00\\
instance n=20 152.alb & 1 & 1 & Solution & 121.01 & 12 &  0.00 & 100.00\\
instance n=20 153.alb & 1 & 1 & Solution & 121.03 & 12 &  0.00 & 100.00\\
instance n=20 154.alb & 1 & 1 & Solution & 121.03 & 11 &  0.00 & 100.00\\
instance n=20 155.alb & 1 & 1 & Solution & 121.02 & 12 &  0.00 & 100.00\\
instance n=20 156.alb & 1 & 1 & Solution & 121.03 & 12 &  0.00 & 100.00\\
instance n=20 157.alb & 1 & 1 & Solution & 120.75 & 11 &  0.00 & 100.00\\
instance n=20 158.alb & 1 & 1 & Solution & 120.41 & 11 &  0.00 & 100.00\\
instance n=20 159.alb & 1 & 1 & Unknown & 121005.00 & - & - & -\\
instance n=20 16.alb & 1 & 1 & Optimal & 38.84 & 12 &  0.00 & 100.00\\
instance n=20 160.alb & 1 & 1 & Solution & 121.02 & 11 &  0.00 & 100.00\\
instance n=20 161.alb & 1 & 1 & Solution & 120.44 & 12 &  0.00 & 100.00\\
instance n=20 162.alb & 1 & 1 & Solution & 120.38 & 12 &  0.00 & 100.00\\
instance n=20 163.alb & 1 & 1 & Solution & 121.02 & 12 &  0.00 & 100.00\\
instance n=20 164.alb & 1 & 1 & Solution & 120.27 & 11 &  0.00 & 100.00\\
instance n=20 165.alb & 1 & 1 & Solution & 120.89 & 11 &  0.00 & 100.00\\
instance n=20 166.alb & 1 & 1 & Optimal & 49.10 & 12 &  0.00 & 100.00\\
instance n=20 167.alb & 1 & 1 & Optimal & 45.68 & 11 &  0.00 & 100.00\\
instance n=20 168.alb & 1 & 1 & Optimal & 35.69 & 10 &  0.00 & 100.00\\
instance n=20 169.alb & 1 & 1 & Optimal & 65.22 & 11 &  0.00 & 100.00\\
instance n=20 17.alb & 1 & 1 & Optimal & 58.20 & 10 &  0.00 & 100.00\\
instance n=20 170.alb & 1 & 1 & Optimal & 53.43 & 11 &  0.00 & 100.00\\
instance n=20 171.alb & 1 & 1 & Optimal & 43.65 & 13 &  0.00 & 100.00\\
instance n=20 172.alb & 1 & 1 & Optimal & 36.96 & 11 &  0.00 & 100.00\\
instance n=20 173.alb & 1 & 1 & Optimal & 47.24 & 11 &  0.00 & 100.00\\
instance n=20 174.alb & 1 & 1 & Optimal & 21.46 & 12 &  0.00 & 100.00\\
instance n=20 175.alb & 1 & 1 & Optimal & 55.40 & 10 &  0.00 & 100.00\\
instance n=20 176.alb & 1 & 1 & Optimal & 41.01 & 11 &  0.00 & 100.00\\
instance n=20 177.alb & 1 & 1 & Optimal & 58.07 & 10 &  0.00 & 100.00\\
instance n=20 178.alb & 1 & 1 & Optimal & 34.50 & 11 &  0.00 & 100.00\\
instance n=20 179.alb & 1 & 1 & Optimal & 38.36 & 11 &  0.00 & 100.00\\
instance n=20 18.alb & 1 & 1 & Optimal & 50.98 & 11 &  0.00 & 100.00\\
instance n=20 180.alb & 1 & 1 & Optimal & 30.12 & 13 &  0.00 & 100.00\\
instance n=20 181.alb & 1 & 1 & Optimal & 32.20 & 11 &  0.00 & 100.00\\
instance n=20 182.alb & 1 & 1 & Optimal & 50.42 & 11 &  0.00 & 100.00\\
instance n=20 183.alb & 1 & 1 & Optimal & 41.37 & 13 &  0.00 & 100.00\\
instance n=20 184.alb & 1 & 1 & Optimal & 20.30 & 12 &  0.00 & 100.00\\
instance n=20 185.alb & 1 & 1 & Optimal & 25.95 & 15 &  0.00 & 100.00\\
instance n=20 186.alb & 1 & 1 & Optimal & 45.42 & 14 &  0.00 & 100.00\\
instance n=20 187.alb & 1 & 1 & Optimal & 59.83 & 10 &  0.00 & 100.00\\
instance n=20 188.alb & 1 & 1 & Optimal & 47.31 & 11 &  0.00 & 100.00\\
instance n=20 189.alb & 1 & 1 & Optimal & 29.07 & 13 &  0.00 & 100.00\\
instance n=20 19.alb & 1 & 1 & Optimal & 42.87 & 14 &  0.00 & 100.00\\
instance n=20 190.alb & 1 & 1 & Optimal & 41.32 & 15 &  0.00 & 100.00\\
instance n=20 191.alb & 1 & 1 & Solution & 120.13 & 11 &  0.00 & 100.00\\
instance n=20 192.alb & 1 & 1 & Solution & 121.02 & 11 &  0.00 & 100.00\\
instance n=20 193.alb & 1 & 1 & Solution & 121.03 & 10 &  0.00 & 100.00\\
instance n=20 194.alb & 1 & 1 & Solution & 120.16 & 11 &  0.00 & 100.00\\
instance n=20 195.alb & 1 & 1 & Solution & 121.03 & 10 &  0.00 & 100.00\\
instance n=20 196.alb & 1 & 1 & Solution & 121.02 & 11 &  0.00 & 100.00\\
instance n=20 197.alb & 1 & 1 & Solution & 120.35 & 11 &  0.00 & 100.00\\
instance n=20 198.alb & 1 & 1 & Solution & 120.80 & 9 &  0.00 & 100.00\\
instance n=20 199.alb & 1 & 1 & Solution & 121.02 & 10 &  0.00 & 100.00\\
instance n=20 2.alb & 1 & 1 & Solution & 121.03 & 12 &  0.00 & 100.00\\
instance n=20 20.alb & 1 & 1 & Optimal & 36.08 & 11 &  0.00 & 100.00\\
instance n=20 200.alb & 1 & 1 & Solution & 120.11 & 10 &  0.00 & 100.00\\
instance n=20 201.alb & 1 & 1 & Solution & 120.30 & 11 &  0.00 & 100.00\\
instance n=20 202.alb & 1 & 1 & Solution & 120.19 & 11 &  0.00 & 100.00\\
instance n=20 203.alb & 1 & 1 & Solution & 120.87 & 12 &  0.00 & 100.00\\
instance n=20 204.alb & 1 & 1 & Solution & 120.99 & 11 &  0.00 & 100.00\\
instance n=20 205.alb & 1 & 1 & Solution & 120.64 & 11 &  0.00 & 100.00\\
instance n=20 206.alb & 1 & 1 & Solution & 121.02 & 10 &  0.00 & 100.00\\
instance n=20 207.alb & 1 & 1 & Solution & 120.13 & 10 &  0.00 & 100.00\\
instance n=20 208.alb & 1 & 1 & Solution & 120.79 & 10 &  0.00 & 100.00\\
instance n=20 209.alb & 1 & 1 & Solution & 120.67 & 12 &  0.00 & 100.00\\
instance n=20 21.alb & 1 & 1 & Optimal & 26.36 & 14 &  0.00 & 100.00\\
instance n=20 210.alb & 1 & 1 & Solution & 120.13 & 10 &  0.00 & 100.00\\
instance n=20 211.alb & 1 & 1 & Solution & 121.03 & 11 &  0.00 & 100.00\\
instance n=20 212.alb & 1 & 1 & Solution & 120.41 & 10 &  0.00 & 100.00\\
instance n=20 213.alb & 1 & 1 & Solution & 120.17 & 10 &  0.00 & 100.00\\
instance n=20 214.alb & 1 & 1 & Unknown & 121013.00 & - & - & -\\
instance n=20 215.alb & 1 & 1 & Solution & 120.42 & 11 &  0.00 & 100.00\\
instance n=20 216.alb & 1 & 1 & Solution & 120.29 & 12 &  0.00 & 100.00\\
instance n=20 217.alb & 1 & 1 & Solution & 121.02 & 11 &  0.00 & 100.00\\
instance n=20 218.alb & 1 & 1 & Solution & 121.02 & 11 &  0.00 & 100.00\\
instance n=20 219.alb & 1 & 1 & Solution & 121.01 & 11 &  0.00 & 100.00\\
instance n=20 22.alb & 1 & 1 & Optimal & 25.01 & 12 &  0.00 & 100.00\\
instance n=20 220.alb & 1 & 1 & Solution & 120.62 & 12 &  0.00 & 100.00\\
instance n=20 221.alb & 1 & 1 & Solution & 121.02 & 7 &  0.00 & 100.00\\
instance n=20 222.alb & 1 & 1 & Solution & 120.12 & 10 &  0.00 & 100.00\\
instance n=20 223.alb & 1 & 1 & Solution & 121.03 & 13 &  0.00 & 100.00\\
instance n=20 224.alb & 1 & 1 & Solution & 120.78 & 13 &  0.00 & 100.00\\
instance n=20 225.alb & 1 & 1 & Solution & 121.03 & 11 &  0.00 & 100.00\\
instance n=20 226.alb & 1 & 1 & Solution & 120.98 & 12 &  0.00 & 100.00\\
instance n=20 227.alb & 1 & 1 & Solution & 121.03 & 12 &  0.00 & 100.00\\
instance n=20 228.alb & 1 & 1 & Solution & 120.55 & 12 &  0.00 & 100.00\\
instance n=20 229.alb & 1 & 1 & Solution & 120.64 & 12 &  0.00 & 100.00\\
instance n=20 23.alb & 1 & 1 & Optimal & 30.80 & 13 &  0.00 & 100.00\\
instance n=20 230.alb & 1 & 1 & Solution & 121.02 & 12 &  0.00 & 100.00\\
instance n=20 231.alb & 1 & 1 & Solution & 120.45 & 12 &  0.00 & 100.00\\
instance n=20 232.alb & 1 & 1 & Solution & 121.02 & 12 &  0.00 & 100.00\\
instance n=20 233.alb & 1 & 1 & Unknown & 121018.00 & - & - & -\\
instance n=20 234.alb & 1 & 1 & Solution & 121.01 & 12 &  0.00 & 100.00\\
instance n=20 235.alb & 1 & 1 & Solution & 121.02 & 12 &  0.00 & 100.00\\
instance n=20 236.alb & 1 & 1 & Solution & 120.62 & 10 &  0.00 & 100.00\\
instance n=20 237.alb & 1 & 1 & Solution & 121.03 & 11 &  0.00 & 100.00\\
instance n=20 238.alb & 1 & 1 & Solution & 120.83 & 12 &  0.00 & 100.00\\
instance n=20 239.alb & 1 & 1 & Solution & 121.02 & 12 &  0.00 & 100.00\\
instance n=20 24.alb & 1 & 1 & Optimal & 54.31 & 11 &  0.00 & 100.00\\
instance n=20 240.alb & 1 & 1 & Solution & 121.02 & 12 &  0.00 & 100.00\\
instance n=20 241.alb & 1 & 1 & Optimal &  3.18 & 13 &  0.00 & 100.00\\
instance n=20 242.alb & 1 & 1 & Optimal &  2.56 & 12 &  0.00 & 100.00\\
instance n=20 243.alb & 1 & 1 & Optimal & 60.79 & 10 &  0.00 & 100.00\\
instance n=20 244.alb & 1 & 1 & Optimal & 13.85 & 11 &  0.00 & 100.00\\
instance n=20 245.alb & 1 & 1 & Optimal &  5.30 & 13 &  0.00 & 100.00\\
instance n=20 246.alb & 1 & 1 & Optimal &  9.48 & 13 &  0.00 & 100.00\\
instance n=20 247.alb & 1 & 1 & Optimal & 23.60 & 11 &  0.00 & 100.00\\
instance n=20 248.alb & 1 & 1 & Optimal & 24.54 & 11 &  0.00 & 100.00\\
instance n=20 249.alb & 1 & 1 & Optimal & 14.82 & 13 &  0.00 & 100.00\\
instance n=20 25.alb & 1 & 1 & Optimal & 18.24 & 11 &  0.00 & 100.00\\
instance n=20 250.alb & 1 & 1 & Optimal & 17.17 & 10 &  0.00 & 100.00\\
instance n=20 251.alb & 1 & 1 & Optimal & 12.73 & 12 &  0.00 & 100.00\\
instance n=20 252.alb & 1 & 1 & Optimal & 13.03 & 11 &  0.00 & 100.00\\
instance n=20 253.alb & 1 & 1 & Optimal &  5.76 & 13 &  0.00 & 100.00\\
instance n=20 254.alb & 1 & 1 & Optimal &  8.94 & 12 &  0.00 & 100.00\\
instance n=20 255.alb & 1 & 1 & Optimal &  4.55 & 13 &  0.00 & 100.00\\
instance n=20 256.alb & 1 & 1 & Optimal &  2.12 & 14 &  0.00 & 100.00\\
instance n=20 257.alb & 1 & 1 & Optimal & 17.03 & 10 &  0.00 & 100.00\\
instance n=20 258.alb & 1 & 1 & Optimal &  6.09 & 13 &  0.00 & 100.00\\
instance n=20 259.alb & 1 & 1 & Optimal &  7.03 & 13 &  0.00 & 100.00\\
instance n=20 26.alb & 1 & 1 & Optimal & 20.62 & 12 &  0.00 & 100.00\\
instance n=20 260.alb & 1 & 1 & Optimal &  3.26 & 12 &  0.00 & 100.00\\
instance n=20 261.alb & 1 & 1 & Optimal &  3.69 & 12 &  0.00 & 100.00\\
instance n=20 262.alb & 1 & 1 & Optimal & 20.54 & 11 &  0.00 & 100.00\\
instance n=20 263.alb & 1 & 1 & Optimal & 17.28 & 12 &  0.00 & 100.00\\
instance n=20 264.alb & 1 & 1 & Optimal & 16.47 & 12 &  0.00 & 100.00\\
instance n=20 265.alb & 1 & 1 & Optimal & 12.84 & 12 &  0.00 & 100.00\\
instance n=20 266.alb & 1 & 1 & Solution & 121.02 & 10 &  0.00 & 100.00\\
instance n=20 267.alb & 1 & 1 & Unknown & 121017.00 & - & - & -\\
instance n=20 268.alb & 1 & 1 & Solution & 120.15 & 9 &  0.00 & 100.00\\
instance n=20 269.alb & 1 & 1 & Solution & 121.01 & 10 &  0.00 & 100.00\\
instance n=20 27.alb & 1 & 1 & Optimal & 19.67 & 13 &  0.00 & 100.00\\
instance n=20 270.alb & 1 & 1 & Solution & 120.26 & 8 &  0.00 & 100.00\\
instance n=20 271.alb & 1 & 1 & Solution & 120.20 & 6 &  0.00 & 100.00\\
instance n=20 272.alb & 1 & 1 & Solution & 120.15 & 5 &  0.00 & 100.00\\
instance n=20 273.alb & 1 & 1 & Solution & 121.01 & 10 &  0.00 & 100.00\\
instance n=20 274.alb & 1 & 1 & Solution & 121.01 & 10 &  0.00 & 100.00\\
instance n=20 275.alb & 1 & 1 & Solution & 121.02 & 9 &  0.00 & 100.00\\
instance n=20 276.alb & 1 & 1 & Solution & 121.02 & 11 &  0.00 & 100.00\\
instance n=20 277.alb & 1 & 1 & Solution & 120.19 & 11 &  0.00 & 100.00\\
instance n=20 278.alb & 1 & 1 & Solution & 120.50 & 10 &  0.00 & 100.00\\
instance n=20 279.alb & 1 & 1 & Optimal & 85.57 & 6 &  0.00 & 100.00\\
instance n=20 28.alb & 1 & 1 & Optimal & 37.77 & 12 &  0.00 & 100.00\\
instance n=20 280.alb & 1 & 1 & Solution & 121.02 & 6 &  0.00 & 100.00\\
instance n=20 281.alb & 1 & 1 & Solution & 121.03 & 11 &  0.00 & 100.00\\
instance n=20 282.alb & 1 & 1 & Solution & 120.52 & 10 &  0.00 & 100.00\\
instance n=20 283.alb & 1 & 1 & Solution & 120.18 & 10 &  0.00 & 100.00\\
instance n=20 284.alb & 1 & 1 & Solution & 121.04 & 9 &  0.00 & 100.00\\
instance n=20 285.alb & 1 & 1 & Solution & 120.10 & 10 &  0.00 & 100.00\\
instance n=20 286.alb & 1 & 1 & Solution & 121.02 & 9 &  0.00 & 100.00\\
instance n=20 287.alb & 1 & 1 & Solution & 121.02 & 11 &  0.00 & 100.00\\
instance n=20 288.alb & 1 & 1 & Solution & 120.88 & 8 &  0.00 & 100.00\\
instance n=20 289.alb & 1 & 1 & Solution & 120.57 & 11 &  0.00 & 100.00\\
instance n=20 29.alb & 1 & 1 & Optimal & 87.95 & 10 &  0.00 & 100.00\\
instance n=20 290.alb & 1 & 1 & Solution & 120.92 & 8 &  0.00 & 100.00\\
instance n=20 291.alb & 1 & 1 & Solution & 120.95 & 13 &  0.00 & 100.00\\
instance n=20 292.alb & 1 & 1 & Solution & 121.01 & 12 &  0.00 & 100.00\\
instance n=20 293.alb & 1 & 1 & Solution & 120.60 & 12 &  0.00 & 100.00\\
instance n=20 294.alb & 1 & 1 & Solution & 120.12 & 11 &  0.00 & 100.00\\
instance n=20 295.alb & 1 & 1 & Solution & 121.02 & 12 &  0.00 & 100.00\\
instance n=20 296.alb & 1 & 1 & Solution & 121.02 & 11 &  0.00 & 100.00\\
instance n=20 297.alb & 1 & 1 & Solution & 120.42 & 12 &  0.00 & 100.00\\
instance n=20 298.alb & 1 & 1 & Solution & 121.01 & 11 &  0.00 & 100.00\\
instance n=20 299.alb & 1 & 1 & Solution & 121.02 & 11 &  0.00 & 100.00\\
instance n=20 3.alb & 1 & 1 & Solution & 120.51 & 12 &  0.00 & 100.00\\
instance n=20 30.alb & 1 & 1 & Optimal & 13.60 & 16 &  0.00 & 100.00\\
instance n=20 300.alb & 1 & 1 & Solution & 121.02 & 12 &  0.00 & 100.00\\
instance n=20 301.alb & 1 & 1 & Solution & 120.77 & 12 &  0.00 & 100.00\\
instance n=20 302.alb & 1 & 1 & Solution & 121.02 & 12 &  0.00 & 100.00\\
instance n=20 303.alb & 1 & 1 & Solution & 121.02 & 11 &  0.00 & 100.00\\
instance n=20 304.alb & 1 & 1 & Solution & 121.03 & 11 &  0.00 & 100.00\\
instance n=20 305.alb & 1 & 1 & Solution & 120.15 & 11 &  0.00 & 100.00\\
instance n=20 306.alb & 1 & 1 & Solution & 121.02 & 12 &  0.00 & 100.00\\
instance n=20 307.alb & 1 & 1 & Solution & 121.03 & 12 &  0.00 & 100.00\\
instance n=20 308.alb & 1 & 1 & Solution & 121.02 & 12 &  0.00 & 100.00\\
instance n=20 309.alb & 1 & 1 & Solution & 120.34 & 12 &  0.00 & 100.00\\
instance n=20 31.alb & 1 & 1 & Optimal & 28.62 & 12 &  0.00 & 100.00\\
instance n=20 310.alb & 1 & 1 & Solution & 120.48 & 11 &  0.00 & 100.00\\
instance n=20 311.alb & 1 & 1 & Unknown & 121028.00 & - & - & -\\
instance n=20 312.alb & 1 & 1 & Solution & 120.28 & 11 &  0.00 & 100.00\\
instance n=20 313.alb & 1 & 1 & Solution & 120.90 & 13 &  0.00 & 100.00\\
instance n=20 314.alb & 1 & 1 & Solution & 121.03 & 12 &  0.00 & 100.00\\
instance n=20 315.alb & 1 & 1 & Solution & 120.33 & 12 &  0.00 & 100.00\\
instance n=20 316.alb & 1 & 1 & Optimal & 94.01 & 10 &  0.00 & 100.00\\
instance n=20 317.alb & 1 & 1 & Solution & 120.11 & 10 &  0.00 & 100.00\\
instance n=20 318.alb & 1 & 1 & Optimal & 52.94 & 10 &  0.00 & 100.00\\
instance n=20 319.alb & 1 & 1 & Optimal & 19.77 & 14 &  0.00 & 100.00\\
instance n=20 32.alb & 1 & 1 & Optimal & 27.33 & 13 &  0.00 & 100.00\\
instance n=20 320.alb & 1 & 1 & Optimal & 47.45 & 12 &  0.00 & 100.00\\
instance n=20 321.alb & 1 & 1 & Optimal & 77.41 & 14 &  0.00 & 100.00\\
instance n=20 322.alb & 1 & 1 & Optimal & 63.66 & 12 &  0.00 & 100.00\\
instance n=20 323.alb & 1 & 1 & Optimal & 53.70 & 13 &  0.00 & 100.00\\
instance n=20 324.alb & 1 & 1 & Solution & 120.12 & 9 &  0.00 & 100.00\\
instance n=20 325.alb & 1 & 1 & Optimal & 32.40 & 14 &  0.00 & 100.00\\
instance n=20 326.alb & 1 & 1 & Optimal & 39.25 & 14 &  0.00 & 100.00\\
instance n=20 327.alb & 1 & 1 & Optimal & 52.29 & 13 &  0.00 & 100.00\\
instance n=20 328.alb & 1 & 1 & Optimal & 44.57 & 13 &  0.00 & 100.00\\
instance n=20 329.alb & 1 & 1 & Optimal & 59.99 & 10 &  0.00 & 100.00\\
instance n=20 33.alb & 1 & 1 & Optimal & 33.23 & 11 &  0.00 & 100.00\\
instance n=20 330.alb & 1 & 1 & Optimal & 102.15 & 12 &  0.00 & 100.00\\
instance n=20 331.alb & 1 & 1 & Optimal & 75.67 & 13 &  0.00 & 100.00\\
instance n=20 332.alb & 1 & 1 & Optimal & 38.67 & 13 &  0.00 & 100.00\\
instance n=20 333.alb & 1 & 1 & Optimal & 48.36 & 11 &  0.00 & 100.00\\
instance n=20 334.alb & 1 & 1 & Optimal & 77.27 & 10 &  0.00 & 100.00\\
instance n=20 335.alb & 1 & 1 & Optimal & 58.23 & 14 &  0.00 & 100.00\\
instance n=20 336.alb & 1 & 1 & Optimal & 40.40 & 11 &  0.00 & 100.00\\
instance n=20 337.alb & 1 & 1 & Optimal & 78.22 & 10 &  0.00 & 100.00\\
instance n=20 338.alb & 1 & 1 & Optimal & 18.83 & 14 &  0.00 & 100.00\\
instance n=20 339.alb & 1 & 1 & Optimal & 50.80 & 13 &  0.00 & 100.00\\
instance n=20 34.alb & 1 & 1 & Optimal & 39.09 & 12 &  0.00 & 100.00\\
instance n=20 340.alb & 1 & 1 & Optimal & 66.27 & 11 &  0.00 & 100.00\\
instance n=20 341.alb & 1 & 1 & Solution & 120.93 & 11 &  0.00 & 100.00\\
instance n=20 342.alb & 1 & 1 & Solution & 120.83 & 10 &  0.00 & 100.00\\
instance n=20 343.alb & 1 & 1 & Solution & 121.01 & 10 &  0.00 & 100.00\\
instance n=20 344.alb & 1 & 1 & Unknown & 121036.00 & - & - & -\\
instance n=20 345.alb & 1 & 1 & Solution & 120.64 & 12 &  0.00 & 100.00\\
instance n=20 346.alb & 1 & 1 & Solution & 121.02 & 11 &  0.00 & 100.00\\
instance n=20 347.alb & 1 & 1 & Solution & 121.01 & 10 &  0.00 & 100.00\\
instance n=20 348.alb & 1 & 1 & Solution & 121.01 & 11 &  0.00 & 100.00\\
instance n=20 349.alb & 1 & 1 & Solution & 121.00 & 10 &  0.00 & 100.00\\
instance n=20 35.alb & 1 & 1 & Optimal & 22.81 & 12 &  0.00 & 100.00\\
instance n=20 350.alb & 1 & 1 & Solution & 121.02 & 11 &  0.00 & 100.00\\
instance n=20 351.alb & 1 & 1 & Solution & 121.02 & 10 &  0.00 & 100.00\\
instance n=20 352.alb & 1 & 1 & Solution & 120.84 & 11 &  0.00 & 100.00\\
instance n=20 353.alb & 1 & 1 & Solution & 121.02 & 10 &  0.00 & 100.00\\
instance n=20 354.alb & 1 & 1 & Solution & 121.02 & 11 &  0.00 & 100.00\\
instance n=20 355.alb & 1 & 1 & Solution & 121.02 & 11 &  0.00 & 100.00\\
instance n=20 356.alb & 1 & 1 & Solution & 120.95 & 11 &  0.00 & 100.00\\
instance n=20 357.alb & 1 & 1 & Solution & 121.02 & 12 &  0.00 & 100.00\\
instance n=20 358.alb & 1 & 1 & Solution & 121.02 & 11 &  0.00 & 100.00\\
instance n=20 359.alb & 1 & 1 & Solution & 120.39 & 11 &  0.00 & 100.00\\
instance n=20 36.alb & 1 & 1 & Optimal & 22.13 & 13 &  0.00 & 100.00\\
instance n=20 360.alb & 1 & 1 & Solution & 121.03 & 10 &  0.00 & 100.00\\
instance n=20 361.alb & 1 & 1 & Solution & 121.03 & 10 &  0.00 & 100.00\\
instance n=20 362.alb & 1 & 1 & Solution & 120.49 & 11 &  0.00 & 100.00\\
instance n=20 363.alb & 1 & 1 & Solution & 120.98 & 10 &  0.00 & 100.00\\
instance n=20 364.alb & 1 & 1 & Solution & 120.91 & 12 &  0.00 & 100.00\\
instance n=20 365.alb & 1 & 1 & Solution & 121.02 & 11 &  0.00 & 100.00\\
instance n=20 366.alb & 1 & 1 & Solution & 120.28 & 12 &  0.00 & 100.00\\
instance n=20 367.alb & 1 & 1 & Solution & 120.36 & 13 &  0.00 & 100.00\\
instance n=20 368.alb & 1 & 1 & Solution & 120.34 & 11 &  0.00 & 100.00\\
instance n=20 369.alb & 1 & 1 & Solution & 121.02 & 12 &  0.00 & 100.00\\
instance n=20 37.alb & 1 & 1 & Optimal & 31.56 & 12 &  0.00 & 100.00\\
instance n=20 370.alb & 1 & 1 & Solution & 120.21 & 12 &  0.00 & 100.00\\
instance n=20 371.alb & 1 & 1 & Solution & 121.02 & 12 &  0.00 & 100.00\\
instance n=20 372.alb & 1 & 1 & Solution & 120.23 & 13 &  0.00 & 100.00\\
instance n=20 373.alb & 1 & 1 & Solution & 120.76 & 12 &  0.00 & 100.00\\
instance n=20 374.alb & 1 & 1 & Solution & 121.02 & 13 &  0.00 & 100.00\\
instance n=20 375.alb & 1 & 1 & Solution & 121.02 & 11 &  0.00 & 100.00\\
instance n=20 376.alb & 1 & 1 & Solution & 120.28 & 12 &  0.00 & 100.00\\
instance n=20 377.alb & 1 & 1 & Solution & 121.01 & 13 &  0.00 & 100.00\\
instance n=20 378.alb & 1 & 1 & Solution & 121.02 & 12 &  0.00 & 100.00\\
instance n=20 379.alb & 1 & 1 & Solution & 120.45 & 12 &  0.00 & 100.00\\
instance n=20 38.alb & 1 & 1 & Optimal & 33.63 & 12 &  0.00 & 100.00\\
instance n=20 380.alb & 1 & 1 & Solution & 120.11 & 13 &  0.00 & 100.00\\
instance n=20 381.alb & 1 & 1 & Solution & 120.12 & 13 &  0.00 & 100.00\\
instance n=20 382.alb & 1 & 1 & Unknown & 121016.00 & - & - & -\\
instance n=20 383.alb & 1 & 1 & Solution & 121.02 & 13 &  0.00 & 100.00\\
instance n=20 384.alb & 1 & 1 & Solution & 120.94 & 12 &  0.00 & 100.00\\
instance n=20 385.alb & 1 & 1 & Solution & 120.77 & 12 &  0.00 & 100.00\\
instance n=20 386.alb & 1 & 1 & Solution & 120.13 & 12 &  0.00 & 100.00\\
instance n=20 387.alb & 1 & 1 & Solution & 121.03 & 12 &  0.00 & 100.00\\
instance n=20 388.alb & 1 & 1 & Solution & 121.02 & 11 &  0.00 & 100.00\\
instance n=20 389.alb & 1 & 1 & Solution & 120.68 & 12 &  0.00 & 100.00\\
instance n=20 39.alb & 1 & 1 & Optimal & 24.09 & 13 &  0.00 & 100.00\\
instance n=20 390.alb & 1 & 1 & Solution & 120.33 & 13 &  0.00 & 100.00\\
instance n=20 391.alb & 1 & 1 & Optimal & 21.72 & 11 &  0.00 & 100.00\\
instance n=20 392.alb & 1 & 1 & Optimal &  8.29 & 14 &  0.00 & 100.00\\
instance n=20 393.alb & 1 & 1 & Optimal & 16.75 & 11 &  0.00 & 100.00\\
instance n=20 394.alb & 1 & 1 & Optimal & 20.11 & 12 &  0.00 & 100.00\\
instance n=20 395.alb & 1 & 1 & Optimal & 16.98 & 12 &  0.00 & 100.00\\
instance n=20 396.alb & 1 & 1 & Optimal & 19.57 & 13 &  0.00 & 100.00\\
instance n=20 397.alb & 1 & 1 & Optimal & 31.90 & 10 &  0.00 & 100.00\\
instance n=20 398.alb & 1 & 1 & Optimal & 31.70 & 11 &  0.00 & 100.00\\
instance n=20 399.alb & 1 & 1 & Optimal &  6.39 & 13 &  0.00 & 100.00\\
instance n=20 4.alb & 1 & 1 & Solution & 121.02 & 12 &  0.00 & 100.00\\
instance n=20 40.alb & 1 & 1 & Optimal & 33.51 & 12 &  0.00 & 100.00\\
instance n=20 400.alb & 1 & 1 & Optimal & 26.79 & 12 &  0.00 & 100.00\\
instance n=20 401.alb & 1 & 1 & Optimal & 13.15 & 12 &  0.00 & 100.00\\
instance n=20 402.alb & 1 & 1 & Optimal & 24.82 & 12 &  0.00 & 100.00\\
instance n=20 403.alb & 1 & 1 & Optimal & 14.34 & 12 &  0.00 & 100.00\\
instance n=20 404.alb & 1 & 1 & Optimal & 56.92 & 10 &  0.00 & 100.00\\
instance n=20 405.alb & 1 & 1 & Optimal & 22.83 & 12 &  0.00 & 100.00\\
instance n=20 406.alb & 1 & 1 & Optimal & 12.78 & 14 &  0.00 & 100.00\\
instance n=20 407.alb & 1 & 1 & Optimal & 47.83 & 10 &  0.00 & 100.00\\
instance n=20 408.alb & 1 & 1 & Optimal &  9.96 & 14 &  0.00 & 100.00\\
instance n=20 409.alb & 1 & 1 & Optimal & 20.44 & 12 &  0.00 & 100.00\\
instance n=20 41.alb & 1 & 1 & Solution & 120.12 & 9 &  0.00 & 100.00\\
instance n=20 410.alb & 1 & 1 & Optimal & 31.82 & 11 &  0.00 & 100.00\\
instance n=20 411.alb & 1 & 1 & Optimal &  6.31 & 15 &  0.00 & 100.00\\
instance n=20 412.alb & 1 & 1 & Optimal & 32.78 & 11 &  0.00 & 100.00\\
instance n=20 413.alb & 1 & 1 & Optimal & 32.70 & 10 &  0.00 & 100.00\\
instance n=20 414.alb & 1 & 1 & Optimal & 19.30 & 12 &  0.00 & 100.00\\
instance n=20 415.alb & 1 & 1 & Optimal & 47.71 & 10 &  0.00 & 100.00\\
instance n=20 416.alb & 1 & 1 & Solution & 120.88 & 11 &  0.00 & 100.00\\
instance n=20 417.alb & 1 & 1 & Solution & 121.02 & 10 &  0.00 & 100.00\\
instance n=20 418.alb & 1 & 1 & Solution & 120.37 & 8 &  0.00 & 100.00\\
instance n=20 419.alb & 1 & 1 & Solution & 121.02 & 11 &  0.00 & 100.00\\
instance n=20 42.alb & 1 & 1 & Solution & 120.57 & 11 &  0.00 & 100.00\\
instance n=20 420.alb & 1 & 1 & Solution & 120.45 & 11 &  0.00 & 100.00\\
instance n=20 421.alb & 1 & 1 & Solution & 120.14 & 11 &  0.00 & 100.00\\
instance n=20 422.alb & 1 & 1 & Solution & 120.40 & 11 &  0.00 & 100.00\\
instance n=20 423.alb & 1 & 1 & Solution & 121.01 & 10 &  0.00 & 100.00\\
instance n=20 424.alb & 1 & 1 & Solution & 120.40 & 10 &  0.00 & 100.00\\
instance n=20 425.alb & 1 & 1 & Solution & 120.11 & 7 &  0.00 & 100.00\\
instance n=20 426.alb & 1 & 1 & Solution & 121.01 & 11 &  0.00 & 100.00\\
instance n=20 427.alb & 1 & 1 & Solution & 121.03 & 7 &  0.00 & 100.00\\
instance n=20 428.alb & 1 & 1 & Solution & 120.26 & 9 &  0.00 & 100.00\\
instance n=20 429.alb & 1 & 1 & Solution & 120.85 & 10 &  0.00 & 100.00\\
instance n=20 43.alb & 1 & 1 & Solution & 120.25 & 10 &  0.00 & 100.00\\
instance n=20 430.alb & 1 & 1 & Solution & 121.01 & 11 &  0.00 & 100.00\\
instance n=20 431.alb & 1 & 1 & Optimal & 112.69 & 6 &  0.00 & 100.00\\
instance n=20 432.alb & 1 & 1 & Solution & 120.16 & 9 &  0.00 & 100.00\\
instance n=20 433.alb & 1 & 1 & Solution & 121.02 & 10 &  0.00 & 100.00\\
instance n=20 434.alb & 1 & 1 & Solution & 120.80 & 11 &  0.00 & 100.00\\
instance n=20 435.alb & 1 & 1 & Solution & 120.13 & 8 &  0.00 & 100.00\\
instance n=20 436.alb & 1 & 1 & Solution & 120.48 & 10 &  0.00 & 100.00\\
instance n=20 437.alb & 1 & 1 & Solution & 120.11 & 10 &  0.00 & 100.00\\
instance n=20 438.alb & 1 & 1 & Solution & 120.92 & 10 &  0.00 & 100.00\\
instance n=20 439.alb & 1 & 1 & Solution & 120.72 & 9 &  0.00 & 100.00\\
instance n=20 44.alb & 1 & 1 & Solution & 120.27 & 9 &  0.00 & 100.00\\
instance n=20 440.alb & 1 & 1 & Solution & 121.03 & 10 &  0.00 & 100.00\\
instance n=20 441.alb & 1 & 1 & Solution & 120.29 & 10 &  0.00 & 100.00\\
instance n=20 442.alb & 1 & 1 & Solution & 121.02 & 11 &  0.00 & 100.00\\
instance n=20 443.alb & 1 & 1 & Solution & 120.97 & 13 &  0.00 & 100.00\\
instance n=20 444.alb & 1 & 1 & Solution & 120.19 & 9 &  0.00 & 100.00\\
instance n=20 445.alb & 1 & 1 & Solution & 121.02 & 10 &  0.00 & 100.00\\
instance n=20 446.alb & 1 & 1 & Solution & 121.01 & 11 &  0.00 & 100.00\\
instance n=20 447.alb & 1 & 1 & Solution & 121.02 & 10 &  0.00 & 100.00\\
instance n=20 448.alb & 1 & 1 & Solution & 121.02 & 8 &  0.00 & 100.00\\
instance n=20 449.alb & 1 & 1 & Unknown & 121021.00 & - & - & -\\
instance n=20 45.alb & 1 & 1 & Solution & 121.02 & 10 &  0.00 & 100.00\\
instance n=20 450.alb & 1 & 1 & Solution & 120.13 & 11 &  0.00 & 100.00\\
instance n=20 451.alb & 1 & 1 & Solution & 120.82 & 11 &  0.00 & 100.00\\
instance n=20 452.alb & 1 & 1 & Solution & 121.03 & 11 &  0.00 & 100.00\\
instance n=20 453.alb & 1 & 1 & Solution & 120.28 & 9 &  0.00 & 100.00\\
instance n=20 454.alb & 1 & 1 & Solution & 121.01 & 11 &  0.00 & 100.00\\
instance n=20 455.alb & 1 & 1 & Solution & 121.02 & 11 &  0.00 & 100.00\\
instance n=20 456.alb & 1 & 1 & Solution & 120.30 & 5 &  0.00 & 100.00\\
instance n=20 457.alb & 1 & 1 & Solution & 121.03 & 10 &  0.00 & 100.00\\
instance n=20 458.alb & 1 & 1 & Solution & 120.54 & 11 &  0.00 & 100.00\\
instance n=20 459.alb & 1 & 1 & Solution & 121.02 & 11 &  0.00 & 100.00\\
instance n=20 46.alb & 1 & 1 & Unknown & 121016.00 & - & - & -\\
instance n=20 460.alb & 1 & 1 & Solution & 121.03 & 12 &  0.00 & 100.00\\
instance n=20 461.alb & 1 & 1 & Solution & 121.01 & 11 &  0.00 & 100.00\\
instance n=20 462.alb & 1 & 1 & Solution & 120.74 & 10 &  0.00 & 100.00\\
instance n=20 463.alb & 1 & 1 & Solution & 120.11 & 10 &  0.00 & 100.00\\
instance n=20 464.alb & 1 & 1 & Solution & 120.09 & 9 &  0.00 & 100.00\\
instance n=20 465.alb & 1 & 1 & Unknown & 121013.00 & - & - & -\\
instance n=20 466.alb & 1 & 1 & Optimal & 13.01 & 13 &  0.00 & 100.00\\
instance n=20 467.alb & 1 & 1 & Optimal &  7.02 & 14 &  0.00 & 100.00\\
instance n=20 468.alb & 1 & 1 & Optimal &  8.82 & 13 &  0.00 & 100.00\\
instance n=20 469.alb & 1 & 1 & Optimal &  1.08 & 14 &  0.00 & 100.00\\
instance n=20 47.alb & 1 & 1 & Solution & 120.16 & 11 &  0.00 & 100.00\\
instance n=20 470.alb & 1 & 1 & Optimal &  6.26 & 12 &  0.00 & 100.00\\
instance n=20 471.alb & 1 & 1 & Optimal & 10.25 & 12 &  0.00 & 100.00\\
instance n=20 472.alb & 1 & 1 & Optimal &  6.44 & 13 &  0.00 & 100.00\\
instance n=20 473.alb & 1 & 1 & Optimal & 18.78 & 10 &  0.00 & 100.00\\
instance n=20 474.alb & 1 & 1 & Optimal &  8.35 & 14 &  0.00 & 100.00\\
instance n=20 475.alb & 1 & 1 & Optimal &  7.56 & 11 &  0.00 & 100.00\\
instance n=20 476.alb & 1 & 1 & Optimal & 13.65 & 11 &  0.00 & 100.00\\
instance n=20 477.alb & 1 & 1 & Optimal & 10.72 & 11 &  0.00 & 100.00\\
instance n=20 478.alb & 1 & 1 & Optimal & 13.64 & 12 &  0.00 & 100.00\\
instance n=20 479.alb & 1 & 1 & Optimal &  4.30 & 13 &  0.00 & 100.00\\
instance n=20 48.alb & 1 & 1 & Solution & 120.13 & 10 &  0.00 & 100.00\\
instance n=20 480.alb & 1 & 1 & Optimal &  8.76 & 13 &  0.00 & 100.00\\
instance n=20 481.alb & 1 & 1 & Optimal &  3.62 & 13 &  0.00 & 100.00\\
instance n=20 482.alb & 1 & 1 & Optimal &  3.99 & 13 &  0.00 & 100.00\\
instance n=20 483.alb & 1 & 1 & Optimal & 11.48 & 12 &  0.00 & 100.00\\
instance n=20 484.alb & 1 & 1 & Optimal &  4.37 & 13 &  0.00 & 100.00\\
instance n=20 485.alb & 1 & 1 & Optimal &  6.95 & 15 &  0.00 & 100.00\\
instance n=20 486.alb & 1 & 1 & Optimal & 10.61 & 11 &  0.00 & 100.00\\
instance n=20 487.alb & 1 & 1 & Optimal &  8.67 & 12 &  0.00 & 100.00\\
instance n=20 488.alb & 1 & 1 & Optimal &  2.72 & 15 &  0.00 & 100.00\\
instance n=20 489.alb & 1 & 1 & Optimal &  7.87 & 12 &  0.00 & 100.00\\
instance n=20 49.alb & 1 & 1 & Solution & 120.26 & 11 &  0.00 & 100.00\\
instance n=20 490.alb & 1 & 1 & Optimal &  6.38 & 12 &  0.00 & 100.00\\
instance n=20 491.alb & 1 & 1 & Optimal & 76.12 & 6 &  0.00 & 100.00\\
instance n=20 492.alb & 1 & 1 & Optimal & 47.61 & 5 &  0.00 & 100.00\\
instance n=20 493.alb & 1 & 1 & Solution & 120.11 & 5 &  0.00 & 100.00\\
instance n=20 494.alb & 1 & 1 & Solution & 120.17 & 7 &  0.00 & 100.00\\
instance n=20 495.alb & 1 & 1 & Optimal & 110.95 & 6 &  0.00 & 100.00\\
instance n=20 496.alb & 1 & 1 & Solution & 120.45 & 7 &  0.00 & 100.00\\
instance n=20 497.alb & 1 & 1 & Optimal & 75.10 & 6 &  0.00 & 100.00\\
instance n=20 498.alb & 1 & 1 & Optimal & 74.54 & 6 &  0.00 & 100.00\\
instance n=20 499.alb & 1 & 1 & Optimal & 102.72 & 5 &  0.00 & 100.00\\
instance n=20 5.alb & 1 & 1 & Solution & 121.00 & 12 &  0.00 & 100.00\\
instance n=20 50.alb & 1 & 1 & Solution & 120.78 & 11 &  0.00 & 100.00\\
instance n=20 500.alb & 1 & 1 & Optimal & 38.36 & 8 &  0.00 & 100.00\\
instance n=20 501.alb & 1 & 1 & Optimal & 105.62 & 5 &  0.00 & 100.00\\
instance n=20 502.alb & 1 & 1 & Solution & 121.02 & 10 &  0.00 & 100.00\\
instance n=20 503.alb & 1 & 1 & Solution & 121.02 & 7 &  0.00 & 100.00\\
instance n=20 504.alb & 1 & 1 & Optimal & 102.12 & 6 &  0.00 & 100.00\\
instance n=20 505.alb & 1 & 1 & Optimal & 90.43 & 6 &  0.00 & 100.00\\
instance n=20 506.alb & 1 & 1 & Solution & 121.02 & 6 &  0.00 & 100.00\\
instance n=20 507.alb & 1 & 1 & Optimal & 26.70 & 5 &  0.00 & 100.00\\
instance n=20 508.alb & 1 & 1 & Solution & 120.28 & 6 &  0.00 & 100.00\\
instance n=20 509.alb & 1 & 1 & Solution & 120.32 & 7 &  0.00 & 100.00\\
instance n=20 51.alb & 1 & 1 & Solution & 120.42 & 10 &  0.00 & 100.00\\
instance n=20 510.alb & 1 & 1 & Solution & 120.27 & 6 &  0.00 & 100.00\\
instance n=20 511.alb & 1 & 1 & Optimal & 37.11 & 5 &  0.00 & 100.00\\
instance n=20 512.alb & 1 & 1 & Optimal & 113.16 & 5 &  0.00 & 100.00\\
instance n=20 513.alb & 1 & 1 & Solution & 121.01 & 8 &  0.00 & 100.00\\
instance n=20 514.alb & 1 & 1 & Solution & 120.92 & 5 &  0.00 & 100.00\\
instance n=20 515.alb & 1 & 1 & Optimal & 72.34 & 6 &  0.00 & 100.00\\
instance n=20 516.alb & 1 & 1 & Solution & 121.02 & 10 &  0.00 & 100.00\\
instance n=20 517.alb & 1 & 1 & Solution & 121.03 & 12 &  0.00 & 100.00\\
instance n=20 518.alb & 1 & 1 & Solution & 121.03 & 12 &  0.00 & 100.00\\
instance n=20 519.alb & 1 & 1 & Solution & 121.02 & 11 &  0.00 & 100.00\\
instance n=20 52.alb & 1 & 1 & Solution & 120.51 & 11 &  0.00 & 100.00\\
instance n=20 520.alb & 1 & 1 & Solution & 121.01 & 12 &  0.00 & 100.00\\
instance n=20 521.alb & 1 & 1 & Solution & 120.72 & 11 &  0.00 & 100.00\\
instance n=20 522.alb & 1 & 1 & Solution & 121.02 & 12 &  0.00 & 100.00\\
instance n=20 523.alb & 1 & 1 & Solution & 120.29 & 11 &  0.00 & 100.00\\
instance n=20 524.alb & 1 & 1 & Solution & 121.01 & 11 &  0.00 & 100.00\\
instance n=20 525.alb & 1 & 1 & Solution & 121.02 & 10 &  0.00 & 100.00\\
instance n=20 53.alb & 1 & 1 & Solution & 120.12 & 11 &  0.00 & 100.00\\
instance n=20 54.alb & 1 & 1 & Solution & 120.37 & 10 &  0.00 & 100.00\\
instance n=20 55.alb & 1 & 1 & Solution & 120.84 & 11 &  0.00 & 100.00\\
instance n=20 56.alb & 1 & 1 & Solution & 121.01 & 11 &  0.00 & 100.00\\
instance n=20 57.alb & 1 & 1 & Solution & 121.03 & 11 &  0.00 & 100.00\\
instance n=20 58.alb & 1 & 1 & Solution & 120.23 & 10 &  0.00 & 100.00\\
instance n=20 59.alb & 1 & 1 & Solution & 120.75 & 10 &  0.00 & 100.00\\
instance n=20 6.alb & 1 & 1 & Solution & 121.02 & 11 &  0.00 & 100.00\\
instance n=20 60.alb & 1 & 1 & Solution & 121.03 & 10 &  0.00 & 100.00\\
instance n=20 61.alb & 1 & 1 & Solution & 121.02 & 9 &  0.00 & 100.00\\
instance n=20 62.alb & 1 & 1 & Solution & 121.00 & 10 &  0.00 & 100.00\\
instance n=20 63.alb & 1 & 1 & Solution & 121.01 & 11 &  0.00 & 100.00\\
instance n=20 64.alb & 1 & 1 & Solution & 120.31 & 10 &  0.00 & 100.00\\
instance n=20 65.alb & 1 & 1 & Solution & 121.02 & 10 &  0.00 & 100.00\\
instance n=20 66.alb & 1 & 1 & Solution & 120.17 & 12 &  0.00 & 100.00\\
instance n=20 67.alb & 1 & 1 & Solution & 120.30 & 12 &  0.00 & 100.00\\
instance n=20 68.alb & 1 & 1 & Solution & 120.67 & 11 &  0.00 & 100.00\\
instance n=20 69.alb & 1 & 1 & Solution & 120.77 & 12 &  0.00 & 100.00\\
instance n=20 7.alb & 1 & 1 & Solution & 121.02 & 12 &  0.00 & 100.00\\
instance n=20 70.alb & 1 & 1 & Solution & 121.02 & 11 &  0.00 & 100.00\\
instance n=20 71.alb & 1 & 1 & Solution & 120.19 & 12 &  0.00 & 100.00\\
instance n=20 72.alb & 1 & 1 & Solution & 120.64 & 12 &  0.00 & 100.00\\
instance n=20 73.alb & 1 & 1 & Solution & 121.02 & 13 &  0.00 & 100.00\\
instance n=20 74.alb & 1 & 1 & Solution & 120.21 & 12 &  0.00 & 100.00\\
instance n=20 75.alb & 1 & 1 & Solution & 120.77 & 12 &  0.00 & 100.00\\
instance n=20 76.alb & 1 & 1 & Solution & 120.45 & 11 &  0.00 & 100.00\\
instance n=20 77.alb & 1 & 1 & Solution & 121.01 & 11 &  0.00 & 100.00\\
instance n=20 78.alb & 1 & 1 & Solution & 120.59 & 11 &  0.00 & 100.00\\
instance n=20 79.alb & 1 & 1 & Solution & 120.27 & 11 &  0.00 & 100.00\\
instance n=20 8.alb & 1 & 1 & Solution & 120.34 & 11 &  0.00 & 100.00\\
instance n=20 80.alb & 1 & 1 & Solution & 120.49 & 12 &  0.00 & 100.00\\
instance n=20 81.alb & 1 & 1 & Solution & 121.01 & 11 &  0.00 & 100.00\\
instance n=20 82.alb & 1 & 1 & Solution & 120.13 & 12 &  0.00 & 100.00\\
instance n=20 83.alb & 1 & 1 & Solution & 120.94 & 11 &  0.00 & 100.00\\
instance n=20 84.alb & 1 & 1 & Solution & 120.50 & 12 &  0.00 & 100.00\\
instance n=20 85.alb & 1 & 1 & Solution & 121.02 & 12 &  0.00 & 100.00\\
instance n=20 86.alb & 1 & 1 & Solution & 120.11 & 11 &  0.00 & 100.00\\
instance n=20 87.alb & 1 & 1 & Solution & 121.02 & 12 &  0.00 & 100.00\\
instance n=20 88.alb & 1 & 1 & Solution & 121.00 & 12 &  0.00 & 100.00\\
instance n=20 89.alb & 1 & 1 & Solution & 121.03 & 12 &  0.00 & 100.00\\
instance n=20 9.alb & 1 & 1 & Solution & 121.01 & 11 &  0.00 & 100.00\\
instance n=20 90.alb & 1 & 1 & Solution & 120.38 & 12 &  0.00 & 100.00\\
instance n=20 91.alb & 1 & 1 & Optimal & 36.25 & 11 &  0.00 & 100.00\\
instance n=20 92.alb & 1 & 1 & Optimal & 19.22 & 11 &  0.00 & 100.00\\
instance n=20 93.alb & 1 & 1 & Optimal & 16.75 & 13 &  0.00 & 100.00\\
instance n=20 94.alb & 1 & 1 & Optimal & 40.16 & 10 &  0.00 & 100.00\\
instance n=20 95.alb & 1 & 1 & Optimal & 26.14 & 12 &  0.00 & 100.00\\
instance n=20 96.alb & 1 & 1 & Optimal & 29.36 & 10 &  0.00 & 100.00\\
instance n=20 97.alb & 1 & 1 & Optimal & 11.54 & 15 &  0.00 & 100.00\\
instance n=20 98.alb & 1 & 1 & Optimal & 14.49 & 13 &  0.00 & 100.00\\
instance n=20 99.alb & 1 & 1 & Optimal & 18.66 & 12 &  0.00 & 100.00\\
\end{longtable}



\subsection{Cplex}

\begin{longtable}{lrrlrrrr}
\caption{Results for SALBP-1 Problems Alternative (Cplex) (1050 Instances)}\\\toprule
Name & \shortstack{Nr\\Jobs} & \shortstack{Nr\\Machines} & Status & Time & Makespan & Bound & \shortstack{Gap\\Percent}\\ \midrule
\endhead
\bottomrule
\endfoot
instance n=20 1.alb & 1 & 1 & Solution & 120.04 & 3 &  0.00 & 100.00\\
instance n=20 10.alb & 1 & 1 & Solution & 120.04 & 3 &  0.00 & 100.00\\
instance n=20 100.alb & 1 & 1 & Solution & 120.04 & 11 &  0.00 & 100.00\\
instance n=20 101.alb & 1 & 1 & Solution & 120.02 & 13 &  0.00 & 100.00\\
instance n=20 102.alb & 1 & 1 & Optimal &  3.92 & 13 &  0.00 & 100.00\\
instance n=20 103.alb & 1 & 1 & Optimal & 16.39 & 12 &  0.00 & 100.00\\
instance n=20 104.alb & 1 & 1 & Optimal &  2.06 & 11 &  0.00 & 100.00\\
instance n=20 105.alb & 1 & 1 & Optimal & 26.34 & 12 &  0.00 & 100.00\\
instance n=20 106.alb & 1 & 1 & Optimal & 12.05 & 10 &  0.00 & 100.00\\
instance n=20 107.alb & 1 & 1 & Solution & 120.03 & 14 &  0.00 & 100.00\\
instance n=20 108.alb & 1 & 1 & Solution & 120.04 & 15 &  0.00 & 100.00\\
instance n=20 109.alb & 1 & 1 & Optimal &  4.47 & 12 &  0.00 & 100.00\\
instance n=20 11.alb & 1 & 1 & Solution & 120.03 & 3 &  0.00 & 100.00\\
instance n=20 110.alb & 1 & 1 & Optimal &  2.03 & 11 &  0.00 & 100.00\\
instance n=20 111.alb & 1 & 1 & Optimal &  4.45 & 13 &  0.00 & 100.00\\
instance n=20 112.alb & 1 & 1 & Optimal &  3.47 & 11 &  0.00 & 100.00\\
instance n=20 113.alb & 1 & 1 & Optimal &  8.33 & 12 &  0.00 & 100.00\\
instance n=20 114.alb & 1 & 1 & Optimal & 46.44 & 13 &  0.00 & 100.00\\
instance n=20 115.alb & 1 & 1 & Optimal &  7.57 & 11 &  0.00 & 100.00\\
instance n=20 116.alb & 1 & 1 & Optimal & 53.33 & 5 &  0.00 & 100.00\\
instance n=20 117.alb & 1 & 1 & Solution & 120.03 & 5 &  0.00 & 100.00\\
instance n=20 118.alb & 1 & 1 & Optimal &  5.14 & 5 &  0.00 & 100.00\\
instance n=20 119.alb & 1 & 1 & Optimal &  1.42 & 6 &  0.00 & 100.00\\
instance n=20 12.alb & 1 & 1 & Solution & 120.04 & 3 &  0.00 & 100.00\\
instance n=20 120.alb & 1 & 1 & Optimal &  7.69 & 6 &  0.00 & 100.00\\
instance n=20 121.alb & 1 & 1 & Optimal & 88.61 & 5 &  0.00 & 100.00\\
instance n=20 122.alb & 1 & 1 & Optimal &  3.49 & 6 &  0.00 & 100.00\\
instance n=20 123.alb & 1 & 1 & Optimal &  7.72 & 5 &  0.00 & 100.00\\
instance n=20 124.alb & 1 & 1 & Optimal &  6.09 & 5 &  0.00 & 100.00\\
instance n=20 125.alb & 1 & 1 & Optimal & 14.03 & 5 &  0.00 & 100.00\\
instance n=20 126.alb & 1 & 1 & Optimal &  1.45 & 5 &  0.00 & 100.00\\
instance n=20 127.alb & 1 & 1 & Optimal & 27.99 & 4 &  0.00 & 100.00\\
instance n=20 128.alb & 1 & 1 & Optimal & 73.10 & 5 &  0.00 & 100.00\\
instance n=20 129.alb & 1 & 1 & Optimal &  2.89 & 5 &  0.00 & 100.00\\
instance n=20 13.alb & 1 & 1 & Solution & 120.02 & 3 &  0.00 & 100.00\\
instance n=20 130.alb & 1 & 1 & Optimal &  2.83 & 6 &  0.00 & 100.00\\
instance n=20 131.alb & 1 & 1 & Optimal &  1.28 & 7 &  0.00 & 100.00\\
instance n=20 132.alb & 1 & 1 & Solution & 120.03 & 4 &  0.00 & 100.00\\
instance n=20 133.alb & 1 & 1 & Optimal & 21.50 & 5 &  0.00 & 100.00\\
instance n=20 134.alb & 1 & 1 & Optimal & 13.06 & 6 &  0.00 & 100.00\\
instance n=20 135.alb & 1 & 1 & Optimal &  5.24 & 6 &  0.00 & 100.00\\
instance n=20 136.alb & 1 & 1 & Optimal &  2.00 & 6 &  0.00 & 100.00\\
instance n=20 137.alb & 1 & 1 & Optimal &  1.59 & 5 &  0.00 & 100.00\\
instance n=20 138.alb & 1 & 1 & Optimal &  6.42 & 5 &  0.00 & 100.00\\
instance n=20 139.alb & 1 & 1 & Optimal &  9.35 & 5 &  0.00 & 100.00\\
instance n=20 14.alb & 1 & 1 & Solution & 120.02 & 3 &  0.00 & 100.00\\
instance n=20 140.alb & 1 & 1 & Optimal & 80.24 & 5 &  0.00 & 100.00\\
instance n=20 141.alb & 1 & 1 & Solution & 120.04 & 3 &  0.00 & 100.00\\
instance n=20 142.alb & 1 & 1 & Solution & 120.02 & 3 &  0.00 & 100.00\\
instance n=20 143.alb & 1 & 1 & Solution & 120.03 & 3 &  0.00 & 100.00\\
instance n=20 144.alb & 1 & 1 & Solution & 120.03 & 4 &  0.00 & 100.00\\
instance n=20 145.alb & 1 & 1 & Solution & 120.03 & 3 &  0.00 & 100.00\\
instance n=20 146.alb & 1 & 1 & Solution & 120.03 & 3 &  0.00 & 100.00\\
instance n=20 147.alb & 1 & 1 & Solution & 120.04 & 3 &  0.00 & 100.00\\
instance n=20 148.alb & 1 & 1 & Solution & 120.03 & 3 &  0.00 & 100.00\\
instance n=20 149.alb & 1 & 1 & Solution & 120.03 & 3 &  0.00 & 100.00\\
instance n=20 15.alb & 1 & 1 & Solution & 120.03 & 3 &  0.00 & 100.00\\
instance n=20 150.alb & 1 & 1 & Solution & 120.04 & 3 &  0.00 & 100.00\\
instance n=20 151.alb & 1 & 1 & Solution & 120.02 & 3 &  0.00 & 100.00\\
instance n=20 152.alb & 1 & 1 & Solution & 120.04 & 3 &  0.00 & 100.00\\
instance n=20 153.alb & 1 & 1 & Solution & 120.04 & 3 &  0.00 & 100.00\\
instance n=20 154.alb & 1 & 1 & Solution & 120.02 & 3 &  0.00 & 100.00\\
instance n=20 155.alb & 1 & 1 & Solution & 120.02 & 3 &  0.00 & 100.00\\
instance n=20 156.alb & 1 & 1 & Solution & 120.04 & 3 &  0.00 & 100.00\\
instance n=20 157.alb & 1 & 1 & Solution & 120.02 & 3 &  0.00 & 100.00\\
instance n=20 158.alb & 1 & 1 & Solution & 120.04 & 3 &  0.00 & 100.00\\
instance n=20 159.alb & 1 & 1 & Solution & 120.03 & 3 &  0.00 & 100.00\\
instance n=20 16.alb & 1 & 1 & Optimal & 45.22 & 12 &  0.00 & 100.00\\
instance n=20 160.alb & 1 & 1 & Solution & 120.03 & 3 &  0.00 & 100.00\\
instance n=20 161.alb & 1 & 1 & Solution & 120.03 & 3 &  0.00 & 100.00\\
instance n=20 162.alb & 1 & 1 & Solution & 120.04 & 3 &  0.00 & 100.00\\
instance n=20 163.alb & 1 & 1 & Solution & 120.03 & 3 &  0.00 & 100.00\\
instance n=20 164.alb & 1 & 1 & Solution & 120.03 & 4 &  0.00 & 100.00\\
instance n=20 165.alb & 1 & 1 & Solution & 120.04 & 3 &  0.00 & 100.00\\
instance n=20 166.alb & 1 & 1 & Solution & 120.02 & 12 &  0.00 & 100.00\\
instance n=20 167.alb & 1 & 1 & Solution & 120.03 & 11 &  0.00 & 100.00\\
instance n=20 168.alb & 1 & 1 & Optimal & 49.28 & 10 &  0.00 & 100.00\\
instance n=20 169.alb & 1 & 1 & Optimal & 117.94 & 11 &  0.00 & 100.00\\
instance n=20 17.alb & 1 & 1 & Solution & 120.03 & 10 &  0.00 & 100.00\\
instance n=20 170.alb & 1 & 1 & Solution & 120.05 & 11 &  0.00 & 100.00\\
instance n=20 171.alb & 1 & 1 & Solution & 120.03 & 13 &  0.00 & 100.00\\
instance n=20 172.alb & 1 & 1 & Optimal & 38.45 & 11 &  0.00 & 100.00\\
instance n=20 173.alb & 1 & 1 & Solution & 120.05 & 11 &  0.00 & 100.00\\
instance n=20 174.alb & 1 & 1 & Optimal & 90.54 & 12 &  0.00 & 100.00\\
instance n=20 175.alb & 1 & 1 & Optimal & 14.02 & 10 &  0.00 & 100.00\\
instance n=20 176.alb & 1 & 1 & Optimal & 45.53 & 11 &  0.00 & 100.00\\
instance n=20 177.alb & 1 & 1 & Solution & 120.04 & 10 &  0.00 & 100.00\\
instance n=20 178.alb & 1 & 1 & Optimal & 37.48 & 11 &  0.00 & 100.00\\
instance n=20 179.alb & 1 & 1 & Solution & 120.03 & 11 &  0.00 & 100.00\\
instance n=20 18.alb & 1 & 1 & Optimal & 74.25 & 11 &  0.00 & 100.00\\
instance n=20 180.alb & 1 & 1 & Solution & 120.04 & 13 &  0.00 & 100.00\\
instance n=20 181.alb & 1 & 1 & Optimal & 27.41 & 11 &  0.00 & 100.00\\
instance n=20 182.alb & 1 & 1 & Solution & 120.03 & 11 &  0.00 & 100.00\\
instance n=20 183.alb & 1 & 1 & Solution & 120.04 & 13 &  0.00 & 100.00\\
instance n=20 184.alb & 1 & 1 & Solution & 120.04 & 12 &  0.00 & 100.00\\
instance n=20 185.alb & 1 & 1 & Solution & 120.03 & 15 &  0.00 & 100.00\\
instance n=20 186.alb & 1 & 1 & Solution & 120.04 & 14 &  0.00 & 100.00\\
instance n=20 187.alb & 1 & 1 & Optimal &  5.15 & 10 &  0.00 & 100.00\\
instance n=20 188.alb & 1 & 1 & Optimal & 102.74 & 11 &  0.00 & 100.00\\
instance n=20 189.alb & 1 & 1 & Solution & 120.03 & 13 &  0.00 & 100.00\\
instance n=20 19.alb & 1 & 1 & Solution & 120.02 & 14 &  0.00 & 100.00\\
instance n=20 190.alb & 1 & 1 & Solution & 120.05 & 15 &  0.00 & 100.00\\
instance n=20 191.alb & 1 & 1 & Solution & 120.03 & 4 &  0.00 & 100.00\\
instance n=20 192.alb & 1 & 1 & Optimal & 66.43 & 5 &  0.00 & 100.00\\
instance n=20 193.alb & 1 & 1 & Solution & 120.02 & 5 &  0.00 & 100.00\\
instance n=20 194.alb & 1 & 1 & Optimal & 51.86 & 6 &  0.00 & 100.00\\
instance n=20 195.alb & 1 & 1 & Solution & 120.04 & 6 &  0.00 & 100.00\\
instance n=20 196.alb & 1 & 1 & Solution & 120.02 & 5 &  0.00 & 100.00\\
instance n=20 197.alb & 1 & 1 & Optimal & 95.90 & 4 &  0.00 & 100.00\\
instance n=20 198.alb & 1 & 1 & Optimal & 47.37 & 6 &  0.00 & 100.00\\
instance n=20 199.alb & 1 & 1 & Solution & 120.03 & 5 &  0.00 & 100.00\\
instance n=20 2.alb & 1 & 1 & Solution & 120.02 & 3 &  0.00 & 100.00\\
instance n=20 20.alb & 1 & 1 & Solution & 120.03 & 11 &  0.00 & 100.00\\
instance n=20 200.alb & 1 & 1 & Optimal & 51.89 & 6 &  0.00 & 100.00\\
instance n=20 201.alb & 1 & 1 & Solution & 120.03 & 6 &  0.00 & 100.00\\
instance n=20 202.alb & 1 & 1 & Solution & 120.03 & 4 &  0.00 & 100.00\\
instance n=20 203.alb & 1 & 1 & Solution & 120.04 & 4 &  0.00 & 100.00\\
instance n=20 204.alb & 1 & 1 & Solution & 120.03 & 5 &  0.00 & 100.00\\
instance n=20 205.alb & 1 & 1 & Solution & 120.03 & 6 &  0.00 & 100.00\\
instance n=20 206.alb & 1 & 1 & Optimal & 41.26 & 5 &  0.00 & 100.00\\
instance n=20 207.alb & 1 & 1 & Optimal & 43.53 & 6 &  0.00 & 100.00\\
instance n=20 208.alb & 1 & 1 & Solution & 120.04 & 5 &  0.00 & 100.00\\
instance n=20 209.alb & 1 & 1 & Solution & 120.04 & 4 &  0.00 & 100.00\\
instance n=20 21.alb & 1 & 1 & Solution & 120.05 & 14 &  0.00 & 100.00\\
instance n=20 210.alb & 1 & 1 & Solution & 120.04 & 5 &  0.00 & 100.00\\
instance n=20 211.alb & 1 & 1 & Solution & 120.03 & 5 &  0.00 & 100.00\\
instance n=20 212.alb & 1 & 1 & Optimal & 109.32 & 5 &  0.00 & 100.00\\
instance n=20 213.alb & 1 & 1 & Optimal & 114.33 & 5 &  0.00 & 100.00\\
instance n=20 214.alb & 1 & 1 & Solution & 120.03 & 5 &  0.00 & 100.00\\
instance n=20 215.alb & 1 & 1 & Solution & 120.03 & 5 &  0.00 & 100.00\\
instance n=20 216.alb & 1 & 1 & Solution & 120.04 & 3 &  0.00 & 100.00\\
instance n=20 217.alb & 1 & 1 & Optimal & 28.60 & 4 &  0.00 & 100.00\\
instance n=20 218.alb & 1 & 1 & Optimal & 17.56 & 3 &  0.00 & 100.00\\
instance n=20 219.alb & 1 & 1 & Solution & 120.02 & 3 &  0.00 & 100.00\\
instance n=20 22.alb & 1 & 1 & Solution & 120.04 & 12 &  0.00 & 100.00\\
instance n=20 220.alb & 1 & 1 & Optimal & 84.78 & 3 &  0.00 & 100.00\\
instance n=20 221.alb & 1 & 1 & Optimal & 19.28 & 3 &  0.00 & 100.00\\
instance n=20 222.alb & 1 & 1 & Optimal & 32.96 & 3 &  0.00 & 100.00\\
instance n=20 223.alb & 1 & 1 & Optimal & 30.98 & 3 &  0.00 & 100.00\\
instance n=20 224.alb & 1 & 1 & Optimal & 47.45 & 3 &  0.00 & 100.00\\
instance n=20 225.alb & 1 & 1 & Optimal & 57.04 & 3 &  0.00 & 100.00\\
instance n=20 226.alb & 1 & 1 & Solution & 120.03 & 3 &  0.00 & 100.00\\
instance n=20 227.alb & 1 & 1 & Optimal & 60.25 & 3 &  0.00 & 100.00\\
instance n=20 228.alb & 1 & 1 & Solution & 120.05 & 2 &  0.00 & 100.00\\
instance n=20 229.alb & 1 & 1 & Optimal & 23.15 & 3 &  0.00 & 100.00\\
instance n=20 23.alb & 1 & 1 & Solution & 120.03 & 13 &  0.00 & 100.00\\
instance n=20 230.alb & 1 & 1 & Optimal & 57.49 & 3 &  0.00 & 100.00\\
instance n=20 231.alb & 1 & 1 & Solution & 120.01 & 3 &  0.00 & 100.00\\
instance n=20 232.alb & 1 & 1 & Optimal & 32.64 & 3 &  0.00 & 100.00\\
instance n=20 233.alb & 1 & 1 & Optimal & 60.68 & 3 &  0.00 & 100.00\\
instance n=20 234.alb & 1 & 1 & Optimal & 39.81 & 3 &  0.00 & 100.00\\
instance n=20 235.alb & 1 & 1 & Optimal & 68.59 & 3 &  0.00 & 100.00\\
instance n=20 236.alb & 1 & 1 & Optimal & 51.20 & 3 &  0.00 & 100.00\\
instance n=20 237.alb & 1 & 1 & Optimal & 15.58 & 3 &  0.00 & 100.00\\
instance n=20 238.alb & 1 & 1 & Optimal &  6.20 & 3 &  0.00 & 100.00\\
instance n=20 239.alb & 1 & 1 & Optimal & 18.25 & 3 &  0.00 & 100.00\\
instance n=20 24.alb & 1 & 1 & Solution & 120.03 & 11 &  0.00 & 100.00\\
instance n=20 240.alb & 1 & 1 & Optimal & 36.93 & 3 &  0.00 & 100.00\\
instance n=20 241.alb & 1 & 1 & Optimal &  3.13 & 13 &  0.00 & 100.00\\
instance n=20 242.alb & 1 & 1 & Optimal &  0.92 & 12 &  0.00 & 100.00\\
instance n=20 243.alb & 1 & 1 & Optimal & 21.28 & 10 &  0.00 & 100.00\\
instance n=20 244.alb & 1 & 1 & Optimal &  2.05 & 11 &  0.00 & 100.00\\
instance n=20 245.alb & 1 & 1 & Optimal &  2.98 & 13 &  0.00 & 100.00\\
instance n=20 246.alb & 1 & 1 & Optimal & 11.53 & 13 &  0.00 & 100.00\\
instance n=20 247.alb & 1 & 1 & Optimal & 16.26 & 11 &  0.00 & 100.00\\
instance n=20 248.alb & 1 & 1 & Optimal &  2.76 & 11 &  0.00 & 100.00\\
instance n=20 249.alb & 1 & 1 & Optimal &  9.09 & 13 &  0.00 & 100.00\\
instance n=20 25.alb & 1 & 1 & Optimal & 36.53 & 11 &  0.00 & 100.00\\
instance n=20 250.alb & 1 & 1 & Optimal &  9.09 & 10 &  0.00 & 100.00\\
instance n=20 251.alb & 1 & 1 & Optimal &  2.09 & 12 &  0.00 & 100.00\\
instance n=20 252.alb & 1 & 1 & Optimal &  2.53 & 11 &  0.00 & 100.00\\
instance n=20 253.alb & 1 & 1 & Optimal &  6.97 & 13 &  0.00 & 100.00\\
instance n=20 254.alb & 1 & 1 & Optimal &  2.12 & 12 &  0.00 & 100.00\\
instance n=20 255.alb & 1 & 1 & Optimal &  8.86 & 13 &  0.00 & 100.00\\
instance n=20 256.alb & 1 & 1 & Optimal &  2.45 & 14 &  0.00 & 100.00\\
instance n=20 257.alb & 1 & 1 & Optimal & 52.13 & 10 &  0.00 & 100.00\\
instance n=20 258.alb & 1 & 1 & Optimal &  2.07 & 13 &  0.00 & 100.00\\
instance n=20 259.alb & 1 & 1 & Optimal &  4.31 & 13 &  0.00 & 100.00\\
instance n=20 26.alb & 1 & 1 & Solution & 120.04 & 12 &  0.00 & 100.00\\
instance n=20 260.alb & 1 & 1 & Optimal & 13.18 & 12 &  0.00 & 100.00\\
instance n=20 261.alb & 1 & 1 & Optimal &  1.86 & 12 &  0.00 & 100.00\\
instance n=20 262.alb & 1 & 1 & Optimal &  1.92 & 11 &  0.00 & 100.00\\
instance n=20 263.alb & 1 & 1 & Optimal &  2.27 & 12 &  0.00 & 100.00\\
instance n=20 264.alb & 1 & 1 & Optimal & 15.42 & 12 &  0.00 & 100.00\\
instance n=20 265.alb & 1 & 1 & Optimal &  2.27 & 12 &  0.00 & 100.00\\
instance n=20 266.alb & 1 & 1 & Optimal &  2.16 & 5 &  0.00 & 100.00\\
instance n=20 267.alb & 1 & 1 & Optimal &  0.92 & 6 &  0.00 & 100.00\\
instance n=20 268.alb & 1 & 1 & Optimal &  1.87 & 6 &  0.00 & 100.00\\
instance n=20 269.alb & 1 & 1 & Optimal & 10.02 & 7 &  0.00 & 100.00\\
instance n=20 27.alb & 1 & 1 & Solution & 120.04 & 13 &  0.00 & 100.00\\
instance n=20 270.alb & 1 & 1 & Optimal &  3.14 & 7 &  0.00 & 100.00\\
instance n=20 271.alb & 1 & 1 & Optimal &  0.93 & 6 &  0.00 & 100.00\\
instance n=20 272.alb & 1 & 1 & Optimal &  1.53 & 5 &  0.00 & 100.00\\
instance n=20 273.alb & 1 & 1 & Optimal &  1.57 & 5 &  0.00 & 100.00\\
instance n=20 274.alb & 1 & 1 & Optimal &  1.77 & 6 &  0.00 & 100.00\\
instance n=20 275.alb & 1 & 1 & Optimal &  4.30 & 5 &  0.00 & 100.00\\
instance n=20 276.alb & 1 & 1 & Optimal &  4.41 & 4 &  0.00 & 100.00\\
instance n=20 277.alb & 1 & 1 & Optimal &  6.43 & 4 &  0.00 & 100.00\\
instance n=20 278.alb & 1 & 1 & Optimal &  5.13 & 6 &  0.00 & 100.00\\
instance n=20 279.alb & 1 & 1 & Optimal &  1.85 & 6 &  0.00 & 100.00\\
instance n=20 28.alb & 1 & 1 & Solution & 120.04 & 12 &  0.00 & 100.00\\
instance n=20 280.alb & 1 & 1 & Optimal &  3.01 & 5 &  0.00 & 100.00\\
instance n=20 281.alb & 1 & 1 & Optimal & 11.62 & 4 &  0.00 & 100.00\\
instance n=20 282.alb & 1 & 1 & Optimal &  6.61 & 4 &  0.00 & 100.00\\
instance n=20 283.alb & 1 & 1 & Optimal &  4.19 & 5 &  0.00 & 100.00\\
instance n=20 284.alb & 1 & 1 & Optimal &  2.48 & 5 &  0.00 & 100.00\\
instance n=20 285.alb & 1 & 1 & Optimal &  5.32 & 5 &  0.00 & 100.00\\
instance n=20 286.alb & 1 & 1 & Optimal &  1.91 & 5 &  0.00 & 100.00\\
instance n=20 287.alb & 1 & 1 & Optimal &  4.14 & 5 &  0.00 & 100.00\\
instance n=20 288.alb & 1 & 1 & Optimal &  2.47 & 6 &  0.00 & 100.00\\
instance n=20 289.alb & 1 & 1 & Optimal &  4.50 & 5 &  0.00 & 100.00\\
instance n=20 29.alb & 1 & 1 & Solution & 120.04 & 10 &  0.00 & 100.00\\
instance n=20 290.alb & 1 & 1 & Optimal &  1.18 & 5 &  0.00 & 100.00\\
instance n=20 291.alb & 1 & 1 & Solution & 120.03 & 3 &  0.00 & 100.00\\
instance n=20 292.alb & 1 & 1 & Solution & 120.03 & 3 &  0.00 & 100.00\\
instance n=20 293.alb & 1 & 1 & Solution & 120.03 & 3 &  0.00 & 100.00\\
instance n=20 294.alb & 1 & 1 & Solution & 120.03 & 3 &  0.00 & 100.00\\
instance n=20 295.alb & 1 & 1 & Solution & 120.03 & 3 &  0.00 & 100.00\\
instance n=20 296.alb & 1 & 1 & Solution & 120.02 & 3 &  0.00 & 100.00\\
instance n=20 297.alb & 1 & 1 & Solution & 120.04 & 3 &  0.00 & 100.00\\
instance n=20 298.alb & 1 & 1 & Solution & 120.03 & 3 &  0.00 & 100.00\\
instance n=20 299.alb & 1 & 1 & Solution & 120.03 & 3 &  0.00 & 100.00\\
instance n=20 3.alb & 1 & 1 & Solution & 120.02 & 3 &  0.00 & 100.00\\
instance n=20 30.alb & 1 & 1 & Solution & 120.03 & 16 &  0.00 & 100.00\\
instance n=20 300.alb & 1 & 1 & Solution & 120.01 & 4 &  0.00 & 100.00\\
instance n=20 301.alb & 1 & 1 & Solution & 120.03 & 3 &  0.00 & 100.00\\
instance n=20 302.alb & 1 & 1 & Solution & 120.03 & 3 &  0.00 & 100.00\\
instance n=20 303.alb & 1 & 1 & Solution & 120.02 & 3 &  0.00 & 100.00\\
instance n=20 304.alb & 1 & 1 & Solution & 120.03 & 3 &  0.00 & 100.00\\
instance n=20 305.alb & 1 & 1 & Solution & 120.03 & 3 &  0.00 & 100.00\\
instance n=20 306.alb & 1 & 1 & Solution & 120.04 & 3 &  0.00 & 100.00\\
instance n=20 307.alb & 1 & 1 & Solution & 120.03 & 3 &  0.00 & 100.00\\
instance n=20 308.alb & 1 & 1 & Solution & 120.02 & 3 &  0.00 & 100.00\\
instance n=20 309.alb & 1 & 1 & Solution & 120.03 & 3 &  0.00 & 100.00\\
instance n=20 31.alb & 1 & 1 & Solution & 120.04 & 12 &  0.00 & 100.00\\
instance n=20 310.alb & 1 & 1 & Solution & 120.03 & 3 &  0.00 & 100.00\\
instance n=20 311.alb & 1 & 1 & Solution & 120.04 & 3 &  0.00 & 100.00\\
instance n=20 312.alb & 1 & 1 & Solution & 120.03 & 4 &  0.00 & 100.00\\
instance n=20 313.alb & 1 & 1 & Solution & 120.02 & 3 &  0.00 & 100.00\\
instance n=20 314.alb & 1 & 1 & Solution & 120.03 & 3 &  0.00 & 100.00\\
instance n=20 315.alb & 1 & 1 & Solution & 120.03 & 3 &  0.00 & 100.00\\
instance n=20 316.alb & 1 & 1 & Solution & 120.03 & 10 &  0.00 & 100.00\\
instance n=20 317.alb & 1 & 1 & Solution & 120.04 & 10 &  0.00 & 100.00\\
instance n=20 318.alb & 1 & 1 & Optimal & 109.73 & 10 &  0.00 & 100.00\\
instance n=20 319.alb & 1 & 1 & Solution & 120.04 & 14 &  0.00 & 100.00\\
instance n=20 32.alb & 1 & 1 & Solution & 120.04 & 13 &  0.00 & 100.00\\
instance n=20 320.alb & 1 & 1 & Optimal & 112.73 & 12 &  0.00 & 100.00\\
instance n=20 321.alb & 1 & 1 & Solution & 120.04 & 14 &  0.00 & 100.00\\
instance n=20 322.alb & 1 & 1 & Solution & 120.03 & 12 &  0.00 & 100.00\\
instance n=20 323.alb & 1 & 1 & Solution & 120.04 & 13 &  0.00 & 100.00\\
instance n=20 324.alb & 1 & 1 & Solution & 120.04 & 9 &  0.00 & 100.00\\
instance n=20 325.alb & 1 & 1 & Solution & 120.04 & 14 &  0.00 & 100.00\\
instance n=20 326.alb & 1 & 1 & Solution & 120.03 & 14 &  0.00 & 100.00\\
instance n=20 327.alb & 1 & 1 & Solution & 120.03 & 13 &  0.00 & 100.00\\
instance n=20 328.alb & 1 & 1 & Solution & 120.03 & 13 &  0.00 & 100.00\\
instance n=20 329.alb & 1 & 1 & Solution & 120.03 & 10 &  0.00 & 100.00\\
instance n=20 33.alb & 1 & 1 & Optimal & 27.23 & 11 &  0.00 & 100.00\\
instance n=20 330.alb & 1 & 1 & Solution & 120.04 & 12 &  0.00 & 100.00\\
instance n=20 331.alb & 1 & 1 & Solution & 120.04 & 13 &  0.00 & 100.00\\
instance n=20 332.alb & 1 & 1 & Solution & 120.04 & 13 &  0.00 & 100.00\\
instance n=20 333.alb & 1 & 1 & Optimal & 41.21 & 11 &  0.00 & 100.00\\
instance n=20 334.alb & 1 & 1 & Solution & 120.03 & 10 &  0.00 & 100.00\\
instance n=20 335.alb & 1 & 1 & Solution & 120.04 & 14 &  0.00 & 100.00\\
instance n=20 336.alb & 1 & 1 & Optimal & 32.17 & 11 &  0.00 & 100.00\\
instance n=20 337.alb & 1 & 1 & Optimal & 81.96 & 10 &  0.00 & 100.00\\
instance n=20 338.alb & 1 & 1 & Solution & 120.04 & 14 &  0.00 & 100.00\\
instance n=20 339.alb & 1 & 1 & Solution & 120.03 & 13 &  0.00 & 100.00\\
instance n=20 34.alb & 1 & 1 & Solution & 120.04 & 12 &  0.00 & 100.00\\
instance n=20 340.alb & 1 & 1 & Optimal & 84.19 & 11 &  0.00 & 100.00\\
instance n=20 341.alb & 1 & 1 & Solution & 120.03 & 6 &  0.00 & 100.00\\
instance n=20 342.alb & 1 & 1 & Solution & 120.04 & 6 &  0.00 & 100.00\\
instance n=20 343.alb & 1 & 1 & Solution & 120.03 & 6 &  0.00 & 100.00\\
instance n=20 344.alb & 1 & 1 & Solution & 120.04 & 6 &  0.00 & 100.00\\
instance n=20 345.alb & 1 & 1 & Solution & 120.03 & 4 &  0.00 & 100.00\\
instance n=20 346.alb & 1 & 1 & Solution & 120.04 & 5 &  0.00 & 100.00\\
instance n=20 347.alb & 1 & 1 & Solution & 120.03 & 6 &  0.00 & 100.00\\
instance n=20 348.alb & 1 & 1 & Solution & 120.03 & 5 &  0.00 & 100.00\\
instance n=20 349.alb & 1 & 1 & Solution & 120.04 & 5 &  0.00 & 100.00\\
instance n=20 35.alb & 1 & 1 & Optimal & 69.47 & 12 &  0.00 & 100.00\\
instance n=20 350.alb & 1 & 1 & Solution & 120.06 & 5 &  0.00 & 100.00\\
instance n=20 351.alb & 1 & 1 & Solution & 120.03 & 5 &  0.00 & 100.00\\
instance n=20 352.alb & 1 & 1 & Solution & 120.03 & 4 &  0.00 & 100.00\\
instance n=20 353.alb & 1 & 1 & Optimal & 58.06 & 6 &  0.00 & 100.00\\
instance n=20 354.alb & 1 & 1 & Solution & 120.04 & 6 &  0.00 & 100.00\\
instance n=20 355.alb & 1 & 1 & Solution & 120.05 & 5 &  0.00 & 100.00\\
instance n=20 356.alb & 1 & 1 & Solution & 120.02 & 5 &  0.00 & 100.00\\
instance n=20 357.alb & 1 & 1 & Solution & 120.03 & 5 &  0.00 & 100.00\\
instance n=20 358.alb & 1 & 1 & Solution & 120.05 & 4 &  0.00 & 100.00\\
instance n=20 359.alb & 1 & 1 & Solution & 120.02 & 4 &  0.00 & 100.00\\
instance n=20 36.alb & 1 & 1 & Optimal & 61.68 & 13 &  0.00 & 100.00\\
instance n=20 360.alb & 1 & 1 & Solution & 120.04 & 6 &  0.00 & 100.00\\
instance n=20 361.alb & 1 & 1 & Solution & 120.04 & 5 &  0.00 & 100.00\\
instance n=20 362.alb & 1 & 1 & Solution & 120.03 & 5 &  0.00 & 100.00\\
instance n=20 363.alb & 1 & 1 & Solution & 120.03 & 7 &  0.00 & 100.00\\
instance n=20 364.alb & 1 & 1 & Solution & 120.02 & 4 &  0.00 & 100.00\\
instance n=20 365.alb & 1 & 1 & Solution & 120.04 & 5 &  0.00 & 100.00\\
instance n=20 366.alb & 1 & 1 & Solution & 120.04 & 3 &  0.00 & 100.00\\
instance n=20 367.alb & 1 & 1 & Solution & 120.03 & 3 &  0.00 & 100.00\\
instance n=20 368.alb & 1 & 1 & Optimal & 57.62 & 3 &  0.00 & 100.00\\
instance n=20 369.alb & 1 & 1 & Solution & 120.03 & 3 &  0.00 & 100.00\\
instance n=20 37.alb & 1 & 1 & Solution & 120.04 & 12 &  0.00 & 100.00\\
instance n=20 370.alb & 1 & 1 & Optimal & 92.47 & 3 &  0.00 & 100.00\\
instance n=20 371.alb & 1 & 1 & Optimal & 109.61 & 3 &  0.00 & 100.00\\
instance n=20 372.alb & 1 & 1 & Solution & 120.04 & 3 &  0.00 & 100.00\\
instance n=20 373.alb & 1 & 1 & Solution & 120.05 & 3 &  0.00 & 100.00\\
instance n=20 374.alb & 1 & 1 & Solution & 120.04 & 3 &  0.00 & 100.00\\
instance n=20 375.alb & 1 & 1 & Solution & 120.04 & 3 &  0.00 & 100.00\\
instance n=20 376.alb & 1 & 1 & Solution & 120.03 & 3 &  0.00 & 100.00\\
instance n=20 377.alb & 1 & 1 & Solution & 120.03 & 3 &  0.00 & 100.00\\
instance n=20 378.alb & 1 & 1 & Solution & 120.04 & 3 &  0.00 & 100.00\\
instance n=20 379.alb & 1 & 1 & Solution & 120.03 & 4 &  0.00 & 100.00\\
instance n=20 38.alb & 1 & 1 & Optimal & 17.94 & 12 &  0.00 & 100.00\\
instance n=20 380.alb & 1 & 1 & Solution & 120.04 & 3 &  0.00 & 100.00\\
instance n=20 381.alb & 1 & 1 & Solution & 120.03 & 3 &  0.00 & 100.00\\
instance n=20 382.alb & 1 & 1 & Optimal & 28.34 & 4 &  0.00 & 100.00\\
instance n=20 383.alb & 1 & 1 & Optimal & 104.69 & 3 &  0.00 & 100.00\\
instance n=20 384.alb & 1 & 1 & Solution & 120.03 & 3 &  0.00 & 100.00\\
instance n=20 385.alb & 1 & 1 & Solution & 120.03 & 3 &  0.00 & 100.00\\
instance n=20 386.alb & 1 & 1 & Solution & 120.02 & 3 &  0.00 & 100.00\\
instance n=20 387.alb & 1 & 1 & Solution & 120.02 & 3 &  0.00 & 100.00\\
instance n=20 388.alb & 1 & 1 & Optimal & 47.41 & 3 &  0.00 & 100.00\\
instance n=20 389.alb & 1 & 1 & Optimal & 33.53 & 3 &  0.00 & 100.00\\
instance n=20 39.alb & 1 & 1 & Solution & 120.04 & 13 &  0.00 & 100.00\\
instance n=20 390.alb & 1 & 1 & Solution & 120.03 & 3 &  0.00 & 100.00\\
instance n=20 391.alb & 1 & 1 & Optimal &  1.66 & 11 &  0.00 & 100.00\\
instance n=20 392.alb & 1 & 1 & Optimal & 15.49 & 14 &  0.00 & 100.00\\
instance n=20 393.alb & 1 & 1 & Optimal &  2.20 & 11 &  0.00 & 100.00\\
instance n=20 394.alb & 1 & 1 & Optimal &  5.89 & 12 &  0.00 & 100.00\\
instance n=20 395.alb & 1 & 1 & Optimal &  3.13 & 12 &  0.00 & 100.00\\
instance n=20 396.alb & 1 & 1 & Optimal &  7.57 & 13 &  0.00 & 100.00\\
instance n=20 397.alb & 1 & 1 & Optimal & 11.91 & 10 &  0.00 & 100.00\\
instance n=20 398.alb & 1 & 1 & Optimal &  1.72 & 11 &  0.00 & 100.00\\
instance n=20 399.alb & 1 & 1 & Optimal &  2.61 & 13 &  0.00 & 100.00\\
instance n=20 4.alb & 1 & 1 & Solution & 120.04 & 3 &  0.00 & 100.00\\
instance n=20 40.alb & 1 & 1 & Solution & 120.03 & 12 &  0.00 & 100.00\\
instance n=20 400.alb & 1 & 1 & Optimal & 17.42 & 12 &  0.00 & 100.00\\
instance n=20 401.alb & 1 & 1 & Optimal & 17.61 & 12 &  0.00 & 100.00\\
instance n=20 402.alb & 1 & 1 & Optimal &  6.33 & 12 &  0.00 & 100.00\\
instance n=20 403.alb & 1 & 1 & Optimal &  2.96 & 12 &  0.00 & 100.00\\
instance n=20 404.alb & 1 & 1 & Optimal & 14.61 & 10 &  0.00 & 100.00\\
instance n=20 405.alb & 1 & 1 & Optimal &  1.94 & 12 &  0.00 & 100.00\\
instance n=20 406.alb & 1 & 1 & Optimal & 29.91 & 14 &  0.00 & 100.00\\
instance n=20 407.alb & 1 & 1 & Optimal &  2.82 & 10 &  0.00 & 100.00\\
instance n=20 408.alb & 1 & 1 & Optimal & 31.09 & 14 &  0.00 & 100.00\\
instance n=20 409.alb & 1 & 1 & Optimal & 10.23 & 12 &  0.00 & 100.00\\
instance n=20 41.alb & 1 & 1 & Solution & 120.02 & 6 &  0.00 & 100.00\\
instance n=20 410.alb & 1 & 1 & Optimal &  7.26 & 11 &  0.00 & 100.00\\
instance n=20 411.alb & 1 & 1 & Optimal & 40.12 & 15 &  0.00 & 100.00\\
instance n=20 412.alb & 1 & 1 & Optimal &  2.40 & 11 &  0.00 & 100.00\\
instance n=20 413.alb & 1 & 1 & Optimal &  1.97 & 10 &  0.00 & 100.00\\
instance n=20 414.alb & 1 & 1 & Optimal & 33.57 & 12 &  0.00 & 100.00\\
instance n=20 415.alb & 1 & 1 & Optimal & 14.81 & 10 &  0.00 & 100.00\\
instance n=20 416.alb & 1 & 1 & Optimal &  2.76 & 6 &  0.00 & 100.00\\
instance n=20 417.alb & 1 & 1 & Optimal & 13.70 & 5 &  0.00 & 100.00\\
instance n=20 418.alb & 1 & 1 & Optimal &  1.25 & 6 &  0.00 & 100.00\\
instance n=20 419.alb & 1 & 1 & Optimal & 51.83 & 4 &  0.00 & 100.00\\
instance n=20 42.alb & 1 & 1 & Solution & 120.02 & 5 &  0.00 & 100.00\\
instance n=20 420.alb & 1 & 1 & Optimal & 14.42 & 5 &  0.00 & 100.00\\
instance n=20 421.alb & 1 & 1 & Optimal &  1.55 & 6 &  0.00 & 100.00\\
instance n=20 422.alb & 1 & 1 & Optimal & 15.77 & 4 &  0.00 & 100.00\\
instance n=20 423.alb & 1 & 1 & Optimal &  2.62 & 6 &  0.00 & 100.00\\
instance n=20 424.alb & 1 & 1 & Optimal & 28.03 & 5 &  0.00 & 100.00\\
instance n=20 425.alb & 1 & 1 & Optimal &  1.38 & 6 &  0.00 & 100.00\\
instance n=20 426.alb & 1 & 1 & Optimal &  1.25 & 5 &  0.00 & 100.00\\
instance n=20 427.alb & 1 & 1 & Optimal &  9.16 & 6 &  0.00 & 100.00\\
instance n=20 428.alb & 1 & 1 & Optimal &  5.19 & 5 &  0.00 & 100.00\\
instance n=20 429.alb & 1 & 1 & Optimal & 10.86 & 4 &  0.00 & 100.00\\
instance n=20 43.alb & 1 & 1 & Solution & 120.03 & 5 &  0.00 & 100.00\\
instance n=20 430.alb & 1 & 1 & Optimal &  4.60 & 5 &  0.00 & 100.00\\
instance n=20 431.alb & 1 & 1 & Optimal &  2.63 & 6 &  0.00 & 100.00\\
instance n=20 432.alb & 1 & 1 & Optimal &  1.77 & 5 &  0.00 & 100.00\\
instance n=20 433.alb & 1 & 1 & Optimal &  2.90 & 5 &  0.00 & 100.00\\
instance n=20 434.alb & 1 & 1 & Optimal &  3.19 & 5 &  0.00 & 100.00\\
instance n=20 435.alb & 1 & 1 & Optimal &  1.10 & 7 &  0.00 & 100.00\\
instance n=20 436.alb & 1 & 1 & Optimal &  6.05 & 5 &  0.00 & 100.00\\
instance n=20 437.alb & 1 & 1 & Optimal &  3.26 & 5 &  0.00 & 100.00\\
instance n=20 438.alb & 1 & 1 & Optimal &  2.33 & 6 &  0.00 & 100.00\\
instance n=20 439.alb & 1 & 1 & Optimal &  1.46 & 5 &  0.00 & 100.00\\
instance n=20 44.alb & 1 & 1 & Solution & 120.04 & 5 &  0.00 & 100.00\\
instance n=20 440.alb & 1 & 1 & Optimal &  4.30 & 5 &  0.00 & 100.00\\
instance n=20 441.alb & 1 & 1 & Optimal &  0.83 & 3 &  0.00 & 100.00\\
instance n=20 442.alb & 1 & 1 & Optimal &  1.08 & 3 &  0.00 & 100.00\\
instance n=20 443.alb & 1 & 1 & Optimal &  0.81 & 3 &  0.00 & 100.00\\
instance n=20 444.alb & 1 & 1 & Optimal &  1.35 & 3 &  0.00 & 100.00\\
instance n=20 445.alb & 1 & 1 & Optimal &  0.80 & 3 &  0.00 & 100.00\\
instance n=20 446.alb & 1 & 1 & Optimal &  0.91 & 3 &  0.00 & 100.00\\
instance n=20 447.alb & 1 & 1 & Optimal &  0.86 & 3 &  0.00 & 100.00\\
instance n=20 448.alb & 1 & 1 & Optimal &  0.89 & 3 &  0.00 & 100.00\\
instance n=20 449.alb & 1 & 1 & Optimal &  1.01 & 3 &  0.00 & 100.00\\
instance n=20 45.alb & 1 & 1 & Solution & 120.04 & 6 &  0.00 & 100.00\\
instance n=20 450.alb & 1 & 1 & Optimal &  0.93 & 3 &  0.00 & 100.00\\
instance n=20 451.alb & 1 & 1 & Optimal &  0.82 & 3 &  0.00 & 100.00\\
instance n=20 452.alb & 1 & 1 & Optimal &  0.92 & 3 &  0.00 & 100.00\\
instance n=20 453.alb & 1 & 1 & Optimal &  0.69 & 3 &  0.00 & 100.00\\
instance n=20 454.alb & 1 & 1 & Optimal &  1.12 & 3 &  0.00 & 100.00\\
instance n=20 455.alb & 1 & 1 & Optimal &  0.75 & 3 &  0.00 & 100.00\\
instance n=20 456.alb & 1 & 1 & Optimal &  0.76 & 4 &  0.00 & 100.00\\
instance n=20 457.alb & 1 & 1 & Optimal &  0.77 & 3 &  0.00 & 100.00\\
instance n=20 458.alb & 1 & 1 & Optimal &  0.87 & 3 &  0.00 & 100.00\\
instance n=20 459.alb & 1 & 1 & Optimal &  0.84 & 3 &  0.00 & 100.00\\
instance n=20 46.alb & 1 & 1 & Solution & 120.02 & 4 &  0.00 & 100.00\\
instance n=20 460.alb & 1 & 1 & Optimal &  0.83 & 3 &  0.00 & 100.00\\
instance n=20 461.alb & 1 & 1 & Optimal &  0.67 & 3 &  0.00 & 100.00\\
instance n=20 462.alb & 1 & 1 & Optimal &  0.72 & 3 &  0.00 & 100.00\\
instance n=20 463.alb & 1 & 1 & Optimal &  1.25 & 3 &  0.00 & 100.00\\
instance n=20 464.alb & 1 & 1 & Optimal &  1.23 & 3 &  0.00 & 100.00\\
instance n=20 465.alb & 1 & 1 & Optimal &  0.77 & 3 &  0.00 & 100.00\\
instance n=20 466.alb & 1 & 1 & Optimal &  0.61 & 13 &  0.00 & 100.00\\
instance n=20 467.alb & 1 & 1 & Optimal &  0.63 & 14 &  0.00 & 100.00\\
instance n=20 468.alb & 1 & 1 & Optimal &  0.67 & 13 &  0.00 & 100.00\\
instance n=20 469.alb & 1 & 1 & Optimal &  0.68 & 14 &  0.00 & 100.00\\
instance n=20 47.alb & 1 & 1 & Solution & 120.03 & 4 &  0.00 & 100.00\\
instance n=20 470.alb & 1 & 1 & Optimal &  0.64 & 12 &  0.00 & 100.00\\
instance n=20 471.alb & 1 & 1 & Optimal &  0.67 & 12 &  0.00 & 100.00\\
instance n=20 472.alb & 1 & 1 & Optimal &  0.74 & 13 &  0.00 & 100.00\\
instance n=20 473.alb & 1 & 1 & Optimal &  0.64 & 10 &  0.00 & 100.00\\
instance n=20 474.alb & 1 & 1 & Optimal &  0.78 & 14 &  0.00 & 100.00\\
instance n=20 475.alb & 1 & 1 & Optimal &  0.65 & 11 &  0.00 & 100.00\\
instance n=20 476.alb & 1 & 1 & Optimal &  0.70 & 11 &  0.00 & 100.00\\
instance n=20 477.alb & 1 & 1 & Optimal &  0.76 & 11 &  0.00 & 100.00\\
instance n=20 478.alb & 1 & 1 & Optimal &  0.63 & 12 &  0.00 & 100.00\\
instance n=20 479.alb & 1 & 1 & Optimal &  0.61 & 13 &  0.00 & 100.00\\
instance n=20 48.alb & 1 & 1 & Solution & 120.02 & 5 &  0.00 & 100.00\\
instance n=20 480.alb & 1 & 1 & Optimal &  0.65 & 13 &  0.00 & 100.00\\
instance n=20 481.alb & 1 & 1 & Optimal &  0.72 & 13 &  0.00 & 100.00\\
instance n=20 482.alb & 1 & 1 & Optimal &  0.57 & 13 &  0.00 & 100.00\\
instance n=20 483.alb & 1 & 1 & Optimal &  0.67 & 12 &  0.00 & 100.00\\
instance n=20 484.alb & 1 & 1 & Optimal &  0.70 & 13 &  0.00 & 100.00\\
instance n=20 485.alb & 1 & 1 & Optimal &  1.45 & 15 &  0.00 & 100.00\\
instance n=20 486.alb & 1 & 1 & Optimal &  0.74 & 11 &  0.00 & 100.00\\
instance n=20 487.alb & 1 & 1 & Optimal &  0.72 & 12 &  0.00 & 100.00\\
instance n=20 488.alb & 1 & 1 & Optimal &  0.75 & 15 &  0.00 & 100.00\\
instance n=20 489.alb & 1 & 1 & Optimal &  0.64 & 12 &  0.00 & 100.00\\
instance n=20 49.alb & 1 & 1 & Solution & 120.04 & 4 &  0.00 & 100.00\\
instance n=20 490.alb & 1 & 1 & Optimal &  0.73 & 12 &  0.00 & 100.00\\
instance n=20 491.alb & 1 & 1 & Optimal &  0.61 & 6 &  0.00 & 100.00\\
instance n=20 492.alb & 1 & 1 & Optimal &  0.62 & 5 &  0.00 & 100.00\\
instance n=20 493.alb & 1 & 1 & Optimal &  0.64 & 5 &  0.00 & 100.00\\
instance n=20 494.alb & 1 & 1 & Optimal &  0.63 & 6 &  0.00 & 100.00\\
instance n=20 495.alb & 1 & 1 & Optimal &  0.59 & 6 &  0.00 & 100.00\\
instance n=20 496.alb & 1 & 1 & Optimal &  0.64 & 5 &  0.00 & 100.00\\
instance n=20 497.alb & 1 & 1 & Optimal &  0.73 & 6 &  0.00 & 100.00\\
instance n=20 498.alb & 1 & 1 & Optimal &  0.63 & 6 &  0.00 & 100.00\\
instance n=20 499.alb & 1 & 1 & Optimal &  0.70 & 5 &  0.00 & 100.00\\
instance n=20 5.alb & 1 & 1 & Solution & 120.03 & 3 &  0.00 & 100.00\\
instance n=20 50.alb & 1 & 1 & Solution & 120.04 & 4 &  0.00 & 100.00\\
instance n=20 500.alb & 1 & 1 & Optimal &  0.62 & 8 &  0.00 & 100.00\\
instance n=20 501.alb & 1 & 1 & Optimal &  0.78 & 5 &  0.00 & 100.00\\
instance n=20 502.alb & 1 & 1 & Optimal &  0.85 & 4 &  0.00 & 100.00\\
instance n=20 503.alb & 1 & 1 & Optimal &  0.76 & 6 &  0.00 & 100.00\\
instance n=20 504.alb & 1 & 1 & Optimal &  0.71 & 6 &  0.00 & 100.00\\
instance n=20 505.alb & 1 & 1 & Optimal &  0.76 & 6 &  0.00 & 100.00\\
instance n=20 506.alb & 1 & 1 & Optimal &  0.68 & 5 &  0.00 & 100.00\\
instance n=20 507.alb & 1 & 1 & Optimal &  0.58 & 5 &  0.00 & 100.00\\
instance n=20 508.alb & 1 & 1 & Optimal &  0.79 & 5 &  0.00 & 100.00\\
instance n=20 509.alb & 1 & 1 & Optimal &  0.67 & 4 &  0.00 & 100.00\\
instance n=20 51.alb & 1 & 1 & Solution & 120.02 & 4 &  0.00 & 100.00\\
instance n=20 510.alb & 1 & 1 & Optimal &  0.62 & 5 &  0.00 & 100.00\\
instance n=20 511.alb & 1 & 1 & Optimal &  0.73 & 5 &  0.00 & 100.00\\
instance n=20 512.alb & 1 & 1 & Optimal &  0.68 & 5 &  0.00 & 100.00\\
instance n=20 513.alb & 1 & 1 & Optimal &  0.76 & 5 &  0.00 & 100.00\\
instance n=20 514.alb & 1 & 1 & Optimal &  0.62 & 5 &  0.00 & 100.00\\
instance n=20 515.alb & 1 & 1 & Optimal &  0.63 & 6 &  0.00 & 100.00\\
instance n=20 516.alb & 1 & 1 & Solution & 120.03 & 3 &  0.00 & 100.00\\
instance n=20 517.alb & 1 & 1 & Solution & 120.03 & 3 &  0.00 & 100.00\\
instance n=20 518.alb & 1 & 1 & Solution & 120.04 & 3 &  0.00 & 100.00\\
instance n=20 519.alb & 1 & 1 & Solution & 120.04 & 3 &  0.00 & 100.00\\
instance n=20 52.alb & 1 & 1 & Solution & 120.04 & 4 &  0.00 & 100.00\\
instance n=20 520.alb & 1 & 1 & Solution & 120.02 & 3 &  0.00 & 100.00\\
instance n=20 521.alb & 1 & 1 & Solution & 120.03 & 3 &  0.00 & 100.00\\
instance n=20 522.alb & 1 & 1 & Solution & 120.04 & 3 &  0.00 & 100.00\\
instance n=20 523.alb & 1 & 1 & Solution & 120.04 & 3 &  0.00 & 100.00\\
instance n=20 524.alb & 1 & 1 & Solution & 120.03 & 3 &  0.00 & 100.00\\
instance n=20 525.alb & 1 & 1 & Solution & 120.02 & 3 &  0.00 & 100.00\\
instance n=20 53.alb & 1 & 1 & Optimal & 49.46 & 5 &  0.00 & 100.00\\
instance n=20 54.alb & 1 & 1 & Solution & 120.04 & 5 &  0.00 & 100.00\\
instance n=20 55.alb & 1 & 1 & Solution & 120.04 & 5 &  0.00 & 100.00\\
instance n=20 56.alb & 1 & 1 & Solution & 120.03 & 4 &  0.00 & 100.00\\
instance n=20 57.alb & 1 & 1 & Solution & 120.04 & 4 &  0.00 & 100.00\\
instance n=20 58.alb & 1 & 1 & Solution & 120.03 & 5 &  0.00 & 100.00\\
instance n=20 59.alb & 1 & 1 & Solution & 120.03 & 4 &  0.00 & 100.00\\
instance n=20 6.alb & 1 & 1 & Solution & 120.03 & 3 &  0.00 & 100.00\\
instance n=20 60.alb & 1 & 1 & Solution & 120.04 & 6 &  0.00 & 100.00\\
instance n=20 61.alb & 1 & 1 & Solution & 120.03 & 7 &  0.00 & 100.00\\
instance n=20 62.alb & 1 & 1 & Solution & 120.03 & 5 &  0.00 & 100.00\\
instance n=20 63.alb & 1 & 1 & Solution & 120.03 & 5 &  0.00 & 100.00\\
instance n=20 64.alb & 1 & 1 & Solution & 120.02 & 5 &  0.00 & 100.00\\
instance n=20 65.alb & 1 & 1 & Solution & 120.03 & 5 &  0.00 & 100.00\\
instance n=20 66.alb & 1 & 1 & Solution & 120.03 & 3 &  0.00 & 100.00\\
instance n=20 67.alb & 1 & 1 & Optimal & 60.43 & 3 &  0.00 & 100.00\\
instance n=20 68.alb & 1 & 1 & Solution & 120.03 & 3 &  0.00 & 100.00\\
instance n=20 69.alb & 1 & 1 & Solution & 120.03 & 2 &  0.00 & 100.00\\
instance n=20 7.alb & 1 & 1 & Solution & 120.03 & 3 &  0.00 & 100.00\\
instance n=20 70.alb & 1 & 1 & Solution & 120.03 & 3 &  0.00 & 100.00\\
instance n=20 71.alb & 1 & 1 & Solution & 120.03 & 3 &  0.00 & 100.00\\
instance n=20 72.alb & 1 & 1 & Solution & 120.02 & 3 &  0.00 & 100.00\\
instance n=20 73.alb & 1 & 1 & Solution & 120.02 & 2 &  0.00 & 100.00\\
instance n=20 74.alb & 1 & 1 & Optimal & 80.33 & 3 &  0.00 & 100.00\\
instance n=20 75.alb & 1 & 1 & Solution & 120.03 & 3 &  0.00 & 100.00\\
instance n=20 76.alb & 1 & 1 & Optimal & 74.03 & 3 &  0.00 & 100.00\\
instance n=20 77.alb & 1 & 1 & Solution & 120.04 & 3 &  0.00 & 100.00\\
instance n=20 78.alb & 1 & 1 & Solution & 120.03 & 3 &  0.00 & 100.00\\
instance n=20 79.alb & 1 & 1 & Optimal & 43.11 & 3 &  0.00 & 100.00\\
instance n=20 8.alb & 1 & 1 & Solution & 120.02 & 3 &  0.00 & 100.00\\
instance n=20 80.alb & 1 & 1 & Solution & 120.03 & 3 &  0.00 & 100.00\\
instance n=20 81.alb & 1 & 1 & Solution & 120.04 & 3 &  0.00 & 100.00\\
instance n=20 82.alb & 1 & 1 & Solution & 120.03 & 4 &  0.00 & 100.00\\
instance n=20 83.alb & 1 & 1 & Solution & 120.03 & 3 &  0.00 & 100.00\\
instance n=20 84.alb & 1 & 1 & Solution & 120.05 & 3 &  0.00 & 100.00\\
instance n=20 85.alb & 1 & 1 & Solution & 120.04 & 3 &  0.00 & 100.00\\
instance n=20 86.alb & 1 & 1 & Solution & 120.03 & 3 &  0.00 & 100.00\\
instance n=20 87.alb & 1 & 1 & Solution & 120.03 & 3 &  0.00 & 100.00\\
instance n=20 88.alb & 1 & 1 & Solution & 120.03 & 3 &  0.00 & 100.00\\
instance n=20 89.alb & 1 & 1 & Solution & 120.04 & 3 &  0.00 & 100.00\\
instance n=20 9.alb & 1 & 1 & Solution & 120.04 & 3 &  0.00 & 100.00\\
instance n=20 90.alb & 1 & 1 & Solution & 120.04 & 3 &  0.00 & 100.00\\
instance n=20 91.alb & 1 & 1 & Optimal &  4.17 & 11 &  0.00 & 100.00\\
instance n=20 92.alb & 1 & 1 & Optimal &  1.93 & 11 &  0.00 & 100.00\\
instance n=20 93.alb & 1 & 1 & Optimal & 96.86 & 13 &  0.00 & 100.00\\
instance n=20 94.alb & 1 & 1 & Optimal &  3.23 & 10 &  0.00 & 100.00\\
instance n=20 95.alb & 1 & 1 & Optimal &  3.09 & 12 &  0.00 & 100.00\\
instance n=20 96.alb & 1 & 1 & Optimal &  3.42 & 10 &  0.00 & 100.00\\
instance n=20 97.alb & 1 & 1 & Solution & 120.02 & 15 &  0.00 & 100.00\\
instance n=20 98.alb & 1 & 1 & Optimal &  8.14 & 13 &  0.00 & 100.00\\
instance n=20 99.alb & 1 & 1 & Optimal & 16.26 & 12 &  0.00 & 100.00\\
instance n=50 1.alb & 1 & 1 & Solution & 120.13 & 8 &  0.00 & 100.00\\
instance n=50 10.alb & 1 & 1 & Solution & 120.18 & 16 &  0.00 & 100.00\\
instance n=50 100.alb & 1 & 1 & Unknown & 120764.00 & - & - & -\\
instance n=50 101.alb & 1 & 1 & Unknown & 120242.00 & - & - & -\\
instance n=50 102.alb & 1 & 1 & Unknown & 120208.00 & - & - & -\\
instance n=50 103.alb & 1 & 1 & Unknown & 120273.00 & - & - & -\\
instance n=50 104.alb & 1 & 1 & Unknown & 120315.00 & - & - & -\\
instance n=50 105.alb & 1 & 1 & Unknown & 120141.00 & - & - & -\\
instance n=50 106.alb & 1 & 1 & Unknown & 120180.00 & - & - & -\\
instance n=50 107.alb & 1 & 1 & Unknown & 120160.00 & - & - & -\\
instance n=50 108.alb & 1 & 1 & Unknown & 120172.00 & - & - & -\\
instance n=50 109.alb & 1 & 1 & Unknown & 120157.00 & - & - & -\\
instance n=50 11.alb & 1 & 1 & Solution & 120.19 & 8 &  0.00 & 100.00\\
instance n=50 110.alb & 1 & 1 & Unknown & 120168.00 & - & - & -\\
instance n=50 111.alb & 1 & 1 & Unknown & 120162.00 & - & - & -\\
instance n=50 112.alb & 1 & 1 & Unknown & 120168.00 & - & - & -\\
instance n=50 113.alb & 1 & 1 & Unknown & 120160.00 & - & - & -\\
instance n=50 114.alb & 1 & 1 & Unknown & 120180.00 & - & - & -\\
instance n=50 115.alb & 1 & 1 & Unknown & 120179.00 & - & - & -\\
instance n=50 116.alb & 1 & 1 & Unknown & 120210.00 & - & - & -\\
instance n=50 117.alb & 1 & 1 & Unknown & 120477.00 & - & - & -\\
instance n=50 118.alb & 1 & 1 & Unknown & 120185.00 & - & - & -\\
instance n=50 119.alb & 1 & 1 & Unknown & 120252.00 & - & - & -\\
instance n=50 12.alb & 1 & 1 & Solution & 120.20 & 22 &  0.00 & 100.00\\
instance n=50 120.alb & 1 & 1 & Unknown & 120166.00 & - & - & -\\
instance n=50 121.alb & 1 & 1 & Unknown & 120172.00 & - & - & -\\
instance n=50 122.alb & 1 & 1 & Solution & 120.19 & 50 &  0.00 & 100.00\\
instance n=50 123.alb & 1 & 1 & Unknown & 120170.00 & - & - & -\\
instance n=50 124.alb & 1 & 1 & Unknown & 120260.00 & - & - & -\\
instance n=50 125.alb & 1 & 1 & Unknown & 120131.00 & - & - & -\\
instance n=50 126.alb & 1 & 1 & Unknown & 120158.00 & - & - & -\\
instance n=50 127.alb & 1 & 1 & Solution & 120.26 & 14 &  0.00 & 100.00\\
instance n=50 128.alb & 1 & 1 & Solution & 120.16 & 13 &  0.00 & 100.00\\
instance n=50 129.alb & 1 & 1 & Solution & 120.18 & 13 &  0.00 & 100.00\\
instance n=50 13.alb & 1 & 1 & Solution & 120.18 & 12 &  0.00 & 100.00\\
instance n=50 130.alb & 1 & 1 & Unknown & 120156.00 & - & - & -\\
instance n=50 131.alb & 1 & 1 & Solution & 120.33 & 12 &  0.00 & 100.00\\
instance n=50 132.alb & 1 & 1 & Unknown & 120878.00 & - & - & -\\
instance n=50 133.alb & 1 & 1 & Solution & 120.17 & 13 &  0.00 & 100.00\\
instance n=50 134.alb & 1 & 1 & Unknown & 120166.00 & - & - & -\\
instance n=50 135.alb & 1 & 1 & Solution & 120.19 & 14 &  0.00 & 100.00\\
instance n=50 136.alb & 1 & 1 & Unknown & 120153.00 & - & - & -\\
instance n=50 137.alb & 1 & 1 & Unknown & 120167.00 & - & - & -\\
instance n=50 138.alb & 1 & 1 & Solution & 120.14 & 12 &  0.00 & 100.00\\
instance n=50 139.alb & 1 & 1 & Unknown & 120145.00 & - & - & -\\
instance n=50 14.alb & 1 & 1 & Solution & 120.16 & 8 &  0.00 & 100.00\\
instance n=50 140.alb & 1 & 1 & Unknown & 120160.00 & - & - & -\\
instance n=50 141.alb & 1 & 1 & Solution & 120.20 & 14 &  0.00 & 100.00\\
instance n=50 142.alb & 1 & 1 & Unknown & 120167.00 & - & - & -\\
instance n=50 143.alb & 1 & 1 & Solution & 120.42 & 12 &  0.00 & 100.00\\
instance n=50 144.alb & 1 & 1 & Solution & 120.16 & 14 &  0.00 & 100.00\\
instance n=50 145.alb & 1 & 1 & Unknown & 120163.00 & - & - & -\\
instance n=50 146.alb & 1 & 1 & Unknown & 120161.00 & - & - & -\\
instance n=50 147.alb & 1 & 1 & Solution & 120.19 & 15 &  0.00 & 100.00\\
instance n=50 148.alb & 1 & 1 & Solution & 120.14 & 10 &  0.00 & 100.00\\
instance n=50 149.alb & 1 & 1 & Unknown & 120168.00 & - & - & -\\
instance n=50 15.alb & 1 & 1 & Unknown & 120645.00 & - & - & -\\
instance n=50 150.alb & 1 & 1 & Unknown & 120181.00 & - & - & -\\
instance n=50 151.alb & 1 & 1 & Unknown & 120187.00 & - & - & -\\
instance n=50 152.alb & 1 & 1 & Unknown & 120182.00 & - & - & -\\
instance n=50 153.alb & 1 & 1 & Unknown & 120157.00 & - & - & -\\
instance n=50 154.alb & 1 & 1 & Solution & 120.18 & 16 &  0.00 & 100.00\\
instance n=50 155.alb & 1 & 1 & Solution & 121.02 & 8 &  0.00 & 100.00\\
instance n=50 156.alb & 1 & 1 & Solution & 120.14 & 7 &  0.00 & 100.00\\
instance n=50 157.alb & 1 & 1 & Unknown & 120239.00 & - & - & -\\
instance n=50 158.alb & 1 & 1 & Solution & 120.16 & 11 &  0.00 & 100.00\\
instance n=50 159.alb & 1 & 1 & Solution & 120.17 & 8 &  0.00 & 100.00\\
instance n=50 16.alb & 1 & 1 & Unknown & 120166.00 & - & - & -\\
instance n=50 160.alb & 1 & 1 & Solution & 120.20 & 23 &  0.00 & 100.00\\
instance n=50 161.alb & 1 & 1 & Solution & 120.17 & 32 &  0.00 & 100.00\\
instance n=50 162.alb & 1 & 1 & Unknown & 120165.00 & - & - & -\\
instance n=50 163.alb & 1 & 1 & Unknown & 120161.00 & - & - & -\\
instance n=50 164.alb & 1 & 1 & Unknown & 120163.00 & - & - & -\\
instance n=50 165.alb & 1 & 1 & Solution & 120.14 & 8 &  0.00 & 100.00\\
instance n=50 166.alb & 1 & 1 & Solution & 120.19 & 11 &  0.00 & 100.00\\
instance n=50 167.alb & 1 & 1 & Unknown & 120155.00 & - & - & -\\
instance n=50 168.alb & 1 & 1 & Unknown & 120156.00 & - & - & -\\
instance n=50 169.alb & 1 & 1 & Solution & 120.14 & 9 &  0.00 & 100.00\\
instance n=50 17.alb & 1 & 1 & Solution & 120.15 & 7 &  0.00 & 100.00\\
instance n=50 170.alb & 1 & 1 & Unknown & 120185.00 & - & - & -\\
instance n=50 171.alb & 1 & 1 & Unknown & 120156.00 & - & - & -\\
instance n=50 172.alb & 1 & 1 & Solution & 120.15 & 7 &  0.00 & 100.00\\
instance n=50 173.alb & 1 & 1 & Unknown & 120165.00 & - & - & -\\
instance n=50 174.alb & 1 & 1 & Unknown & 120161.00 & - & - & -\\
instance n=50 175.alb & 1 & 1 & Unknown & 120166.00 & - & - & -\\
instance n=50 176.alb & 1 & 1 & Solution & 120.24 & 30 &  0.00 & 100.00\\
instance n=50 177.alb & 1 & 1 & Solution & 121.11 & 41 &  0.00 & 100.00\\
instance n=50 178.alb & 1 & 1 & Solution & 120.21 & 29 &  0.00 & 100.00\\
instance n=50 179.alb & 1 & 1 & Solution & 120.18 & 35 &  0.00 & 100.00\\
instance n=50 18.alb & 1 & 1 & Unknown & 120172.00 & - & - & -\\
instance n=50 180.alb & 1 & 1 & Solution & 120.17 & 30 &  0.00 & 100.00\\
instance n=50 181.alb & 1 & 1 & Solution & 120.20 & 33 &  0.00 & 100.00\\
instance n=50 182.alb & 1 & 1 & Unknown & 120231.00 & - & - & -\\
instance n=50 183.alb & 1 & 1 & Solution & 120.18 & 34 &  0.00 & 100.00\\
instance n=50 184.alb & 1 & 1 & Solution & 120.17 & 42 &  0.00 & 100.00\\
instance n=50 185.alb & 1 & 1 & Unknown & 120171.00 & - & - & -\\
instance n=50 186.alb & 1 & 1 & Unknown & 120176.00 & - & - & -\\
instance n=50 187.alb & 1 & 1 & Solution & 120.17 & 29 &  0.00 & 100.00\\
instance n=50 188.alb & 1 & 1 & Unknown & 120154.00 & - & - & -\\
instance n=50 189.alb & 1 & 1 & Unknown & 120160.00 & - & - & -\\
instance n=50 19.alb & 1 & 1 & Unknown & 120177.00 & - & - & -\\
instance n=50 190.alb & 1 & 1 & Solution & 120.17 & 37 &  0.00 & 100.00\\
instance n=50 191.alb & 1 & 1 & Solution & 120.53 & 32 &  0.00 & 100.00\\
instance n=50 192.alb & 1 & 1 & Solution & 120.18 & 33 &  0.00 & 100.00\\
instance n=50 193.alb & 1 & 1 & Unknown & 120182.00 & - & - & -\\
instance n=50 194.alb & 1 & 1 & Unknown & 120159.00 & - & - & -\\
instance n=50 195.alb & 1 & 1 & Solution & 120.30 & 30 &  0.00 & 100.00\\
instance n=50 196.alb & 1 & 1 & Unknown & 120759.00 & - & - & -\\
instance n=50 197.alb & 1 & 1 & Solution & 120.16 & 32 &  0.00 & 100.00\\
instance n=50 198.alb & 1 & 1 & Solution & 121.33 & 35 &  0.00 & 100.00\\
instance n=50 199.alb & 1 & 1 & Solution & 120.16 & 32 &  0.00 & 100.00\\
instance n=50 2.alb & 1 & 1 & Unknown & 120176.00 & - & - & -\\
instance n=50 20.alb & 1 & 1 & Unknown & 120180.00 & - & - & -\\
instance n=50 200.alb & 1 & 1 & Solution & 120.17 & 28 &  0.00 & 100.00\\
instance n=50 201.alb & 1 & 1 & Unknown & 120178.00 & - & - & -\\
instance n=50 202.alb & 1 & 1 & Solution & 120.17 & 18 &  0.00 & 100.00\\
instance n=50 203.alb & 1 & 1 & Unknown & 120243.00 & - & - & -\\
instance n=50 204.alb & 1 & 1 & Unknown & 120157.00 & - & - & -\\
instance n=50 205.alb & 1 & 1 & Unknown & 120155.00 & - & - & -\\
instance n=50 206.alb & 1 & 1 & Unknown & 120169.00 & - & - & -\\
instance n=50 207.alb & 1 & 1 & Solution & 120.18 & 11 &  0.00 & 100.00\\
instance n=50 208.alb & 1 & 1 & Unknown & 120143.00 & - & - & -\\
instance n=50 209.alb & 1 & 1 & Unknown & 120154.00 & - & - & -\\
instance n=50 21.alb & 1 & 1 & Unknown & 120173.00 & - & - & -\\
instance n=50 210.alb & 1 & 1 & Unknown & 120171.00 & - & - & -\\
instance n=50 211.alb & 1 & 1 & Solution & 120.16 & 13 &  0.00 & 100.00\\
instance n=50 212.alb & 1 & 1 & Solution & 120.16 & 11 &  0.00 & 100.00\\
instance n=50 213.alb & 1 & 1 & Solution & 120.15 & 13 &  0.00 & 100.00\\
instance n=50 214.alb & 1 & 1 & Unknown & 120162.00 & - & - & -\\
instance n=50 215.alb & 1 & 1 & Unknown & 120155.00 & - & - & -\\
instance n=50 216.alb & 1 & 1 & Unknown & 120177.00 & - & - & -\\
instance n=50 217.alb & 1 & 1 & Unknown & 120151.00 & - & - & -\\
instance n=50 218.alb & 1 & 1 & Solution & 120.17 & 13 &  0.00 & 100.00\\
instance n=50 219.alb & 1 & 1 & Unknown & 120170.00 & - & - & -\\
instance n=50 22.alb & 1 & 1 & Unknown & 120173.00 & - & - & -\\
instance n=50 220.alb & 1 & 1 & Solution & 120.16 & 13 &  0.00 & 100.00\\
instance n=50 221.alb & 1 & 1 & Unknown & 120202.00 & - & - & -\\
instance n=50 222.alb & 1 & 1 & Unknown & 120167.00 & - & - & -\\
instance n=50 223.alb & 1 & 1 & Unknown & 120164.00 & - & - & -\\
instance n=50 224.alb & 1 & 1 & Solution & 120.19 & 18 &  0.00 & 100.00\\
instance n=50 225.alb & 1 & 1 & Unknown & 120507.00 & - & - & -\\
instance n=50 226.alb & 1 & 1 & Unknown & 120183.00 & - & - & -\\
instance n=50 227.alb & 1 & 1 & Unknown & 120150.00 & - & - & -\\
instance n=50 228.alb & 1 & 1 & Unknown & 120265.00 & - & - & -\\
instance n=50 229.alb & 1 & 1 & Unknown & 120166.00 & - & - & -\\
instance n=50 23.alb & 1 & 1 & Unknown & 120194.00 & - & - & -\\
instance n=50 230.alb & 1 & 1 & Unknown & 120148.00 & - & - & -\\
instance n=50 231.alb & 1 & 1 & Solution & 120.12 & 9 &  0.00 & 100.00\\
instance n=50 232.alb & 1 & 1 & Unknown & 120152.00 & - & - & -\\
instance n=50 233.alb & 1 & 1 & Unknown & 120164.00 & - & - & -\\
instance n=50 234.alb & 1 & 1 & Solution & 120.18 & 10 &  0.00 & 100.00\\
instance n=50 235.alb & 1 & 1 & Unknown & 120165.00 & - & - & -\\
instance n=50 236.alb & 1 & 1 & Unknown & 120151.00 & - & - & -\\
instance n=50 237.alb & 1 & 1 & Unknown & 120172.00 & - & - & -\\
instance n=50 238.alb & 1 & 1 & Solution & 120.15 & 7 &  0.00 & 100.00\\
instance n=50 239.alb & 1 & 1 & Unknown & 120163.00 & - & - & -\\
instance n=50 24.alb & 1 & 1 & Solution & 120.16 & 7 &  0.00 & 100.00\\
instance n=50 240.alb & 1 & 1 & Unknown & 120182.00 & - & - & -\\
instance n=50 241.alb & 1 & 1 & Unknown & 120265.00 & - & - & -\\
instance n=50 242.alb & 1 & 1 & Solution & 120.25 & 15 &  0.00 & 100.00\\
instance n=50 243.alb & 1 & 1 & Unknown & 120180.00 & - & - & -\\
instance n=50 244.alb & 1 & 1 & Unknown & 120251.00 & - & - & -\\
instance n=50 245.alb & 1 & 1 & Unknown & 120876.00 & - & - & -\\
instance n=50 246.alb & 1 & 1 & Unknown & 120181.00 & - & - & -\\
instance n=50 247.alb & 1 & 1 & Unknown & 120170.00 & - & - & -\\
instance n=50 248.alb & 1 & 1 & Unknown & 120169.00 & - & - & -\\
instance n=50 249.alb & 1 & 1 & Solution & 120.17 & 8 &  0.00 & 100.00\\
instance n=50 25.alb & 1 & 1 & Unknown & 120174.00 & - & - & -\\
instance n=50 250.alb & 1 & 1 & Unknown & 120185.00 & - & - & -\\
instance n=50 251.alb & 1 & 1 & Unknown & 120158.00 & - & - & -\\
instance n=50 252.alb & 1 & 1 & Solution & 120.15 & 37 &  0.00 & 100.00\\
instance n=50 253.alb & 1 & 1 & Unknown & 120151.00 & - & - & -\\
instance n=50 254.alb & 1 & 1 & Unknown & 120156.00 & - & - & -\\
instance n=50 255.alb & 1 & 1 & Unknown & 120163.00 & - & - & -\\
instance n=50 256.alb & 1 & 1 & Solution & 120.16 & 50 &  0.00 & 100.00\\
instance n=50 257.alb & 1 & 1 & Unknown & 120177.00 & - & - & -\\
instance n=50 258.alb & 1 & 1 & Unknown & 120154.00 & - & - & -\\
instance n=50 259.alb & 1 & 1 & Unknown & 120274.00 & - & - & -\\
instance n=50 26.alb & 1 & 1 & Solution & 120.19 & 37 &  0.00 & 100.00\\
instance n=50 260.alb & 1 & 1 & Solution & 120.19 & 34 &  0.00 & 100.00\\
instance n=50 261.alb & 1 & 1 & Solution & 120.16 & 28 &  0.00 & 100.00\\
instance n=50 262.alb & 1 & 1 & Unknown & 120187.00 & - & - & -\\
instance n=50 263.alb & 1 & 1 & Unknown & 120159.00 & - & - & -\\
instance n=50 264.alb & 1 & 1 & Unknown & 120140.00 & - & - & -\\
instance n=50 265.alb & 1 & 1 & Unknown & 120169.00 & - & - & -\\
instance n=50 266.alb & 1 & 1 & Unknown & 120171.00 & - & - & -\\
instance n=50 267.alb & 1 & 1 & Unknown & 120196.00 & - & - & -\\
instance n=50 268.alb & 1 & 1 & Solution & 120.16 & 34 &  0.00 & 100.00\\
instance n=50 269.alb & 1 & 1 & Unknown & 120159.00 & - & - & -\\
instance n=50 27.alb & 1 & 1 & Solution & 120.19 & 33 &  0.00 & 100.00\\
instance n=50 270.alb & 1 & 1 & Unknown & 120152.00 & - & - & -\\
instance n=50 271.alb & 1 & 1 & Unknown & 120163.00 & - & - & -\\
instance n=50 272.alb & 1 & 1 & Solution & 120.18 & 42 &  0.00 & 100.00\\
instance n=50 273.alb & 1 & 1 & Solution & 121.44 & 31 &  0.00 & 100.00\\
instance n=50 274.alb & 1 & 1 & Solution & 120.21 & 35 &  0.00 & 100.00\\
instance n=50 275.alb & 1 & 1 & Solution & 120.15 & 30 &  0.00 & 100.00\\
instance n=50 276.alb & 1 & 1 & Solution & 120.15 & 13 &  0.00 & 100.00\\
instance n=50 277.alb & 1 & 1 & Solution & 120.18 & 18 &  0.00 & 100.00\\
instance n=50 278.alb & 1 & 1 & Unknown & 120150.00 & - & - & -\\
instance n=50 279.alb & 1 & 1 & Solution & 120.18 & 11 &  0.00 & 100.00\\
instance n=50 28.alb & 1 & 1 & Solution & 120.18 & 50 &  0.00 & 100.00\\
instance n=50 280.alb & 1 & 1 & Solution & 120.27 & 14 &  0.00 & 100.00\\
instance n=50 281.alb & 1 & 1 & Unknown & 120177.00 & - & - & -\\
instance n=50 282.alb & 1 & 1 & Unknown & 120185.00 & - & - & -\\
instance n=50 283.alb & 1 & 1 & Solution & 120.15 & 50 &  0.00 & 100.00\\
instance n=50 284.alb & 1 & 1 & Unknown & 120403.00 & - & - & -\\
instance n=50 285.alb & 1 & 1 & Unknown & 120180.00 & - & - & -\\
instance n=50 286.alb & 1 & 1 & Unknown & 120166.00 & - & - & -\\
instance n=50 287.alb & 1 & 1 & Unknown & 120160.00 & - & - & -\\
instance n=50 288.alb & 1 & 1 & Unknown & 120172.00 & - & - & -\\
instance n=50 289.alb & 1 & 1 & Unknown & 120240.00 & - & - & -\\
instance n=50 29.alb & 1 & 1 & Solution & 121.49 & 33 &  0.00 & 100.00\\
instance n=50 290.alb & 1 & 1 & Unknown & 120150.00 & - & - & -\\
instance n=50 291.alb & 1 & 1 & Unknown & 120156.00 & - & - & -\\
instance n=50 292.alb & 1 & 1 & Solution & 120.17 & 13 &  0.00 & 100.00\\
instance n=50 293.alb & 1 & 1 & Solution & 120.15 & 12 &  0.00 & 100.00\\
instance n=50 294.alb & 1 & 1 & Unknown & 120226.00 & - & - & -\\
instance n=50 295.alb & 1 & 1 & Solution & 120.16 & 16 &  0.00 & 100.00\\
instance n=50 296.alb & 1 & 1 & Unknown & 120163.00 & - & - & -\\
instance n=50 297.alb & 1 & 1 & Solution & 120.15 & 14 &  0.00 & 100.00\\
instance n=50 298.alb & 1 & 1 & Solution & 120.15 & 11 &  0.00 & 100.00\\
instance n=50 299.alb & 1 & 1 & Unknown & 120150.00 & - & - & -\\
instance n=50 3.alb & 1 & 1 & Unknown & 120166.00 & - & - & -\\
instance n=50 30.alb & 1 & 1 & Unknown & 120167.00 & - & - & -\\
instance n=50 300.alb & 1 & 1 & Unknown & 120267.00 & - & - & -\\
instance n=50 301.alb & 1 & 1 & Unknown & 120176.00 & - & - & -\\
instance n=50 302.alb & 1 & 1 & Unknown & 120180.00 & - & - & -\\
instance n=50 303.alb & 1 & 1 & Solution & 120.16 & 11 &  0.00 & 100.00\\
instance n=50 304.alb & 1 & 1 & Solution & 120.17 & 10 &  0.00 & 100.00\\
instance n=50 305.alb & 1 & 1 & Solution & 120.17 & 29 &  0.00 & 100.00\\
instance n=50 306.alb & 1 & 1 & Unknown & 120151.00 & - & - & -\\
instance n=50 307.alb & 1 & 1 & Solution & 120.17 & 7 &  0.00 & 100.00\\
instance n=50 308.alb & 1 & 1 & Solution & 120.18 & 10 &  0.00 & 100.00\\
instance n=50 309.alb & 1 & 1 & Solution & 120.15 & 19 &  0.00 & 100.00\\
instance n=50 31.alb & 1 & 1 & Solution & 120.19 & 30 &  0.00 & 100.00\\
instance n=50 310.alb & 1 & 1 & Solution & 120.18 & 11 &  0.00 & 100.00\\
instance n=50 311.alb & 1 & 1 & Unknown & 120176.00 & - & - & -\\
instance n=50 312.alb & 1 & 1 & Solution & 120.14 & 7 &  0.00 & 100.00\\
instance n=50 313.alb & 1 & 1 & Solution & 120.16 & 8 &  0.00 & 100.00\\
instance n=50 314.alb & 1 & 1 & Solution & 121.54 & 50 &  0.00 & 100.00\\
instance n=50 315.alb & 1 & 1 & Solution & 120.16 & 8 &  0.00 & 100.00\\
instance n=50 316.alb & 1 & 1 & Solution & 120.21 & 10 &  0.00 & 100.00\\
instance n=50 317.alb & 1 & 1 & Solution & 120.20 & 7 &  0.00 & 100.00\\
instance n=50 318.alb & 1 & 1 & Unknown & 120299.00 & - & - & -\\
instance n=50 319.alb & 1 & 1 & Solution & 120.16 & 7 &  0.00 & 100.00\\
instance n=50 32.alb & 1 & 1 & Unknown & 120207.00 & - & - & -\\
instance n=50 320.alb & 1 & 1 & Solution & 120.15 & 8 &  0.00 & 100.00\\
instance n=50 321.alb & 1 & 1 & Solution & 120.15 & 6 &  0.00 & 100.00\\
instance n=50 322.alb & 1 & 1 & Solution & 120.21 & 7 &  0.00 & 100.00\\
instance n=50 323.alb & 1 & 1 & Solution & 120.29 & 13 &  0.00 & 100.00\\
instance n=50 324.alb & 1 & 1 & Solution & 120.14 & 7 &  0.00 & 100.00\\
instance n=50 325.alb & 1 & 1 & Unknown & 120155.00 & - & - & -\\
instance n=50 326.alb & 1 & 1 & Unknown & 120148.00 & - & - & -\\
instance n=50 327.alb & 1 & 1 & Solution & 120.16 & 31 &  0.00 & 100.00\\
instance n=50 328.alb & 1 & 1 & Solution & 120.15 & 48 &  0.00 & 100.00\\
instance n=50 329.alb & 1 & 1 & Unknown & 120152.00 & - & - & -\\
instance n=50 33.alb & 1 & 1 & Unknown & 120436.00 & - & - & -\\
instance n=50 330.alb & 1 & 1 & Unknown & 120264.00 & - & - & -\\
instance n=50 331.alb & 1 & 1 & Solution & 120.21 & 34 &  0.00 & 100.00\\
instance n=50 332.alb & 1 & 1 & Solution & 120.19 & 28 &  0.00 & 100.00\\
instance n=50 333.alb & 1 & 1 & Solution & 120.17 & 50 &  0.00 & 100.00\\
instance n=50 334.alb & 1 & 1 & Solution & 120.20 & 32 &  0.00 & 100.00\\
instance n=50 335.alb & 1 & 1 & Solution & 120.16 & 33 &  0.00 & 100.00\\
instance n=50 336.alb & 1 & 1 & Solution & 120.21 & 31 &  0.00 & 100.00\\
instance n=50 337.alb & 1 & 1 & Solution & 121.08 & 29 &  0.00 & 100.00\\
instance n=50 338.alb & 1 & 1 & Solution & 120.16 & 29 &  0.00 & 100.00\\
instance n=50 339.alb & 1 & 1 & Solution & 120.18 & 34 &  0.00 & 100.00\\
instance n=50 34.alb & 1 & 1 & Solution & 120.19 & 31 &  0.00 & 100.00\\
instance n=50 340.alb & 1 & 1 & Solution & 120.18 & 31 &  0.00 & 100.00\\
instance n=50 341.alb & 1 & 1 & Unknown & 120302.00 & - & - & -\\
instance n=50 342.alb & 1 & 1 & Unknown & 120175.00 & - & - & -\\
instance n=50 343.alb & 1 & 1 & Solution & 120.30 & 29 &  0.00 & 100.00\\
instance n=50 344.alb & 1 & 1 & Solution & 120.24 & 32 &  0.00 & 100.00\\
instance n=50 345.alb & 1 & 1 & Solution & 120.18 & 33 &  0.00 & 100.00\\
instance n=50 346.alb & 1 & 1 & Unknown & 120176.00 & - & - & -\\
instance n=50 347.alb & 1 & 1 & Solution & 120.16 & 30 &  0.00 & 100.00\\
instance n=50 348.alb & 1 & 1 & Unknown & 120258.00 & - & - & -\\
instance n=50 349.alb & 1 & 1 & Solution & 120.16 & 32 &  0.00 & 100.00\\
instance n=50 35.alb & 1 & 1 & Solution & 120.18 & 43 &  0.00 & 100.00\\
instance n=50 350.alb & 1 & 1 & Solution & 120.17 & 27 &  0.00 & 100.00\\
instance n=50 351.alb & 1 & 1 & Solution & 120.17 & 18 &  0.00 & 100.00\\
instance n=50 352.alb & 1 & 1 & Solution & 120.18 & 11 &  0.00 & 100.00\\
instance n=50 353.alb & 1 & 1 & Unknown & 120269.00 & - & - & -\\
instance n=50 354.alb & 1 & 1 & Solution & 120.20 & 15 &  0.00 & 100.00\\
instance n=50 355.alb & 1 & 1 & Solution & 120.17 & 21 &  0.00 & 100.00\\
instance n=50 356.alb & 1 & 1 & Solution & 120.19 & 15 &  0.00 & 100.00\\
instance n=50 357.alb & 1 & 1 & Unknown & 120183.00 & - & - & -\\
instance n=50 358.alb & 1 & 1 & Solution & 120.16 & 13 &  0.00 & 100.00\\
instance n=50 359.alb & 1 & 1 & Unknown & 120159.00 & - & - & -\\
instance n=50 36.alb & 1 & 1 & Solution & 120.19 & 33 &  0.00 & 100.00\\
instance n=50 360.alb & 1 & 1 & Solution & 120.17 & 13 &  0.00 & 100.00\\
instance n=50 361.alb & 1 & 1 & Unknown & 120175.00 & - & - & -\\
instance n=50 362.alb & 1 & 1 & Solution & 120.19 & 14 &  0.00 & 100.00\\
instance n=50 363.alb & 1 & 1 & Solution & 120.19 & 14 &  0.00 & 100.00\\
instance n=50 364.alb & 1 & 1 & Unknown & 120167.00 & - & - & -\\
instance n=50 365.alb & 1 & 1 & Solution & 120.16 & 13 &  0.00 & 100.00\\
instance n=50 366.alb & 1 & 1 & Unknown & 120147.00 & - & - & -\\
instance n=50 367.alb & 1 & 1 & Unknown & 120163.00 & - & - & -\\
instance n=50 368.alb & 1 & 1 & Unknown & 120146.00 & - & - & -\\
instance n=50 369.alb & 1 & 1 & Solution & 120.16 & 13 &  0.00 & 100.00\\
instance n=50 37.alb & 1 & 1 & Unknown & 120623.00 & - & - & -\\
instance n=50 370.alb & 1 & 1 & Solution & 120.14 & 12 &  0.00 & 100.00\\
instance n=50 371.alb & 1 & 1 & Unknown & 120169.00 & - & - & -\\
instance n=50 372.alb & 1 & 1 & Solution & 120.17 & 18 &  0.00 & 100.00\\
instance n=50 373.alb & 1 & 1 & Unknown & 120173.00 & - & - & -\\
instance n=50 374.alb & 1 & 1 & Solution & 120.20 & 11 &  0.00 & 100.00\\
instance n=50 375.alb & 1 & 1 & Unknown & 120171.00 & - & - & -\\
instance n=50 376.alb & 1 & 1 & Unknown & 120173.00 & - & - & -\\
instance n=50 377.alb & 1 & 1 & Unknown & 120166.00 & - & - & -\\
instance n=50 378.alb & 1 & 1 & Unknown & 120191.00 & - & - & -\\
instance n=50 379.alb & 1 & 1 & Solution & 120.20 & 23 &  0.00 & 100.00\\
instance n=50 38.alb & 1 & 1 & Solution & 120.19 & 39 &  0.00 & 100.00\\
instance n=50 380.alb & 1 & 1 & Unknown & 120182.00 & - & - & -\\
instance n=50 381.alb & 1 & 1 & Unknown & 120711.00 & - & - & -\\
instance n=50 382.alb & 1 & 1 & Unknown & 120176.00 & - & - & -\\
instance n=50 383.alb & 1 & 1 & Unknown & 120180.00 & - & - & -\\
instance n=50 384.alb & 1 & 1 & Unknown & 120644.00 & - & - & -\\
instance n=50 385.alb & 1 & 1 & Unknown & 120253.00 & - & - & -\\
instance n=50 386.alb & 1 & 1 & Solution & 120.16 & 7 &  0.00 & 100.00\\
instance n=50 387.alb & 1 & 1 & Unknown & 120166.00 & - & - & -\\
instance n=50 388.alb & 1 & 1 & Unknown & 120178.00 & - & - & -\\
instance n=50 389.alb & 1 & 1 & Unknown & 120397.00 & - & - & -\\
instance n=50 39.alb & 1 & 1 & Unknown & 120161.00 & - & - & -\\
instance n=50 390.alb & 1 & 1 & Unknown & 120170.00 & - & - & -\\
instance n=50 391.alb & 1 & 1 & Solution & 120.16 & 9 &  0.00 & 100.00\\
instance n=50 392.alb & 1 & 1 & Unknown & 120171.00 & - & - & -\\
instance n=50 393.alb & 1 & 1 & Unknown & 120170.00 & - & - & -\\
instance n=50 394.alb & 1 & 1 & Unknown & 120150.00 & - & - & -\\
instance n=50 395.alb & 1 & 1 & Unknown & 120153.00 & - & - & -\\
instance n=50 396.alb & 1 & 1 & Unknown & 120168.00 & - & - & -\\
instance n=50 397.alb & 1 & 1 & Solution & 120.20 & 7 &  0.00 & 100.00\\
instance n=50 398.alb & 1 & 1 & Unknown & 120159.00 & - & - & -\\
instance n=50 399.alb & 1 & 1 & Unknown & 120203.00 & - & - & -\\
instance n=50 4.alb & 1 & 1 & Unknown & 120329.00 & - & - & -\\
instance n=50 40.alb & 1 & 1 & Unknown & 120171.00 & - & - & -\\
instance n=50 400.alb & 1 & 1 & Unknown & 120168.00 & - & - & -\\
instance n=50 401.alb & 1 & 1 & Unknown & 120323.00 & - & - & -\\
instance n=50 402.alb & 1 & 1 & Unknown & 120229.00 & - & - & -\\
instance n=50 403.alb & 1 & 1 & Unknown & 120169.00 & - & - & -\\
instance n=50 404.alb & 1 & 1 & Unknown & 120160.00 & - & - & -\\
instance n=50 405.alb & 1 & 1 & Unknown & 120314.00 & - & - & -\\
instance n=50 406.alb & 1 & 1 & Solution & 120.16 & 35 &  0.00 & 100.00\\
instance n=50 407.alb & 1 & 1 & Solution & 120.15 & 30 &  0.00 & 100.00\\
instance n=50 408.alb & 1 & 1 & Unknown & 120230.00 & - & - & -\\
instance n=50 409.alb & 1 & 1 & Unknown & 120170.00 & - & - & -\\
instance n=50 41.alb & 1 & 1 & Solution & 120.20 & 31 &  0.00 & 100.00\\
instance n=50 410.alb & 1 & 1 & Unknown & 120171.00 & - & - & -\\
instance n=50 411.alb & 1 & 1 & Unknown & 120165.00 & - & - & -\\
instance n=50 412.alb & 1 & 1 & Unknown & 120221.00 & - & - & -\\
instance n=50 413.alb & 1 & 1 & Unknown & 120170.00 & - & - & -\\
instance n=50 414.alb & 1 & 1 & Unknown & 120165.00 & - & - & -\\
instance n=50 415.alb & 1 & 1 & Unknown & 120223.00 & - & - & -\\
instance n=50 416.alb & 1 & 1 & Unknown & 120287.00 & - & - & -\\
instance n=50 417.alb & 1 & 1 & Solution & 120.15 & 31 &  0.00 & 100.00\\
instance n=50 418.alb & 1 & 1 & Unknown & 120306.00 & - & - & -\\
instance n=50 419.alb & 1 & 1 & Unknown & 120239.00 & - & - & -\\
instance n=50 42.alb & 1 & 1 & Solution & 120.19 & 26 &  0.00 & 100.00\\
instance n=50 420.alb & 1 & 1 & Unknown & 120220.00 & - & - & -\\
instance n=50 421.alb & 1 & 1 & Unknown & 120329.00 & - & - & -\\
instance n=50 422.alb & 1 & 1 & Solution & 120.15 & 31 &  0.00 & 100.00\\
instance n=50 423.alb & 1 & 1 & Unknown & 120158.00 & - & - & -\\
instance n=50 424.alb & 1 & 1 & Solution & 121.23 & 31 &  0.00 & 100.00\\
instance n=50 425.alb & 1 & 1 & Solution & 120.17 & 37 &  0.00 & 100.00\\
instance n=50 426.alb & 1 & 1 & Unknown & 120219.00 & - & - & -\\
instance n=50 427.alb & 1 & 1 & Solution & 120.17 & 12 &  0.00 & 100.00\\
instance n=50 428.alb & 1 & 1 & Unknown & 120185.00 & - & - & -\\
instance n=50 429.alb & 1 & 1 & Unknown & 120166.00 & - & - & -\\
instance n=50 43.alb & 1 & 1 & Unknown & 120855.00 & - & - & -\\
instance n=50 430.alb & 1 & 1 & Unknown & 120698.00 & - & - & -\\
instance n=50 431.alb & 1 & 1 & Unknown & 120898.00 & - & - & -\\
instance n=50 432.alb & 1 & 1 & Solution & 120.18 & 13 &  0.00 & 100.00\\
instance n=50 433.alb & 1 & 1 & Unknown & 120163.00 & - & - & -\\
instance n=50 434.alb & 1 & 1 & Solution & 120.17 & 11 &  0.00 & 100.00\\
instance n=50 435.alb & 1 & 1 & Unknown & 120153.00 & - & - & -\\
instance n=50 436.alb & 1 & 1 & Unknown & 120163.00 & - & - & -\\
instance n=50 437.alb & 1 & 1 & Unknown & 120143.00 & - & - & -\\
instance n=50 438.alb & 1 & 1 & Unknown & 120402.00 & - & - & -\\
instance n=50 439.alb & 1 & 1 & Unknown & 120162.00 & - & - & -\\
instance n=50 44.alb & 1 & 1 & Solution & 120.14 & 28 &  0.00 & 100.00\\
instance n=50 440.alb & 1 & 1 & Unknown & 120171.00 & - & - & -\\
instance n=50 441.alb & 1 & 1 & Solution & 120.15 & 11 &  0.00 & 100.00\\
instance n=50 442.alb & 1 & 1 & Unknown & 120168.00 & - & - & -\\
instance n=50 443.alb & 1 & 1 & Unknown & 120166.00 & - & - & -\\
instance n=50 444.alb & 1 & 1 & Unknown & 120189.00 & - & - & -\\
instance n=50 445.alb & 1 & 1 & Unknown & 120159.00 & - & - & -\\
instance n=50 446.alb & 1 & 1 & Unknown & 120171.00 & - & - & -\\
instance n=50 447.alb & 1 & 1 & Solution & 120.17 & 16 &  0.00 & 100.00\\
instance n=50 448.alb & 1 & 1 & Unknown & 120149.00 & - & - & -\\
instance n=50 449.alb & 1 & 1 & Unknown & 120170.00 & - & - & -\\
instance n=50 45.alb & 1 & 1 & Solution & 120.18 & 31 &  0.00 & 100.00\\
instance n=50 450.alb & 1 & 1 & Unknown & 120169.00 & - & - & -\\
instance n=50 451.alb & 1 & 1 & Solution & 120.12 & 8 &  0.00 & 100.00\\
instance n=50 452.alb & 1 & 1 & Solution & 120.14 & 8 &  0.00 & 100.00\\
instance n=50 453.alb & 1 & 1 & Solution & 120.17 & 7 &  0.00 & 100.00\\
instance n=50 454.alb & 1 & 1 & Solution & 120.17 & 8 &  0.00 & 100.00\\
instance n=50 455.alb & 1 & 1 & Solution & 120.15 & 6 &  0.00 & 100.00\\
instance n=50 456.alb & 1 & 1 & Solution & 120.14 & 8 &  0.00 & 100.00\\
instance n=50 457.alb & 1 & 1 & Solution & 120.14 & 8 &  0.00 & 100.00\\
instance n=50 458.alb & 1 & 1 & Solution & 120.15 & 7 &  0.00 & 100.00\\
instance n=50 459.alb & 1 & 1 & Solution & 120.14 & 7 &  0.00 & 100.00\\
instance n=50 46.alb & 1 & 1 & Unknown & 120179.00 & - & - & -\\
instance n=50 460.alb & 1 & 1 & Solution & 120.98 & 7 &  0.00 & 100.00\\
instance n=50 461.alb & 1 & 1 & Solution & 120.16 & 6 &  0.00 & 100.00\\
instance n=50 462.alb & 1 & 1 & Solution & 120.81 & 7 &  0.00 & 100.00\\
instance n=50 463.alb & 1 & 1 & Solution & 120.18 & 8 &  0.00 & 100.00\\
instance n=50 464.alb & 1 & 1 & Solution & 120.15 & 6 &  0.00 & 100.00\\
instance n=50 465.alb & 1 & 1 & Solution & 120.14 & 8 &  0.00 & 100.00\\
instance n=50 466.alb & 1 & 1 & Solution & 120.18 & 7 &  0.00 & 100.00\\
instance n=50 467.alb & 1 & 1 & Solution & 120.13 & 9 &  0.00 & 100.00\\
instance n=50 468.alb & 1 & 1 & Solution & 120.15 & 7 &  0.00 & 100.00\\
instance n=50 469.alb & 1 & 1 & Solution & 120.82 & 8 &  0.00 & 100.00\\
instance n=50 47.alb & 1 & 1 & Unknown & 120168.00 & - & - & -\\
instance n=50 470.alb & 1 & 1 & Solution & 120.13 & 8 &  0.00 & 100.00\\
instance n=50 471.alb & 1 & 1 & Solution & 120.19 & 8 &  0.00 & 100.00\\
instance n=50 472.alb & 1 & 1 & Solution & 120.15 & 8 &  0.00 & 100.00\\
instance n=50 473.alb & 1 & 1 & Solution & 120.15 & 7 &  0.00 & 100.00\\
instance n=50 474.alb & 1 & 1 & Solution & 120.13 & 7 &  0.00 & 100.00\\
instance n=50 475.alb & 1 & 1 & Solution & 120.13 & 6 &  0.00 & 100.00\\
instance n=50 476.alb & 1 & 1 & Solution & 120.17 & 28 &  0.00 & 100.00\\
instance n=50 477.alb & 1 & 1 & Solution & 120.33 & 29 &  0.00 & 100.00\\
instance n=50 478.alb & 1 & 1 & Solution & 120.15 & 32 &  0.00 & 100.00\\
instance n=50 479.alb & 1 & 1 & Optimal & 93.04 & 28 &  0.00 & 100.00\\
instance n=50 48.alb & 1 & 1 & Solution & 120.20 & 29 &  0.00 & 100.00\\
instance n=50 480.alb & 1 & 1 & Solution & 120.33 & 34 &  0.00 & 100.00\\
instance n=50 481.alb & 1 & 1 & Solution & 120.40 & 29 &  0.00 & 100.00\\
instance n=50 482.alb & 1 & 1 & Solution & 120.17 & 27 &  0.00 & 100.00\\
instance n=50 483.alb & 1 & 1 & Solution & 120.15 & 30 &  0.00 & 100.00\\
instance n=50 484.alb & 1 & 1 & Optimal & 27.19 & 32 &  0.00 & 100.00\\
instance n=50 485.alb & 1 & 1 & Solution & 120.17 & 31 &  0.00 & 100.00\\
instance n=50 486.alb & 1 & 1 & Optimal & 40.84 & 32 &  0.00 & 100.00\\
instance n=50 487.alb & 1 & 1 & Solution & 120.17 & 31 &  0.00 & 100.00\\
instance n=50 488.alb & 1 & 1 & Solution & 120.16 & 31 &  0.00 & 100.00\\
instance n=50 489.alb & 1 & 1 & Solution & 120.16 & 35 &  0.00 & 100.00\\
instance n=50 49.alb & 1 & 1 & Unknown & 120143.00 & - & - & -\\
instance n=50 490.alb & 1 & 1 & Solution & 120.14 & 29 &  0.00 & 100.00\\
instance n=50 491.alb & 1 & 1 & Solution & 120.15 & 35 &  0.00 & 100.00\\
instance n=50 492.alb & 1 & 1 & Solution & 120.15 & 29 &  0.00 & 100.00\\
instance n=50 493.alb & 1 & 1 & Solution & 120.17 & 30 &  0.00 & 100.00\\
instance n=50 494.alb & 1 & 1 & Solution & 120.15 & 32 &  0.00 & 100.00\\
instance n=50 495.alb & 1 & 1 & Solution & 120.15 & 34 &  0.00 & 100.00\\
instance n=50 496.alb & 1 & 1 & Solution & 120.17 & 29 &  0.00 & 100.00\\
instance n=50 497.alb & 1 & 1 & Solution & 120.15 & 30 &  0.00 & 100.00\\
instance n=50 498.alb & 1 & 1 & Solution & 120.13 & 30 &  0.00 & 100.00\\
instance n=50 499.alb & 1 & 1 & Solution & 120.17 & 34 &  0.00 & 100.00\\
instance n=50 5.alb & 1 & 1 & Solution & 120.17 & 34 &  0.00 & 100.00\\
instance n=50 50.alb & 1 & 1 & Solution & 120.19 & 46 &  0.00 & 100.00\\
instance n=50 500.alb & 1 & 1 & Solution & 120.14 & 37 &  0.00 & 100.00\\
instance n=50 501.alb & 1 & 1 & Solution & 120.14 & 12 &  0.00 & 100.00\\
instance n=50 502.alb & 1 & 1 & Solution & 120.14 & 10 &  0.00 & 100.00\\
instance n=50 503.alb & 1 & 1 & Optimal &  8.93 & 13 &  0.00 & 100.00\\
instance n=50 504.alb & 1 & 1 & Solution & 120.15 & 11 &  0.00 & 100.00\\
instance n=50 505.alb & 1 & 1 & Optimal &  8.29 & 12 &  0.00 & 100.00\\
instance n=50 506.alb & 1 & 1 & Solution & 120.15 & 11 &  0.00 & 100.00\\
instance n=50 507.alb & 1 & 1 & Solution & 120.14 & 13 &  0.00 & 100.00\\
instance n=50 508.alb & 1 & 1 & Solution & 120.14 & 14 &  0.00 & 100.00\\
instance n=50 509.alb & 1 & 1 & Solution & 120.14 & 13 &  0.00 & 100.00\\
instance n=50 51.alb & 1 & 1 & Solution & 120.17 & 12 &  0.00 & 100.00\\
instance n=50 510.alb & 1 & 1 & Solution & 120.15 & 11 &  0.00 & 100.00\\
instance n=50 511.alb & 1 & 1 & Solution & 120.13 & 13 &  0.00 & 100.00\\
instance n=50 512.alb & 1 & 1 & Solution & 120.14 & 13 &  0.00 & 100.00\\
instance n=50 513.alb & 1 & 1 & Solution & 120.12 & 12 &  0.00 & 100.00\\
instance n=50 514.alb & 1 & 1 & Solution & 120.30 & 12 &  0.00 & 100.00\\
instance n=50 515.alb & 1 & 1 & Solution & 120.15 & 11 &  0.00 & 100.00\\
instance n=50 516.alb & 1 & 1 & Solution & 120.13 & 13 &  0.00 & 100.00\\
instance n=50 517.alb & 1 & 1 & Optimal & 108.98 & 14 &  0.00 & 100.00\\
instance n=50 518.alb & 1 & 1 & Solution & 120.15 & 11 &  0.00 & 100.00\\
instance n=50 519.alb & 1 & 1 & Solution & 120.13 & 12 &  0.00 & 100.00\\
instance n=50 52.alb & 1 & 1 & Unknown & 120187.00 & - & - & -\\
instance n=50 520.alb & 1 & 1 & Solution & 120.45 & 11 &  0.00 & 100.00\\
instance n=50 521.alb & 1 & 1 & Solution & 120.16 & 10 &  0.00 & 100.00\\
instance n=50 522.alb & 1 & 1 & Solution & 120.14 & 11 &  0.00 & 100.00\\
instance n=50 523.alb & 1 & 1 & Solution & 120.16 & 11 &  0.00 & 100.00\\
instance n=50 524.alb & 1 & 1 & Solution & 120.16 & 14 &  0.00 & 100.00\\
instance n=50 525.alb & 1 & 1 & Solution & 120.16 & 11 &  0.00 & 100.00\\
instance n=50 53.alb & 1 & 1 & Unknown & 120175.00 & - & - & -\\
instance n=50 54.alb & 1 & 1 & Unknown & 120192.00 & - & - & -\\
instance n=50 55.alb & 1 & 1 & Unknown & 120176.00 & - & - & -\\
instance n=50 56.alb & 1 & 1 & Solution & 120.20 & 12 &  0.00 & 100.00\\
instance n=50 57.alb & 1 & 1 & Solution & 120.16 & 13 &  0.00 & 100.00\\
instance n=50 58.alb & 1 & 1 & Unknown & 120182.00 & - & - & -\\
instance n=50 59.alb & 1 & 1 & Unknown & 120219.00 & - & - & -\\
instance n=50 6.alb & 1 & 1 & Solution & 120.16 & 11 &  0.00 & 100.00\\
instance n=50 60.alb & 1 & 1 & Unknown & 120173.00 & - & - & -\\
instance n=50 61.alb & 1 & 1 & Unknown & 120188.00 & - & - & -\\
instance n=50 62.alb & 1 & 1 & Unknown & 120181.00 & - & - & -\\
instance n=50 63.alb & 1 & 1 & Unknown & 120165.00 & - & - & -\\
instance n=50 64.alb & 1 & 1 & Unknown & 120179.00 & - & - & -\\
instance n=50 65.alb & 1 & 1 & Unknown & 120166.00 & - & - & -\\
instance n=50 66.alb & 1 & 1 & Solution & 120.16 & 14 &  0.00 & 100.00\\
instance n=50 67.alb & 1 & 1 & Unknown & 120171.00 & - & - & -\\
instance n=50 68.alb & 1 & 1 & Unknown & 120178.00 & - & - & -\\
instance n=50 69.alb & 1 & 1 & Unknown & 120172.00 & - & - & -\\
instance n=50 7.alb & 1 & 1 & Unknown & 120208.00 & - & - & -\\
instance n=50 70.alb & 1 & 1 & Solution & 120.97 & 13 &  0.00 & 100.00\\
instance n=50 71.alb & 1 & 1 & Unknown & 120153.00 & - & - & -\\
instance n=50 72.alb & 1 & 1 & Unknown & 120166.00 & - & - & -\\
instance n=50 73.alb & 1 & 1 & Solution & 120.16 & 13 &  0.00 & 100.00\\
instance n=50 74.alb & 1 & 1 & Solution & 120.18 & 12 &  0.00 & 100.00\\
instance n=50 75.alb & 1 & 1 & Unknown & 120523.00 & - & - & -\\
instance n=50 76.alb & 1 & 1 & Unknown & 120820.00 & - & - & -\\
instance n=50 77.alb & 1 & 1 & Unknown & 120169.00 & - & - & -\\
instance n=50 78.alb & 1 & 1 & Unknown & 120614.00 & - & - & -\\
instance n=50 79.alb & 1 & 1 & Unknown & 120159.00 & - & - & -\\
instance n=50 8.alb & 1 & 1 & Solution & 120.20 & 24 &  0.00 & 100.00\\
instance n=50 80.alb & 1 & 1 & Unknown & 120168.00 & - & - & -\\
instance n=50 81.alb & 1 & 1 & Solution & 120.18 & 7 &  0.00 & 100.00\\
instance n=50 82.alb & 1 & 1 & Solution & 120.93 & 7 &  0.00 & 100.00\\
instance n=50 83.alb & 1 & 1 & Unknown & 120169.00 & - & - & -\\
instance n=50 84.alb & 1 & 1 & Unknown & 120173.00 & - & - & -\\
instance n=50 85.alb & 1 & 1 & Unknown & 120167.00 & - & - & -\\
instance n=50 86.alb & 1 & 1 & Solution & 121.15 & 8 &  0.00 & 100.00\\
instance n=50 87.alb & 1 & 1 & Unknown & 120182.00 & - & - & -\\
instance n=50 88.alb & 1 & 1 & Solution & 120.44 & 10 &  0.00 & 100.00\\
instance n=50 89.alb & 1 & 1 & Unknown & 120174.00 & - & - & -\\
instance n=50 9.alb & 1 & 1 & Unknown & 120180.00 & - & - & -\\
instance n=50 90.alb & 1 & 1 & Unknown & 120699.00 & - & - & -\\
instance n=50 91.alb & 1 & 1 & Unknown & 120169.00 & - & - & -\\
instance n=50 92.alb & 1 & 1 & Unknown & 120181.00 & - & - & -\\
instance n=50 93.alb & 1 & 1 & Unknown & 120158.00 & - & - & -\\
instance n=50 94.alb & 1 & 1 & Unknown & 120177.00 & - & - & -\\
instance n=50 95.alb & 1 & 1 & Unknown & 120178.00 & - & - & -\\
instance n=50 96.alb & 1 & 1 & Unknown & 120172.00 & - & - & -\\
instance n=50 97.alb & 1 & 1 & Solution & 120.17 & 9 &  0.00 & 100.00\\
instance n=50 98.alb & 1 & 1 & Unknown & 120161.00 & - & - & -\\
instance n=50 99.alb & 1 & 1 & Unknown & 120236.00 & - & - & -\\
\end{longtable}




\clearpage
\chapter{Test Scheduling Problems}

Due to the number of instances given, we only run problems for 30 seconds, some results are still missing. The original instance data was given in Prolog format, we generate a JSON equivalent, which is used as input to create the problems.

\section{Results for CPOptimizer}

\begin{longtable}{lrrlrrrr}
\caption{Results for Test Scheduling Problems (544 Instances)}\\\toprule
Name & \shortstack{Nr\\Jobs} & \shortstack{Nr\\Machines} & Status & Time & Makespan & Bound & \shortstack{Gap\\Percent}\\ \midrule
\endhead
\bottomrule
\endfoot
t100m10r10-1.pl.json & 100 & 10 & Solution & 30.24 & 10491 & 9055.00 & 13.69\\
t100m10r10-10.pl.json & 100 & 10 & Solution & 30.05 & 9593 & 8369.00 & 12.76\\
t100m10r10-11.pl.json & 100 & 10 & Solution & 30.06 & 5317 & 5100.00 &  4.08\\
t100m10r10-12.pl.json & 100 & 10 & Solution & 30.07 & 6539 & 5613.00 & 14.16\\
t100m10r10-13.pl.json & 100 & 10 & Solution & 30.05 & 6831 & 6786.00 &  0.66\\
t100m10r10-14.pl.json & 100 & 10 & Solution & 30.04 & 5775 & 5257.00 &  8.97\\
t100m10r10-15.pl.json & 100 & 10 & Solution & 30.04 & 6105 & 5012.00 & 17.90\\
t100m10r10-16.pl.json & 100 & 10 & Solution & 30.08 & 12563 & 11589.00 &  7.75\\
t100m10r10-17.pl.json & 100 & 10 & Solution & 30.09 & 8954 & 8114.00 &  9.38\\
t100m10r10-18.pl.json & 100 & 10 & Solution & 30.04 & 10180 & 9304.00 &  8.61\\
t100m10r10-19.pl.json & 100 & 10 & Solution & 30.09 & 9812 & 8514.00 & 13.23\\
t100m10r10-2.pl.json & 100 & 10 & Solution & 30.07 & 11593 & 9807.00 & 15.41\\
t100m10r10-20.pl.json & 100 & 10 & Solution & 30.15 & 12287 & 10686.00 & 13.03\\
t100m10r10-3.pl.json & 100 & 10 & Solution & 30.06 & 6878 & 6379.00 &  7.26\\
t100m10r10-4.pl.json & 100 & 10 & Solution & 30.11 & 11041 & 9111.00 & 17.48\\
t100m10r10-5.pl.json & 100 & 10 & Solution & 30.09 & 12157 & 11823.00 &  2.75\\
t100m10r10-6.pl.json & 100 & 10 & Solution & 30.06 & 11688 & 10914.00 &  6.62\\
t100m10r10-7.pl.json & 100 & 10 & Solution & 30.05 & 6435 & 5732.00 & 10.92\\
t100m10r10-8.pl.json & 100 & 10 & Solution & 30.10 & 11056 & 10010.00 &  9.46\\
t100m10r10-9.pl.json & 100 & 10 & Solution & 30.11 & 9878 & 7991.00 & 19.10\\
t100m10r3-1.pl.json & 100 & 10 & Optimal &  0.62 & 8711 & 8711.00 &  0.00\\
t100m10r3-10.pl.json & 100 & 10 & Optimal &  0.43 & 8958 & 8958.00 &  0.00\\
t100m10r3-11.pl.json & 100 & 10 & Optimal &  0.15 & 9560 & 9560.00 &  0.00\\
t100m10r3-12.pl.json & 100 & 10 & Optimal &  0.38 & 7892 & 7892.00 &  0.00\\
t100m10r3-13.pl.json & 100 & 10 & Optimal &  0.09 & 10078 & 10077.00 &  0.01\\
t100m10r3-14.pl.json & 100 & 10 & Optimal &  0.36 & 8681 & 8681.00 &  0.00\\
t100m10r3-15.pl.json & 100 & 10 & Optimal &  0.17 & 8810 & 8810.00 &  0.00\\
t100m10r3-16.pl.json & 100 & 10 & Optimal &  0.47 & 11182 & 11182.00 &  0.00\\
t100m10r3-17.pl.json & 100 & 10 & Optimal &  0.74 & 7534 & 7534.00 &  0.00\\
t100m10r3-18.pl.json & 100 & 10 & Solution & 30.10 & 10376 & 9934.00 &  4.26\\
t100m10r3-19.pl.json & 100 & 10 & Solution & 30.03 & 7706 & 6970.00 &  9.55\\
t100m10r3-2.pl.json & 100 & 10 & Optimal &  0.29 & 7082 & 7082.00 &  0.00\\
t100m10r3-20.pl.json & 100 & 10 & Optimal &  0.17 & 9025 & 9025.00 &  0.00\\
t100m10r3-3.pl.json & 100 & 10 & Optimal &  0.42 & 10054 & 10053.00 &  0.01\\
t100m10r3-4.pl.json & 100 & 10 & Optimal &  0.10 & 13122 & 13121.00 &  0.01\\
t100m10r3-5.pl.json & 100 & 10 & Optimal &  1.50 & 7545 & 7545.00 &  0.00\\
t100m10r3-6.pl.json & 100 & 10 & Optimal &  0.93 & 7840 & 7840.00 &  0.00\\
t100m10r3-7.pl.json & 100 & 10 & Optimal &  0.16 & 11010 & 11009.00 &  0.01\\
t100m10r3-8.pl.json & 100 & 10 & Optimal &  0.16 & 9112 & 9112.00 &  0.00\\
t100m10r3-9.pl.json & 100 & 10 & Optimal &  0.34 & 8532 & 8532.00 &  0.00\\
t100m10r5-1.pl.json & 100 & 10 & Solution & 30.04 & 7304 & 7300.00 &  0.05\\
t100m10r5-10.pl.json & 100 & 10 & Optimal &  1.42 & 6972 & 6972.00 &  0.00\\
t100m10r5-11.pl.json & 100 & 10 & Solution & 30.08 & 9091 & 8568.00 &  5.75\\
t100m10r5-12.pl.json & 100 & 10 & Optimal &  0.66 & 6538 & 6538.00 &  0.00\\
t100m10r5-13.pl.json & 100 & 10 & Optimal &  0.67 & 8972 & 8972.00 &  0.00\\
t100m10r5-14.pl.json & 100 & 10 & Solution & 30.07 & 10478 & 10347.00 &  1.25\\
t100m10r5-15.pl.json & 100 & 10 & Solution & 30.05 & 5762 & 5647.00 &  2.00\\
t100m10r5-16.pl.json & 100 & 10 & Solution & 30.04 & 7019 & 6207.00 & 11.57\\
t100m10r5-17.pl.json & 100 & 10 & Optimal &  0.23 & 6728 & 6728.00 &  0.00\\
t100m10r5-18.pl.json & 100 & 10 & Solution & 30.12 & 8987 & 8811.00 &  1.96\\
t100m10r5-19.pl.json & 100 & 10 & Optimal &  0.98 & 8885 & 8885.00 &  0.00\\
t100m10r5-2.pl.json & 100 & 10 & Optimal &  2.05 & 9010 & 9010.00 &  0.00\\
t100m10r5-20.pl.json & 100 & 10 & Optimal &  0.91 & 7022 & 7022.00 &  0.00\\
t100m10r5-3.pl.json & 100 & 10 & Optimal &  0.99 & 8820 & 8820.00 &  0.00\\
t100m10r5-4.pl.json & 100 & 10 & Optimal &  1.02 & 10753 & 10753.00 &  0.00\\
t100m10r5-5.pl.json & 100 & 10 & Optimal &  2.03 & 6608 & 6608.00 &  0.00\\
t100m10r5-6.pl.json & 100 & 10 & Solution & 30.06 & 9452 & 8456.00 & 10.54\\
t100m10r5-7.pl.json & 100 & 10 & Solution & 30.05 & 8186 & 7664.00 &  6.38\\
t100m10r5-8.pl.json & 100 & 10 & Solution & 30.12 & 11383 & 10079.00 & 11.46\\
t100m10r5-9.pl.json & 100 & 10 & Solution & 30.05 & 11649 & 10683.00 &  8.29\\
t100m20r10-1.pl.json & 100 & 20 & Solution & 30.19 & 12412 & 12180.00 &  1.87\\
t100m20r10-10.pl.json & 100 & 20 & Solution & 30.05 & 12646 & 10953.00 & 13.39\\
t100m20r10-11.pl.json & 100 & 20 & Solution & 30.09 & 8687 & 7289.00 & 16.09\\
t100m20r10-12.pl.json & 100 & 20 & Solution & 30.20 & 7391 & 6774.00 &  8.35\\
t100m20r10-13.pl.json & 100 & 20 & Solution & 30.08 & 9695 & 9229.00 &  4.81\\
t100m20r10-14.pl.json & 100 & 20 & Solution & 30.16 & 10027 & 8652.00 & 13.71\\
t100m20r10-15.pl.json & 100 & 20 & Solution & 30.04 & 6544 & 5362.00 & 18.06\\
t100m20r10-16.pl.json & 100 & 20 & Solution & 30.10 & 9264 & 8343.00 &  9.94\\
t100m20r10-17.pl.json & 100 & 20 & Solution & 30.15 & 8611 & 7381.00 & 14.28\\
t100m20r10-18.pl.json & 100 & 20 & Optimal &  1.74 & 4843 & 4843.00 &  0.00\\
t100m20r10-19.pl.json & 100 & 20 & Solution & 30.16 & 12320 & 11752.00 &  4.61\\
t100m20r10-2.pl.json & 100 & 20 & Solution & 30.14 & 7740 & 6890.00 & 10.98\\
t100m20r10-20.pl.json & 100 & 20 & Solution & 30.11 & 9873 & 8562.00 & 13.28\\
t100m20r10-3.pl.json & 100 & 20 & Solution & 30.07 & 7133 & 6295.00 & 11.75\\
t100m20r10-4.pl.json & 100 & 20 & Solution & 30.21 & 9510 & 9052.00 &  4.82\\
t100m20r10-5.pl.json & 100 & 20 & Solution & 30.13 & 9230 & 8459.00 &  8.35\\
t100m20r10-6.pl.json & 100 & 20 & Solution & 30.10 & 8781 & 7619.00 & 13.23\\
t100m20r10-7.pl.json & 100 & 20 & Solution & 30.18 & 11313 & 9767.00 & 13.67\\
t100m20r10-8.pl.json & 100 & 20 & Solution & 30.12 & 7096 & 7041.00 &  0.78\\
t100m20r10-9.pl.json & 100 & 20 & Solution & 30.19 & 10835 & 10019.00 &  7.53\\
t100m20r3-1.pl.json & 100 & 20 & Optimal &  0.59 & 6585 & 6585.00 &  0.00\\
t100m20r3-10.pl.json & 100 & 20 & Optimal &  0.28 & 8535 & 8535.00 &  0.00\\
t100m20r3-11.pl.json & 100 & 20 & Optimal &  0.60 & 9084 & 9084.00 &  0.00\\
t100m20r3-12.pl.json & 100 & 20 & Optimal &  0.28 & 9066 & 9066.00 &  0.00\\
t100m20r3-13.pl.json & 100 & 20 & Solution & 30.09 & 11412 & 9974.00 & 12.60\\
t100m20r3-14.pl.json & 100 & 20 & Optimal &  0.54 & 8786 & 8786.00 &  0.00\\
t100m20r3-15.pl.json & 100 & 20 & Optimal &  0.27 & 10205 & 10204.00 &  0.01\\
t100m20r3-16.pl.json & 100 & 20 & Optimal &  0.28 & 8856 & 8856.00 &  0.00\\
t100m20r3-17.pl.json & 100 & 20 & Optimal &  1.30 & 5451 & 5451.00 &  0.00\\
t100m20r3-18.pl.json & 100 & 20 & Optimal &  0.51 & 8752 & 8752.00 &  0.00\\
t100m20r3-19.pl.json & 100 & 20 & Solution & 30.13 & 8909 & 8860.00 &  0.55\\
t100m20r3-2.pl.json & 100 & 20 & Optimal &  0.26 & 8498 & 8498.00 &  0.00\\
t100m20r3-20.pl.json & 100 & 20 & Optimal &  0.87 & 7880 & 7880.00 &  0.00\\
t100m20r3-3.pl.json & 100 & 20 & Solution & 30.21 & 12170 & 11987.00 &  1.50\\
t100m20r3-4.pl.json & 100 & 20 & Optimal &  0.53 & 12258 & 12257.00 &  0.01\\
t100m20r3-5.pl.json & 100 & 20 & Optimal &  0.25 & 11932 & 11931.00 &  0.01\\
t100m20r3-6.pl.json & 100 & 20 & Optimal &  0.28 & 8531 & 8531.00 &  0.00\\
t100m20r3-7.pl.json & 100 & 20 & Optimal &  0.28 & 6512 & 6512.00 &  0.00\\
t100m20r3-8.pl.json & 100 & 20 & Optimal &  3.31 & 10690 & 10689.00 &  0.01\\
t100m20r3-9.pl.json & 100 & 20 & Optimal &  0.30 & 8255 & 8255.00 &  0.00\\
t100m20r5-1.pl.json & 100 & 20 & Optimal &  0.34 & 9098 & 9098.00 &  0.00\\
t100m20r5-10.pl.json & 100 & 20 & Solution & 30.04 & 8340 & 7964.00 &  4.51\\
t100m20r5-11.pl.json & 100 & 20 & Solution & 30.11 & 6828 & 5564.00 & 18.51\\
t100m20r5-12.pl.json & 100 & 20 & Optimal &  3.25 & 8704 & 8704.00 &  0.00\\
t100m20r5-13.pl.json & 100 & 20 & Optimal &  0.70 & 8880 & 8880.00 &  0.00\\
t100m20r5-14.pl.json & 100 & 20 & Solution & 30.26 & 10590 & 9727.00 &  8.15\\
t100m20r5-15.pl.json & 100 & 20 & Optimal &  0.59 & 8953 & 8953.00 &  0.00\\
t100m20r5-16.pl.json & 100 & 20 & Solution & 30.15 & 7864 & 7594.00 &  3.43\\
t100m20r5-17.pl.json & 100 & 20 & Solution & 30.15 & 5685 & 5524.00 &  2.83\\
t100m20r5-18.pl.json & 100 & 20 & Optimal &  1.06 & 6617 & 6617.00 &  0.00\\
t100m20r5-19.pl.json & 100 & 20 & Optimal &  0.42 & 9461 & 9461.00 &  0.00\\
t100m20r5-2.pl.json & 100 & 20 & Optimal &  0.38 & 9566 & 9566.00 &  0.00\\
t100m20r5-20.pl.json & 100 & 20 & Solution & 30.06 & 11569 & 10228.00 & 11.59\\
t100m20r5-3.pl.json & 100 & 20 & Optimal &  1.74 & 9366 & 9366.00 &  0.00\\
t100m20r5-4.pl.json & 100 & 20 & Solution & 30.07 & 14108 & 12456.00 & 11.71\\
t100m20r5-5.pl.json & 100 & 20 & Optimal &  0.35 & 8585 & 8585.00 &  0.00\\
t100m20r5-6.pl.json & 100 & 20 & Solution & 30.12 & 7528 & 6539.00 & 13.14\\
t100m20r5-7.pl.json & 100 & 20 & Solution & 30.13 & 11254 & 10099.00 & 10.26\\
t100m20r5-8.pl.json & 100 & 20 & Optimal &  2.49 & 5812 & 5812.00 &  0.00\\
t100m20r5-9.pl.json & 100 & 20 & Solution & 30.16 & 6634 & 6496.00 &  2.08\\
t100m50r10-1.pl.json & 100 & 50 & Solution & 30.17 & 7299 & 6941.00 &  4.90\\
t100m50r10-10.pl.json & 100 & 50 & Solution & 30.23 & 5201 & 5108.00 &  1.79\\
t100m50r10-11.pl.json & 100 & 50 & Solution & 30.09 & 4970 & 4782.00 &  3.78\\
t100m50r10-12.pl.json & 100 & 50 & Solution & 30.06 & 9335 & 9122.00 &  2.28\\
t100m50r10-13.pl.json & 100 & 50 & Solution & 30.26 & 9759 & 8828.00 &  9.54\\
t100m50r10-14.pl.json & 100 & 50 & Solution & 30.10 & 10704 & 8290.00 & 22.55\\
t100m50r10-15.pl.json & 100 & 50 & Solution & 30.08 & 8637 & 7804.00 &  9.64\\
t100m50r10-16.pl.json & 100 & 50 & Solution & 30.14 & 14087 & 12381.00 & 12.11\\
t100m50r10-17.pl.json & 100 & 50 & Solution & 30.18 & 9600 & 9151.00 &  4.68\\
t100m50r10-18.pl.json & 100 & 50 & Solution & 30.34 & 7214 & 7120.00 &  1.30\\
t100m50r10-19.pl.json & 100 & 50 & Solution & 30.18 & 8559 & 8059.00 &  5.84\\
t100m50r10-2.pl.json & 100 & 50 & Solution & 30.25 & 7968 & 7568.00 &  5.02\\
t100m50r10-20.pl.json & 100 & 50 & Solution & 30.09 & 8421 & 7939.00 &  5.72\\
t100m50r10-3.pl.json & 100 & 50 & Optimal &  0.33 & 6937 & 6937.00 &  0.00\\
t100m50r10-4.pl.json & 100 & 50 & Solution & 30.16 & 9952 & 8525.00 & 14.34\\
t100m50r10-5.pl.json & 100 & 50 & Optimal &  1.35 & 9859 & 9859.00 &  0.00\\
t100m50r10-6.pl.json & 100 & 50 & Solution & 30.31 & 7696 & 6837.00 & 11.16\\
t100m50r10-7.pl.json & 100 & 50 & Optimal &  1.17 & 9542 & 9542.00 &  0.00\\
t100m50r10-8.pl.json & 100 & 50 & Solution & 30.07 & 10719 & 9176.00 & 14.39\\
t100m50r10-9.pl.json & 100 & 50 & Solution & 30.07 & 10411 & 9375.00 &  9.95\\
t100m50r3-1.pl.json & 100 & 50 & Optimal &  0.46 & 9937 & 9937.00 &  0.00\\
t100m50r3-10.pl.json & 100 & 50 & Solution & 30.06 & 8946 & 8877.00 &  0.77\\
t100m50r3-11.pl.json & 100 & 50 & Optimal &  1.01 & 6141 & 6141.00 &  0.00\\
t100m50r3-12.pl.json & 100 & 50 & Optimal &  0.87 & 6473 & 6473.00 &  0.00\\
t100m50r3-13.pl.json & 100 & 50 & Optimal &  0.47 & 8653 & 8653.00 &  0.00\\
t100m50r3-14.pl.json & 100 & 50 & Solution & 30.09 & 13018 & 12796.00 &  1.71\\
t100m50r3-15.pl.json & 100 & 50 & Optimal &  3.29 & 9056 & 9056.00 &  0.00\\
t100m50r3-16.pl.json & 100 & 50 & Optimal &  0.41 & 8680 & 8680.00 &  0.00\\
t100m50r3-17.pl.json & 100 & 50 & Optimal &  0.55 & 8197 & 8197.00 &  0.00\\
t100m50r3-18.pl.json & 100 & 50 & Optimal &  0.38 & 9318 & 9318.00 &  0.00\\
t100m50r3-19.pl.json & 100 & 50 & Optimal &  0.35 & 12265 & 12264.00 &  0.01\\
t100m50r3-2.pl.json & 100 & 50 & Optimal &  0.79 & 11030 & 11029.00 &  0.01\\
t100m50r3-20.pl.json & 100 & 50 & Optimal &  0.38 & 7662 & 7662.00 &  0.00\\
t100m50r3-3.pl.json & 100 & 50 & Optimal &  0.46 & 5348 & 5348.00 &  0.00\\
t100m50r3-4.pl.json & 100 & 50 & Optimal &  2.02 & 7800 & 7800.00 &  0.00\\
t100m50r3-5.pl.json & 100 & 50 & Optimal &  0.83 & 4207 & 4207.00 &  0.00\\
t100m50r3-6.pl.json & 100 & 50 & Optimal &  6.31 & 10596 & 10596.00 &  0.00\\
t100m50r3-7.pl.json & 100 & 50 & Optimal &  0.43 & 7826 & 7826.00 &  0.00\\
t100m50r3-8.pl.json & 100 & 50 & Optimal &  0.81 & 7865 & 7865.00 &  0.00\\
t100m50r3-9.pl.json & 100 & 50 & Optimal &  0.48 & 7891 & 7891.00 &  0.00\\
t100m50r5-1.pl.json & 100 & 50 & Optimal &  0.78 & 7926 & 7926.00 &  0.00\\
t100m50r5-10.pl.json & 100 & 50 & Solution & 30.23 & 7299 & 6521.00 & 10.66\\
t100m50r5-11.pl.json & 100 & 50 & Optimal &  1.56 & 9417 & 9417.00 &  0.00\\
t100m50r5-12.pl.json & 100 & 50 & Optimal &  3.81 & 8824 & 8824.00 &  0.00\\
t100m50r5-13.pl.json & 100 & 50 & Solution & 30.05 & 10473 & 9115.00 & 12.97\\
t100m50r5-14.pl.json & 100 & 50 & Solution & 30.33 & 7503 & 7134.00 &  4.92\\
t100m50r5-15.pl.json & 100 & 50 & Solution & 30.06 & 10141 & 9853.00 &  2.84\\
t100m50r5-16.pl.json & 100 & 50 & Optimal &  0.47 & 6481 & 6481.00 &  0.00\\
t100m50r5-17.pl.json & 100 & 50 & Optimal &  0.50 & 6129 & 6129.00 &  0.00\\
t100m50r5-18.pl.json & 100 & 50 & Solution & 30.06 & 9100 & 8337.00 &  8.38\\
t100m50r5-19.pl.json & 100 & 50 & Solution & 30.20 & 6762 & 6356.00 &  6.00\\
t100m50r5-2.pl.json & 100 & 50 & Optimal &  1.00 & 6651 & 6651.00 &  0.00\\
t100m50r5-20.pl.json & 100 & 50 & Solution & 30.05 & 6894 & 6667.00 &  3.29\\
t100m50r5-3.pl.json & 100 & 50 & Solution & 30.19 & 7944 & 7857.00 &  1.10\\
t100m50r5-4.pl.json & 100 & 50 & Optimal &  1.39 & 8296 & 8296.00 &  0.00\\
t100m50r5-5.pl.json & 100 & 50 & Optimal &  1.26 & 9977 & 9977.00 &  0.00\\
t100m50r5-6.pl.json & 100 & 50 & Optimal &  0.91 & 8240 & 8240.00 &  0.00\\
t100m50r5-7.pl.json & 100 & 50 & Optimal &  1.34 & 10904 & 10903.00 &  0.01\\
t100m50r5-8.pl.json & 100 & 50 & Optimal &  0.90 & 8293 & 8293.00 &  0.00\\
t100m50r5-9.pl.json & 100 & 50 & Solution & 30.06 & 7879 & 7622.00 &  3.26\\
t20m10r10-1.pl.json & 20 & 10 & Optimal &  0.07 & 1337 & 1337.00 &  0.00\\
t20m10r10-10.pl.json & 20 & 10 & Optimal &  0.05 & 3882 & 3882.00 &  0.00\\
t20m10r10-11.pl.json & 20 & 10 & Optimal &  0.06 & 2002 & 2002.00 &  0.00\\
t20m10r10-12.pl.json & 20 & 10 & Optimal &  0.31 & 1257 & 1257.00 &  0.00\\
t20m10r10-13.pl.json & 20 & 10 & Optimal &  0.06 & 2110 & 2110.00 &  0.00\\
t20m10r10-14.pl.json & 20 & 10 & Optimal &  2.43 & 2546 & 2546.00 &  0.00\\
t20m10r10-15.pl.json & 20 & 10 & Optimal &  0.05 & 3344 & 3344.00 &  0.00\\
t20m10r10-16.pl.json & 20 & 10 & Optimal &  3.87 & 1643 & 1643.00 &  0.00\\
t20m10r10-17.pl.json & 20 & 10 & Optimal &  0.43 & 1069 & 1069.00 &  0.00\\
t20m10r10-18.pl.json & 20 & 10 & Optimal &  0.04 & 3041 & 3041.00 &  0.00\\
t20m10r10-19.pl.json & 20 & 10 & Optimal &  0.04 & 2422 & 2422.00 &  0.00\\
t20m10r10-2.pl.json & 20 & 10 & Optimal &  0.05 & 1819 & 1819.00 &  0.00\\
t20m10r10-20.pl.json & 20 & 10 & Optimal &  0.05 & 1595 & 1595.00 &  0.00\\
t20m10r10-3.pl.json & 20 & 10 & Solution & 30.02 & 843 & 771.00 &  8.54\\
t20m10r10-4.pl.json & 20 & 10 & Optimal &  0.04 & 1396 & 1396.00 &  0.00\\
t20m10r10-5.pl.json & 20 & 10 & Optimal &  0.05 & 1710 & 1710.00 &  0.00\\
t20m10r10-6.pl.json & 20 & 10 & Optimal &  0.03 & 2434 & 2434.00 &  0.00\\
t20m10r10-7.pl.json & 20 & 10 & Optimal &  0.41 & 2696 & 2696.00 &  0.00\\
t20m10r10-8.pl.json & 20 & 10 & Optimal &  0.03 & 1329 & 1329.00 &  0.00\\
t20m10r10-9.pl.json & 20 & 10 & Optimal &  4.48 & 2933 & 2933.00 &  0.00\\
t20m10r3-1.pl.json & 20 & 10 & Optimal &  0.05 & 1876 & 1876.00 &  0.00\\
t20m10r3-10.pl.json & 20 & 10 & Optimal &  0.05 & 1652 & 1652.00 &  0.00\\
t20m10r3-11.pl.json & 20 & 10 & Optimal &  0.04 & 1640 & 1640.00 &  0.00\\
t20m10r3-12.pl.json & 20 & 10 & Optimal &  0.03 & 1758 & 1758.00 &  0.00\\
t20m10r3-13.pl.json & 20 & 10 & Optimal &  0.03 & 3099 & 3099.00 &  0.00\\
t20m10r3-14.pl.json & 20 & 10 & Solution & 30.01 & 3891 & 3520.00 &  9.53\\
t20m10r3-15.pl.json & 20 & 10 & Optimal &  0.05 & 1433 & 1433.00 &  0.00\\
t20m10r3-16.pl.json & 20 & 10 & Optimal &  0.04 & 1564 & 1564.00 &  0.00\\
t20m10r3-17.pl.json & 20 & 10 & Optimal &  0.04 & 2321 & 2321.00 &  0.00\\
t20m10r3-18.pl.json & 20 & 10 & Solution & 30.01 & 821 & 746.00 &  9.14\\
t20m10r3-19.pl.json & 20 & 10 & Optimal &  0.09 & 1236 & 1236.00 &  0.00\\
t20m10r3-2.pl.json & 20 & 10 & Optimal &  0.05 & 3258 & 3258.00 &  0.00\\
t20m10r3-20.pl.json & 20 & 10 & Optimal &  0.04 & 2168 & 2168.00 &  0.00\\
t20m10r3-3.pl.json & 20 & 10 & Optimal &  0.03 & 2255 & 2255.00 &  0.00\\
t20m10r3-4.pl.json & 20 & 10 & Optimal &  0.03 & 2707 & 2707.00 &  0.00\\
t20m10r3-5.pl.json & 20 & 10 & Optimal &  0.05 & 2381 & 2381.00 &  0.00\\
t20m10r3-6.pl.json & 20 & 10 & Optimal &  0.03 & 3043 & 3043.00 &  0.00\\
t20m10r3-7.pl.json & 20 & 10 & Optimal &  0.05 & 1738 & 1738.00 &  0.00\\
t20m10r3-8.pl.json & 20 & 10 & Optimal &  2.74 & 1278 & 1278.00 &  0.00\\
t20m10r3-9.pl.json & 20 & 10 & Optimal &  0.04 & 2874 & 2874.00 &  0.00\\
t20m10r5-1.pl.json & 20 & 10 & Optimal &  0.04 & 2586 & 2586.00 &  0.00\\
t20m10r5-10.pl.json & 20 & 10 & Optimal &  0.05 & 2260 & 2260.00 &  0.00\\
t20m10r5-11.pl.json & 20 & 10 & Optimal &  0.03 & 3487 & 3487.00 &  0.00\\
t20m10r5-12.pl.json & 20 & 10 & Optimal &  0.03 & 1559 & 1559.00 &  0.00\\
t20m10r5-13.pl.json & 20 & 10 & Optimal &  0.22 & 1457 & 1457.00 &  0.00\\
t20m10r5-14.pl.json & 20 & 10 & Optimal &  0.06 & 1141 & 1141.00 &  0.00\\
t20m10r5-15.pl.json & 20 & 10 & Optimal &  0.18 & 821 & 821.00 &  0.00\\
t20m10r5-16.pl.json & 20 & 10 & Optimal &  0.03 & 2910 & 2910.00 &  0.00\\
t20m10r5-17.pl.json & 20 & 10 & Optimal &  0.05 & 2337 & 2337.00 &  0.00\\
t20m10r5-18.pl.json & 20 & 10 & Optimal &  3.96 & 2920 & 2920.00 &  0.00\\
t20m10r5-19.pl.json & 20 & 10 & Optimal &  0.03 & 1952 & 1952.00 &  0.00\\
t20m10r5-2.pl.json & 20 & 10 & Optimal &  0.03 & 1639 & 1639.00 &  0.00\\
t20m10r5-20.pl.json & 20 & 10 & Optimal &  0.03 & 2660 & 2660.00 &  0.00\\
t20m10r5-3.pl.json & 20 & 10 & Optimal &  0.05 & 1406 & 1406.00 &  0.00\\
t20m10r5-4.pl.json & 20 & 10 & Optimal &  0.05 & 2658 & 2658.00 &  0.00\\
t20m10r5-5.pl.json & 20 & 10 & Optimal &  0.08 & 794 & 794.00 &  0.00\\
t20m10r5-6.pl.json & 20 & 10 & Optimal &  0.03 & 2398 & 2398.00 &  0.00\\
t20m10r5-7.pl.json & 20 & 10 & Optimal &  0.04 & 1430 & 1430.00 &  0.00\\
t20m10r5-8.pl.json & 20 & 10 & Optimal &  0.06 & 976 & 976.00 &  0.00\\
t20m10r5-9.pl.json & 20 & 10 & Optimal &  0.04 & 2953 & 2953.00 &  0.00\\
t30m10r10-1.pl.json & 30 & 10 & Optimal &  6.81 & 3344 & 3344.00 &  0.00\\
t30m10r10-10.pl.json & 30 & 10 & Solution & 30.03 & 4692 & 4146.00 & 11.64\\
t30m10r10-11.pl.json & 30 & 10 & Optimal &  0.06 & 2905 & 2905.00 &  0.00\\
t30m10r10-12.pl.json & 30 & 10 & Optimal &  0.06 & 3672 & 3672.00 &  0.00\\
t30m10r10-13.pl.json & 30 & 10 & Optimal &  0.36 & 2778 & 2778.00 &  0.00\\
t30m10r10-14.pl.json & 30 & 10 & Optimal &  2.31 & 2741 & 2741.00 &  0.00\\
t30m10r10-15.pl.json & 30 & 10 & Optimal &  0.05 & 2388 & 2388.00 &  0.00\\
t30m10r10-16.pl.json & 30 & 10 & Solution & 30.03 & 4225 & 3900.00 &  7.69\\
t30m10r10-17.pl.json & 30 & 10 & Optimal &  0.08 & 1504 & 1504.00 &  0.00\\
t30m10r10-18.pl.json & 30 & 10 & Solution & 30.03 & 3287 & 2730.00 & 16.95\\
t30m10r10-19.pl.json & 30 & 10 & Optimal &  0.05 & 3874 & 3874.00 &  0.00\\
t30m10r10-2.pl.json & 30 & 10 & Optimal &  0.03 & 3169 & 3169.00 &  0.00\\
t30m10r10-20.pl.json & 30 & 10 & Optimal &  0.05 & 2691 & 2691.00 &  0.00\\
t30m10r10-3.pl.json & 30 & 10 & Solution & 30.01 & 3360 & 2851.00 & 15.15\\
t30m10r10-4.pl.json & 30 & 10 & Optimal &  0.06 & 3452 & 3452.00 &  0.00\\
t30m10r10-5.pl.json & 30 & 10 & Optimal &  0.05 & 2785 & 2785.00 &  0.00\\
t30m10r10-6.pl.json & 30 & 10 & Solution & 30.03 & 1013 & 775.00 & 23.49\\
t30m10r10-7.pl.json & 30 & 10 & Optimal & 27.69 & 3755 & 3755.00 &  0.00\\
t30m10r10-8.pl.json & 30 & 10 & Solution & 30.02 & 4613 & 4160.00 &  9.82\\
t30m10r10-9.pl.json & 30 & 10 & Optimal &  0.03 & 2770 & 2770.00 &  0.00\\
t30m10r3-1.pl.json & 30 & 10 & Optimal &  0.05 & 2901 & 2901.00 &  0.00\\
t30m10r3-10.pl.json & 30 & 10 & Optimal &  0.04 & 4829 & 4829.00 &  0.00\\
t30m10r3-11.pl.json & 30 & 10 & Optimal &  0.04 & 2584 & 2584.00 &  0.00\\
t30m10r3-12.pl.json & 30 & 10 & Optimal &  0.03 & 2130 & 2130.00 &  0.00\\
t30m10r3-13.pl.json & 30 & 10 & Optimal &  0.03 & 4253 & 4253.00 &  0.00\\
t30m10r3-14.pl.json & 30 & 10 & Optimal &  0.17 & 1393 & 1393.00 &  0.00\\
t30m10r3-15.pl.json & 30 & 10 & Optimal &  0.03 & 4149 & 4149.00 &  0.00\\
t30m10r3-16.pl.json & 30 & 10 & Optimal &  0.05 & 2027 & 2027.00 &  0.00\\
t30m10r3-17.pl.json & 30 & 10 & Optimal &  0.05 & 2975 & 2975.00 &  0.00\\
t30m10r3-18.pl.json & 30 & 10 & Optimal &  0.05 & 5477 & 5477.00 &  0.00\\
t30m10r3-19.pl.json & 30 & 10 & Solution & 30.01 & 1289 & 1042.00 & 19.16\\
t30m10r3-2.pl.json & 30 & 10 & Optimal &  0.14 & 2523 & 2523.00 &  0.00\\
t30m10r3-20.pl.json & 30 & 10 & Optimal &  0.05 & 4754 & 4754.00 &  0.00\\
t30m10r3-3.pl.json & 30 & 10 & Optimal &  0.04 & 2793 & 2793.00 &  0.00\\
t30m10r3-4.pl.json & 30 & 10 & Optimal &  0.69 & 2809 & 2809.00 &  0.00\\
t30m10r3-5.pl.json & 30 & 10 & Optimal &  0.04 & 3758 & 3758.00 &  0.00\\
t30m10r3-6.pl.json & 30 & 10 & Optimal &  0.05 & 2870 & 2870.00 &  0.00\\
t30m10r3-7.pl.json & 30 & 10 & Optimal &  0.05 & 2122 & 2122.00 &  0.00\\
t30m10r3-8.pl.json & 30 & 10 & Optimal &  0.03 & 2862 & 2862.00 &  0.00\\
t30m10r3-9.pl.json & 30 & 10 & Optimal &  0.08 & 2754 & 2754.00 &  0.00\\
t30m10r5-1.pl.json & 30 & 10 & Optimal &  0.04 & 1998 & 1998.00 &  0.00\\
t30m10r5-10.pl.json & 30 & 10 & Optimal &  0.04 & 3743 & 3743.00 &  0.00\\
t30m10r5-11.pl.json & 30 & 10 & Optimal &  0.05 & 2138 & 2138.00 &  0.00\\
t30m10r5-12.pl.json & 30 & 10 & Optimal &  0.05 & 2251 & 2251.00 &  0.00\\
t30m10r5-13.pl.json & 30 & 10 & Optimal &  0.05 & 2632 & 2632.00 &  0.00\\
t30m10r5-14.pl.json & 30 & 10 & Optimal &  0.06 & 2201 & 2201.00 &  0.00\\
t30m10r5-15.pl.json & 30 & 10 & Optimal &  0.09 & 2339 & 2339.00 &  0.00\\
t30m10r5-16.pl.json & 30 & 10 & Optimal &  0.05 & 4293 & 4293.00 &  0.00\\
t30m10r5-17.pl.json & 30 & 10 & Optimal &  0.11 & 1314 & 1314.00 &  0.00\\
t30m10r5-18.pl.json & 30 & 10 & Optimal &  0.07 & 2169 & 2169.00 &  0.00\\
t30m10r5-19.pl.json & 30 & 10 & Solution & 30.01 & 1346 & 1279.00 &  4.98\\
t30m10r5-2.pl.json & 30 & 10 & Optimal &  0.05 & 2399 & 2399.00 &  0.00\\
t30m10r5-20.pl.json & 30 & 10 & Optimal &  0.05 & 1486 & 1486.00 &  0.00\\
t30m10r5-3.pl.json & 30 & 10 & Optimal &  0.05 & 2494 & 2494.00 &  0.00\\
t30m10r5-4.pl.json & 30 & 10 & Optimal &  0.03 & 3405 & 3405.00 &  0.00\\
t30m10r5-5.pl.json & 30 & 10 & Solution & 30.02 & 5243 & 4550.00 & 13.22\\
t30m10r5-6.pl.json & 30 & 10 & Optimal &  0.05 & 2382 & 2382.00 &  0.00\\
t30m10r5-7.pl.json & 30 & 10 & Optimal &  0.06 & 2018 & 2018.00 &  0.00\\
t30m10r5-8.pl.json & 30 & 10 & Optimal &  0.04 & 3089 & 3089.00 &  0.00\\
t30m10r5-9.pl.json & 30 & 10 & Optimal &  0.05 & 3704 & 3704.00 &  0.00\\
t30m20r10-1.pl.json & 30 & 20 & Solution & 30.03 & 3702 & 2850.00 & 23.01\\
t30m20r10-10.pl.json & 30 & 20 & Optimal &  4.79 & 2508 & 2508.00 &  0.00\\
t30m20r10-11.pl.json & 30 & 20 & Solution & 30.02 & 3648 & 3482.00 &  4.55\\
t30m20r10-12.pl.json & 30 & 20 & Optimal &  0.09 & 4214 & 4214.00 &  0.00\\
t30m20r10-13.pl.json & 30 & 20 & Optimal & 15.77 & 3980 & 3980.00 &  0.00\\
t30m20r10-14.pl.json & 30 & 20 & Optimal & 13.92 & 3141 & 3141.00 &  0.00\\
t30m20r10-15.pl.json & 30 & 20 & Solution & 30.02 & 4322 & 3457.00 & 20.01\\
t30m20r10-16.pl.json & 30 & 20 & Optimal &  0.11 & 4002 & 4002.00 &  0.00\\
t30m20r10-17.pl.json & 30 & 20 & Solution & 30.02 & 4161 & 3363.00 & 19.18\\
t30m20r10-18.pl.json & 30 & 20 & Optimal &  6.32 & 1992 & 1992.00 &  0.00\\
t30m20r10-19.pl.json & 30 & 20 & Solution & 30.04 & 2789 & 2250.00 & 19.33\\
t30m20r10-2.pl.json & 30 & 20 & Solution & 30.02 & 3982 & 3447.00 & 13.44\\
t30m20r10-20.pl.json & 30 & 20 & Optimal &  5.60 & 2314 & 2314.00 &  0.00\\
t30m20r10-3.pl.json & 30 & 20 & Optimal &  0.09 & 2158 & 2158.00 &  0.00\\
t30m20r10-4.pl.json & 30 & 20 & Solution & 30.03 & 4040 & 3217.00 & 20.37\\
t30m20r10-5.pl.json & 30 & 20 & Optimal &  0.09 & 1237 & 1237.00 &  0.00\\
t30m20r10-6.pl.json & 30 & 20 & Solution & 30.04 & 3770 & 3600.00 &  4.51\\
t30m20r10-7.pl.json & 30 & 20 & Optimal &  0.08 & 2266 & 2266.00 &  0.00\\
t30m20r10-8.pl.json & 30 & 20 & Optimal &  2.08 & 1855 & 1855.00 &  0.00\\
t30m20r10-9.pl.json & 30 & 20 & Optimal &  3.60 & 2028 & 2028.00 &  0.00\\
t30m20r3-1.pl.json & 30 & 20 & Optimal &  0.08 & 2200 & 2200.00 &  0.00\\
t30m20r3-10.pl.json & 30 & 20 & Optimal &  0.07 & 3291 & 3291.00 &  0.00\\
t30m20r3-11.pl.json & 30 & 20 & Optimal &  0.08 & 4473 & 4473.00 &  0.00\\
t30m20r3-12.pl.json & 30 & 20 & Solution & 30.02 & 5060 & 4931.00 &  2.55\\
t30m20r3-13.pl.json & 30 & 20 & Optimal &  0.07 & 3536 & 3536.00 &  0.00\\
t30m20r3-14.pl.json & 30 & 20 & Optimal &  0.08 & 3432 & 3432.00 &  0.00\\
t30m20r3-15.pl.json & 30 & 20 & Optimal &  0.08 & 3463 & 3463.00 &  0.00\\
t30m20r3-16.pl.json & 30 & 20 & Optimal &  0.07 & 3893 & 3893.00 &  0.00\\
t30m20r3-17.pl.json & 30 & 20 & Optimal &  0.07 & 1892 & 1892.00 &  0.00\\
t30m20r3-18.pl.json & 30 & 20 & Optimal &  0.08 & 2653 & 2653.00 &  0.00\\
t30m20r3-19.pl.json & 30 & 20 & Optimal &  0.08 & 3141 & 3141.00 &  0.00\\
t30m20r3-2.pl.json & 30 & 20 & Optimal &  0.08 & 1251 & 1251.00 &  0.00\\
t30m20r3-20.pl.json & 30 & 20 & Optimal &  5.77 & 2745 & 2745.00 &  0.00\\
t30m20r3-3.pl.json & 30 & 20 & Optimal &  0.08 & 3434 & 3434.00 &  0.00\\
t30m20r3-4.pl.json & 30 & 20 & Optimal &  0.10 & 2394 & 2394.00 &  0.00\\
t30m20r3-5.pl.json & 30 & 20 & Optimal &  0.06 & 3776 & 3776.00 &  0.00\\
t30m20r3-6.pl.json & 30 & 20 & Optimal &  0.08 & 2250 & 2250.00 &  0.00\\
t30m20r3-7.pl.json & 30 & 20 & Optimal &  0.12 & 1693 & 1693.00 &  0.00\\
t30m20r3-8.pl.json & 30 & 20 & Optimal &  0.08 & 4997 & 4997.00 &  0.00\\
t30m20r3-9.pl.json & 30 & 20 & Optimal &  0.08 & 4898 & 4898.00 &  0.00\\
t30m20r5-1.pl.json & 30 & 20 & Solution & 30.02 & 3195 & 2787.00 & 12.77\\
t30m20r5-10.pl.json & 30 & 20 & Optimal &  5.14 & 2133 & 2133.00 &  0.00\\
t30m20r5-11.pl.json & 30 & 20 & Optimal &  0.08 & 3974 & 3974.00 &  0.00\\
t30m20r5-12.pl.json & 30 & 20 & Optimal &  0.08 & 2197 & 2197.00 &  0.00\\
t30m20r5-13.pl.json & 30 & 20 & Optimal &  0.09 & 2296 & 2296.00 &  0.00\\
t30m20r5-14.pl.json & 30 & 20 & Optimal &  0.07 & 3861 & 3861.00 &  0.00\\
t30m20r5-15.pl.json & 30 & 20 & Optimal &  0.08 & 2353 & 2353.00 &  0.00\\
t30m20r5-16.pl.json & 30 & 20 & Optimal &  4.27 & 2751 & 2751.00 &  0.00\\
t30m20r5-17.pl.json & 30 & 20 & Optimal &  0.08 & 3555 & 3555.00 &  0.00\\
t30m20r5-18.pl.json & 30 & 20 & Optimal &  0.06 & 2384 & 2384.00 &  0.00\\
t30m20r5-19.pl.json & 30 & 20 & Optimal &  0.11 & 2080 & 2080.00 &  0.00\\
t30m20r5-2.pl.json & 30 & 20 & Optimal &  0.10 & 1715 & 1715.00 &  0.00\\
t30m20r5-20.pl.json & 30 & 20 & Optimal &  0.10 & 4176 & 4176.00 &  0.00\\
t30m20r5-3.pl.json & 30 & 20 & Solution & 30.05 & 4528 & 4037.00 & 10.84\\
t30m20r5-4.pl.json & 30 & 20 & Optimal &  0.09 & 3083 & 3083.00 &  0.00\\
t30m20r5-5.pl.json & 30 & 20 & Optimal &  0.08 & 1969 & 1969.00 &  0.00\\
t30m20r5-6.pl.json & 30 & 20 & Optimal &  0.08 & 4250 & 4250.00 &  0.00\\
t30m20r5-7.pl.json & 30 & 20 & Optimal &  0.08 & 3036 & 3036.00 &  0.00\\
t30m20r5-8.pl.json & 30 & 20 & Optimal &  1.55 & 2834 & 2834.00 &  0.00\\
t30m20r5-9.pl.json & 30 & 20 & Optimal &  0.10 & 2343 & 2343.00 &  0.00\\
t40m10r10-1.pl.json & 40 & 10 & Optimal &  0.11 & 2514 & 2514.00 &  0.00\\
t40m10r10-10.pl.json & 40 & 10 & Optimal &  0.08 & 3557 & 3557.00 &  0.00\\
t40m10r10-11.pl.json & 40 & 10 & Solution & 30.03 & 4556 & 4262.00 &  6.45\\
t40m10r10-12.pl.json & 40 & 10 & Solution & 30.01 & 5225 & 4355.00 & 16.65\\
t40m10r10-13.pl.json & 40 & 10 & Optimal & 16.47 & 2789 & 2789.00 &  0.00\\
t40m10r10-14.pl.json & 40 & 10 & Optimal &  0.47 & 1648 & 1648.00 &  0.00\\
t40m10r10-15.pl.json & 40 & 10 & Optimal &  2.03 & 1844 & 1844.00 &  0.00\\
t40m10r10-16.pl.json & 40 & 10 & Solution & 30.02 & 3749 & 3380.00 &  9.84\\
t40m10r10-17.pl.json & 40 & 10 & Optimal &  0.14 & 2363 & 2363.00 &  0.00\\
t40m10r10-18.pl.json & 40 & 10 & Optimal &  0.06 & 4973 & 4973.00 &  0.00\\
t40m10r10-19.pl.json & 40 & 10 & Optimal &  0.06 & 3181 & 3181.00 &  0.00\\
t40m10r10-2.pl.json & 40 & 10 & Optimal &  0.20 & 2350 & 2350.00 &  0.00\\
t40m10r10-20.pl.json & 40 & 10 & Solution & 30.04 & 2730 & 2470.00 &  9.52\\
t40m10r10-3.pl.json & 40 & 10 & Optimal &  0.06 & 3717 & 3717.00 &  0.00\\
t40m10r10-4.pl.json & 40 & 10 & Optimal &  0.08 & 3414 & 3414.00 &  0.00\\
t40m10r10-5.pl.json & 40 & 10 & Optimal &  5.68 & 2852 & 2852.00 &  0.00\\
t40m10r10-6.pl.json & 40 & 10 & Solution & 30.02 & 3262 & 2910.00 & 10.79\\
t40m10r10-7.pl.json & 40 & 10 & Optimal &  0.08 & 4572 & 4572.00 &  0.00\\
t40m10r10-8.pl.json & 40 & 10 & Solution & 30.03 & 3776 & 3385.00 & 10.35\\
t40m10r10-9.pl.json & 40 & 10 & Optimal &  0.11 & 2524 & 2524.00 &  0.00\\
t40m10r3-1.pl.json & 40 & 10 & Optimal &  0.09 & 4832 & 4832.00 &  0.00\\
t40m10r3-10.pl.json & 40 & 10 & Optimal &  0.15 & 2442 & 2442.00 &  0.00\\
t40m10r3-11.pl.json & 40 & 10 & Optimal &  0.06 & 3218 & 3218.00 &  0.00\\
t40m10r3-12.pl.json & 40 & 10 & Optimal &  0.06 & 3863 & 3863.00 &  0.00\\
t40m10r3-13.pl.json & 40 & 10 & Optimal &  0.07 & 3564 & 3564.00 &  0.00\\
t40m10r3-14.pl.json & 40 & 10 & Optimal &  0.08 & 4913 & 4913.00 &  0.00\\
t40m10r3-15.pl.json & 40 & 10 & Optimal &  0.26 & 3785 & 3785.00 &  0.00\\
t40m10r3-16.pl.json & 40 & 10 & Optimal &  0.11 & 2840 & 2840.00 &  0.00\\
t40m10r3-17.pl.json & 40 & 10 & Optimal &  0.06 & 5506 & 5506.00 &  0.00\\
t40m10r3-18.pl.json & 40 & 10 & Optimal &  0.08 & 3848 & 3848.00 &  0.00\\
t40m10r3-19.pl.json & 40 & 10 & Optimal &  0.11 & 2259 & 2259.00 &  0.00\\
t40m10r3-2.pl.json & 40 & 10 & Solution & 30.04 & 1727 & 1589.00 &  7.99\\
t40m10r3-20.pl.json & 40 & 10 & Optimal &  0.09 & 4157 & 4157.00 &  0.00\\
t40m10r3-3.pl.json & 40 & 10 & Optimal &  0.08 & 4903 & 4903.00 &  0.00\\
t40m10r3-4.pl.json & 40 & 10 & Solution & 30.03 & 1635 & 1341.00 & 17.98\\
t40m10r3-5.pl.json & 40 & 10 & Optimal &  0.16 & 1984 & 1984.00 &  0.00\\
t40m10r3-6.pl.json & 40 & 10 & Optimal &  0.06 & 5005 & 5005.00 &  0.00\\
t40m10r3-7.pl.json & 40 & 10 & Solution & 30.03 & 5545 & 5188.00 &  6.44\\
t40m10r3-8.pl.json & 40 & 10 & Optimal &  0.08 & 3658 & 3658.00 &  0.00\\
t40m10r3-9.pl.json & 40 & 10 & Optimal &  0.19 & 3830 & 3830.00 &  0.00\\
t40m10r5-1.pl.json & 40 & 10 & Optimal &  0.08 & 4857 & 4857.00 &  0.00\\
t40m10r5-10.pl.json & 40 & 10 & Optimal &  0.08 & 3989 & 3989.00 &  0.00\\
t40m10r5-11.pl.json & 40 & 10 & Optimal &  0.08 & 5238 & 5238.00 &  0.00\\
t40m10r5-12.pl.json & 40 & 10 & Optimal &  0.08 & 4584 & 4584.00 &  0.00\\
t40m10r5-13.pl.json & 40 & 10 & Optimal &  0.09 & 2307 & 2307.00 &  0.00\\
t40m10r5-14.pl.json & 40 & 10 & Optimal &  0.30 & 1826 & 1826.00 &  0.00\\
t40m10r5-15.pl.json & 40 & 10 & Optimal &  0.11 & 1926 & 1926.00 &  0.00\\
t40m10r5-16.pl.json & 40 & 10 & Optimal &  0.11 & 5216 & 5216.00 &  0.00\\
t40m10r5-17.pl.json & 40 & 10 & Optimal &  0.08 & 7162 & 7162.00 &  0.00\\
t40m10r5-18.pl.json & 40 & 10 & Optimal &  0.11 & 4892 & 4892.00 &  0.00\\
t40m10r5-19.pl.json & 40 & 10 & Optimal &  0.08 & 4027 & 4027.00 &  0.00\\
t40m10r5-2.pl.json & 40 & 10 & Optimal &  8.38 & 4099 & 4099.00 &  0.00\\
t40m10r5-20.pl.json & 40 & 10 & Solution & 30.02 & 4899 & 4755.00 &  2.94\\
t40m10r5-3.pl.json & 40 & 10 & Optimal &  0.08 & 3113 & 3113.00 &  0.00\\
t40m10r5-4.pl.json & 40 & 10 & Optimal &  0.10 & 6626 & 6626.00 &  0.00\\
t40m10r5-5.pl.json & 40 & 10 & Optimal &  0.08 & 3828 & 3828.00 &  0.00\\
t40m10r5-6.pl.json & 40 & 10 & Optimal &  0.09 & 4213 & 4213.00 &  0.00\\
t40m10r5-7.pl.json & 40 & 10 & Optimal &  0.28 & 4303 & 4303.00 &  0.00\\
t40m10r5-8.pl.json & 40 & 10 & Solution & 30.03 & 3559 & 3189.00 & 10.40\\
t40m10r5-9.pl.json & 40 & 10 & Optimal &  0.41 & 1953 & 1953.00 &  0.00\\
t40m20r10-1.pl.json & 40 & 20 & Solution & 30.09 & 4518 & 3972.00 & 12.08\\
t40m20r10-10.pl.json & 40 & 20 & Optimal & 12.43 & 3862 & 3862.00 &  0.00\\
t40m20r10-11.pl.json & 40 & 20 & Optimal &  0.14 & 1952 & 1952.00 &  0.00\\
t40m20r10-12.pl.json & 40 & 20 & Optimal &  0.14 & 4129 & 4129.00 &  0.00\\
t40m20r10-13.pl.json & 40 & 20 & Optimal &  0.28 & 2927 & 2927.00 &  0.00\\
t40m20r10-14.pl.json & 40 & 20 & Solution & 30.05 & 2701 & 2381.00 & 11.85\\
t40m20r10-15.pl.json & 40 & 20 & Optimal & 11.77 & 3168 & 3168.00 &  0.00\\
t40m20r10-16.pl.json & 40 & 20 & Optimal &  0.14 & 2812 & 2812.00 &  0.00\\
t40m20r10-17.pl.json & 40 & 20 & Solution & 30.07 & 4288 & 3718.00 & 13.29\\
t40m20r10-18.pl.json & 40 & 20 & Solution & 30.05 & 3611 & 3194.00 & 11.55\\
t40m20r10-19.pl.json & 40 & 20 & Optimal & 12.23 & 2891 & 2891.00 &  0.00\\
t40m20r10-2.pl.json & 40 & 20 & Optimal &  8.74 & 3284 & 3284.00 &  0.00\\
t40m20r10-20.pl.json & 40 & 20 & Solution & 30.04 & 5506 & 4945.00 & 10.19\\
t40m20r10-3.pl.json & 40 & 20 & Solution & 30.08 & 5981 & 5478.00 &  8.41\\
t40m20r10-4.pl.json & 40 & 20 & Optimal &  0.14 & 3409 & 3409.00 &  0.00\\
t40m20r10-5.pl.json & 40 & 20 & Solution & 30.06 & 5113 & 4278.00 & 16.33\\
t40m20r10-6.pl.json & 40 & 20 & Solution & 30.03 & 2376 & 2333.00 &  1.81\\
t40m20r10-7.pl.json & 40 & 20 & Solution & 30.06 & 4799 & 4243.00 & 11.59\\
t40m20r10-8.pl.json & 40 & 20 & Solution & 30.02 & 3924 & 3327.00 & 15.21\\
t40m20r10-9.pl.json & 40 & 20 & Optimal &  3.86 & 2043 & 2043.00 &  0.00\\
t40m20r3-1.pl.json & 40 & 20 & Optimal &  0.16 & 3524 & 3524.00 &  0.00\\
t40m20r3-10.pl.json & 40 & 20 & Optimal &  0.19 & 3110 & 3110.00 &  0.00\\
t40m20r3-11.pl.json & 40 & 20 & Optimal &  0.15 & 3695 & 3695.00 &  0.00\\
t40m20r3-12.pl.json & 40 & 20 & Optimal &  0.24 & 4828 & 4828.00 &  0.00\\
t40m20r3-13.pl.json & 40 & 20 & Optimal &  0.25 & 4010 & 4010.00 &  0.00\\
t40m20r3-14.pl.json & 40 & 20 & Optimal &  0.14 & 2752 & 2752.00 &  0.00\\
t40m20r3-15.pl.json & 40 & 20 & Optimal &  0.16 & 3312 & 3312.00 &  0.00\\
t40m20r3-16.pl.json & 40 & 20 & Optimal &  0.16 & 4304 & 4304.00 &  0.00\\
t40m20r3-17.pl.json & 40 & 20 & Optimal &  0.17 & 3991 & 3991.00 &  0.00\\
t40m20r3-18.pl.json & 40 & 20 & Optimal &  0.17 & 5733 & 5733.00 &  0.00\\
t40m20r3-19.pl.json & 40 & 20 & Optimal &  0.17 & 3581 & 3581.00 &  0.00\\
t40m20r3-2.pl.json & 40 & 20 & Optimal &  0.17 & 4869 & 4869.00 &  0.00\\
t40m20r3-20.pl.json & 40 & 20 & Optimal &  0.17 & 3514 & 3514.00 &  0.00\\
t40m20r3-3.pl.json & 40 & 20 & Optimal &  0.24 & 2503 & 2503.00 &  0.00\\
t40m20r3-4.pl.json & 40 & 20 & Optimal &  0.13 & 4323 & 4323.00 &  0.00\\
t40m20r3-5.pl.json & 40 & 20 & Optimal &  0.17 & 3626 & 3626.00 &  0.00\\
t40m20r3-6.pl.json & 40 & 20 & Optimal &  0.17 & 2488 & 2488.00 &  0.00\\
t40m20r3-7.pl.json & 40 & 20 & Optimal &  0.17 & 3470 & 3470.00 &  0.00\\
t40m20r3-8.pl.json & 40 & 20 & Optimal &  0.24 & 6730 & 6730.00 &  0.00\\
t40m20r3-9.pl.json & 40 & 20 & Optimal &  0.20 & 4656 & 4656.00 &  0.00\\
t40m20r5-1.pl.json & 40 & 20 & Optimal &  0.28 & 1318 & 1318.00 &  0.00\\
t40m20r5-10.pl.json & 40 & 20 & Optimal &  0.25 & 2216 & 2216.00 &  0.00\\
t40m20r5-11.pl.json & 40 & 20 & Optimal &  0.25 & 3538 & 3538.00 &  0.00\\
t40m20r5-12.pl.json & 40 & 20 & Optimal &  0.23 & 5346 & 5346.00 &  0.00\\
t40m20r5-13.pl.json & 40 & 20 & Solution & 30.03 & 4589 & 4393.00 &  4.27\\
t40m20r5-14.pl.json & 40 & 20 & Optimal &  0.17 & 2243 & 2243.00 &  0.00\\
t40m20r5-15.pl.json & 40 & 20 & Solution & 30.08 & 3869 & 3590.00 &  7.21\\
t40m20r5-16.pl.json & 40 & 20 & Optimal &  0.17 & 4319 & 4319.00 &  0.00\\
t40m20r5-17.pl.json & 40 & 20 & Optimal &  0.18 & 4866 & 4866.00 &  0.00\\
t40m20r5-18.pl.json & 40 & 20 & Optimal &  0.39 & 5802 & 5802.00 &  0.00\\
t40m20r5-19.pl.json & 40 & 20 & Solution & 30.06 & 4197 & 4072.00 &  2.98\\
t40m20r5-2.pl.json & 40 & 20 & Optimal &  0.16 & 2634 & 2634.00 &  0.00\\
t40m20r5-20.pl.json & 40 & 20 & Solution & 30.03 & 6482 & 6232.00 &  3.86\\
t40m20r5-3.pl.json & 40 & 20 & Optimal &  0.19 & 4391 & 4391.00 &  0.00\\
t40m20r5-4.pl.json & 40 & 20 & Optimal &  9.64 & 4610 & 4610.00 &  0.00\\
t40m20r5-5.pl.json & 40 & 20 & Optimal &  0.17 & 3105 & 3105.00 &  0.00\\
t40m20r5-6.pl.json & 40 & 20 & Optimal &  0.16 & 4760 & 4760.00 &  0.00\\
t40m20r5-7.pl.json & 40 & 20 & Optimal &  0.31 & 1218 & 1218.00 &  0.00\\
t40m20r5-8.pl.json & 40 & 20 & Solution & 30.05 & 2601 & 2190.00 & 15.80\\
t40m20r5-9.pl.json & 40 & 20 & Optimal &  0.19 & 3141 & 3141.00 &  0.00\\
t500m100r10-10.pl.json & 500 & 100 & Solution & 30.54 & 43793 & 795.00 & 98.18\\
t500m100r10-11.pl.json & 500 & 100 & Solution & 30.92 & 36367 & 801.00 & 97.80\\
t500m100r10-13.pl.json & 500 & 100 & Solution & 30.63 & 45030 & 801.00 & 98.22\\
t500m100r10-14.pl.json & 500 & 100 & Solution & 30.54 & 40089 & 800.00 & 98.00\\
t500m100r10-15.pl.json & 500 & 100 & Solution & 30.45 & 41425 & 801.00 & 98.07\\
t500m100r10-16.pl.json & 500 & 100 & Solution & 30.65 & 40463 & 801.00 & 98.02\\
t500m100r10-17.pl.json & 500 & 100 & Solution & 30.43 & 33209 & 798.00 & 97.60\\
t500m100r10-18.pl.json & 500 & 100 & Solution & 30.44 & 41028 & 801.00 & 98.05\\
t500m100r10-19.pl.json & 500 & 100 & Solution & 30.94 & 49137 & 801.00 & 98.37\\
t500m100r10-2.pl.json & 500 & 100 & Solution & 30.54 & 42142 & 796.00 & 98.11\\
t500m100r10-20.pl.json & 500 & 100 & Solution & 30.35 & 38167 & 801.00 & 97.90\\
t500m100r10-3.pl.json & 500 & 100 & Solution & 30.39 & 37653 & 801.00 & 97.87\\
t500m100r10-4.pl.json & 500 & 100 & Solution & 30.67 & 39921 & 798.00 & 98.00\\
t500m100r10-5.pl.json & 500 & 100 & Solution & 30.47 & 35252 & 800.00 & 97.73\\
t500m100r10-6.pl.json & 500 & 100 & Solution & 30.65 & 41172 & 801.00 & 98.05\\
t500m100r10-7.pl.json & 500 & 100 & Solution & 30.97 & 41044 & 800.00 & 98.05\\
t500m100r10-8.pl.json & 500 & 100 & Solution & 30.52 & 46351 & 800.00 & 98.27\\
t500m100r10-9.pl.json & 500 & 100 & Solution & 30.51 & 40539 & 800.00 & 98.03\\
t500m100r3-1.pl.json & 500 & 100 & Solution & 30.55 & 39303 & 801.00 & 97.96\\
t500m100r3-10.pl.json & 500 & 100 & Solution & 30.65 & 42052 & 801.00 & 98.10\\
t500m100r3-11.pl.json & 500 & 100 & Solution & 30.62 & 38084 & 794.00 & 97.92\\
t500m100r3-12.pl.json & 500 & 100 & Solution & 30.70 & 38483 & 800.00 & 97.92\\
t500m100r3-13.pl.json & 500 & 100 & Solution & 30.57 & 35447 & 801.00 & 97.74\\
t500m100r3-14.pl.json & 500 & 100 & Solution & 30.42 & 40571 & 798.00 & 98.03\\
t500m100r3-15.pl.json & 500 & 100 & Solution & 30.45 & 38987 & 801.00 & 97.95\\
t500m100r3-16.pl.json & 500 & 100 & Solution & 30.59 & 41984 & 798.00 & 98.10\\
t500m100r3-18.pl.json & 500 & 100 & Solution & 30.89 & 39919 & 801.00 & 97.99\\
t500m100r3-19.pl.json & 500 & 100 & Optimal & 10.63 & 41896 & 41892.00 &  0.01\\
t500m100r3-2.pl.json & 500 & 100 & Optimal & 10.86 & 41211 & 41207.00 &  0.01\\
t500m100r3-20.pl.json & 500 & 100 & Solution & 30.78 & 38551 & 800.00 & 97.92\\
t500m100r3-3.pl.json & 500 & 100 & Solution & 30.79 & 35516 & 798.00 & 97.75\\
t500m100r3-4.pl.json & 500 & 100 & Solution & 30.36 & 32084 & 798.00 & 97.51\\
t500m100r3-5.pl.json & 500 & 100 & Solution & 30.66 & 38761 & 801.00 & 97.93\\
t500m100r3-6.pl.json & 500 & 100 & Solution & 30.52 & 46048 & 800.00 & 98.26\\
t500m100r3-7.pl.json & 500 & 100 & Solution & 30.45 & 37680 & 800.00 & 97.88\\
t500m100r3-8.pl.json & 500 & 100 & Solution & 30.69 & 40838 & 799.00 & 98.04\\
t500m100r3-9.pl.json & 500 & 100 & Solution & 30.85 & 44803 & 801.00 & 98.21\\
t500m100r5-1.pl.json & 500 & 100 & Solution & 30.49 & 36936 & 797.00 & 97.84\\
t500m100r5-10.pl.json & 500 & 100 & Solution & 31.15 & 30332 & 800.00 & 97.36\\
t500m100r5-11.pl.json & 500 & 100 & Solution & 30.80 & 37660 & 801.00 & 97.87\\
t500m100r5-12.pl.json & 500 & 100 & Solution & 30.42 & 39090 & 799.00 & 97.96\\
t500m100r5-13.pl.json & 500 & 100 & Solution & 30.39 & 44171 & 801.00 & 98.19\\
t500m100r5-14.pl.json & 500 & 100 & Solution & 30.45 & 39568 & 800.00 & 97.98\\
t500m100r5-15.pl.json & 500 & 100 & Solution & 30.57 & 38257 & 800.00 & 97.91\\
t500m100r5-16.pl.json & 500 & 100 & Solution & 30.61 & 35151 & 798.00 & 97.73\\
t500m100r5-17.pl.json & 500 & 100 & Solution & 30.72 & 39749 & 797.00 & 97.99\\
t500m100r5-18.pl.json & 500 & 100 & Solution & 30.54 & 45868 & 801.00 & 98.25\\
t500m100r5-19.pl.json & 500 & 100 & Solution & 30.40 & 46018 & 801.00 & 98.26\\
t500m100r5-2.pl.json & 500 & 100 & Solution & 30.58 & 43708 & 800.00 & 98.17\\
t500m100r5-20.pl.json & 500 & 100 & Solution & 30.82 & 39466 & 800.00 & 97.97\\
t500m100r5-3.pl.json & 500 & 100 & Solution & 30.77 & 42468 & 801.00 & 98.11\\
t500m100r5-4.pl.json & 500 & 100 & Solution & 30.57 & 33936 & 801.00 & 97.64\\
t500m100r5-5.pl.json & 500 & 100 & Solution & 30.69 & 38103 & 795.00 & 97.91\\
t500m100r5-6.pl.json & 500 & 100 & Solution & 30.62 & 45271 & 801.00 & 98.23\\
t500m100r5-7.pl.json & 500 & 100 & Solution & 30.68 & 43542 & 800.00 & 98.16\\
t500m100r5-8.pl.json & 500 & 100 & Solution & 30.74 & 38116 & 796.00 & 97.91\\
t500m100r5-9.pl.json & 500 & 100 & Solution & 30.43 & 39282 & 801.00 & 97.96\\
t500m10r10-1.pl.json & 500 & 10 & Solution & 30.06 & 48213 & 42756.00 & 11.32\\
t500m10r10-10.pl.json & 500 & 10 & Solution & 30.08 & 35490 & 30745.00 & 13.37\\
t500m10r10-11.pl.json & 500 & 10 & Solution & 30.10 & 47651 & 42832.00 & 10.11\\
t500m10r10-12.pl.json & 500 & 10 & Solution & 30.09 & 43253 & 35908.00 & 16.98\\
\end{longtable}



\section{Results for CPSat}

\begin{longtable}{lrrlrrrr}
\caption{Results for Test Scheduling Problems (CPSat) (840 Instances)}\\\toprule
Name & \shortstack{Nr\\Jobs} & \shortstack{Nr\\Machines} & Status & Time & Makespan & Bound & \shortstack{Gap\\Percent}\\ \midrule
\endhead
\bottomrule
\endfoot
t100m10r10-1.pl.json & 100 & 10 & Solution & 30.04 & 10491 & 9055.00 & 13.69\\
t100m10r10-10.pl.json & 100 & 10 & Solution & 30.04 & 9599 & 8369.00 & 12.81\\
t100m10r10-11.pl.json & 100 & 10 & Solution & 30.04 & 5336 & 5100.00 &  4.42\\
t100m10r10-12.pl.json & 100 & 10 & Solution & 30.04 & 6564 & 5613.00 & 14.49\\
t100m10r10-13.pl.json & 100 & 10 & Solution & 30.05 & 6831 & 6786.00 &  0.66\\
t100m10r10-14.pl.json & 100 & 10 & Solution & 30.03 & 5775 & 5257.00 &  8.97\\
t100m10r10-15.pl.json & 100 & 10 & Solution & 30.02 & 6105 & 5012.00 & 17.90\\
t100m10r10-16.pl.json & 100 & 10 & Solution & 30.04 & 12563 & 11589.00 &  7.75\\
t100m10r10-17.pl.json & 100 & 10 & Solution & 30.05 & 8954 & 8114.00 &  9.38\\
t100m10r10-18.pl.json & 100 & 10 & Solution & 30.05 & 10180 & 9304.00 &  8.61\\
t100m10r10-19.pl.json & 100 & 10 & Solution & 30.03 & 9812 & 8514.00 & 13.23\\
t100m10r10-2.pl.json & 100 & 10 & Solution & 30.02 & 11593 & 9807.00 & 15.41\\
t100m10r10-20.pl.json & 100 & 10 & Solution & 30.04 & 12342 & 10686.00 & 13.42\\
t100m10r10-3.pl.json & 100 & 10 & Solution & 30.03 & 6884 & 6379.00 &  7.34\\
t100m10r10-4.pl.json & 100 & 10 & Solution & 30.03 & 11041 & 9111.00 & 17.48\\
t100m10r10-5.pl.json & 100 & 10 & Solution & 30.04 & 12241 & 11823.00 &  3.41\\
t100m10r10-6.pl.json & 100 & 10 & Solution & 30.04 & 11906 & 10914.00 &  8.33\\
t100m10r10-7.pl.json & 100 & 10 & Solution & 30.05 & 6435 & 5732.00 & 10.92\\
t100m10r10-8.pl.json & 100 & 10 & Solution & 30.03 & 11070 & 10010.00 &  9.58\\
t100m10r10-9.pl.json & 100 & 10 & Solution & 30.03 & 9878 & 7991.00 & 19.10\\
t100m10r3-1.pl.json & 100 & 10 & Optimal & 12.19 & 8711 & 8711.00 &  0.00\\
t100m10r3-10.pl.json & 100 & 10 & Optimal & 15.43 & 8958 & 8958.00 &  0.00\\
t100m10r3-11.pl.json & 100 & 10 & Optimal &  3.27 & 9560 & 9560.00 &  0.00\\
t100m10r3-12.pl.json & 100 & 10 & Optimal &  2.92 & 7892 & 7892.00 &  0.00\\
t100m10r3-13.pl.json & 100 & 10 & Optimal & 19.27 & 10078 & 10078.00 &  0.00\\
t100m10r3-14.pl.json & 100 & 10 & Optimal & 18.02 & 8681 & 8681.00 &  0.00\\
t100m10r3-15.pl.json & 100 & 10 & Optimal &  2.13 & 8810 & 8810.00 &  0.00\\
t100m10r3-16.pl.json & 100 & 10 & Optimal & 10.48 & 11182 & 11182.00 &  0.00\\
t100m10r3-17.pl.json & 100 & 10 & Optimal & 10.25 & 7534 & 7534.00 &  0.00\\
t100m10r3-18.pl.json & 100 & 10 & Solution & 30.05 & 10376 & 9934.00 &  4.26\\
t100m10r3-19.pl.json & 100 & 10 & Solution & 30.04 & 7706 & 6970.00 &  9.55\\
t100m10r3-2.pl.json & 100 & 10 & Optimal &  1.65 & 7082 & 7082.00 &  0.00\\
t100m10r3-20.pl.json & 100 & 10 & Optimal &  0.47 & 9025 & 9025.00 &  0.00\\
t100m10r3-3.pl.json & 100 & 10 & Optimal &  3.03 & 10054 & 10054.00 &  0.00\\
t100m10r3-4.pl.json & 100 & 10 & Optimal &  1.74 & 13122 & 13122.00 &  0.00\\
t100m10r3-5.pl.json & 100 & 10 & Optimal & 13.52 & 7545 & 7545.00 &  0.00\\
t100m10r3-6.pl.json & 100 & 10 & Optimal & 14.51 & 7840 & 7840.00 &  0.00\\
t100m10r3-7.pl.json & 100 & 10 & Optimal &  3.63 & 11010 & 11010.00 &  0.00\\
t100m10r3-8.pl.json & 100 & 10 & Optimal &  5.99 & 9112 & 9112.00 &  0.00\\
t100m10r3-9.pl.json & 100 & 10 & Optimal & 11.16 & 8532 & 8532.00 &  0.00\\
t100m10r5-1.pl.json & 100 & 10 & Solution & 30.04 & 7304 & 7300.00 &  0.05\\
t100m10r5-10.pl.json & 100 & 10 & Optimal & 12.84 & 6972 & 6972.00 &  0.00\\
t100m10r5-11.pl.json & 100 & 10 & Solution & 30.04 & 9098 & 8568.00 &  5.83\\
t100m10r5-12.pl.json & 100 & 10 & Optimal &  5.79 & 6538 & 6538.00 &  0.00\\
t100m10r5-13.pl.json & 100 & 10 & Optimal & 18.18 & 8972 & 8972.00 &  0.00\\
t100m10r5-14.pl.json & 100 & 10 & Solution & 30.03 & 10539 & 10347.00 &  1.82\\
t100m10r5-15.pl.json & 100 & 10 & Solution & 30.03 & 5762 & 5647.00 &  2.00\\
t100m10r5-16.pl.json & 100 & 10 & Solution & 30.04 & 7019 & 6207.00 & 11.57\\
t100m10r5-17.pl.json & 100 & 10 & Optimal &  4.08 & 6728 & 6728.00 &  0.00\\
t100m10r5-18.pl.json & 100 & 10 & Solution & 30.04 & 9019 & 8811.00 &  2.31\\
t100m10r5-19.pl.json & 100 & 10 & Optimal & 13.99 & 8885 & 8885.00 &  0.00\\
t100m10r5-2.pl.json & 100 & 10 & Optimal &  8.51 & 9010 & 9010.00 &  0.00\\
t100m10r5-20.pl.json & 100 & 10 & Optimal & 12.69 & 7022 & 7022.00 &  0.00\\
t100m10r5-3.pl.json & 100 & 10 & Solution & 30.04 & 8857 & 8820.00 &  0.42\\
t100m10r5-4.pl.json & 100 & 10 & Optimal & 22.41 & 10753 & 10753.00 &  0.00\\
t100m10r5-5.pl.json & 100 & 10 & Optimal & 11.65 & 6608 & 6608.00 &  0.00\\
t100m10r5-6.pl.json & 100 & 10 & Solution & 30.04 & 9452 & 8456.00 & 10.54\\
t100m10r5-7.pl.json & 100 & 10 & Solution & 30.04 & 8186 & 7664.00 &  6.38\\
t100m10r5-8.pl.json & 100 & 10 & Solution & 30.03 & 11383 & 10079.00 & 11.46\\
t100m10r5-9.pl.json & 100 & 10 & Solution & 30.03 & 11649 & 10683.00 &  8.29\\
t100m20r10-1.pl.json & 100 & 20 & Solution & 30.06 & 12643 & 12180.00 &  3.66\\
t100m20r10-10.pl.json & 100 & 20 & Solution & 30.05 & 12653 & 10953.00 & 13.44\\
t100m20r10-11.pl.json & 100 & 20 & Solution & 30.06 & 8724 & 7289.00 & 16.45\\
t100m20r10-12.pl.json & 100 & 20 & Solution & 30.05 & 7404 & 6774.00 &  8.51\\
t100m20r10-13.pl.json & 100 & 20 & Solution & 30.05 & 9695 & 9229.00 &  4.81\\
t100m20r10-14.pl.json & 100 & 20 & Solution & 30.05 & 10027 & 8652.00 & 13.71\\
t100m20r10-15.pl.json & 100 & 20 & Solution & 30.05 & 6544 & 5362.00 & 18.06\\
t100m20r10-16.pl.json & 100 & 20 & Solution & 30.05 & 9264 & 8343.00 &  9.94\\
t100m20r10-17.pl.json & 100 & 20 & Solution & 30.07 & 8691 & 7381.00 & 15.07\\
t100m20r10-18.pl.json & 100 & 20 & Optimal & 15.69 & 4843 & 4843.00 &  0.00\\
t100m20r10-19.pl.json & 100 & 20 & Solution & 30.05 & 12320 & 11752.00 &  4.61\\
t100m20r10-2.pl.json & 100 & 20 & Solution & 30.05 & 7760 & 6890.00 & 11.21\\
t100m20r10-20.pl.json & 100 & 20 & Solution & 30.03 & 10030 & 8562.00 & 14.64\\
t100m20r10-3.pl.json & 100 & 20 & Solution & 30.06 & 7133 & 6295.00 & 11.75\\
t100m20r10-4.pl.json & 100 & 20 & Solution & 30.06 & 9671 & 9052.00 &  6.40\\
t100m20r10-5.pl.json & 100 & 20 & Solution & 30.04 & 9230 & 8459.00 &  8.35\\
t100m20r10-6.pl.json & 100 & 20 & Solution & 30.06 & 8781 & 7619.00 & 13.23\\
t100m20r10-7.pl.json & 100 & 20 & Solution & 30.05 & 11318 & 9767.00 & 13.70\\
t100m20r10-8.pl.json & 100 & 20 & Solution & 30.06 & 7852 & 7041.00 & 10.33\\
t100m20r10-9.pl.json & 100 & 20 & Solution & 30.05 & 10856 & 10019.00 &  7.71\\
t100m20r3-1.pl.json & 100 & 20 & Optimal &  9.73 & 6585 & 6585.00 &  0.00\\
t100m20r3-10.pl.json & 100 & 20 & Optimal &  4.77 & 8535 & 8535.00 &  0.00\\
t100m20r3-11.pl.json & 100 & 20 & Optimal & 13.99 & 9084 & 9084.00 &  0.00\\
t100m20r3-12.pl.json & 100 & 20 & Optimal &  2.36 & 9066 & 9066.00 &  0.00\\
t100m20r3-13.pl.json & 100 & 20 & Solution & 30.07 & 11429 & 9974.00 & 12.73\\
t100m20r3-14.pl.json & 100 & 20 & Optimal &  8.11 & 8786 & 8786.00 &  0.00\\
t100m20r3-15.pl.json & 100 & 20 & Optimal & 12.26 & 10205 & 10205.00 &  0.00\\
t100m20r3-16.pl.json & 100 & 20 & Optimal & 10.67 & 8856 & 8856.00 &  0.00\\
t100m20r3-17.pl.json & 100 & 20 & Optimal & 10.75 & 5451 & 5451.00 &  0.00\\
t100m20r3-18.pl.json & 100 & 20 & Optimal & 11.28 & 8752 & 8752.00 &  0.00\\
t100m20r3-19.pl.json & 100 & 20 & Solution & 30.04 & 8909 & 8860.00 &  0.55\\
t100m20r3-2.pl.json & 100 & 20 & Optimal & 13.78 & 8498 & 8498.00 &  0.00\\
t100m20r3-20.pl.json & 100 & 20 & Optimal &  3.73 & 7880 & 7880.00 &  0.00\\
t100m20r3-3.pl.json & 100 & 20 & Solution & 30.04 & 12192 & 11987.00 &  1.68\\
t100m20r3-4.pl.json & 100 & 20 & Optimal & 18.10 & 12258 & 12258.00 &  0.00\\
t100m20r3-5.pl.json & 100 & 20 & Optimal &  7.98 & 11932 & 11932.00 &  0.00\\
t100m20r3-6.pl.json & 100 & 20 & Optimal & 10.86 & 8531 & 8531.00 &  0.00\\
t100m20r3-7.pl.json & 100 & 20 & Optimal &  7.63 & 6512 & 6512.00 &  0.00\\
t100m20r3-8.pl.json & 100 & 20 & Optimal & 18.30 & 10690 & 10690.00 &  0.00\\
t100m20r3-9.pl.json & 100 & 20 & Optimal &  2.31 & 8255 & 8255.00 &  0.00\\
t100m20r5-1.pl.json & 100 & 20 & Optimal & 12.04 & 9098 & 9098.00 &  0.00\\
t100m20r5-10.pl.json & 100 & 20 & Solution & 30.05 & 8340 & 7964.00 &  4.51\\
t100m20r5-11.pl.json & 100 & 20 & Solution & 30.05 & 6828 & 5564.00 & 18.51\\
t100m20r5-12.pl.json & 100 & 20 & Solution & 30.04 & 8722 & 8704.00 &  0.21\\
t100m20r5-13.pl.json & 100 & 20 & Optimal & 16.31 & 8880 & 8880.00 &  0.00\\
t100m20r5-14.pl.json & 100 & 20 & Solution & 30.06 & 10621 & 9727.00 &  8.42\\
t100m20r5-15.pl.json & 100 & 20 & Optimal & 18.85 & 8953 & 8953.00 &  0.00\\
t100m20r5-16.pl.json & 100 & 20 & Solution & 30.05 & 8020 & 7594.00 &  5.31\\
t100m20r5-17.pl.json & 100 & 20 & Solution & 30.05 & 5685 & 5524.00 &  2.83\\
t100m20r5-18.pl.json & 100 & 20 & Solution & 30.03 & 6637 & 6617.00 &  0.30\\
t100m20r5-19.pl.json & 100 & 20 & Optimal & 22.81 & 9461 & 9461.00 &  0.00\\
t100m20r5-2.pl.json & 100 & 20 & Optimal & 13.63 & 9566 & 9566.00 &  0.00\\
t100m20r5-20.pl.json & 100 & 20 & Solution & 30.03 & 11569 & 10228.00 & 11.59\\
t100m20r5-3.pl.json & 100 & 20 & Solution & 30.06 & 9470 & 9366.00 &  1.10\\
t100m20r5-4.pl.json & 100 & 20 & Solution & 30.04 & 14465 & 12456.00 & 13.89\\
t100m20r5-5.pl.json & 100 & 20 & Optimal & 12.10 & 8585 & 8585.00 &  0.00\\
t100m20r5-6.pl.json & 100 & 20 & Solution & 30.05 & 7528 & 6539.00 & 13.14\\
t100m20r5-7.pl.json & 100 & 20 & Solution & 30.05 & 11413 & 10099.00 & 11.51\\
t100m20r5-8.pl.json & 100 & 20 & Optimal & 17.27 & 5812 & 5812.00 &  0.00\\
t100m20r5-9.pl.json & 100 & 20 & Solution & 30.06 & 6657 & 6496.00 &  2.42\\
t100m50r10-1.pl.json & 100 & 50 & Solution & 30.10 & 7299 & 6941.00 &  4.90\\
t100m50r10-10.pl.json & 100 & 50 & Solution & 30.08 & 5201 & 5108.00 &  1.79\\
t100m50r10-11.pl.json & 100 & 50 & Solution & 30.11 & 4970 & 4782.00 &  3.78\\
t100m50r10-12.pl.json & 100 & 50 & Solution & 30.12 & 9335 & 9122.00 &  2.28\\
t100m50r10-13.pl.json & 100 & 50 & Solution & 30.11 & 9759 & 8828.00 &  9.54\\
t100m50r10-14.pl.json & 100 & 50 & Solution & 30.08 & 10724 & 8290.00 & 22.70\\
t100m50r10-15.pl.json & 100 & 50 & Solution & 30.08 & 8640 & 7804.00 &  9.68\\
t100m50r10-16.pl.json & 100 & 50 & Solution & 30.12 & 14211 & 12381.00 & 12.88\\
t100m50r10-17.pl.json & 100 & 50 & Solution & 30.10 & 9826 & 9151.00 &  6.87\\
t100m50r10-18.pl.json & 100 & 50 & Solution & 30.09 & 7384 & 7120.00 &  3.58\\
t100m50r10-19.pl.json & 100 & 50 & Solution & 30.07 & 8559 & 8059.00 &  5.84\\
t100m50r10-2.pl.json & 100 & 50 & Solution & 30.12 & 7968 & 7568.00 &  5.02\\
t100m50r10-20.pl.json & 100 & 50 & Solution & 30.11 & 8421 & 7939.00 &  5.72\\
t100m50r10-3.pl.json & 100 & 50 & Optimal &  2.98 & 6937 & 6937.00 &  0.00\\
t100m50r10-4.pl.json & 100 & 50 & Solution & 30.10 & 10208 & 8525.00 & 16.49\\
t100m50r10-5.pl.json & 100 & 50 & Optimal & 18.08 & 9859 & 9859.00 &  0.00\\
t100m50r10-6.pl.json & 100 & 50 & Solution & 30.09 & 7715 & 6837.00 & 11.38\\
t100m50r10-7.pl.json & 100 & 50 & Solution & 30.10 & 9691 & 9542.00 &  1.54\\
t100m50r10-8.pl.json & 100 & 50 & Solution & 30.13 & 10719 & 9176.00 & 14.39\\
t100m50r10-9.pl.json & 100 & 50 & Solution & 30.08 & 10453 & 9375.00 & 10.31\\
t100m50r3-1.pl.json & 100 & 50 & Optimal & 10.47 & 9937 & 9937.00 &  0.00\\
t100m50r3-10.pl.json & 100 & 50 & Solution & 30.12 & 8957 & 8877.00 &  0.89\\
t100m50r3-11.pl.json & 100 & 50 & Optimal & 16.48 & 6141 & 6141.00 &  0.00\\
t100m50r3-12.pl.json & 100 & 50 & Optimal &  3.37 & 6473 & 6473.00 &  0.00\\
t100m50r3-13.pl.json & 100 & 50 & Optimal &  7.08 & 8653 & 8653.00 &  0.00\\
t100m50r3-14.pl.json & 100 & 50 & Solution & 30.07 & 13039 & 12796.00 &  1.86\\
t100m50r3-15.pl.json & 100 & 50 & Solution & 30.13 & 9271 & 9056.00 &  2.32\\
t100m50r3-16.pl.json & 100 & 50 & Optimal & 15.74 & 8680 & 8680.00 &  0.00\\
t100m50r3-17.pl.json & 100 & 50 & Optimal &  5.79 & 8197 & 8197.00 &  0.00\\
t100m50r3-18.pl.json & 100 & 50 & Optimal &  6.21 & 9318 & 9318.00 &  0.00\\
t100m50r3-19.pl.json & 100 & 50 & Optimal &  4.24 & 12265 & 12265.00 &  0.00\\
t100m50r3-2.pl.json & 100 & 50 & Optimal & 25.96 & 11030 & 11030.00 &  0.00\\
t100m50r3-20.pl.json & 100 & 50 & Optimal &  2.53 & 7662 & 7662.00 &  0.00\\
t100m50r3-3.pl.json & 100 & 50 & Optimal &  2.34 & 5348 & 5348.00 &  0.00\\
t100m50r3-4.pl.json & 100 & 50 & Optimal & 14.63 & 7800 & 7800.00 &  0.00\\
t100m50r3-5.pl.json & 100 & 50 & Optimal & 13.56 & 4207 & 4207.00 &  0.00\\
t100m50r3-6.pl.json & 100 & 50 & Solution & 30.08 & 10674 & 10596.00 &  0.73\\
t100m50r3-7.pl.json & 100 & 50 & Optimal &  3.88 & 7826 & 7826.00 &  0.00\\
t100m50r3-8.pl.json & 100 & 50 & Optimal & 14.67 & 7865 & 7865.00 &  0.00\\
t100m50r3-9.pl.json & 100 & 50 & Optimal &  3.79 & 7891 & 7891.00 &  0.00\\
t100m50r5-1.pl.json & 100 & 50 & Solution & 30.07 & 8016 & 7926.00 &  1.12\\
t100m50r5-10.pl.json & 100 & 50 & Solution & 30.08 & 7299 & 6521.00 & 10.66\\
t100m50r5-11.pl.json & 100 & 50 & Optimal & 18.72 & 9417 & 9417.00 &  0.00\\
t100m50r5-12.pl.json & 100 & 50 & Optimal &  4.77 & 8824 & 8824.00 &  0.00\\
t100m50r5-13.pl.json & 100 & 50 & Solution & 30.12 & 10473 & 9115.00 & 12.97\\
t100m50r5-14.pl.json & 100 & 50 & Solution & 30.08 & 7503 & 7134.00 &  4.92\\
t100m50r5-15.pl.json & 100 & 50 & Solution & 30.10 & 10141 & 9853.00 &  2.84\\
t100m50r5-16.pl.json & 100 & 50 & Optimal &  9.40 & 6481 & 6481.00 &  0.00\\
t100m50r5-17.pl.json & 100 & 50 & Optimal &  5.97 & 6129 & 6129.00 &  0.00\\
t100m50r5-18.pl.json & 100 & 50 & Solution & 30.08 & 9100 & 8337.00 &  8.38\\
t100m50r5-19.pl.json & 100 & 50 & Solution & 30.09 & 6762 & 6356.00 &  6.00\\
t100m50r5-2.pl.json & 100 & 50 & Optimal &  4.94 & 6651 & 6651.00 &  0.00\\
t100m50r5-20.pl.json & 100 & 50 & Solution & 30.08 & 6894 & 6667.00 &  3.29\\
t100m50r5-3.pl.json & 100 & 50 & Solution & 30.11 & 7944 & 7857.00 &  1.10\\
t100m50r5-4.pl.json & 100 & 50 & Optimal & 18.31 & 8296 & 8296.00 &  0.00\\
t100m50r5-5.pl.json & 100 & 50 & Optimal &  9.79 & 9977 & 9977.00 &  0.00\\
t100m50r5-6.pl.json & 100 & 50 & Optimal &  5.27 & 8240 & 8240.00 &  0.00\\
t100m50r5-7.pl.json & 100 & 50 & Solution & 30.11 & 10917 & 10904.00 &  0.12\\
t100m50r5-8.pl.json & 100 & 50 & Optimal & 17.90 & 8293 & 8293.00 &  0.00\\
t100m50r5-9.pl.json & 100 & 50 & Solution & 30.12 & 7879 & 7622.00 &  3.26\\
t20m10r10-1.pl.json & 20 & 10 & Optimal &  0.06 & 1337 & 1337.00 &  0.00\\
t20m10r10-10.pl.json & 20 & 10 & Optimal &  0.05 & 3882 & 3882.00 &  0.00\\
t20m10r10-11.pl.json & 20 & 10 & Optimal &  0.08 & 2002 & 2002.00 &  0.00\\
t20m10r10-12.pl.json & 20 & 10 & Optimal &  0.05 & 1257 & 1257.00 &  0.00\\
t20m10r10-13.pl.json & 20 & 10 & Optimal &  0.08 & 2110 & 2110.00 &  0.00\\
t20m10r10-14.pl.json & 20 & 10 & Optimal &  0.04 & 2546 & 2546.00 &  0.00\\
t20m10r10-15.pl.json & 20 & 10 & Optimal &  0.05 & 3344 & 3344.00 &  0.00\\
t20m10r10-16.pl.json & 20 & 10 & Optimal &  0.68 & 1643 & 1643.00 &  0.00\\
t20m10r10-17.pl.json & 20 & 10 & Optimal &  0.06 & 1069 & 1069.00 &  0.00\\
t20m10r10-18.pl.json & 20 & 10 & Optimal &  0.06 & 3041 & 3041.00 &  0.00\\
t20m10r10-19.pl.json & 20 & 10 & Optimal &  0.05 & 2422 & 2422.00 &  0.00\\
t20m10r10-2.pl.json & 20 & 10 & Optimal &  0.07 & 1819 & 1819.00 &  0.00\\
t20m10r10-20.pl.json & 20 & 10 & Optimal &  0.05 & 1595 & 1595.00 &  0.00\\
t20m10r10-3.pl.json & 20 & 10 & Solution & 30.02 & 843 & 771.00 &  8.54\\
t20m10r10-4.pl.json & 20 & 10 & Optimal &  0.07 & 1396 & 1396.00 &  0.00\\
t20m10r10-5.pl.json & 20 & 10 & Optimal &  0.07 & 1710 & 1710.00 &  0.00\\
t20m10r10-6.pl.json & 20 & 10 & Optimal &  0.06 & 2434 & 2434.00 &  0.00\\
t20m10r10-7.pl.json & 20 & 10 & Optimal &  0.12 & 2696 & 2696.00 &  0.00\\
t20m10r10-8.pl.json & 20 & 10 & Optimal &  0.05 & 1329 & 1329.00 &  0.00\\
t20m10r10-9.pl.json & 20 & 10 & Optimal &  0.93 & 2933 & 2933.00 &  0.00\\
t20m10r3-1.pl.json & 20 & 10 & Optimal &  0.05 & 1876 & 1876.00 &  0.00\\
t20m10r3-10.pl.json & 20 & 10 & Optimal &  0.06 & 1652 & 1652.00 &  0.00\\
t20m10r3-11.pl.json & 20 & 10 & Optimal &  0.04 & 1640 & 1640.00 &  0.00\\
t20m10r3-12.pl.json & 20 & 10 & Optimal &  0.07 & 1758 & 1758.00 &  0.00\\
t20m10r3-13.pl.json & 20 & 10 & Optimal &  0.06 & 3099 & 3099.00 &  0.00\\
t20m10r3-14.pl.json & 20 & 10 & Optimal &  1.76 & 3891 & 3891.00 &  0.00\\
t20m10r3-15.pl.json & 20 & 10 & Optimal &  0.07 & 1433 & 1433.00 &  0.00\\
t20m10r3-16.pl.json & 20 & 10 & Optimal &  0.06 & 1564 & 1564.00 &  0.00\\
t20m10r3-17.pl.json & 20 & 10 & Optimal &  0.06 & 2321 & 2321.00 &  0.00\\
t20m10r3-18.pl.json & 20 & 10 & Solution & 30.05 & 821 & 746.00 &  9.14\\
t20m10r3-19.pl.json & 20 & 10 & Optimal &  0.07 & 1236 & 1236.00 &  0.00\\
t20m10r3-2.pl.json & 20 & 10 & Optimal &  0.06 & 3258 & 3258.00 &  0.00\\
t20m10r3-20.pl.json & 20 & 10 & Optimal &  0.04 & 2168 & 2168.00 &  0.00\\
t20m10r3-3.pl.json & 20 & 10 & Optimal &  0.04 & 2255 & 2255.00 &  0.00\\
t20m10r3-4.pl.json & 20 & 10 & Optimal &  0.09 & 2707 & 2707.00 &  0.00\\
t20m10r3-5.pl.json & 20 & 10 & Optimal &  0.06 & 2381 & 2381.00 &  0.00\\
t20m10r3-6.pl.json & 20 & 10 & Optimal &  0.07 & 3043 & 3043.00 &  0.00\\
t20m10r3-7.pl.json & 20 & 10 & Optimal &  0.05 & 1738 & 1738.00 &  0.00\\
t20m10r3-8.pl.json & 20 & 10 & Optimal &  0.18 & 1278 & 1278.00 &  0.00\\
t20m10r3-9.pl.json & 20 & 10 & Optimal &  0.05 & 2874 & 2874.00 &  0.00\\
t20m10r5-1.pl.json & 20 & 10 & Optimal &  0.06 & 2586 & 2586.00 &  0.00\\
t20m10r5-10.pl.json & 20 & 10 & Optimal &  0.07 & 2260 & 2260.00 &  0.00\\
t20m10r5-11.pl.json & 20 & 10 & Optimal &  0.05 & 3487 & 3487.00 &  0.00\\
t20m10r5-12.pl.json & 20 & 10 & Optimal &  0.05 & 1559 & 1559.00 &  0.00\\
t20m10r5-13.pl.json & 20 & 10 & Optimal &  0.06 & 1457 & 1457.00 &  0.00\\
t20m10r5-14.pl.json & 20 & 10 & Optimal &  0.08 & 1141 & 1141.00 &  0.00\\
t20m10r5-15.pl.json & 20 & 10 & Optimal &  0.14 & 821 & 821.00 &  0.00\\
t20m10r5-16.pl.json & 20 & 10 & Optimal &  0.06 & 2910 & 2910.00 &  0.00\\
t20m10r5-17.pl.json & 20 & 10 & Optimal &  0.07 & 2337 & 2337.00 &  0.00\\
t20m10r5-18.pl.json & 20 & 10 & Optimal &  0.80 & 2920 & 2920.00 &  0.00\\
t20m10r5-19.pl.json & 20 & 10 & Optimal &  0.04 & 1952 & 1952.00 &  0.00\\
t20m10r5-2.pl.json & 20 & 10 & Optimal &  0.06 & 1639 & 1639.00 &  0.00\\
t20m10r5-20.pl.json & 20 & 10 & Optimal &  0.04 & 2660 & 2660.00 &  0.00\\
t20m10r5-3.pl.json & 20 & 10 & Optimal &  0.06 & 1406 & 1406.00 &  0.00\\
t20m10r5-4.pl.json & 20 & 10 & Optimal &  0.07 & 2658 & 2658.00 &  0.00\\
t20m10r5-5.pl.json & 20 & 10 & Optimal &  0.09 & 794 & 794.00 &  0.00\\
t20m10r5-6.pl.json & 20 & 10 & Optimal &  0.06 & 2398 & 2398.00 &  0.00\\
t20m10r5-7.pl.json & 20 & 10 & Optimal &  0.04 & 1430 & 1430.00 &  0.00\\
t20m10r5-8.pl.json & 20 & 10 & Optimal &  0.09 & 976 & 976.00 &  0.00\\
t20m10r5-9.pl.json & 20 & 10 & Optimal &  0.06 & 2953 & 2953.00 &  0.00\\
t30m10r10-1.pl.json & 30 & 10 & Optimal &  3.50 & 3344 & 3344.00 &  0.00\\
t30m10r10-10.pl.json & 30 & 10 & Solution & 30.03 & 4692 & 4146.00 & 11.64\\
t30m10r10-11.pl.json & 30 & 10 & Optimal &  0.12 & 2905 & 2905.00 &  0.00\\
t30m10r10-12.pl.json & 30 & 10 & Optimal &  0.11 & 3672 & 3672.00 &  0.00\\
t30m10r10-13.pl.json & 30 & 10 & Optimal &  0.15 & 2778 & 2778.00 &  0.00\\
t30m10r10-14.pl.json & 30 & 10 & Optimal &  1.59 & 2741 & 2741.00 &  0.00\\
t30m10r10-15.pl.json & 30 & 10 & Optimal &  0.12 & 2388 & 2388.00 &  0.00\\
t30m10r10-16.pl.json & 30 & 10 & Optimal &  3.04 & 4225 & 4225.00 &  0.00\\
t30m10r10-17.pl.json & 30 & 10 & Optimal &  0.11 & 1504 & 1504.00 &  0.00\\
t30m10r10-18.pl.json & 30 & 10 & Optimal &  7.37 & 3287 & 3287.00 &  0.00\\
t30m10r10-19.pl.json & 30 & 10 & Optimal &  0.11 & 3874 & 3874.00 &  0.00\\
t30m10r10-2.pl.json & 30 & 10 & Optimal &  0.09 & 3169 & 3169.00 &  0.00\\
t30m10r10-20.pl.json & 30 & 10 & Optimal &  0.07 & 2691 & 2691.00 &  0.00\\
t30m10r10-3.pl.json & 30 & 10 & Solution & 30.02 & 3360 & 2851.00 & 15.15\\
t30m10r10-4.pl.json & 30 & 10 & Optimal &  0.08 & 3452 & 3452.00 &  0.00\\
t30m10r10-5.pl.json & 30 & 10 & Optimal &  0.08 & 2785 & 2785.00 &  0.00\\
t30m10r10-6.pl.json & 30 & 10 & Solution & 30.09 & 1011 & 775.00 & 23.34\\
t30m10r10-7.pl.json & 30 & 10 & Optimal &  4.10 & 3755 & 3755.00 &  0.00\\
t30m10r10-8.pl.json & 30 & 10 & Optimal & 11.44 & 4613 & 4613.00 &  0.00\\
t30m10r10-9.pl.json & 30 & 10 & Optimal &  0.08 & 2770 & 2770.00 &  0.00\\
t30m10r3-1.pl.json & 30 & 10 & Optimal &  0.17 & 2901 & 2901.00 &  0.00\\
t30m10r3-10.pl.json & 30 & 10 & Optimal &  0.10 & 4829 & 4829.00 &  0.00\\
t30m10r3-11.pl.json & 30 & 10 & Optimal &  0.09 & 2584 & 2584.00 &  0.00\\
t30m10r3-12.pl.json & 30 & 10 & Optimal &  0.08 & 2130 & 2130.00 &  0.00\\
t30m10r3-13.pl.json & 30 & 10 & Optimal &  0.07 & 4253 & 4253.00 &  0.00\\
t30m10r3-14.pl.json & 30 & 10 & Optimal &  0.16 & 1393 & 1393.00 &  0.00\\
t30m10r3-15.pl.json & 30 & 10 & Optimal &  0.11 & 4149 & 4149.00 &  0.00\\
t30m10r3-16.pl.json & 30 & 10 & Optimal &  0.23 & 2027 & 2027.00 &  0.00\\
t30m10r3-17.pl.json & 30 & 10 & Optimal &  0.11 & 2975 & 2975.00 &  0.00\\
t30m10r3-18.pl.json & 30 & 10 & Optimal &  0.13 & 5477 & 5477.00 &  0.00\\
t30m10r3-19.pl.json & 30 & 10 & Solution & 30.02 & 1288 & 1042.00 & 19.10\\
t30m10r3-2.pl.json & 30 & 10 & Optimal &  0.15 & 2523 & 2523.00 &  0.00\\
t30m10r3-20.pl.json & 30 & 10 & Optimal &  0.09 & 4754 & 4754.00 &  0.00\\
t30m10r3-3.pl.json & 30 & 10 & Optimal &  0.07 & 2793 & 2793.00 &  0.00\\
t30m10r3-4.pl.json & 30 & 10 & Optimal &  0.97 & 2809 & 2809.00 &  0.00\\
t30m10r3-5.pl.json & 30 & 10 & Optimal &  0.14 & 3758 & 3758.00 &  0.00\\
t30m10r3-6.pl.json & 30 & 10 & Optimal &  0.06 & 2870 & 2870.00 &  0.00\\
t30m10r3-7.pl.json & 30 & 10 & Optimal &  0.13 & 2122 & 2122.00 &  0.00\\
t30m10r3-8.pl.json & 30 & 10 & Optimal &  0.13 & 2862 & 2862.00 &  0.00\\
t30m10r3-9.pl.json & 30 & 10 & Optimal &  0.09 & 2754 & 2754.00 &  0.00\\
t30m10r5-1.pl.json & 30 & 10 & Optimal &  0.09 & 1998 & 1998.00 &  0.00\\
t30m10r5-10.pl.json & 30 & 10 & Optimal &  0.12 & 3743 & 3743.00 &  0.00\\
t30m10r5-11.pl.json & 30 & 10 & Optimal &  0.12 & 2138 & 2138.00 &  0.00\\
t30m10r5-12.pl.json & 30 & 10 & Optimal &  0.08 & 2251 & 2251.00 &  0.00\\
t30m10r5-13.pl.json & 30 & 10 & Optimal &  0.10 & 2632 & 2632.00 &  0.00\\
t30m10r5-14.pl.json & 30 & 10 & Optimal &  0.11 & 2201 & 2201.00 &  0.00\\
t30m10r5-15.pl.json & 30 & 10 & Optimal &  0.10 & 2339 & 2339.00 &  0.00\\
t30m10r5-16.pl.json & 30 & 10 & Optimal &  0.17 & 4293 & 4293.00 &  0.00\\
t30m10r5-17.pl.json & 30 & 10 & Optimal &  0.15 & 1314 & 1314.00 &  0.00\\
t30m10r5-18.pl.json & 30 & 10 & Optimal &  0.09 & 2169 & 2169.00 &  0.00\\
t30m10r5-19.pl.json & 30 & 10 & Solution & 30.14 & 1346 & 1279.00 &  4.98\\
t30m10r5-2.pl.json & 30 & 10 & Optimal &  0.05 & 2399 & 2399.00 &  0.00\\
t30m10r5-20.pl.json & 30 & 10 & Optimal &  0.18 & 1486 & 1486.00 &  0.00\\
t30m10r5-3.pl.json & 30 & 10 & Optimal &  0.08 & 2494 & 2494.00 &  0.00\\
t30m10r5-4.pl.json & 30 & 10 & Optimal &  0.11 & 3405 & 3405.00 &  0.00\\
t30m10r5-5.pl.json & 30 & 10 & Optimal &  3.99 & 5243 & 5243.00 &  0.00\\
t30m10r5-6.pl.json & 30 & 10 & Optimal &  0.09 & 2382 & 2382.00 &  0.00\\
t30m10r5-7.pl.json & 30 & 10 & Optimal &  0.10 & 2018 & 2018.00 &  0.00\\
t30m10r5-8.pl.json & 30 & 10 & Optimal &  0.13 & 3089 & 3089.00 &  0.00\\
t30m10r5-9.pl.json & 30 & 10 & Optimal &  0.12 & 3704 & 3704.00 &  0.00\\
t30m20r10-1.pl.json & 30 & 20 & Solution & 30.01 & 3702 & 2850.00 & 23.01\\
t30m20r10-10.pl.json & 30 & 20 & Optimal &  0.11 & 2508 & 2508.00 &  0.00\\
t30m20r10-11.pl.json & 30 & 20 & Optimal &  1.75 & 3648 & 3648.00 &  0.00\\
t30m20r10-12.pl.json & 30 & 20 & Optimal &  0.36 & 4214 & 4214.00 &  0.00\\
t30m20r10-13.pl.json & 30 & 20 & Optimal &  6.20 & 3980 & 3980.00 &  0.00\\
t30m20r10-14.pl.json & 30 & 20 & Optimal &  0.17 & 3141 & 3141.00 &  0.00\\
t30m20r10-15.pl.json & 30 & 20 & Solution & 30.03 & 4322 & 3457.00 & 20.01\\
t30m20r10-16.pl.json & 30 & 20 & Optimal &  0.22 & 4002 & 4002.00 &  0.00\\
t30m20r10-17.pl.json & 30 & 20 & Optimal & 19.13 & 4161 & 4161.00 &  0.00\\
t30m20r10-18.pl.json & 30 & 20 & Optimal &  3.59 & 1992 & 1992.00 &  0.00\\
t30m20r10-19.pl.json & 30 & 20 & Optimal &  0.17 & 2789 & 2789.00 &  0.00\\
t30m20r10-2.pl.json & 30 & 20 & Optimal & 20.44 & 3982 & 3982.00 &  0.00\\
t30m20r10-20.pl.json & 30 & 20 & Optimal &  0.18 & 2314 & 2314.00 &  0.00\\
t30m20r10-3.pl.json & 30 & 20 & Optimal &  0.15 & 2158 & 2158.00 &  0.00\\
t30m20r10-4.pl.json & 30 & 20 & Optimal &  6.52 & 4040 & 4040.00 &  0.00\\
t30m20r10-5.pl.json & 30 & 20 & Optimal &  0.14 & 1237 & 1237.00 &  0.00\\
t30m20r10-6.pl.json & 30 & 20 & Optimal &  3.53 & 3770 & 3770.00 &  0.00\\
t30m20r10-7.pl.json & 30 & 20 & Optimal &  0.18 & 2266 & 2266.00 &  0.00\\
t30m20r10-8.pl.json & 30 & 20 & Optimal &  0.45 & 1855 & 1855.00 &  0.00\\
t30m20r10-9.pl.json & 30 & 20 & Optimal &  0.80 & 2028 & 2028.00 &  0.00\\
t30m20r3-1.pl.json & 30 & 20 & Optimal &  0.17 & 2200 & 2200.00 &  0.00\\
t30m20r3-10.pl.json & 30 & 20 & Optimal &  0.13 & 3291 & 3291.00 &  0.00\\
t30m20r3-11.pl.json & 30 & 20 & Optimal &  0.22 & 4473 & 4473.00 &  0.00\\
t30m20r3-12.pl.json & 30 & 20 & Optimal &  3.75 & 5060 & 5060.00 &  0.00\\
t30m20r3-13.pl.json & 30 & 20 & Optimal &  0.14 & 3536 & 3536.00 &  0.00\\
t30m20r3-14.pl.json & 30 & 20 & Optimal &  0.15 & 3432 & 3432.00 &  0.00\\
t30m20r3-15.pl.json & 30 & 20 & Optimal &  0.14 & 3463 & 3463.00 &  0.00\\
t30m20r3-16.pl.json & 30 & 20 & Optimal &  0.16 & 3893 & 3893.00 &  0.00\\
t30m20r3-17.pl.json & 30 & 20 & Optimal &  0.19 & 1892 & 1892.00 &  0.00\\
t30m20r3-18.pl.json & 30 & 20 & Optimal &  0.16 & 2653 & 2653.00 &  0.00\\
t30m20r3-19.pl.json & 30 & 20 & Optimal &  0.18 & 3141 & 3141.00 &  0.00\\
t30m20r3-2.pl.json & 30 & 20 & Optimal &  0.15 & 1251 & 1251.00 &  0.00\\
t30m20r3-20.pl.json & 30 & 20 & Optimal &  2.08 & 2745 & 2745.00 &  0.00\\
t30m20r3-3.pl.json & 30 & 20 & Optimal &  0.18 & 3434 & 3434.00 &  0.00\\
t30m20r3-4.pl.json & 30 & 20 & Optimal &  0.19 & 2394 & 2394.00 &  0.00\\
t30m20r3-5.pl.json & 30 & 20 & Optimal &  0.12 & 3776 & 3776.00 &  0.00\\
t30m20r3-6.pl.json & 30 & 20 & Optimal &  0.20 & 2250 & 2250.00 &  0.00\\
t30m20r3-7.pl.json & 30 & 20 & Optimal &  0.21 & 1693 & 1693.00 &  0.00\\
t30m20r3-8.pl.json & 30 & 20 & Optimal &  0.12 & 4997 & 4997.00 &  0.00\\
t30m20r3-9.pl.json & 30 & 20 & Optimal &  0.16 & 4898 & 4898.00 &  0.00\\
t30m20r5-1.pl.json & 30 & 20 & Optimal &  2.62 & 3195 & 3195.00 &  0.00\\
t30m20r5-10.pl.json & 30 & 20 & Optimal &  0.72 & 2133 & 2133.00 &  0.00\\
t30m20r5-11.pl.json & 30 & 20 & Optimal &  0.17 & 3974 & 3974.00 &  0.00\\
t30m20r5-12.pl.json & 30 & 20 & Optimal &  0.16 & 2197 & 2197.00 &  0.00\\
t30m20r5-13.pl.json & 30 & 20 & Optimal &  0.15 & 2296 & 2296.00 &  0.00\\
t30m20r5-14.pl.json & 30 & 20 & Optimal &  0.21 & 3861 & 3861.00 &  0.00\\
t30m20r5-15.pl.json & 30 & 20 & Optimal &  0.16 & 2353 & 2353.00 &  0.00\\
t30m20r5-16.pl.json & 30 & 20 & Optimal &  1.80 & 2751 & 2751.00 &  0.00\\
t30m20r5-17.pl.json & 30 & 20 & Optimal &  0.22 & 3555 & 3555.00 &  0.00\\
t30m20r5-18.pl.json & 30 & 20 & Optimal &  0.14 & 2384 & 2384.00 &  0.00\\
t30m20r5-19.pl.json & 30 & 20 & Optimal &  0.17 & 2080 & 2080.00 &  0.00\\
t30m20r5-2.pl.json & 30 & 20 & Optimal &  0.11 & 1715 & 1715.00 &  0.00\\
t30m20r5-20.pl.json & 30 & 20 & Optimal &  0.15 & 4176 & 4176.00 &  0.00\\
t30m20r5-3.pl.json & 30 & 20 & Optimal & 15.15 & 4528 & 4528.00 &  0.00\\
t30m20r5-4.pl.json & 30 & 20 & Optimal &  0.20 & 3083 & 3083.00 &  0.00\\
t30m20r5-5.pl.json & 30 & 20 & Optimal &  0.12 & 1969 & 1969.00 &  0.00\\
t30m20r5-6.pl.json & 30 & 20 & Optimal &  0.15 & 4250 & 4250.00 &  0.00\\
t30m20r5-7.pl.json & 30 & 20 & Optimal &  0.19 & 3036 & 3036.00 &  0.00\\
t30m20r5-8.pl.json & 30 & 20 & Optimal &  2.07 & 2834 & 2834.00 &  0.00\\
t30m20r5-9.pl.json & 30 & 20 & Optimal &  0.16 & 2343 & 2343.00 &  0.00\\
t40m10r10-1.pl.json & 40 & 10 & Optimal &  0.18 & 2514 & 2514.00 &  0.00\\
t40m10r10-10.pl.json & 40 & 10 & Optimal &  0.18 & 3557 & 3557.00 &  0.00\\
t40m10r10-11.pl.json & 40 & 10 & Solution & 30.03 & 4556 & 4262.00 &  6.45\\
t40m10r10-12.pl.json & 40 & 10 & Solution & 30.04 & 5225 & 4355.00 & 16.65\\
t40m10r10-13.pl.json & 40 & 10 & Optimal &  5.66 & 2789 & 2789.00 &  0.00\\
t40m10r10-14.pl.json & 40 & 10 & Optimal &  0.34 & 1648 & 1648.00 &  0.00\\
t40m10r10-15.pl.json & 40 & 10 & Optimal &  1.53 & 1844 & 1844.00 &  0.00\\
t40m10r10-16.pl.json & 40 & 10 & Optimal &  9.44 & 3749 & 3749.00 &  0.00\\
t40m10r10-17.pl.json & 40 & 10 & Optimal &  0.15 & 2363 & 2363.00 &  0.00\\
t40m10r10-18.pl.json & 40 & 10 & Optimal &  0.23 & 4973 & 4973.00 &  0.00\\
t40m10r10-19.pl.json & 40 & 10 & Optimal &  0.26 & 3181 & 3181.00 &  0.00\\
t40m10r10-2.pl.json & 40 & 10 & Optimal &  0.24 & 2350 & 2350.00 &  0.00\\
t40m10r10-20.pl.json & 40 & 10 & Optimal & 11.15 & 2730 & 2730.00 &  0.00\\
t40m10r10-3.pl.json & 40 & 10 & Optimal &  0.17 & 3717 & 3717.00 &  0.00\\
t40m10r10-4.pl.json & 40 & 10 & Optimal &  0.17 & 3414 & 3414.00 &  0.00\\
t40m10r10-5.pl.json & 40 & 10 & Optimal &  2.53 & 2852 & 2852.00 &  0.00\\
t40m10r10-6.pl.json & 40 & 10 & Optimal &  8.05 & 3262 & 3262.00 &  0.00\\
t40m10r10-7.pl.json & 40 & 10 & Optimal &  0.15 & 4572 & 4572.00 &  0.00\\
t40m10r10-8.pl.json & 40 & 10 & Optimal &  6.04 & 3776 & 3776.00 &  0.00\\
t40m10r10-9.pl.json & 40 & 10 & Optimal &  0.34 & 2524 & 2524.00 &  0.00\\
t40m10r3-1.pl.json & 40 & 10 & Optimal &  0.17 & 4832 & 4832.00 &  0.00\\
t40m10r3-10.pl.json & 40 & 10 & Optimal &  0.12 & 2442 & 2442.00 &  0.00\\
t40m10r3-11.pl.json & 40 & 10 & Optimal &  0.52 & 3218 & 3218.00 &  0.00\\
t40m10r3-12.pl.json & 40 & 10 & Optimal &  0.11 & 3863 & 3863.00 &  0.00\\
t40m10r3-13.pl.json & 40 & 10 & Optimal &  0.41 & 3564 & 3564.00 &  0.00\\
t40m10r3-14.pl.json & 40 & 10 & Optimal &  0.15 & 4913 & 4913.00 &  0.00\\
t40m10r3-15.pl.json & 40 & 10 & Optimal &  0.21 & 3785 & 3785.00 &  0.00\\
t40m10r3-16.pl.json & 40 & 10 & Optimal &  0.37 & 2840 & 2840.00 &  0.00\\
t40m10r3-17.pl.json & 40 & 10 & Optimal &  0.20 & 5506 & 5506.00 &  0.00\\
t40m10r3-18.pl.json & 40 & 10 & Optimal &  0.38 & 3848 & 3848.00 &  0.00\\
t40m10r3-19.pl.json & 40 & 10 & Optimal &  0.27 & 2259 & 2259.00 &  0.00\\
t40m10r3-2.pl.json & 40 & 10 & Solution & 30.18 & 1729 & 1589.00 &  8.10\\
t40m10r3-20.pl.json & 40 & 10 & Optimal &  0.26 & 4157 & 4157.00 &  0.00\\
t40m10r3-3.pl.json & 40 & 10 & Optimal &  0.33 & 4903 & 4903.00 &  0.00\\
t40m10r3-4.pl.json & 40 & 10 & Solution & 30.02 & 1633 & 1341.00 & 17.88\\
t40m10r3-5.pl.json & 40 & 10 & Optimal &  0.34 & 1984 & 1984.00 &  0.00\\
t40m10r3-6.pl.json & 40 & 10 & Optimal &  0.35 & 5005 & 5005.00 &  0.00\\
t40m10r3-7.pl.json & 40 & 10 & Solution & 30.02 & 5545 & 5188.00 &  6.44\\
t40m10r3-8.pl.json & 40 & 10 & Optimal &  0.24 & 3658 & 3658.00 &  0.00\\
t40m10r3-9.pl.json & 40 & 10 & Optimal &  0.36 & 3830 & 3830.00 &  0.00\\
t40m10r5-1.pl.json & 40 & 10 & Optimal &  0.20 & 4857 & 4857.00 &  0.00\\
t40m10r5-10.pl.json & 40 & 10 & Optimal &  0.20 & 3989 & 3989.00 &  0.00\\
t40m10r5-11.pl.json & 40 & 10 & Optimal &  0.33 & 5238 & 5238.00 &  0.00\\
t40m10r5-12.pl.json & 40 & 10 & Optimal &  0.42 & 4584 & 4584.00 &  0.00\\
t40m10r5-13.pl.json & 40 & 10 & Optimal &  0.40 & 2307 & 2307.00 &  0.00\\
t40m10r5-14.pl.json & 40 & 10 & Optimal &  0.21 & 1826 & 1826.00 &  0.00\\
t40m10r5-15.pl.json & 40 & 10 & Optimal &  0.17 & 1926 & 1926.00 &  0.00\\
t40m10r5-16.pl.json & 40 & 10 & Optimal &  0.26 & 5216 & 5216.00 &  0.00\\
t40m10r5-17.pl.json & 40 & 10 & Optimal &  0.14 & 7162 & 7162.00 &  0.00\\
t40m10r5-18.pl.json & 40 & 10 & Optimal &  0.24 & 4892 & 4892.00 &  0.00\\
t40m10r5-19.pl.json & 40 & 10 & Optimal &  0.18 & 4027 & 4027.00 &  0.00\\
t40m10r5-2.pl.json & 40 & 10 & Optimal &  3.51 & 4099 & 4099.00 &  0.00\\
t40m10r5-20.pl.json & 40 & 10 & Optimal & 10.41 & 4899 & 4899.00 &  0.00\\
t40m10r5-3.pl.json & 40 & 10 & Optimal &  0.64 & 3113 & 3113.00 &  0.00\\
t40m10r5-4.pl.json & 40 & 10 & Optimal &  0.21 & 6626 & 6626.00 &  0.00\\
t40m10r5-5.pl.json & 40 & 10 & Optimal &  0.25 & 3828 & 3828.00 &  0.00\\
t40m10r5-6.pl.json & 40 & 10 & Optimal &  0.33 & 4213 & 4213.00 &  0.00\\
t40m10r5-7.pl.json & 40 & 10 & Optimal &  0.21 & 4303 & 4303.00 &  0.00\\
t40m10r5-8.pl.json & 40 & 10 & Solution & 30.02 & 3559 & 3189.00 & 10.40\\
t40m10r5-9.pl.json & 40 & 10 & Optimal &  0.30 & 1953 & 1953.00 &  0.00\\
t40m20r10-1.pl.json & 40 & 20 & Solution & 30.05 & 4518 & 3972.00 & 12.08\\
t40m20r10-10.pl.json & 40 & 20 & Optimal &  4.24 & 3862 & 3862.00 &  0.00\\
t40m20r10-11.pl.json & 40 & 20 & Optimal &  0.21 & 1952 & 1952.00 &  0.00\\
t40m20r10-12.pl.json & 40 & 20 & Optimal &  0.71 & 4129 & 4129.00 &  0.00\\
t40m20r10-13.pl.json & 40 & 20 & Optimal &  0.23 & 2927 & 2927.00 &  0.00\\
t40m20r10-14.pl.json & 40 & 20 & Optimal &  6.14 & 2701 & 2701.00 &  0.00\\
t40m20r10-15.pl.json & 40 & 20 & Optimal &  6.72 & 3168 & 3168.00 &  0.00\\
t40m20r10-16.pl.json & 40 & 20 & Optimal &  0.15 & 2812 & 2812.00 &  0.00\\
t40m20r10-17.pl.json & 40 & 20 & Optimal &  8.83 & 4288 & 4288.00 &  0.00\\
t40m20r10-18.pl.json & 40 & 20 & Optimal &  8.25 & 3611 & 3611.00 &  0.00\\
t40m20r10-19.pl.json & 40 & 20 & Optimal &  1.71 & 2891 & 2891.00 &  0.00\\
t40m20r10-2.pl.json & 40 & 20 & Optimal &  0.16 & 3284 & 3284.00 &  0.00\\
t40m20r10-20.pl.json & 40 & 20 & Optimal & 23.67 & 5506 & 5506.00 &  0.00\\
t40m20r10-3.pl.json & 40 & 20 & Solution & 30.03 & 5981 & 5478.00 &  8.41\\
t40m20r10-4.pl.json & 40 & 20 & Optimal &  0.16 & 3409 & 3409.00 &  0.00\\
t40m20r10-5.pl.json & 40 & 20 & Solution & 30.04 & 5113 & 4278.00 & 16.33\\
t40m20r10-6.pl.json & 40 & 20 & Optimal & 21.04 & 2376 & 2376.00 &  0.00\\
t40m20r10-7.pl.json & 40 & 20 & Optimal & 18.53 & 4799 & 4799.00 &  0.00\\
t40m20r10-8.pl.json & 40 & 20 & Optimal &  6.17 & 3924 & 3924.00 &  0.00\\
t40m20r10-9.pl.json & 40 & 20 & Optimal &  4.28 & 2043 & 2043.00 &  0.00\\
t40m20r3-1.pl.json & 40 & 20 & Optimal &  0.26 & 3524 & 3524.00 &  0.00\\
t40m20r3-10.pl.json & 40 & 20 & Optimal &  0.60 & 3110 & 3110.00 &  0.00\\
t40m20r3-11.pl.json & 40 & 20 & Optimal &  0.22 & 3695 & 3695.00 &  0.00\\
t40m20r3-12.pl.json & 40 & 20 & Optimal &  0.31 & 4828 & 4828.00 &  0.00\\
t40m20r3-13.pl.json & 40 & 20 & Optimal &  0.35 & 4010 & 4010.00 &  0.00\\
t40m20r3-14.pl.json & 40 & 20 & Optimal &  0.14 & 2752 & 2752.00 &  0.00\\
t40m20r3-15.pl.json & 40 & 20 & Optimal &  0.22 & 3312 & 3312.00 &  0.00\\
t40m20r3-16.pl.json & 40 & 20 & Optimal &  0.41 & 4304 & 4304.00 &  0.00\\
t40m20r3-17.pl.json & 40 & 20 & Optimal &  0.29 & 3991 & 3991.00 &  0.00\\
t40m20r3-18.pl.json & 40 & 20 & Optimal &  0.27 & 5733 & 5733.00 &  0.00\\
t40m20r3-19.pl.json & 40 & 20 & Optimal &  0.24 & 3581 & 3581.00 &  0.00\\
t40m20r3-2.pl.json & 40 & 20 & Optimal &  0.30 & 4869 & 4869.00 &  0.00\\
t40m20r3-20.pl.json & 40 & 20 & Optimal &  0.34 & 3514 & 3514.00 &  0.00\\
t40m20r3-3.pl.json & 40 & 20 & Optimal &  0.27 & 2503 & 2503.00 &  0.00\\
t40m20r3-4.pl.json & 40 & 20 & Optimal &  0.21 & 4323 & 4323.00 &  0.00\\
t40m20r3-5.pl.json & 40 & 20 & Optimal &  0.29 & 3626 & 3626.00 &  0.00\\
t40m20r3-6.pl.json & 40 & 20 & Optimal &  0.22 & 2488 & 2488.00 &  0.00\\
t40m20r3-7.pl.json & 40 & 20 & Optimal &  0.16 & 3470 & 3470.00 &  0.00\\
t40m20r3-8.pl.json & 40 & 20 & Optimal &  0.84 & 6730 & 6730.00 &  0.00\\
t40m20r3-9.pl.json & 40 & 20 & Optimal &  0.23 & 4656 & 4656.00 &  0.00\\
t40m20r5-1.pl.json & 40 & 20 & Optimal &  0.20 & 1318 & 1318.00 &  0.00\\
t40m20r5-10.pl.json & 40 & 20 & Optimal &  0.27 & 2216 & 2216.00 &  0.00\\
t40m20r5-11.pl.json & 40 & 20 & Optimal &  0.23 & 3538 & 3538.00 &  0.00\\
t40m20r5-12.pl.json & 40 & 20 & Optimal &  0.33 & 5346 & 5346.00 &  0.00\\
t40m20r5-13.pl.json & 40 & 20 & Optimal & 21.99 & 4589 & 4589.00 &  0.00\\
t40m20r5-14.pl.json & 40 & 20 & Optimal &  0.21 & 2243 & 2243.00 &  0.00\\
t40m20r5-15.pl.json & 40 & 20 & Optimal &  8.96 & 3869 & 3869.00 &  0.00\\
t40m20r5-16.pl.json & 40 & 20 & Optimal &  0.28 & 4319 & 4319.00 &  0.00\\
t40m20r5-17.pl.json & 40 & 20 & Optimal &  0.21 & 4866 & 4866.00 &  0.00\\
t40m20r5-18.pl.json & 40 & 20 & Optimal &  0.66 & 5802 & 5802.00 &  0.00\\
t40m20r5-19.pl.json & 40 & 20 & Optimal & 10.46 & 4197 & 4197.00 &  0.00\\
t40m20r5-2.pl.json & 40 & 20 & Optimal &  0.17 & 2634 & 2634.00 &  0.00\\
t40m20r5-20.pl.json & 40 & 20 & Solution & 30.05 & 6482 & 6232.00 &  3.86\\
t40m20r5-3.pl.json & 40 & 20 & Optimal &  0.38 & 4391 & 4391.00 &  0.00\\
t40m20r5-4.pl.json & 40 & 20 & Optimal &  5.34 & 4610 & 4610.00 &  0.00\\
t40m20r5-5.pl.json & 40 & 20 & Optimal &  0.17 & 3105 & 3105.00 &  0.00\\
t40m20r5-6.pl.json & 40 & 20 & Optimal &  0.21 & 4760 & 4760.00 &  0.00\\
t40m20r5-7.pl.json & 40 & 20 & Optimal &  0.22 & 1218 & 1218.00 &  0.00\\
t40m20r5-8.pl.json & 40 & 20 & Optimal &  0.20 & 2601 & 2601.00 &  0.00\\
t40m20r5-9.pl.json & 40 & 20 & Optimal &  0.17 & 3141 & 3141.00 &  0.00\\
t500m100r10-1.pl.json & 500 & 100 & Solution & 34.47 & 99985 & 44508.00 & 55.49\\
t500m100r10-10.pl.json & 500 & 100 & Solution & 36.70 & 99989 & 35930.00 & 64.07\\
t500m100r10-11.pl.json & 500 & 100 & Solution & 37.37 & 99998 & 31878.00 & 68.12\\
t500m100r10-12.pl.json & 500 & 100 & Solution & 41.73 & 99997 & 44533.00 & 55.47\\
t500m100r10-13.pl.json & 500 & 100 & Solution & 40.49 & 99999 & 37955.00 & 62.04\\
t500m100r10-14.pl.json & 500 & 100 & Solution & 36.97 & 99990 & 34723.00 & 65.27\\
t500m100r10-15.pl.json & 500 & 100 & Solution & 40.76 & 100000 & 35403.00 & 64.60\\
t500m100r10-16.pl.json & 500 & 100 & Solution & 39.01 & 99999 & 33693.00 & 66.31\\
t500m100r10-17.pl.json & 500 & 100 & Solution & 42.46 & 99997 & 28688.00 & 71.31\\
t500m100r10-18.pl.json & 500 & 100 & Solution & 42.66 & 100000 & 37334.00 & 62.67\\
t500m100r10-19.pl.json & 500 & 100 & Solution & 43.40 & 100000 & 40128.00 & 59.87\\
t500m100r10-2.pl.json & 500 & 100 & Solution & 43.11 & 100000 & 37597.00 & 62.40\\
t500m100r10-20.pl.json & 500 & 100 & Solution & 42.96 & 100000 & 30194.00 & 69.81\\
t500m100r10-3.pl.json & 500 & 100 & Solution & 43.67 & 99999 & 31662.00 & 68.34\\
t500m100r10-4.pl.json & 500 & 100 & Solution & 43.11 & 99993 & 35350.00 & 64.65\\
t500m100r10-5.pl.json & 500 & 100 & Solution & 44.75 & 99996 & 30335.00 & 69.66\\
t500m100r10-6.pl.json & 500 & 100 & Solution & 41.60 & 99986 & 35654.00 & 64.34\\
t500m100r10-7.pl.json & 500 & 100 & Solution & 89.79 & 99991 & 35760.00 & 64.24\\
t500m100r10-8.pl.json & 500 & 100 & Solution & 81.73 & 99990 & 37775.00 & 62.22\\
t500m100r10-9.pl.json & 500 & 100 & Solution & 43.73 & 100000 & 34951.00 & 65.05\\
t500m100r3-1.pl.json & 500 & 100 & Solution & 42.91 & 99985 & 37887.00 & 62.11\\
t500m100r3-10.pl.json & 500 & 100 & Solution & 43.92 & 99993 & 41592.00 & 58.41\\
t500m100r3-11.pl.json & 500 & 100 & Solution & 46.47 & 99996 & 36331.00 & 63.67\\
t500m100r3-12.pl.json & 500 & 100 & Solution & 45.82 & 100000 & 36704.00 & 63.30\\
t500m100r3-13.pl.json & 500 & 100 & Solution & 42.12 & 99995 & 35381.00 & 64.62\\
t500m100r3-14.pl.json & 500 & 100 & Solution & 88.74 & 99982 & 40411.00 & 59.58\\
t500m100r3-15.pl.json & 500 & 100 & Solution & 44.26 & 99992 & 38658.00 & 61.34\\
t500m100r3-16.pl.json & 500 & 100 & Solution & 93.53 & 99986 & 39443.00 & 60.55\\
t500m100r3-17.pl.json & 500 & 100 & Solution & 42.16 & 99994 & 54487.00 & 45.51\\
t500m100r3-18.pl.json & 500 & 100 & Solution & 91.03 & 100000 & 38068.00 & 61.93\\
t500m100r3-19.pl.json & 500 & 100 & Solution & 48.15 & 100000 & 41896.00 & 58.10\\
t500m100r3-2.pl.json & 500 & 100 & Solution & 48.01 & 99993 & 41211.00 & 58.79\\
t500m100r3-20.pl.json & 500 & 100 & Solution & 42.84 & 100000 & 37671.00 & 62.33\\
t500m100r3-3.pl.json & 500 & 100 & Solution & 88.52 & 99990 & 35084.00 & 64.91\\
t500m100r3-4.pl.json & 500 & 100 & Solution & 47.32 & 99997 & 32016.00 & 67.98\\
t500m100r3-5.pl.json & 500 & 100 & Solution & 47.73 & 100000 & 38298.00 & 61.70\\
t500m100r3-6.pl.json & 500 & 100 & Solution & 44.73 & 99979 & 46003.00 & 53.99\\
t500m100r3-7.pl.json & 500 & 100 & Solution & 47.33 & 99998 & 37262.00 & 62.74\\
t500m100r3-8.pl.json & 500 & 100 & Solution & 46.35 & 99996 & 40827.00 & 59.17\\
t500m100r3-9.pl.json & 500 & 100 & Solution & 44.71 & 99998 & 44625.00 & 55.37\\
t500m100r5-1.pl.json & 500 & 100 & Solution & 34.78 & 99995 & 34446.00 & 65.55\\
t500m100r5-10.pl.json & 500 & 100 & Solution & 37.04 & 100000 & 27639.00 & 72.36\\
t500m100r5-11.pl.json & 500 & 100 & Solution & 35.72 & 100000 & 35280.00 & 64.72\\
t500m100r5-12.pl.json & 500 & 100 & Solution & 31.74 & 99993 & 37187.00 & 62.81\\
t500m100r5-13.pl.json & 500 & 100 & Solution & 39.48 & 99998 & 43728.00 & 56.27\\
t500m100r5-14.pl.json & 500 & 100 & Solution & 37.64 & 100000 & 38862.00 & 61.14\\
t500m100r5-15.pl.json & 500 & 100 & Solution & 37.56 & 99991 & 36096.00 & 63.90\\
t500m100r5-16.pl.json & 500 & 100 & Solution & 41.37 & 100000 & 34669.00 & 65.33\\
t500m100r5-17.pl.json & 500 & 100 & Solution & 41.38 & 99999 & 37944.00 & 62.06\\
t500m100r5-18.pl.json & 500 & 100 & Solution & 41.58 & 99996 & 42744.00 & 57.25\\
t500m100r5-19.pl.json & 500 & 100 & Solution & 42.75 & 99997 & 44310.00 & 55.69\\
t500m100r5-2.pl.json & 500 & 100 & Solution & 41.26 & 99999 & 40905.00 & 59.09\\
t500m100r5-20.pl.json & 500 & 100 & Solution & 41.73 & 99999 & 38404.00 & 61.60\\
t500m100r5-3.pl.json & 500 & 100 & Solution & 43.39 & 99997 & 38651.00 & 61.35\\
t500m100r5-4.pl.json & 500 & 100 & Solution & 43.76 & 99991 & 30938.00 & 69.06\\
t500m100r5-5.pl.json & 500 & 100 & Solution & 38.67 & 100000 & 37915.00 & 62.09\\
t500m100r5-6.pl.json & 500 & 100 & Solution & 44.47 & 99984 & 40363.00 & 59.63\\
t500m100r5-7.pl.json & 500 & 100 & Solution & 43.93 & 99992 & 40749.00 & 59.25\\
t500m100r5-8.pl.json & 500 & 100 & Solution & 46.34 & 100000 & 37050.00 & 62.95\\
t500m100r5-9.pl.json & 500 & 100 & Solution & 44.35 & 99992 & 39160.00 & 60.84\\
t500m10r10-1.pl.json & 500 & 10 & Solution & 30.11 & 95746 & 42756.00 & 55.34\\
t500m10r10-10.pl.json & 500 & 10 & Solution & 30.08 & 95000 & 30745.00 & 67.64\\
t500m10r10-11.pl.json & 500 & 10 & Solution & 30.06 & 94588 & 42832.00 & 54.72\\
t500m10r10-12.pl.json & 500 & 10 & Solution & 30.09 & 93713 & 35908.00 & 61.68\\
t500m10r10-13.pl.json & 500 & 10 & Solution & 30.07 & 95952 & 38554.00 & 59.82\\
t500m10r10-14.pl.json & 500 & 10 & Solution & 30.06 & 94768 & 34152.00 & 63.96\\
t500m10r10-15.pl.json & 500 & 10 & Solution & 30.07 & 96018 & 32118.00 & 66.55\\
t500m10r10-16.pl.json & 500 & 10 & Solution & 30.07 & 94780 & 32243.00 & 65.98\\
t500m10r10-17.pl.json & 500 & 10 & Solution & 30.07 & 96565 & 32882.00 & 65.95\\
t500m10r10-18.pl.json & 500 & 10 & Solution & 30.08 & 94982 & 33101.00 & 65.15\\
t500m10r10-19.pl.json & 500 & 10 & Solution & 30.07 & 95235 & 40550.00 & 57.42\\
t500m10r10-2.pl.json & 500 & 10 & Solution & 30.06 & 93974 & 34094.00 & 63.72\\
t500m10r10-20.pl.json & 500 & 10 & Solution & 30.09 & 94572 & 36034.00 & 61.90\\
t500m10r10-3.pl.json & 500 & 10 & Solution & 30.06 & 95991 & 34790.00 & 63.76\\
t500m10r10-4.pl.json & 500 & 10 & Solution & 30.05 & 94949 & 40391.00 & 57.46\\
t500m10r10-5.pl.json & 500 & 10 & Solution & 30.09 & 96784 & 40910.00 & 57.73\\
t500m10r10-6.pl.json & 500 & 10 & Solution & 30.07 & 94288 & 31591.00 & 66.50\\
t500m10r10-7.pl.json & 500 & 10 & Solution & 30.06 & 96950 & 33091.00 & 65.87\\
t500m10r10-8.pl.json & 500 & 10 & Solution & 30.05 & 95149 & 37700.00 & 60.38\\
t500m10r10-9.pl.json & 500 & 10 & Solution & 30.07 & 93849 & 31331.00 & 66.62\\
t500m10r3-1.pl.json & 500 & 10 & Solution & 30.06 & 92705 & 38470.00 & 58.50\\
t500m10r3-10.pl.json & 500 & 10 & Solution & 30.06 & 96160 & 46481.00 & 51.66\\
t500m10r3-11.pl.json & 500 & 10 & Solution & 30.10 & 95135 & 37621.00 & 60.46\\
t500m10r3-12.pl.json & 500 & 10 & Solution & 30.06 & 93775 & 41276.00 & 55.98\\
t500m10r3-13.pl.json & 500 & 10 & Solution & 30.06 & 96699 & 36639.00 & 62.11\\
t500m10r3-14.pl.json & 500 & 10 & Solution & 30.09 & 95937 & 39052.00 & 59.29\\
t500m10r3-15.pl.json & 500 & 10 & Solution & 30.08 & 96302 & 40506.00 & 57.94\\
t500m10r3-16.pl.json & 500 & 10 & Solution & 30.08 & 94188 & 32654.00 & 65.33\\
t500m10r3-17.pl.json & 500 & 10 & Solution & 30.09 & 94889 & 48574.00 & 48.81\\
t500m10r3-18.pl.json & 500 & 10 & Solution & 30.07 & 94265 & 37386.00 & 60.34\\
t500m10r3-19.pl.json & 500 & 10 & Solution & 30.06 & 95914 & 49330.00 & 48.57\\
t500m10r3-2.pl.json & 500 & 10 & Solution & 30.16 & 97153 & 40595.00 & 58.22\\
t500m10r3-20.pl.json & 500 & 10 & Solution & 30.07 & 92943 & 46331.00 & 50.15\\
t500m10r3-3.pl.json & 500 & 10 & Solution & 30.08 & 94467 & 37399.00 & 60.41\\
t500m10r3-4.pl.json & 500 & 10 & Solution & 30.05 & 97560 & 48637.00 & 50.15\\
t500m10r3-5.pl.json & 500 & 10 & Solution & 30.07 & 94536 & 38945.00 & 58.80\\
t500m10r3-6.pl.json & 500 & 10 & Solution & 30.06 & 96686 & 39113.00 & 59.55\\
t500m10r3-7.pl.json & 500 & 10 & Solution & 30.09 & 96742 & 36212.00 & 62.57\\
t500m10r3-8.pl.json & 500 & 10 & Solution & 30.14 & 94423 & 42992.00 & 54.47\\
t500m10r3-9.pl.json & 500 & 10 & Solution & 30.06 & 94916 & 41201.00 & 56.59\\
t500m10r5-1.pl.json & 500 & 10 & Solution & 30.07 & 95693 & 38422.00 & 59.85\\
t500m10r5-10.pl.json & 500 & 10 & Solution & 30.10 & 96968 & 40616.00 & 58.11\\
t500m10r5-11.pl.json & 500 & 10 & Solution & 30.05 & 96445 & 43447.00 & 54.95\\
t500m10r5-12.pl.json & 500 & 10 & Solution & 30.06 & 96045 & 35447.00 & 63.09\\
t500m10r5-13.pl.json & 500 & 10 & Solution & 30.08 & 95556 & 41212.00 & 56.87\\
t500m10r5-14.pl.json & 500 & 10 & Solution & 30.05 & 95732 & 37546.00 & 60.78\\
t500m10r5-15.pl.json & 500 & 10 & Solution & 30.08 & 77582 & 36409.00 & 53.07\\
t500m10r5-16.pl.json & 500 & 10 & Solution & 30.09 & 94243 & 37966.00 & 59.71\\
t500m10r5-17.pl.json & 500 & 10 & Solution & 30.08 & 95414 & 41333.00 & 56.68\\
t500m10r5-18.pl.json & 500 & 10 & Solution & 30.07 & 95623 & 40205.00 & 57.95\\
t500m10r5-19.pl.json & 500 & 10 & Solution & 30.06 & 94847 & 38862.00 & 59.03\\
t500m10r5-2.pl.json & 500 & 10 & Solution & 30.08 & 95895 & 36135.00 & 62.32\\
t500m10r5-20.pl.json & 500 & 10 & Solution & 30.08 & 94987 & 42789.00 & 54.95\\
t500m10r5-3.pl.json & 500 & 10 & Solution & 30.08 & 94696 & 41375.00 & 56.31\\
t500m10r5-4.pl.json & 500 & 10 & Solution & 30.08 & 95774 & 34710.00 & 63.76\\
t500m10r5-5.pl.json & 500 & 10 & Solution & 30.06 & 95351 & 33781.00 & 64.57\\
t500m10r5-6.pl.json & 500 & 10 & Solution & 30.06 & 94254 & 41208.00 & 56.28\\
t500m10r5-7.pl.json & 500 & 10 & Solution & 30.08 & 71786 & 37543.00 & 47.70\\
t500m10r5-8.pl.json & 500 & 10 & Solution & 30.06 & 94893 & 40616.00 & 57.20\\
t500m10r5-9.pl.json & 500 & 10 & Solution & 30.06 & 93998 & 37557.00 & 60.04\\
t500m20r10-1.pl.json & 500 & 20 & Solution & 30.14 & 97697 & 35120.00 & 64.05\\
t500m20r10-10.pl.json & 500 & 20 & Solution & 30.11 & 97516 & 34269.00 & 64.86\\
t500m20r10-11.pl.json & 500 & 20 & Solution & 30.13 & 97580 & 33469.00 & 65.70\\
t500m20r10-12.pl.json & 500 & 20 & Solution & 30.13 & 95009 & 36943.00 & 61.12\\
t500m20r10-13.pl.json & 500 & 20 & Solution & 30.11 & 98196 & 36175.00 & 63.16\\
t500m20r10-14.pl.json & 500 & 20 & Solution & 30.19 & 94915 & 34601.00 & 63.55\\
t500m20r10-15.pl.json & 500 & 20 & Solution & 30.11 & 96944 & 32963.00 & 66.00\\
t500m20r10-16.pl.json & 500 & 20 & Solution & 30.25 & 95596 & 37875.00 & 60.38\\
t500m20r10-17.pl.json & 500 & 20 & Solution & 30.10 & 96973 & 34515.00 & 64.41\\
t500m20r10-18.pl.json & 500 & 20 & Solution & 30.14 & 97844 & 35137.00 & 64.09\\
t500m20r10-19.pl.json & 500 & 20 & Solution & 30.14 & 96900 & 37146.00 & 61.67\\
t500m20r10-2.pl.json & 500 & 20 & Solution & 30.12 & 95672 & 39857.00 & 58.34\\
t500m20r10-20.pl.json & 500 & 20 & Solution & 30.14 & 96470 & 35785.00 & 62.91\\
t500m20r10-3.pl.json & 500 & 20 & Solution & 30.15 & 95282 & 35332.00 & 62.92\\
t500m20r10-4.pl.json & 500 & 20 & Solution & 30.13 & 96463 & 30197.00 & 68.70\\
t500m20r10-5.pl.json & 500 & 20 & Solution & 30.16 & 97742 & 39933.00 & 59.14\\
t500m20r10-6.pl.json & 500 & 20 & Solution & 30.25 & 96682 & 33282.00 & 65.58\\
t500m20r10-7.pl.json & 500 & 20 & Solution & 30.10 & 95513 & 30485.00 & 68.08\\
t500m20r10-8.pl.json & 500 & 20 & Solution & 30.11 & 97048 & 37688.00 & 61.17\\
t500m20r10-9.pl.json & 500 & 20 & Solution & 30.17 & 95122 & 40863.00 & 57.04\\
t500m20r3-1.pl.json & 500 & 20 & Solution & 30.11 & 96331 & 36188.00 & 62.43\\
t500m20r3-10.pl.json & 500 & 20 & Solution & 30.13 & 95729 & 42859.00 & 55.23\\
t500m20r3-11.pl.json & 500 & 20 & Solution & 30.11 & 95560 & 38401.00 & 59.81\\
t500m20r3-12.pl.json & 500 & 20 & Solution & 30.12 & 95608 & 40309.00 & 57.84\\
t500m20r3-13.pl.json & 500 & 20 & Solution & 30.11 & 97160 & 33374.00 & 65.65\\
t500m20r3-14.pl.json & 500 & 20 & Solution & 30.10 & 47664 & 34978.00 & 26.62\\
t500m20r3-15.pl.json & 500 & 20 & Solution & 30.25 & 94244 & 37664.00 & 60.04\\
t500m20r3-16.pl.json & 500 & 20 & Solution & 30.13 & 95521 & 42848.00 & 55.14\\
t500m20r3-17.pl.json & 500 & 20 & Solution & 30.15 & 97072 & 39524.00 & 59.28\\
t500m20r3-18.pl.json & 500 & 20 & Solution & 30.11 & 95122 & 43126.00 & 54.66\\
t500m20r3-19.pl.json & 500 & 20 & Solution & 30.11 & 44926 & 37033.00 & 17.57\\
t500m20r3-2.pl.json & 500 & 20 & Solution & 30.12 & 96028 & 42127.00 & 56.13\\
t500m20r3-20.pl.json & 500 & 20 & Solution & 30.10 & 94804 & 45628.00 & 51.87\\
t500m20r3-3.pl.json & 500 & 20 & Solution & 30.25 & 97763 & 31170.00 & 68.12\\
t500m20r3-4.pl.json & 500 & 20 & Solution & 30.11 & 94497 & 43640.00 & 53.82\\
t500m20r3-5.pl.json & 500 & 20 & Solution & 30.15 & 96748 & 48397.00 & 49.98\\
t500m20r3-6.pl.json & 500 & 20 & Solution & 30.11 & 96780 & 35195.00 & 63.63\\
t500m20r3-7.pl.json & 500 & 20 & Solution & 30.12 & 96251 & 45611.00 & 52.61\\
t500m20r3-8.pl.json & 500 & 20 & Solution & 30.25 & 97074 & 44320.00 & 54.34\\
t500m20r3-9.pl.json & 500 & 20 & Solution & 30.11 & 95614 & 41018.00 & 57.10\\
t500m20r5-1.pl.json & 500 & 20 & Solution & 30.25 & 97130 & 35280.00 & 63.68\\
t500m20r5-10.pl.json & 500 & 20 & Solution & 30.11 & 96985 & 42735.00 & 55.94\\
t500m20r5-11.pl.json & 500 & 20 & Solution & 30.13 & 94840 & 33780.00 & 64.38\\
t500m20r5-12.pl.json & 500 & 20 & Solution & 30.11 & 94597 & 37117.00 & 60.76\\
t500m20r5-13.pl.json & 500 & 20 & Solution & 30.13 & 97220 & 39429.00 & 59.44\\
t500m20r5-14.pl.json & 500 & 20 & Solution & 30.11 & 96568 & 45311.00 & 53.08\\
t500m20r5-15.pl.json & 500 & 20 & Solution & 30.12 & 95130 & 38015.00 & 60.04\\
t500m20r5-16.pl.json & 500 & 20 & Solution & 30.15 & 94779 & 36087.00 & 61.93\\
t500m20r5-17.pl.json & 500 & 20 & Solution & 30.11 & 98195 & 41447.00 & 57.79\\
t500m20r5-18.pl.json & 500 & 20 & Solution & 30.13 & 94881 & 40783.00 & 57.02\\
t500m20r5-19.pl.json & 500 & 20 & Solution & 30.11 & 95921 & 40972.00 & 57.29\\
t500m20r5-2.pl.json & 500 & 20 & Solution & 30.12 & 95081 & 39591.00 & 58.36\\
t500m20r5-20.pl.json & 500 & 20 & Solution & 30.14 & 95319 & 37542.00 & 60.61\\
t500m20r5-3.pl.json & 500 & 20 & Solution & 30.14 & 95802 & 39647.00 & 58.62\\
t500m20r5-4.pl.json & 500 & 20 & Solution & 30.23 & 96496 & 40300.00 & 58.24\\
t500m20r5-5.pl.json & 500 & 20 & Solution & 30.16 & 96540 & 41014.00 & 57.52\\
t500m20r5-6.pl.json & 500 & 20 & Solution & 30.13 & 94656 & 41439.00 & 56.22\\
t500m20r5-7.pl.json & 500 & 20 & Solution & 30.14 & 96949 & 37547.00 & 61.27\\
t500m20r5-8.pl.json & 500 & 20 & Solution & 30.11 & 95312 & 41282.00 & 56.69\\
t500m20r5-9.pl.json & 500 & 20 & Solution & 30.20 & 95186 & 39384.00 & 58.62\\
t500m50r10-1.pl.json & 500 & 50 & Solution & 30.26 & 98574 & 39376.00 & 60.05\\
t500m50r10-10.pl.json & 500 & 50 & Solution & 30.53 & 97898 & 34844.00 & 64.41\\
t500m50r10-11.pl.json & 500 & 50 & Solution & 30.31 & 97432 & 39722.00 & 59.23\\
t500m50r10-12.pl.json & 500 & 50 & Solution & 30.27 & 97544 & 32276.00 & 66.91\\
t500m50r10-13.pl.json & 500 & 50 & Solution & 30.33 & 98339 & 29550.00 & 69.95\\
t500m50r10-14.pl.json & 500 & 50 & Solution & 34.32 & 97956 & 37824.00 & 61.39\\
t500m50r10-15.pl.json & 500 & 50 & Solution & 30.30 & 97960 & 33997.00 & 65.30\\
t500m50r10-16.pl.json & 500 & 50 & Solution & 30.55 & 99118 & 31567.00 & 68.15\\
t500m50r10-17.pl.json & 500 & 50 & Solution & 30.30 & 99163 & 28277.00 & 71.48\\
t500m50r10-18.pl.json & 500 & 50 & Solution & 30.41 & 98458 & 39127.00 & 60.26\\
t500m50r10-19.pl.json & 500 & 50 & Solution & 30.30 & 96340 & 38100.00 & 60.45\\
t500m50r10-2.pl.json & 500 & 50 & Solution & 34.68 & 97717 & 37318.00 & 61.81\\
t500m50r10-20.pl.json & 500 & 50 & Solution & 30.49 & 97088 & 32654.00 & 66.37\\
t500m50r10-3.pl.json & 500 & 50 & Solution & 30.25 & 97120 & 36737.00 & 62.17\\
t500m50r10-4.pl.json & 500 & 50 & Solution & 30.35 & 98732 & 36302.00 & 63.23\\
t500m50r10-5.pl.json & 500 & 50 & Solution & 34.52 & 98061 & 31982.00 & 67.39\\
t500m50r10-6.pl.json & 500 & 50 & Solution & 30.34 & 96562 & 28608.00 & 70.37\\
t500m50r10-7.pl.json & 500 & 50 & Solution & 30.55 & 96332 & 30074.00 & 68.78\\
t500m50r10-8.pl.json & 500 & 50 & Solution & 30.28 & 97888 & 39978.00 & 59.16\\
t500m50r10-9.pl.json & 500 & 50 & Solution & 30.33 & 96470 & 35216.00 & 63.50\\
t500m50r3-1.pl.json & 500 & 50 & Solution & 30.29 & 96953 & 43548.00 & 55.08\\
t500m50r3-10.pl.json & 500 & 50 & Solution & 34.27 & 96965 & 43200.00 & 55.45\\
t500m50r3-11.pl.json & 500 & 50 & Solution & 30.47 & 97740 & 40426.00 & 58.64\\
t500m50r3-12.pl.json & 500 & 50 & Solution & 30.28 & 97264 & 36948.00 & 62.01\\
t500m50r3-13.pl.json & 500 & 50 & Solution & 30.30 & 97299 & 38482.00 & 60.45\\
t500m50r3-14.pl.json & 500 & 50 & Solution & 30.26 & 95702 & 33747.00 & 64.74\\
t500m50r3-15.pl.json & 500 & 50 & Solution & 30.44 & 95916 & 39597.00 & 58.72\\
t500m50r3-16.pl.json & 500 & 50 & Solution & 34.46 & 97474 & 42361.00 & 56.54\\
t500m50r3-17.pl.json & 500 & 50 & Solution & 30.47 & 98815 & 36939.00 & 62.62\\
t500m50r3-18.pl.json & 500 & 50 & Solution & 30.33 & 97270 & 42601.00 & 56.20\\
t500m50r3-19.pl.json & 500 & 50 & Solution & 30.27 & 97126 & 34933.00 & 64.03\\
t500m50r3-2.pl.json & 500 & 50 & Solution & 34.38 & 97040 & 39261.00 & 59.54\\
t500m50r3-20.pl.json & 500 & 50 & Solution & 30.32 & 97582 & 41275.00 & 57.70\\
t500m50r3-3.pl.json & 500 & 50 & Solution & 34.67 & 98223 & 45600.00 & 53.58\\
t500m50r3-4.pl.json & 500 & 50 & Solution & 30.50 & 97899 & 43554.00 & 55.51\\
t500m50r3-5.pl.json & 500 & 50 & Solution & 30.33 & 98115 & 44963.00 & 54.17\\
t500m50r3-6.pl.json & 500 & 50 & Solution & 30.31 & 97331 & 38374.00 & 60.57\\
t500m50r3-7.pl.json & 500 & 50 & Solution & 30.30 & 96331 & 41410.00 & 57.01\\
t500m50r3-8.pl.json & 500 & 50 & Solution & 30.35 & 97725 & 46945.00 & 51.96\\
t500m50r3-9.pl.json & 500 & 50 & Solution & 30.28 & 97526 & 45689.00 & 53.15\\
t500m50r5-1.pl.json & 500 & 50 & Solution & 30.27 & 97780 & 43579.00 & 55.43\\
t500m50r5-10.pl.json & 500 & 50 & Solution & 30.31 & 97631 & 41522.00 & 57.47\\
t500m50r5-11.pl.json & 500 & 50 & Solution & 34.24 & 97571 & 40447.00 & 58.55\\
t500m50r5-12.pl.json & 500 & 50 & Solution & 30.34 & 97678 & 41246.00 & 57.77\\
t500m50r5-13.pl.json & 500 & 50 & Solution & 30.28 & 97497 & 37668.00 & 61.36\\
t500m50r5-14.pl.json & 500 & 50 & Solution & 30.29 & 99505 & 37897.00 & 61.91\\
t500m50r5-15.pl.json & 500 & 50 & Solution & 30.45 & 97382 & 44019.00 & 54.80\\
t500m50r5-16.pl.json & 500 & 50 & Solution & 30.28 & 97763 & 41798.00 & 57.25\\
t500m50r5-17.pl.json & 500 & 50 & Solution & 30.32 & 96938 & 36155.00 & 62.70\\
t500m50r5-18.pl.json & 500 & 50 & Solution & 30.26 & 97819 & 33100.00 & 66.16\\
t500m50r5-19.pl.json & 500 & 50 & Solution & 34.56 & 97278 & 36464.00 & 62.52\\
t500m50r5-2.pl.json & 500 & 50 & Solution & 30.29 & 96060 & 40840.00 & 57.48\\
t500m50r5-20.pl.json & 500 & 50 & Solution & 34.18 & 98245 & 41452.00 & 57.81\\
t500m50r5-3.pl.json & 500 & 50 & Solution & 30.49 & 99069 & 37737.00 & 61.91\\
t500m50r5-4.pl.json & 500 & 50 & Solution & 30.26 & 98094 & 33092.00 & 66.27\\
t500m50r5-5.pl.json & 500 & 50 & Solution & 30.35 & 97837 & 33529.00 & 65.73\\
t500m50r5-6.pl.json & 500 & 50 & Solution & 30.28 & 97882 & 39918.00 & 59.22\\
t500m50r5-7.pl.json & 500 & 50 & Solution & 30.30 & 97935 & 41726.00 & 57.39\\
t500m50r5-8.pl.json & 500 & 50 & Solution & 30.28 & 96977 & 34249.00 & 64.68\\
t500m50r5-9.pl.json & 500 & 50 & Solution & 30.27 & 96065 & 30499.00 & 68.25\\
t50m10r10-1.pl.json & 50 & 10 & Solution & 30.04 & 6499 & 5840.00 & 10.14\\
t50m10r10-10.pl.json & 50 & 10 & Optimal &  6.15 & 3396 & 3396.00 &  0.00\\
t50m10r10-11.pl.json & 50 & 10 & Optimal &  7.37 & 3398 & 3398.00 &  0.00\\
t50m10r10-12.pl.json & 50 & 10 & Solution & 30.04 & 7550 & 6544.00 & 13.32\\
t50m10r10-13.pl.json & 50 & 10 & Optimal & 16.73 & 5484 & 5484.00 &  0.00\\
t50m10r10-14.pl.json & 50 & 10 & Solution & 30.03 & 4666 & 3431.00 & 26.47\\
t50m10r10-15.pl.json & 50 & 10 & Solution & 30.03 & 6640 & 5903.00 & 11.10\\
t50m10r10-16.pl.json & 50 & 10 & Optimal & 21.47 & 4914 & 4914.00 &  0.00\\
t50m10r10-17.pl.json & 50 & 10 & Optimal &  0.60 & 2252 & 2252.00 &  0.00\\
t50m10r10-18.pl.json & 50 & 10 & Solution & 30.04 & 4034 & 3841.00 &  4.78\\
t50m10r10-19.pl.json & 50 & 10 & Solution & 30.04 & 4873 & 4532.00 &  7.00\\
t50m10r10-2.pl.json & 50 & 10 & Solution & 30.02 & 4148 & 3646.00 & 12.10\\
t50m10r10-20.pl.json & 50 & 10 & Optimal &  0.38 & 3158 & 3158.00 &  0.00\\
t50m10r10-3.pl.json & 50 & 10 & Solution & 30.04 & 4334 & 4190.00 &  3.32\\
t50m10r10-4.pl.json & 50 & 10 & Solution & 30.01 & 4259 & 3715.00 & 12.77\\
t50m10r10-5.pl.json & 50 & 10 & Optimal &  5.78 & 2211 & 2211.00 &  0.00\\
t50m10r10-6.pl.json & 50 & 10 & Solution & 30.04 & 5752 & 5457.00 &  5.13\\
t50m10r10-7.pl.json & 50 & 10 & Optimal & 10.99 & 3239 & 3239.00 &  0.00\\
t50m10r10-8.pl.json & 50 & 10 & Optimal &  0.80 & 2624 & 2624.00 &  0.00\\
t50m10r10-9.pl.json & 50 & 10 & Solution & 30.02 & 5109 & 5015.00 &  1.84\\
t50m10r3-1.pl.json & 50 & 10 & Optimal &  0.54 & 7067 & 7067.00 &  0.00\\
t50m10r3-10.pl.json & 50 & 10 & Optimal &  0.35 & 4504 & 4504.00 &  0.00\\
t50m10r3-11.pl.json & 50 & 10 & Solution & 30.03 & 3856 & 3811.00 &  1.17\\
t50m10r3-12.pl.json & 50 & 10 & Optimal &  0.35 & 3063 & 3063.00 &  0.00\\
t50m10r3-13.pl.json & 50 & 10 & Optimal &  0.22 & 5368 & 5368.00 &  0.00\\
t50m10r3-14.pl.json & 50 & 10 & Optimal &  0.22 & 5759 & 5759.00 &  0.00\\
t50m10r3-15.pl.json & 50 & 10 & Optimal &  1.89 & 6360 & 6360.00 &  0.00\\
t50m10r3-16.pl.json & 50 & 10 & Optimal &  0.54 & 7616 & 7616.00 &  0.00\\
t50m10r3-17.pl.json & 50 & 10 & Solution & 30.03 & 5429 & 5233.00 &  3.61\\
t50m10r3-18.pl.json & 50 & 10 & Optimal &  0.94 & 5186 & 5186.00 &  0.00\\
t50m10r3-19.pl.json & 50 & 10 & Optimal &  0.48 & 4197 & 4197.00 &  0.00\\
t50m10r3-2.pl.json & 50 & 10 & Optimal &  0.43 & 5680 & 5680.00 &  0.00\\
t50m10r3-20.pl.json & 50 & 10 & Optimal &  0.44 & 7792 & 7792.00 &  0.00\\
t50m10r3-3.pl.json & 50 & 10 & Optimal &  0.79 & 3752 & 3752.00 &  0.00\\
t50m10r3-4.pl.json & 50 & 10 & Optimal &  0.66 & 4942 & 4942.00 &  0.00\\
t50m10r3-5.pl.json & 50 & 10 & Optimal &  0.44 & 6159 & 6159.00 &  0.00\\
t50m10r3-6.pl.json & 50 & 10 & Optimal &  0.52 & 3804 & 3804.00 &  0.00\\
t50m10r3-7.pl.json & 50 & 10 & Optimal &  0.16 & 6186 & 6186.00 &  0.00\\
t50m10r3-8.pl.json & 50 & 10 & Optimal &  0.71 & 5142 & 5142.00 &  0.00\\
t50m10r3-9.pl.json & 50 & 10 & Solution & 30.02 & 7279 & 7191.00 &  1.21\\
t50m10r5-1.pl.json & 50 & 10 & Optimal &  0.68 & 5397 & 5397.00 &  0.00\\
t50m10r5-10.pl.json & 50 & 10 & Optimal &  0.26 & 4926 & 4926.00 &  0.00\\
t50m10r5-11.pl.json & 50 & 10 & Optimal &  0.54 & 3620 & 3620.00 &  0.00\\
t50m10r5-12.pl.json & 50 & 10 & Optimal &  0.26 & 5183 & 5183.00 &  0.00\\
t50m10r5-13.pl.json & 50 & 10 & Solution & 30.03 & 5716 & 5394.00 &  5.63\\
t50m10r5-14.pl.json & 50 & 10 & Optimal &  0.60 & 2828 & 2828.00 &  0.00\\
t50m10r5-15.pl.json & 50 & 10 & Solution & 30.01 & 6385 & 6283.00 &  1.60\\
t50m10r5-16.pl.json & 50 & 10 & Solution & 30.04 & 4548 & 3970.00 & 12.71\\
t50m10r5-17.pl.json & 50 & 10 & Optimal &  0.33 & 5129 & 5129.00 &  0.00\\
t50m10r5-18.pl.json & 50 & 10 & Solution & 30.02 & 5831 & 5303.00 &  9.06\\
t50m10r5-19.pl.json & 50 & 10 & Solution & 30.04 & 5552 & 5213.00 &  6.11\\
t50m10r5-2.pl.json & 50 & 10 & Optimal &  0.33 & 5153 & 5153.00 &  0.00\\
t50m10r5-20.pl.json & 50 & 10 & Optimal &  9.30 & 3900 & 3900.00 &  0.00\\
t50m10r5-3.pl.json & 50 & 10 & Solution & 30.03 & 4708 & 4667.00 &  0.87\\
t50m10r5-4.pl.json & 50 & 10 & Solution & 30.02 & 5551 & 4986.00 & 10.18\\
t50m10r5-5.pl.json & 50 & 10 & Optimal &  0.31 & 7451 & 7451.00 &  0.00\\
t50m10r5-6.pl.json & 50 & 10 & Optimal &  0.53 & 3781 & 3781.00 &  0.00\\
t50m10r5-7.pl.json & 50 & 10 & Optimal & 17.68 & 3323 & 3323.00 &  0.00\\
t50m10r5-8.pl.json & 50 & 10 & Solution & 30.02 & 5559 & 4986.00 & 10.31\\
t50m10r5-9.pl.json & 50 & 10 & Solution & 30.02 & 6385 & 6082.00 &  4.75\\
t50m20r10-1.pl.json & 50 & 20 & Solution & 30.04 & 5211 & 4457.00 & 14.47\\
t50m20r10-10.pl.json & 50 & 20 & Optimal &  0.59 & 7934 & 7934.00 &  0.00\\
t50m20r10-11.pl.json & 50 & 20 & Optimal & 21.37 & 5509 & 5509.00 &  0.00\\
t50m20r10-12.pl.json & 50 & 20 & Solution & 30.04 & 5023 & 4256.00 & 15.27\\
t50m20r10-13.pl.json & 50 & 20 & Optimal &  0.36 & 4143 & 4143.00 &  0.00\\
t50m20r10-14.pl.json & 50 & 20 & Optimal &  0.43 & 6048 & 6048.00 &  0.00\\
t50m20r10-15.pl.json & 50 & 20 & Solution & 30.03 & 5992 & 5301.00 & 11.53\\
t50m20r10-16.pl.json & 50 & 20 & Optimal &  0.66 & 5032 & 5032.00 &  0.00\\
t50m20r10-17.pl.json & 50 & 20 & Optimal &  0.40 & 4488 & 4488.00 &  0.00\\
t50m20r10-18.pl.json & 50 & 20 & Solution & 30.02 & 4848 & 4599.00 &  5.14\\
t50m20r10-19.pl.json & 50 & 20 & Solution & 30.03 & 5430 & 4555.00 & 16.11\\
t50m20r10-2.pl.json & 50 & 20 & Solution & 30.03 & 6192 & 5348.00 & 13.63\\
t50m20r10-20.pl.json & 50 & 20 & Solution & 30.03 & 6271 & 5680.00 &  9.42\\
t50m20r10-3.pl.json & 50 & 20 & Solution & 30.03 & 6582 & 6278.00 &  4.62\\
t50m20r10-4.pl.json & 50 & 20 & Solution & 30.03 & 5686 & 5160.00 &  9.25\\
t50m20r10-5.pl.json & 50 & 20 & Optimal &  0.37 & 3301 & 3301.00 &  0.00\\
t50m20r10-6.pl.json & 50 & 20 & Optimal & 20.69 & 4425 & 4425.00 &  0.00\\
t50m20r10-7.pl.json & 50 & 20 & Optimal &  1.52 & 3519 & 3519.00 &  0.00\\
t50m20r10-8.pl.json & 50 & 20 & Solution & 30.02 & 4630 & 4569.00 &  1.32\\
t50m20r10-9.pl.json & 50 & 20 & Solution & 30.05 & 5869 & 5303.00 &  9.64\\
t50m20r3-1.pl.json & 50 & 20 & Optimal &  0.26 & 3869 & 3869.00 &  0.00\\
t50m20r3-10.pl.json & 50 & 20 & Optimal &  0.33 & 3982 & 3982.00 &  0.00\\
t50m20r3-11.pl.json & 50 & 20 & Optimal &  0.38 & 4144 & 4144.00 &  0.00\\
t50m20r3-12.pl.json & 50 & 20 & Optimal &  0.39 & 2791 & 2791.00 &  0.00\\
t50m20r3-13.pl.json & 50 & 20 & Optimal &  1.02 & 6449 & 6449.00 &  0.00\\
t50m20r3-14.pl.json & 50 & 20 & Optimal &  0.36 & 4933 & 4933.00 &  0.00\\
t50m20r3-15.pl.json & 50 & 20 & Optimal & 22.87 & 2436 & 2436.00 &  0.00\\
t50m20r3-16.pl.json & 50 & 20 & Optimal &  0.19 & 5872 & 5872.00 &  0.00\\
t50m20r3-17.pl.json & 50 & 20 & Optimal &  0.54 & 6880 & 6880.00 &  0.00\\
t50m20r3-18.pl.json & 50 & 20 & Optimal &  0.32 & 2811 & 2811.00 &  0.00\\
t50m20r3-19.pl.json & 50 & 20 & Optimal &  0.44 & 3465 & 3465.00 &  0.00\\
t50m20r3-2.pl.json & 50 & 20 & Optimal &  0.38 & 5570 & 5570.00 &  0.00\\
t50m20r3-20.pl.json & 50 & 20 & Optimal &  0.44 & 6364 & 6364.00 &  0.00\\
t50m20r3-3.pl.json & 50 & 20 & Optimal &  0.51 & 3081 & 3081.00 &  0.00\\
t50m20r3-4.pl.json & 50 & 20 & Optimal &  0.43 & 3505 & 3505.00 &  0.00\\
t50m20r3-5.pl.json & 50 & 20 & Optimal &  0.44 & 2228 & 2228.00 &  0.00\\
t50m20r3-6.pl.json & 50 & 20 & Optimal &  0.82 & 5713 & 5713.00 &  0.00\\
t50m20r3-7.pl.json & 50 & 20 & Optimal &  0.74 & 3173 & 3173.00 &  0.00\\
t50m20r3-8.pl.json & 50 & 20 & Optimal & 14.05 & 3908 & 3908.00 &  0.00\\
t50m20r3-9.pl.json & 50 & 20 & Optimal &  0.44 & 4661 & 4661.00 &  0.00\\
t50m20r5-1.pl.json & 50 & 20 & Solution & 30.04 & 6273 & 5304.00 & 15.45\\
t50m20r5-10.pl.json & 50 & 20 & Optimal &  0.55 & 2328 & 2328.00 &  0.00\\
t50m20r5-11.pl.json & 50 & 20 & Optimal &  1.81 & 6403 & 6403.00 &  0.00\\
t50m20r5-12.pl.json & 50 & 20 & Optimal &  1.21 & 4281 & 4281.00 &  0.00\\
t50m20r5-13.pl.json & 50 & 20 & Optimal &  0.58 & 5754 & 5754.00 &  0.00\\
t50m20r5-14.pl.json & 50 & 20 & Solution & 30.03 & 6639 & 5359.00 & 19.28\\
t50m20r5-15.pl.json & 50 & 20 & Optimal &  0.41 & 3472 & 3472.00 &  0.00\\
t50m20r5-16.pl.json & 50 & 20 & Solution & 30.04 & 5934 & 5042.00 & 15.03\\
t50m20r5-17.pl.json & 50 & 20 & Optimal &  0.28 & 4745 & 4745.00 &  0.00\\
t50m20r5-18.pl.json & 50 & 20 & Optimal &  0.31 & 3147 & 3147.00 &  0.00\\
t50m20r5-19.pl.json & 50 & 20 & Optimal &  0.65 & 5960 & 5960.00 &  0.00\\
t50m20r5-2.pl.json & 50 & 20 & Solution & 30.03 & 5547 & 5417.00 &  2.34\\
t50m20r5-20.pl.json & 50 & 20 & Optimal &  0.30 & 3913 & 3913.00 &  0.00\\
t50m20r5-3.pl.json & 50 & 20 & Solution & 30.04 & 5598 & 4754.00 & 15.08\\
t50m20r5-4.pl.json & 50 & 20 & Solution & 30.03 & 5367 & 4465.00 & 16.81\\
t50m20r5-5.pl.json & 50 & 20 & Optimal &  1.74 & 3648 & 3648.00 &  0.00\\
t50m20r5-6.pl.json & 50 & 20 & Optimal &  0.38 & 5449 & 5449.00 &  0.00\\
t50m20r5-7.pl.json & 50 & 20 & Solution & 30.04 & 4127 & 3794.00 &  8.07\\
t50m20r5-8.pl.json & 50 & 20 & Solution & 30.05 & 5003 & 4535.00 &  9.35\\
t50m20r5-9.pl.json & 50 & 20 & Optimal &  0.40 & 4022 & 4022.00 &  0.00\\
\end{longtable}



\clearpage
\chapter{J\&J Hybrid Flexible Flowshop with Transportation Times}

\section{Without Transportation Times}

\subsection{Results for CPOptimizer}

\begin{longtable}{lrrlrrrr}
\caption{Results for Factory Design (CPO) (25 Instances)}\\\toprule
Name & \shortstack{Nr\\Jobs} & \shortstack{Nr\\Machines} & Status & Time & Makespan & Bound & \shortstack{Gap\\Percent}\\ \midrule
\endhead
\bottomrule
\endfoot
instance20 1.txt & 20 & 80 & Optimal &  0.60 & 55 & 55.00 &  0.00\\
instance20 10.txt & 20 & 80 & Optimal &  0.46 & 53 & 53.00 &  0.00\\
instance20 11.txt & 20 & 80 & Optimal &  0.45 & 61 & 61.00 &  0.00\\
instance20 12.txt & 20 & 80 & Optimal &  0.48 & 56 & 56.00 &  0.00\\
instance20 13.txt & 20 & 80 & Optimal &  0.45 & 61 & 61.00 &  0.00\\
instance20 14.txt & 20 & 80 & Solution & 300.06 & 54 & 53.00 &  1.85\\
instance20 15.txt & 20 & 80 & Solution & 300.04 & 49 & 45.00 &  8.16\\
instance20 16.txt & 20 & 80 & Optimal &  0.42 & 52 & 52.00 &  0.00\\
instance20 17.txt & 20 & 80 & Optimal &  0.41 & 53 & 53.00 &  0.00\\
instance20 18.txt & 20 & 80 & Optimal &  0.38 & 56 & 56.00 &  0.00\\
instance20 19.txt & 20 & 80 & Optimal &  0.44 & 56 & 56.00 &  0.00\\
instance20 2.txt & 20 & 80 & Optimal &  0.57 & 53 & 53.00 &  0.00\\
instance20 20.txt & 20 & 80 & Optimal &  0.43 & 55 & 55.00 &  0.00\\
instance20 21.txt & 20 & 80 & Optimal &  0.46 & 58 & 58.00 &  0.00\\
instance20 22.txt & 20 & 80 & Optimal & 10.34 & 56 & 56.00 &  0.00\\
instance20 23.txt & 20 & 80 & Optimal &  0.46 & 47 & 47.00 &  0.00\\
instance20 24.txt & 20 & 80 & Optimal &  0.47 & 59 & 59.00 &  0.00\\
instance20 25.txt & 20 & 80 & Optimal &  0.49 & 59 & 59.00 &  0.00\\
instance20 3.txt & 20 & 80 & Optimal &  0.53 & 51 & 51.00 &  0.00\\
instance20 4.txt & 20 & 80 & Solution & 300.03 & 50 & 49.00 &  2.00\\
instance20 5.txt & 20 & 80 & Solution & 300.09 & 56 & 55.00 &  1.79\\
instance20 6.txt & 20 & 80 & Solution & 300.06 & 56 & 52.00 &  7.14\\
instance20 7.txt & 20 & 80 & Optimal &  0.53 & 61 & 61.00 &  0.00\\
instance20 8.txt & 20 & 80 & Solution & 300.07 & 52 & 51.00 &  1.92\\
instance20 9.txt & 20 & 80 & Optimal &  0.53 & 65 & 65.00 &  0.00\\
\end{longtable}



\subsection{Results for CPSat}

\begin{longtable}{lrrlrrrr}
\caption{Results for Factory Design (CPSat) (25 Instances)}\\\toprule
Name & \shortstack{Nr\\Jobs} & \shortstack{Nr\\Machines} & Status & Time & Makespan & Bound & \shortstack{Gap\\Percent}\\ \midrule
\endhead
\bottomrule
\endfoot
instance20 1.txt & 20 & 80 & Optimal &  2.47 & 55 & 55.00 &  0.00\\
instance20 10.txt & 20 & 80 & Optimal &  2.61 & 53 & 53.00 &  0.00\\
instance20 11.txt & 20 & 80 & Optimal &  1.74 & 61 & 61.00 &  0.00\\
instance20 12.txt & 20 & 80 & Optimal &  2.48 & 56 & 56.00 &  0.00\\
instance20 13.txt & 20 & 80 & Optimal &  1.47 & 61 & 61.00 &  0.00\\
instance20 14.txt & 20 & 80 & Optimal & 52.64 & 54 & 54.00 &  0.00\\
instance20 15.txt & 20 & 80 & Optimal & 10.57 & 49 & 49.00 &  0.00\\
instance20 16.txt & 20 & 80 & Optimal &  2.97 & 52 & 52.00 &  0.00\\
instance20 17.txt & 20 & 80 & Optimal &  4.02 & 53 & 53.00 &  0.00\\
instance20 18.txt & 20 & 80 & Optimal &  1.89 & 56 & 56.00 &  0.00\\
instance20 19.txt & 20 & 80 & Optimal &  2.43 & 56 & 56.00 &  0.00\\
instance20 2.txt & 20 & 80 & Optimal &  4.55 & 53 & 53.00 &  0.00\\
instance20 20.txt & 20 & 80 & Optimal &  4.03 & 55 & 55.00 &  0.00\\
instance20 21.txt & 20 & 80 & Optimal &  2.20 & 58 & 58.00 &  0.00\\
instance20 22.txt & 20 & 80 & Optimal & 56.65 & 56 & 56.00 &  0.00\\
instance20 23.txt & 20 & 80 & Optimal &  4.08 & 47 & 47.00 &  0.00\\
instance20 24.txt & 20 & 80 & Optimal &  2.76 & 59 & 59.00 &  0.00\\
instance20 25.txt & 20 & 80 & Optimal &  3.67 & 59 & 59.00 &  0.00\\
instance20 3.txt & 20 & 80 & Optimal &  2.44 & 51 & 51.00 &  0.00\\
instance20 4.txt & 20 & 80 & Solution & 300.15 & 50 & 49.00 &  2.00\\
instance20 5.txt & 20 & 80 & Optimal & 18.43 & 56 & 56.00 &  0.00\\
instance20 6.txt & 20 & 80 & Optimal & 42.43 & 56 & 56.00 &  0.00\\
instance20 7.txt & 20 & 80 & Optimal &  3.56 & 61 & 61.00 &  0.00\\
instance20 8.txt & 20 & 80 & Optimal & 27.78 & 52 & 52.00 &  0.00\\
instance20 9.txt & 20 & 80 & Optimal &  3.64 & 65 & 65.00 &  0.00\\
\end{longtable}



\clearpage
\chapter{RCPSP SingleMode}

The detailed result tables for the individual RCPSP instances show that many instances are solved with either solver in less than a second, but that there are a few families of problems which are more difficult to solve. For problem type J30, the sets 13, 29, and 45 are such examples, they are still solved to optimality, but the time required is larger. For bigger problem instances, the solvers are not able to find and prove the optimal solutions for such families, for example sets 9, 13 , 25, 29, 41, 45 for j60. It would be interesting to understand this better, and see which generator settings make these instance families more difficult to solve.

\section{Size J30}
\subsection{CPO}
\begin{longtable}{lrrlrrrr}
\caption{Results for RCPSP J30 (CPO) (480 Instances)}\\\toprule
Name & \shortstack{Nr\\Jobs} & \shortstack{Nr\\Machines} & Status & Time & Makespan & Bound & \shortstack{Gap\\Percent}\\ \midrule
\endhead
\bottomrule
\endfoot
j3010 1.json & 1 & 0 & Optimal &  0.12 & 42 & 42.00 &  0.00\\
j3010 10.json & 1 & 0 & Optimal &  0.11 & 41 & 41.00 &  0.00\\
j3010 2.json & 1 & 0 & Optimal &  0.20 & 56 & 56.00 &  0.00\\
j3010 3.json & 1 & 0 & Optimal &  0.09 & 62 & 62.00 &  0.00\\
j3010 4.json & 1 & 0 & Optimal &  0.10 & 58 & 58.00 &  0.00\\
j3010 5.json & 1 & 0 & Optimal &  0.03 & 41 & 41.00 &  0.00\\
j3010 6.json & 1 & 0 & Optimal &  0.10 & 44 & 44.00 &  0.00\\
j3010 7.json & 1 & 0 & Optimal &  0.02 & 49 & 49.00 &  0.00\\
j3010 8.json & 1 & 0 & Optimal &  0.10 & 54 & 54.00 &  0.00\\
j3010 9.json & 1 & 0 & Optimal &  0.03 & 49 & 49.00 &  0.00\\
j3011 1.json & 1 & 0 & Optimal &  0.02 & 54 & 54.00 &  0.00\\
j3011 10.json & 1 & 0 & Optimal &  0.02 & 38 & 38.00 &  0.00\\
j3011 2.json & 1 & 0 & Optimal &  0.02 & 56 & 56.00 &  0.00\\
j3011 3.json & 1 & 0 & Optimal &  0.02 & 81 & 81.00 &  0.00\\
j3011 4.json & 1 & 0 & Optimal &  0.03 & 63 & 63.00 &  0.00\\
j3011 5.json & 1 & 0 & Optimal &  0.11 & 49 & 49.00 &  0.00\\
j3011 6.json & 1 & 0 & Optimal &  0.02 & 44 & 44.00 &  0.00\\
j3011 7.json & 1 & 0 & Optimal &  0.02 & 36 & 36.00 &  0.00\\
j3011 8.json & 1 & 0 & Optimal &  0.03 & 62 & 62.00 &  0.00\\
j3011 9.json & 1 & 0 & Optimal &  0.02 & 67 & 67.00 &  0.00\\
j3012 1.json & 1 & 0 & Optimal &  0.02 & 47 & 47.00 &  0.00\\
j3012 10.json & 1 & 0 & Optimal &  0.02 & 57 & 57.00 &  0.00\\
j3012 2.json & 1 & 0 & Optimal &  0.02 & 46 & 46.00 &  0.00\\
j3012 3.json & 1 & 0 & Optimal &  0.02 & 37 & 37.00 &  0.00\\
j3012 4.json & 1 & 0 & Optimal &  0.02 & 63 & 63.00 &  0.00\\
j3012 5.json & 1 & 0 & Optimal &  0.02 & 47 & 47.00 &  0.00\\
j3012 6.json & 1 & 0 & Optimal &  0.02 & 53 & 53.00 &  0.00\\
j3012 7.json & 1 & 0 & Optimal &  0.02 & 55 & 55.00 &  0.00\\
j3012 8.json & 1 & 0 & Optimal &  0.02 & 35 & 35.00 &  0.00\\
j3012 9.json & 1 & 0 & Optimal &  0.03 & 52 & 52.00 &  0.00\\
j3013 1.json & 1 & 0 & Optimal &  7.46 & 58 & 58.00 &  0.00\\
j3013 10.json & 1 & 0 & Optimal &  1.82 & 64 & 64.00 &  0.00\\
j3013 2.json & 1 & 0 & Optimal & 29.16 & 62 & 62.00 &  0.00\\
j3013 3.json & 1 & 0 & Optimal &  5.16 & 76 & 76.00 &  0.00\\
j3013 4.json & 1 & 0 & Optimal &  2.17 & 72 & 72.00 &  0.00\\
j3013 5.json & 1 & 0 & Optimal & 11.34 & 67 & 67.00 &  0.00\\
j3013 6.json & 1 & 0 & Optimal & 17.89 & 64 & 64.00 &  0.00\\
j3013 7.json & 1 & 0 & Optimal &  4.13 & 77 & 77.00 &  0.00\\
j3013 8.json & 1 & 0 & Optimal & 11.54 & 106 & 106.00 &  0.00\\
j3013 9.json & 1 & 0 & Optimal &  1.04 & 71 & 69.00 &  2.82\\
j3014 1.json & 1 & 0 & Optimal &  0.18 & 50 & 50.00 &  0.00\\
j3014 10.json & 1 & 0 & Optimal &  0.04 & 61 & 61.00 &  0.00\\
j3014 2.json & 1 & 0 & Optimal &  0.65 & 53 & 53.00 &  0.00\\
j3014 3.json & 1 & 0 & Optimal &  0.08 & 58 & 58.00 &  0.00\\
j3014 4.json & 1 & 0 & Optimal &  0.46 & 50 & 50.00 &  0.00\\
j3014 5.json & 1 & 0 & Optimal &  0.03 & 52 & 52.00 &  0.00\\
j3014 6.json & 1 & 0 & Optimal &  0.02 & 35 & 35.00 &  0.00\\
j3014 7.json & 1 & 0 & Optimal &  0.47 & 50 & 50.00 &  0.00\\
j3014 8.json & 1 & 0 & Optimal &  0.02 & 54 & 54.00 &  0.00\\
j3014 9.json & 1 & 0 & Optimal &  0.27 & 46 & 46.00 &  0.00\\
j3015 1.json & 1 & 0 & Optimal &  0.02 & 46 & 46.00 &  0.00\\
j3015 10.json & 1 & 0 & Optimal &  0.02 & 65 & 65.00 &  0.00\\
j3015 2.json & 1 & 0 & Optimal &  0.02 & 47 & 47.00 &  0.00\\
j3015 3.json & 1 & 0 & Optimal &  0.02 & 48 & 48.00 &  0.00\\
j3015 4.json & 1 & 0 & Optimal &  0.02 & 48 & 48.00 &  0.00\\
j3015 5.json & 1 & 0 & Optimal &  0.32 & 58 & 58.00 &  0.00\\
j3015 6.json & 1 & 0 & Optimal &  0.02 & 67 & 67.00 &  0.00\\
j3015 7.json & 1 & 0 & Optimal &  0.02 & 47 & 47.00 &  0.00\\
j3015 8.json & 1 & 0 & Optimal &  0.02 & 50 & 50.00 &  0.00\\
j3015 9.json & 1 & 0 & Optimal &  0.02 & 54 & 54.00 &  0.00\\
j3016 1.json & 1 & 0 & Optimal &  0.02 & 51 & 51.00 &  0.00\\
j3016 10.json & 1 & 0 & Optimal &  0.02 & 51 & 51.00 &  0.00\\
j3016 2.json & 1 & 0 & Optimal &  0.02 & 48 & 48.00 &  0.00\\
j3016 3.json & 1 & 0 & Optimal &  0.02 & 36 & 36.00 &  0.00\\
j3016 4.json & 1 & 0 & Optimal &  0.02 & 47 & 47.00 &  0.00\\
j3016 5.json & 1 & 0 & Optimal &  0.02 & 51 & 51.00 &  0.00\\
j3016 6.json & 1 & 0 & Optimal &  0.02 & 51 & 51.00 &  0.00\\
j3016 7.json & 1 & 0 & Optimal &  0.02 & 34 & 34.00 &  0.00\\
j3016 8.json & 1 & 0 & Optimal &  0.03 & 44 & 44.00 &  0.00\\
j3016 9.json & 1 & 0 & Optimal &  0.02 & 44 & 44.00 &  0.00\\
j3017 1.json & 1 & 0 & Optimal &  0.08 & 64 & 64.00 &  0.00\\
j3017 10.json & 1 & 0 & Optimal &  0.02 & 66 & 66.00 &  0.00\\
j3017 2.json & 1 & 0 & Optimal &  0.02 & 68 & 68.00 &  0.00\\
j3017 3.json & 1 & 0 & Optimal &  0.02 & 60 & 60.00 &  0.00\\
j3017 4.json & 1 & 0 & Optimal &  0.02 & 49 & 49.00 &  0.00\\
j3017 5.json & 1 & 0 & Optimal &  0.08 & 47 & 47.00 &  0.00\\
j3017 6.json & 1 & 0 & Optimal &  0.02 & 63 & 63.00 &  0.00\\
j3017 7.json & 1 & 0 & Optimal &  0.07 & 57 & 57.00 &  0.00\\
j3017 8.json & 1 & 0 & Optimal &  0.02 & 61 & 61.00 &  0.00\\
j3017 9.json & 1 & 0 & Optimal &  0.02 & 48 & 48.00 &  0.00\\
j3018 1.json & 1 & 0 & Optimal &  0.02 & 53 & 53.00 &  0.00\\
j3018 10.json & 1 & 0 & Optimal &  0.03 & 49 & 49.00 &  0.00\\
j3018 2.json & 1 & 0 & Optimal &  0.02 & 55 & 55.00 &  0.00\\
j3018 3.json & 1 & 0 & Optimal &  0.02 & 56 & 56.00 &  0.00\\
j3018 4.json & 1 & 0 & Optimal &  0.02 & 70 & 70.00 &  0.00\\
j3018 5.json & 1 & 0 & Optimal &  0.02 & 52 & 52.00 &  0.00\\
j3018 6.json & 1 & 0 & Optimal &  0.02 & 62 & 62.00 &  0.00\\
j3018 7.json & 1 & 0 & Optimal &  0.02 & 48 & 48.00 &  0.00\\
j3018 8.json & 1 & 0 & Optimal &  0.02 & 52 & 52.00 &  0.00\\
j3018 9.json & 1 & 0 & Optimal &  0.02 & 47 & 47.00 &  0.00\\
j3019 1.json & 1 & 0 & Optimal &  0.02 & 40 & 40.00 &  0.00\\
j3019 10.json & 1 & 0 & Optimal &  0.02 & 47 & 47.00 &  0.00\\
j3019 2.json & 1 & 0 & Optimal &  0.02 & 58 & 58.00 &  0.00\\
j3019 3.json & 1 & 0 & Optimal &  0.02 & 83 & 83.00 &  0.00\\
j3019 4.json & 1 & 0 & Optimal &  0.02 & 39 & 39.00 &  0.00\\
j3019 5.json & 1 & 0 & Optimal &  0.02 & 48 & 48.00 &  0.00\\
j3019 6.json & 1 & 0 & Optimal &  0.02 & 49 & 49.00 &  0.00\\
j3019 7.json & 1 & 0 & Optimal &  0.02 & 57 & 57.00 &  0.00\\
j3019 8.json & 1 & 0 & Optimal &  0.02 & 55 & 55.00 &  0.00\\
j3019 9.json & 1 & 0 & Optimal &  0.02 & 38 & 38.00 &  0.00\\
j301 1.json & 1 & 0 & Optimal &  0.02 & 43 & 43.00 &  0.00\\
j301 10.json & 1 & 0 & Optimal &  0.02 & 45 & 45.00 &  0.00\\
j301 2.json & 1 & 0 & Optimal &  0.02 & 47 & 47.00 &  0.00\\
j301 3.json & 1 & 0 & Optimal &  0.02 & 47 & 47.00 &  0.00\\
j301 4.json & 1 & 0 & Optimal &  0.02 & 62 & 62.00 &  0.00\\
j301 5.json & 1 & 0 & Optimal &  0.09 & 39 & 39.00 &  0.00\\
j301 6.json & 1 & 0 & Optimal &  0.07 & 48 & 48.00 &  0.00\\
j301 7.json & 1 & 0 & Optimal &  0.02 & 60 & 60.00 &  0.00\\
j301 8.json & 1 & 0 & Optimal &  0.02 & 53 & 53.00 &  0.00\\
j301 9.json & 1 & 0 & Optimal &  0.03 & 49 & 49.00 &  0.00\\
j3020 1.json & 1 & 0 & Optimal &  0.02 & 57 & 57.00 &  0.00\\
j3020 10.json & 1 & 0 & Optimal &  0.02 & 37 & 37.00 &  0.00\\
j3020 2.json & 1 & 0 & Optimal &  0.02 & 70 & 70.00 &  0.00\\
j3020 3.json & 1 & 0 & Optimal &  0.02 & 49 & 49.00 &  0.00\\
j3020 4.json & 1 & 0 & Optimal &  0.02 & 43 & 43.00 &  0.00\\
j3020 5.json & 1 & 0 & Optimal &  0.02 & 61 & 61.00 &  0.00\\
j3020 6.json & 1 & 0 & Optimal &  0.02 & 51 & 51.00 &  0.00\\
j3020 7.json & 1 & 0 & Optimal &  0.02 & 42 & 42.00 &  0.00\\
j3020 8.json & 1 & 0 & Optimal &  0.02 & 51 & 51.00 &  0.00\\
j3020 9.json & 1 & 0 & Optimal &  0.02 & 41 & 41.00 &  0.00\\
j3021 1.json & 1 & 0 & Optimal &  0.14 & 84 & 84.00 &  0.00\\
j3021 10.json & 1 & 0 & Optimal &  0.08 & 69 & 69.00 &  0.00\\
j3021 2.json & 1 & 0 & Optimal &  0.15 & 59 & 59.00 &  0.00\\
j3021 3.json & 1 & 0 & Optimal &  0.16 & 76 & 76.00 &  0.00\\
j3021 4.json & 1 & 0 & Optimal &  0.14 & 70 & 70.00 &  0.00\\
j3021 5.json & 1 & 0 & Optimal &  0.08 & 55 & 55.00 &  0.00\\
j3021 6.json & 1 & 0 & Optimal &  0.18 & 76 & 76.00 &  0.00\\
j3021 7.json & 1 & 0 & Optimal &  0.15 & 65 & 65.00 &  0.00\\
j3021 8.json & 1 & 0 & Optimal &  0.15 & 62 & 62.00 &  0.00\\
j3021 9.json & 1 & 0 & Optimal &  0.29 & 69 & 69.00 &  0.00\\
j3022 1.json & 1 & 0 & Optimal &  0.03 & 42 & 42.00 &  0.00\\
j3022 10.json & 1 & 0 & Optimal &  0.02 & 55 & 55.00 &  0.00\\
j3022 2.json & 1 & 0 & Optimal &  0.02 & 45 & 45.00 &  0.00\\
j3022 3.json & 1 & 0 & Optimal &  0.02 & 63 & 63.00 &  0.00\\
j3022 4.json & 1 & 0 & Optimal &  0.02 & 42 & 42.00 &  0.00\\
j3022 5.json & 1 & 0 & Optimal &  0.02 & 52 & 52.00 &  0.00\\
j3022 6.json & 1 & 0 & Optimal &  0.04 & 52 & 52.00 &  0.00\\
j3022 7.json & 1 & 0 & Optimal &  0.10 & 60 & 60.00 &  0.00\\
j3022 8.json & 1 & 0 & Optimal &  0.09 & 55 & 55.00 &  0.00\\
j3022 9.json & 1 & 0 & Optimal &  0.02 & 76 & 76.00 &  0.00\\
j3023 1.json & 1 & 0 & Optimal &  0.02 & 63 & 63.00 &  0.00\\
j3023 10.json & 1 & 0 & Optimal &  0.02 & 61 & 61.00 &  0.00\\
j3023 2.json & 1 & 0 & Optimal &  0.02 & 53 & 53.00 &  0.00\\
j3023 3.json & 1 & 0 & Optimal &  0.02 & 46 & 46.00 &  0.00\\
j3023 4.json & 1 & 0 & Optimal &  0.02 & 65 & 65.00 &  0.00\\
j3023 5.json & 1 & 0 & Optimal &  0.02 & 52 & 52.00 &  0.00\\
j3023 6.json & 1 & 0 & Optimal &  0.02 & 48 & 48.00 &  0.00\\
j3023 7.json & 1 & 0 & Optimal &  0.02 & 60 & 60.00 &  0.00\\
j3023 8.json & 1 & 0 & Optimal &  0.02 & 48 & 48.00 &  0.00\\
j3023 9.json & 1 & 0 & Optimal &  0.02 & 63 & 63.00 &  0.00\\
j3024 1.json & 1 & 0 & Optimal &  0.02 & 53 & 53.00 &  0.00\\
j3024 10.json & 1 & 0 & Optimal &  0.02 & 53 & 53.00 &  0.00\\
j3024 2.json & 1 & 0 & Optimal &  0.02 & 58 & 58.00 &  0.00\\
j3024 3.json & 1 & 0 & Optimal &  0.02 & 69 & 69.00 &  0.00\\
j3024 4.json & 1 & 0 & Optimal &  0.02 & 53 & 53.00 &  0.00\\
j3024 5.json & 1 & 0 & Optimal &  0.02 & 51 & 51.00 &  0.00\\
j3024 6.json & 1 & 0 & Optimal &  0.02 & 56 & 56.00 &  0.00\\
j3024 7.json & 1 & 0 & Optimal &  0.02 & 44 & 44.00 &  0.00\\
j3024 8.json & 1 & 0 & Optimal &  0.02 & 38 & 38.00 &  0.00\\
j3024 9.json & 1 & 0 & Optimal &  0.02 & 43 & 43.00 &  0.00\\
j3025 1.json & 1 & 0 & Optimal &  1.03 & 93 & 93.00 &  0.00\\
j3025 10.json & 1 & 0 & Optimal &  0.31 & 58 & 58.00 &  0.00\\
j3025 2.json & 1 & 0 & Optimal &  0.91 & 75 & 75.00 &  0.00\\
j3025 3.json & 1 & 0 & Optimal &  1.49 & 76 & 76.00 &  0.00\\
j3025 4.json & 1 & 0 & Optimal &  1.08 & 81 & 81.00 &  0.00\\
j3025 5.json & 1 & 0 & Optimal &  1.03 & 72 & 72.00 &  0.00\\
j3025 6.json & 1 & 0 & Optimal &  0.75 & 58 & 58.00 &  0.00\\
j3025 7.json & 1 & 0 & Optimal &  0.77 & 95 & 95.00 &  0.00\\
j3025 8.json & 1 & 0 & Optimal &  0.67 & 69 & 69.00 &  0.00\\
j3025 9.json & 1 & 0 & Optimal &  0.68 & 84 & 84.00 &  0.00\\
j3026 1.json & 1 & 0 & Optimal &  0.02 & 59 & 59.00 &  0.00\\
j3026 10.json & 1 & 0 & Optimal &  0.02 & 49 & 49.00 &  0.00\\
j3026 2.json & 1 & 0 & Optimal &  0.02 & 40 & 40.00 &  0.00\\
j3026 3.json & 1 & 0 & Optimal &  0.02 & 58 & 58.00 &  0.00\\
j3026 4.json & 1 & 0 & Optimal &  0.02 & 62 & 62.00 &  0.00\\
j3026 5.json & 1 & 0 & Optimal &  0.03 & 74 & 74.00 &  0.00\\
j3026 6.json & 1 & 0 & Optimal &  0.09 & 53 & 53.00 &  0.00\\
j3026 7.json & 1 & 0 & Optimal &  0.02 & 56 & 56.00 &  0.00\\
j3026 8.json & 1 & 0 & Optimal &  0.02 & 66 & 66.00 &  0.00\\
j3026 9.json & 1 & 0 & Optimal &  0.10 & 43 & 43.00 &  0.00\\
j3027 1.json & 1 & 0 & Optimal &  0.02 & 43 & 43.00 &  0.00\\
j3027 10.json & 1 & 0 & Optimal &  0.02 & 62 & 62.00 &  0.00\\
j3027 2.json & 1 & 0 & Optimal &  0.02 & 58 & 58.00 &  0.00\\
j3027 3.json & 1 & 0 & Optimal &  0.02 & 60 & 60.00 &  0.00\\
j3027 4.json & 1 & 0 & Optimal &  0.02 & 64 & 64.00 &  0.00\\
j3027 5.json & 1 & 0 & Optimal &  0.02 & 49 & 49.00 &  0.00\\
j3027 6.json & 1 & 0 & Optimal &  0.02 & 59 & 59.00 &  0.00\\
j3027 7.json & 1 & 0 & Optimal &  0.02 & 49 & 49.00 &  0.00\\
j3027 8.json & 1 & 0 & Optimal &  0.02 & 66 & 66.00 &  0.00\\
j3027 9.json & 1 & 0 & Optimal &  0.02 & 55 & 55.00 &  0.00\\
j3028 1.json & 1 & 0 & Optimal &  0.02 & 69 & 69.00 &  0.00\\
j3028 10.json & 1 & 0 & Optimal &  0.02 & 59 & 59.00 &  0.00\\
j3028 2.json & 1 & 0 & Optimal &  0.02 & 57 & 57.00 &  0.00\\
j3028 3.json & 1 & 0 & Optimal &  0.02 & 40 & 40.00 &  0.00\\
j3028 4.json & 1 & 0 & Optimal &  0.02 & 49 & 49.00 &  0.00\\
j3028 5.json & 1 & 0 & Optimal &  0.02 & 73 & 73.00 &  0.00\\
j3028 6.json & 1 & 0 & Optimal &  0.02 & 55 & 55.00 &  0.00\\
j3028 7.json & 1 & 0 & Optimal &  0.02 & 48 & 48.00 &  0.00\\
j3028 8.json & 1 & 0 & Optimal &  0.02 & 53 & 53.00 &  0.00\\
j3028 9.json & 1 & 0 & Optimal &  0.02 & 62 & 62.00 &  0.00\\
j3029 1.json & 1 & 0 & Optimal &  0.79 & 85 & 85.00 &  0.00\\
j3029 10.json & 1 & 0 & Optimal &  0.47 & 76 & 76.00 &  0.00\\
j3029 2.json & 1 & 0 & Optimal &  1.36 & 90 & 90.00 &  0.00\\
j3029 3.json & 1 & 0 & Optimal & 56.03 & 78 & 78.00 &  0.00\\
j3029 4.json & 1 & 0 & Optimal &  7.43 & 103 & 103.00 &  0.00\\
j3029 5.json & 1 & 0 & Optimal &  1.36 & 98 & 98.00 &  0.00\\
j3029 6.json & 1 & 0 & Optimal & 25.83 & 92 & 92.00 &  0.00\\
j3029 7.json & 1 & 0 & Optimal &  1.99 & 73 & 73.00 &  0.00\\
j3029 8.json & 1 & 0 & Optimal & 17.32 & 80 & 80.00 &  0.00\\
j3029 9.json & 1 & 0 & Optimal &  4.06 & 97 & 97.00 &  0.00\\
j302 1.json & 1 & 0 & Optimal &  0.02 & 38 & 38.00 &  0.00\\
j302 10.json & 1 & 0 & Optimal &  0.02 & 43 & 43.00 &  0.00\\
j302 2.json & 1 & 0 & Optimal &  0.02 & 51 & 51.00 &  0.00\\
j302 3.json & 1 & 0 & Optimal &  0.02 & 43 & 43.00 &  0.00\\
j302 4.json & 1 & 0 & Optimal &  0.02 & 43 & 43.00 &  0.00\\
j302 5.json & 1 & 0 & Optimal &  0.02 & 51 & 51.00 &  0.00\\
j302 6.json & 1 & 0 & Optimal &  0.02 & 47 & 47.00 &  0.00\\
j302 7.json & 1 & 0 & Optimal &  0.02 & 47 & 47.00 &  0.00\\
j302 8.json & 1 & 0 & Optimal &  0.02 & 54 & 54.00 &  0.00\\
j302 9.json & 1 & 0 & Optimal &  0.02 & 54 & 54.00 &  0.00\\
j3030 1.json & 1 & 0 & Optimal &  0.27 & 47 & 47.00 &  0.00\\
j3030 10.json & 1 & 0 & Optimal &  0.45 & 53 & 53.00 &  0.00\\
j3030 2.json & 1 & 0 & Optimal &  0.49 & 68 & 68.00 &  0.00\\
j3030 3.json & 1 & 0 & Optimal &  0.28 & 55 & 55.00 &  0.00\\
j3030 4.json & 1 & 0 & Optimal &  0.09 & 53 & 53.00 &  0.00\\
j3030 5.json & 1 & 0 & Optimal &  0.18 & 54 & 54.00 &  0.00\\
j3030 6.json & 1 & 0 & Optimal &  0.54 & 62 & 62.00 &  0.00\\
j3030 7.json & 1 & 0 & Optimal &  0.09 & 68 & 68.00 &  0.00\\
j3030 8.json & 1 & 0 & Optimal &  0.10 & 46 & 46.00 &  0.00\\
j3030 9.json & 1 & 0 & Optimal &  0.10 & 46 & 46.00 &  0.00\\
j3031 1.json & 1 & 0 & Optimal &  0.02 & 43 & 43.00 &  0.00\\
j3031 10.json & 1 & 0 & Optimal &  0.20 & 55 & 55.00 &  0.00\\
j3031 2.json & 1 & 0 & Optimal &  0.02 & 63 & 63.00 &  0.00\\
j3031 3.json & 1 & 0 & Optimal &  0.02 & 58 & 58.00 &  0.00\\
j3031 4.json & 1 & 0 & Optimal &  0.02 & 50 & 50.00 &  0.00\\
j3031 5.json & 1 & 0 & Optimal &  0.12 & 52 & 52.00 &  0.00\\
j3031 6.json & 1 & 0 & Optimal &  0.02 & 53 & 53.00 &  0.00\\
j3031 7.json & 1 & 0 & Optimal &  0.02 & 61 & 61.00 &  0.00\\
j3031 8.json & 1 & 0 & Optimal &  0.02 & 58 & 58.00 &  0.00\\
j3031 9.json & 1 & 0 & Optimal &  0.10 & 50 & 50.00 &  0.00\\
j3032 1.json & 1 & 0 & Optimal &  0.02 & 61 & 61.00 &  0.00\\
j3032 10.json & 1 & 0 & Optimal &  0.02 & 51 & 51.00 &  0.00\\
j3032 2.json & 1 & 0 & Optimal &  0.02 & 60 & 60.00 &  0.00\\
j3032 3.json & 1 & 0 & Optimal &  0.02 & 57 & 57.00 &  0.00\\
j3032 4.json & 1 & 0 & Optimal &  0.02 & 68 & 68.00 &  0.00\\
j3032 5.json & 1 & 0 & Optimal &  0.02 & 54 & 54.00 &  0.00\\
j3032 6.json & 1 & 0 & Optimal &  0.02 & 44 & 44.00 &  0.00\\
j3032 7.json & 1 & 0 & Optimal &  0.02 & 35 & 35.00 &  0.00\\
j3032 8.json & 1 & 0 & Optimal &  0.02 & 54 & 54.00 &  0.00\\
j3032 9.json & 1 & 0 & Optimal &  0.02 & 65 & 65.00 &  0.00\\
j3033 1.json & 1 & 0 & Optimal &  0.02 & 65 & 65.00 &  0.00\\
j3033 10.json & 1 & 0 & Optimal &  0.02 & 53 & 53.00 &  0.00\\
j3033 2.json & 1 & 0 & Optimal &  0.02 & 60 & 60.00 &  0.00\\
j3033 3.json & 1 & 0 & Optimal &  0.04 & 55 & 55.00 &  0.00\\
j3033 4.json & 1 & 0 & Optimal &  0.02 & 77 & 77.00 &  0.00\\
j3033 5.json & 1 & 0 & Optimal &  0.02 & 53 & 53.00 &  0.00\\
j3033 6.json & 1 & 0 & Optimal &  0.02 & 59 & 59.00 &  0.00\\
j3033 7.json & 1 & 0 & Optimal &  0.02 & 58 & 58.00 &  0.00\\
j3033 8.json & 1 & 0 & Optimal &  0.10 & 61 & 61.00 &  0.00\\
j3033 9.json & 1 & 0 & Optimal &  0.09 & 65 & 65.00 &  0.00\\
j3034 1.json & 1 & 0 & Optimal &  0.02 & 68 & 68.00 &  0.00\\
j3034 10.json & 1 & 0 & Optimal &  0.02 & 47 & 47.00 &  0.00\\
j3034 2.json & 1 & 0 & Optimal &  0.02 & 44 & 44.00 &  0.00\\
j3034 3.json & 1 & 0 & Optimal &  0.02 & 69 & 69.00 &  0.00\\
j3034 4.json & 1 & 0 & Optimal &  0.02 & 67 & 67.00 &  0.00\\
j3034 5.json & 1 & 0 & Optimal &  0.02 & 63 & 63.00 &  0.00\\
j3034 6.json & 1 & 0 & Optimal &  0.02 & 52 & 52.00 &  0.00\\
j3034 7.json & 1 & 0 & Optimal &  0.02 & 58 & 58.00 &  0.00\\
j3034 8.json & 1 & 0 & Optimal &  0.02 & 58 & 58.00 &  0.00\\
j3034 9.json & 1 & 0 & Optimal &  0.02 & 60 & 60.00 &  0.00\\
j3035 1.json & 1 & 0 & Optimal &  0.02 & 57 & 57.00 &  0.00\\
j3035 10.json & 1 & 0 & Optimal &  0.02 & 59 & 59.00 &  0.00\\
j3035 2.json & 1 & 0 & Optimal &  0.02 & 53 & 53.00 &  0.00\\
j3035 3.json & 1 & 0 & Optimal &  0.02 & 60 & 60.00 &  0.00\\
j3035 4.json & 1 & 0 & Optimal &  0.02 & 50 & 50.00 &  0.00\\
j3035 5.json & 1 & 0 & Optimal &  0.02 & 60 & 60.00 &  0.00\\
j3035 6.json & 1 & 0 & Optimal &  0.02 & 58 & 58.00 &  0.00\\
j3035 7.json & 1 & 0 & Optimal &  0.02 & 61 & 61.00 &  0.00\\
j3035 8.json & 1 & 0 & Optimal &  0.02 & 63 & 63.00 &  0.00\\
j3035 9.json & 1 & 0 & Optimal &  0.02 & 59 & 59.00 &  0.00\\
j3036 1.json & 1 & 0 & Optimal &  0.02 & 66 & 66.00 &  0.00\\
j3036 10.json & 1 & 0 & Optimal &  0.02 & 59 & 59.00 &  0.00\\
j3036 2.json & 1 & 0 & Optimal &  0.02 & 44 & 44.00 &  0.00\\
j3036 3.json & 1 & 0 & Optimal &  0.02 & 61 & 61.00 &  0.00\\
j3036 4.json & 1 & 0 & Optimal &  0.02 & 59 & 59.00 &  0.00\\
j3036 5.json & 1 & 0 & Optimal &  0.02 & 64 & 64.00 &  0.00\\
j3036 6.json & 1 & 0 & Optimal &  0.02 & 46 & 46.00 &  0.00\\
j3036 7.json & 1 & 0 & Optimal &  0.02 & 56 & 56.00 &  0.00\\
j3036 8.json & 1 & 0 & Optimal &  0.02 & 63 & 63.00 &  0.00\\
j3036 9.json & 1 & 0 & Optimal &  0.02 & 59 & 59.00 &  0.00\\
j3037 1.json & 1 & 0 & Optimal &  0.23 & 79 & 79.00 &  0.00\\
j3037 10.json & 1 & 0 & Optimal &  0.16 & 81 & 81.00 &  0.00\\
j3037 2.json & 1 & 0 & Optimal &  0.08 & 69 & 69.00 &  0.00\\
j3037 3.json & 1 & 0 & Optimal &  0.22 & 81 & 81.00 &  0.00\\
j3037 4.json & 1 & 0 & Optimal &  0.24 & 83 & 83.00 &  0.00\\
j3037 5.json & 1 & 0 & Optimal &  0.08 & 80 & 80.00 &  0.00\\
j3037 6.json & 1 & 0 & Optimal &  0.08 & 73 & 73.00 &  0.00\\
j3037 7.json & 1 & 0 & Optimal &  0.39 & 92 & 92.00 &  0.00\\
j3037 8.json & 1 & 0 & Optimal &  0.15 & 72 & 72.00 &  0.00\\
j3037 9.json & 1 & 0 & Optimal &  0.15 & 57 & 57.00 &  0.00\\
j3038 1.json & 1 & 0 & Optimal &  0.02 & 48 & 48.00 &  0.00\\
j3038 10.json & 1 & 0 & Optimal &  0.02 & 60 & 60.00 &  0.00\\
j3038 2.json & 1 & 0 & Optimal &  0.02 & 54 & 54.00 &  0.00\\
j3038 3.json & 1 & 0 & Optimal &  0.02 & 59 & 59.00 &  0.00\\
j3038 4.json & 1 & 0 & Optimal &  0.02 & 59 & 59.00 &  0.00\\
j3038 5.json & 1 & 0 & Optimal &  0.10 & 71 & 71.00 &  0.00\\
j3038 6.json & 1 & 0 & Optimal &  0.02 & 63 & 63.00 &  0.00\\
j3038 7.json & 1 & 0 & Optimal &  0.02 & 65 & 65.00 &  0.00\\
j3038 8.json & 1 & 0 & Optimal &  0.03 & 61 & 61.00 &  0.00\\
j3038 9.json & 1 & 0 & Optimal &  0.02 & 63 & 63.00 &  0.00\\
j3039 1.json & 1 & 0 & Optimal &  0.02 & 55 & 55.00 &  0.00\\
j3039 10.json & 1 & 0 & Optimal &  0.02 & 60 & 60.00 &  0.00\\
j3039 2.json & 1 & 0 & Optimal &  0.02 & 54 & 54.00 &  0.00\\
j3039 3.json & 1 & 0 & Optimal &  0.02 & 54 & 54.00 &  0.00\\
j3039 4.json & 1 & 0 & Optimal &  0.02 & 53 & 53.00 &  0.00\\
j3039 5.json & 1 & 0 & Optimal &  0.02 & 55 & 55.00 &  0.00\\
j3039 6.json & 1 & 0 & Optimal &  0.02 & 69 & 69.00 &  0.00\\
j3039 7.json & 1 & 0 & Optimal &  0.02 & 56 & 56.00 &  0.00\\
j3039 8.json & 1 & 0 & Optimal &  0.02 & 67 & 67.00 &  0.00\\
j3039 9.json & 1 & 0 & Optimal &  0.02 & 64 & 64.00 &  0.00\\
j303 1.json & 1 & 0 & Optimal &  0.02 & 72 & 72.00 &  0.00\\
j303 10.json & 1 & 0 & Optimal &  0.02 & 59 & 59.00 &  0.00\\
j303 2.json & 1 & 0 & Optimal &  0.02 & 40 & 40.00 &  0.00\\
j303 3.json & 1 & 0 & Optimal &  0.02 & 57 & 57.00 &  0.00\\
j303 4.json & 1 & 0 & Optimal &  0.02 & 98 & 98.00 &  0.00\\
j303 5.json & 1 & 0 & Optimal &  0.02 & 53 & 53.00 &  0.00\\
j303 6.json & 1 & 0 & Optimal &  0.02 & 54 & 54.00 &  0.00\\
j303 7.json & 1 & 0 & Optimal &  0.02 & 48 & 48.00 &  0.00\\
j303 8.json & 1 & 0 & Optimal &  0.02 & 54 & 54.00 &  0.00\\
j303 9.json & 1 & 0 & Optimal &  0.02 & 65 & 65.00 &  0.00\\
j3040 1.json & 1 & 0 & Optimal &  0.02 & 51 & 51.00 &  0.00\\
j3040 10.json & 1 & 0 & Optimal &  0.02 & 51 & 51.00 &  0.00\\
j3040 2.json & 1 & 0 & Optimal &  0.02 & 56 & 56.00 &  0.00\\
j3040 3.json & 1 & 0 & Optimal &  0.02 & 57 & 57.00 &  0.00\\
j3040 4.json & 1 & 0 & Optimal &  0.02 & 57 & 57.00 &  0.00\\
j3040 5.json & 1 & 0 & Optimal &  0.02 & 65 & 65.00 &  0.00\\
j3040 6.json & 1 & 0 & Optimal &  0.02 & 60 & 60.00 &  0.00\\
j3040 7.json & 1 & 0 & Optimal &  0.02 & 46 & 46.00 &  0.00\\
j3040 8.json & 1 & 0 & Optimal &  0.02 & 57 & 57.00 &  0.00\\
j3040 9.json & 1 & 0 & Optimal &  0.02 & 64 & 64.00 &  0.00\\
j3041 1.json & 1 & 0 & Optimal &  0.23 & 86 & 86.00 &  0.00\\
j3041 10.json & 1 & 0 & Optimal &  2.52 & 99 & 99.00 &  0.00\\
j3041 2.json & 1 & 0 & Optimal &  1.14 & 89 & 89.00 &  0.00\\
j3041 3.json & 1 & 0 & Optimal &  0.42 & 85 & 85.00 &  0.00\\
j3041 4.json & 1 & 0 & Optimal &  0.64 & 78 & 78.00 &  0.00\\
j3041 5.json & 1 & 0 & Optimal &  0.40 & 99 & 99.00 &  0.00\\
j3041 6.json & 1 & 0 & Optimal &  2.30 & 103 & 103.00 &  0.00\\
j3041 7.json & 1 & 0 & Optimal &  0.78 & 92 & 92.00 &  0.00\\
j3041 8.json & 1 & 0 & Optimal &  1.22 & 88 & 88.00 &  0.00\\
j3041 9.json & 1 & 0 & Optimal &  0.29 & 92 & 92.00 &  0.00\\
j3042 1.json & 1 & 0 & Optimal &  0.02 & 58 & 58.00 &  0.00\\
j3042 10.json & 1 & 0 & Optimal &  0.02 & 75 & 75.00 &  0.00\\
j3042 2.json & 1 & 0 & Optimal &  0.10 & 50 & 50.00 &  0.00\\
j3042 3.json & 1 & 0 & Optimal &  0.08 & 60 & 60.00 &  0.00\\
j3042 4.json & 1 & 0 & Optimal &  0.15 & 49 & 49.00 &  0.00\\
j3042 5.json & 1 & 0 & Optimal &  0.02 & 52 & 52.00 &  0.00\\
j3042 6.json & 1 & 0 & Optimal &  0.02 & 66 & 66.00 &  0.00\\
j3042 7.json & 1 & 0 & Optimal &  0.02 & 66 & 66.00 &  0.00\\
j3042 8.json & 1 & 0 & Optimal &  0.09 & 82 & 82.00 &  0.00\\
j3042 9.json & 1 & 0 & Optimal &  0.03 & 60 & 60.00 &  0.00\\
j3043 1.json & 1 & 0 & Optimal &  0.09 & 55 & 55.00 &  0.00\\
j3043 10.json & 1 & 0 & Optimal &  0.02 & 60 & 60.00 &  0.00\\
j3043 2.json & 1 & 0 & Optimal &  0.02 & 43 & 43.00 &  0.00\\
j3043 3.json & 1 & 0 & Optimal &  0.07 & 57 & 57.00 &  0.00\\
j3043 4.json & 1 & 0 & Optimal &  0.02 & 67 & 67.00 &  0.00\\
j3043 5.json & 1 & 0 & Optimal &  0.02 & 64 & 64.00 &  0.00\\
j3043 6.json & 1 & 0 & Optimal &  0.03 & 58 & 58.00 &  0.00\\
j3043 7.json & 1 & 0 & Optimal &  0.02 & 52 & 52.00 &  0.00\\
j3043 8.json & 1 & 0 & Optimal &  0.03 & 62 & 62.00 &  0.00\\
j3043 9.json & 1 & 0 & Optimal &  0.12 & 57 & 57.00 &  0.00\\
j3044 1.json & 1 & 0 & Optimal &  0.02 & 50 & 50.00 &  0.00\\
j3044 10.json & 1 & 0 & Optimal &  0.02 & 63 & 63.00 &  0.00\\
j3044 2.json & 1 & 0 & Optimal &  0.02 & 54 & 54.00 &  0.00\\
j3044 3.json & 1 & 0 & Optimal &  0.03 & 51 & 51.00 &  0.00\\
j3044 4.json & 1 & 0 & Optimal &  0.02 & 57 & 57.00 &  0.00\\
j3044 5.json & 1 & 0 & Optimal &  0.02 & 55 & 55.00 &  0.00\\
j3044 6.json & 1 & 0 & Optimal &  0.02 & 56 & 56.00 &  0.00\\
j3044 7.json & 1 & 0 & Optimal &  0.02 & 42 & 42.00 &  0.00\\
j3044 8.json & 1 & 0 & Optimal &  0.02 & 49 & 49.00 &  0.00\\
j3044 9.json & 1 & 0 & Optimal &  0.02 & 64 & 64.00 &  0.00\\
j3045 1.json & 1 & 0 & Optimal &  1.63 & 82 & 82.00 &  0.00\\
j3045 10.json & 1 & 0 & Optimal &  1.73 & 90 & 90.00 &  0.00\\
j3045 2.json & 1 & 0 & Optimal & 66.04 & 125 & 125.00 &  0.00\\
j3045 3.json & 1 & 0 & Optimal &  0.67 & 92 & 92.00 &  0.00\\
j3045 4.json & 1 & 0 & Optimal &  0.75 & 84 & 84.00 &  0.00\\
j3045 5.json & 1 & 0 & Optimal &  0.82 & 86 & 86.00 &  0.00\\
j3045 6.json & 1 & 0 & Optimal & 44.91 & 129 & 129.00 &  0.00\\
j3045 7.json & 1 & 0 & Optimal &  1.12 & 101 & 101.00 &  0.00\\
j3045 8.json & 1 & 0 & Optimal &  1.39 & 94 & 94.00 &  0.00\\
j3045 9.json & 1 & 0 & Optimal &  0.61 & 82 & 82.00 &  0.00\\
j3046 1.json & 1 & 0 & Optimal &  0.10 & 59 & 59.00 &  0.00\\
j3046 10.json & 1 & 0 & Optimal &  0.46 & 55 & 55.00 &  0.00\\
j3046 2.json & 1 & 0 & Optimal &  0.18 & 67 & 67.00 &  0.00\\
j3046 3.json & 1 & 0 & Optimal &  0.18 & 65 & 65.00 &  0.00\\
j3046 4.json & 1 & 0 & Optimal &  0.03 & 64 & 64.00 &  0.00\\
j3046 5.json & 1 & 0 & Optimal &  0.02 & 57 & 57.00 &  0.00\\
j3046 6.json & 1 & 0 & Optimal &  0.40 & 59 & 59.00 &  0.00\\
j3046 7.json & 1 & 0 & Optimal &  0.49 & 59 & 59.00 &  0.00\\
j3046 8.json & 1 & 0 & Optimal &  0.09 & 58 & 58.00 &  0.00\\
j3046 9.json & 1 & 0 & Optimal &  0.02 & 49 & 49.00 &  0.00\\
j3047 1.json & 1 & 0 & Optimal &  0.02 & 58 & 58.00 &  0.00\\
j3047 10.json & 1 & 0 & Optimal &  0.10 & 60 & 60.00 &  0.00\\
j3047 2.json & 1 & 0 & Optimal &  0.02 & 59 & 59.00 &  0.00\\
j3047 3.json & 1 & 0 & Optimal &  0.02 & 55 & 55.00 &  0.00\\
j3047 4.json & 1 & 0 & Optimal &  0.02 & 49 & 49.00 &  0.00\\
j3047 5.json & 1 & 0 & Optimal &  0.02 & 47 & 47.00 &  0.00\\
j3047 6.json & 1 & 0 & Optimal &  0.02 & 53 & 53.00 &  0.00\\
j3047 7.json & 1 & 0 & Optimal &  0.04 & 66 & 66.00 &  0.00\\
j3047 8.json & 1 & 0 & Optimal &  0.02 & 48 & 48.00 &  0.00\\
j3047 9.json & 1 & 0 & Optimal &  0.02 & 65 & 65.00 &  0.00\\
j3048 1.json & 1 & 0 & Optimal &  0.02 & 63 & 63.00 &  0.00\\
j3048 10.json & 1 & 0 & Optimal &  0.02 & 54 & 54.00 &  0.00\\
j3048 2.json & 1 & 0 & Optimal &  0.02 & 54 & 54.00 &  0.00\\
j3048 3.json & 1 & 0 & Optimal &  0.02 & 50 & 50.00 &  0.00\\
j3048 4.json & 1 & 0 & Optimal &  0.02 & 57 & 57.00 &  0.00\\
j3048 5.json & 1 & 0 & Optimal &  0.02 & 58 & 58.00 &  0.00\\
j3048 6.json & 1 & 0 & Optimal &  0.02 & 58 & 58.00 &  0.00\\
j3048 7.json & 1 & 0 & Optimal &  0.02 & 55 & 55.00 &  0.00\\
j3048 8.json & 1 & 0 & Optimal &  0.02 & 44 & 44.00 &  0.00\\
j3048 9.json & 1 & 0 & Optimal &  0.02 & 59 & 59.00 &  0.00\\
j304 1.json & 1 & 0 & Optimal &  0.02 & 49 & 49.00 &  0.00\\
j304 10.json & 1 & 0 & Optimal &  0.02 & 48 & 48.00 &  0.00\\
j304 2.json & 1 & 0 & Optimal &  0.02 & 60 & 60.00 &  0.00\\
j304 3.json & 1 & 0 & Optimal &  0.02 & 47 & 47.00 &  0.00\\
j304 4.json & 1 & 0 & Optimal &  0.02 & 57 & 57.00 &  0.00\\
j304 5.json & 1 & 0 & Optimal &  0.02 & 59 & 59.00 &  0.00\\
j304 6.json & 1 & 0 & Optimal &  0.02 & 45 & 45.00 &  0.00\\
j304 7.json & 1 & 0 & Optimal &  0.02 & 56 & 56.00 &  0.00\\
j304 8.json & 1 & 0 & Optimal &  0.02 & 55 & 55.00 &  0.00\\
j304 9.json & 1 & 0 & Optimal &  0.02 & 38 & 38.00 &  0.00\\
j305 1.json & 1 & 0 & Optimal &  0.10 & 53 & 53.00 &  0.00\\
j305 10.json & 1 & 0 & Optimal &  0.18 & 70 & 70.00 &  0.00\\
j305 2.json & 1 & 0 & Optimal &  0.23 & 82 & 82.00 &  0.00\\
j305 3.json & 1 & 0 & Optimal &  0.22 & 76 & 76.00 &  0.00\\
j305 4.json & 1 & 0 & Optimal &  0.25 & 63 & 63.00 &  0.00\\
j305 5.json & 1 & 0 & Optimal &  0.22 & 76 & 76.00 &  0.00\\
j305 6.json & 1 & 0 & Optimal &  0.14 & 64 & 64.00 &  0.00\\
j305 7.json & 1 & 0 & Optimal &  0.23 & 76 & 76.00 &  0.00\\
j305 8.json & 1 & 0 & Optimal &  0.23 & 67 & 67.00 &  0.00\\
j305 9.json & 1 & 0 & Optimal &  0.08 & 49 & 49.00 &  0.00\\
j306 1.json & 1 & 0 & Optimal &  0.09 & 59 & 59.00 &  0.00\\
j306 10.json & 1 & 0 & Optimal &  0.09 & 61 & 61.00 &  0.00\\
j306 2.json & 1 & 0 & Optimal &  0.03 & 51 & 51.00 &  0.00\\
j306 3.json & 1 & 0 & Optimal &  0.02 & 48 & 48.00 &  0.00\\
j306 4.json & 1 & 0 & Optimal &  0.11 & 42 & 42.00 &  0.00\\
j306 5.json & 1 & 0 & Optimal &  0.09 & 67 & 67.00 &  0.00\\
j306 6.json & 1 & 0 & Optimal &  0.02 & 37 & 37.00 &  0.00\\
j306 7.json & 1 & 0 & Optimal &  0.02 & 46 & 46.00 &  0.00\\
j306 8.json & 1 & 0 & Optimal &  0.02 & 39 & 39.00 &  0.00\\
j306 9.json & 1 & 0 & Optimal &  0.02 & 51 & 51.00 &  0.00\\
j307 1.json & 1 & 0 & Optimal &  0.02 & 55 & 55.00 &  0.00\\
j307 10.json & 1 & 0 & Optimal &  0.03 & 49 & 49.00 &  0.00\\
j307 2.json & 1 & 0 & Optimal &  0.02 & 42 & 42.00 &  0.00\\
j307 3.json & 1 & 0 & Optimal &  0.02 & 42 & 42.00 &  0.00\\
j307 4.json & 1 & 0 & Optimal &  0.02 & 44 & 44.00 &  0.00\\
j307 5.json & 1 & 0 & Optimal &  0.02 & 44 & 44.00 &  0.00\\
j307 6.json & 1 & 0 & Optimal &  0.02 & 35 & 35.00 &  0.00\\
j307 7.json & 1 & 0 & Optimal &  0.02 & 50 & 50.00 &  0.00\\
j307 8.json & 1 & 0 & Optimal &  0.02 & 44 & 44.00 &  0.00\\
j307 9.json & 1 & 0 & Optimal &  0.02 & 60 & 60.00 &  0.00\\
j308 1.json & 1 & 0 & Optimal &  0.02 & 44 & 44.00 &  0.00\\
j308 10.json & 1 & 0 & Optimal &  0.02 & 67 & 67.00 &  0.00\\
j308 2.json & 1 & 0 & Optimal &  0.02 & 51 & 51.00 &  0.00\\
j308 3.json & 1 & 0 & Optimal &  0.02 & 53 & 53.00 &  0.00\\
j308 4.json & 1 & 0 & Optimal &  0.02 & 48 & 48.00 &  0.00\\
j308 5.json & 1 & 0 & Optimal &  0.02 & 58 & 58.00 &  0.00\\
j308 6.json & 1 & 0 & Optimal &  0.02 & 47 & 47.00 &  0.00\\
j308 7.json & 1 & 0 & Optimal &  0.02 & 41 & 41.00 &  0.00\\
j308 8.json & 1 & 0 & Optimal &  0.02 & 51 & 51.00 &  0.00\\
j308 9.json & 1 & 0 & Optimal &  0.02 & 39 & 39.00 &  0.00\\
j309 1.json & 1 & 0 & Optimal &  0.92 & 83 & 83.00 &  0.00\\
j309 10.json & 1 & 0 & Optimal &  1.21 & 88 & 88.00 &  0.00\\
j309 2.json & 1 & 0 & Optimal & 14.64 & 92 & 92.00 &  0.00\\
j309 3.json & 1 & 0 & Optimal &  0.40 & 68 & 68.00 &  0.00\\
j309 4.json & 1 & 0 & Optimal &  0.38 & 71 & 71.00 &  0.00\\
j309 5.json & 1 & 0 & Optimal &  0.23 & 70 & 70.00 &  0.00\\
j309 6.json & 1 & 0 & Optimal &  0.48 & 59 & 59.00 &  0.00\\
j309 7.json & 1 & 0 & Optimal &  0.75 & 63 & 63.00 &  0.00\\
j309 8.json & 1 & 0 & Optimal &  0.53 & 91 & 91.00 &  0.00\\
j309 9.json & 1 & 0 & Optimal &  0.85 & 63 & 63.00 &  0.00\\
\end{longtable}



\subsection{CPSat}
\input{resultsrcpspj30CPSat}

\section{Size J60}
\subsection{CPO}
\begin{longtable}{lrrlrrrr}
\caption{Results for RCPSP J60 (CPO) (480 Instances)}\\\toprule
Name & \shortstack{Nr\\Jobs} & \shortstack{Nr\\Machines} & Status & Time & Makespan & Bound & \shortstack{Gap\\Percent}\\ \midrule
\endhead
\bottomrule
\endfoot
j6010 1.json & 1 & 0 & Optimal &  0.12 & 85 & 85.00 &  0.00\\
j6010 10.json & 1 & 0 & Optimal &  0.03 & 73 & 73.00 &  0.00\\
j6010 2.json & 1 & 0 & Optimal &  0.02 & 62 & 62.00 &  0.00\\
j6010 3.json & 1 & 0 & Optimal &  0.03 & 72 & 72.00 &  0.00\\
j6010 4.json & 1 & 0 & Optimal &  0.02 & 80 & 80.00 &  0.00\\
j6010 5.json & 1 & 0 & Optimal &  0.02 & 79 & 79.00 &  0.00\\
j6010 6.json & 1 & 0 & Optimal &  0.02 & 67 & 67.00 &  0.00\\
j6010 7.json & 1 & 0 & Optimal &  0.04 & 69 & 69.00 &  0.00\\
j6010 8.json & 1 & 0 & Optimal &  0.03 & 65 & 65.00 &  0.00\\
j6010 9.json & 1 & 0 & Optimal &  0.05 & 73 & 73.00 &  0.00\\
j6011 1.json & 1 & 0 & Optimal &  0.02 & 71 & 71.00 &  0.00\\
j6011 10.json & 1 & 0 & Optimal &  0.03 & 58 & 58.00 &  0.00\\
j6011 2.json & 1 & 0 & Optimal &  0.02 & 61 & 61.00 &  0.00\\
j6011 3.json & 1 & 0 & Optimal &  0.02 & 76 & 76.00 &  0.00\\
j6011 4.json & 1 & 0 & Optimal &  0.02 & 69 & 69.00 &  0.00\\
j6011 5.json & 1 & 0 & Optimal &  0.02 & 65 & 65.00 &  0.00\\
j6011 6.json & 1 & 0 & Optimal &  0.02 & 70 & 70.00 &  0.00\\
j6011 7.json & 1 & 0 & Optimal &  0.02 & 70 & 70.00 &  0.00\\
j6011 8.json & 1 & 0 & Optimal &  0.02 & 69 & 69.00 &  0.00\\
j6011 9.json & 1 & 0 & Optimal &  0.02 & 62 & 62.00 &  0.00\\
j6012 1.json & 1 & 0 & Optimal &  0.02 & 59 & 59.00 &  0.00\\
j6012 10.json & 1 & 0 & Optimal &  0.02 & 79 & 79.00 &  0.00\\
j6012 2.json & 1 & 0 & Optimal &  0.02 & 58 & 58.00 &  0.00\\
j6012 3.json & 1 & 0 & Optimal &  0.02 & 75 & 75.00 &  0.00\\
j6012 4.json & 1 & 0 & Optimal &  0.02 & 69 & 69.00 &  0.00\\
j6012 5.json & 1 & 0 & Optimal &  0.02 & 63 & 63.00 &  0.00\\
j6012 6.json & 1 & 0 & Optimal &  0.02 & 54 & 54.00 &  0.00\\
j6012 7.json & 1 & 0 & Optimal &  0.02 & 71 & 71.00 &  0.00\\
j6012 8.json & 1 & 0 & Optimal &  0.02 & 60 & 60.00 &  0.00\\
j6012 9.json & 1 & 0 & Optimal &  0.03 & 59 & 59.00 &  0.00\\
j6013 1.json & 1 & 0 & Solution & 600.01 & 114 & 105.00 &  7.89\\
j6013 10.json & 1 & 0 & Solution & 600.01 & 117 & 114.00 &  2.56\\
j6013 2.json & 1 & 0 & Solution & 600.01 & 108 & 103.00 &  4.63\\
j6013 3.json & 1 & 0 & Solution & 600.01 & 88 & 84.00 &  4.55\\
j6013 4.json & 1 & 0 & Solution & 600.01 & 105 & 98.00 &  6.67\\
j6013 5.json & 1 & 0 & Solution & 600.01 & 98 & 93.00 &  5.10\\
j6013 6.json & 1 & 0 & Solution & 600.01 & 95 & 91.00 &  4.21\\
j6013 7.json & 1 & 0 & Solution & 600.01 & 89 & 83.00 &  6.74\\
j6013 8.json & 1 & 0 & Solution & 600.01 & 123 & 112.00 &  8.94\\
j6013 9.json & 1 & 0 & Solution & 600.01 & 103 & 97.00 &  5.83\\
j6014 1.json & 1 & 0 & Optimal & 33.24 & 61 & 61.00 &  0.00\\
j6014 10.json & 1 & 0 & Optimal & 56.92 & 72 & 72.00 &  0.00\\
j6014 2.json & 1 & 0 & Optimal &  0.03 & 65 & 65.00 &  0.00\\
j6014 3.json & 1 & 0 & Optimal & 351.54 & 61 & 61.00 &  0.00\\
j6014 4.json & 1 & 0 & Optimal &  0.18 & 65 & 65.00 &  0.00\\
j6014 5.json & 1 & 0 & Optimal &  0.03 & 59 & 59.00 &  0.00\\
j6014 6.json & 1 & 0 & Optimal &  0.03 & 65 & 65.00 &  0.00\\
j6014 7.json & 1 & 0 & Optimal &  0.02 & 69 & 69.00 &  0.00\\
j6014 8.json & 1 & 0 & Optimal &  0.03 & 88 & 88.00 &  0.00\\
j6014 9.json & 1 & 0 & Optimal &  0.02 & 61 & 61.00 &  0.00\\
j6015 1.json & 1 & 0 & Optimal &  0.03 & 84 & 84.00 &  0.00\\
j6015 10.json & 1 & 0 & Optimal &  0.02 & 61 & 61.00 &  0.00\\
j6015 2.json & 1 & 0 & Optimal &  0.03 & 89 & 89.00 &  0.00\\
j6015 3.json & 1 & 0 & Optimal &  0.03 & 72 & 72.00 &  0.00\\
j6015 4.json & 1 & 0 & Optimal &  0.03 & 75 & 75.00 &  0.00\\
j6015 5.json & 1 & 0 & Optimal &  0.03 & 70 & 70.00 &  0.00\\
j6015 6.json & 1 & 0 & Optimal &  0.02 & 76 & 76.00 &  0.00\\
j6015 7.json & 1 & 0 & Optimal &  0.02 & 64 & 64.00 &  0.00\\
j6015 8.json & 1 & 0 & Optimal &  0.02 & 79 & 79.00 &  0.00\\
j6015 9.json & 1 & 0 & Optimal &  0.03 & 72 & 72.00 &  0.00\\
j6016 1.json & 1 & 0 & Optimal &  0.03 & 64 & 64.00 &  0.00\\
j6016 10.json & 1 & 0 & Optimal &  0.02 & 68 & 68.00 &  0.00\\
j6016 2.json & 1 & 0 & Optimal &  0.02 & 64 & 64.00 &  0.00\\
j6016 3.json & 1 & 0 & Optimal &  0.02 & 53 & 53.00 &  0.00\\
j6016 4.json & 1 & 0 & Optimal &  0.02 & 60 & 60.00 &  0.00\\
j6016 5.json & 1 & 0 & Optimal &  0.02 & 66 & 66.00 &  0.00\\
j6016 6.json & 1 & 0 & Optimal &  0.03 & 66 & 66.00 &  0.00\\
j6016 7.json & 1 & 0 & Optimal &  0.02 & 82 & 82.00 &  0.00\\
j6016 8.json & 1 & 0 & Optimal &  0.03 & 68 & 68.00 &  0.00\\
j6016 9.json & 1 & 0 & Optimal &  0.02 & 54 & 54.00 &  0.00\\
j6017 1.json & 1 & 0 & Optimal &  0.07 & 86 & 86.00 &  0.00\\
j6017 10.json & 1 & 0 & Optimal &  0.02 & 72 & 72.00 &  0.00\\
j6017 2.json & 1 & 0 & Optimal &  0.07 & 69 & 69.00 &  0.00\\
j6017 3.json & 1 & 0 & Optimal &  0.02 & 89 & 89.00 &  0.00\\
j6017 4.json & 1 & 0 & Optimal &  0.02 & 71 & 71.00 &  0.00\\
j6017 5.json & 1 & 0 & Optimal &  0.06 & 59 & 59.00 &  0.00\\
j6017 6.json & 1 & 0 & Optimal &  0.06 & 69 & 69.00 &  0.00\\
j6017 7.json & 1 & 0 & Optimal &  0.02 & 83 & 83.00 &  0.00\\
j6017 8.json & 1 & 0 & Optimal &  0.19 & 85 & 85.00 &  0.00\\
j6017 9.json & 1 & 0 & Optimal &  0.02 & 76 & 76.00 &  0.00\\
j6018 1.json & 1 & 0 & Optimal &  0.02 & 81 & 81.00 &  0.00\\
j6018 10.json & 1 & 0 & Optimal &  0.02 & 97 & 97.00 &  0.00\\
j6018 2.json & 1 & 0 & Optimal &  0.02 & 69 & 69.00 &  0.00\\
j6018 3.json & 1 & 0 & Optimal &  0.02 & 77 & 77.00 &  0.00\\
j6018 4.json & 1 & 0 & Optimal &  0.02 & 71 & 71.00 &  0.00\\
j6018 5.json & 1 & 0 & Optimal &  0.02 & 80 & 80.00 &  0.00\\
j6018 6.json & 1 & 0 & Optimal &  0.02 & 61 & 61.00 &  0.00\\
j6018 7.json & 1 & 0 & Optimal &  0.03 & 93 & 93.00 &  0.00\\
j6018 8.json & 1 & 0 & Optimal &  0.02 & 78 & 78.00 &  0.00\\
j6018 9.json & 1 & 0 & Optimal &  0.02 & 69 & 69.00 &  0.00\\
j6019 1.json & 1 & 0 & Optimal &  0.02 & 62 & 62.00 &  0.00\\
j6019 10.json & 1 & 0 & Optimal &  0.02 & 78 & 78.00 &  0.00\\
j6019 2.json & 1 & 0 & Optimal &  0.02 & 83 & 83.00 &  0.00\\
j6019 3.json & 1 & 0 & Optimal &  0.02 & 83 & 83.00 &  0.00\\
j6019 4.json & 1 & 0 & Optimal &  0.02 & 67 & 67.00 &  0.00\\
j6019 5.json & 1 & 0 & Optimal &  0.02 & 73 & 73.00 &  0.00\\
j6019 6.json & 1 & 0 & Optimal &  0.02 & 69 & 69.00 &  0.00\\
j6019 7.json & 1 & 0 & Optimal &  0.02 & 60 & 60.00 &  0.00\\
j6019 8.json & 1 & 0 & Optimal &  0.02 & 87 & 87.00 &  0.00\\
j6019 9.json & 1 & 0 & Optimal &  0.02 & 69 & 69.00 &  0.00\\
j601 1.json & 1 & 0 & Optimal &  0.02 & 77 & 77.00 &  0.00\\
j601 10.json & 1 & 0 & Optimal &  0.02 & 80 & 80.00 &  0.00\\
j601 2.json & 1 & 0 & Optimal &  0.02 & 68 & 68.00 &  0.00\\
j601 3.json & 1 & 0 & Optimal &  0.02 & 68 & 68.00 &  0.00\\
j601 4.json & 1 & 0 & Optimal &  0.08 & 91 & 91.00 &  0.00\\
j601 5.json & 1 & 0 & Optimal &  0.04 & 73 & 73.00 &  0.00\\
j601 6.json & 1 & 0 & Optimal &  0.36 & 66 & 66.00 &  0.00\\
j601 7.json & 1 & 0 & Optimal &  0.21 & 72 & 72.00 &  0.00\\
j601 8.json & 1 & 0 & Optimal &  0.02 & 75 & 75.00 &  0.00\\
j601 9.json & 1 & 0 & Optimal &  0.03 & 85 & 85.00 &  0.00\\
j6020 1.json & 1 & 0 & Optimal &  0.02 & 60 & 60.00 &  0.00\\
j6020 10.json & 1 & 0 & Optimal &  0.02 & 70 & 70.00 &  0.00\\
j6020 2.json & 1 & 0 & Optimal &  0.02 & 78 & 78.00 &  0.00\\
j6020 3.json & 1 & 0 & Optimal &  0.02 & 69 & 69.00 &  0.00\\
j6020 4.json & 1 & 0 & Optimal &  0.02 & 86 & 86.00 &  0.00\\
j6020 5.json & 1 & 0 & Optimal &  0.02 & 71 & 71.00 &  0.00\\
j6020 6.json & 1 & 0 & Optimal &  0.02 & 97 & 97.00 &  0.00\\
j6020 7.json & 1 & 0 & Optimal &  0.02 & 74 & 74.00 &  0.00\\
j6020 8.json & 1 & 0 & Optimal &  0.02 & 65 & 65.00 &  0.00\\
j6020 9.json & 1 & 0 & Optimal &  0.02 & 74 & 74.00 &  0.00\\
j6021 1.json & 1 & 0 & Optimal &  1.13 & 103 & 103.00 &  0.00\\
j6021 10.json & 1 & 0 & Optimal &  0.85 & 80 & 80.00 &  0.00\\
j6021 2.json & 1 & 0 & Optimal &  0.60 & 108 & 108.00 &  0.00\\
j6021 3.json & 1 & 0 & Optimal &  0.51 & 87 & 87.00 &  0.00\\
j6021 4.json & 1 & 0 & Optimal &  2.82 & 95 & 95.00 &  0.00\\
j6021 5.json & 1 & 0 & Optimal &  3.61 & 89 & 89.00 &  0.00\\
j6021 6.json & 1 & 0 & Optimal &  1.18 & 84 & 84.00 &  0.00\\
j6021 7.json & 1 & 0 & Optimal &  1.55 & 103 & 103.00 &  0.00\\
j6021 8.json & 1 & 0 & Optimal &  1.57 & 110 & 110.00 &  0.00\\
j6021 9.json & 1 & 0 & Optimal &  9.88 & 89 & 89.00 &  0.00\\
j6022 1.json & 1 & 0 & Optimal &  0.02 & 64 & 64.00 &  0.00\\
j6022 10.json & 1 & 0 & Optimal &  0.03 & 70 & 70.00 &  0.00\\
j6022 2.json & 1 & 0 & Optimal &  0.02 & 83 & 83.00 &  0.00\\
j6022 3.json & 1 & 0 & Optimal &  0.03 & 70 & 70.00 &  0.00\\
j6022 4.json & 1 & 0 & Optimal &  0.43 & 73 & 73.00 &  0.00\\
j6022 5.json & 1 & 0 & Optimal &  0.02 & 76 & 76.00 &  0.00\\
j6022 6.json & 1 & 0 & Optimal &  0.07 & 79 & 79.00 &  0.00\\
j6022 7.json & 1 & 0 & Optimal &  0.03 & 69 & 69.00 &  0.00\\
j6022 8.json & 1 & 0 & Optimal &  0.02 & 59 & 59.00 &  0.00\\
j6022 9.json & 1 & 0 & Optimal &  0.02 & 65 & 65.00 &  0.00\\
j6023 1.json & 1 & 0 & Optimal &  0.02 & 75 & 75.00 &  0.00\\
j6023 10.json & 1 & 0 & Optimal &  0.02 & 68 & 68.00 &  0.00\\
j6023 2.json & 1 & 0 & Optimal &  0.02 & 69 & 69.00 &  0.00\\
j6023 3.json & 1 & 0 & Optimal &  0.02 & 78 & 78.00 &  0.00\\
j6023 4.json & 1 & 0 & Optimal &  0.02 & 83 & 83.00 &  0.00\\
j6023 5.json & 1 & 0 & Optimal &  0.02 & 72 & 72.00 &  0.00\\
j6023 6.json & 1 & 0 & Optimal &  0.02 & 81 & 81.00 &  0.00\\
j6023 7.json & 1 & 0 & Optimal &  0.02 & 60 & 60.00 &  0.00\\
j6023 8.json & 1 & 0 & Optimal &  0.02 & 72 & 72.00 &  0.00\\
j6023 9.json & 1 & 0 & Optimal &  0.02 & 64 & 64.00 &  0.00\\
j6024 1.json & 1 & 0 & Optimal &  0.02 & 65 & 65.00 &  0.00\\
j6024 10.json & 1 & 0 & Optimal &  0.02 & 66 & 66.00 &  0.00\\
j6024 2.json & 1 & 0 & Optimal &  0.02 & 55 & 55.00 &  0.00\\
j6024 3.json & 1 & 0 & Optimal &  0.02 & 67 & 67.00 &  0.00\\
j6024 4.json & 1 & 0 & Optimal &  0.02 & 78 & 78.00 &  0.00\\
j6024 5.json & 1 & 0 & Optimal &  0.02 & 76 & 76.00 &  0.00\\
j6024 6.json & 1 & 0 & Optimal &  0.02 & 75 & 75.00 &  0.00\\
j6024 7.json & 1 & 0 & Optimal &  0.02 & 68 & 68.00 &  0.00\\
j6024 8.json & 1 & 0 & Optimal &  0.03 & 81 & 81.00 &  0.00\\
j6024 9.json & 1 & 0 & Optimal &  0.02 & 80 & 80.00 &  0.00\\
j6025 1.json & 1 & 0 & Solution & 600.01 & 114 & 102.00 & 10.53\\
j6025 10.json & 1 & 0 & Solution & 600.01 & 108 & 102.00 &  5.56\\
j6025 2.json & 1 & 0 & Solution & 600.01 & 99 & 91.00 &  8.08\\
j6025 3.json & 1 & 0 & Optimal & 105.03 & 113 & 113.00 &  0.00\\
j6025 4.json & 1 & 0 & Solution & 600.01 & 108 & 103.00 &  4.63\\
j6025 5.json & 1 & 0 & Solution & 600.01 & 98 & 86.00 & 12.24\\
j6025 6.json & 1 & 0 & Solution & 600.01 & 112 & 99.00 & 11.61\\
j6025 7.json & 1 & 0 & Solution & 600.01 & 91 & 86.00 &  5.49\\
j6025 8.json & 1 & 0 & Solution & 600.01 & 99 & 92.00 &  7.07\\
j6025 9.json & 1 & 0 & Optimal & 103.66 & 99 & 99.00 &  0.00\\
j6026 1.json & 1 & 0 & Optimal &  0.03 & 80 & 80.00 &  0.00\\
j6026 10.json & 1 & 0 & Optimal &  0.02 & 85 & 85.00 &  0.00\\
j6026 2.json & 1 & 0 & Optimal &  0.19 & 66 & 66.00 &  0.00\\
j6026 3.json & 1 & 0 & Optimal &  2.41 & 76 & 76.00 &  0.00\\
j6026 4.json & 1 & 0 & Optimal &  0.95 & 67 & 67.00 &  0.00\\
j6026 5.json & 1 & 0 & Optimal &  0.02 & 61 & 61.00 &  0.00\\
j6026 6.json & 1 & 0 & Optimal &  0.52 & 74 & 74.00 &  0.00\\
j6026 7.json & 1 & 0 & Optimal &  0.03 & 72 & 72.00 &  0.00\\
j6026 8.json & 1 & 0 & Optimal &  0.02 & 89 & 89.00 &  0.00\\
j6026 9.json & 1 & 0 & Optimal &  1.30 & 65 & 65.00 &  0.00\\
j6027 1.json & 1 & 0 & Optimal &  0.03 & 96 & 96.00 &  0.00\\
j6027 10.json & 1 & 0 & Optimal &  0.02 & 57 & 57.00 &  0.00\\
j6027 2.json & 1 & 0 & Optimal &  0.02 & 74 & 74.00 &  0.00\\
j6027 3.json & 1 & 0 & Optimal &  0.03 & 76 & 76.00 &  0.00\\
j6027 4.json & 1 & 0 & Optimal &  0.02 & 60 & 60.00 &  0.00\\
j6027 5.json & 1 & 0 & Optimal &  0.02 & 78 & 78.00 &  0.00\\
j6027 6.json & 1 & 0 & Optimal &  0.02 & 64 & 64.00 &  0.00\\
j6027 7.json & 1 & 0 & Optimal &  0.02 & 83 & 83.00 &  0.00\\
j6027 8.json & 1 & 0 & Optimal &  0.03 & 88 & 88.00 &  0.00\\
j6027 9.json & 1 & 0 & Optimal &  0.02 & 76 & 76.00 &  0.00\\
j6028 1.json & 1 & 0 & Optimal &  0.02 & 92 & 92.00 &  0.00\\
j6028 10.json & 1 & 0 & Optimal &  0.03 & 74 & 74.00 &  0.00\\
j6028 2.json & 1 & 0 & Optimal &  0.02 & 64 & 64.00 &  0.00\\
j6028 3.json & 1 & 0 & Optimal &  0.02 & 72 & 72.00 &  0.00\\
j6028 4.json & 1 & 0 & Optimal &  0.02 & 84 & 84.00 &  0.00\\
j6028 5.json & 1 & 0 & Optimal &  0.02 & 71 & 71.00 &  0.00\\
j6028 6.json & 1 & 0 & Optimal &  0.02 & 89 & 89.00 &  0.00\\
j6028 7.json & 1 & 0 & Optimal &  0.02 & 75 & 75.00 &  0.00\\
j6028 8.json & 1 & 0 & Optimal &  0.02 & 62 & 62.00 &  0.00\\
j6028 9.json & 1 & 0 & Optimal &  0.02 & 74 & 74.00 &  0.00\\
j6029 1.json & 1 & 0 & Solution & 600.01 & 104 & 95.00 &  8.65\\
j6029 10.json & 1 & 0 & Solution & 600.01 & 120 & 112.00 &  6.67\\
j6029 2.json & 1 & 0 & Solution & 600.01 & 133 & 116.00 & 12.78\\
j6029 3.json & 1 & 0 & Solution & 600.01 & 122 & 114.00 &  6.56\\
j6029 4.json & 1 & 0 & Solution & 600.01 & 137 & 124.00 &  9.49\\
j6029 5.json & 1 & 0 & Solution & 600.02 & 110 & 101.00 &  8.18\\
j6029 6.json & 1 & 0 & Solution & 600.01 & 156 & 145.00 &  7.05\\
j6029 7.json & 1 & 0 & Solution & 600.01 & 125 & 115.00 &  8.00\\
j6029 8.json & 1 & 0 & Solution & 600.01 & 103 & 97.00 &  5.83\\
j6029 9.json & 1 & 0 & Solution & 600.01 & 113 & 97.00 & 14.16\\
j602 1.json & 1 & 0 & Optimal &  0.02 & 65 & 65.00 &  0.00\\
j602 10.json & 1 & 0 & Optimal &  0.02 & 69 & 69.00 &  0.00\\
j602 2.json & 1 & 0 & Optimal &  0.02 & 82 & 82.00 &  0.00\\
j602 3.json & 1 & 0 & Optimal &  0.02 & 78 & 78.00 &  0.00\\
j602 4.json & 1 & 0 & Optimal &  0.02 & 78 & 78.00 &  0.00\\
j602 5.json & 1 & 0 & Optimal &  0.02 & 54 & 54.00 &  0.00\\
j602 6.json & 1 & 0 & Optimal &  0.02 & 64 & 64.00 &  0.00\\
j602 7.json & 1 & 0 & Optimal &  0.02 & 53 & 53.00 &  0.00\\
j602 8.json & 1 & 0 & Optimal &  0.03 & 66 & 66.00 &  0.00\\
j602 9.json & 1 & 0 & Optimal &  0.02 & 65 & 65.00 &  0.00\\
j6030 1.json & 1 & 0 & Optimal &  0.04 & 70 & 70.00 &  0.00\\
j6030 10.json & 1 & 0 & Optimal & 67.18 & 86 & 86.00 &  0.00\\
j6030 2.json & 1 & 0 & Solution & 600.01 & 70 & 69.00 &  1.43\\
j6030 3.json & 1 & 0 & Optimal &  0.31 & 82 & 82.00 &  0.00\\
j6030 4.json & 1 & 0 & Optimal &  0.02 & 76 & 76.00 &  0.00\\
j6030 5.json & 1 & 0 & Optimal & 103.34 & 76 & 76.00 &  0.00\\
j6030 6.json & 1 & 0 & Optimal &  0.02 & 68 & 68.00 &  0.00\\
j6030 7.json & 1 & 0 & Optimal & 46.96 & 86 & 86.00 &  0.00\\
j6030 8.json & 1 & 0 & Optimal &  0.03 & 63 & 63.00 &  0.00\\
j6030 9.json & 1 & 0 & Optimal &  0.02 & 98 & 98.00 &  0.00\\
j6031 1.json & 1 & 0 & Optimal &  0.02 & 65 & 65.00 &  0.00\\
j6031 10.json & 1 & 0 & Optimal &  0.02 & 56 & 56.00 &  0.00\\
j6031 2.json & 1 & 0 & Optimal &  0.02 & 74 & 74.00 &  0.00\\
j6031 3.json & 1 & 0 & Optimal &  0.03 & 66 & 66.00 &  0.00\\
j6031 4.json & 1 & 0 & Optimal &  0.03 & 68 & 68.00 &  0.00\\
j6031 5.json & 1 & 0 & Optimal &  0.03 & 72 & 72.00 &  0.00\\
j6031 6.json & 1 & 0 & Optimal &  0.02 & 72 & 72.00 &  0.00\\
j6031 7.json & 1 & 0 & Optimal &  0.02 & 76 & 76.00 &  0.00\\
j6031 8.json & 1 & 0 & Optimal &  0.02 & 75 & 75.00 &  0.00\\
j6031 9.json & 1 & 0 & Optimal &  0.02 & 86 & 86.00 &  0.00\\
j6032 1.json & 1 & 0 & Optimal &  0.02 & 69 & 69.00 &  0.00\\
j6032 10.json & 1 & 0 & Optimal &  0.03 & 77 & 77.00 &  0.00\\
j6032 2.json & 1 & 0 & Optimal &  0.03 & 114 & 114.00 &  0.00\\
j6032 3.json & 1 & 0 & Optimal &  0.03 & 85 & 85.00 &  0.00\\
j6032 4.json & 1 & 0 & Optimal &  0.02 & 56 & 56.00 &  0.00\\
j6032 5.json & 1 & 0 & Optimal &  0.02 & 77 & 77.00 &  0.00\\
j6032 6.json & 1 & 0 & Optimal &  0.02 & 93 & 93.00 &  0.00\\
j6032 7.json & 1 & 0 & Optimal &  0.02 & 76 & 76.00 &  0.00\\
j6032 8.json & 1 & 0 & Optimal &  0.02 & 76 & 76.00 &  0.00\\
j6032 9.json & 1 & 0 & Optimal &  0.02 & 74 & 74.00 &  0.00\\
j6033 1.json & 1 & 0 & Optimal &  0.07 & 105 & 105.00 &  0.00\\
j6033 10.json & 1 & 0 & Optimal &  0.05 & 84 & 84.00 &  0.00\\
j6033 2.json & 1 & 0 & Optimal &  0.02 & 100 & 100.00 &  0.00\\
j6033 3.json & 1 & 0 & Optimal &  0.02 & 79 & 79.00 &  0.00\\
j6033 4.json & 1 & 0 & Optimal &  0.02 & 81 & 81.00 &  0.00\\
j6033 5.json & 1 & 0 & Optimal &  0.08 & 108 & 108.00 &  0.00\\
j6033 6.json & 1 & 0 & Optimal &  0.22 & 75 & 75.00 &  0.00\\
j6033 7.json & 1 & 0 & Optimal &  0.14 & 78 & 78.00 &  0.00\\
j6033 8.json & 1 & 0 & Optimal &  0.03 & 79 & 79.00 &  0.00\\
j6033 9.json & 1 & 0 & Optimal &  0.04 & 108 & 108.00 &  0.00\\
j6034 1.json & 1 & 0 & Optimal &  0.04 & 72 & 72.00 &  0.00\\
j6034 10.json & 1 & 0 & Optimal &  0.02 & 92 & 92.00 &  0.00\\
j6034 2.json & 1 & 0 & Optimal &  0.02 & 68 & 68.00 &  0.00\\
j6034 3.json & 1 & 0 & Optimal &  0.03 & 61 & 61.00 &  0.00\\
j6034 4.json & 1 & 0 & Optimal &  0.02 & 83 & 83.00 &  0.00\\
j6034 5.json & 1 & 0 & Optimal &  0.02 & 80 & 80.00 &  0.00\\
j6034 6.json & 1 & 0 & Optimal &  0.03 & 81 & 81.00 &  0.00\\
j6034 7.json & 1 & 0 & Optimal &  0.02 & 85 & 85.00 &  0.00\\
j6034 8.json & 1 & 0 & Optimal &  0.02 & 63 & 63.00 &  0.00\\
j6034 9.json & 1 & 0 & Optimal &  0.02 & 77 & 77.00 &  0.00\\
j6035 1.json & 1 & 0 & Optimal &  0.03 & 78 & 78.00 &  0.00\\
j6035 10.json & 1 & 0 & Optimal &  0.02 & 71 & 71.00 &  0.00\\
j6035 2.json & 1 & 0 & Optimal &  0.02 & 77 & 77.00 &  0.00\\
j6035 3.json & 1 & 0 & Optimal &  0.02 & 89 & 89.00 &  0.00\\
j6035 4.json & 1 & 0 & Optimal &  0.02 & 72 & 72.00 &  0.00\\
j6035 5.json & 1 & 0 & Optimal &  0.02 & 76 & 76.00 &  0.00\\
j6035 6.json & 1 & 0 & Optimal &  0.02 & 79 & 79.00 &  0.00\\
j6035 7.json & 1 & 0 & Optimal &  0.02 & 73 & 73.00 &  0.00\\
j6035 8.json & 1 & 0 & Optimal &  0.02 & 78 & 78.00 &  0.00\\
j6035 9.json & 1 & 0 & Optimal &  0.02 & 76 & 76.00 &  0.00\\
j6036 1.json & 1 & 0 & Optimal &  0.02 & 61 & 61.00 &  0.00\\
j6036 10.json & 1 & 0 & Optimal &  0.02 & 77 & 77.00 &  0.00\\
j6036 2.json & 1 & 0 & Optimal &  0.02 & 75 & 75.00 &  0.00\\
j6036 3.json & 1 & 0 & Optimal &  0.03 & 81 & 81.00 &  0.00\\
j6036 4.json & 1 & 0 & Optimal &  0.02 & 85 & 85.00 &  0.00\\
j6036 5.json & 1 & 0 & Optimal &  0.02 & 57 & 57.00 &  0.00\\
j6036 6.json & 1 & 0 & Optimal &  0.02 & 76 & 76.00 &  0.00\\
j6036 7.json & 1 & 0 & Optimal &  0.02 & 71 & 71.00 &  0.00\\
j6036 8.json & 1 & 0 & Optimal &  0.02 & 69 & 69.00 &  0.00\\
j6036 9.json & 1 & 0 & Optimal &  0.02 & 86 & 86.00 &  0.00\\
j6037 1.json & 1 & 0 & Optimal &  2.54 & 97 & 97.00 &  0.00\\
j6037 10.json & 1 & 0 & Optimal &  0.52 & 96 & 96.00 &  0.00\\
j6037 2.json & 1 & 0 & Optimal &  3.65 & 95 & 95.00 &  0.00\\
j6037 3.json & 1 & 0 & Optimal &  1.87 & 139 & 139.00 &  0.00\\
j6037 4.json & 1 & 0 & Optimal &  0.64 & 101 & 101.00 &  0.00\\
j6037 5.json & 1 & 0 & Optimal &  0.94 & 98 & 98.00 &  0.00\\
j6037 6.json & 1 & 0 & Optimal & 21.45 & 102 & 102.00 &  0.00\\
j6037 7.json & 1 & 0 & Optimal &  5.92 & 110 & 110.00 &  0.00\\
j6037 8.json & 1 & 0 & Optimal &  0.76 & 93 & 93.00 &  0.00\\
j6037 9.json & 1 & 0 & Optimal &  1.49 & 96 & 96.00 &  0.00\\
j6038 1.json & 1 & 0 & Optimal &  0.02 & 73 & 73.00 &  0.00\\
j6038 10.json & 1 & 0 & Optimal &  0.19 & 66 & 66.00 &  0.00\\
j6038 2.json & 1 & 0 & Optimal &  1.51 & 76 & 76.00 &  0.00\\
j6038 3.json & 1 & 0 & Optimal &  0.04 & 77 & 77.00 &  0.00\\
j6038 4.json & 1 & 0 & Optimal &  0.03 & 58 & 58.00 &  0.00\\
j6038 5.json & 1 & 0 & Optimal &  0.02 & 103 & 103.00 &  0.00\\
j6038 6.json & 1 & 0 & Optimal &  0.02 & 86 & 86.00 &  0.00\\
j6038 7.json & 1 & 0 & Optimal &  0.02 & 74 & 74.00 &  0.00\\
j6038 8.json & 1 & 0 & Optimal &  0.06 & 71 & 71.00 &  0.00\\
j6038 9.json & 1 & 0 & Optimal &  0.03 & 66 & 66.00 &  0.00\\
j6039 1.json & 1 & 0 & Optimal &  0.03 & 80 & 80.00 &  0.00\\
j6039 10.json & 1 & 0 & Optimal &  0.03 & 74 & 74.00 &  0.00\\
j6039 2.json & 1 & 0 & Optimal &  0.02 & 84 & 84.00 &  0.00\\
j6039 3.json & 1 & 0 & Optimal &  0.03 & 83 & 83.00 &  0.00\\
j6039 4.json & 1 & 0 & Optimal &  0.02 & 92 & 92.00 &  0.00\\
j6039 5.json & 1 & 0 & Optimal &  0.02 & 73 & 73.00 &  0.00\\
j6039 6.json & 1 & 0 & Optimal &  0.02 & 84 & 84.00 &  0.00\\
j6039 7.json & 1 & 0 & Optimal &  0.02 & 68 & 68.00 &  0.00\\
j6039 8.json & 1 & 0 & Optimal &  0.02 & 77 & 77.00 &  0.00\\
j6039 9.json & 1 & 0 & Optimal &  0.02 & 72 & 72.00 &  0.00\\
j603 1.json & 1 & 0 & Optimal &  0.02 & 60 & 60.00 &  0.00\\
j603 10.json & 1 & 0 & Optimal &  0.02 & 69 & 69.00 &  0.00\\
j603 2.json & 1 & 0 & Optimal &  0.02 & 69 & 69.00 &  0.00\\
j603 3.json & 1 & 0 & Optimal &  0.02 & 105 & 105.00 &  0.00\\
j603 4.json & 1 & 0 & Optimal &  0.02 & 81 & 81.00 &  0.00\\
j603 5.json & 1 & 0 & Optimal &  0.02 & 83 & 83.00 &  0.00\\
j603 6.json & 1 & 0 & Optimal &  0.02 & 57 & 57.00 &  0.00\\
j603 7.json & 1 & 0 & Optimal &  0.02 & 59 & 59.00 &  0.00\\
j603 8.json & 1 & 0 & Optimal &  0.09 & 55 & 55.00 &  0.00\\
j603 9.json & 1 & 0 & Optimal &  0.02 & 67 & 67.00 &  0.00\\
j6040 1.json & 1 & 0 & Optimal &  0.02 & 86 & 86.00 &  0.00\\
j6040 10.json & 1 & 0 & Optimal &  0.02 & 73 & 73.00 &  0.00\\
j6040 2.json & 1 & 0 & Optimal &  0.02 & 81 & 81.00 &  0.00\\
j6040 3.json & 1 & 0 & Optimal &  0.02 & 70 & 70.00 &  0.00\\
j6040 4.json & 1 & 0 & Optimal &  0.02 & 87 & 87.00 &  0.00\\
j6040 5.json & 1 & 0 & Optimal &  0.02 & 83 & 83.00 &  0.00\\
j6040 6.json & 1 & 0 & Optimal &  0.03 & 69 & 69.00 &  0.00\\
j6040 7.json & 1 & 0 & Optimal &  0.02 & 68 & 68.00 &  0.00\\
j6040 8.json & 1 & 0 & Optimal &  0.02 & 80 & 80.00 &  0.00\\
j6040 9.json & 1 & 0 & Optimal &  0.02 & 90 & 90.00 &  0.00\\
j6041 1.json & 1 & 0 & Optimal & 40.76 & 122 & 122.00 &  0.00\\
j6041 10.json & 1 & 0 & Solution & 600.01 & 111 & 108.00 &  2.70\\
j6041 2.json & 1 & 0 & Optimal & 31.96 & 113 & 113.00 &  0.00\\
j6041 3.json & 1 & 0 & Solution & 600.01 & 99 & 89.00 & 10.10\\
j6041 4.json & 1 & 0 & Optimal &  8.26 & 133 & 133.00 &  0.00\\
j6041 5.json & 1 & 0 & Solution & 600.01 & 117 & 101.00 & 13.68\\
j6041 6.json & 1 & 0 & Optimal & 65.80 & 134 & 134.00 &  0.00\\
j6041 7.json & 1 & 0 & Optimal & 18.86 & 132 & 132.00 &  0.00\\
j6041 8.json & 1 & 0 & Optimal & 21.69 & 135 & 135.00 &  0.00\\
j6041 9.json & 1 & 0 & Optimal & 35.00 & 131 & 131.00 &  0.00\\
j6042 1.json & 1 & 0 & Optimal &  0.02 & 83 & 83.00 &  0.00\\
j6042 10.json & 1 & 0 & Optimal &  0.02 & 87 & 87.00 &  0.00\\
j6042 2.json & 1 & 0 & Optimal &  0.02 & 68 & 68.00 &  0.00\\
j6042 3.json & 1 & 0 & Optimal &  4.44 & 78 & 78.00 &  0.00\\
j6042 4.json & 1 & 0 & Optimal &  0.56 & 103 & 103.00 &  0.00\\
j6042 5.json & 1 & 0 & Optimal &  0.02 & 73 & 73.00 &  0.00\\
j6042 6.json & 1 & 0 & Optimal &  0.02 & 82 & 82.00 &  0.00\\
j6042 7.json & 1 & 0 & Optimal &  2.18 & 59 & 59.00 &  0.00\\
j6042 8.json & 1 & 0 & Optimal &  0.78 & 82 & 82.00 &  0.00\\
j6042 9.json & 1 & 0 & Optimal &  0.03 & 71 & 71.00 &  0.00\\
j6043 1.json & 1 & 0 & Optimal &  0.02 & 108 & 108.00 &  0.00\\
j6043 10.json & 1 & 0 & Optimal &  0.02 & 78 & 78.00 &  0.00\\
j6043 2.json & 1 & 0 & Optimal &  0.03 & 85 & 85.00 &  0.00\\
j6043 3.json & 1 & 0 & Optimal &  0.02 & 74 & 74.00 &  0.00\\
j6043 4.json & 1 & 0 & Optimal &  0.02 & 75 & 75.00 &  0.00\\
j6043 5.json & 1 & 0 & Optimal &  0.02 & 64 & 64.00 &  0.00\\
j6043 6.json & 1 & 0 & Optimal &  0.03 & 84 & 84.00 &  0.00\\
j6043 7.json & 1 & 0 & Optimal &  0.03 & 89 & 89.00 &  0.00\\
j6043 8.json & 1 & 0 & Optimal &  0.02 & 69 & 69.00 &  0.00\\
j6043 9.json & 1 & 0 & Optimal &  0.02 & 70 & 70.00 &  0.00\\
j6044 1.json & 1 & 0 & Optimal &  0.03 & 84 & 84.00 &  0.00\\
j6044 10.json & 1 & 0 & Optimal &  0.02 & 65 & 65.00 &  0.00\\
j6044 2.json & 1 & 0 & Optimal &  0.02 & 68 & 68.00 &  0.00\\
j6044 3.json & 1 & 0 & Optimal &  0.02 & 87 & 87.00 &  0.00\\
j6044 4.json & 1 & 0 & Optimal &  0.02 & 77 & 77.00 &  0.00\\
j6044 5.json & 1 & 0 & Optimal &  0.03 & 74 & 74.00 &  0.00\\
j6044 6.json & 1 & 0 & Optimal &  0.02 & 81 & 81.00 &  0.00\\
j6044 7.json & 1 & 0 & Optimal &  0.02 & 76 & 76.00 &  0.00\\
j6044 8.json & 1 & 0 & Optimal &  0.02 & 83 & 83.00 &  0.00\\
j6044 9.json & 1 & 0 & Optimal &  0.02 & 65 & 65.00 &  0.00\\
j6045 1.json & 1 & 0 & Solution & 600.01 & 96 & 89.00 &  7.29\\
j6045 10.json & 1 & 0 & Solution & 600.01 & 115 & 104.00 &  9.57\\
j6045 2.json & 1 & 0 & Solution & 600.01 & 144 & 122.00 & 15.28\\
j6045 3.json & 1 & 0 & Solution & 600.01 & 144 & 130.00 &  9.72\\
j6045 4.json & 1 & 0 & Solution & 600.01 & 109 & 100.00 &  8.26\\
j6045 5.json & 1 & 0 & Solution & 600.01 & 106 & 100.00 &  5.66\\
j6045 6.json & 1 & 0 & Solution & 600.01 & 145 & 129.00 & 11.03\\
j6045 7.json & 1 & 0 & Solution & 600.01 & 122 & 110.00 &  9.84\\
j6045 8.json & 1 & 0 & Solution & 600.01 & 130 & 112.00 & 13.85\\
j6045 9.json & 1 & 0 & Solution & 600.02 & 124 & 111.00 & 10.48\\
j6046 1.json & 1 & 0 & Optimal &  0.90 & 79 & 78.00 &  1.27\\
j6046 10.json & 1 & 0 & Optimal & 135.10 & 88 & 88.00 &  0.00\\
j6046 2.json & 1 & 0 & Optimal &  0.02 & 78 & 78.00 &  0.00\\
j6046 3.json & 1 & 0 & Optimal &  1.54 & 79 & 79.00 &  0.00\\
j6046 4.json & 1 & 0 & Optimal & 32.94 & 74 & 74.00 &  0.00\\
j6046 5.json & 1 & 0 & Optimal & 17.77 & 91 & 91.00 &  0.00\\
j6046 6.json & 1 & 0 & Optimal &  8.76 & 90 & 90.00 &  0.00\\
j6046 7.json & 1 & 0 & Optimal & 67.16 & 78 & 78.00 &  0.00\\
j6046 8.json & 1 & 0 & Optimal &  1.25 & 75 & 75.00 &  0.00\\
j6046 9.json & 1 & 0 & Optimal & 314.39 & 69 & 69.00 &  0.00\\
j6047 1.json & 1 & 0 & Optimal &  0.02 & 75 & 75.00 &  0.00\\
j6047 10.json & 1 & 0 & Optimal &  0.02 & 66 & 66.00 &  0.00\\
j6047 2.json & 1 & 0 & Optimal &  0.03 & 66 & 66.00 &  0.00\\
j6047 3.json & 1 & 0 & Optimal &  0.02 & 69 & 69.00 &  0.00\\
j6047 4.json & 1 & 0 & Optimal &  0.02 & 76 & 76.00 &  0.00\\
j6047 5.json & 1 & 0 & Optimal &  0.02 & 87 & 87.00 &  0.00\\
j6047 6.json & 1 & 0 & Optimal &  0.02 & 76 & 76.00 &  0.00\\
j6047 7.json & 1 & 0 & Optimal &  0.02 & 68 & 68.00 &  0.00\\
j6047 8.json & 1 & 0 & Optimal &  0.03 & 71 & 71.00 &  0.00\\
j6047 9.json & 1 & 0 & Optimal &  0.02 & 76 & 76.00 &  0.00\\
j6048 1.json & 1 & 0 & Optimal &  0.02 & 71 & 71.00 &  0.00\\
j6048 10.json & 1 & 0 & Optimal &  0.02 & 70 & 70.00 &  0.00\\
j6048 2.json & 1 & 0 & Optimal &  0.02 & 87 & 87.00 &  0.00\\
j6048 3.json & 1 & 0 & Optimal &  0.02 & 84 & 84.00 &  0.00\\
j6048 4.json & 1 & 0 & Optimal &  0.03 & 62 & 62.00 &  0.00\\
j6048 5.json & 1 & 0 & Optimal &  0.02 & 101 & 101.00 &  0.00\\
j6048 6.json & 1 & 0 & Optimal &  0.03 & 66 & 66.00 &  0.00\\
j6048 7.json & 1 & 0 & Optimal &  0.03 & 77 & 77.00 &  0.00\\
j6048 8.json & 1 & 0 & Optimal &  0.03 & 88 & 88.00 &  0.00\\
j6048 9.json & 1 & 0 & Optimal &  0.03 & 82 & 82.00 &  0.00\\
j604 1.json & 1 & 0 & Optimal &  0.02 & 84 & 84.00 &  0.00\\
j604 10.json & 1 & 0 & Optimal &  0.02 & 77 & 77.00 &  0.00\\
j604 2.json & 1 & 0 & Optimal &  0.02 & 60 & 60.00 &  0.00\\
j604 3.json & 1 & 0 & Optimal &  0.02 & 58 & 58.00 &  0.00\\
j604 4.json & 1 & 0 & Optimal &  0.02 & 65 & 65.00 &  0.00\\
j604 5.json & 1 & 0 & Optimal &  0.02 & 75 & 75.00 &  0.00\\
j604 6.json & 1 & 0 & Optimal &  0.02 & 71 & 71.00 &  0.00\\
j604 7.json & 1 & 0 & Optimal &  0.02 & 67 & 67.00 &  0.00\\
j604 8.json & 1 & 0 & Optimal &  0.02 & 65 & 65.00 &  0.00\\
j604 9.json & 1 & 0 & Optimal &  0.02 & 75 & 75.00 &  0.00\\
j605 1.json & 1 & 0 & Optimal &  5.94 & 76 & 76.00 &  0.00\\
j605 10.json & 1 & 0 & Optimal & 25.80 & 81 & 81.00 &  0.00\\
j605 2.json & 1 & 0 & Optimal &  4.69 & 106 & 106.00 &  0.00\\
j605 3.json & 1 & 0 & Optimal &  2.77 & 80 & 80.00 &  0.00\\
j605 4.json & 1 & 0 & Optimal &  5.12 & 72 & 72.00 &  0.00\\
j605 5.json & 1 & 0 & Optimal &  2.27 & 108 & 108.00 &  0.00\\
j605 6.json & 1 & 0 & Optimal &  0.92 & 74 & 74.00 &  0.00\\
j605 7.json & 1 & 0 & Optimal & 11.15 & 75 & 75.00 &  0.00\\
j605 8.json & 1 & 0 & Optimal &  0.64 & 78 & 76.00 &  2.56\\
j605 9.json & 1 & 0 & Optimal &  0.44 & 83 & 83.00 &  0.00\\
j606 1.json & 1 & 0 & Optimal &  0.02 & 60 & 60.00 &  0.00\\
j606 10.json & 1 & 0 & Optimal &  0.02 & 74 & 74.00 &  0.00\\
j606 2.json & 1 & 0 & Optimal &  0.03 & 67 & 67.00 &  0.00\\
j606 3.json & 1 & 0 & Optimal &  0.02 & 72 & 72.00 &  0.00\\
j606 4.json & 1 & 0 & Optimal &  0.02 & 67 & 67.00 &  0.00\\
j606 5.json & 1 & 0 & Optimal &  0.02 & 78 & 78.00 &  0.00\\
j606 6.json & 1 & 0 & Optimal &  0.31 & 55 & 55.00 &  0.00\\
j606 7.json & 1 & 0 & Optimal &  0.02 & 61 & 61.00 &  0.00\\
j606 8.json & 1 & 0 & Optimal &  0.02 & 72 & 72.00 &  0.00\\
j606 9.json & 1 & 0 & Optimal &  0.02 & 64 & 64.00 &  0.00\\
j607 1.json & 1 & 0 & Optimal &  0.02 & 77 & 77.00 &  0.00\\
j607 10.json & 1 & 0 & Optimal &  0.02 & 82 & 82.00 &  0.00\\
j607 2.json & 1 & 0 & Optimal &  0.02 & 85 & 85.00 &  0.00\\
j607 3.json & 1 & 0 & Optimal &  0.02 & 62 & 62.00 &  0.00\\
j607 4.json & 1 & 0 & Optimal &  0.02 & 63 & 63.00 &  0.00\\
j607 5.json & 1 & 0 & Optimal &  0.02 & 71 & 71.00 &  0.00\\
j607 6.json & 1 & 0 & Optimal &  0.02 & 65 & 65.00 &  0.00\\
j607 7.json & 1 & 0 & Optimal &  0.02 & 89 & 89.00 &  0.00\\
j607 8.json & 1 & 0 & Optimal &  0.02 & 66 & 66.00 &  0.00\\
j607 9.json & 1 & 0 & Optimal &  0.02 & 44 & 44.00 &  0.00\\
j608 1.json & 1 & 0 & Optimal &  0.02 & 64 & 64.00 &  0.00\\
j608 10.json & 1 & 0 & Optimal &  0.02 & 97 & 97.00 &  0.00\\
j608 2.json & 1 & 0 & Optimal &  0.02 & 61 & 61.00 &  0.00\\
j608 3.json & 1 & 0 & Optimal &  0.02 & 79 & 79.00 &  0.00\\
j608 4.json & 1 & 0 & Optimal &  0.02 & 64 & 64.00 &  0.00\\
j608 5.json & 1 & 0 & Optimal &  0.02 & 83 & 83.00 &  0.00\\
j608 6.json & 1 & 0 & Optimal &  0.02 & 56 & 56.00 &  0.00\\
j608 7.json & 1 & 0 & Optimal &  0.02 & 62 & 62.00 &  0.00\\
j608 8.json & 1 & 0 & Optimal &  0.02 & 66 & 66.00 &  0.00\\
j608 9.json & 1 & 0 & Optimal &  0.02 & 58 & 58.00 &  0.00\\
j609 1.json & 1 & 0 & Solution & 600.01 & 87 & 85.00 &  2.30\\
j609 10.json & 1 & 0 & Solution & 600.02 & 95 & 90.00 &  5.26\\
j609 2.json & 1 & 0 & Optimal & 173.20 & 82 & 82.00 &  0.00\\
j609 3.json & 1 & 0 & Solution & 600.01 & 101 & 93.00 &  7.92\\
j609 4.json & 1 & 0 & Optimal & 19.41 & 87 & 87.00 &  0.00\\
j609 5.json & 1 & 0 & Solution & 600.01 & 87 & 81.00 &  6.90\\
j609 6.json & 1 & 0 & Solution & 600.01 & 112 & 104.00 &  7.14\\
j609 7.json & 1 & 0 & Solution & 600.01 & 111 & 104.00 &  6.31\\
j609 8.json & 1 & 0 & Solution & 600.03 & 96 & 90.00 &  6.25\\
j609 9.json & 1 & 0 & Solution & 600.06 & 99 & 98.00 &  1.01\\
\end{longtable}



\subsection{CPSat}
\begin{longtable}{lrrlrrrr}
\caption{Results for RCPSP J60 (CPSat) (480 Instances)}\\\toprule
Name & \shortstack{Nr\\Jobs} & \shortstack{Nr\\Machines} & Status & Time & Makespan & Bound & \shortstack{Gap\\Percent}\\ \midrule
\endhead
\bottomrule
\endfoot
j6010 1.json & 1 & 0 & Optimal &  0.10 & 85 & 85.00 &  0.00\\
j6010 10.json & 1 & 0 & Optimal &  0.03 & 73 & 73.00 &  0.00\\
j6010 2.json & 1 & 0 & Optimal &  0.05 & 62 & 62.00 &  0.00\\
j6010 3.json & 1 & 0 & Optimal &  0.04 & 72 & 72.00 &  0.00\\
j6010 4.json & 1 & 0 & Optimal &  0.07 & 80 & 80.00 &  0.00\\
j6010 5.json & 1 & 0 & Optimal &  0.07 & 79 & 79.00 &  0.00\\
j6010 6.json & 1 & 0 & Optimal &  0.07 & 67 & 67.00 &  0.00\\
j6010 7.json & 1 & 0 & Optimal &  0.06 & 69 & 69.00 &  0.00\\
j6010 8.json & 1 & 0 & Optimal &  0.07 & 65 & 65.00 &  0.00\\
j6010 9.json & 1 & 0 & Optimal &  0.10 & 73 & 73.00 &  0.00\\
j6011 1.json & 1 & 0 & Optimal &  0.03 & 71 & 71.00 &  0.00\\
j6011 10.json & 1 & 0 & Optimal &  0.02 & 58 & 58.00 &  0.00\\
j6011 2.json & 1 & 0 & Optimal &  0.04 & 61 & 61.00 &  0.00\\
j6011 3.json & 1 & 0 & Optimal &  0.05 & 76 & 76.00 &  0.00\\
j6011 4.json & 1 & 0 & Optimal &  0.07 & 69 & 69.00 &  0.00\\
j6011 5.json & 1 & 0 & Optimal &  0.03 & 65 & 65.00 &  0.00\\
j6011 6.json & 1 & 0 & Optimal &  0.03 & 70 & 70.00 &  0.00\\
j6011 7.json & 1 & 0 & Optimal &  0.02 & 70 & 70.00 &  0.00\\
j6011 8.json & 1 & 0 & Optimal &  0.03 & 69 & 69.00 &  0.00\\
j6011 9.json & 1 & 0 & Optimal &  0.03 & 62 & 62.00 &  0.00\\
j6012 1.json & 1 & 0 & Optimal &  0.03 & 59 & 59.00 &  0.00\\
j6012 10.json & 1 & 0 & Optimal &  0.02 & 79 & 79.00 &  0.00\\
j6012 2.json & 1 & 0 & Optimal &  0.02 & 58 & 58.00 &  0.00\\
j6012 3.json & 1 & 0 & Optimal &  0.04 & 75 & 75.00 &  0.00\\
j6012 4.json & 1 & 0 & Optimal &  0.03 & 69 & 69.00 &  0.00\\
j6012 5.json & 1 & 0 & Optimal &  0.03 & 63 & 63.00 &  0.00\\
j6012 6.json & 1 & 0 & Optimal &  0.02 & 54 & 54.00 &  0.00\\
j6012 7.json & 1 & 0 & Optimal &  0.03 & 71 & 71.00 &  0.00\\
j6012 8.json & 1 & 0 & Optimal &  0.02 & 60 & 60.00 &  0.00\\
j6012 9.json & 1 & 0 & Optimal &  0.02 & 59 & 59.00 &  0.00\\
j6013 1.json & 1 & 0 & Solution & 600.98 & 113 & 104.00 &  7.96\\
j6013 10.json & 1 & 0 & Solution & 600.25 & 120 & 112.00 &  6.67\\
j6013 2.json & 1 & 0 & Solution & 600.19 & 107 & 102.00 &  4.67\\
j6013 3.json & 1 & 0 & Solution & 600.14 & 91 & 83.00 &  8.79\\
j6013 4.json & 1 & 0 & Solution & 600.21 & 105 & 97.00 &  7.62\\
j6013 5.json & 1 & 0 & Solution & 600.17 & 99 & 91.00 &  8.08\\
j6013 6.json & 1 & 0 & Solution & 601.31 & 96 & 91.00 &  5.21\\
j6013 7.json & 1 & 0 & Solution & 600.15 & 89 & 81.00 &  8.99\\
j6013 8.json & 1 & 0 & Solution & 600.18 & 122 & 114.00 &  6.56\\
j6013 9.json & 1 & 0 & Solution & 600.34 & 103 & 95.00 &  7.77\\
j6014 1.json & 1 & 0 & Optimal & 595.18 & 61 & 61.00 &  0.00\\
j6014 10.json & 1 & 0 & Optimal & 600.02 & 72 & 72.00 &  0.00\\
j6014 2.json & 1 & 0 & Optimal &  0.07 & 65 & 65.00 &  0.00\\
j6014 3.json & 1 & 0 & Optimal & 600.01 & 61 & 61.00 &  0.00\\
j6014 4.json & 1 & 0 & Optimal & 19.55 & 65 & 65.00 &  0.00\\
j6014 5.json & 1 & 0 & Optimal &  0.04 & 59 & 59.00 &  0.00\\
j6014 6.json & 1 & 0 & Optimal &  0.06 & 65 & 65.00 &  0.00\\
j6014 7.json & 1 & 0 & Optimal &  0.06 & 69 & 69.00 &  0.00\\
j6014 8.json & 1 & 0 & Optimal &  0.04 & 88 & 88.00 &  0.00\\
j6014 9.json & 1 & 0 & Optimal &  0.04 & 61 & 61.00 &  0.00\\
j6015 1.json & 1 & 0 & Optimal &  0.03 & 84 & 84.00 &  0.00\\
j6015 10.json & 1 & 0 & Optimal &  0.03 & 61 & 61.00 &  0.00\\
j6015 2.json & 1 & 0 & Optimal &  0.04 & 89 & 89.00 &  0.00\\
j6015 3.json & 1 & 0 & Optimal &  0.03 & 72 & 72.00 &  0.00\\
j6015 4.json & 1 & 0 & Optimal &  0.05 & 75 & 75.00 &  0.00\\
j6015 5.json & 1 & 0 & Optimal &  0.03 & 70 & 70.00 &  0.00\\
j6015 6.json & 1 & 0 & Optimal &  0.02 & 76 & 76.00 &  0.00\\
j6015 7.json & 1 & 0 & Optimal &  0.02 & 64 & 64.00 &  0.00\\
j6015 8.json & 1 & 0 & Optimal &  0.03 & 79 & 79.00 &  0.00\\
j6015 9.json & 1 & 0 & Optimal &  0.02 & 72 & 72.00 &  0.00\\
j6016 1.json & 1 & 0 & Optimal &  0.02 & 64 & 64.00 &  0.00\\
j6016 10.json & 1 & 0 & Optimal &  0.03 & 68 & 68.00 &  0.00\\
j6016 2.json & 1 & 0 & Optimal &  0.02 & 64 & 64.00 &  0.00\\
j6016 3.json & 1 & 0 & Optimal &  0.03 & 53 & 53.00 &  0.00\\
j6016 4.json & 1 & 0 & Optimal &  0.03 & 60 & 60.00 &  0.00\\
j6016 5.json & 1 & 0 & Optimal &  0.03 & 66 & 66.00 &  0.00\\
j6016 6.json & 1 & 0 & Optimal &  0.03 & 66 & 66.00 &  0.00\\
j6016 7.json & 1 & 0 & Optimal &  0.02 & 82 & 82.00 &  0.00\\
j6016 8.json & 1 & 0 & Optimal &  0.03 & 68 & 68.00 &  0.00\\
j6016 9.json & 1 & 0 & Optimal &  0.03 & 54 & 54.00 &  0.00\\
j6017 1.json & 1 & 0 & Optimal &  0.04 & 86 & 86.00 &  0.00\\
j6017 10.json & 1 & 0 & Optimal &  0.04 & 72 & 72.00 &  0.00\\
j6017 2.json & 1 & 0 & Optimal &  0.06 & 69 & 69.00 &  0.00\\
j6017 3.json & 1 & 0 & Optimal &  0.07 & 89 & 89.00 &  0.00\\
j6017 4.json & 1 & 0 & Optimal &  0.04 & 71 & 71.00 &  0.00\\
j6017 5.json & 1 & 0 & Optimal &  0.06 & 59 & 59.00 &  0.00\\
j6017 6.json & 1 & 0 & Optimal &  0.06 & 69 & 69.00 &  0.00\\
j6017 7.json & 1 & 0 & Optimal &  0.06 & 83 & 83.00 &  0.00\\
j6017 8.json & 1 & 0 & Optimal &  0.09 & 85 & 85.00 &  0.00\\
j6017 9.json & 1 & 0 & Optimal &  0.04 & 76 & 76.00 &  0.00\\
j6018 1.json & 1 & 0 & Optimal &  0.05 & 81 & 81.00 &  0.00\\
j6018 10.json & 1 & 0 & Optimal &  0.04 & 97 & 97.00 &  0.00\\
j6018 2.json & 1 & 0 & Optimal &  0.03 & 69 & 69.00 &  0.00\\
j6018 3.json & 1 & 0 & Optimal &  0.02 & 77 & 77.00 &  0.00\\
j6018 4.json & 1 & 0 & Optimal &  0.06 & 71 & 71.00 &  0.00\\
j6018 5.json & 1 & 0 & Optimal &  0.03 & 80 & 80.00 &  0.00\\
j6018 6.json & 1 & 0 & Optimal &  0.04 & 61 & 61.00 &  0.00\\
j6018 7.json & 1 & 0 & Optimal &  0.03 & 93 & 93.00 &  0.00\\
j6018 8.json & 1 & 0 & Optimal &  0.04 & 78 & 78.00 &  0.00\\
j6018 9.json & 1 & 0 & Optimal &  0.04 & 69 & 69.00 &  0.00\\
j6019 1.json & 1 & 0 & Optimal &  0.04 & 62 & 62.00 &  0.00\\
j6019 10.json & 1 & 0 & Optimal &  0.03 & 78 & 78.00 &  0.00\\
j6019 2.json & 1 & 0 & Optimal &  0.02 & 83 & 83.00 &  0.00\\
j6019 3.json & 1 & 0 & Optimal &  0.03 & 83 & 83.00 &  0.00\\
j6019 4.json & 1 & 0 & Optimal &  0.03 & 67 & 67.00 &  0.00\\
j6019 5.json & 1 & 0 & Optimal &  0.03 & 73 & 73.00 &  0.00\\
j6019 6.json & 1 & 0 & Optimal &  0.03 & 69 & 69.00 &  0.00\\
j6019 7.json & 1 & 0 & Optimal &  0.02 & 60 & 60.00 &  0.00\\
j6019 8.json & 1 & 0 & Optimal &  0.03 & 87 & 87.00 &  0.00\\
j6019 9.json & 1 & 0 & Optimal &  0.06 & 69 & 69.00 &  0.00\\
j601 1.json & 1 & 0 & Optimal &  0.06 & 77 & 77.00 &  0.00\\
j601 10.json & 1 & 0 & Optimal &  0.04 & 80 & 80.00 &  0.00\\
j601 2.json & 1 & 0 & Optimal &  0.04 & 68 & 68.00 &  0.00\\
j601 3.json & 1 & 0 & Optimal &  0.07 & 68 & 68.00 &  0.00\\
j601 4.json & 1 & 0 & Optimal &  0.04 & 91 & 91.00 &  0.00\\
j601 5.json & 1 & 0 & Optimal &  0.12 & 73 & 73.00 &  0.00\\
j601 6.json & 1 & 0 & Optimal &  0.11 & 66 & 66.00 &  0.00\\
j601 7.json & 1 & 0 & Optimal &  0.11 & 72 & 72.00 &  0.00\\
j601 8.json & 1 & 0 & Optimal &  0.07 & 75 & 75.00 &  0.00\\
j601 9.json & 1 & 0 & Optimal &  0.11 & 85 & 85.00 &  0.00\\
j6020 1.json & 1 & 0 & Optimal &  0.03 & 60 & 60.00 &  0.00\\
j6020 10.json & 1 & 0 & Optimal &  0.03 & 70 & 70.00 &  0.00\\
j6020 2.json & 1 & 0 & Optimal &  0.03 & 78 & 78.00 &  0.00\\
j6020 3.json & 1 & 0 & Optimal &  0.03 & 69 & 69.00 &  0.00\\
j6020 4.json & 1 & 0 & Optimal &  0.03 & 86 & 86.00 &  0.00\\
j6020 5.json & 1 & 0 & Optimal &  0.03 & 71 & 71.00 &  0.00\\
j6020 6.json & 1 & 0 & Optimal &  0.02 & 97 & 97.00 &  0.00\\
j6020 7.json & 1 & 0 & Optimal &  0.03 & 74 & 74.00 &  0.00\\
j6020 8.json & 1 & 0 & Optimal &  0.03 & 65 & 65.00 &  0.00\\
j6020 9.json & 1 & 0 & Optimal &  0.03 & 74 & 74.00 &  0.00\\
j6021 1.json & 1 & 0 & Optimal &  1.09 & 103 & 103.00 &  0.00\\
j6021 10.json & 1 & 0 & Optimal &  0.47 & 80 & 80.00 &  0.00\\
j6021 2.json & 1 & 0 & Optimal &  0.32 & 108 & 108.00 &  0.00\\
j6021 3.json & 1 & 0 & Optimal &  0.82 & 87 & 87.00 &  0.00\\
j6021 4.json & 1 & 0 & Optimal &  2.42 & 95 & 95.00 &  0.00\\
j6021 5.json & 1 & 0 & Optimal &  2.05 & 89 & 89.00 &  0.00\\
j6021 6.json & 1 & 0 & Optimal &  1.23 & 84 & 84.00 &  0.00\\
j6021 7.json & 1 & 0 & Optimal &  0.58 & 103 & 103.00 &  0.00\\
j6021 8.json & 1 & 0 & Optimal &  1.48 & 110 & 110.00 &  0.00\\
j6021 9.json & 1 & 0 & Optimal & 31.08 & 89 & 89.00 &  0.00\\
j6022 1.json & 1 & 0 & Optimal &  0.05 & 64 & 64.00 &  0.00\\
j6022 10.json & 1 & 0 & Optimal &  0.06 & 70 & 70.00 &  0.00\\
j6022 2.json & 1 & 0 & Optimal &  0.09 & 83 & 83.00 &  0.00\\
j6022 3.json & 1 & 0 & Optimal &  0.10 & 70 & 70.00 &  0.00\\
j6022 4.json & 1 & 0 & Optimal &  0.10 & 73 & 73.00 &  0.00\\
j6022 5.json & 1 & 0 & Optimal &  0.06 & 76 & 76.00 &  0.00\\
j6022 6.json & 1 & 0 & Optimal &  0.04 & 79 & 79.00 &  0.00\\
j6022 7.json & 1 & 0 & Optimal &  0.06 & 69 & 69.00 &  0.00\\
j6022 8.json & 1 & 0 & Optimal &  0.06 & 59 & 59.00 &  0.00\\
j6022 9.json & 1 & 0 & Optimal &  0.06 & 65 & 65.00 &  0.00\\
j6023 1.json & 1 & 0 & Optimal &  0.03 & 75 & 75.00 &  0.00\\
j6023 10.json & 1 & 0 & Optimal &  0.05 & 68 & 68.00 &  0.00\\
j6023 2.json & 1 & 0 & Optimal &  0.03 & 69 & 69.00 &  0.00\\
j6023 3.json & 1 & 0 & Optimal &  0.03 & 78 & 78.00 &  0.00\\
j6023 4.json & 1 & 0 & Optimal &  0.04 & 83 & 83.00 &  0.00\\
j6023 5.json & 1 & 0 & Optimal &  0.04 & 72 & 72.00 &  0.00\\
j6023 6.json & 1 & 0 & Optimal &  0.04 & 81 & 81.00 &  0.00\\
j6023 7.json & 1 & 0 & Optimal &  0.03 & 60 & 60.00 &  0.00\\
j6023 8.json & 1 & 0 & Optimal &  0.02 & 72 & 72.00 &  0.00\\
j6023 9.json & 1 & 0 & Optimal &  0.04 & 64 & 64.00 &  0.00\\
j6024 1.json & 1 & 0 & Optimal &  0.02 & 65 & 65.00 &  0.00\\
j6024 10.json & 1 & 0 & Optimal &  0.03 & 66 & 66.00 &  0.00\\
j6024 2.json & 1 & 0 & Optimal &  0.03 & 55 & 55.00 &  0.00\\
j6024 3.json & 1 & 0 & Optimal &  0.03 & 67 & 67.00 &  0.00\\
j6024 4.json & 1 & 0 & Optimal &  0.02 & 78 & 78.00 &  0.00\\
j6024 5.json & 1 & 0 & Optimal &  0.03 & 76 & 76.00 &  0.00\\
j6024 6.json & 1 & 0 & Optimal &  0.02 & 75 & 75.00 &  0.00\\
j6024 7.json & 1 & 0 & Optimal &  0.03 & 68 & 68.00 &  0.00\\
j6024 8.json & 1 & 0 & Optimal &  0.02 & 81 & 81.00 &  0.00\\
j6024 9.json & 1 & 0 & Optimal &  0.03 & 80 & 80.00 &  0.00\\
j6025 1.json & 1 & 0 & Optimal & 600.03 & 114 & 114.00 &  0.00\\
j6025 10.json & 1 & 0 & Optimal & 586.36 & 108 & 108.00 &  0.00\\
j6025 2.json & 1 & 0 & Solution & 601.75 & 99 & 91.00 &  8.08\\
j6025 3.json & 1 & 0 & Optimal & 67.98 & 113 & 113.00 &  0.00\\
j6025 4.json & 1 & 0 & Solution & 600.30 & 108 & 100.00 &  7.41\\
j6025 5.json & 1 & 0 & Optimal & 600.02 & 98 & 98.00 &  0.00\\
j6025 6.json & 1 & 0 & Solution & 600.18 & 113 & 102.00 &  9.73\\
j6025 7.json & 1 & 0 & Solution & 601.43 & 90 & 84.00 &  6.67\\
j6025 8.json & 1 & 0 & Solution & 601.73 & 99 & 92.00 &  7.07\\
j6025 9.json & 1 & 0 & Optimal & 160.54 & 99 & 99.00 &  0.00\\
j6026 1.json & 1 & 0 & Optimal &  0.08 & 80 & 80.00 &  0.00\\
j6026 10.json & 1 & 0 & Optimal &  0.06 & 85 & 85.00 &  0.00\\
j6026 2.json & 1 & 0 & Optimal &  0.10 & 66 & 66.00 &  0.00\\
j6026 3.json & 1 & 0 & Optimal &  0.34 & 76 & 76.00 &  0.00\\
j6026 4.json & 1 & 0 & Optimal &  1.99 & 67 & 67.00 &  0.00\\
j6026 5.json & 1 & 0 & Optimal &  0.07 & 61 & 61.00 &  0.00\\
j6026 6.json & 1 & 0 & Optimal &  0.09 & 74 & 74.00 &  0.00\\
j6026 7.json & 1 & 0 & Optimal &  0.07 & 72 & 72.00 &  0.00\\
j6026 8.json & 1 & 0 & Optimal &  0.07 & 89 & 89.00 &  0.00\\
j6026 9.json & 1 & 0 & Optimal &  0.29 & 65 & 65.00 &  0.00\\
j6027 1.json & 1 & 0 & Optimal &  0.03 & 96 & 96.00 &  0.00\\
j6027 10.json & 1 & 0 & Optimal &  0.03 & 57 & 57.00 &  0.00\\
j6027 2.json & 1 & 0 & Optimal &  0.04 & 74 & 74.00 &  0.00\\
j6027 3.json & 1 & 0 & Optimal &  0.07 & 76 & 76.00 &  0.00\\
j6027 4.json & 1 & 0 & Optimal &  0.03 & 60 & 60.00 &  0.00\\
j6027 5.json & 1 & 0 & Optimal &  0.02 & 78 & 78.00 &  0.00\\
j6027 6.json & 1 & 0 & Optimal &  0.04 & 64 & 64.00 &  0.00\\
j6027 7.json & 1 & 0 & Optimal &  0.02 & 83 & 83.00 &  0.00\\
j6027 8.json & 1 & 0 & Optimal &  0.04 & 88 & 88.00 &  0.00\\
j6027 9.json & 1 & 0 & Optimal &  0.03 & 76 & 76.00 &  0.00\\
j6028 1.json & 1 & 0 & Optimal &  0.02 & 92 & 92.00 &  0.00\\
j6028 10.json & 1 & 0 & Optimal &  0.02 & 74 & 74.00 &  0.00\\
j6028 2.json & 1 & 0 & Optimal &  0.03 & 64 & 64.00 &  0.00\\
j6028 3.json & 1 & 0 & Optimal &  0.02 & 72 & 72.00 &  0.00\\
j6028 4.json & 1 & 0 & Optimal &  0.03 & 84 & 84.00 &  0.00\\
j6028 5.json & 1 & 0 & Optimal &  0.02 & 71 & 71.00 &  0.00\\
j6028 6.json & 1 & 0 & Optimal &  0.03 & 89 & 89.00 &  0.00\\
j6028 7.json & 1 & 0 & Optimal &  0.02 & 75 & 75.00 &  0.00\\
j6028 8.json & 1 & 0 & Optimal &  0.02 & 62 & 62.00 &  0.00\\
j6028 9.json & 1 & 0 & Optimal &  0.02 & 74 & 74.00 &  0.00\\
j6029 1.json & 1 & 0 & Solution & 601.51 & 106 & 96.00 &  9.43\\
j6029 10.json & 1 & 0 & Solution & 600.22 & 121 & 110.00 &  9.09\\
j6029 2.json & 1 & 0 & Solution & 600.18 & 135 & 116.00 & 14.07\\
j6029 3.json & 1 & 0 & Solution & 600.48 & 122 & 113.00 &  7.38\\
j6029 4.json & 1 & 0 & Solution & 601.36 & 137 & 120.00 & 12.41\\
j6029 5.json & 1 & 0 & Solution & 600.19 & 111 & 100.00 &  9.91\\
j6029 6.json & 1 & 0 & Solution & 601.84 & 156 & 143.00 &  8.33\\
j6029 7.json & 1 & 0 & Solution & 601.58 & 124 & 113.00 &  8.87\\
j6029 8.json & 1 & 0 & Solution & 600.20 & 104 & 95.00 &  8.65\\
j6029 9.json & 1 & 0 & Solution & 600.18 & 113 & 100.00 & 11.50\\
j602 1.json & 1 & 0 & Optimal &  0.04 & 65 & 65.00 &  0.00\\
j602 10.json & 1 & 0 & Optimal &  0.04 & 69 & 69.00 &  0.00\\
j602 2.json & 1 & 0 & Optimal &  0.04 & 82 & 82.00 &  0.00\\
j602 3.json & 1 & 0 & Optimal &  0.04 & 78 & 78.00 &  0.00\\
j602 4.json & 1 & 0 & Optimal &  0.06 & 78 & 78.00 &  0.00\\
j602 5.json & 1 & 0 & Optimal &  0.04 & 54 & 54.00 &  0.00\\
j602 6.json & 1 & 0 & Optimal &  0.04 & 64 & 64.00 &  0.00\\
j602 7.json & 1 & 0 & Optimal &  0.03 & 53 & 53.00 &  0.00\\
j602 8.json & 1 & 0 & Optimal &  0.06 & 66 & 66.00 &  0.00\\
j602 9.json & 1 & 0 & Optimal &  0.04 & 65 & 65.00 &  0.00\\
j6030 1.json & 1 & 0 & Optimal &  0.07 & 70 & 70.00 &  0.00\\
j6030 10.json & 1 & 0 & Optimal & 600.01 & 86 & 86.00 &  0.00\\
j6030 2.json & 1 & 0 & Solution & 600.28 & 70 & 68.00 &  2.86\\
j6030 3.json & 1 & 0 & Optimal &  0.08 & 82 & 82.00 &  0.00\\
j6030 4.json & 1 & 0 & Optimal &  0.02 & 76 & 76.00 &  0.00\\
j6030 5.json & 1 & 0 & Optimal & 600.02 & 76 & 76.00 &  0.00\\
j6030 6.json & 1 & 0 & Optimal &  0.07 & 68 & 68.00 &  0.00\\
j6030 7.json & 1 & 0 & Optimal & 600.03 & 86 & 86.00 &  0.00\\
j6030 8.json & 1 & 0 & Optimal &  0.05 & 63 & 63.00 &  0.00\\
j6030 9.json & 1 & 0 & Optimal &  0.09 & 98 & 98.00 &  0.00\\
j6031 1.json & 1 & 0 & Optimal &  0.03 & 65 & 65.00 &  0.00\\
j6031 10.json & 1 & 0 & Optimal &  0.03 & 56 & 56.00 &  0.00\\
j6031 2.json & 1 & 0 & Optimal &  0.02 & 74 & 74.00 &  0.00\\
j6031 3.json & 1 & 0 & Optimal &  0.03 & 66 & 66.00 &  0.00\\
j6031 4.json & 1 & 0 & Optimal &  0.02 & 68 & 68.00 &  0.00\\
j6031 5.json & 1 & 0 & Optimal &  0.02 & 72 & 72.00 &  0.00\\
j6031 6.json & 1 & 0 & Optimal &  0.03 & 72 & 72.00 &  0.00\\
j6031 7.json & 1 & 0 & Optimal &  0.03 & 76 & 76.00 &  0.00\\
j6031 8.json & 1 & 0 & Optimal &  0.03 & 75 & 75.00 &  0.00\\
j6031 9.json & 1 & 0 & Optimal &  0.04 & 86 & 86.00 &  0.00\\
j6032 1.json & 1 & 0 & Optimal &  0.03 & 69 & 69.00 &  0.00\\
j6032 10.json & 1 & 0 & Optimal &  0.03 & 77 & 77.00 &  0.00\\
j6032 2.json & 1 & 0 & Optimal &  0.02 & 114 & 114.00 &  0.00\\
j6032 3.json & 1 & 0 & Optimal &  0.03 & 85 & 85.00 &  0.00\\
j6032 4.json & 1 & 0 & Optimal &  0.02 & 56 & 56.00 &  0.00\\
j6032 5.json & 1 & 0 & Optimal &  0.03 & 77 & 77.00 &  0.00\\
j6032 6.json & 1 & 0 & Optimal &  0.02 & 93 & 93.00 &  0.00\\
j6032 7.json & 1 & 0 & Optimal &  0.02 & 76 & 76.00 &  0.00\\
j6032 8.json & 1 & 0 & Optimal &  0.03 & 76 & 76.00 &  0.00\\
j6032 9.json & 1 & 0 & Optimal &  0.02 & 74 & 74.00 &  0.00\\
j6033 1.json & 1 & 0 & Optimal &  0.06 & 105 & 105.00 &  0.00\\
j6033 10.json & 1 & 0 & Optimal &  0.07 & 84 & 84.00 &  0.00\\
j6033 2.json & 1 & 0 & Optimal &  0.04 & 100 & 100.00 &  0.00\\
j6033 3.json & 1 & 0 & Optimal &  0.04 & 79 & 79.00 &  0.00\\
j6033 4.json & 1 & 0 & Optimal &  0.07 & 81 & 81.00 &  0.00\\
j6033 5.json & 1 & 0 & Optimal &  0.04 & 108 & 108.00 &  0.00\\
j6033 6.json & 1 & 0 & Optimal &  0.07 & 75 & 75.00 &  0.00\\
j6033 7.json & 1 & 0 & Optimal &  0.12 & 78 & 78.00 &  0.00\\
j6033 8.json & 1 & 0 & Optimal &  0.11 & 79 & 79.00 &  0.00\\
j6033 9.json & 1 & 0 & Optimal &  0.06 & 108 & 108.00 &  0.00\\
j6034 1.json & 1 & 0 & Optimal &  0.04 & 72 & 72.00 &  0.00\\
j6034 10.json & 1 & 0 & Optimal &  0.04 & 92 & 92.00 &  0.00\\
j6034 2.json & 1 & 0 & Optimal &  0.03 & 68 & 68.00 &  0.00\\
j6034 3.json & 1 & 0 & Optimal &  0.06 & 61 & 61.00 &  0.00\\
j6034 4.json & 1 & 0 & Optimal &  0.03 & 83 & 83.00 &  0.00\\
j6034 5.json & 1 & 0 & Optimal &  0.03 & 80 & 80.00 &  0.00\\
j6034 6.json & 1 & 0 & Optimal &  0.04 & 81 & 81.00 &  0.00\\
j6034 7.json & 1 & 0 & Optimal &  0.04 & 85 & 85.00 &  0.00\\
j6034 8.json & 1 & 0 & Optimal &  0.04 & 63 & 63.00 &  0.00\\
j6034 9.json & 1 & 0 & Optimal &  0.03 & 77 & 77.00 &  0.00\\
j6035 1.json & 1 & 0 & Optimal &  0.03 & 78 & 78.00 &  0.00\\
j6035 10.json & 1 & 0 & Optimal &  0.03 & 71 & 71.00 &  0.00\\
j6035 2.json & 1 & 0 & Optimal &  0.04 & 77 & 77.00 &  0.00\\
j6035 3.json & 1 & 0 & Optimal &  0.04 & 89 & 89.00 &  0.00\\
j6035 4.json & 1 & 0 & Optimal &  0.04 & 72 & 72.00 &  0.00\\
j6035 5.json & 1 & 0 & Optimal &  0.04 & 76 & 76.00 &  0.00\\
j6035 6.json & 1 & 0 & Optimal &  0.03 & 79 & 79.00 &  0.00\\
j6035 7.json & 1 & 0 & Optimal &  0.03 & 73 & 73.00 &  0.00\\
j6035 8.json & 1 & 0 & Optimal &  0.03 & 78 & 78.00 &  0.00\\
j6035 9.json & 1 & 0 & Optimal &  0.04 & 76 & 76.00 &  0.00\\
j6036 1.json & 1 & 0 & Optimal &  0.03 & 61 & 61.00 &  0.00\\
j6036 10.json & 1 & 0 & Optimal &  0.03 & 77 & 77.00 &  0.00\\
j6036 2.json & 1 & 0 & Optimal &  0.03 & 75 & 75.00 &  0.00\\
j6036 3.json & 1 & 0 & Optimal &  0.02 & 81 & 81.00 &  0.00\\
j6036 4.json & 1 & 0 & Optimal &  0.03 & 85 & 85.00 &  0.00\\
j6036 5.json & 1 & 0 & Optimal &  0.02 & 57 & 57.00 &  0.00\\
j6036 6.json & 1 & 0 & Optimal &  0.03 & 76 & 76.00 &  0.00\\
j6036 7.json & 1 & 0 & Optimal &  0.03 & 71 & 71.00 &  0.00\\
j6036 8.json & 1 & 0 & Optimal &  0.03 & 69 & 69.00 &  0.00\\
j6036 9.json & 1 & 0 & Optimal &  0.02 & 86 & 86.00 &  0.00\\
j6037 1.json & 1 & 0 & Optimal &  1.70 & 97 & 97.00 &  0.00\\
j6037 10.json & 1 & 0 & Optimal &  0.28 & 96 & 96.00 &  0.00\\
j6037 2.json & 1 & 0 & Optimal & 37.52 & 95 & 95.00 &  0.00\\
j6037 3.json & 1 & 0 & Optimal &  1.52 & 139 & 139.00 &  0.00\\
j6037 4.json & 1 & 0 & Optimal &  0.34 & 101 & 101.00 &  0.00\\
j6037 5.json & 1 & 0 & Optimal &  1.40 & 98 & 98.00 &  0.00\\
j6037 6.json & 1 & 0 & Optimal & 18.04 & 102 & 102.00 &  0.00\\
j6037 7.json & 1 & 0 & Optimal &  5.39 & 110 & 110.00 &  0.00\\
j6037 8.json & 1 & 0 & Optimal &  0.23 & 93 & 93.00 &  0.00\\
j6037 9.json & 1 & 0 & Optimal &  0.60 & 96 & 96.00 &  0.00\\
j6038 1.json & 1 & 0 & Optimal &  0.06 & 73 & 73.00 &  0.00\\
j6038 10.json & 1 & 0 & Optimal &  0.06 & 66 & 66.00 &  0.00\\
j6038 2.json & 1 & 0 & Optimal &  0.98 & 76 & 76.00 &  0.00\\
j6038 3.json & 1 & 0 & Optimal &  0.06 & 77 & 77.00 &  0.00\\
j6038 4.json & 1 & 0 & Optimal &  0.09 & 58 & 58.00 &  0.00\\
j6038 5.json & 1 & 0 & Optimal &  0.06 & 103 & 103.00 &  0.00\\
j6038 6.json & 1 & 0 & Optimal &  0.06 & 86 & 86.00 &  0.00\\
j6038 7.json & 1 & 0 & Optimal &  0.06 & 74 & 74.00 &  0.00\\
j6038 8.json & 1 & 0 & Optimal &  0.09 & 71 & 71.00 &  0.00\\
j6038 9.json & 1 & 0 & Optimal &  0.09 & 66 & 66.00 &  0.00\\
j6039 1.json & 1 & 0 & Optimal &  0.03 & 80 & 80.00 &  0.00\\
j6039 10.json & 1 & 0 & Optimal &  0.03 & 74 & 74.00 &  0.00\\
j6039 2.json & 1 & 0 & Optimal &  0.03 & 84 & 84.00 &  0.00\\
j6039 3.json & 1 & 0 & Optimal &  0.03 & 83 & 83.00 &  0.00\\
j6039 4.json & 1 & 0 & Optimal &  0.04 & 92 & 92.00 &  0.00\\
j6039 5.json & 1 & 0 & Optimal &  0.03 & 73 & 73.00 &  0.00\\
j6039 6.json & 1 & 0 & Optimal &  0.04 & 84 & 84.00 &  0.00\\
j6039 7.json & 1 & 0 & Optimal &  0.03 & 68 & 68.00 &  0.00\\
j6039 8.json & 1 & 0 & Optimal &  0.03 & 77 & 77.00 &  0.00\\
j6039 9.json & 1 & 0 & Optimal &  0.03 & 72 & 72.00 &  0.00\\
j603 1.json & 1 & 0 & Optimal &  0.02 & 60 & 60.00 &  0.00\\
j603 10.json & 1 & 0 & Optimal &  0.04 & 69 & 69.00 &  0.00\\
j603 2.json & 1 & 0 & Optimal &  0.03 & 69 & 69.00 &  0.00\\
j603 3.json & 1 & 0 & Optimal &  0.03 & 105 & 105.00 &  0.00\\
j603 4.json & 1 & 0 & Optimal &  0.03 & 81 & 81.00 &  0.00\\
j603 5.json & 1 & 0 & Optimal &  0.03 & 83 & 83.00 &  0.00\\
j603 6.json & 1 & 0 & Optimal &  0.05 & 57 & 57.00 &  0.00\\
j603 7.json & 1 & 0 & Optimal &  0.03 & 59 & 59.00 &  0.00\\
j603 8.json & 1 & 0 & Optimal &  0.04 & 55 & 55.00 &  0.00\\
j603 9.json & 1 & 0 & Optimal &  0.03 & 67 & 67.00 &  0.00\\
j6040 1.json & 1 & 0 & Optimal &  0.03 & 86 & 86.00 &  0.00\\
j6040 10.json & 1 & 0 & Optimal &  0.03 & 73 & 73.00 &  0.00\\
j6040 2.json & 1 & 0 & Optimal &  0.02 & 81 & 81.00 &  0.00\\
j6040 3.json & 1 & 0 & Optimal &  0.02 & 70 & 70.00 &  0.00\\
j6040 4.json & 1 & 0 & Optimal &  0.03 & 87 & 87.00 &  0.00\\
j6040 5.json & 1 & 0 & Optimal &  0.03 & 83 & 83.00 &  0.00\\
j6040 6.json & 1 & 0 & Optimal &  0.02 & 69 & 69.00 &  0.00\\
j6040 7.json & 1 & 0 & Optimal &  0.03 & 68 & 68.00 &  0.00\\
j6040 8.json & 1 & 0 & Optimal &  0.03 & 80 & 80.00 &  0.00\\
j6040 9.json & 1 & 0 & Optimal &  0.02 & 90 & 90.00 &  0.00\\
j6041 1.json & 1 & 0 & Optimal & 29.14 & 122 & 122.00 &  0.00\\
j6041 10.json & 1 & 0 & Solution & 600.16 & 111 & 105.00 &  5.41\\
j6041 2.json & 1 & 0 & Optimal & 93.30 & 113 & 113.00 &  0.00\\
j6041 3.json & 1 & 0 & Solution & 600.16 & 98 & 87.00 & 11.22\\
j6041 4.json & 1 & 0 & Optimal &  5.70 & 133 & 133.00 &  0.00\\
j6041 5.json & 1 & 0 & Solution & 600.16 & 117 & 105.00 & 10.26\\
j6041 6.json & 1 & 0 & Optimal & 34.86 & 134 & 134.00 &  0.00\\
j6041 7.json & 1 & 0 & Optimal & 28.37 & 132 & 132.00 &  0.00\\
j6041 8.json & 1 & 0 & Optimal & 26.60 & 135 & 135.00 &  0.00\\
j6041 9.json & 1 & 0 & Optimal & 50.88 & 131 & 131.00 &  0.00\\
j6042 1.json & 1 & 0 & Optimal &  0.05 & 83 & 83.00 &  0.00\\
j6042 10.json & 1 & 0 & Optimal &  0.09 & 87 & 87.00 &  0.00\\
j6042 2.json & 1 & 0 & Optimal &  0.07 & 68 & 68.00 &  0.00\\
j6042 3.json & 1 & 0 & Optimal &  3.54 & 78 & 78.00 &  0.00\\
j6042 4.json & 1 & 0 & Optimal &  0.11 & 103 & 103.00 &  0.00\\
j6042 5.json & 1 & 0 & Optimal &  0.06 & 73 & 73.00 &  0.00\\
j6042 6.json & 1 & 0 & Optimal &  0.10 & 82 & 82.00 &  0.00\\
j6042 7.json & 1 & 0 & Optimal &  0.41 & 59 & 59.00 &  0.00\\
j6042 8.json & 1 & 0 & Optimal &  0.14 & 82 & 82.00 &  0.00\\
j6042 9.json & 1 & 0 & Optimal &  0.07 & 71 & 71.00 &  0.00\\
j6043 1.json & 1 & 0 & Optimal &  0.08 & 108 & 108.00 &  0.00\\
j6043 10.json & 1 & 0 & Optimal &  0.04 & 78 & 78.00 &  0.00\\
j6043 2.json & 1 & 0 & Optimal &  0.08 & 85 & 85.00 &  0.00\\
j6043 3.json & 1 & 0 & Optimal &  0.04 & 74 & 74.00 &  0.00\\
j6043 4.json & 1 & 0 & Optimal &  0.04 & 75 & 75.00 &  0.00\\
j6043 5.json & 1 & 0 & Optimal &  0.05 & 64 & 64.00 &  0.00\\
j6043 6.json & 1 & 0 & Optimal &  0.04 & 84 & 84.00 &  0.00\\
j6043 7.json & 1 & 0 & Optimal &  0.04 & 89 & 89.00 &  0.00\\
j6043 8.json & 1 & 0 & Optimal &  0.02 & 69 & 69.00 &  0.00\\
j6043 9.json & 1 & 0 & Optimal &  0.03 & 70 & 70.00 &  0.00\\
j6044 1.json & 1 & 0 & Optimal &  0.02 & 84 & 84.00 &  0.00\\
j6044 10.json & 1 & 0 & Optimal &  0.02 & 65 & 65.00 &  0.00\\
j6044 2.json & 1 & 0 & Optimal &  0.02 & 68 & 68.00 &  0.00\\
j6044 3.json & 1 & 0 & Optimal &  0.02 & 87 & 87.00 &  0.00\\
j6044 4.json & 1 & 0 & Optimal &  0.02 & 77 & 77.00 &  0.00\\
j6044 5.json & 1 & 0 & Optimal &  0.02 & 74 & 74.00 &  0.00\\
j6044 6.json & 1 & 0 & Optimal &  0.02 & 81 & 81.00 &  0.00\\
j6044 7.json & 1 & 0 & Optimal &  0.02 & 76 & 76.00 &  0.00\\
j6044 8.json & 1 & 0 & Optimal &  0.02 & 83 & 83.00 &  0.00\\
j6044 9.json & 1 & 0 & Optimal &  0.02 & 65 & 65.00 &  0.00\\
j6045 1.json & 1 & 0 & Solution & 600.17 & 97 & 87.00 & 10.31\\
j6045 10.json & 1 & 0 & Solution & 600.23 & 114 & 99.00 & 13.16\\
j6045 2.json & 1 & 0 & Solution & 601.26 & 145 & 126.00 & 13.10\\
j6045 3.json & 1 & 0 & Solution & 600.33 & 147 & 129.00 & 12.24\\
j6045 4.json & 1 & 0 & Solution & 600.19 & 108 & 97.00 & 10.19\\
j6045 5.json & 1 & 0 & Solution & 601.71 & 108 & 98.00 &  9.26\\
j6045 6.json & 1 & 0 & Solution & 601.08 & 147 & 125.00 & 14.97\\
j6045 7.json & 1 & 0 & Solution & 600.28 & 122 & 109.00 & 10.66\\
j6045 8.json & 1 & 0 & Solution & 600.26 & 129 & 114.00 & 11.63\\
j6045 9.json & 1 & 0 & Solution & 600.27 & 124 & 108.00 & 12.90\\
j6046 1.json & 1 & 0 & Optimal &  0.10 & 79 & 79.00 &  0.00\\
j6046 10.json & 1 & 0 & Optimal & 142.88 & 88 & 88.00 &  0.00\\
j6046 2.json & 1 & 0 & Optimal &  0.04 & 78 & 78.00 &  0.00\\
j6046 3.json & 1 & 0 & Optimal &  0.78 & 79 & 79.00 &  0.00\\
j6046 4.json & 1 & 0 & Optimal & 11.37 & 74 & 74.00 &  0.00\\
j6046 5.json & 1 & 0 & Optimal &  8.15 & 91 & 91.00 &  0.00\\
j6046 6.json & 1 & 0 & Optimal & 14.21 & 90 & 90.00 &  0.00\\
j6046 7.json & 1 & 0 & Optimal & 119.87 & 78 & 78.00 &  0.00\\
j6046 8.json & 1 & 0 & Optimal &  0.64 & 75 & 75.00 &  0.00\\
j6046 9.json & 1 & 0 & Optimal & 600.02 & 69 & 69.00 &  0.00\\
j6047 1.json & 1 & 0 & Optimal &  0.03 & 75 & 75.00 &  0.00\\
j6047 10.json & 1 & 0 & Optimal &  0.04 & 66 & 66.00 &  0.00\\
j6047 2.json & 1 & 0 & Optimal &  0.04 & 66 & 66.00 &  0.00\\
j6047 3.json & 1 & 0 & Optimal &  0.05 & 69 & 69.00 &  0.00\\
j6047 4.json & 1 & 0 & Optimal &  0.04 & 76 & 76.00 &  0.00\\
j6047 5.json & 1 & 0 & Optimal &  0.06 & 87 & 87.00 &  0.00\\
j6047 6.json & 1 & 0 & Optimal &  0.06 & 76 & 76.00 &  0.00\\
j6047 7.json & 1 & 0 & Optimal &  0.04 & 68 & 68.00 &  0.00\\
j6047 8.json & 1 & 0 & Optimal &  0.04 & 71 & 71.00 &  0.00\\
j6047 9.json & 1 & 0 & Optimal &  0.05 & 76 & 76.00 &  0.00\\
j6048 1.json & 1 & 0 & Optimal &  0.02 & 71 & 71.00 &  0.00\\
j6048 10.json & 1 & 0 & Optimal &  0.02 & 70 & 70.00 &  0.00\\
j6048 2.json & 1 & 0 & Optimal &  0.02 & 87 & 87.00 &  0.00\\
j6048 3.json & 1 & 0 & Optimal &  0.02 & 84 & 84.00 &  0.00\\
j6048 4.json & 1 & 0 & Optimal &  0.03 & 62 & 62.00 &  0.00\\
j6048 5.json & 1 & 0 & Optimal &  0.03 & 101 & 101.00 &  0.00\\
j6048 6.json & 1 & 0 & Optimal &  0.03 & 66 & 66.00 &  0.00\\
j6048 7.json & 1 & 0 & Optimal &  0.03 & 77 & 77.00 &  0.00\\
j6048 8.json & 1 & 0 & Optimal &  0.02 & 88 & 88.00 &  0.00\\
j6048 9.json & 1 & 0 & Optimal &  0.02 & 82 & 82.00 &  0.00\\
j604 1.json & 1 & 0 & Optimal &  0.02 & 84 & 84.00 &  0.00\\
j604 10.json & 1 & 0 & Optimal &  0.03 & 77 & 77.00 &  0.00\\
j604 2.json & 1 & 0 & Optimal &  0.03 & 60 & 60.00 &  0.00\\
j604 3.json & 1 & 0 & Optimal &  0.03 & 58 & 58.00 &  0.00\\
j604 4.json & 1 & 0 & Optimal &  0.03 & 65 & 65.00 &  0.00\\
j604 5.json & 1 & 0 & Optimal &  0.03 & 75 & 75.00 &  0.00\\
j604 6.json & 1 & 0 & Optimal &  0.03 & 71 & 71.00 &  0.00\\
j604 7.json & 1 & 0 & Optimal &  0.03 & 67 & 67.00 &  0.00\\
j604 8.json & 1 & 0 & Optimal &  0.03 & 65 & 65.00 &  0.00\\
j604 9.json & 1 & 0 & Optimal &  0.03 & 75 & 75.00 &  0.00\\
j605 1.json & 1 & 0 & Optimal & 46.92 & 76 & 76.00 &  0.00\\
j605 10.json & 1 & 0 & Optimal & 600.03 & 81 & 81.00 &  0.00\\
j605 2.json & 1 & 0 & Optimal & 18.55 & 106 & 106.00 &  0.00\\
j605 3.json & 1 & 0 & Optimal &  4.17 & 80 & 80.00 &  0.00\\
j605 4.json & 1 & 0 & Optimal & 49.00 & 72 & 72.00 &  0.00\\
j605 5.json & 1 & 0 & Optimal &  2.75 & 108 & 108.00 &  0.00\\
j605 6.json & 1 & 0 & Optimal &  0.33 & 74 & 74.00 &  0.00\\
j605 7.json & 1 & 0 & Optimal & 22.18 & 75 & 75.00 &  0.00\\
j605 8.json & 1 & 0 & Optimal &  0.38 & 78 & 78.00 &  0.00\\
j605 9.json & 1 & 0 & Optimal &  0.18 & 83 & 83.00 &  0.00\\
j606 1.json & 1 & 0 & Optimal &  0.09 & 60 & 60.00 &  0.00\\
j606 10.json & 1 & 0 & Optimal &  0.03 & 74 & 74.00 &  0.00\\
j606 2.json & 1 & 0 & Optimal &  0.06 & 67 & 67.00 &  0.00\\
j606 3.json & 1 & 0 & Optimal &  0.09 & 72 & 72.00 &  0.00\\
j606 4.json & 1 & 0 & Optimal &  0.04 & 67 & 67.00 &  0.00\\
j606 5.json & 1 & 0 & Optimal &  0.06 & 78 & 78.00 &  0.00\\
j606 6.json & 1 & 0 & Optimal &  0.10 & 55 & 55.00 &  0.00\\
j606 7.json & 1 & 0 & Optimal &  0.07 & 61 & 61.00 &  0.00\\
j606 8.json & 1 & 0 & Optimal &  0.06 & 72 & 72.00 &  0.00\\
j606 9.json & 1 & 0 & Optimal &  0.03 & 64 & 64.00 &  0.00\\
j607 1.json & 1 & 0 & Optimal &  0.02 & 77 & 77.00 &  0.00\\
j607 10.json & 1 & 0 & Optimal &  0.04 & 82 & 82.00 &  0.00\\
j607 2.json & 1 & 0 & Optimal &  0.03 & 85 & 85.00 &  0.00\\
j607 3.json & 1 & 0 & Optimal &  0.02 & 62 & 62.00 &  0.00\\
j607 4.json & 1 & 0 & Optimal &  0.04 & 63 & 63.00 &  0.00\\
j607 5.json & 1 & 0 & Optimal &  0.04 & 71 & 71.00 &  0.00\\
j607 6.json & 1 & 0 & Optimal &  0.04 & 65 & 65.00 &  0.00\\
j607 7.json & 1 & 0 & Optimal &  0.04 & 89 & 89.00 &  0.00\\
j607 8.json & 1 & 0 & Optimal &  0.04 & 66 & 66.00 &  0.00\\
j607 9.json & 1 & 0 & Optimal &  0.04 & 44 & 44.00 &  0.00\\
j608 1.json & 1 & 0 & Optimal &  0.02 & 64 & 64.00 &  0.00\\
j608 10.json & 1 & 0 & Optimal &  0.02 & 97 & 97.00 &  0.00\\
j608 2.json & 1 & 0 & Optimal &  0.02 & 61 & 61.00 &  0.00\\
j608 3.json & 1 & 0 & Optimal &  0.02 & 79 & 79.00 &  0.00\\
j608 4.json & 1 & 0 & Optimal &  0.02 & 64 & 64.00 &  0.00\\
j608 5.json & 1 & 0 & Optimal &  0.03 & 83 & 83.00 &  0.00\\
j608 6.json & 1 & 0 & Optimal &  0.02 & 56 & 56.00 &  0.00\\
j608 7.json & 1 & 0 & Optimal &  0.02 & 62 & 62.00 &  0.00\\
j608 8.json & 1 & 0 & Optimal &  0.02 & 66 & 66.00 &  0.00\\
j608 9.json & 1 & 0 & Optimal &  0.03 & 58 & 58.00 &  0.00\\
j609 1.json & 1 & 0 & Solution & 600.21 & 87 & 84.00 &  3.45\\
j609 10.json & 1 & 0 & Solution & 600.23 & 95 & 86.00 &  9.47\\
j609 2.json & 1 & 0 & Optimal & 130.11 & 82 & 82.00 &  0.00\\
j609 3.json & 1 & 0 & Optimal & 600.03 & 100 & 100.00 &  0.00\\
j609 4.json & 1 & 0 & Optimal & 554.99 & 87 & 87.00 &  0.00\\
j609 5.json & 1 & 0 & Solution & 600.19 & 86 & 80.00 &  6.98\\
j609 6.json & 1 & 0 & Solution & 600.17 & 112 & 100.00 & 10.71\\
j609 7.json & 1 & 0 & Solution & 600.26 & 111 & 103.00 &  7.21\\
j609 8.json & 1 & 0 & Solution & 600.19 & 96 & 90.00 &  6.25\\
j609 9.json & 1 & 0 & Optimal & 600.09 & 99 & 99.00 &  0.00\\
\end{longtable}



\section{Size J90}
\subsection{CPO}
\begin{longtable}{lrrlrrrr}
\caption{Results for RCPSP J90 (CPO) (480 Instances)}\\\toprule
Name & \shortstack{Nr\\Jobs} & \shortstack{Nr\\Machines} & Status & Time & Makespan & Bound & \shortstack{Gap\\Percent}\\ \midrule
\endhead
\bottomrule
\endfoot
j9010 1.json & 1 & 0 & Optimal &  0.13 & 77 & 77.00 &  0.00\\
j9010 10.json & 1 & 0 & Optimal &  0.07 & 75 & 75.00 &  0.00\\
j9010 2.json & 1 & 0 & Optimal &  0.04 & 95 & 95.00 &  0.00\\
j9010 3.json & 1 & 0 & Optimal &  0.03 & 112 & 112.00 &  0.00\\
j9010 4.json & 1 & 0 & Optimal &  0.03 & 94 & 94.00 &  0.00\\
j9010 5.json & 1 & 0 & Optimal &  0.03 & 78 & 78.00 &  0.00\\
j9010 6.json & 1 & 0 & Optimal &  0.04 & 92 & 92.00 &  0.00\\
j9010 7.json & 1 & 0 & Optimal &  0.04 & 83 & 83.00 &  0.00\\
j9010 8.json & 1 & 0 & Optimal &  0.03 & 81 & 81.00 &  0.00\\
j9010 9.json & 1 & 0 & Optimal &  0.03 & 88 & 88.00 &  0.00\\
j9011 1.json & 1 & 0 & Optimal &  0.03 & 86 & 86.00 &  0.00\\
j9011 10.json & 1 & 0 & Optimal &  0.03 & 81 & 81.00 &  0.00\\
j9011 2.json & 1 & 0 & Optimal &  0.04 & 99 & 99.00 &  0.00\\
j9011 3.json & 1 & 0 & Optimal &  0.03 & 69 & 69.00 &  0.00\\
j9011 4.json & 1 & 0 & Optimal &  0.02 & 64 & 64.00 &  0.00\\
j9011 5.json & 1 & 0 & Optimal &  0.03 & 81 & 81.00 &  0.00\\
j9011 6.json & 1 & 0 & Optimal &  0.03 & 78 & 78.00 &  0.00\\
j9011 7.json & 1 & 0 & Optimal &  0.04 & 95 & 95.00 &  0.00\\
j9011 8.json & 1 & 0 & Optimal &  0.04 & 82 & 82.00 &  0.00\\
j9011 9.json & 1 & 0 & Optimal &  0.03 & 81 & 81.00 &  0.00\\
j9012 1.json & 1 & 0 & Optimal &  0.03 & 71 & 71.00 &  0.00\\
j9012 10.json & 1 & 0 & Optimal &  0.03 & 86 & 86.00 &  0.00\\
j9012 2.json & 1 & 0 & Optimal &  0.02 & 71 & 71.00 &  0.00\\
j9012 3.json & 1 & 0 & Optimal &  0.03 & 93 & 93.00 &  0.00\\
j9012 4.json & 1 & 0 & Optimal &  0.02 & 73 & 73.00 &  0.00\\
j9012 5.json & 1 & 0 & Optimal &  0.02 & 83 & 83.00 &  0.00\\
j9012 6.json & 1 & 0 & Optimal &  0.02 & 81 & 81.00 &  0.00\\
j9012 7.json & 1 & 0 & Optimal &  0.03 & 77 & 77.00 &  0.00\\
j9012 8.json & 1 & 0 & Optimal &  0.02 & 83 & 83.00 &  0.00\\
j9012 9.json & 1 & 0 & Optimal &  0.03 & 77 & 77.00 &  0.00\\
j9013 1.json & 1 & 0 & Solution & 30.01 & 143 & 128.00 & 10.49\\
j9013 10.json & 1 & 0 & Solution & 30.02 & 123 & 113.00 &  8.13\\
j9013 2.json & 1 & 0 & Solution & 30.02 & 132 & 119.00 &  9.85\\
j9013 3.json & 1 & 0 & Solution & 30.02 & 110 & 104.00 &  5.45\\
j9013 4.json & 1 & 0 & Solution & 30.02 & 115 & 109.00 &  5.22\\
j9013 5.json & 1 & 0 & Solution & 30.01 & 117 & 108.00 &  7.69\\
j9013 6.json & 1 & 0 & Solution & 30.01 & 127 & 117.00 &  7.87\\
j9013 7.json & 1 & 0 & Solution & 30.02 & 127 & 116.00 &  8.66\\
j9013 8.json & 1 & 0 & Solution & 30.01 & 120 & 113.00 &  5.83\\
j9013 9.json & 1 & 0 & Solution & 30.02 & 127 & 117.00 &  7.87\\
j9014 1.json & 1 & 0 & Optimal &  0.02 & 89 & 89.00 &  0.00\\
j9014 10.json & 1 & 0 & Optimal &  0.04 & 85 & 85.00 &  0.00\\
j9014 2.json & 1 & 0 & Optimal &  0.04 & 79 & 79.00 &  0.00\\
j9014 3.json & 1 & 0 & Optimal &  0.03 & 94 & 94.00 &  0.00\\
j9014 4.json & 1 & 0 & Optimal &  0.03 & 88 & 88.00 &  0.00\\
j9014 5.json & 1 & 0 & Optimal &  0.04 & 84 & 84.00 &  0.00\\
j9014 6.json & 1 & 0 & Optimal &  3.56 & 76 & 76.00 &  0.00\\
j9014 7.json & 1 & 0 & Optimal &  0.03 & 86 & 86.00 &  0.00\\
j9014 8.json & 1 & 0 & Optimal &  0.02 & 80 & 80.00 &  0.00\\
j9014 9.json & 1 & 0 & Optimal &  0.03 & 112 & 112.00 &  0.00\\
j9015 1.json & 1 & 0 & Optimal &  0.04 & 76 & 76.00 &  0.00\\
j9015 10.json & 1 & 0 & Optimal &  0.03 & 78 & 78.00 &  0.00\\
j9015 2.json & 1 & 0 & Optimal &  0.03 & 71 & 71.00 &  0.00\\
j9015 3.json & 1 & 0 & Optimal &  0.03 & 82 & 82.00 &  0.00\\
j9015 4.json & 1 & 0 & Optimal &  0.04 & 92 & 92.00 &  0.00\\
j9015 5.json & 1 & 0 & Optimal &  0.04 & 93 & 93.00 &  0.00\\
j9015 6.json & 1 & 0 & Optimal &  0.03 & 61 & 61.00 &  0.00\\
j9015 7.json & 1 & 0 & Optimal &  0.03 & 82 & 82.00 &  0.00\\
j9015 8.json & 1 & 0 & Optimal &  0.03 & 82 & 82.00 &  0.00\\
j9015 9.json & 1 & 0 & Optimal &  0.04 & 83 & 83.00 &  0.00\\
j9016 1.json & 1 & 0 & Optimal &  0.03 & 85 & 85.00 &  0.00\\
j9016 10.json & 1 & 0 & Optimal &  0.03 & 71 & 71.00 &  0.00\\
j9016 2.json & 1 & 0 & Optimal &  0.02 & 71 & 71.00 &  0.00\\
j9016 3.json & 1 & 0 & Optimal &  0.03 & 73 & 73.00 &  0.00\\
j9016 4.json & 1 & 0 & Optimal &  0.03 & 69 & 69.00 &  0.00\\
j9016 5.json & 1 & 0 & Optimal &  0.03 & 71 & 71.00 &  0.00\\
j9016 6.json & 1 & 0 & Optimal &  0.03 & 74 & 74.00 &  0.00\\
j9016 7.json & 1 & 0 & Optimal &  0.03 & 65 & 65.00 &  0.00\\
j9016 8.json & 1 & 0 & Optimal &  0.03 & 71 & 71.00 &  0.00\\
j9016 9.json & 1 & 0 & Optimal &  0.03 & 66 & 66.00 &  0.00\\
j9017 1.json & 1 & 0 & Optimal &  0.28 & 92 & 92.00 &  0.00\\
j9017 10.json & 1 & 0 & Optimal &  0.47 & 89 & 89.00 &  0.00\\
j9017 2.json & 1 & 0 & Optimal &  0.54 & 100 & 100.00 &  0.00\\
j9017 3.json & 1 & 0 & Optimal &  0.06 & 89 & 89.00 &  0.00\\
j9017 4.json & 1 & 0 & Optimal &  0.04 & 94 & 94.00 &  0.00\\
j9017 5.json & 1 & 0 & Optimal &  0.02 & 113 & 113.00 &  0.00\\
j9017 6.json & 1 & 0 & Optimal &  0.05 & 94 & 94.00 &  0.00\\
j9017 7.json & 1 & 0 & Optimal &  0.02 & 80 & 80.00 &  0.00\\
j9017 8.json & 1 & 0 & Optimal &  0.43 & 113 & 113.00 &  0.00\\
j9017 9.json & 1 & 0 & Optimal &  0.32 & 96 & 96.00 &  0.00\\
j9018 1.json & 1 & 0 & Optimal &  0.02 & 101 & 101.00 &  0.00\\
j9018 10.json & 1 & 0 & Optimal &  0.02 & 94 & 94.00 &  0.00\\
j9018 2.json & 1 & 0 & Optimal &  0.02 & 94 & 94.00 &  0.00\\
j9018 3.json & 1 & 0 & Optimal &  0.01 & 83 & 83.00 &  0.00\\
j9018 4.json & 1 & 0 & Optimal &  0.02 & 98 & 98.00 &  0.00\\
j9018 5.json & 1 & 0 & Optimal &  0.02 & 90 & 90.00 &  0.00\\
j9018 6.json & 1 & 0 & Optimal &  0.01 & 83 & 83.00 &  0.00\\
j9018 7.json & 1 & 0 & Optimal &  0.02 & 73 & 73.00 &  0.00\\
j9018 8.json & 1 & 0 & Optimal &  0.02 & 92 & 92.00 &  0.00\\
j9018 9.json & 1 & 0 & Optimal &  0.02 & 79 & 79.00 &  0.00\\
j9019 1.json & 1 & 0 & Optimal &  0.03 & 98 & 98.00 &  0.00\\
j9019 10.json & 1 & 0 & Optimal &  0.03 & 85 & 85.00 &  0.00\\
j9019 2.json & 1 & 0 & Optimal &  0.03 & 83 & 83.00 &  0.00\\
j9019 3.json & 1 & 0 & Optimal &  0.01 & 89 & 89.00 &  0.00\\
j9019 4.json & 1 & 0 & Optimal &  0.03 & 77 & 77.00 &  0.00\\
j9019 5.json & 1 & 0 & Optimal &  0.02 & 66 & 66.00 &  0.00\\
j9019 6.json & 1 & 0 & Optimal &  0.02 & 136 & 136.00 &  0.00\\
j9019 7.json & 1 & 0 & Optimal &  0.02 & 66 & 66.00 &  0.00\\
j9019 8.json & 1 & 0 & Optimal &  0.03 & 91 & 91.00 &  0.00\\
j9019 9.json & 1 & 0 & Optimal &  0.03 & 121 & 121.00 &  0.00\\
j901 1.json & 1 & 0 & Optimal &  0.46 & 73 & 73.00 &  0.00\\
j901 10.json & 1 & 0 & Optimal &  0.35 & 90 & 90.00 &  0.00\\
j901 2.json & 1 & 0 & Optimal &  0.02 & 92 & 92.00 &  0.00\\
j901 3.json & 1 & 0 & Optimal &  0.58 & 66 & 66.00 &  0.00\\
j901 4.json & 1 & 0 & Optimal &  0.77 & 86 & 86.00 &  0.00\\
j901 5.json & 1 & 0 & Optimal &  0.03 & 87 & 87.00 &  0.00\\
j901 6.json & 1 & 0 & Optimal &  0.31 & 74 & 74.00 &  0.00\\
j901 7.json & 1 & 0 & Optimal &  0.09 & 91 & 91.00 &  0.00\\
j901 8.json & 1 & 0 & Optimal &  0.21 & 95 & 95.00 &  0.00\\
j901 9.json & 1 & 0 & Optimal &  0.16 & 72 & 72.00 &  0.00\\
j9020 1.json & 1 & 0 & Optimal &  0.02 & 85 & 85.00 &  0.00\\
j9020 10.json & 1 & 0 & Optimal &  0.03 & 89 & 89.00 &  0.00\\
j9020 2.json & 1 & 0 & Optimal &  0.04 & 76 & 76.00 &  0.00\\
j9020 3.json & 1 & 0 & Optimal &  0.03 & 86 & 86.00 &  0.00\\
j9020 4.json & 1 & 0 & Optimal &  0.03 & 86 & 86.00 &  0.00\\
j9020 5.json & 1 & 0 & Optimal &  0.02 & 88 & 88.00 &  0.00\\
j9020 6.json & 1 & 0 & Optimal &  0.02 & 83 & 83.00 &  0.00\\
j9020 7.json & 1 & 0 & Optimal &  0.02 & 82 & 82.00 &  0.00\\
j9020 8.json & 1 & 0 & Optimal &  0.03 & 85 & 85.00 &  0.00\\
j9020 9.json & 1 & 0 & Optimal &  0.02 & 76 & 76.00 &  0.00\\
j9021 1.json & 1 & 0 & Solution & 30.03 & 114 & 102.00 & 10.53\\
j9021 10.json & 1 & 0 & Solution & 30.02 & 109 & 105.00 &  3.67\\
j9021 2.json & 1 & 0 & Solution & 30.00 & 117 & 111.00 &  5.13\\
j9021 3.json & 1 & 0 & Solution & 30.01 & 125 & 119.00 &  4.80\\
j9021 4.json & 1 & 0 & Optimal &  8.19 & 106 & 106.00 &  0.00\\
j9021 5.json & 1 & 0 & Solution & 30.01 & 112 & 104.00 &  7.14\\
j9021 6.json & 1 & 0 & Solution & 30.02 & 108 & 104.00 &  3.70\\
j9021 7.json & 1 & 0 & Solution & 30.00 & 112 & 100.00 & 10.71\\
j9021 8.json & 1 & 0 & Solution & 30.02 & 112 & 101.00 &  9.82\\
j9021 9.json & 1 & 0 & Solution & 30.00 & 122 & 110.00 &  9.84\\
j9022 1.json & 1 & 0 & Optimal &  0.03 & 108 & 108.00 &  0.00\\
j9022 10.json & 1 & 0 & Optimal &  0.18 & 75 & 75.00 &  0.00\\
j9022 2.json & 1 & 0 & Optimal &  0.02 & 85 & 85.00 &  0.00\\
j9022 3.json & 1 & 0 & Optimal &  0.17 & 83 & 83.00 &  0.00\\
j9022 4.json & 1 & 0 & Optimal &  0.02 & 96 & 96.00 &  0.00\\
j9022 5.json & 1 & 0 & Optimal &  0.02 & 96 & 96.00 &  0.00\\
j9022 6.json & 1 & 0 & Optimal &  0.03 & 71 & 71.00 &  0.00\\
j9022 7.json & 1 & 0 & Optimal &  0.03 & 90 & 90.00 &  0.00\\
j9022 8.json & 1 & 0 & Optimal &  0.02 & 97 & 97.00 &  0.00\\
j9022 9.json & 1 & 0 & Optimal &  0.03 & 101 & 101.00 &  0.00\\
j9023 1.json & 1 & 0 & Optimal &  0.03 & 90 & 90.00 &  0.00\\
j9023 10.json & 1 & 0 & Optimal &  0.03 & 87 & 87.00 &  0.00\\
j9023 2.json & 1 & 0 & Optimal &  0.03 & 84 & 84.00 &  0.00\\
j9023 3.json & 1 & 0 & Optimal &  0.03 & 116 & 116.00 &  0.00\\
j9023 4.json & 1 & 0 & Optimal &  0.03 & 85 & 85.00 &  0.00\\
j9023 5.json & 1 & 0 & Optimal &  0.02 & 95 & 95.00 &  0.00\\
j9023 6.json & 1 & 0 & Optimal &  0.04 & 87 & 87.00 &  0.00\\
j9023 7.json & 1 & 0 & Optimal &  0.03 & 77 & 77.00 &  0.00\\
j9023 8.json & 1 & 0 & Optimal &  0.03 & 92 & 92.00 &  0.00\\
j9023 9.json & 1 & 0 & Optimal &  0.03 & 126 & 126.00 &  0.00\\
j9024 1.json & 1 & 0 & Optimal &  0.03 & 84 & 84.00 &  0.00\\
j9024 10.json & 1 & 0 & Optimal &  0.03 & 89 & 89.00 &  0.00\\
j9024 2.json & 1 & 0 & Optimal &  0.03 & 92 & 92.00 &  0.00\\
j9024 3.json & 1 & 0 & Optimal &  0.03 & 69 & 69.00 &  0.00\\
j9024 4.json & 1 & 0 & Optimal &  0.02 & 81 & 81.00 &  0.00\\
j9024 5.json & 1 & 0 & Optimal &  0.04 & 85 & 85.00 &  0.00\\
j9024 6.json & 1 & 0 & Optimal &  0.02 & 79 & 79.00 &  0.00\\
j9024 7.json & 1 & 0 & Optimal &  0.02 & 87 & 87.00 &  0.00\\
j9024 8.json & 1 & 0 & Optimal &  0.03 & 88 & 88.00 &  0.00\\
j9024 9.json & 1 & 0 & Optimal &  0.03 & 80 & 80.00 &  0.00\\
j9025 1.json & 1 & 0 & Solution & 30.01 & 131 & 116.00 & 11.45\\
j9025 10.json & 1 & 0 & Solution & 30.01 & 135 & 119.00 & 11.85\\
j9025 2.json & 1 & 0 & Solution & 30.01 & 134 & 122.00 &  8.96\\
j9025 3.json & 1 & 0 & Solution & 30.01 & 128 & 111.00 & 13.28\\
j9025 4.json & 1 & 0 & Solution & 30.01 & 140 & 128.00 &  8.57\\
j9025 5.json & 1 & 0 & Solution & 30.01 & 119 & 109.00 &  8.40\\
j9025 6.json & 1 & 0 & Solution & 30.01 & 124 & 113.00 &  8.87\\
j9025 7.json & 1 & 0 & Solution & 30.01 & 133 & 122.00 &  8.27\\
j9025 8.json & 1 & 0 & Solution & 30.01 & 143 & 130.00 &  9.09\\
j9025 9.json & 1 & 0 & Solution & 30.01 & 109 & 97.00 & 11.01\\
j9026 1.json & 1 & 0 & Optimal &  0.02 & 90 & 90.00 &  0.00\\
j9026 10.json & 1 & 0 & Optimal &  0.03 & 92 & 92.00 &  0.00\\
j9026 2.json & 1 & 0 & Optimal &  0.50 & 85 & 85.00 &  0.00\\
j9026 3.json & 1 & 0 & Optimal &  0.02 & 80 & 80.00 &  0.00\\
j9026 4.json & 1 & 0 & Solution & 30.01 & 98 & 96.00 &  2.04\\
j9026 5.json & 1 & 0 & Solution & 30.01 & 86 & 84.00 &  2.33\\
j9026 6.json & 1 & 0 & Optimal &  0.03 & 108 & 108.00 &  0.00\\
j9026 7.json & 1 & 0 & Optimal &  1.82 & 82 & 82.00 &  0.00\\
j9026 8.json & 1 & 0 & Solution & 30.02 & 83 & 82.00 &  1.20\\
j9026 9.json & 1 & 0 & Optimal &  0.05 & 87 & 87.00 &  0.00\\
j9027 1.json & 1 & 0 & Optimal &  0.02 & 96 & 96.00 &  0.00\\
j9027 10.json & 1 & 0 & Optimal &  0.04 & 97 & 97.00 &  0.00\\
j9027 2.json & 1 & 0 & Optimal &  0.02 & 81 & 81.00 &  0.00\\
j9027 3.json & 1 & 0 & Optimal &  0.03 & 91 & 91.00 &  0.00\\
j9027 4.json & 1 & 0 & Optimal &  0.02 & 79 & 79.00 &  0.00\\
j9027 5.json & 1 & 0 & Optimal &  0.03 & 99 & 99.00 &  0.00\\
j9027 6.json & 1 & 0 & Optimal &  0.04 & 87 & 87.00 &  0.00\\
j9027 7.json & 1 & 0 & Optimal &  0.01 & 73 & 73.00 &  0.00\\
j9027 8.json & 1 & 0 & Optimal &  0.03 & 72 & 72.00 &  0.00\\
j9027 9.json & 1 & 0 & Optimal &  0.03 & 84 & 84.00 &  0.00\\
j9028 1.json & 1 & 0 & Optimal &  0.02 & 80 & 80.00 &  0.00\\
j9028 10.json & 1 & 0 & Optimal &  0.03 & 68 & 68.00 &  0.00\\
j9028 2.json & 1 & 0 & Optimal &  0.03 & 76 & 76.00 &  0.00\\
j9028 3.json & 1 & 0 & Optimal &  0.02 & 86 & 86.00 &  0.00\\
j9028 4.json & 1 & 0 & Optimal &  0.03 & 78 & 78.00 &  0.00\\
j9028 5.json & 1 & 0 & Optimal &  0.02 & 88 & 88.00 &  0.00\\
j9028 6.json & 1 & 0 & Optimal &  0.02 & 102 & 102.00 &  0.00\\
j9028 7.json & 1 & 0 & Optimal &  0.04 & 97 & 97.00 &  0.00\\
j9028 8.json & 1 & 0 & Optimal &  0.03 & 110 & 110.00 &  0.00\\
j9028 9.json & 1 & 0 & Optimal &  0.02 & 120 & 120.00 &  0.00\\
j9029 1.json & 1 & 0 & Solution & 30.00 & 138 & 125.00 &  9.42\\
j9029 10.json & 1 & 0 & Solution & 30.02 & 128 & 118.00 &  7.81\\
j9029 2.json & 1 & 0 & Solution & 30.01 & 132 & 121.00 &  8.33\\
j9029 3.json & 1 & 0 & Solution & 30.02 & 147 & 135.00 &  8.16\\
j9029 4.json & 1 & 0 & Solution & 30.01 & 153 & 138.00 &  9.80\\
j9029 5.json & 1 & 0 & Solution & 30.01 & 125 & 115.00 &  8.00\\
j9029 6.json & 1 & 0 & Solution & 30.01 & 127 & 116.00 &  8.66\\
j9029 7.json & 1 & 0 & Solution & 30.01 & 176 & 156.00 & 11.36\\
j9029 8.json & 1 & 0 & Solution & 30.01 & 160 & 146.00 &  8.75\\
j9029 9.json & 1 & 0 & Solution & 30.01 & 132 & 119.00 &  9.85\\
j902 1.json & 1 & 0 & Optimal &  0.02 & 96 & 96.00 &  0.00\\
j902 10.json & 1 & 0 & Optimal &  0.02 & 80 & 80.00 &  0.00\\
j902 2.json & 1 & 0 & Optimal &  0.02 & 114 & 114.00 &  0.00\\
j902 3.json & 1 & 0 & Optimal &  0.02 & 75 & 75.00 &  0.00\\
j902 4.json & 1 & 0 & Optimal &  0.02 & 70 & 70.00 &  0.00\\
j902 5.json & 1 & 0 & Optimal &  0.02 & 100 & 100.00 &  0.00\\
j902 6.json & 1 & 0 & Optimal &  0.02 & 67 & 67.00 &  0.00\\
j902 7.json & 1 & 0 & Optimal &  0.03 & 92 & 92.00 &  0.00\\
j902 8.json & 1 & 0 & Optimal &  0.02 & 82 & 82.00 &  0.00\\
j902 9.json & 1 & 0 & Optimal &  0.03 & 79 & 79.00 &  0.00\\
j9030 1.json & 1 & 0 & Optimal &  0.03 & 102 & 102.00 &  0.00\\
j9030 10.json & 1 & 0 & Optimal &  0.03 & 90 & 90.00 &  0.00\\
j9030 2.json & 1 & 0 & Optimal &  0.02 & 76 & 76.00 &  0.00\\
j9030 3.json & 1 & 0 & Optimal &  0.05 & 102 & 102.00 &  0.00\\
j9030 4.json & 1 & 0 & Optimal &  0.04 & 104 & 104.00 &  0.00\\
j9030 5.json & 1 & 0 & Solution & 30.02 & 84 & 83.00 &  1.19\\
j9030 6.json & 1 & 0 & Optimal &  0.03 & 90 & 90.00 &  0.00\\
j9030 7.json & 1 & 0 & Solution & 30.01 & 85 & 84.00 &  1.18\\
j9030 8.json & 1 & 0 & Optimal &  0.03 & 82 & 82.00 &  0.00\\
j9030 9.json & 1 & 0 & Solution & 30.01 & 95 & 91.00 &  4.21\\
j9031 1.json & 1 & 0 & Optimal &  0.02 & 79 & 79.00 &  0.00\\
j9031 10.json & 1 & 0 & Optimal &  0.03 & 99 & 99.00 &  0.00\\
j9031 2.json & 1 & 0 & Optimal &  0.03 & 69 & 69.00 &  0.00\\
j9031 3.json & 1 & 0 & Optimal &  0.03 & 106 & 106.00 &  0.00\\
j9031 4.json & 1 & 0 & Optimal &  0.03 & 79 & 79.00 &  0.00\\
j9031 5.json & 1 & 0 & Optimal &  0.03 & 79 & 79.00 &  0.00\\
j9031 6.json & 1 & 0 & Optimal &  0.03 & 80 & 80.00 &  0.00\\
j9031 7.json & 1 & 0 & Optimal &  0.03 & 97 & 97.00 &  0.00\\
j9031 8.json & 1 & 0 & Optimal &  0.03 & 83 & 83.00 &  0.00\\
j9031 9.json & 1 & 0 & Optimal &  0.03 & 72 & 72.00 &  0.00\\
j9032 1.json & 1 & 0 & Optimal &  0.03 & 78 & 78.00 &  0.00\\
j9032 10.json & 1 & 0 & Optimal &  0.03 & 91 & 91.00 &  0.00\\
j9032 2.json & 1 & 0 & Optimal &  0.03 & 78 & 78.00 &  0.00\\
j9032 3.json & 1 & 0 & Optimal &  0.03 & 89 & 89.00 &  0.00\\
j9032 4.json & 1 & 0 & Optimal &  0.03 & 104 & 104.00 &  0.00\\
j9032 5.json & 1 & 0 & Optimal &  0.03 & 93 & 93.00 &  0.00\\
j9032 6.json & 1 & 0 & Optimal &  0.03 & 86 & 86.00 &  0.00\\
j9032 7.json & 1 & 0 & Optimal &  0.03 & 87 & 87.00 &  0.00\\
j9032 8.json & 1 & 0 & Optimal &  0.03 & 79 & 79.00 &  0.00\\
j9032 9.json & 1 & 0 & Optimal &  0.03 & 95 & 95.00 &  0.00\\
j9033 1.json & 1 & 0 & Optimal &  0.30 & 99 & 99.00 &  0.00\\
j9033 10.json & 1 & 0 & Optimal &  0.08 & 114 & 114.00 &  0.00\\
j9033 2.json & 1 & 0 & Optimal &  0.05 & 112 & 112.00 &  0.00\\
j9033 3.json & 1 & 0 & Optimal &  0.02 & 108 & 108.00 &  0.00\\
j9033 4.json & 1 & 0 & Optimal &  0.08 & 92 & 92.00 &  0.00\\
j9033 5.json & 1 & 0 & Optimal &  0.18 & 109 & 109.00 &  0.00\\
j9033 6.json & 1 & 0 & Optimal &  0.03 & 88 & 88.00 &  0.00\\
j9033 7.json & 1 & 0 & Optimal &  0.10 & 109 & 109.00 &  0.00\\
j9033 8.json & 1 & 0 & Optimal &  0.17 & 110 & 110.00 &  0.00\\
j9033 9.json & 1 & 0 & Optimal &  0.56 & 95 & 95.00 &  0.00\\
j9034 1.json & 1 & 0 & Optimal &  0.02 & 83 & 83.00 &  0.00\\
j9034 10.json & 1 & 0 & Optimal &  0.02 & 101 & 101.00 &  0.00\\
j9034 2.json & 1 & 0 & Optimal &  0.02 & 89 & 89.00 &  0.00\\
j9034 3.json & 1 & 0 & Optimal &  0.02 & 82 & 82.00 &  0.00\\
j9034 4.json & 1 & 0 & Optimal &  0.12 & 81 & 81.00 &  0.00\\
j9034 5.json & 1 & 0 & Optimal &  0.07 & 83 & 83.00 &  0.00\\
j9034 6.json & 1 & 0 & Optimal &  0.02 & 89 & 89.00 &  0.00\\
j9034 7.json & 1 & 0 & Optimal &  0.02 & 92 & 92.00 &  0.00\\
j9034 8.json & 1 & 0 & Optimal &  0.02 & 81 & 81.00 &  0.00\\
j9034 9.json & 1 & 0 & Optimal &  0.02 & 109 & 109.00 &  0.00\\
j9035 1.json & 1 & 0 & Optimal &  0.02 & 98 & 98.00 &  0.00\\
j9035 10.json & 1 & 0 & Optimal &  0.02 & 82 & 82.00 &  0.00\\
j9035 2.json & 1 & 0 & Optimal &  0.02 & 92 & 92.00 &  0.00\\
j9035 3.json & 1 & 0 & Optimal &  0.02 & 96 & 96.00 &  0.00\\
j9035 4.json & 1 & 0 & Optimal &  0.02 & 86 & 86.00 &  0.00\\
j9035 5.json & 1 & 0 & Optimal &  0.02 & 103 & 103.00 &  0.00\\
j9035 6.json & 1 & 0 & Optimal &  0.02 & 72 & 72.00 &  0.00\\
j9035 7.json & 1 & 0 & Optimal &  0.02 & 78 & 78.00 &  0.00\\
j9035 8.json & 1 & 0 & Optimal &  0.02 & 85 & 85.00 &  0.00\\
j9035 9.json & 1 & 0 & Optimal &  0.02 & 76 & 76.00 &  0.00\\
j9036 1.json & 1 & 0 & Optimal &  0.02 & 97 & 97.00 &  0.00\\
j9036 10.json & 1 & 0 & Optimal &  0.02 & 109 & 109.00 &  0.00\\
j9036 2.json & 1 & 0 & Optimal &  0.02 & 114 & 114.00 &  0.00\\
j9036 3.json & 1 & 0 & Optimal &  0.02 & 84 & 84.00 &  0.00\\
j9036 4.json & 1 & 0 & Optimal &  0.03 & 79 & 79.00 &  0.00\\
j9036 5.json & 1 & 0 & Optimal &  0.02 & 98 & 98.00 &  0.00\\
j9036 6.json & 1 & 0 & Optimal &  0.02 & 99 & 99.00 &  0.00\\
j9036 7.json & 1 & 0 & Optimal &  0.02 & 89 & 89.00 &  0.00\\
j9036 8.json & 1 & 0 & Optimal &  0.02 & 84 & 84.00 &  0.00\\
j9036 9.json & 1 & 0 & Optimal &  0.02 & 102 & 102.00 &  0.00\\
j9037 1.json & 1 & 0 & Solution & 30.01 & 112 & 101.00 &  9.82\\
j9037 10.json & 1 & 0 & Solution & 30.02 & 123 & 108.00 & 12.20\\
j9037 2.json & 1 & 0 & Solution & 30.01 & 115 & 106.00 &  7.83\\
j9037 3.json & 1 & 0 & Optimal &  5.87 & 132 & 132.00 &  0.00\\
j9037 4.json & 1 & 0 & Optimal & 20.92 & 123 & 123.00 &  0.00\\
j9037 5.json & 1 & 0 & Solution & 30.01 & 127 & 114.00 & 10.24\\
j9037 6.json & 1 & 0 & Solution & 30.01 & 133 & 120.00 &  9.77\\
j9037 7.json & 1 & 0 & Optimal &  9.09 & 123 & 123.00 &  0.00\\
j9037 8.json & 1 & 0 & Solution & 30.01 & 120 & 106.00 & 11.67\\
j9037 9.json & 1 & 0 & Optimal &  7.51 & 123 & 123.00 &  0.00\\
j9038 1.json & 1 & 0 & Optimal &  0.40 & 85 & 85.00 &  0.00\\
j9038 10.json & 1 & 0 & Optimal &  0.02 & 108 & 108.00 &  0.00\\
j9038 2.json & 1 & 0 & Optimal &  0.03 & 78 & 78.00 &  0.00\\
j9038 3.json & 1 & 0 & Optimal &  0.47 & 89 & 89.00 &  0.00\\
j9038 4.json & 1 & 0 & Optimal &  0.03 & 89 & 89.00 &  0.00\\
j9038 5.json & 1 & 0 & Optimal &  0.42 & 86 & 86.00 &  0.00\\
j9038 6.json & 1 & 0 & Optimal &  0.07 & 88 & 88.00 &  0.00\\
j9038 7.json & 1 & 0 & Optimal &  0.03 & 85 & 85.00 &  0.00\\
j9038 8.json & 1 & 0 & Optimal &  0.02 & 91 & 91.00 &  0.00\\
j9038 9.json & 1 & 0 & Optimal &  0.03 & 95 & 95.00 &  0.00\\
j9039 1.json & 1 & 0 & Optimal &  0.02 & 106 & 106.00 &  0.00\\
j9039 10.json & 1 & 0 & Optimal &  0.02 & 100 & 100.00 &  0.00\\
j9039 2.json & 1 & 0 & Optimal &  0.02 & 119 & 119.00 &  0.00\\
j9039 3.json & 1 & 0 & Optimal &  0.03 & 83 & 83.00 &  0.00\\
j9039 4.json & 1 & 0 & Optimal &  0.03 & 81 & 81.00 &  0.00\\
j9039 5.json & 1 & 0 & Optimal &  0.03 & 85 & 85.00 &  0.00\\
j9039 6.json & 1 & 0 & Optimal &  0.02 & 102 & 102.00 &  0.00\\
j9039 7.json & 1 & 0 & Optimal &  0.02 & 85 & 85.00 &  0.00\\
j9039 8.json & 1 & 0 & Optimal &  0.03 & 81 & 81.00 &  0.00\\
j9039 9.json & 1 & 0 & Optimal &  0.02 & 79 & 79.00 &  0.00\\
j903 1.json & 1 & 0 & Optimal &  0.02 & 81 & 81.00 &  0.00\\
j903 10.json & 1 & 0 & Optimal &  0.02 & 65 & 65.00 &  0.00\\
j903 2.json & 1 & 0 & Optimal &  0.02 & 84 & 84.00 &  0.00\\
j903 3.json & 1 & 0 & Optimal &  0.02 & 71 & 71.00 &  0.00\\
j903 4.json & 1 & 0 & Optimal &  0.02 & 104 & 104.00 &  0.00\\
j903 5.json & 1 & 0 & Optimal &  0.02 & 75 & 75.00 &  0.00\\
j903 6.json & 1 & 0 & Optimal &  0.02 & 68 & 68.00 &  0.00\\
j903 7.json & 1 & 0 & Optimal &  0.02 & 87 & 87.00 &  0.00\\
j903 8.json & 1 & 0 & Optimal &  0.02 & 86 & 86.00 &  0.00\\
j903 9.json & 1 & 0 & Optimal &  0.02 & 61 & 61.00 &  0.00\\
j9040 1.json & 1 & 0 & Optimal &  0.02 & 95 & 95.00 &  0.00\\
j9040 10.json & 1 & 0 & Optimal &  0.02 & 86 & 86.00 &  0.00\\
j9040 2.json & 1 & 0 & Optimal &  0.02 & 91 & 91.00 &  0.00\\
j9040 3.json & 1 & 0 & Optimal &  0.02 & 77 & 77.00 &  0.00\\
j9040 4.json & 1 & 0 & Optimal &  0.02 & 106 & 106.00 &  0.00\\
j9040 5.json & 1 & 0 & Optimal &  0.03 & 92 & 92.00 &  0.00\\
j9040 6.json & 1 & 0 & Optimal &  0.03 & 86 & 86.00 &  0.00\\
j9040 7.json & 1 & 0 & Optimal &  0.02 & 87 & 87.00 &  0.00\\
j9040 8.json & 1 & 0 & Optimal &  0.03 & 79 & 79.00 &  0.00\\
j9040 9.json & 1 & 0 & Optimal &  0.02 & 98 & 98.00 &  0.00\\
j9041 1.json & 1 & 0 & Solution & 30.01 & 147 & 128.00 & 12.93\\
j9041 10.json & 1 & 0 & Solution & 30.01 & 154 & 143.00 &  7.14\\
j9041 2.json & 1 & 0 & Solution & 30.01 & 170 & 153.00 & 10.00\\
j9041 3.json & 1 & 0 & Solution & 30.01 & 164 & 145.00 & 11.59\\
j9041 4.json & 1 & 0 & Solution & 30.01 & 158 & 140.00 & 11.39\\
j9041 5.json & 1 & 0 & Solution & 30.01 & 129 & 115.00 & 10.85\\
j9041 6.json & 1 & 0 & Solution & 30.01 & 137 & 127.00 &  7.30\\
j9041 7.json & 1 & 0 & Solution & 30.01 & 158 & 142.00 & 10.13\\
j9041 8.json & 1 & 0 & Solution & 30.01 & 168 & 147.00 & 12.50\\
j9041 9.json & 1 & 0 & Solution & 30.01 & 125 & 110.00 & 12.00\\
j9042 1.json & 1 & 0 & Optimal &  0.03 & 106 & 106.00 &  0.00\\
j9042 10.json & 1 & 0 & Solution & 30.01 & 91 & 89.00 &  2.20\\
j9042 2.json & 1 & 0 & Solution & 30.01 & 103 & 101.00 &  1.94\\
j9042 3.json & 1 & 0 & Optimal &  0.12 & 94 & 94.00 &  0.00\\
j9042 4.json & 1 & 0 & Optimal &  0.03 & 102 & 102.00 &  0.00\\
j9042 5.json & 1 & 0 & Optimal &  0.03 & 105 & 105.00 &  0.00\\
j9042 6.json & 1 & 0 & Optimal &  0.03 & 89 & 89.00 &  0.00\\
j9042 7.json & 1 & 0 & Solution & 30.00 & 87 & 86.00 &  1.15\\
j9042 8.json & 1 & 0 & Optimal &  0.02 & 105 & 105.00 &  0.00\\
j9042 9.json & 1 & 0 & Optimal &  2.78 & 83 & 83.00 &  0.00\\
j9043 1.json & 1 & 0 & Optimal &  0.02 & 99 & 99.00 &  0.00\\
j9043 10.json & 1 & 0 & Optimal &  0.02 & 92 & 92.00 &  0.00\\
j9043 2.json & 1 & 0 & Optimal &  0.02 & 91 & 91.00 &  0.00\\
j9043 3.json & 1 & 0 & Optimal &  0.01 & 102 & 102.00 &  0.00\\
j9043 4.json & 1 & 0 & Optimal &  0.03 & 94 & 94.00 &  0.00\\
j9043 5.json & 1 & 0 & Optimal &  0.03 & 98 & 98.00 &  0.00\\
j9043 6.json & 1 & 0 & Optimal &  0.01 & 114 & 114.00 &  0.00\\
j9043 7.json & 1 & 0 & Optimal &  0.02 & 88 & 88.00 &  0.00\\
j9043 8.json & 1 & 0 & Optimal &  0.02 & 100 & 100.00 &  0.00\\
j9043 9.json & 1 & 0 & Optimal &  0.03 & 88 & 88.00 &  0.00\\
j9044 1.json & 1 & 0 & Optimal &  0.03 & 100 & 100.00 &  0.00\\
j9044 10.json & 1 & 0 & Optimal &  0.03 & 86 & 86.00 &  0.00\\
j9044 2.json & 1 & 0 & Optimal &  0.03 & 92 & 92.00 &  0.00\\
j9044 3.json & 1 & 0 & Optimal &  0.03 & 110 & 110.00 &  0.00\\
j9044 4.json & 1 & 0 & Optimal &  0.02 & 89 & 89.00 &  0.00\\
j9044 5.json & 1 & 0 & Optimal &  0.03 & 84 & 84.00 &  0.00\\
j9044 6.json & 1 & 0 & Optimal &  0.02 & 96 & 96.00 &  0.00\\
j9044 7.json & 1 & 0 & Optimal &  0.02 & 93 & 93.00 &  0.00\\
j9044 8.json & 1 & 0 & Optimal &  0.03 & 99 & 99.00 &  0.00\\
j9044 9.json & 1 & 0 & Optimal &  0.03 & 96 & 96.00 &  0.00\\
j9045 1.json & 1 & 0 & Solution & 30.02 & 151 & 142.00 &  5.96\\
j9045 10.json & 1 & 0 & Solution & 30.01 & 170 & 156.00 &  8.24\\
j9045 2.json & 1 & 0 & Solution & 30.01 & 150 & 138.00 &  8.00\\
j9045 3.json & 1 & 0 & Solution & 30.01 & 160 & 142.00 & 11.25\\
j9045 4.json & 1 & 0 & Solution & 30.00 & 139 & 125.00 & 10.07\\
j9045 5.json & 1 & 0 & Solution & 30.00 & 180 & 163.00 &  9.44\\
j9045 6.json & 1 & 0 & Solution & 30.01 & 178 & 157.00 & 11.80\\
j9045 7.json & 1 & 0 & Solution & 30.02 & 140 & 127.00 &  9.29\\
j9045 8.json & 1 & 0 & Solution & 30.01 & 163 & 147.00 &  9.82\\
j9045 9.json & 1 & 0 & Solution & 30.02 & 161 & 141.00 & 12.42\\
j9046 1.json & 1 & 0 & Optimal &  4.15 & 104 & 104.00 &  0.00\\
j9046 10.json & 1 & 0 & Optimal &  0.02 & 114 & 114.00 &  0.00\\
j9046 2.json & 1 & 0 & Optimal &  0.03 & 98 & 98.00 &  0.00\\
j9046 3.json & 1 & 0 & Optimal &  2.76 & 113 & 113.00 &  0.00\\
j9046 4.json & 1 & 0 & Solution & 30.02 & 93 & 92.00 &  1.08\\
j9046 5.json & 1 & 0 & Optimal &  0.02 & 91 & 91.00 &  0.00\\
j9046 6.json & 1 & 0 & Optimal &  0.05 & 83 & 83.00 &  0.00\\
j9046 7.json & 1 & 0 & Optimal &  0.05 & 89 & 89.00 &  0.00\\
j9046 8.json & 1 & 0 & Solution & 30.01 & 97 & 93.00 &  4.12\\
j9046 9.json & 1 & 0 & Solution & 30.02 & 90 & 86.00 &  4.44\\
j9047 1.json & 1 & 0 & Optimal &  0.03 & 82 & 82.00 &  0.00\\
j9047 10.json & 1 & 0 & Optimal &  0.03 & 65 & 65.00 &  0.00\\
j9047 2.json & 1 & 0 & Optimal &  0.03 & 90 & 90.00 &  0.00\\
j9047 3.json & 1 & 0 & Optimal &  0.02 & 102 & 102.00 &  0.00\\
j9047 4.json & 1 & 0 & Optimal &  0.03 & 93 & 93.00 &  0.00\\
j9047 5.json & 1 & 0 & Optimal &  0.03 & 93 & 93.00 &  0.00\\
j9047 6.json & 1 & 0 & Optimal &  0.03 & 98 & 98.00 &  0.00\\
j9047 7.json & 1 & 0 & Optimal &  0.03 & 94 & 94.00 &  0.00\\
j9047 8.json & 1 & 0 & Optimal &  0.03 & 98 & 98.00 &  0.00\\
j9047 9.json & 1 & 0 & Optimal &  0.03 & 86 & 86.00 &  0.00\\
j9048 1.json & 1 & 0 & Optimal &  0.03 & 83 & 83.00 &  0.00\\
j9048 10.json & 1 & 0 & Optimal &  0.02 & 93 & 93.00 &  0.00\\
j9048 2.json & 1 & 0 & Optimal &  0.02 & 89 & 89.00 &  0.00\\
j9048 3.json & 1 & 0 & Optimal &  0.01 & 86 & 86.00 &  0.00\\
j9048 4.json & 1 & 0 & Optimal &  0.02 & 91 & 91.00 &  0.00\\
j9048 5.json & 1 & 0 & Optimal &  0.03 & 75 & 75.00 &  0.00\\
j9048 6.json & 1 & 0 & Optimal &  0.03 & 114 & 114.00 &  0.00\\
j9048 7.json & 1 & 0 & Optimal &  0.03 & 103 & 103.00 &  0.00\\
j9048 8.json & 1 & 0 & Optimal &  0.03 & 74 & 74.00 &  0.00\\
j9048 9.json & 1 & 0 & Optimal &  0.03 & 89 & 89.00 &  0.00\\
j904 1.json & 1 & 0 & Optimal &  0.03 & 93 & 93.00 &  0.00\\
j904 10.json & 1 & 0 & Optimal &  0.02 & 68 & 68.00 &  0.00\\
j904 2.json & 1 & 0 & Optimal &  0.02 & 89 & 89.00 &  0.00\\
j904 3.json & 1 & 0 & Optimal &  0.02 & 67 & 67.00 &  0.00\\
j904 4.json & 1 & 0 & Optimal &  0.02 & 92 & 92.00 &  0.00\\
j904 5.json & 1 & 0 & Optimal &  0.03 & 88 & 88.00 &  0.00\\
j904 6.json & 1 & 0 & Optimal &  0.03 & 78 & 78.00 &  0.00\\
j904 7.json & 1 & 0 & Optimal &  0.02 & 80 & 80.00 &  0.00\\
j904 8.json & 1 & 0 & Optimal &  0.02 & 69 & 69.00 &  0.00\\
j904 9.json & 1 & 0 & Optimal &  0.03 & 79 & 79.00 &  0.00\\
j905 1.json & 1 & 0 & Optimal &  7.71 & 78 & 78.00 &  0.00\\
j905 10.json & 1 & 0 & Solution & 30.01 & 97 & 94.00 &  3.09\\
j905 2.json & 1 & 0 & Optimal & 12.63 & 93 & 93.00 &  0.00\\
j905 3.json & 1 & 0 & Solution & 30.01 & 89 & 84.00 &  5.62\\
j905 4.json & 1 & 0 & Solution & 30.01 & 103 & 98.00 &  4.85\\
j905 5.json & 1 & 0 & Solution & 30.02 & 113 & 109.00 &  3.54\\
j905 6.json & 1 & 0 & Solution & 30.02 & 88 & 85.00 &  3.41\\
j905 7.json & 1 & 0 & Solution & 30.01 & 110 & 106.00 &  3.64\\
j905 8.json & 1 & 0 & Solution & 30.01 & 104 & 95.00 &  8.65\\
j905 9.json & 1 & 0 & Solution & 30.01 & 119 & 109.00 &  8.40\\
j906 1.json & 1 & 0 & Optimal &  0.02 & 82 & 82.00 &  0.00\\
j906 10.json & 1 & 0 & Optimal &  0.02 & 94 & 94.00 &  0.00\\
j906 2.json & 1 & 0 & Optimal &  0.02 & 86 & 86.00 &  0.00\\
j906 3.json & 1 & 0 & Optimal &  0.59 & 77 & 77.00 &  0.00\\
j906 4.json & 1 & 0 & Optimal &  0.02 & 80 & 80.00 &  0.00\\
j906 5.json & 1 & 0 & Optimal &  0.02 & 71 & 71.00 &  0.00\\
j906 6.json & 1 & 0 & Optimal &  0.02 & 98 & 98.00 &  0.00\\
j906 7.json & 1 & 0 & Optimal &  0.02 & 71 & 71.00 &  0.00\\
j906 8.json & 1 & 0 & Optimal &  1.79 & 68 & 68.00 &  0.00\\
j906 9.json & 1 & 0 & Optimal &  0.02 & 68 & 68.00 &  0.00\\
j907 1.json & 1 & 0 & Optimal &  0.02 & 88 & 88.00 &  0.00\\
j907 10.json & 1 & 0 & Optimal &  0.02 & 98 & 98.00 &  0.00\\
j907 2.json & 1 & 0 & Optimal &  0.02 & 77 & 77.00 &  0.00\\
j907 3.json & 1 & 0 & Optimal &  0.02 & 80 & 80.00 &  0.00\\
j907 4.json & 1 & 0 & Optimal &  0.02 & 86 & 86.00 &  0.00\\
j907 5.json & 1 & 0 & Optimal &  0.02 & 79 & 79.00 &  0.00\\
j907 6.json & 1 & 0 & Optimal &  0.03 & 90 & 90.00 &  0.00\\
j907 7.json & 1 & 0 & Optimal &  0.03 & 90 & 90.00 &  0.00\\
j907 8.json & 1 & 0 & Optimal &  0.02 & 60 & 60.00 &  0.00\\
j907 9.json & 1 & 0 & Optimal &  0.02 & 83 & 83.00 &  0.00\\
j908 1.json & 1 & 0 & Optimal &  0.02 & 96 & 96.00 &  0.00\\
j908 10.json & 1 & 0 & Optimal &  0.02 & 88 & 88.00 &  0.00\\
j908 2.json & 1 & 0 & Optimal &  0.03 & 78 & 78.00 &  0.00\\
j908 3.json & 1 & 0 & Optimal &  0.02 & 70 & 70.00 &  0.00\\
j908 4.json & 1 & 0 & Optimal &  0.02 & 77 & 77.00 &  0.00\\
j908 5.json & 1 & 0 & Optimal &  0.03 & 63 & 63.00 &  0.00\\
j908 6.json & 1 & 0 & Optimal &  0.02 & 70 & 70.00 &  0.00\\
j908 7.json & 1 & 0 & Optimal &  0.02 & 77 & 77.00 &  0.00\\
j908 8.json & 1 & 0 & Optimal &  0.02 & 68 & 68.00 &  0.00\\
j908 9.json & 1 & 0 & Optimal &  0.02 & 97 & 97.00 &  0.00\\
j909 1.json & 1 & 0 & Solution & 30.01 & 108 & 99.00 &  8.33\\
j909 10.json & 1 & 0 & Solution & 30.01 & 115 & 104.00 &  9.57\\
j909 2.json & 1 & 0 & Solution & 30.01 & 133 & 120.00 &  9.77\\
j909 3.json & 1 & 0 & Solution & 30.01 & 106 & 98.00 &  7.55\\
j909 4.json & 1 & 0 & Solution & 30.01 & 131 & 119.00 &  9.16\\
j909 5.json & 1 & 0 & Solution & 30.00 & 143 & 123.00 & 13.99\\
j909 6.json & 1 & 0 & Solution & 30.01 & 122 & 112.00 &  8.20\\
j909 7.json & 1 & 0 & Solution & 30.01 & 110 & 103.00 &  6.36\\
j909 8.json & 1 & 0 & Solution & 30.02 & 119 & 110.00 &  7.56\\
j909 9.json & 1 & 0 & Solution & 30.01 & 119 & 106.00 & 10.92\\
\end{longtable}



\subsection{CPSat}
\begin{longtable}{lrrlrrrr}
\caption{Results for RCPSP J90 (CPSat) (480 Instances)}\\\toprule
Name & \shortstack{Nr\\Jobs} & \shortstack{Nr\\Machines} & Status & Time & Makespan & Bound & \shortstack{Gap\\Percent}\\ \midrule
\endhead
\bottomrule
\endfoot
j9010 1.json & 1 & 0 & Optimal &  0.06 & 77 & 77.00 &  0.00\\
j9010 10.json & 1 & 0 & Optimal &  0.05 & 75 & 75.00 &  0.00\\
j9010 2.json & 1 & 0 & Optimal &  0.08 & 95 & 95.00 &  0.00\\
j9010 3.json & 1 & 0 & Optimal &  0.07 & 112 & 112.00 &  0.00\\
j9010 4.json & 1 & 0 & Optimal &  0.05 & 94 & 94.00 &  0.00\\
j9010 5.json & 1 & 0 & Optimal &  0.05 & 78 & 78.00 &  0.00\\
j9010 6.json & 1 & 0 & Optimal &  0.03 & 92 & 92.00 &  0.00\\
j9010 7.json & 1 & 0 & Optimal &  0.05 & 83 & 83.00 &  0.00\\
j9010 8.json & 1 & 0 & Optimal &  0.09 & 81 & 81.00 &  0.00\\
j9010 9.json & 1 & 0 & Optimal &  0.04 & 88 & 88.00 &  0.00\\
j9011 1.json & 1 & 0 & Optimal &  0.06 & 86 & 86.00 &  0.00\\
j9011 10.json & 1 & 0 & Optimal &  0.09 & 81 & 81.00 &  0.00\\
j9011 2.json & 1 & 0 & Optimal &  0.07 & 99 & 99.00 &  0.00\\
j9011 3.json & 1 & 0 & Optimal &  0.05 & 69 & 69.00 &  0.00\\
j9011 4.json & 1 & 0 & Optimal &  0.05 & 64 & 64.00 &  0.00\\
j9011 5.json & 1 & 0 & Optimal &  0.02 & 81 & 81.00 &  0.00\\
j9011 6.json & 1 & 0 & Optimal &  0.04 & 78 & 78.00 &  0.00\\
j9011 7.json & 1 & 0 & Optimal &  0.02 & 95 & 95.00 &  0.00\\
j9011 8.json & 1 & 0 & Optimal &  0.04 & 82 & 82.00 &  0.00\\
j9011 9.json & 1 & 0 & Optimal &  0.04 & 81 & 81.00 &  0.00\\
j9012 1.json & 1 & 0 & Optimal &  0.02 & 71 & 71.00 &  0.00\\
j9012 10.json & 1 & 0 & Optimal &  0.02 & 86 & 86.00 &  0.00\\
j9012 2.json & 1 & 0 & Optimal &  0.02 & 71 & 71.00 &  0.00\\
j9012 3.json & 1 & 0 & Optimal &  0.02 & 93 & 93.00 &  0.00\\
j9012 4.json & 1 & 0 & Optimal &  0.02 & 73 & 73.00 &  0.00\\
j9012 5.json & 1 & 0 & Optimal &  0.03 & 83 & 83.00 &  0.00\\
j9012 6.json & 1 & 0 & Optimal &  0.02 & 81 & 81.00 &  0.00\\
j9012 7.json & 1 & 0 & Optimal &  0.03 & 77 & 77.00 &  0.00\\
j9012 8.json & 1 & 0 & Optimal &  0.02 & 83 & 83.00 &  0.00\\
j9012 9.json & 1 & 0 & Optimal &  0.02 & 77 & 77.00 &  0.00\\
j9013 1.json & 1 & 0 & Solution & 30.02 & 144 & 127.00 & 11.81\\
j9013 10.json & 1 & 0 & Solution & 30.04 & 127 & 112.00 & 11.81\\
j9013 2.json & 1 & 0 & Solution & 30.02 & 134 & 116.00 & 13.43\\
j9013 3.json & 1 & 0 & Solution & 30.03 & 112 & 104.00 &  7.14\\
j9013 4.json & 1 & 0 & Solution & 30.03 & 118 & 108.00 &  8.47\\
j9013 5.json & 1 & 0 & Solution & 30.02 & 119 & 108.00 &  9.24\\
j9013 6.json & 1 & 0 & Solution & 30.03 & 131 & 116.00 & 11.45\\
j9013 7.json & 1 & 0 & Solution & 30.02 & 131 & 114.00 & 12.98\\
j9013 8.json & 1 & 0 & Solution & 30.03 & 124 & 112.00 &  9.68\\
j9013 9.json & 1 & 0 & Solution & 30.01 & 128 & 115.00 & 10.16\\
j9014 1.json & 1 & 0 & Optimal &  0.07 & 89 & 89.00 &  0.00\\
j9014 10.json & 1 & 0 & Optimal &  0.08 & 85 & 85.00 &  0.00\\
j9014 2.json & 1 & 0 & Optimal &  0.04 & 79 & 79.00 &  0.00\\
j9014 3.json & 1 & 0 & Optimal &  0.06 & 94 & 94.00 &  0.00\\
j9014 4.json & 1 & 0 & Optimal &  0.07 & 88 & 88.00 &  0.00\\
j9014 5.json & 1 & 0 & Optimal &  0.07 & 84 & 84.00 &  0.00\\
j9014 6.json & 1 & 0 & Optimal & 30.01 & 76 & 76.00 &  0.00\\
j9014 7.json & 1 & 0 & Optimal &  0.04 & 86 & 86.00 &  0.00\\
j9014 8.json & 1 & 0 & Optimal &  0.05 & 80 & 80.00 &  0.00\\
j9014 9.json & 1 & 0 & Optimal &  0.06 & 112 & 112.00 &  0.00\\
j9015 1.json & 1 & 0 & Optimal &  0.04 & 76 & 76.00 &  0.00\\
j9015 10.json & 1 & 0 & Optimal &  0.06 & 78 & 78.00 &  0.00\\
j9015 2.json & 1 & 0 & Optimal &  0.04 & 71 & 71.00 &  0.00\\
j9015 3.json & 1 & 0 & Optimal &  0.04 & 82 & 82.00 &  0.00\\
j9015 4.json & 1 & 0 & Optimal &  0.04 & 92 & 92.00 &  0.00\\
j9015 5.json & 1 & 0 & Optimal &  0.07 & 93 & 93.00 &  0.00\\
j9015 6.json & 1 & 0 & Optimal &  0.02 & 61 & 61.00 &  0.00\\
j9015 7.json & 1 & 0 & Optimal &  0.04 & 82 & 82.00 &  0.00\\
j9015 8.json & 1 & 0 & Optimal &  0.04 & 82 & 82.00 &  0.00\\
j9015 9.json & 1 & 0 & Optimal &  0.04 & 83 & 83.00 &  0.00\\
j9016 1.json & 1 & 0 & Optimal &  0.04 & 85 & 85.00 &  0.00\\
j9016 10.json & 1 & 0 & Optimal &  0.03 & 71 & 71.00 &  0.00\\
j9016 2.json & 1 & 0 & Optimal &  0.04 & 71 & 71.00 &  0.00\\
j9016 3.json & 1 & 0 & Optimal &  0.04 & 73 & 73.00 &  0.00\\
j9016 4.json & 1 & 0 & Optimal &  0.03 & 69 & 69.00 &  0.00\\
j9016 5.json & 1 & 0 & Optimal &  0.04 & 71 & 71.00 &  0.00\\
j9016 6.json & 1 & 0 & Optimal &  0.03 & 74 & 74.00 &  0.00\\
j9016 7.json & 1 & 0 & Optimal &  0.02 & 65 & 65.00 &  0.00\\
j9016 8.json & 1 & 0 & Optimal &  0.04 & 71 & 71.00 &  0.00\\
j9016 9.json & 1 & 0 & Optimal &  0.03 & 66 & 66.00 &  0.00\\
j9017 1.json & 1 & 0 & Optimal &  0.13 & 92 & 92.00 &  0.00\\
j9017 10.json & 1 & 0 & Optimal &  0.18 & 89 & 89.00 &  0.00\\
j9017 2.json & 1 & 0 & Optimal &  0.31 & 100 & 100.00 &  0.00\\
j9017 3.json & 1 & 0 & Optimal &  0.13 & 89 & 89.00 &  0.00\\
j9017 4.json & 1 & 0 & Optimal &  0.11 & 94 & 94.00 &  0.00\\
j9017 5.json & 1 & 0 & Optimal &  0.12 & 113 & 113.00 &  0.00\\
j9017 6.json & 1 & 0 & Optimal &  0.12 & 94 & 94.00 &  0.00\\
j9017 7.json & 1 & 0 & Optimal &  0.12 & 80 & 80.00 &  0.00\\
j9017 8.json & 1 & 0 & Optimal &  0.12 & 113 & 113.00 &  0.00\\
j9017 9.json & 1 & 0 & Optimal &  0.14 & 96 & 96.00 &  0.00\\
j9018 1.json & 1 & 0 & Optimal &  0.06 & 101 & 101.00 &  0.00\\
j9018 10.json & 1 & 0 & Optimal &  0.07 & 94 & 94.00 &  0.00\\
j9018 2.json & 1 & 0 & Optimal &  0.04 & 94 & 94.00 &  0.00\\
j9018 3.json & 1 & 0 & Optimal &  0.07 & 83 & 83.00 &  0.00\\
j9018 4.json & 1 & 0 & Optimal &  0.06 & 98 & 98.00 &  0.00\\
j9018 5.json & 1 & 0 & Optimal &  0.04 & 90 & 90.00 &  0.00\\
j9018 6.json & 1 & 0 & Optimal &  0.09 & 83 & 83.00 &  0.00\\
j9018 7.json & 1 & 0 & Optimal &  0.07 & 73 & 73.00 &  0.00\\
j9018 8.json & 1 & 0 & Optimal &  0.05 & 92 & 92.00 &  0.00\\
j9018 9.json & 1 & 0 & Optimal &  0.05 & 79 & 79.00 &  0.00\\
j9019 1.json & 1 & 0 & Optimal &  0.04 & 98 & 98.00 &  0.00\\
j9019 10.json & 1 & 0 & Optimal &  0.03 & 85 & 85.00 &  0.00\\
j9019 2.json & 1 & 0 & Optimal &  0.03 & 83 & 83.00 &  0.00\\
j9019 3.json & 1 & 0 & Optimal &  0.02 & 89 & 89.00 &  0.00\\
j9019 4.json & 1 & 0 & Optimal &  0.02 & 77 & 77.00 &  0.00\\
j9019 5.json & 1 & 0 & Optimal &  0.03 & 66 & 66.00 &  0.00\\
j9019 6.json & 1 & 0 & Optimal &  0.04 & 136 & 136.00 &  0.00\\
j9019 7.json & 1 & 0 & Optimal &  0.06 & 66 & 66.00 &  0.00\\
j9019 8.json & 1 & 0 & Optimal &  0.03 & 91 & 91.00 &  0.00\\
j9019 9.json & 1 & 0 & Optimal &  0.02 & 121 & 121.00 &  0.00\\
j901 1.json & 1 & 0 & Optimal &  0.12 & 73 & 73.00 &  0.00\\
j901 10.json & 1 & 0 & Optimal &  0.09 & 90 & 90.00 &  0.00\\
j901 2.json & 1 & 0 & Optimal &  0.13 & 92 & 92.00 &  0.00\\
j901 3.json & 1 & 0 & Optimal &  0.21 & 66 & 66.00 &  0.00\\
j901 4.json & 1 & 0 & Optimal &  1.51 & 86 & 86.00 &  0.00\\
j901 5.json & 1 & 0 & Optimal &  0.11 & 87 & 87.00 &  0.00\\
j901 6.json & 1 & 0 & Optimal &  0.14 & 74 & 74.00 &  0.00\\
j901 7.json & 1 & 0 & Optimal &  0.08 & 91 & 91.00 &  0.00\\
j901 8.json & 1 & 0 & Optimal &  0.14 & 95 & 95.00 &  0.00\\
j901 9.json & 1 & 0 & Optimal &  0.25 & 72 & 72.00 &  0.00\\
j9020 1.json & 1 & 0 & Optimal &  0.02 & 85 & 85.00 &  0.00\\
j9020 10.json & 1 & 0 & Optimal &  0.02 & 89 & 89.00 &  0.00\\
j9020 2.json & 1 & 0 & Optimal &  0.02 & 76 & 76.00 &  0.00\\
j9020 3.json & 1 & 0 & Optimal &  0.02 & 86 & 86.00 &  0.00\\
j9020 4.json & 1 & 0 & Optimal &  0.02 & 86 & 86.00 &  0.00\\
j9020 5.json & 1 & 0 & Optimal &  0.02 & 88 & 88.00 &  0.00\\
j9020 6.json & 1 & 0 & Optimal &  0.02 & 83 & 83.00 &  0.00\\
j9020 7.json & 1 & 0 & Optimal &  0.03 & 82 & 82.00 &  0.00\\
j9020 8.json & 1 & 0 & Optimal &  0.02 & 85 & 85.00 &  0.00\\
j9020 9.json & 1 & 0 & Optimal &  0.03 & 76 & 76.00 &  0.00\\
j9021 1.json & 1 & 0 & Solution & 30.03 & 111 & 100.00 &  9.91\\
j9021 10.json & 1 & 0 & Solution & 30.03 & 109 & 102.00 &  6.42\\
j9021 2.json & 1 & 0 & Solution & 30.02 & 116 & 108.00 &  6.90\\
j9021 3.json & 1 & 0 & Solution & 30.03 & 124 & 117.00 &  5.65\\
j9021 4.json & 1 & 0 & Optimal & 30.01 & 106 & 106.00 &  0.00\\
j9021 5.json & 1 & 0 & Solution & 30.03 & 112 & 101.00 &  9.82\\
j9021 6.json & 1 & 0 & Solution & 30.03 & 106 & 103.00 &  2.83\\
j9021 7.json & 1 & 0 & Solution & 30.06 & 110 & 101.00 &  8.18\\
j9021 8.json & 1 & 0 & Solution & 30.05 & 112 & 101.00 &  9.82\\
j9021 9.json & 1 & 0 & Solution & 30.04 & 121 & 113.00 &  6.61\\
j9022 1.json & 1 & 0 & Optimal &  0.05 & 108 & 108.00 &  0.00\\
j9022 10.json & 1 & 0 & Optimal &  0.12 & 75 & 75.00 &  0.00\\
j9022 2.json & 1 & 0 & Optimal &  0.09 & 85 & 85.00 &  0.00\\
j9022 3.json & 1 & 0 & Optimal &  0.12 & 83 & 83.00 &  0.00\\
j9022 4.json & 1 & 0 & Optimal &  0.04 & 96 & 96.00 &  0.00\\
j9022 5.json & 1 & 0 & Optimal &  0.05 & 96 & 96.00 &  0.00\\
j9022 6.json & 1 & 0 & Optimal &  0.07 & 71 & 71.00 &  0.00\\
j9022 7.json & 1 & 0 & Optimal &  0.07 & 90 & 90.00 &  0.00\\
j9022 8.json & 1 & 0 & Optimal &  0.04 & 97 & 97.00 &  0.00\\
j9022 9.json & 1 & 0 & Optimal &  0.15 & 101 & 101.00 &  0.00\\
j9023 1.json & 1 & 0 & Optimal &  0.07 & 90 & 90.00 &  0.00\\
j9023 10.json & 1 & 0 & Optimal &  0.04 & 87 & 87.00 &  0.00\\
j9023 2.json & 1 & 0 & Optimal &  0.02 & 84 & 84.00 &  0.00\\
j9023 3.json & 1 & 0 & Optimal &  0.04 & 116 & 116.00 &  0.00\\
j9023 4.json & 1 & 0 & Optimal &  0.06 & 85 & 85.00 &  0.00\\
j9023 5.json & 1 & 0 & Optimal &  0.02 & 95 & 95.00 &  0.00\\
j9023 6.json & 1 & 0 & Optimal &  0.06 & 87 & 87.00 &  0.00\\
j9023 7.json & 1 & 0 & Optimal &  0.04 & 77 & 77.00 &  0.00\\
j9023 8.json & 1 & 0 & Optimal &  0.02 & 92 & 92.00 &  0.00\\
j9023 9.json & 1 & 0 & Optimal &  0.04 & 126 & 126.00 &  0.00\\
j9024 1.json & 1 & 0 & Optimal &  0.02 & 84 & 84.00 &  0.00\\
j9024 10.json & 1 & 0 & Optimal &  0.04 & 89 & 89.00 &  0.00\\
j9024 2.json & 1 & 0 & Optimal &  0.02 & 92 & 92.00 &  0.00\\
j9024 3.json & 1 & 0 & Optimal &  0.02 & 69 & 69.00 &  0.00\\
j9024 4.json & 1 & 0 & Optimal &  0.02 & 81 & 81.00 &  0.00\\
j9024 5.json & 1 & 0 & Optimal &  0.02 & 85 & 85.00 &  0.00\\
j9024 6.json & 1 & 0 & Optimal &  0.03 & 79 & 79.00 &  0.00\\
j9024 7.json & 1 & 0 & Optimal &  0.02 & 87 & 87.00 &  0.00\\
j9024 8.json & 1 & 0 & Optimal &  0.02 & 88 & 88.00 &  0.00\\
j9024 9.json & 1 & 0 & Optimal &  0.03 & 80 & 80.00 &  0.00\\
j9025 1.json & 1 & 0 & Solution & 30.02 & 128 & 115.00 & 10.16\\
j9025 10.json & 1 & 0 & Solution & 30.07 & 132 & 118.00 & 10.61\\
j9025 2.json & 1 & 0 & Solution & 30.03 & 135 & 120.00 & 11.11\\
j9025 3.json & 1 & 0 & Solution & 30.03 & 125 & 111.00 & 11.20\\
j9025 4.json & 1 & 0 & Solution & 30.04 & 143 & 128.00 & 10.49\\
j9025 5.json & 1 & 0 & Solution & 30.04 & 120 & 109.00 &  9.17\\
j9025 6.json & 1 & 0 & Solution & 30.02 & 126 & 112.00 & 11.11\\
j9025 7.json & 1 & 0 & Solution & 30.07 & 136 & 121.00 & 11.03\\
j9025 8.json & 1 & 0 & Solution & 30.03 & 142 & 130.00 &  8.45\\
j9025 9.json & 1 & 0 & Solution & 30.02 & 109 & 96.00 & 11.93\\
j9026 1.json & 1 & 0 & Optimal &  0.05 & 90 & 90.00 &  0.00\\
j9026 10.json & 1 & 0 & Optimal &  0.08 & 92 & 92.00 &  0.00\\
j9026 2.json & 1 & 0 & Optimal & 30.01 & 85 & 85.00 &  0.00\\
j9026 3.json & 1 & 0 & Optimal &  0.07 & 80 & 80.00 &  0.00\\
j9026 4.json & 1 & 0 & Solution & 30.03 & 97 & 96.00 &  1.03\\
j9026 5.json & 1 & 0 & Solution & 30.03 & 86 & 83.00 &  3.49\\
j9026 6.json & 1 & 0 & Optimal &  0.08 & 108 & 108.00 &  0.00\\
j9026 7.json & 1 & 0 & Optimal &  0.80 & 82 & 82.00 &  0.00\\
j9026 8.json & 1 & 0 & Optimal & 30.01 & 82 & 82.00 &  0.00\\
j9026 9.json & 1 & 0 & Optimal &  0.09 & 87 & 87.00 &  0.00\\
j9027 1.json & 1 & 0 & Optimal &  0.05 & 96 & 96.00 &  0.00\\
j9027 10.json & 1 & 0 & Optimal &  0.05 & 97 & 97.00 &  0.00\\
j9027 2.json & 1 & 0 & Optimal &  0.05 & 81 & 81.00 &  0.00\\
j9027 3.json & 1 & 0 & Optimal &  0.04 & 91 & 91.00 &  0.00\\
j9027 4.json & 1 & 0 & Optimal &  0.06 & 79 & 79.00 &  0.00\\
j9027 5.json & 1 & 0 & Optimal &  0.04 & 99 & 99.00 &  0.00\\
j9027 6.json & 1 & 0 & Optimal &  0.06 & 87 & 87.00 &  0.00\\
j9027 7.json & 1 & 0 & Optimal &  0.04 & 73 & 73.00 &  0.00\\
j9027 8.json & 1 & 0 & Optimal &  0.05 & 72 & 72.00 &  0.00\\
j9027 9.json & 1 & 0 & Optimal &  0.05 & 84 & 84.00 &  0.00\\
j9028 1.json & 1 & 0 & Optimal &  0.03 & 80 & 80.00 &  0.00\\
j9028 10.json & 1 & 0 & Optimal &  0.03 & 68 & 68.00 &  0.00\\
j9028 2.json & 1 & 0 & Optimal &  0.03 & 76 & 76.00 &  0.00\\
j9028 3.json & 1 & 0 & Optimal &  0.03 & 86 & 86.00 &  0.00\\
j9028 4.json & 1 & 0 & Optimal &  0.03 & 78 & 78.00 &  0.00\\
j9028 5.json & 1 & 0 & Optimal &  0.03 & 88 & 88.00 &  0.00\\
j9028 6.json & 1 & 0 & Optimal &  0.03 & 102 & 102.00 &  0.00\\
j9028 7.json & 1 & 0 & Optimal &  0.03 & 97 & 97.00 &  0.00\\
j9028 8.json & 1 & 0 & Optimal &  0.03 & 110 & 110.00 &  0.00\\
j9028 9.json & 1 & 0 & Optimal &  0.03 & 120 & 120.00 &  0.00\\
j9029 1.json & 1 & 0 & Solution & 30.02 & 139 & 123.00 & 11.51\\
j9029 10.json & 1 & 0 & Solution & 30.02 & 129 & 117.00 &  9.30\\
j9029 2.json & 1 & 0 & Solution & 30.03 & 130 & 120.00 &  7.69\\
j9029 3.json & 1 & 0 & Solution & 30.05 & 148 & 135.00 &  8.78\\
j9029 4.json & 1 & 0 & Solution & 30.04 & 156 & 136.00 & 12.82\\
j9029 5.json & 1 & 0 & Solution & 30.03 & 127 & 114.00 & 10.24\\
j9029 6.json & 1 & 0 & Solution & 30.03 & 131 & 116.00 & 11.45\\
j9029 7.json & 1 & 0 & Solution & 30.03 & 177 & 158.00 & 10.73\\
j9029 8.json & 1 & 0 & Solution & 30.03 & 162 & 145.00 & 10.49\\
j9029 9.json & 1 & 0 & Solution & 30.03 & 134 & 118.00 & 11.94\\
j902 1.json & 1 & 0 & Optimal &  0.05 & 96 & 96.00 &  0.00\\
j902 10.json & 1 & 0 & Optimal &  0.09 & 80 & 80.00 &  0.00\\
j902 2.json & 1 & 0 & Optimal &  0.07 & 114 & 114.00 &  0.00\\
j902 3.json & 1 & 0 & Optimal &  0.05 & 75 & 75.00 &  0.00\\
j902 4.json & 1 & 0 & Optimal &  0.03 & 70 & 70.00 &  0.00\\
j902 5.json & 1 & 0 & Optimal &  0.02 & 100 & 100.00 &  0.00\\
j902 6.json & 1 & 0 & Optimal &  0.11 & 67 & 67.00 &  0.00\\
j902 7.json & 1 & 0 & Optimal &  0.06 & 92 & 92.00 &  0.00\\
j902 8.json & 1 & 0 & Optimal &  0.04 & 82 & 82.00 &  0.00\\
j902 9.json & 1 & 0 & Optimal &  0.06 & 79 & 79.00 &  0.00\\
j9030 1.json & 1 & 0 & Optimal &  0.07 & 102 & 102.00 &  0.00\\
j9030 10.json & 1 & 0 & Optimal &  0.09 & 90 & 90.00 &  0.00\\
j9030 2.json & 1 & 0 & Optimal &  0.05 & 76 & 76.00 &  0.00\\
j9030 3.json & 1 & 0 & Optimal &  0.09 & 102 & 102.00 &  0.00\\
j9030 4.json & 1 & 0 & Optimal &  0.08 & 104 & 104.00 &  0.00\\
j9030 5.json & 1 & 0 & Solution & 30.02 & 85 & 83.00 &  2.35\\
j9030 6.json & 1 & 0 & Optimal &  0.06 & 90 & 90.00 &  0.00\\
j9030 7.json & 1 & 0 & Solution & 30.03 & 85 & 84.00 &  1.18\\
j9030 8.json & 1 & 0 & Optimal &  0.08 & 82 & 82.00 &  0.00\\
j9030 9.json & 1 & 0 & Solution & 30.02 & 96 & 90.00 &  6.25\\
j9031 1.json & 1 & 0 & Optimal &  0.05 & 79 & 79.00 &  0.00\\
j9031 10.json & 1 & 0 & Optimal &  0.06 & 99 & 99.00 &  0.00\\
j9031 2.json & 1 & 0 & Optimal &  0.04 & 69 & 69.00 &  0.00\\
j9031 3.json & 1 & 0 & Optimal &  0.07 & 106 & 106.00 &  0.00\\
j9031 4.json & 1 & 0 & Optimal &  0.05 & 79 & 79.00 &  0.00\\
j9031 5.json & 1 & 0 & Optimal &  0.02 & 79 & 79.00 &  0.00\\
j9031 6.json & 1 & 0 & Optimal &  0.05 & 80 & 80.00 &  0.00\\
j9031 7.json & 1 & 0 & Optimal &  0.03 & 97 & 97.00 &  0.00\\
j9031 8.json & 1 & 0 & Optimal &  0.04 & 83 & 83.00 &  0.00\\
j9031 9.json & 1 & 0 & Optimal &  0.04 & 72 & 72.00 &  0.00\\
j9032 1.json & 1 & 0 & Optimal &  0.04 & 78 & 78.00 &  0.00\\
j9032 10.json & 1 & 0 & Optimal &  0.03 & 91 & 91.00 &  0.00\\
j9032 2.json & 1 & 0 & Optimal &  0.04 & 78 & 78.00 &  0.00\\
j9032 3.json & 1 & 0 & Optimal &  0.03 & 89 & 89.00 &  0.00\\
j9032 4.json & 1 & 0 & Optimal &  0.04 & 104 & 104.00 &  0.00\\
j9032 5.json & 1 & 0 & Optimal &  0.04 & 93 & 93.00 &  0.00\\
j9032 6.json & 1 & 0 & Optimal &  0.04 & 86 & 86.00 &  0.00\\
j9032 7.json & 1 & 0 & Optimal &  0.03 & 87 & 87.00 &  0.00\\
j9032 8.json & 1 & 0 & Optimal &  0.04 & 79 & 79.00 &  0.00\\
j9032 9.json & 1 & 0 & Optimal &  0.03 & 95 & 95.00 &  0.00\\
j9033 1.json & 1 & 0 & Optimal &  0.16 & 99 & 99.00 &  0.00\\
j9033 10.json & 1 & 0 & Optimal &  0.11 & 114 & 114.00 &  0.00\\
j9033 2.json & 1 & 0 & Optimal &  0.12 & 112 & 112.00 &  0.00\\
j9033 3.json & 1 & 0 & Optimal &  0.09 & 108 & 108.00 &  0.00\\
j9033 4.json & 1 & 0 & Optimal &  0.12 & 92 & 92.00 &  0.00\\
j9033 5.json & 1 & 0 & Optimal &  0.12 & 109 & 109.00 &  0.00\\
j9033 6.json & 1 & 0 & Optimal &  0.12 & 88 & 88.00 &  0.00\\
j9033 7.json & 1 & 0 & Optimal &  0.10 & 109 & 109.00 &  0.00\\
j9033 8.json & 1 & 0 & Optimal &  0.13 & 110 & 110.00 &  0.00\\
j9033 9.json & 1 & 0 & Optimal &  0.12 & 95 & 95.00 &  0.00\\
j9034 1.json & 1 & 0 & Optimal &  0.07 & 83 & 83.00 &  0.00\\
j9034 10.json & 1 & 0 & Optimal &  0.04 & 101 & 101.00 &  0.00\\
j9034 2.json & 1 & 0 & Optimal &  0.04 & 89 & 89.00 &  0.00\\
j9034 3.json & 1 & 0 & Optimal &  0.05 & 82 & 82.00 &  0.00\\
j9034 4.json & 1 & 0 & Optimal &  0.05 & 81 & 81.00 &  0.00\\
j9034 5.json & 1 & 0 & Optimal &  0.07 & 83 & 83.00 &  0.00\\
j9034 6.json & 1 & 0 & Optimal &  0.05 & 89 & 89.00 &  0.00\\
j9034 7.json & 1 & 0 & Optimal &  0.06 & 92 & 92.00 &  0.00\\
j9034 8.json & 1 & 0 & Optimal &  0.07 & 81 & 81.00 &  0.00\\
j9034 9.json & 1 & 0 & Optimal &  0.05 & 109 & 109.00 &  0.00\\
j9035 1.json & 1 & 0 & Optimal &  0.03 & 98 & 98.00 &  0.00\\
j9035 10.json & 1 & 0 & Optimal &  0.04 & 82 & 82.00 &  0.00\\
j9035 2.json & 1 & 0 & Optimal &  0.04 & 92 & 92.00 &  0.00\\
j9035 3.json & 1 & 0 & Optimal &  0.05 & 96 & 96.00 &  0.00\\
j9035 4.json & 1 & 0 & Optimal &  0.04 & 86 & 86.00 &  0.00\\
j9035 5.json & 1 & 0 & Optimal &  0.04 & 103 & 103.00 &  0.00\\
j9035 6.json & 1 & 0 & Optimal &  0.07 & 72 & 72.00 &  0.00\\
j9035 7.json & 1 & 0 & Optimal &  0.04 & 78 & 78.00 &  0.00\\
j9035 8.json & 1 & 0 & Optimal &  0.04 & 85 & 85.00 &  0.00\\
j9035 9.json & 1 & 0 & Optimal &  0.04 & 76 & 76.00 &  0.00\\
j9036 1.json & 1 & 0 & Optimal &  0.03 & 97 & 97.00 &  0.00\\
j9036 10.json & 1 & 0 & Optimal &  0.02 & 109 & 109.00 &  0.00\\
j9036 2.json & 1 & 0 & Optimal &  0.03 & 114 & 114.00 &  0.00\\
j9036 3.json & 1 & 0 & Optimal &  0.03 & 84 & 84.00 &  0.00\\
j9036 4.json & 1 & 0 & Optimal &  0.02 & 79 & 79.00 &  0.00\\
j9036 5.json & 1 & 0 & Optimal &  0.02 & 98 & 98.00 &  0.00\\
j9036 6.json & 1 & 0 & Optimal &  0.04 & 99 & 99.00 &  0.00\\
j9036 7.json & 1 & 0 & Optimal &  0.03 & 89 & 89.00 &  0.00\\
j9036 8.json & 1 & 0 & Optimal &  0.03 & 84 & 84.00 &  0.00\\
j9036 9.json & 1 & 0 & Optimal &  0.03 & 102 & 102.00 &  0.00\\
j9037 1.json & 1 & 0 & Optimal & 30.01 & 110 & 110.00 &  0.00\\
j9037 10.json & 1 & 0 & Solution & 30.05 & 124 & 112.00 &  9.68\\
j9037 2.json & 1 & 0 & Solution & 30.05 & 116 & 105.00 &  9.48\\
j9037 3.json & 1 & 0 & Optimal &  8.76 & 132 & 132.00 &  0.00\\
j9037 4.json & 1 & 0 & Optimal & 30.01 & 123 & 123.00 &  0.00\\
j9037 5.json & 1 & 0 & Solution & 30.04 & 126 & 114.00 &  9.52\\
j9037 6.json & 1 & 0 & Solution & 30.03 & 132 & 120.00 &  9.09\\
j9037 7.json & 1 & 0 & Optimal & 30.02 & 123 & 123.00 &  0.00\\
j9037 8.json & 1 & 0 & Solution & 30.03 & 119 & 105.00 & 11.76\\
j9037 9.json & 1 & 0 & Optimal & 30.02 & 123 & 123.00 &  0.00\\
j9038 1.json & 1 & 0 & Optimal &  0.11 & 85 & 85.00 &  0.00\\
j9038 10.json & 1 & 0 & Optimal &  0.09 & 108 & 108.00 &  0.00\\
j9038 2.json & 1 & 0 & Optimal &  0.10 & 78 & 78.00 &  0.00\\
j9038 3.json & 1 & 0 & Optimal &  0.42 & 89 & 89.00 &  0.00\\
j9038 4.json & 1 & 0 & Optimal &  0.10 & 89 & 89.00 &  0.00\\
j9038 5.json & 1 & 0 & Optimal &  0.40 & 86 & 86.00 &  0.00\\
j9038 6.json & 1 & 0 & Optimal &  0.09 & 88 & 88.00 &  0.00\\
j9038 7.json & 1 & 0 & Optimal &  0.09 & 85 & 85.00 &  0.00\\
j9038 8.json & 1 & 0 & Optimal &  0.09 & 91 & 91.00 &  0.00\\
j9038 9.json & 1 & 0 & Optimal &  0.17 & 95 & 95.00 &  0.00\\
j9039 1.json & 1 & 0 & Optimal &  0.08 & 106 & 106.00 &  0.00\\
j9039 10.json & 1 & 0 & Optimal &  0.04 & 100 & 100.00 &  0.00\\
j9039 2.json & 1 & 0 & Optimal &  0.03 & 119 & 119.00 &  0.00\\
j9039 3.json & 1 & 0 & Optimal &  0.04 & 83 & 83.00 &  0.00\\
j9039 4.json & 1 & 0 & Optimal &  0.04 & 81 & 81.00 &  0.00\\
j9039 5.json & 1 & 0 & Optimal &  0.04 & 85 & 85.00 &  0.00\\
j9039 6.json & 1 & 0 & Optimal &  0.08 & 102 & 102.00 &  0.00\\
j9039 7.json & 1 & 0 & Optimal &  0.04 & 85 & 85.00 &  0.00\\
j9039 8.json & 1 & 0 & Optimal &  0.05 & 81 & 81.00 &  0.00\\
j9039 9.json & 1 & 0 & Optimal &  0.09 & 79 & 79.00 &  0.00\\
j903 1.json & 1 & 0 & Optimal &  0.04 & 81 & 81.00 &  0.00\\
j903 10.json & 1 & 0 & Optimal &  0.04 & 65 & 65.00 &  0.00\\
j903 2.json & 1 & 0 & Optimal &  0.03 & 84 & 84.00 &  0.00\\
j903 3.json & 1 & 0 & Optimal &  0.04 & 71 & 71.00 &  0.00\\
j903 4.json & 1 & 0 & Optimal &  0.04 & 104 & 104.00 &  0.00\\
j903 5.json & 1 & 0 & Optimal &  0.03 & 75 & 75.00 &  0.00\\
j903 6.json & 1 & 0 & Optimal &  0.04 & 68 & 68.00 &  0.00\\
j903 7.json & 1 & 0 & Optimal &  0.03 & 87 & 87.00 &  0.00\\
j903 8.json & 1 & 0 & Optimal &  0.04 & 86 & 86.00 &  0.00\\
j903 9.json & 1 & 0 & Optimal &  0.06 & 61 & 61.00 &  0.00\\
j9040 1.json & 1 & 0 & Optimal &  0.02 & 95 & 95.00 &  0.00\\
j9040 10.json & 1 & 0 & Optimal &  0.04 & 86 & 86.00 &  0.00\\
j9040 2.json & 1 & 0 & Optimal &  0.02 & 91 & 91.00 &  0.00\\
j9040 3.json & 1 & 0 & Optimal &  0.02 & 77 & 77.00 &  0.00\\
j9040 4.json & 1 & 0 & Optimal &  0.04 & 106 & 106.00 &  0.00\\
j9040 5.json & 1 & 0 & Optimal &  0.03 & 92 & 92.00 &  0.00\\
j9040 6.json & 1 & 0 & Optimal &  0.04 & 86 & 86.00 &  0.00\\
j9040 7.json & 1 & 0 & Optimal &  0.02 & 87 & 87.00 &  0.00\\
j9040 8.json & 1 & 0 & Optimal &  0.02 & 79 & 79.00 &  0.00\\
j9040 9.json & 1 & 0 & Optimal &  0.02 & 98 & 98.00 &  0.00\\
j9041 1.json & 1 & 0 & Solution & 30.02 & 147 & 128.00 & 12.93\\
j9041 10.json & 1 & 0 & Solution & 30.02 & 153 & 143.00 &  6.54\\
j9041 2.json & 1 & 0 & Solution & 30.03 & 173 & 152.00 & 12.14\\
j9041 3.json & 1 & 0 & Solution & 30.02 & 171 & 145.00 & 15.20\\
j9041 4.json & 1 & 0 & Solution & 30.02 & 161 & 140.00 & 13.04\\
j9041 5.json & 1 & 0 & Solution & 30.02 & 131 & 114.00 & 12.98\\
j9041 6.json & 1 & 0 & Solution & 30.03 & 137 & 119.00 & 13.14\\
j9041 7.json & 1 & 0 & Solution & 30.03 & 162 & 140.00 & 13.58\\
j9041 8.json & 1 & 0 & Solution & 30.07 & 170 & 144.00 & 15.29\\
j9041 9.json & 1 & 0 & Solution & 30.03 & 123 & 109.00 & 11.38\\
j9042 1.json & 1 & 0 & Optimal &  0.15 & 106 & 106.00 &  0.00\\
j9042 10.json & 1 & 0 & Solution & 30.03 & 91 & 88.00 &  3.30\\
j9042 2.json & 1 & 0 & Optimal & 30.01 & 102 & 102.00 &  0.00\\
j9042 3.json & 1 & 0 & Optimal &  0.08 & 94 & 94.00 &  0.00\\
j9042 4.json & 1 & 0 & Optimal &  0.07 & 102 & 102.00 &  0.00\\
j9042 5.json & 1 & 0 & Optimal &  0.09 & 105 & 105.00 &  0.00\\
j9042 6.json & 1 & 0 & Optimal &  0.07 & 89 & 89.00 &  0.00\\
j9042 7.json & 1 & 0 & Solution & 30.04 & 87 & 85.00 &  2.30\\
j9042 8.json & 1 & 0 & Optimal &  0.09 & 105 & 105.00 &  0.00\\
j9042 9.json & 1 & 0 & Optimal & 10.18 & 83 & 83.00 &  0.00\\
j9043 1.json & 1 & 0 & Optimal &  0.07 & 99 & 99.00 &  0.00\\
j9043 10.json & 1 & 0 & Optimal &  0.07 & 92 & 92.00 &  0.00\\
j9043 2.json & 1 & 0 & Optimal &  0.02 & 91 & 91.00 &  0.00\\
j9043 3.json & 1 & 0 & Optimal &  0.06 & 102 & 102.00 &  0.00\\
j9043 4.json & 1 & 0 & Optimal &  0.02 & 94 & 94.00 &  0.00\\
j9043 5.json & 1 & 0 & Optimal &  0.04 & 98 & 98.00 &  0.00\\
j9043 6.json & 1 & 0 & Optimal &  0.04 & 114 & 114.00 &  0.00\\
j9043 7.json & 1 & 0 & Optimal &  0.07 & 88 & 88.00 &  0.00\\
j9043 8.json & 1 & 0 & Optimal &  0.05 & 100 & 100.00 &  0.00\\
j9043 9.json & 1 & 0 & Optimal &  0.02 & 88 & 88.00 &  0.00\\
j9044 1.json & 1 & 0 & Optimal &  0.02 & 100 & 100.00 &  0.00\\
j9044 10.json & 1 & 0 & Optimal &  0.03 & 86 & 86.00 &  0.00\\
j9044 2.json & 1 & 0 & Optimal &  0.02 & 92 & 92.00 &  0.00\\
j9044 3.json & 1 & 0 & Optimal &  0.02 & 110 & 110.00 &  0.00\\
j9044 4.json & 1 & 0 & Optimal &  0.04 & 89 & 89.00 &  0.00\\
j9044 5.json & 1 & 0 & Optimal &  0.03 & 84 & 84.00 &  0.00\\
j9044 6.json & 1 & 0 & Optimal &  0.02 & 96 & 96.00 &  0.00\\
j9044 7.json & 1 & 0 & Optimal &  0.02 & 93 & 93.00 &  0.00\\
j9044 8.json & 1 & 0 & Optimal &  0.02 & 99 & 99.00 &  0.00\\
j9044 9.json & 1 & 0 & Optimal &  0.02 & 96 & 96.00 &  0.00\\
j9045 1.json & 1 & 0 & Solution & 30.03 & 152 & 142.00 &  6.58\\
j9045 10.json & 1 & 0 & Solution & 30.02 & 177 & 155.00 & 12.43\\
j9045 2.json & 1 & 0 & Solution & 30.03 & 153 & 136.00 & 11.11\\
j9045 3.json & 1 & 0 & Solution & 30.09 & 161 & 139.00 & 13.66\\
j9045 4.json & 1 & 0 & Solution & 30.03 & 140 & 123.00 & 12.14\\
j9045 5.json & 1 & 0 & Solution & 30.03 & 186 & 162.00 & 12.90\\
j9045 6.json & 1 & 0 & Solution & 30.03 & 186 & 160.00 & 13.98\\
j9045 7.json & 1 & 0 & Solution & 30.03 & 146 & 127.00 & 13.01\\
j9045 8.json & 1 & 0 & Solution & 30.03 & 167 & 147.00 & 11.98\\
j9045 9.json & 1 & 0 & Solution & 30.03 & 163 & 141.00 & 13.50\\
j9046 1.json & 1 & 0 & Optimal & 30.01 & 104 & 104.00 &  0.00\\
j9046 10.json & 1 & 0 & Optimal &  0.09 & 114 & 114.00 &  0.00\\
j9046 2.json & 1 & 0 & Optimal &  0.12 & 98 & 98.00 &  0.00\\
j9046 3.json & 1 & 0 & Optimal &  0.16 & 113 & 113.00 &  0.00\\
j9046 4.json & 1 & 0 & Solution & 30.01 & 94 & 90.00 &  4.26\\
j9046 5.json & 1 & 0 & Optimal &  0.08 & 91 & 91.00 &  0.00\\
j9046 6.json & 1 & 0 & Optimal &  0.09 & 83 & 83.00 &  0.00\\
j9046 7.json & 1 & 0 & Optimal &  0.21 & 89 & 89.00 &  0.00\\
j9046 8.json & 1 & 0 & Solution & 30.03 & 98 & 93.00 &  5.10\\
j9046 9.json & 1 & 0 & Solution & 30.02 & 90 & 85.00 &  5.56\\
j9047 1.json & 1 & 0 & Optimal &  0.05 & 82 & 82.00 &  0.00\\
j9047 10.json & 1 & 0 & Optimal &  0.05 & 65 & 65.00 &  0.00\\
j9047 2.json & 1 & 0 & Optimal &  0.05 & 90 & 90.00 &  0.00\\
j9047 3.json & 1 & 0 & Optimal &  0.07 & 102 & 102.00 &  0.00\\
j9047 4.json & 1 & 0 & Optimal &  0.07 & 93 & 93.00 &  0.00\\
j9047 5.json & 1 & 0 & Optimal &  0.03 & 93 & 93.00 &  0.00\\
j9047 6.json & 1 & 0 & Optimal &  0.04 & 98 & 98.00 &  0.00\\
j9047 7.json & 1 & 0 & Optimal &  0.04 & 94 & 94.00 &  0.00\\
j9047 8.json & 1 & 0 & Optimal &  0.04 & 98 & 98.00 &  0.00\\
j9047 9.json & 1 & 0 & Optimal &  0.04 & 86 & 86.00 &  0.00\\
j9048 1.json & 1 & 0 & Optimal &  0.03 & 83 & 83.00 &  0.00\\
j9048 10.json & 1 & 0 & Optimal &  0.04 & 93 & 93.00 &  0.00\\
j9048 2.json & 1 & 0 & Optimal &  0.04 & 89 & 89.00 &  0.00\\
j9048 3.json & 1 & 0 & Optimal &  0.04 & 86 & 86.00 &  0.00\\
j9048 4.json & 1 & 0 & Optimal &  0.04 & 91 & 91.00 &  0.00\\
j9048 5.json & 1 & 0 & Optimal &  0.04 & 75 & 75.00 &  0.00\\
j9048 6.json & 1 & 0 & Optimal &  0.03 & 114 & 114.00 &  0.00\\
j9048 7.json & 1 & 0 & Optimal &  0.03 & 103 & 103.00 &  0.00\\
j9048 8.json & 1 & 0 & Optimal &  0.04 & 74 & 74.00 &  0.00\\
j9048 9.json & 1 & 0 & Optimal &  0.04 & 89 & 89.00 &  0.00\\
j904 1.json & 1 & 0 & Optimal &  0.02 & 93 & 93.00 &  0.00\\
j904 10.json & 1 & 0 & Optimal &  0.02 & 68 & 68.00 &  0.00\\
j904 2.json & 1 & 0 & Optimal &  0.02 & 89 & 89.00 &  0.00\\
j904 3.json & 1 & 0 & Optimal &  0.02 & 67 & 67.00 &  0.00\\
j904 4.json & 1 & 0 & Optimal &  0.02 & 92 & 92.00 &  0.00\\
j904 5.json & 1 & 0 & Optimal &  0.02 & 88 & 88.00 &  0.00\\
j904 6.json & 1 & 0 & Optimal &  0.02 & 78 & 78.00 &  0.00\\
j904 7.json & 1 & 0 & Optimal &  0.03 & 80 & 80.00 &  0.00\\
j904 8.json & 1 & 0 & Optimal &  0.03 & 69 & 69.00 &  0.00\\
j904 9.json & 1 & 0 & Optimal &  0.02 & 79 & 79.00 &  0.00\\
j905 1.json & 1 & 0 & Optimal & 30.01 & 78 & 78.00 &  0.00\\
j905 10.json & 1 & 0 & Solution & 30.04 & 98 & 94.00 &  4.08\\
j905 2.json & 1 & 0 & Optimal & 30.00 & 93 & 93.00 &  0.00\\
j905 3.json & 1 & 0 & Solution & 30.04 & 91 & 83.00 &  8.79\\
j905 4.json & 1 & 0 & Solution & 30.02 & 103 & 98.00 &  4.85\\
j905 5.json & 1 & 0 & Solution & 30.02 & 113 & 108.00 &  4.42\\
j905 6.json & 1 & 0 & Solution & 30.03 & 87 & 83.00 &  4.60\\
j905 7.json & 1 & 0 & Solution & 30.03 & 109 & 106.00 &  2.75\\
j905 8.json & 1 & 0 & Solution & 30.02 & 105 & 96.00 &  8.57\\
j905 9.json & 1 & 0 & Solution & 30.02 & 116 & 107.00 &  7.76\\
j906 1.json & 1 & 0 & Optimal &  0.06 & 82 & 82.00 &  0.00\\
j906 10.json & 1 & 0 & Optimal &  0.09 & 94 & 94.00 &  0.00\\
j906 2.json & 1 & 0 & Optimal &  0.06 & 86 & 86.00 &  0.00\\
j906 3.json & 1 & 0 & Optimal &  0.18 & 77 & 77.00 &  0.00\\
j906 4.json & 1 & 0 & Optimal &  0.05 & 80 & 80.00 &  0.00\\
j906 5.json & 1 & 0 & Optimal &  0.08 & 71 & 71.00 &  0.00\\
j906 6.json & 1 & 0 & Optimal &  0.04 & 98 & 98.00 &  0.00\\
j906 7.json & 1 & 0 & Optimal &  0.04 & 71 & 71.00 &  0.00\\
j906 8.json & 1 & 0 & Optimal & 18.15 & 68 & 68.00 &  0.00\\
j906 9.json & 1 & 0 & Optimal &  0.04 & 68 & 68.00 &  0.00\\
j907 1.json & 1 & 0 & Optimal &  0.03 & 88 & 88.00 &  0.00\\
j907 10.json & 1 & 0 & Optimal &  0.02 & 98 & 98.00 &  0.00\\
j907 2.json & 1 & 0 & Optimal &  0.02 & 77 & 77.00 &  0.00\\
j907 3.json & 1 & 0 & Optimal &  0.03 & 80 & 80.00 &  0.00\\
j907 4.json & 1 & 0 & Optimal &  0.03 & 86 & 86.00 &  0.00\\
j907 5.json & 1 & 0 & Optimal &  0.04 & 79 & 79.00 &  0.00\\
j907 6.json & 1 & 0 & Optimal &  0.04 & 90 & 90.00 &  0.00\\
j907 7.json & 1 & 0 & Optimal &  0.02 & 90 & 90.00 &  0.00\\
j907 8.json & 1 & 0 & Optimal &  0.02 & 60 & 60.00 &  0.00\\
j907 9.json & 1 & 0 & Optimal &  0.06 & 83 & 83.00 &  0.00\\
j908 1.json & 1 & 0 & Optimal &  0.02 & 96 & 96.00 &  0.00\\
j908 10.json & 1 & 0 & Optimal &  0.02 & 88 & 88.00 &  0.00\\
j908 2.json & 1 & 0 & Optimal &  0.02 & 78 & 78.00 &  0.00\\
j908 3.json & 1 & 0 & Optimal &  0.02 & 70 & 70.00 &  0.00\\
j908 4.json & 1 & 0 & Optimal &  0.02 & 77 & 77.00 &  0.00\\
j908 5.json & 1 & 0 & Optimal &  0.03 & 63 & 63.00 &  0.00\\
j908 6.json & 1 & 0 & Optimal &  0.02 & 70 & 70.00 &  0.00\\
j908 7.json & 1 & 0 & Optimal &  0.02 & 77 & 77.00 &  0.00\\
j908 8.json & 1 & 0 & Optimal &  0.02 & 68 & 68.00 &  0.00\\
j908 9.json & 1 & 0 & Optimal &  0.02 & 97 & 97.00 &  0.00\\
j909 1.json & 1 & 0 & Solution & 30.02 & 108 & 98.00 &  9.26\\
j909 10.json & 1 & 0 & Solution & 30.02 & 115 & 104.00 &  9.57\\
j909 2.json & 1 & 0 & Solution & 30.04 & 132 & 120.00 &  9.09\\
j909 3.json & 1 & 0 & Solution & 30.02 & 106 & 97.00 &  8.49\\
j909 4.json & 1 & 0 & Solution & 30.02 & 132 & 118.00 & 10.61\\
j909 5.json & 1 & 0 & Solution & 30.06 & 145 & 125.00 & 13.79\\
j909 6.json & 1 & 0 & Solution & 30.03 & 121 & 111.00 &  8.26\\
j909 7.json & 1 & 0 & Solution & 30.02 & 113 & 102.00 &  9.73\\
j909 8.json & 1 & 0 & Solution & 30.07 & 121 & 109.00 &  9.92\\
j909 9.json & 1 & 0 & Solution & 30.04 & 122 & 105.00 & 13.93\\
\end{longtable}



\section{Size J120}
\subsection{CPO}
\begin{longtable}{lrrlrrrr}
\caption{Results for RCPSP J120 (CPO) (600 Instances)}\\\toprule
Name & \shortstack{Nr\\Jobs} & \shortstack{Nr\\Machines} & Status & Time & Makespan & Bound & \shortstack{Gap\\Percent}\\ \midrule
\endhead
\bottomrule
\endfoot
j12010 1.json & 1 & 0 & Optimal &  0.12 & 111 & 111.00 &  0.00\\
j12010 10.json & 1 & 0 & Optimal &  0.03 & 66 & 66.00 &  0.00\\
j12010 2.json & 1 & 0 & Optimal &  0.04 & 91 & 91.00 &  0.00\\
j12010 3.json & 1 & 0 & Optimal &  0.05 & 99 & 99.00 &  0.00\\
j12010 4.json & 1 & 0 & Optimal &  0.03 & 95 & 95.00 &  0.00\\
j12010 5.json & 1 & 0 & Optimal &  0.03 & 97 & 97.00 &  0.00\\
j12010 6.json & 1 & 0 & Optimal &  0.03 & 92 & 92.00 &  0.00\\
j12010 7.json & 1 & 0 & Optimal &  0.06 & 79 & 79.00 &  0.00\\
j12010 8.json & 1 & 0 & Optimal &  0.03 & 114 & 114.00 &  0.00\\
j12010 9.json & 1 & 0 & Optimal &  0.02 & 77 & 77.00 &  0.00\\
j12011 1.json & 1 & 0 & Solution & 60.02 & 180 & 155.00 & 13.89\\
j12011 10.json & 1 & 0 & Solution & 60.03 & 188 & 163.00 & 13.30\\
j12011 2.json & 1 & 0 & Solution & 60.02 & 164 & 144.00 & 12.20\\
j12011 3.json & 1 & 0 & Solution & 60.03 & 212 & 186.00 & 12.26\\
j12011 4.json & 1 & 0 & Solution & 60.03 & 207 & 175.00 & 15.46\\
j12011 5.json & 1 & 0 & Solution & 60.03 & 222 & 191.00 & 13.96\\
j12011 6.json & 1 & 0 & Solution & 60.02 & 223 & 188.00 & 15.70\\
j12011 7.json & 1 & 0 & Solution & 60.03 & 170 & 147.00 & 13.53\\
j12011 8.json & 1 & 0 & Solution & 60.02 & 167 & 151.00 &  9.58\\
j12011 9.json & 1 & 0 & Solution & 60.05 & 181 & 167.00 &  7.73\\
j12012 1.json & 1 & 0 & Solution & 60.02 & 143 & 127.00 & 11.19\\
j12012 10.json & 1 & 0 & Solution & 60.02 & 146 & 142.00 &  2.74\\
j12012 2.json & 1 & 0 & Solution & 60.01 & 119 & 111.00 &  6.72\\
j12012 3.json & 1 & 0 & Solution & 60.02 & 140 & 132.00 &  5.71\\
j12012 4.json & 1 & 0 & Solution & 60.02 & 128 & 122.00 &  4.69\\
j12012 5.json & 1 & 0 & Solution & 60.01 & 168 & 153.00 &  8.93\\
j12012 6.json & 1 & 0 & Solution & 60.02 & 126 & 116.00 &  7.94\\
j12012 7.json & 1 & 0 & Solution & 60.03 & 123 & 116.00 &  5.69\\
j12012 8.json & 1 & 0 & Solution & 60.02 & 123 & 113.00 &  8.13\\
j12012 9.json & 1 & 0 & Solution & 60.02 & 109 & 101.00 &  7.34\\
j12013 1.json & 1 & 0 & Solution & 60.03 & 130 & 123.00 &  5.38\\
j12013 10.json & 1 & 0 & Solution & 60.02 & 95 & 89.00 &  6.32\\
j12013 2.json & 1 & 0 & Solution & 60.02 & 89 & 88.00 &  1.12\\
j12013 3.json & 1 & 0 & Solution & 60.02 & 121 & 115.00 &  4.96\\
j12013 4.json & 1 & 0 & Solution & 60.01 & 115 & 108.00 &  6.09\\
j12013 5.json & 1 & 0 & Solution & 60.02 & 93 & 90.00 &  3.23\\
j12013 6.json & 1 & 0 & Solution & 60.03 & 101 & 95.00 &  5.94\\
j12013 7.json & 1 & 0 & Solution & 60.01 & 112 & 107.00 &  4.46\\
j12013 8.json & 1 & 0 & Solution & 60.01 & 97 & 91.00 &  6.19\\
j12013 9.json & 1 & 0 & Solution & 60.01 & 86 & 83.00 &  3.49\\
j12014 1.json & 1 & 0 & Solution & 60.02 & 87 & 84.00 &  3.45\\
j12014 10.json & 1 & 0 & Solution & 60.02 & 83 & 80.00 &  3.61\\
j12014 2.json & 1 & 0 & Solution & 60.02 & 94 & 90.00 &  4.26\\
j12014 3.json & 1 & 0 & Optimal &  1.53 & 88 & 88.00 &  0.00\\
j12014 4.json & 1 & 0 & Solution & 60.02 & 90 & 85.00 &  5.56\\
j12014 5.json & 1 & 0 & Solution & 60.01 & 99 & 93.00 &  6.06\\
j12014 6.json & 1 & 0 & Optimal &  0.62 & 91 & 91.00 &  0.00\\
j12014 7.json & 1 & 0 & Solution & 60.03 & 91 & 90.00 &  1.10\\
j12014 8.json & 1 & 0 & Solution & 60.02 & 114 & 110.00 &  3.51\\
j12014 9.json & 1 & 0 & Optimal &  0.05 & 101 & 101.00 &  0.00\\
j12015 1.json & 1 & 0 & Optimal &  0.02 & 81 & 81.00 &  0.00\\
j12015 10.json & 1 & 0 & Optimal &  0.03 & 91 & 91.00 &  0.00\\
j12015 2.json & 1 & 0 & Optimal &  0.03 & 75 & 75.00 &  0.00\\
j12015 3.json & 1 & 0 & Optimal &  0.04 & 87 & 87.00 &  0.00\\
j12015 4.json & 1 & 0 & Optimal &  0.02 & 82 & 82.00 &  0.00\\
j12015 5.json & 1 & 0 & Optimal &  0.03 & 87 & 87.00 &  0.00\\
j12015 6.json & 1 & 0 & Optimal &  0.03 & 97 & 97.00 &  0.00\\
j12015 7.json & 1 & 0 & Optimal &  0.03 & 75 & 75.00 &  0.00\\
j12015 8.json & 1 & 0 & Optimal &  0.02 & 126 & 126.00 &  0.00\\
j12015 9.json & 1 & 0 & Optimal &  0.04 & 109 & 109.00 &  0.00\\
j12016 1.json & 1 & 0 & Solution & 60.03 & 204 & 179.00 & 12.25\\
j12016 10.json & 1 & 0 & Solution & 60.01 & 224 & 201.00 & 10.27\\
j12016 2.json & 1 & 0 & Solution & 60.02 & 243 & 221.00 &  9.05\\
j12016 3.json & 1 & 0 & Solution & 60.02 & 245 & 219.00 & 10.61\\
j12016 4.json & 1 & 0 & Solution & 60.01 & 207 & 189.00 &  8.70\\
j12016 5.json & 1 & 0 & Solution & 60.01 & 207 & 182.00 & 12.08\\
j12016 6.json & 1 & 0 & Solution & 60.03 & 213 & 194.00 &  8.92\\
j12016 7.json & 1 & 0 & Solution & 60.02 & 191 & 174.00 &  8.90\\
j12016 8.json & 1 & 0 & Solution & 60.02 & 200 & 180.00 & 10.00\\
j12016 9.json & 1 & 0 & Solution & 60.02 & 214 & 188.00 & 12.15\\
j12017 1.json & 1 & 0 & Solution & 60.01 & 145 & 135.00 &  6.90\\
j12017 10.json & 1 & 0 & Solution & 60.02 & 139 & 131.00 &  5.76\\
j12017 2.json & 1 & 0 & Solution & 60.02 & 128 & 121.00 &  5.47\\
j12017 3.json & 1 & 0 & Solution & 60.02 & 111 & 106.00 &  4.50\\
j12017 4.json & 1 & 0 & Solution & 60.02 & 123 & 118.00 &  4.07\\
j12017 5.json & 1 & 0 & Solution & 60.01 & 133 & 123.00 &  7.52\\
j12017 6.json & 1 & 0 & Solution & 60.02 & 140 & 133.00 &  5.00\\
j12017 7.json & 1 & 0 & Solution & 60.00 & 150 & 141.00 &  6.00\\
j12017 8.json & 1 & 0 & Solution & 60.02 & 131 & 126.00 &  3.82\\
j12017 9.json & 1 & 0 & Solution & 60.01 & 138 & 129.00 &  6.52\\
j12018 1.json & 1 & 0 & Solution & 60.01 & 142 & 137.00 &  3.52\\
j12018 10.json & 1 & 0 & Solution & 60.01 & 100 & 97.00 &  3.00\\
j12018 2.json & 1 & 0 & Solution & 60.00 & 119 & 111.00 &  6.72\\
j12018 3.json & 1 & 0 & Solution & 60.01 & 103 & 100.00 &  2.91\\
j12018 4.json & 1 & 0 & Solution & 60.01 & 103 & 98.00 &  4.85\\
j12018 5.json & 1 & 0 & Solution & 60.01 & 121 & 117.00 &  3.31\\
j12018 6.json & 1 & 0 & Solution & 60.01 & 137 & 131.00 &  4.38\\
j12018 7.json & 1 & 0 & Solution & 60.02 & 120 & 112.00 &  6.67\\
j12018 8.json & 1 & 0 & Solution & 60.00 & 107 & 102.00 &  4.67\\
j12018 9.json & 1 & 0 & Solution & 60.02 & 94 & 89.00 &  5.32\\
j12019 1.json & 1 & 0 & Optimal &  0.08 & 88 & 88.00 &  0.00\\
j12019 10.json & 1 & 0 & Optimal &  0.03 & 88 & 88.00 &  0.00\\
j12019 2.json & 1 & 0 & Solution & 60.02 & 84 & 81.00 &  3.57\\
j12019 3.json & 1 & 0 & Solution & 60.02 & 86 & 83.00 &  3.49\\
j12019 4.json & 1 & 0 & Solution & 60.02 & 110 & 103.00 &  6.36\\
j12019 5.json & 1 & 0 & Solution & 60.02 & 106 & 101.00 &  4.72\\
j12019 6.json & 1 & 0 & Solution & 60.00 & 91 & 89.00 &  2.20\\
j12019 7.json & 1 & 0 & Optimal &  0.05 & 93 & 93.00 &  0.00\\
j12019 8.json & 1 & 0 & Solution & 60.01 & 94 & 93.00 &  1.06\\
j12019 9.json & 1 & 0 & Solution & 60.02 & 90 & 88.00 &  2.22\\
j1201 1.json & 1 & 0 & Optimal & 22.53 & 105 & 105.00 &  0.00\\
j1201 10.json & 1 & 0 & Optimal &  2.36 & 108 & 108.00 &  0.00\\
j1201 2.json & 1 & 0 & Optimal &  2.20 & 109 & 109.00 &  0.00\\
j1201 3.json & 1 & 0 & Solution & 60.02 & 126 & 115.00 &  8.73\\
j1201 4.json & 1 & 0 & Optimal &  1.90 & 97 & 97.00 &  0.00\\
j1201 5.json & 1 & 0 & Optimal &  3.17 & 112 & 112.00 &  0.00\\
j1201 6.json & 1 & 0 & Optimal &  0.86 & 84 & 84.00 &  0.00\\
j1201 7.json & 1 & 0 & Optimal &  0.97 & 117 & 117.00 &  0.00\\
j1201 8.json & 1 & 0 & Optimal &  5.19 & 109 & 109.00 &  0.00\\
j1201 9.json & 1 & 0 & Optimal &  0.86 & 112 & 112.00 &  0.00\\
j12020 1.json & 1 & 0 & Optimal &  0.17 & 89 & 89.00 &  0.00\\
j12020 10.json & 1 & 0 & Optimal &  0.03 & 81 & 81.00 &  0.00\\
j12020 2.json & 1 & 0 & Optimal &  0.03 & 99 & 99.00 &  0.00\\
j12020 3.json & 1 & 0 & Solution & 60.00 & 79 & 75.00 &  5.06\\
j12020 4.json & 1 & 0 & Optimal &  0.02 & 89 & 89.00 &  0.00\\
j12020 5.json & 1 & 0 & Optimal &  0.02 & 69 & 69.00 &  0.00\\
j12020 6.json & 1 & 0 & Optimal &  0.02 & 80 & 80.00 &  0.00\\
j12020 7.json & 1 & 0 & Optimal &  0.02 & 81 & 81.00 &  0.00\\
j12020 8.json & 1 & 0 & Optimal & 10.16 & 107 & 107.00 &  0.00\\
j12020 9.json & 1 & 0 & Optimal &  0.03 & 80 & 80.00 &  0.00\\
j12021 1.json & 1 & 0 & Optimal &  8.08 & 114 & 114.00 &  0.00\\
j12021 10.json & 1 & 0 & Optimal &  4.05 & 102 & 102.00 &  0.00\\
j12021 2.json & 1 & 0 & Optimal & 15.64 & 117 & 117.00 &  0.00\\
j12021 3.json & 1 & 0 & Optimal &  2.07 & 143 & 143.00 &  0.00\\
j12021 4.json & 1 & 0 & Optimal &  6.50 & 135 & 135.00 &  0.00\\
j12021 5.json & 1 & 0 & Optimal &  1.65 & 110 & 110.00 &  0.00\\
j12021 6.json & 1 & 0 & Optimal &  2.84 & 109 & 109.00 &  0.00\\
j12021 7.json & 1 & 0 & Optimal & 15.91 & 111 & 111.00 &  0.00\\
j12021 8.json & 1 & 0 & Optimal &  0.56 & 127 & 127.00 &  0.00\\
j12021 9.json & 1 & 0 & Optimal &  0.80 & 102 & 102.00 &  0.00\\
j12022 1.json & 1 & 0 & Optimal &  0.91 & 101 & 101.00 &  0.00\\
j12022 10.json & 1 & 0 & Optimal &  0.33 & 79 & 79.00 &  0.00\\
j12022 2.json & 1 & 0 & Optimal &  0.03 & 107 & 107.00 &  0.00\\
j12022 3.json & 1 & 0 & Optimal & 19.99 & 96 & 96.00 &  0.00\\
j12022 4.json & 1 & 0 & Optimal &  0.50 & 90 & 90.00 &  0.00\\
j12022 5.json & 1 & 0 & Optimal &  0.39 & 93 & 93.00 &  0.00\\
j12022 6.json & 1 & 0 & Optimal &  0.53 & 103 & 103.00 &  0.00\\
j12022 7.json & 1 & 0 & Optimal &  0.03 & 133 & 133.00 &  0.00\\
j12022 8.json & 1 & 0 & Optimal &  3.32 & 103 & 103.00 &  0.00\\
j12022 9.json & 1 & 0 & Optimal &  0.61 & 109 & 109.00 &  0.00\\
j12023 1.json & 1 & 0 & Optimal &  0.03 & 107 & 107.00 &  0.00\\
j12023 10.json & 1 & 0 & Optimal &  0.02 & 100 & 100.00 &  0.00\\
j12023 2.json & 1 & 0 & Optimal &  0.02 & 116 & 116.00 &  0.00\\
j12023 3.json & 1 & 0 & Optimal &  0.02 & 99 & 99.00 &  0.00\\
j12023 4.json & 1 & 0 & Optimal &  0.02 & 106 & 106.00 &  0.00\\
j12023 5.json & 1 & 0 & Optimal &  0.06 & 99 & 99.00 &  0.00\\
j12023 6.json & 1 & 0 & Optimal &  0.03 & 106 & 106.00 &  0.00\\
j12023 7.json & 1 & 0 & Optimal &  0.02 & 104 & 104.00 &  0.00\\
j12023 8.json & 1 & 0 & Optimal &  0.02 & 101 & 101.00 &  0.00\\
j12023 9.json & 1 & 0 & Optimal &  0.03 & 107 & 107.00 &  0.00\\
j12024 1.json & 1 & 0 & Optimal &  0.02 & 93 & 93.00 &  0.00\\
j12024 10.json & 1 & 0 & Optimal &  0.02 & 91 & 91.00 &  0.00\\
j12024 2.json & 1 & 0 & Optimal &  0.05 & 91 & 91.00 &  0.00\\
j12024 3.json & 1 & 0 & Optimal &  0.03 & 89 & 89.00 &  0.00\\
j12024 4.json & 1 & 0 & Optimal &  0.03 & 101 & 101.00 &  0.00\\
j12024 5.json & 1 & 0 & Optimal &  0.03 & 86 & 86.00 &  0.00\\
j12024 6.json & 1 & 0 & Optimal &  0.02 & 95 & 95.00 &  0.00\\
j12024 7.json & 1 & 0 & Optimal &  0.03 & 112 & 112.00 &  0.00\\
j12024 8.json & 1 & 0 & Optimal &  0.02 & 104 & 104.00 &  0.00\\
j12024 9.json & 1 & 0 & Optimal &  0.38 & 82 & 82.00 &  0.00\\
j12025 1.json & 1 & 0 & Optimal &  0.02 & 82 & 82.00 &  0.00\\
j12025 10.json & 1 & 0 & Optimal &  0.02 & 92 & 92.00 &  0.00\\
j12025 2.json & 1 & 0 & Optimal &  0.02 & 108 & 108.00 &  0.00\\
j12025 3.json & 1 & 0 & Optimal &  0.02 & 100 & 100.00 &  0.00\\
j12025 4.json & 1 & 0 & Optimal &  0.02 & 117 & 117.00 &  0.00\\
j12025 5.json & 1 & 0 & Optimal &  0.02 & 100 & 100.00 &  0.00\\
j12025 6.json & 1 & 0 & Optimal &  0.03 & 92 & 92.00 &  0.00\\
j12025 7.json & 1 & 0 & Optimal &  0.08 & 92 & 92.00 &  0.00\\
j12025 8.json & 1 & 0 & Optimal &  0.02 & 80 & 80.00 &  0.00\\
j12025 9.json & 1 & 0 & Optimal &  0.02 & 94 & 94.00 &  0.00\\
j12026 1.json & 1 & 0 & Solution & 60.02 & 177 & 148.00 & 16.38\\
j12026 10.json & 1 & 0 & Solution & 60.01 & 191 & 157.00 & 17.80\\
j12026 2.json & 1 & 0 & Solution & 60.01 & 176 & 147.00 & 16.48\\
j12026 3.json & 1 & 0 & Solution & 60.00 & 173 & 154.00 & 10.98\\
j12026 4.json & 1 & 0 & Solution & 60.02 & 176 & 152.00 & 13.64\\
j12026 5.json & 1 & 0 & Solution & 60.01 & 164 & 138.00 & 15.85\\
j12026 6.json & 1 & 0 & Solution & 60.00 & 191 & 170.00 & 10.99\\
j12026 7.json & 1 & 0 & Solution & 60.02 & 163 & 143.00 & 12.27\\
j12026 8.json & 1 & 0 & Solution & 60.01 & 178 & 155.00 & 12.92\\
j12026 9.json & 1 & 0 & Solution & 60.01 & 181 & 157.00 & 13.26\\
j12027 1.json & 1 & 0 & Solution & 60.02 & 111 & 107.00 &  3.60\\
j12027 10.json & 1 & 0 & Solution & 60.01 & 118 & 109.00 &  7.63\\
j12027 2.json & 1 & 0 & Solution & 60.01 & 119 & 107.00 & 10.08\\
j12027 3.json & 1 & 0 & Solution & 60.01 & 146 & 140.00 &  4.11\\
j12027 4.json & 1 & 0 & Solution & 60.02 & 108 & 105.00 &  2.78\\
j12027 5.json & 1 & 0 & Solution & 60.02 & 114 & 103.00 &  9.65\\
j12027 6.json & 1 & 0 & Solution & 60.00 & 149 & 131.00 & 12.08\\
j12027 7.json & 1 & 0 & Solution & 60.01 & 127 & 118.00 &  7.09\\
j12027 8.json & 1 & 0 & Solution & 60.00 & 142 & 135.00 &  4.93\\
j12027 9.json & 1 & 0 & Solution & 60.02 & 131 & 120.00 &  8.40\\
j12028 1.json & 1 & 0 & Solution & 60.02 & 108 & 105.00 &  2.78\\
j12028 10.json & 1 & 0 & Solution & 60.00 & 117 & 114.00 &  2.56\\
j12028 2.json & 1 & 0 & Optimal &  5.74 & 110 & 110.00 &  0.00\\
j12028 3.json & 1 & 0 & Optimal &  0.02 & 101 & 101.00 &  0.00\\
j12028 4.json & 1 & 0 & Optimal & 25.01 & 112 & 112.00 &  0.00\\
j12028 5.json & 1 & 0 & Optimal &  0.02 & 102 & 102.00 &  0.00\\
j12028 6.json & 1 & 0 & Optimal &  1.77 & 103 & 103.00 &  0.00\\
j12028 7.json & 1 & 0 & Solution & 60.00 & 111 & 107.00 &  3.60\\
j12028 8.json & 1 & 0 & Solution & 60.00 & 100 & 98.00 &  2.00\\
j12028 9.json & 1 & 0 & Solution & 60.02 & 98 & 97.00 &  1.02\\
j12029 1.json & 1 & 0 & Optimal &  0.03 & 104 & 104.00 &  0.00\\
j12029 10.json & 1 & 0 & Optimal &  0.03 & 96 & 96.00 &  0.00\\
j12029 2.json & 1 & 0 & Optimal &  0.03 & 91 & 91.00 &  0.00\\
j12029 3.json & 1 & 0 & Solution & 60.02 & 99 & 95.00 &  4.04\\
j12029 4.json & 1 & 0 & Optimal &  3.90 & 80 & 80.00 &  0.00\\
j12029 5.json & 1 & 0 & Optimal &  0.30 & 102 & 102.00 &  0.00\\
j12029 6.json & 1 & 0 & Solution & 60.01 & 91 & 88.00 &  3.30\\
j12029 7.json & 1 & 0 & Optimal &  0.02 & 97 & 97.00 &  0.00\\
j12029 8.json & 1 & 0 & Optimal &  1.56 & 80 & 80.00 &  0.00\\
j12029 9.json & 1 & 0 & Optimal &  0.03 & 97 & 97.00 &  0.00\\
j1202 1.json & 1 & 0 & Optimal &  0.85 & 87 & 87.00 &  0.00\\
j1202 10.json & 1 & 0 & Optimal &  0.55 & 96 & 96.00 &  0.00\\
j1202 2.json & 1 & 0 & Optimal &  1.13 & 75 & 75.00 &  0.00\\
j1202 3.json & 1 & 0 & Optimal &  1.63 & 92 & 92.00 &  0.00\\
j1202 4.json & 1 & 0 & Optimal &  0.47 & 95 & 95.00 &  0.00\\
j1202 5.json & 1 & 0 & Optimal &  0.60 & 103 & 103.00 &  0.00\\
j1202 6.json & 1 & 0 & Optimal &  0.37 & 92 & 92.00 &  0.00\\
j1202 7.json & 1 & 0 & Optimal &  0.20 & 90 & 90.00 &  0.00\\
j1202 8.json & 1 & 0 & Optimal &  0.41 & 83 & 83.00 &  0.00\\
j1202 9.json & 1 & 0 & Optimal &  1.27 & 94 & 94.00 &  0.00\\
j12030 1.json & 1 & 0 & Optimal &  0.03 & 102 & 102.00 &  0.00\\
j12030 10.json & 1 & 0 & Optimal &  0.03 & 86 & 86.00 &  0.00\\
j12030 2.json & 1 & 0 & Optimal &  0.03 & 112 & 112.00 &  0.00\\
j12030 3.json & 1 & 0 & Optimal &  0.02 & 108 & 108.00 &  0.00\\
j12030 4.json & 1 & 0 & Optimal &  0.03 & 83 & 83.00 &  0.00\\
j12030 5.json & 1 & 0 & Optimal &  2.12 & 83 & 83.00 &  0.00\\
j12030 6.json & 1 & 0 & Optimal &  0.03 & 79 & 79.00 &  0.00\\
j12030 7.json & 1 & 0 & Optimal &  0.63 & 93 & 93.00 &  0.00\\
j12030 8.json & 1 & 0 & Optimal &  0.02 & 79 & 79.00 &  0.00\\
j12030 9.json & 1 & 0 & Optimal &  0.02 & 93 & 93.00 &  0.00\\
j12031 1.json & 1 & 0 & Solution & 60.02 & 206 & 178.00 & 13.59\\
j12031 10.json & 1 & 0 & Solution & 60.03 & 234 & 201.00 & 14.10\\
j12031 2.json & 1 & 0 & Solution & 60.02 & 200 & 174.00 & 13.00\\
j12031 3.json & 1 & 0 & Solution & 60.02 & 181 & 158.00 & 12.71\\
j12031 4.json & 1 & 0 & Solution & 60.01 & 234 & 188.00 & 19.66\\
j12031 5.json & 1 & 0 & Solution & 60.02 & 211 & 184.00 & 12.80\\
j12031 6.json & 1 & 0 & Solution & 60.02 & 201 & 182.00 &  9.45\\
j12031 7.json & 1 & 0 & Solution & 60.02 & 214 & 189.00 & 11.68\\
j12031 8.json & 1 & 0 & Solution & 60.00 & 199 & 172.00 & 13.57\\
j12031 9.json & 1 & 0 & Solution & 60.02 & 198 & 173.00 & 12.63\\
j12032 1.json & 1 & 0 & Solution & 60.01 & 150 & 141.00 &  6.00\\
j12032 10.json & 1 & 0 & Solution & 60.02 & 133 & 125.00 &  6.02\\
j12032 2.json & 1 & 0 & Solution & 60.01 & 135 & 123.00 &  8.89\\
j12032 3.json & 1 & 0 & Solution & 60.01 & 150 & 134.00 & 10.67\\
j12032 4.json & 1 & 0 & Solution & 60.02 & 139 & 127.00 &  8.63\\
j12032 5.json & 1 & 0 & Solution & 60.00 & 142 & 133.00 &  6.34\\
j12032 6.json & 1 & 0 & Solution & 60.01 & 132 & 122.00 &  7.58\\
j12032 7.json & 1 & 0 & Solution & 60.01 & 125 & 118.00 &  5.60\\
j12032 8.json & 1 & 0 & Solution & 60.01 & 139 & 132.00 &  5.04\\
j12032 9.json & 1 & 0 & Solution & 60.00 & 130 & 125.00 &  3.85\\
j12033 1.json & 1 & 0 & Solution & 60.00 & 109 & 105.00 &  3.67\\
j12033 10.json & 1 & 0 & Solution & 60.02 & 109 & 102.00 &  6.42\\
j12033 2.json & 1 & 0 & Solution & 60.02 & 116 & 107.00 &  7.76\\
j12033 3.json & 1 & 0 & Solution & 60.03 & 110 & 102.00 &  7.27\\
j12033 4.json & 1 & 0 & Solution & 60.02 & 114 & 106.00 &  7.02\\
j12033 5.json & 1 & 0 & Solution & 60.02 & 144 & 133.00 &  7.64\\
j12033 6.json & 1 & 0 & Solution & 60.01 & 117 & 115.00 &  1.71\\
j12033 7.json & 1 & 0 & Solution & 60.01 & 125 & 121.00 &  3.20\\
j12033 8.json & 1 & 0 & Solution & 60.02 & 114 & 107.00 &  6.14\\
j12033 9.json & 1 & 0 & Solution & 60.01 & 117 & 109.00 &  6.84\\
j12034 1.json & 1 & 0 & Solution & 60.02 & 79 & 76.00 &  3.80\\
j12034 10.json & 1 & 0 & Optimal &  0.05 & 101 & 101.00 &  0.00\\
j12034 2.json & 1 & 0 & Solution & 60.00 & 107 & 103.00 &  3.74\\
j12034 3.json & 1 & 0 & Solution & 60.01 & 103 & 100.00 &  2.91\\
j12034 4.json & 1 & 0 & Optimal &  3.83 & 95 & 95.00 &  0.00\\
j12034 5.json & 1 & 0 & Solution & 60.01 & 104 & 101.00 &  2.88\\
j12034 6.json & 1 & 0 & Optimal &  0.05 & 100 & 100.00 &  0.00\\
j12034 7.json & 1 & 0 & Optimal &  2.62 & 105 & 105.00 &  0.00\\
j12034 8.json & 1 & 0 & Solution & 60.01 & 91 & 86.00 &  5.49\\
j12034 9.json & 1 & 0 & Solution & 60.01 & 96 & 91.00 &  5.21\\
j12035 1.json & 1 & 0 & Optimal &  0.03 & 87 & 87.00 &  0.00\\
j12035 10.json & 1 & 0 & Optimal &  0.03 & 86 & 86.00 &  0.00\\
j12035 2.json & 1 & 0 & Solution & 60.02 & 112 & 111.00 &  0.89\\
j12035 3.json & 1 & 0 & Optimal &  1.08 & 77 & 77.00 &  0.00\\
j12035 4.json & 1 & 0 & Optimal &  0.05 & 101 & 101.00 &  0.00\\
j12035 5.json & 1 & 0 & Solution & 60.00 & 93 & 92.00 &  1.08\\
j12035 6.json & 1 & 0 & Optimal &  0.02 & 86 & 86.00 &  0.00\\
j12035 7.json & 1 & 0 & Optimal &  0.02 & 99 & 99.00 &  0.00\\
j12035 8.json & 1 & 0 & Optimal &  0.03 & 101 & 101.00 &  0.00\\
j12035 9.json & 1 & 0 & Optimal &  0.79 & 91 & 91.00 &  0.00\\
j12036 1.json & 1 & 0 & Solution & 60.01 & 218 & 199.00 &  8.72\\
j12036 10.json & 1 & 0 & Solution & 60.00 & 224 & 197.00 & 12.05\\
j12036 2.json & 1 & 0 & Solution & 60.02 & 231 & 201.00 & 12.99\\
j12036 3.json & 1 & 0 & Solution & 60.01 & 237 & 217.00 &  8.44\\
j12036 4.json & 1 & 0 & Solution & 60.01 & 250 & 215.00 & 14.00\\
j12036 5.json & 1 & 0 & Solution & 60.02 & 240 & 210.00 & 12.50\\
j12036 6.json & 1 & 0 & Solution & 60.01 & 236 & 204.00 & 13.56\\
j12036 7.json & 1 & 0 & Solution & 60.01 & 215 & 195.00 &  9.30\\
j12036 8.json & 1 & 0 & Solution & 60.00 & 182 & 152.00 & 16.48\\
j12036 9.json & 1 & 0 & Solution & 60.01 & 228 & 198.00 & 13.16\\
j12037 1.json & 1 & 0 & Solution & 60.01 & 149 & 138.00 &  7.38\\
j12037 10.json & 1 & 0 & Solution & 60.02 & 136 & 127.00 &  6.62\\
j12037 2.json & 1 & 0 & Solution & 60.01 & 149 & 141.00 &  5.37\\
j12037 3.json & 1 & 0 & Solution & 60.02 & 143 & 135.00 &  5.59\\
j12037 4.json & 1 & 0 & Solution & 60.03 & 166 & 156.00 &  6.02\\
j12037 5.json & 1 & 0 & Solution & 60.02 & 212 & 194.00 &  8.49\\
j12037 6.json & 1 & 0 & Solution & 60.03 & 169 & 154.00 &  8.88\\
j12037 7.json & 1 & 0 & Solution & 60.02 & 166 & 150.00 &  9.64\\
j12037 8.json & 1 & 0 & Solution & 60.01 & 184 & 166.00 &  9.78\\
j12037 9.json & 1 & 0 & Solution & 60.00 & 149 & 138.00 &  7.38\\
j12038 1.json & 1 & 0 & Solution & 60.02 & 111 & 105.00 &  5.41\\
j12038 10.json & 1 & 0 & Solution & 60.02 & 143 & 137.00 &  4.20\\
j12038 2.json & 1 & 0 & Solution & 60.02 & 129 & 119.00 &  7.75\\
j12038 3.json & 1 & 0 & Solution & 60.02 & 158 & 153.00 &  3.16\\
j12038 4.json & 1 & 0 & Solution & 60.01 & 143 & 138.00 &  3.50\\
j12038 5.json & 1 & 0 & Solution & 60.00 & 116 & 110.00 &  5.17\\
j12038 6.json & 1 & 0 & Solution & 60.02 & 125 & 118.00 &  5.60\\
j12038 7.json & 1 & 0 & Solution & 60.02 & 107 & 102.00 &  4.67\\
j12038 8.json & 1 & 0 & Solution & 60.02 & 128 & 121.00 &  5.47\\
j12038 9.json & 1 & 0 & Solution & 60.01 & 135 & 134.00 &  0.74\\
j12039 1.json & 1 & 0 & Optimal & 27.11 & 95 & 95.00 &  0.00\\
j12039 10.json & 1 & 0 & Solution & 60.00 & 112 & 105.00 &  6.25\\
j12039 2.json & 1 & 0 & Solution & 60.02 & 111 & 105.00 &  5.41\\
j12039 3.json & 1 & 0 & Solution & 60.02 & 114 & 109.00 &  4.39\\
j12039 4.json & 1 & 0 & Solution & 60.01 & 100 & 97.00 &  3.00\\
j12039 5.json & 1 & 0 & Optimal &  0.05 & 106 & 106.00 &  0.00\\
j12039 6.json & 1 & 0 & Optimal &  0.05 & 95 & 95.00 &  0.00\\
j12039 7.json & 1 & 0 & Solution & 60.01 & 106 & 101.00 &  4.72\\
j12039 8.json & 1 & 0 & Solution & 60.00 & 98 & 93.00 &  5.10\\
j12039 9.json & 1 & 0 & Solution & 60.01 & 94 & 89.00 &  5.32\\
j1203 1.json & 1 & 0 & Optimal &  0.96 & 80 & 80.00 &  0.00\\
j1203 10.json & 1 & 0 & Optimal &  0.02 & 103 & 103.00 &  0.00\\
j1203 2.json & 1 & 0 & Optimal &  0.02 & 88 & 88.00 &  0.00\\
j1203 3.json & 1 & 0 & Optimal &  0.03 & 100 & 100.00 &  0.00\\
j1203 4.json & 1 & 0 & Optimal &  0.11 & 71 & 71.00 &  0.00\\
j1203 5.json & 1 & 0 & Optimal &  0.05 & 84 & 84.00 &  0.00\\
j1203 6.json & 1 & 0 & Optimal &  0.03 & 102 & 102.00 &  0.00\\
j1203 7.json & 1 & 0 & Optimal &  0.05 & 93 & 93.00 &  0.00\\
j1203 8.json & 1 & 0 & Optimal &  0.03 & 77 & 77.00 &  0.00\\
j1203 9.json & 1 & 0 & Optimal &  0.02 & 86 & 86.00 &  0.00\\
j12040 1.json & 1 & 0 & Solution & 60.02 & 82 & 80.00 &  2.44\\
j12040 10.json & 1 & 0 & Optimal &  0.04 & 96 & 96.00 &  0.00\\
j12040 2.json & 1 & 0 & Optimal &  4.82 & 90 & 90.00 &  0.00\\
j12040 3.json & 1 & 0 & Optimal &  1.41 & 87 & 87.00 &  0.00\\
j12040 4.json & 1 & 0 & Optimal &  0.02 & 112 & 112.00 &  0.00\\
j12040 5.json & 1 & 0 & Optimal &  0.03 & 101 & 101.00 &  0.00\\
j12040 6.json & 1 & 0 & Optimal &  0.02 & 90 & 90.00 &  0.00\\
j12040 7.json & 1 & 0 & Optimal &  0.02 & 91 & 91.00 &  0.00\\
j12040 8.json & 1 & 0 & Optimal &  0.03 & 97 & 97.00 &  0.00\\
j12040 9.json & 1 & 0 & Optimal &  0.05 & 117 & 117.00 &  0.00\\
j12041 1.json & 1 & 0 & Optimal &  0.58 & 127 & 127.00 &  0.00\\
j12041 10.json & 1 & 0 & Optimal &  0.85 & 136 & 136.00 &  0.00\\
j12041 2.json & 1 & 0 & Optimal & 19.30 & 141 & 141.00 &  0.00\\
j12041 3.json & 1 & 0 & Optimal &  1.80 & 141 & 141.00 &  0.00\\
j12041 4.json & 1 & 0 & Optimal &  2.32 & 116 & 116.00 &  0.00\\
j12041 5.json & 1 & 0 & Optimal &  0.87 & 138 & 138.00 &  0.00\\
j12041 6.json & 1 & 0 & Optimal &  1.18 & 113 & 113.00 &  0.00\\
j12041 7.json & 1 & 0 & Optimal &  7.03 & 109 & 109.00 &  0.00\\
j12041 8.json & 1 & 0 & Optimal &  3.97 & 138 & 138.00 &  0.00\\
j12041 9.json & 1 & 0 & Optimal &  5.53 & 121 & 121.00 &  0.00\\
j12042 1.json & 1 & 0 & Solution & 60.01 & 109 & 104.00 &  4.59\\
j12042 10.json & 1 & 0 & Optimal &  0.89 & 118 & 118.00 &  0.00\\
j12042 2.json & 1 & 0 & Optimal &  0.02 & 126 & 126.00 &  0.00\\
j12042 3.json & 1 & 0 & Optimal &  0.41 & 106 & 106.00 &  0.00\\
j12042 4.json & 1 & 0 & Optimal &  0.43 & 104 & 104.00 &  0.00\\
j12042 5.json & 1 & 0 & Optimal &  4.11 & 120 & 120.00 &  0.00\\
j12042 6.json & 1 & 0 & Optimal &  2.77 & 119 & 119.00 &  0.00\\
j12042 7.json & 1 & 0 & Optimal &  0.03 & 123 & 123.00 &  0.00\\
j12042 8.json & 1 & 0 & Optimal &  5.00 & 113 & 113.00 &  0.00\\
j12042 9.json & 1 & 0 & Optimal &  0.41 & 104 & 104.00 &  0.00\\
j12043 1.json & 1 & 0 & Optimal &  0.02 & 105 & 105.00 &  0.00\\
j12043 10.json & 1 & 0 & Optimal &  0.02 & 113 & 113.00 &  0.00\\
j12043 2.json & 1 & 0 & Optimal &  0.02 & 120 & 120.00 &  0.00\\
j12043 3.json & 1 & 0 & Optimal &  0.06 & 95 & 95.00 &  0.00\\
j12043 4.json & 1 & 0 & Optimal &  0.49 & 105 & 105.00 &  0.00\\
j12043 5.json & 1 & 0 & Optimal &  0.11 & 105 & 105.00 &  0.00\\
j12043 6.json & 1 & 0 & Optimal &  0.86 & 98 & 98.00 &  0.00\\
j12043 7.json & 1 & 0 & Optimal &  0.06 & 122 & 122.00 &  0.00\\
j12043 8.json & 1 & 0 & Optimal &  0.03 & 115 & 115.00 &  0.00\\
j12043 9.json & 1 & 0 & Optimal &  0.03 & 105 & 105.00 &  0.00\\
j12044 1.json & 1 & 0 & Optimal &  0.02 & 100 & 100.00 &  0.00\\
j12044 10.json & 1 & 0 & Optimal &  0.02 & 98 & 98.00 &  0.00\\
j12044 2.json & 1 & 0 & Optimal &  0.13 & 112 & 112.00 &  0.00\\
j12044 3.json & 1 & 0 & Optimal &  0.03 & 107 & 107.00 &  0.00\\
j12044 4.json & 1 & 0 & Optimal &  0.03 & 95 & 95.00 &  0.00\\
j12044 5.json & 1 & 0 & Optimal &  0.03 & 98 & 98.00 &  0.00\\
j12044 6.json & 1 & 0 & Optimal &  0.02 & 106 & 106.00 &  0.00\\
j12044 7.json & 1 & 0 & Optimal &  0.03 & 98 & 98.00 &  0.00\\
j12044 8.json & 1 & 0 & Optimal &  0.14 & 108 & 108.00 &  0.00\\
j12044 9.json & 1 & 0 & Optimal &  0.03 & 91 & 91.00 &  0.00\\
j12045 1.json & 1 & 0 & Optimal &  0.02 & 108 & 108.00 &  0.00\\
j12045 10.json & 1 & 0 & Optimal &  0.02 & 99 & 99.00 &  0.00\\
j12045 2.json & 1 & 0 & Optimal &  0.03 & 91 & 91.00 &  0.00\\
j12045 3.json & 1 & 0 & Optimal &  0.03 & 98 & 98.00 &  0.00\\
j12045 4.json & 1 & 0 & Optimal &  0.03 & 103 & 103.00 &  0.00\\
j12045 5.json & 1 & 0 & Optimal &  0.04 & 116 & 116.00 &  0.00\\
j12045 6.json & 1 & 0 & Optimal &  0.02 & 125 & 125.00 &  0.00\\
j12045 7.json & 1 & 0 & Optimal &  0.02 & 103 & 103.00 &  0.00\\
j12045 8.json & 1 & 0 & Optimal &  0.02 & 103 & 103.00 &  0.00\\
j12045 9.json & 1 & 0 & Optimal &  0.02 & 114 & 114.00 &  0.00\\
j12046 1.json & 1 & 0 & Solution & 60.02 & 197 & 158.00 & 19.80\\
j12046 10.json & 1 & 0 & Solution & 60.01 & 195 & 168.00 & 13.85\\
j12046 2.json & 1 & 0 & Solution & 60.02 & 204 & 174.00 & 14.71\\
j12046 3.json & 1 & 0 & Solution & 60.01 & 181 & 150.00 & 17.13\\
j12046 4.json & 1 & 0 & Solution & 60.01 & 173 & 154.00 & 10.98\\
j12046 5.json & 1 & 0 & Solution & 60.02 & 158 & 137.00 & 13.29\\
j12046 6.json & 1 & 0 & Solution & 60.01 & 181 & 157.00 & 13.26\\
j12046 7.json & 1 & 0 & Solution & 60.02 & 172 & 149.00 & 13.37\\
j12046 8.json & 1 & 0 & Solution & 60.01 & 180 & 156.00 & 13.33\\
j12046 9.json & 1 & 0 & Solution & 60.02 & 168 & 146.00 & 13.10\\
j12047 1.json & 1 & 0 & Solution & 60.01 & 146 & 120.00 & 17.81\\
j12047 10.json & 1 & 0 & Solution & 60.02 & 134 & 130.00 &  2.99\\
j12047 2.json & 1 & 0 & Solution & 60.00 & 134 & 122.00 &  8.96\\
j12047 3.json & 1 & 0 & Solution & 60.01 & 127 & 120.00 &  5.51\\
j12047 4.json & 1 & 0 & Solution & 60.01 & 135 & 120.00 & 11.11\\
j12047 5.json & 1 & 0 & Solution & 60.01 & 131 & 120.00 &  8.40\\
j12047 6.json & 1 & 0 & Solution & 60.02 & 139 & 129.00 &  7.19\\
j12047 7.json & 1 & 0 & Solution & 60.02 & 121 & 113.00 &  6.61\\
j12047 8.json & 1 & 0 & Solution & 60.01 & 136 & 120.00 & 11.76\\
j12047 9.json & 1 & 0 & Solution & 60.00 & 146 & 136.00 &  6.85\\
j12048 1.json & 1 & 0 & Optimal & 53.72 & 100 & 100.00 &  0.00\\
j12048 10.json & 1 & 0 & Solution & 60.02 & 112 & 110.00 &  1.79\\
j12048 2.json & 1 & 0 & Solution & 60.01 & 114 & 111.00 &  2.63\\
j12048 3.json & 1 & 0 & Solution & 60.01 & 113 & 108.00 &  4.42\\
j12048 4.json & 1 & 0 & Solution & 60.01 & 128 & 123.00 &  3.91\\
j12048 5.json & 1 & 0 & Solution & 60.01 & 112 & 109.00 &  2.68\\
j12048 6.json & 1 & 0 & Solution & 60.01 & 106 & 101.00 &  4.72\\
j12048 7.json & 1 & 0 & Solution & 60.04 & 108 & 104.00 &  3.70\\
j12048 8.json & 1 & 0 & Solution & 60.01 & 116 & 112.00 &  3.45\\
j12048 9.json & 1 & 0 & Optimal & 12.52 & 113 & 113.00 &  0.00\\
j12049 1.json & 1 & 0 & Optimal &  0.04 & 96 & 96.00 &  0.00\\
j12049 10.json & 1 & 0 & Solution & 60.02 & 97 & 96.00 &  1.03\\
j12049 2.json & 1 & 0 & Solution & 60.01 & 109 & 105.00 &  3.67\\
j12049 3.json & 1 & 0 & Solution & 60.01 & 96 & 95.00 &  1.04\\
j12049 4.json & 1 & 0 & Solution & 60.01 & 97 & 95.00 &  2.06\\
j12049 5.json & 1 & 0 & Optimal & 11.35 & 89 & 89.00 &  0.00\\
j12049 6.json & 1 & 0 & Optimal &  0.03 & 128 & 128.00 &  0.00\\
j12049 7.json & 1 & 0 & Optimal &  5.93 & 99 & 99.00 &  0.00\\
j12049 8.json & 1 & 0 & Optimal &  4.76 & 113 & 113.00 &  0.00\\
j12049 9.json & 1 & 0 & Optimal &  1.70 & 97 & 97.00 &  0.00\\
j1204 1.json & 1 & 0 & Optimal &  0.02 & 74 & 74.00 &  0.00\\
j1204 10.json & 1 & 0 & Optimal &  0.02 & 77 & 77.00 &  0.00\\
j1204 2.json & 1 & 0 & Optimal &  0.02 & 107 & 107.00 &  0.00\\
j1204 3.json & 1 & 0 & Optimal &  0.03 & 95 & 95.00 &  0.00\\
j1204 4.json & 1 & 0 & Optimal &  0.02 & 75 & 75.00 &  0.00\\
j1204 5.json & 1 & 0 & Optimal &  0.02 & 74 & 74.00 &  0.00\\
j1204 6.json & 1 & 0 & Optimal &  0.03 & 90 & 90.00 &  0.00\\
j1204 7.json & 1 & 0 & Optimal &  0.02 & 81 & 81.00 &  0.00\\
j1204 8.json & 1 & 0 & Optimal &  0.02 & 90 & 90.00 &  0.00\\
j1204 9.json & 1 & 0 & Optimal &  0.02 & 79 & 79.00 &  0.00\\
j12050 1.json & 1 & 0 & Optimal &  0.03 & 116 & 116.00 &  0.00\\
j12050 10.json & 1 & 0 & Optimal &  0.07 & 103 & 103.00 &  0.00\\
j12050 2.json & 1 & 0 & Optimal &  4.06 & 112 & 112.00 &  0.00\\
j12050 3.json & 1 & 0 & Optimal &  0.03 & 111 & 111.00 &  0.00\\
j12050 4.json & 1 & 0 & Solution & 60.02 & 100 & 99.00 &  1.00\\
j12050 5.json & 1 & 0 & Optimal &  0.16 & 100 & 100.00 &  0.00\\
j12050 6.json & 1 & 0 & Optimal &  0.02 & 102 & 102.00 &  0.00\\
j12050 7.json & 1 & 0 & Optimal &  0.03 & 137 & 137.00 &  0.00\\
j12050 8.json & 1 & 0 & Optimal &  0.03 & 112 & 112.00 &  0.00\\
j12050 9.json & 1 & 0 & Optimal &  0.03 & 101 & 101.00 &  0.00\\
j12051 1.json & 1 & 0 & Solution & 60.02 & 215 & 178.00 & 17.21\\
j12051 10.json & 1 & 0 & Solution & 60.01 & 239 & 192.00 & 19.67\\
j12051 2.json & 1 & 0 & Solution & 60.01 & 227 & 191.00 & 15.86\\
j12051 3.json & 1 & 0 & Solution & 60.01 & 233 & 190.00 & 18.45\\
j12051 4.json & 1 & 0 & Solution & 60.01 & 219 & 195.00 & 10.96\\
j12051 5.json & 1 & 0 & Solution & 60.02 & 234 & 194.00 & 17.09\\
j12051 6.json & 1 & 0 & Solution & 60.02 & 221 & 188.00 & 14.93\\
j12051 7.json & 1 & 0 & Solution & 60.01 & 218 & 180.00 & 17.43\\
j12051 8.json & 1 & 0 & Solution & 60.01 & 215 & 182.00 & 15.35\\
j12051 9.json & 1 & 0 & Solution & 60.01 & 222 & 187.00 & 15.77\\
j12052 1.json & 1 & 0 & Solution & 60.01 & 182 & 154.00 & 15.38\\
j12052 10.json & 1 & 0 & Solution & 60.01 & 148 & 133.00 & 10.14\\
j12052 2.json & 1 & 0 & Solution & 60.02 & 188 & 167.00 & 11.17\\
j12052 3.json & 1 & 0 & Solution & 60.01 & 139 & 124.00 & 10.79\\
j12052 4.json & 1 & 0 & Solution & 60.01 & 176 & 157.00 & 10.80\\
j12052 5.json & 1 & 0 & Solution & 60.01 & 173 & 157.00 &  9.25\\
j12052 6.json & 1 & 0 & Solution & 60.01 & 207 & 182.00 & 12.08\\
j12052 7.json & 1 & 0 & Solution & 60.01 & 153 & 142.00 &  7.19\\
j12052 8.json & 1 & 0 & Solution & 60.01 & 164 & 147.00 & 10.37\\
j12052 9.json & 1 & 0 & Solution & 60.02 & 153 & 142.00 &  7.19\\
j12053 1.json & 1 & 0 & Solution & 60.01 & 146 & 137.00 &  6.16\\
j12053 10.json & 1 & 0 & Solution & 60.01 & 135 & 124.00 &  8.15\\
j12053 2.json & 1 & 0 & Solution & 60.01 & 118 & 109.00 &  7.63\\
j12053 3.json & 1 & 0 & Solution & 60.01 & 113 & 106.00 &  6.19\\
j12053 4.json & 1 & 0 & Solution & 60.01 & 147 & 137.00 &  6.80\\
j12053 5.json & 1 & 0 & Solution & 60.01 & 115 & 109.00 &  5.22\\
j12053 6.json & 1 & 0 & Solution & 60.01 & 107 & 101.00 &  5.61\\
j12053 7.json & 1 & 0 & Solution & 60.01 & 121 & 116.00 &  4.13\\
j12053 8.json & 1 & 0 & Solution & 60.01 & 143 & 135.00 &  5.59\\
j12053 9.json & 1 & 0 & Solution & 60.01 & 168 & 150.00 & 10.71\\
j12054 1.json & 1 & 0 & Solution & 60.01 & 106 & 101.00 &  4.72\\
j12054 10.json & 1 & 0 & Solution & 60.01 & 109 & 105.00 &  3.67\\
j12054 2.json & 1 & 0 & Optimal &  0.03 & 134 & 134.00 &  0.00\\
j12054 3.json & 1 & 0 & Optimal &  0.68 & 111 & 111.00 &  0.00\\
j12054 4.json & 1 & 0 & Solution & 60.01 & 121 & 119.00 &  1.65\\
j12054 5.json & 1 & 0 & Solution & 60.01 & 111 & 107.00 &  3.60\\
j12054 6.json & 1 & 0 & Solution & 60.01 & 110 & 103.00 &  6.36\\
j12054 7.json & 1 & 0 & Solution & 60.01 & 112 & 106.00 &  5.36\\
j12054 8.json & 1 & 0 & Solution & 60.01 & 103 & 99.00 &  3.88\\
j12054 9.json & 1 & 0 & Solution & 60.02 & 108 & 104.00 &  3.70\\
j12055 1.json & 1 & 0 & Solution & 60.03 & 102 & 99.00 &  2.94\\
j12055 10.json & 1 & 0 & Optimal &  0.02 & 100 & 100.00 &  0.00\\
j12055 2.json & 1 & 0 & Optimal &  0.03 & 83 & 83.00 &  0.00\\
j12055 3.json & 1 & 0 & Optimal &  0.03 & 126 & 126.00 &  0.00\\
j12055 4.json & 1 & 0 & Optimal &  0.27 & 90 & 90.00 &  0.00\\
j12055 5.json & 1 & 0 & Optimal &  0.03 & 106 & 106.00 &  0.00\\
j12055 6.json & 1 & 0 & Solution & 60.01 & 101 & 98.00 &  2.97\\
j12055 7.json & 1 & 0 & Optimal &  0.03 & 105 & 105.00 &  0.00\\
j12055 8.json & 1 & 0 & Optimal &  1.26 & 101 & 101.00 &  0.00\\
j12055 9.json & 1 & 0 & Optimal &  0.05 & 94 & 94.00 &  0.00\\
j12056 1.json & 1 & 0 & Solution & 60.00 & 245 & 215.00 & 12.24\\
j12056 10.json & 1 & 0 & Solution & 60.02 & 268 & 227.00 & 15.30\\
j12056 2.json & 1 & 0 & Solution & 60.03 & 214 & 183.00 & 14.49\\
j12056 3.json & 1 & 0 & Solution & 60.01 & 249 & 216.00 & 13.25\\
j12056 4.json & 1 & 0 & Solution & 60.00 & 234 & 201.00 & 14.10\\
j12056 5.json & 1 & 0 & Solution & 60.02 & 289 & 246.00 & 14.88\\
j12056 6.json & 1 & 0 & Solution & 60.01 & 223 & 194.00 & 13.00\\
j12056 7.json & 1 & 0 & Solution & 60.01 & 294 & 242.00 & 17.69\\
j12056 8.json & 1 & 0 & Solution & 60.01 & 298 & 250.00 & 16.11\\
j12056 9.json & 1 & 0 & Solution & 60.02 & 297 & 254.00 & 14.48\\
j12057 1.json & 1 & 0 & Solution & 60.01 & 189 & 173.00 &  8.47\\
j12057 10.json & 1 & 0 & Solution & 60.01 & 172 & 156.00 &  9.30\\
j12057 2.json & 1 & 0 & Solution & 60.01 & 166 & 151.00 &  9.04\\
j12057 3.json & 1 & 0 & Solution & 60.01 & 189 & 175.00 &  7.41\\
j12057 4.json & 1 & 0 & Solution & 60.01 & 208 & 186.00 & 10.58\\
j12057 5.json & 1 & 0 & Solution & 60.02 & 184 & 169.00 &  8.15\\
j12057 6.json & 1 & 0 & Solution & 60.01 & 193 & 173.00 & 10.36\\
j12057 7.json & 1 & 0 & Solution & 60.01 & 170 & 155.00 &  8.82\\
j12057 8.json & 1 & 0 & Solution & 60.02 & 167 & 155.00 &  7.19\\
j12057 9.json & 1 & 0 & Solution & 60.01 & 171 & 155.00 &  9.36\\
j12058 1.json & 1 & 0 & Solution & 60.01 & 144 & 132.00 &  8.33\\
j12058 10.json & 1 & 0 & Solution & 60.02 & 137 & 125.00 &  8.76\\
j12058 2.json & 1 & 0 & Solution & 60.01 & 128 & 122.00 &  4.69\\
j12058 3.json & 1 & 0 & Solution & 60.01 & 123 & 116.00 &  5.69\\
j12058 4.json & 1 & 0 & Solution & 60.02 & 150 & 139.00 &  7.33\\
j12058 5.json & 1 & 0 & Solution & 60.02 & 122 & 116.00 &  4.92\\
j12058 6.json & 1 & 0 & Solution & 60.02 & 144 & 135.00 &  6.25\\
j12058 7.json & 1 & 0 & Solution & 60.02 & 150 & 143.00 &  4.67\\
j12058 8.json & 1 & 0 & Solution & 60.02 & 136 & 125.00 &  8.09\\
j12058 9.json & 1 & 0 & Solution & 60.02 & 133 & 126.00 &  5.26\\
j12059 1.json & 1 & 0 & Solution & 60.01 & 115 & 112.00 &  2.61\\
j12059 10.json & 1 & 0 & Solution & 60.02 & 135 & 126.00 &  6.67\\
j12059 2.json & 1 & 0 & Solution & 60.02 & 108 & 103.00 &  4.63\\
j12059 3.json & 1 & 0 & Solution & 60.02 & 109 & 108.00 &  0.92\\
j12059 4.json & 1 & 0 & Solution & 60.02 & 110 & 107.00 &  2.73\\
j12059 5.json & 1 & 0 & Solution & 60.02 & 107 & 104.00 &  2.80\\
j12059 6.json & 1 & 0 & Solution & 60.02 & 118 & 111.00 &  5.93\\
j12059 7.json & 1 & 0 & Solution & 60.02 & 115 & 109.00 &  5.22\\
j12059 8.json & 1 & 0 & Solution & 60.02 & 112 & 106.00 &  5.36\\
j12059 9.json & 1 & 0 & Solution & 60.02 & 119 & 117.00 &  1.68\\
j1205 1.json & 1 & 0 & Optimal &  0.02 & 92 & 92.00 &  0.00\\
j1205 10.json & 1 & 0 & Optimal &  0.02 & 92 & 92.00 &  0.00\\
j1205 2.json & 1 & 0 & Optimal &  0.02 & 80 & 80.00 &  0.00\\
j1205 3.json & 1 & 0 & Optimal &  0.02 & 72 & 72.00 &  0.00\\
j1205 4.json & 1 & 0 & Optimal &  0.02 & 97 & 97.00 &  0.00\\
j1205 5.json & 1 & 0 & Optimal &  0.02 & 77 & 77.00 &  0.00\\
j1205 6.json & 1 & 0 & Optimal &  0.02 & 88 & 88.00 &  0.00\\
j1205 7.json & 1 & 0 & Optimal &  0.02 & 84 & 84.00 &  0.00\\
j1205 8.json & 1 & 0 & Optimal &  0.03 & 78 & 78.00 &  0.00\\
j1205 9.json & 1 & 0 & Optimal &  0.03 & 106 & 106.00 &  0.00\\
j12060 1.json & 1 & 0 & Optimal &  0.03 & 101 & 101.00 &  0.00\\
j12060 10.json & 1 & 0 & Solution & 60.02 & 90 & 88.00 &  2.22\\
j12060 2.json & 1 & 0 & Solution & 60.01 & 84 & 81.00 &  3.57\\
j12060 3.json & 1 & 0 & Solution & 60.01 & 90 & 88.00 &  2.22\\
j12060 4.json & 1 & 0 & Solution & 60.02 & 104 & 101.00 &  2.88\\
j12060 5.json & 1 & 0 & Solution & 60.02 & 106 & 103.00 &  2.83\\
j12060 6.json & 1 & 0 & Optimal &  0.05 & 110 & 110.00 &  0.00\\
j12060 7.json & 1 & 0 & Solution & 60.01 & 95 & 90.00 &  5.26\\
j12060 8.json & 1 & 0 & Optimal & 36.75 & 101 & 101.00 &  0.00\\
j12060 9.json & 1 & 0 & Optimal &  0.02 & 101 & 101.00 &  0.00\\
j1206 1.json & 1 & 0 & Solution & 60.01 & 150 & 132.00 & 12.00\\
j1206 10.json & 1 & 0 & Solution & 60.02 & 178 & 157.00 & 11.80\\
j1206 2.json & 1 & 0 & Solution & 60.01 & 140 & 126.00 & 10.00\\
j1206 3.json & 1 & 0 & Solution & 60.01 & 137 & 126.00 &  8.03\\
j1206 4.json & 1 & 0 & Solution & 60.01 & 157 & 143.00 &  8.92\\
j1206 5.json & 1 & 0 & Solution & 60.01 & 129 & 116.00 & 10.08\\
j1206 6.json & 1 & 0 & Solution & 60.01 & 158 & 138.00 & 12.66\\
j1206 7.json & 1 & 0 & Solution & 60.01 & 172 & 152.00 & 11.63\\
j1206 8.json & 1 & 0 & Solution & 60.02 & 150 & 137.00 &  8.67\\
j1206 9.json & 1 & 0 & Solution & 60.02 & 164 & 144.00 & 12.20\\
j1207 1.json & 1 & 0 & Solution & 60.01 & 104 & 98.00 &  5.77\\
j1207 10.json & 1 & 0 & Solution & 60.01 & 121 & 111.00 &  8.26\\
j1207 2.json & 1 & 0 & Solution & 60.01 & 116 & 112.00 &  3.45\\
j1207 3.json & 1 & 0 & Solution & 60.02 & 102 & 97.00 &  4.90\\
j1207 4.json & 1 & 0 & Solution & 60.01 & 115 & 106.00 &  7.83\\
j1207 5.json & 1 & 0 & Solution & 60.01 & 135 & 126.00 &  6.67\\
j1207 6.json & 1 & 0 & Solution & 60.01 & 129 & 114.00 & 11.63\\
j1207 7.json & 1 & 0 & Solution & 60.01 & 120 & 114.00 &  5.00\\
j1207 8.json & 1 & 0 & Solution & 60.01 & 99 & 93.00 &  6.06\\
j1207 9.json & 1 & 0 & Solution & 60.02 & 91 & 86.00 &  5.49\\
j1208 1.json & 1 & 0 & Optimal &  0.56 & 95 & 95.00 &  0.00\\
j1208 10.json & 1 & 0 & Solution & 60.01 & 94 & 92.00 &  2.13\\
j1208 2.json & 1 & 0 & Solution & 60.01 & 105 & 100.00 &  4.76\\
j1208 3.json & 1 & 0 & Solution & 60.02 & 96 & 94.00 &  2.08\\
j1208 4.json & 1 & 0 & Solution & 60.01 & 95 & 90.00 &  5.26\\
j1208 5.json & 1 & 0 & Solution & 60.01 & 105 & 100.00 &  4.76\\
j1208 6.json & 1 & 0 & Solution & 60.01 & 85 & 84.00 &  1.18\\
j1208 7.json & 1 & 0 & Solution & 60.02 & 88 & 87.00 &  1.14\\
j1208 8.json & 1 & 0 & Solution & 60.01 & 88 & 87.00 &  1.14\\
j1208 9.json & 1 & 0 & Solution & 60.02 & 95 & 90.00 &  5.26\\
j1209 1.json & 1 & 0 & Optimal &  0.04 & 88 & 88.00 &  0.00\\
j1209 10.json & 1 & 0 & Optimal &  0.05 & 84 & 84.00 &  0.00\\
j1209 2.json & 1 & 0 & Optimal &  0.03 & 94 & 94.00 &  0.00\\
j1209 3.json & 1 & 0 & Optimal &  0.02 & 87 & 87.00 &  0.00\\
j1209 4.json & 1 & 0 & Solution & 60.01 & 87 & 84.00 &  3.45\\
j1209 5.json & 1 & 0 & Optimal &  0.03 & 114 & 114.00 &  0.00\\
j1209 6.json & 1 & 0 & Optimal &  2.12 & 98 & 98.00 &  0.00\\
j1209 7.json & 1 & 0 & Optimal &  0.02 & 80 & 80.00 &  0.00\\
j1209 8.json & 1 & 0 & Optimal &  0.03 & 80 & 80.00 &  0.00\\
j1209 9.json & 1 & 0 & Optimal &  0.10 & 87 & 87.00 &  0.00\\
\end{longtable}



\subsection{CPSat}
\begin{longtable}{lrrlrrrr}
\caption{Results for RCPSP J120 (CPSat) (600 Instances)}\\\toprule
Name & \shortstack{Nr\\Jobs} & \shortstack{Nr\\Machines} & Status & Time & Makespan & Bound & \shortstack{Gap\\Percent}\\ \midrule
\endhead
\bottomrule
\endfoot
j12010 1.json & 1 & 0 & Optimal &  0.11 & 111 & 111.00 &  0.00\\
j12010 10.json & 1 & 0 & Optimal &  0.06 & 66 & 66.00 &  0.00\\
j12010 2.json & 1 & 0 & Optimal &  0.07 & 91 & 91.00 &  0.00\\
j12010 3.json & 1 & 0 & Optimal &  0.10 & 99 & 99.00 &  0.00\\
j12010 4.json & 1 & 0 & Optimal &  0.14 & 95 & 95.00 &  0.00\\
j12010 5.json & 1 & 0 & Optimal &  0.10 & 97 & 97.00 &  0.00\\
j12010 6.json & 1 & 0 & Optimal &  0.03 & 92 & 92.00 &  0.00\\
j12010 7.json & 1 & 0 & Optimal &  0.36 & 79 & 79.00 &  0.00\\
j12010 8.json & 1 & 0 & Optimal &  0.17 & 114 & 114.00 &  0.00\\
j12010 9.json & 1 & 0 & Optimal &  0.04 & 77 & 77.00 &  0.00\\
j12011 1.json & 1 & 0 & Solution & 60.03 & 183 & 155.00 & 15.30\\
j12011 10.json & 1 & 0 & Solution & 60.03 & 194 & 163.00 & 15.98\\
j12011 2.json & 1 & 0 & Solution & 60.05 & 166 & 145.00 & 12.65\\
j12011 3.json & 1 & 0 & Solution & 60.04 & 219 & 187.00 & 14.61\\
j12011 4.json & 1 & 0 & Solution & 60.03 & 210 & 176.00 & 16.19\\
j12011 5.json & 1 & 0 & Solution & 60.03 & 224 & 192.00 & 14.29\\
j12011 6.json & 1 & 0 & Solution & 60.07 & 223 & 187.00 & 16.14\\
j12011 7.json & 1 & 0 & Solution & 60.04 & 172 & 148.00 & 13.95\\
j12011 8.json & 1 & 0 & Solution & 60.04 & 173 & 152.00 & 12.14\\
j12011 9.json & 1 & 0 & Solution & 60.05 & 184 & 168.00 &  8.70\\
j12012 1.json & 1 & 0 & Solution & 60.05 & 146 & 125.00 & 14.38\\
j12012 10.json & 1 & 0 & Solution & 60.04 & 148 & 142.00 &  4.05\\
j12012 2.json & 1 & 0 & Solution & 60.02 & 124 & 111.00 & 10.48\\
j12012 3.json & 1 & 0 & Solution & 60.03 & 142 & 132.00 &  7.04\\
j12012 4.json & 1 & 0 & Solution & 60.03 & 129 & 121.00 &  6.20\\
j12012 5.json & 1 & 0 & Solution & 60.07 & 169 & 154.00 &  8.88\\
j12012 6.json & 1 & 0 & Solution & 60.06 & 129 & 115.00 & 10.85\\
j12012 7.json & 1 & 0 & Solution & 60.03 & 124 & 116.00 &  6.45\\
j12012 8.json & 1 & 0 & Solution & 60.03 & 125 & 113.00 &  9.60\\
j12012 9.json & 1 & 0 & Solution & 60.01 & 109 & 101.00 &  7.34\\
j12013 1.json & 1 & 0 & Solution & 60.03 & 131 & 122.00 &  6.87\\
j12013 10.json & 1 & 0 & Solution & 60.03 & 97 & 88.00 &  9.28\\
j12013 2.json & 1 & 0 & Solution & 60.04 & 90 & 88.00 &  2.22\\
j12013 3.json & 1 & 0 & Solution & 60.02 & 120 & 115.00 &  4.17\\
j12013 4.json & 1 & 0 & Solution & 60.01 & 116 & 108.00 &  6.90\\
j12013 5.json & 1 & 0 & Solution & 60.03 & 94 & 90.00 &  4.26\\
j12013 6.json & 1 & 0 & Solution & 60.02 & 102 & 95.00 &  6.86\\
j12013 7.json & 1 & 0 & Solution & 60.03 & 112 & 107.00 &  4.46\\
j12013 8.json & 1 & 0 & Solution & 60.02 & 96 & 91.00 &  5.21\\
j12013 9.json & 1 & 0 & Solution & 60.03 & 87 & 82.00 &  5.75\\
j12014 1.json & 1 & 0 & Solution & 60.03 & 87 & 84.00 &  3.45\\
j12014 10.json & 1 & 0 & Solution & 60.03 & 83 & 80.00 &  3.61\\
j12014 2.json & 1 & 0 & Solution & 60.02 & 95 & 90.00 &  5.26\\
j12014 3.json & 1 & 0 & Optimal & 28.27 & 88 & 88.00 &  0.00\\
j12014 4.json & 1 & 0 & Solution & 60.02 & 89 & 85.00 &  4.49\\
j12014 5.json & 1 & 0 & Solution & 60.03 & 99 & 93.00 &  6.06\\
j12014 6.json & 1 & 0 & Optimal & 60.01 & 91 & 91.00 &  0.00\\
j12014 7.json & 1 & 0 & Solution & 60.03 & 92 & 89.00 &  3.26\\
j12014 8.json & 1 & 0 & Solution & 60.03 & 113 & 109.00 &  3.54\\
j12014 9.json & 1 & 0 & Optimal &  0.15 & 101 & 101.00 &  0.00\\
j12015 1.json & 1 & 0 & Optimal &  0.10 & 81 & 81.00 &  0.00\\
j12015 10.json & 1 & 0 & Optimal &  0.10 & 91 & 91.00 &  0.00\\
j12015 2.json & 1 & 0 & Optimal &  0.04 & 75 & 75.00 &  0.00\\
j12015 3.json & 1 & 0 & Optimal &  0.10 & 87 & 87.00 &  0.00\\
j12015 4.json & 1 & 0 & Optimal &  0.10 & 82 & 82.00 &  0.00\\
j12015 5.json & 1 & 0 & Optimal &  0.09 & 87 & 87.00 &  0.00\\
j12015 6.json & 1 & 0 & Optimal &  0.08 & 97 & 97.00 &  0.00\\
j12015 7.json & 1 & 0 & Optimal &  0.05 & 75 & 75.00 &  0.00\\
j12015 8.json & 1 & 0 & Optimal &  0.07 & 126 & 126.00 &  0.00\\
j12015 9.json & 1 & 0 & Optimal &  0.06 & 109 & 109.00 &  0.00\\
j12016 1.json & 1 & 0 & Solution & 60.05 & 212 & 178.00 & 16.04\\
j12016 10.json & 1 & 0 & Solution & 60.04 & 228 & 201.00 & 11.84\\
j12016 2.json & 1 & 0 & Solution & 60.04 & 253 & 220.00 & 13.04\\
j12016 3.json & 1 & 0 & Solution & 60.02 & 252 & 219.00 & 13.10\\
j12016 4.json & 1 & 0 & Solution & 60.05 & 213 & 188.00 & 11.74\\
j12016 5.json & 1 & 0 & Solution & 60.05 & 213 & 181.00 & 15.02\\
j12016 6.json & 1 & 0 & Solution & 60.05 & 217 & 193.00 & 11.06\\
j12016 7.json & 1 & 0 & Solution & 60.03 & 199 & 172.00 & 13.57\\
j12016 8.json & 1 & 0 & Solution & 60.05 & 206 & 181.00 & 12.14\\
j12016 9.json & 1 & 0 & Solution & 60.09 & 223 & 187.00 & 16.14\\
j12017 1.json & 1 & 0 & Solution & 60.02 & 147 & 133.00 &  9.52\\
j12017 10.json & 1 & 0 & Solution & 60.03 & 141 & 121.00 & 14.18\\
j12017 2.json & 1 & 0 & Solution & 60.02 & 130 & 110.00 & 15.38\\
j12017 3.json & 1 & 0 & Solution & 60.03 & 113 & 98.00 & 13.27\\
j12017 4.json & 1 & 0 & Solution & 60.03 & 124 & 108.00 & 12.90\\
j12017 5.json & 1 & 0 & Solution & 60.02 & 136 & 122.00 & 10.29\\
j12017 6.json & 1 & 0 & Solution & 60.02 & 143 & 131.00 &  8.39\\
j12017 7.json & 1 & 0 & Solution & 60.02 & 153 & 140.00 &  8.50\\
j12017 8.json & 1 & 0 & Solution & 60.03 & 132 & 124.00 &  6.06\\
j12017 9.json & 1 & 0 & Solution & 60.03 & 142 & 127.00 & 10.56\\
j12018 1.json & 1 & 0 & Solution & 60.03 & 142 & 115.00 & 19.01\\
j12018 10.json & 1 & 0 & Solution & 60.04 & 100 & 90.00 & 10.00\\
j12018 2.json & 1 & 0 & Solution & 60.02 & 121 & 111.00 &  8.26\\
j12018 3.json & 1 & 0 & Solution & 60.02 & 103 & 90.00 & 12.62\\
j12018 4.json & 1 & 0 & Solution & 60.02 & 104 & 96.00 &  7.69\\
j12018 5.json & 1 & 0 & Solution & 60.02 & 123 & 106.00 & 13.82\\
j12018 6.json & 1 & 0 & Solution & 60.03 & 139 & 124.00 & 10.79\\
j12018 7.json & 1 & 0 & Solution & 60.02 & 122 & 111.00 &  9.02\\
j12018 8.json & 1 & 0 & Solution & 60.04 & 108 & 95.00 & 12.04\\
j12018 9.json & 1 & 0 & Solution & 60.03 & 95 & 83.00 & 12.63\\
j12019 1.json & 1 & 0 & Optimal &  0.10 & 88 & 88.00 &  0.00\\
j12019 10.json & 1 & 0 & Optimal &  0.06 & 88 & 88.00 &  0.00\\
j12019 2.json & 1 & 0 & Solution & 60.02 & 84 & 81.00 &  3.57\\
j12019 3.json & 1 & 0 & Solution & 60.03 & 88 & 82.00 &  6.82\\
j12019 4.json & 1 & 0 & Solution & 60.02 & 109 & 97.00 & 11.01\\
j12019 5.json & 1 & 0 & Solution & 60.03 & 108 & 99.00 &  8.33\\
j12019 6.json & 1 & 0 & Solution & 60.02 & 92 & 80.00 & 13.04\\
j12019 7.json & 1 & 0 & Optimal &  0.09 & 93 & 93.00 &  0.00\\
j12019 8.json & 1 & 0 & Solution & 60.03 & 95 & 93.00 &  2.11\\
j12019 9.json & 1 & 0 & Solution & 60.02 & 90 & 75.00 & 16.67\\
j1201 1.json & 1 & 0 & Solution & 60.02 & 105 & 104.00 &  0.95\\
j1201 10.json & 1 & 0 & Optimal & 15.90 & 108 & 108.00 &  0.00\\
j1201 2.json & 1 & 0 & Optimal &  1.55 & 109 & 109.00 &  0.00\\
j1201 3.json & 1 & 0 & Solution & 60.06 & 126 & 118.00 &  6.35\\
j1201 4.json & 1 & 0 & Optimal &  0.79 & 97 & 97.00 &  0.00\\
j1201 5.json & 1 & 0 & Optimal &  0.74 & 112 & 112.00 &  0.00\\
j1201 6.json & 1 & 0 & Optimal &  0.71 & 84 & 84.00 &  0.00\\
j1201 7.json & 1 & 0 & Optimal &  4.35 & 117 & 117.00 &  0.00\\
j1201 8.json & 1 & 0 & Optimal & 60.02 & 109 & 109.00 &  0.00\\
j1201 9.json & 1 & 0 & Optimal &  0.33 & 112 & 112.00 &  0.00\\
j12020 1.json & 1 & 0 & Optimal & 60.02 & 89 & 89.00 &  0.00\\
j12020 10.json & 1 & 0 & Optimal &  0.09 & 81 & 81.00 &  0.00\\
j12020 2.json & 1 & 0 & Optimal &  0.08 & 99 & 99.00 &  0.00\\
j12020 3.json & 1 & 0 & Solution & 60.02 & 78 & 74.00 &  5.13\\
j12020 4.json & 1 & 0 & Optimal &  0.08 & 89 & 89.00 &  0.00\\
j12020 5.json & 1 & 0 & Optimal &  0.08 & 69 & 69.00 &  0.00\\
j12020 6.json & 1 & 0 & Optimal &  0.07 & 80 & 80.00 &  0.00\\
j12020 7.json & 1 & 0 & Optimal &  0.07 & 81 & 81.00 &  0.00\\
j12020 8.json & 1 & 0 & Optimal & 60.01 & 107 & 107.00 &  0.00\\
j12020 9.json & 1 & 0 & Optimal &  0.07 & 80 & 80.00 &  0.00\\
j12021 1.json & 1 & 0 & Optimal & 60.01 & 114 & 114.00 &  0.00\\
j12021 10.json & 1 & 0 & Optimal & 13.02 & 102 & 102.00 &  0.00\\
j12021 2.json & 1 & 0 & Optimal & 37.55 & 117 & 117.00 &  0.00\\
j12021 3.json & 1 & 0 & Optimal &  1.65 & 143 & 143.00 &  0.00\\
j12021 4.json & 1 & 0 & Optimal & 11.74 & 135 & 135.00 &  0.00\\
j12021 5.json & 1 & 0 & Optimal &  5.01 & 110 & 110.00 &  0.00\\
j12021 6.json & 1 & 0 & Optimal &  1.60 & 109 & 109.00 &  0.00\\
j12021 7.json & 1 & 0 & Optimal &  5.83 & 111 & 111.00 &  0.00\\
j12021 8.json & 1 & 0 & Optimal &  0.23 & 127 & 127.00 &  0.00\\
j12021 9.json & 1 & 0 & Optimal &  0.41 & 102 & 102.00 &  0.00\\
j12022 1.json & 1 & 0 & Optimal &  1.52 & 101 & 101.00 &  0.00\\
j12022 10.json & 1 & 0 & Optimal &  0.17 & 79 & 79.00 &  0.00\\
j12022 2.json & 1 & 0 & Optimal &  0.18 & 107 & 107.00 &  0.00\\
j12022 3.json & 1 & 0 & Optimal & 60.01 & 96 & 96.00 &  0.00\\
j12022 4.json & 1 & 0 & Optimal &  0.19 & 90 & 90.00 &  0.00\\
j12022 5.json & 1 & 0 & Optimal &  0.18 & 93 & 93.00 &  0.00\\
j12022 6.json & 1 & 0 & Optimal &  0.21 & 103 & 103.00 &  0.00\\
j12022 7.json & 1 & 0 & Optimal &  0.09 & 133 & 133.00 &  0.00\\
j12022 8.json & 1 & 0 & Optimal &  3.11 & 103 & 103.00 &  0.00\\
j12022 9.json & 1 & 0 & Optimal &  0.21 & 109 & 109.00 &  0.00\\
j12023 1.json & 1 & 0 & Optimal &  0.09 & 107 & 107.00 &  0.00\\
j12023 10.json & 1 & 0 & Optimal &  0.14 & 100 & 100.00 &  0.00\\
j12023 2.json & 1 & 0 & Optimal &  0.08 & 116 & 116.00 &  0.00\\
j12023 3.json & 1 & 0 & Optimal &  0.10 & 99 & 99.00 &  0.00\\
j12023 4.json & 1 & 0 & Optimal &  0.15 & 106 & 106.00 &  0.00\\
j12023 5.json & 1 & 0 & Optimal &  0.09 & 99 & 99.00 &  0.00\\
j12023 6.json & 1 & 0 & Optimal &  0.12 & 106 & 106.00 &  0.00\\
j12023 7.json & 1 & 0 & Optimal &  0.14 & 104 & 104.00 &  0.00\\
j12023 8.json & 1 & 0 & Optimal &  0.15 & 101 & 101.00 &  0.00\\
j12023 9.json & 1 & 0 & Optimal &  0.13 & 107 & 107.00 &  0.00\\
j12024 1.json & 1 & 0 & Optimal &  0.06 & 93 & 93.00 &  0.00\\
j12024 10.json & 1 & 0 & Optimal &  0.14 & 91 & 91.00 &  0.00\\
j12024 2.json & 1 & 0 & Optimal &  0.12 & 91 & 91.00 &  0.00\\
j12024 3.json & 1 & 0 & Optimal &  0.10 & 89 & 89.00 &  0.00\\
j12024 4.json & 1 & 0 & Optimal &  0.12 & 101 & 101.00 &  0.00\\
j12024 5.json & 1 & 0 & Optimal &  0.10 & 86 & 86.00 &  0.00\\
j12024 6.json & 1 & 0 & Optimal &  0.09 & 95 & 95.00 &  0.00\\
j12024 7.json & 1 & 0 & Optimal &  0.12 & 112 & 112.00 &  0.00\\
j12024 8.json & 1 & 0 & Optimal &  0.13 & 104 & 104.00 &  0.00\\
j12024 9.json & 1 & 0 & Optimal &  0.12 & 82 & 82.00 &  0.00\\
j12025 1.json & 1 & 0 & Optimal &  0.12 & 82 & 82.00 &  0.00\\
j12025 10.json & 1 & 0 & Optimal &  0.05 & 92 & 92.00 &  0.00\\
j12025 2.json & 1 & 0 & Optimal &  0.10 & 108 & 108.00 &  0.00\\
j12025 3.json & 1 & 0 & Optimal &  0.06 & 100 & 100.00 &  0.00\\
j12025 4.json & 1 & 0 & Optimal &  0.05 & 117 & 117.00 &  0.00\\
j12025 5.json & 1 & 0 & Optimal &  0.05 & 100 & 100.00 &  0.00\\
j12025 6.json & 1 & 0 & Optimal &  0.07 & 92 & 92.00 &  0.00\\
j12025 7.json & 1 & 0 & Optimal &  0.12 & 92 & 92.00 &  0.00\\
j12025 8.json & 1 & 0 & Optimal &  0.10 & 80 & 80.00 &  0.00\\
j12025 9.json & 1 & 0 & Optimal &  0.05 & 94 & 94.00 &  0.00\\
j12026 1.json & 1 & 0 & Solution & 60.03 & 173 & 150.00 & 13.29\\
j12026 10.json & 1 & 0 & Solution & 60.02 & 186 & 160.00 & 13.98\\
j12026 2.json & 1 & 0 & Solution & 60.06 & 172 & 150.00 & 12.79\\
j12026 3.json & 1 & 0 & Solution & 60.04 & 172 & 154.00 & 10.47\\
j12026 4.json & 1 & 0 & Solution & 60.07 & 177 & 151.00 & 14.69\\
j12026 5.json & 1 & 0 & Solution & 60.05 & 159 & 136.00 & 14.47\\
j12026 6.json & 1 & 0 & Solution & 60.03 & 190 & 170.00 & 10.53\\
j12026 7.json & 1 & 0 & Solution & 60.03 & 161 & 143.00 & 11.18\\
j12026 8.json & 1 & 0 & Solution & 60.06 & 178 & 159.00 & 10.67\\
j12026 9.json & 1 & 0 & Solution & 60.03 & 178 & 160.00 & 10.11\\
j12027 1.json & 1 & 0 & Solution & 60.03 & 110 & 105.00 &  4.55\\
j12027 10.json & 1 & 0 & Solution & 60.02 & 117 & 108.00 &  7.69\\
j12027 2.json & 1 & 0 & Solution & 60.05 & 116 & 108.00 &  6.90\\
j12027 3.json & 1 & 0 & Solution & 60.07 & 149 & 141.00 &  5.37\\
j12027 4.json & 1 & 0 & Solution & 60.09 & 110 & 104.00 &  5.45\\
j12027 5.json & 1 & 0 & Solution & 60.03 & 114 & 102.00 & 10.53\\
j12027 6.json & 1 & 0 & Solution & 60.05 & 150 & 132.00 & 12.00\\
j12027 7.json & 1 & 0 & Solution & 60.08 & 126 & 115.00 &  8.73\\
j12027 8.json & 1 & 0 & Solution & 60.07 & 144 & 135.00 &  6.25\\
j12027 9.json & 1 & 0 & Solution & 60.05 & 128 & 120.00 &  6.25\\
j12028 1.json & 1 & 0 & Solution & 60.04 & 110 & 105.00 &  4.55\\
j12028 10.json & 1 & 0 & Solution & 60.03 & 117 & 111.00 &  5.13\\
j12028 2.json & 1 & 0 & Optimal & 60.02 & 110 & 110.00 &  0.00\\
j12028 3.json & 1 & 0 & Optimal &  0.20 & 101 & 101.00 &  0.00\\
j12028 4.json & 1 & 0 & Optimal & 60.02 & 112 & 112.00 &  0.00\\
j12028 5.json & 1 & 0 & Optimal &  0.21 & 102 & 102.00 &  0.00\\
j12028 6.json & 1 & 0 & Optimal &  3.89 & 103 & 103.00 &  0.00\\
j12028 7.json & 1 & 0 & Solution & 60.02 & 109 & 103.00 &  5.50\\
j12028 8.json & 1 & 0 & Solution & 60.05 & 100 & 97.00 &  3.00\\
j12028 9.json & 1 & 0 & Solution & 60.03 & 98 & 96.00 &  2.04\\
j12029 1.json & 1 & 0 & Optimal &  0.12 & 104 & 104.00 &  0.00\\
j12029 10.json & 1 & 0 & Optimal &  0.14 & 96 & 96.00 &  0.00\\
j12029 2.json & 1 & 0 & Optimal &  0.12 & 91 & 91.00 &  0.00\\
j12029 3.json & 1 & 0 & Solution & 60.05 & 98 & 94.00 &  4.08\\
j12029 4.json & 1 & 0 & Optimal &  0.88 & 80 & 80.00 &  0.00\\
j12029 5.json & 1 & 0 & Optimal &  0.58 & 102 & 102.00 &  0.00\\
j12029 6.json & 1 & 0 & Solution & 60.03 & 92 & 88.00 &  4.35\\
j12029 7.json & 1 & 0 & Optimal &  0.11 & 97 & 97.00 &  0.00\\
j12029 8.json & 1 & 0 & Optimal &  0.63 & 80 & 80.00 &  0.00\\
j12029 9.json & 1 & 0 & Optimal &  0.11 & 97 & 97.00 &  0.00\\
j1202 1.json & 1 & 0 & Optimal &  7.10 & 87 & 87.00 &  0.00\\
j1202 10.json & 1 & 0 & Optimal &  0.38 & 96 & 96.00 &  0.00\\
j1202 2.json & 1 & 0 & Optimal &  1.81 & 75 & 75.00 &  0.00\\
j1202 3.json & 1 & 0 & Optimal &  5.29 & 92 & 92.00 &  0.00\\
j1202 4.json & 1 & 0 & Optimal &  0.19 & 95 & 95.00 &  0.00\\
j1202 5.json & 1 & 0 & Optimal &  0.39 & 103 & 103.00 &  0.00\\
j1202 6.json & 1 & 0 & Optimal &  0.29 & 92 & 92.00 &  0.00\\
j1202 7.json & 1 & 0 & Optimal &  0.17 & 90 & 90.00 &  0.00\\
j1202 8.json & 1 & 0 & Optimal &  0.19 & 83 & 83.00 &  0.00\\
j1202 9.json & 1 & 0 & Optimal &  6.64 & 94 & 94.00 &  0.00\\
j12030 1.json & 1 & 0 & Optimal &  0.08 & 102 & 102.00 &  0.00\\
j12030 10.json & 1 & 0 & Optimal &  0.08 & 86 & 86.00 &  0.00\\
j12030 2.json & 1 & 0 & Optimal &  0.09 & 112 & 112.00 &  0.00\\
j12030 3.json & 1 & 0 & Optimal &  0.08 & 108 & 108.00 &  0.00\\
j12030 4.json & 1 & 0 & Optimal &  0.09 & 83 & 83.00 &  0.00\\
j12030 5.json & 1 & 0 & Optimal & 53.55 & 83 & 83.00 &  0.00\\
j12030 6.json & 1 & 0 & Optimal &  0.05 & 79 & 79.00 &  0.00\\
j12030 7.json & 1 & 0 & Optimal &  0.57 & 93 & 93.00 &  0.00\\
j12030 8.json & 1 & 0 & Optimal &  0.07 & 79 & 79.00 &  0.00\\
j12030 9.json & 1 & 0 & Optimal &  0.10 & 93 & 93.00 &  0.00\\
j12031 1.json & 1 & 0 & Solution & 60.06 & 205 & 179.00 & 12.68\\
j12031 10.json & 1 & 0 & Solution & 60.03 & 241 & 199.00 & 17.43\\
j12031 2.json & 1 & 0 & Solution & 60.04 & 202 & 174.00 & 13.86\\
j12031 3.json & 1 & 0 & Solution & 60.04 & 183 & 158.00 & 13.66\\
j12031 4.json & 1 & 0 & Solution & 60.03 & 235 & 186.00 & 20.85\\
j12031 5.json & 1 & 0 & Solution & 60.03 & 213 & 185.00 & 13.15\\
j12031 6.json & 1 & 0 & Solution & 60.04 & 204 & 181.00 & 11.27\\
j12031 7.json & 1 & 0 & Solution & 60.02 & 218 & 190.00 & 12.84\\
j12031 8.json & 1 & 0 & Solution & 60.03 & 203 & 173.00 & 14.78\\
j12031 9.json & 1 & 0 & Solution & 60.07 & 206 & 175.00 & 15.05\\
j12032 1.json & 1 & 0 & Solution & 60.03 & 150 & 143.00 &  4.67\\
j12032 10.json & 1 & 0 & Solution & 60.03 & 137 & 125.00 &  8.76\\
j12032 2.json & 1 & 0 & Solution & 60.03 & 139 & 122.00 & 12.23\\
j12032 3.json & 1 & 0 & Solution & 60.04 & 152 & 133.00 & 12.50\\
j12032 4.json & 1 & 0 & Solution & 60.04 & 141 & 126.00 & 10.64\\
j12032 5.json & 1 & 0 & Solution & 60.04 & 144 & 132.00 &  8.33\\
j12032 6.json & 1 & 0 & Solution & 60.02 & 134 & 121.00 &  9.70\\
j12032 7.json & 1 & 0 & Solution & 60.03 & 127 & 118.00 &  7.09\\
j12032 8.json & 1 & 0 & Solution & 60.04 & 141 & 131.00 &  7.09\\
j12032 9.json & 1 & 0 & Solution & 60.03 & 132 & 124.00 &  6.06\\
j12033 1.json & 1 & 0 & Solution & 60.03 & 110 & 104.00 &  5.45\\
j12033 10.json & 1 & 0 & Solution & 60.03 & 111 & 102.00 &  8.11\\
j12033 2.json & 1 & 0 & Solution & 60.03 & 117 & 105.00 & 10.26\\
j12033 3.json & 1 & 0 & Solution & 60.03 & 111 & 101.00 &  9.01\\
j12033 4.json & 1 & 0 & Solution & 60.03 & 117 & 105.00 & 10.26\\
j12033 5.json & 1 & 0 & Solution & 60.03 & 148 & 132.00 & 10.81\\
j12033 6.json & 1 & 0 & Solution & 60.05 & 118 & 115.00 &  2.54\\
j12033 7.json & 1 & 0 & Solution & 60.02 & 126 & 121.00 &  3.97\\
j12033 8.json & 1 & 0 & Solution & 60.02 & 114 & 106.00 &  7.02\\
j12033 9.json & 1 & 0 & Solution & 60.03 & 120 & 108.00 & 10.00\\
j12034 1.json & 1 & 0 & Solution & 60.01 & 79 & 75.00 &  5.06\\
j12034 10.json & 1 & 0 & Optimal &  0.18 & 101 & 101.00 &  0.00\\
j12034 2.json & 1 & 0 & Solution & 60.02 & 107 & 102.00 &  4.67\\
j12034 3.json & 1 & 0 & Solution & 60.02 & 103 & 98.00 &  4.85\\
j12034 4.json & 1 & 0 & Optimal &  8.06 & 95 & 95.00 &  0.00\\
j12034 5.json & 1 & 0 & Solution & 60.03 & 105 & 101.00 &  3.81\\
j12034 6.json & 1 & 0 & Optimal &  0.29 & 100 & 100.00 &  0.00\\
j12034 7.json & 1 & 0 & Optimal & 14.52 & 105 & 105.00 &  0.00\\
j12034 8.json & 1 & 0 & Solution & 60.02 & 90 & 85.00 &  5.56\\
j12034 9.json & 1 & 0 & Solution & 60.03 & 97 & 90.00 &  7.22\\
j12035 1.json & 1 & 0 & Optimal &  0.09 & 87 & 87.00 &  0.00\\
j12035 10.json & 1 & 0 & Optimal &  0.10 & 86 & 86.00 &  0.00\\
j12035 2.json & 1 & 0 & Solution & 60.03 & 112 & 111.00 &  0.89\\
j12035 3.json & 1 & 0 & Optimal & 60.01 & 77 & 77.00 &  0.00\\
j12035 4.json & 1 & 0 & Optimal &  0.12 & 101 & 101.00 &  0.00\\
j12035 5.json & 1 & 0 & Optimal & 60.01 & 92 & 92.00 &  0.00\\
j12035 6.json & 1 & 0 & Optimal &  0.07 & 86 & 86.00 &  0.00\\
j12035 7.json & 1 & 0 & Optimal &  0.08 & 99 & 99.00 &  0.00\\
j12035 8.json & 1 & 0 & Optimal &  0.11 & 101 & 101.00 &  0.00\\
j12035 9.json & 1 & 0 & Optimal & 29.79 & 91 & 91.00 &  0.00\\
j12036 1.json & 1 & 0 & Solution & 60.04 & 222 & 186.00 & 16.22\\
j12036 10.json & 1 & 0 & Solution & 60.03 & 231 & 191.00 & 17.32\\
j12036 2.json & 1 & 0 & Solution & 60.11 & 238 & 201.00 & 15.55\\
j12036 3.json & 1 & 0 & Solution & 60.04 & 243 & 216.00 & 11.11\\
j12036 4.json & 1 & 0 & Solution & 60.04 & 257 & 215.00 & 16.34\\
j12036 5.json & 1 & 0 & Solution & 60.09 & 247 & 208.00 & 15.79\\
j12036 6.json & 1 & 0 & Solution & 60.04 & 244 & 201.00 & 17.62\\
j12036 7.json & 1 & 0 & Solution & 60.04 & 220 & 193.00 & 12.27\\
j12036 8.json & 1 & 0 & Solution & 60.03 & 185 & 151.00 & 18.38\\
j12036 9.json & 1 & 0 & Solution & 60.13 & 233 & 200.00 & 14.16\\
j12037 1.json & 1 & 0 & Solution & 60.02 & 151 & 137.00 &  9.27\\
j12037 10.json & 1 & 0 & Solution & 60.03 & 138 & 126.00 &  8.70\\
j12037 2.json & 1 & 0 & Solution & 60.03 & 152 & 132.00 & 13.16\\
j12037 3.json & 1 & 0 & Solution & 60.03 & 146 & 125.00 & 14.38\\
j12037 4.json & 1 & 0 & Solution & 60.05 & 170 & 147.00 & 13.53\\
j12037 5.json & 1 & 0 & Solution & 60.03 & 217 & 191.00 & 11.98\\
j12037 6.json & 1 & 0 & Solution & 60.04 & 172 & 140.00 & 18.60\\
j12037 7.json & 1 & 0 & Solution & 60.03 & 168 & 148.00 & 11.90\\
j12037 8.json & 1 & 0 & Solution & 60.03 & 189 & 165.00 & 12.70\\
j12037 9.json & 1 & 0 & Solution & 60.04 & 151 & 126.00 & 16.56\\
j12038 1.json & 1 & 0 & Solution & 60.03 & 111 & 104.00 &  6.31\\
j12038 10.json & 1 & 0 & Solution & 60.02 & 145 & 114.00 & 21.38\\
j12038 2.json & 1 & 0 & Solution & 60.03 & 131 & 117.00 & 10.69\\
j12038 3.json & 1 & 0 & Solution & 60.03 & 161 & 147.00 &  8.70\\
j12038 4.json & 1 & 0 & Solution & 60.03 & 145 & 122.00 & 15.86\\
j12038 5.json & 1 & 0 & Solution & 60.03 & 117 & 101.00 & 13.68\\
j12038 6.json & 1 & 0 & Solution & 60.02 & 127 & 109.00 & 14.17\\
j12038 7.json & 1 & 0 & Solution & 60.03 & 108 & 97.00 & 10.19\\
j12038 8.json & 1 & 0 & Solution & 60.03 & 129 & 108.00 & 16.28\\
j12038 9.json & 1 & 0 & Solution & 60.03 & 138 & 134.00 &  2.90\\
j12039 1.json & 1 & 0 & Optimal & 60.01 & 95 & 95.00 &  0.00\\
j12039 10.json & 1 & 0 & Solution & 60.02 & 113 & 99.00 & 12.39\\
j12039 2.json & 1 & 0 & Solution & 60.03 & 112 & 104.00 &  7.14\\
j12039 3.json & 1 & 0 & Solution & 60.02 & 114 & 103.00 &  9.65\\
j12039 4.json & 1 & 0 & Solution & 60.02 & 101 & 89.00 & 11.88\\
j12039 5.json & 1 & 0 & Optimal &  0.12 & 106 & 106.00 &  0.00\\
j12039 6.json & 1 & 0 & Optimal &  0.23 & 95 & 95.00 &  0.00\\
j12039 7.json & 1 & 0 & Solution & 60.01 & 108 & 94.00 & 12.96\\
j12039 8.json & 1 & 0 & Solution & 60.04 & 100 & 93.00 &  7.00\\
j12039 9.json & 1 & 0 & Solution & 60.03 & 95 & 87.00 &  8.42\\
j1203 1.json & 1 & 0 & Optimal &  0.25 & 80 & 80.00 &  0.00\\
j1203 10.json & 1 & 0 & Optimal &  0.13 & 103 & 103.00 &  0.00\\
j1203 2.json & 1 & 0 & Optimal &  0.11 & 88 & 88.00 &  0.00\\
j1203 3.json & 1 & 0 & Optimal &  0.11 & 100 & 100.00 &  0.00\\
j1203 4.json & 1 & 0 & Optimal &  0.12 & 71 & 71.00 &  0.00\\
j1203 5.json & 1 & 0 & Optimal &  0.12 & 84 & 84.00 &  0.00\\
j1203 6.json & 1 & 0 & Optimal &  0.12 & 102 & 102.00 &  0.00\\
j1203 7.json & 1 & 0 & Optimal &  0.09 & 93 & 93.00 &  0.00\\
j1203 8.json & 1 & 0 & Optimal &  0.17 & 77 & 77.00 &  0.00\\
j1203 9.json & 1 & 0 & Optimal &  0.12 & 86 & 86.00 &  0.00\\
j12040 1.json & 1 & 0 & Solution & 60.03 & 82 & 78.00 &  4.88\\
j12040 10.json & 1 & 0 & Optimal &  0.10 & 96 & 96.00 &  0.00\\
j12040 2.json & 1 & 0 & Optimal &  0.53 & 90 & 90.00 &  0.00\\
j12040 3.json & 1 & 0 & Optimal & 60.01 & 87 & 87.00 &  0.00\\
j12040 4.json & 1 & 0 & Optimal &  0.07 & 112 & 112.00 &  0.00\\
j12040 5.json & 1 & 0 & Optimal &  0.08 & 101 & 101.00 &  0.00\\
j12040 6.json & 1 & 0 & Optimal &  0.06 & 90 & 90.00 &  0.00\\
j12040 7.json & 1 & 0 & Optimal &  0.07 & 91 & 91.00 &  0.00\\
j12040 8.json & 1 & 0 & Optimal &  0.09 & 97 & 97.00 &  0.00\\
j12040 9.json & 1 & 0 & Optimal &  0.16 & 117 & 117.00 &  0.00\\
j12041 1.json & 1 & 0 & Optimal &  0.34 & 127 & 127.00 &  0.00\\
j12041 10.json & 1 & 0 & Optimal &  0.86 & 136 & 136.00 &  0.00\\
j12041 2.json & 1 & 0 & Optimal & 26.27 & 141 & 141.00 &  0.00\\
j12041 3.json & 1 & 0 & Optimal &  3.91 & 141 & 141.00 &  0.00\\
j12041 4.json & 1 & 0 & Optimal &  1.15 & 116 & 116.00 &  0.00\\
j12041 5.json & 1 & 0 & Optimal &  0.32 & 138 & 138.00 &  0.00\\
j12041 6.json & 1 & 0 & Optimal &  1.61 & 113 & 113.00 &  0.00\\
j12041 7.json & 1 & 0 & Optimal &  4.27 & 109 & 109.00 &  0.00\\
j12041 8.json & 1 & 0 & Optimal &  2.72 & 138 & 138.00 &  0.00\\
j12041 9.json & 1 & 0 & Optimal & 60.01 & 121 & 121.00 &  0.00\\
j12042 1.json & 1 & 0 & Solution & 60.15 & 108 & 105.00 &  2.78\\
j12042 10.json & 1 & 0 & Optimal &  1.19 & 118 & 118.00 &  0.00\\
j12042 2.json & 1 & 0 & Optimal &  0.07 & 126 & 126.00 &  0.00\\
j12042 3.json & 1 & 0 & Optimal &  0.23 & 106 & 106.00 &  0.00\\
j12042 4.json & 1 & 0 & Optimal &  0.18 & 104 & 104.00 &  0.00\\
j12042 5.json & 1 & 0 & Optimal & 22.61 & 120 & 120.00 &  0.00\\
j12042 6.json & 1 & 0 & Optimal & 13.42 & 119 & 119.00 &  0.00\\
j12042 7.json & 1 & 0 & Optimal &  0.18 & 123 & 123.00 &  0.00\\
j12042 8.json & 1 & 0 & Optimal & 60.00 & 113 & 113.00 &  0.00\\
j12042 9.json & 1 & 0 & Optimal &  0.17 & 104 & 104.00 &  0.00\\
j12043 1.json & 1 & 0 & Optimal &  0.13 & 105 & 105.00 &  0.00\\
j12043 10.json & 1 & 0 & Optimal &  0.15 & 113 & 113.00 &  0.00\\
j12043 2.json & 1 & 0 & Optimal &  0.13 & 120 & 120.00 &  0.00\\
j12043 3.json & 1 & 0 & Optimal &  0.15 & 95 & 95.00 &  0.00\\
j12043 4.json & 1 & 0 & Optimal &  0.17 & 105 & 105.00 &  0.00\\
j12043 5.json & 1 & 0 & Optimal &  0.15 & 105 & 105.00 &  0.00\\
j12043 6.json & 1 & 0 & Optimal &  0.92 & 98 & 98.00 &  0.00\\
j12043 7.json & 1 & 0 & Optimal &  0.16 & 122 & 122.00 &  0.00\\
j12043 8.json & 1 & 0 & Optimal &  0.12 & 115 & 115.00 &  0.00\\
j12043 9.json & 1 & 0 & Optimal &  0.14 & 105 & 105.00 &  0.00\\
j12044 1.json & 1 & 0 & Optimal &  0.11 & 100 & 100.00 &  0.00\\
j12044 10.json & 1 & 0 & Optimal &  0.12 & 98 & 98.00 &  0.00\\
j12044 2.json & 1 & 0 & Optimal &  0.12 & 112 & 112.00 &  0.00\\
j12044 3.json & 1 & 0 & Optimal &  0.09 & 107 & 107.00 &  0.00\\
j12044 4.json & 1 & 0 & Optimal &  0.07 & 95 & 95.00 &  0.00\\
j12044 5.json & 1 & 0 & Optimal &  0.12 & 98 & 98.00 &  0.00\\
j12044 6.json & 1 & 0 & Optimal &  0.13 & 106 & 106.00 &  0.00\\
j12044 7.json & 1 & 0 & Optimal &  0.12 & 98 & 98.00 &  0.00\\
j12044 8.json & 1 & 0 & Optimal &  0.09 & 108 & 108.00 &  0.00\\
j12044 9.json & 1 & 0 & Optimal &  0.11 & 91 & 91.00 &  0.00\\
j12045 1.json & 1 & 0 & Optimal &  0.10 & 108 & 108.00 &  0.00\\
j12045 10.json & 1 & 0 & Optimal &  0.12 & 99 & 99.00 &  0.00\\
j12045 2.json & 1 & 0 & Optimal &  0.09 & 91 & 91.00 &  0.00\\
j12045 3.json & 1 & 0 & Optimal &  0.09 & 98 & 98.00 &  0.00\\
j12045 4.json & 1 & 0 & Optimal &  0.10 & 103 & 103.00 &  0.00\\
j12045 5.json & 1 & 0 & Optimal &  0.10 & 116 & 116.00 &  0.00\\
j12045 6.json & 1 & 0 & Optimal &  0.04 & 125 & 125.00 &  0.00\\
j12045 7.json & 1 & 0 & Optimal &  0.09 & 103 & 103.00 &  0.00\\
j12045 8.json & 1 & 0 & Optimal &  0.13 & 103 & 103.00 &  0.00\\
j12045 9.json & 1 & 0 & Optimal &  0.06 & 114 & 114.00 &  0.00\\
j12046 1.json & 1 & 0 & Solution & 60.15 & 193 & 159.00 & 17.62\\
j12046 10.json & 1 & 0 & Solution & 60.05 & 191 & 173.00 &  9.42\\
j12046 2.json & 1 & 0 & Solution & 60.03 & 199 & 174.00 & 12.56\\
j12046 3.json & 1 & 0 & Solution & 60.07 & 183 & 153.00 & 16.39\\
j12046 4.json & 1 & 0 & Solution & 60.05 & 173 & 162.00 &  6.36\\
j12046 5.json & 1 & 0 & Solution & 60.04 & 154 & 135.00 & 12.34\\
j12046 6.json & 1 & 0 & Solution & 60.07 & 181 & 159.00 & 12.15\\
j12046 7.json & 1 & 0 & Solution & 60.08 & 173 & 155.00 & 10.40\\
j12046 8.json & 1 & 0 & Solution & 60.03 & 181 & 157.00 & 13.26\\
j12046 9.json & 1 & 0 & Solution & 60.04 & 170 & 147.00 & 13.53\\
j12047 1.json & 1 & 0 & Solution & 60.06 & 140 & 121.00 & 13.57\\
j12047 10.json & 1 & 0 & Solution & 60.02 & 135 & 127.00 &  5.93\\
j12047 2.json & 1 & 0 & Solution & 60.09 & 133 & 120.00 &  9.77\\
j12047 3.json & 1 & 0 & Solution & 60.02 & 127 & 118.00 &  7.09\\
j12047 4.json & 1 & 0 & Solution & 60.05 & 137 & 117.00 & 14.60\\
j12047 5.json & 1 & 0 & Solution & 60.03 & 129 & 119.00 &  7.75\\
j12047 6.json & 1 & 0 & Solution & 60.03 & 142 & 127.00 & 10.56\\
j12047 7.json & 1 & 0 & Solution & 60.04 & 122 & 112.00 &  8.20\\
j12047 8.json & 1 & 0 & Solution & 60.04 & 138 & 122.00 & 11.59\\
j12047 9.json & 1 & 0 & Solution & 60.05 & 146 & 136.00 &  6.85\\
j12048 1.json & 1 & 0 & Optimal & 60.02 & 100 & 100.00 &  0.00\\
j12048 10.json & 1 & 0 & Solution & 60.06 & 111 & 106.00 &  4.50\\
j12048 2.json & 1 & 0 & Solution & 60.04 & 114 & 111.00 &  2.63\\
j12048 3.json & 1 & 0 & Solution & 60.05 & 114 & 106.00 &  7.02\\
j12048 4.json & 1 & 0 & Solution & 60.03 & 129 & 121.00 &  6.20\\
j12048 5.json & 1 & 0 & Solution & 60.03 & 111 & 104.00 &  6.31\\
j12048 6.json & 1 & 0 & Solution & 60.03 & 106 & 99.00 &  6.60\\
j12048 7.json & 1 & 0 & Solution & 60.05 & 107 & 102.00 &  4.67\\
j12048 8.json & 1 & 0 & Solution & 60.03 & 116 & 109.00 &  6.03\\
j12048 9.json & 1 & 0 & Optimal & 60.02 & 113 & 113.00 &  0.00\\
j12049 1.json & 1 & 0 & Optimal &  0.11 & 96 & 96.00 &  0.00\\
j12049 10.json & 1 & 0 & Solution & 60.09 & 97 & 95.00 &  2.06\\
j12049 2.json & 1 & 0 & Solution & 60.03 & 109 & 104.00 &  4.59\\
j12049 3.json & 1 & 0 & Optimal & 60.02 & 96 & 96.00 &  0.00\\
j12049 4.json & 1 & 0 & Solution & 60.03 & 97 & 94.00 &  3.09\\
j12049 5.json & 1 & 0 & Optimal & 60.01 & 89 & 89.00 &  0.00\\
j12049 6.json & 1 & 0 & Optimal &  0.16 & 128 & 128.00 &  0.00\\
j12049 7.json & 1 & 0 & Optimal &  3.41 & 99 & 99.00 &  0.00\\
j12049 8.json & 1 & 0 & Optimal & 49.19 & 113 & 113.00 &  0.00\\
j12049 9.json & 1 & 0 & Optimal &  7.50 & 97 & 97.00 &  0.00\\
j1204 1.json & 1 & 0 & Optimal &  0.11 & 74 & 74.00 &  0.00\\
j1204 10.json & 1 & 0 & Optimal &  0.08 & 77 & 77.00 &  0.00\\
j1204 2.json & 1 & 0 & Optimal &  0.09 & 107 & 107.00 &  0.00\\
j1204 3.json & 1 & 0 & Optimal &  0.09 & 95 & 95.00 &  0.00\\
j1204 4.json & 1 & 0 & Optimal &  0.10 & 75 & 75.00 &  0.00\\
j1204 5.json & 1 & 0 & Optimal &  0.09 & 74 & 74.00 &  0.00\\
j1204 6.json & 1 & 0 & Optimal &  0.11 & 90 & 90.00 &  0.00\\
j1204 7.json & 1 & 0 & Optimal &  0.10 & 81 & 81.00 &  0.00\\
j1204 8.json & 1 & 0 & Optimal &  0.05 & 90 & 90.00 &  0.00\\
j1204 9.json & 1 & 0 & Optimal &  0.07 & 79 & 79.00 &  0.00\\
j12050 1.json & 1 & 0 & Optimal &  0.07 & 116 & 116.00 &  0.00\\
j12050 10.json & 1 & 0 & Optimal &  0.15 & 103 & 103.00 &  0.00\\
j12050 2.json & 1 & 0 & Optimal &  8.68 & 112 & 112.00 &  0.00\\
j12050 3.json & 1 & 0 & Optimal &  0.11 & 111 & 111.00 &  0.00\\
j12050 4.json & 1 & 0 & Solution & 60.04 & 100 & 98.00 &  2.00\\
j12050 5.json & 1 & 0 & Optimal &  0.12 & 100 & 100.00 &  0.00\\
j12050 6.json & 1 & 0 & Optimal &  0.09 & 102 & 102.00 &  0.00\\
j12050 7.json & 1 & 0 & Optimal &  0.09 & 137 & 137.00 &  0.00\\
j12050 8.json & 1 & 0 & Optimal &  0.11 & 112 & 112.00 &  0.00\\
j12050 9.json & 1 & 0 & Optimal &  0.09 & 101 & 101.00 &  0.00\\
j12051 1.json & 1 & 0 & Solution & 60.05 & 221 & 180.00 & 18.55\\
j12051 10.json & 1 & 0 & Solution & 60.05 & 238 & 193.00 & 18.91\\
j12051 2.json & 1 & 0 & Solution & 60.03 & 233 & 192.00 & 17.60\\
j12051 3.json & 1 & 0 & Solution & 60.07 & 235 & 189.00 & 19.57\\
j12051 4.json & 1 & 0 & Solution & 60.04 & 222 & 196.00 & 11.71\\
j12051 5.json & 1 & 0 & Solution & 60.02 & 244 & 193.00 & 20.90\\
j12051 6.json & 1 & 0 & Solution & 60.07 & 225 & 190.00 & 15.56\\
j12051 7.json & 1 & 0 & Solution & 60.03 & 220 & 179.00 & 18.64\\
j12051 8.json & 1 & 0 & Solution & 60.05 & 216 & 184.00 & 14.81\\
j12051 9.json & 1 & 0 & Solution & 60.03 & 228 & 188.00 & 17.54\\
j12052 1.json & 1 & 0 & Solution & 60.04 & 183 & 157.00 & 14.21\\
j12052 10.json & 1 & 0 & Solution & 60.03 & 150 & 129.00 & 14.00\\
j12052 2.json & 1 & 0 & Solution & 60.04 & 192 & 167.00 & 13.02\\
j12052 3.json & 1 & 0 & Solution & 60.05 & 141 & 124.00 & 12.06\\
j12052 4.json & 1 & 0 & Solution & 60.04 & 178 & 155.00 & 12.92\\
j12052 5.json & 1 & 0 & Solution & 60.04 & 174 & 157.00 &  9.77\\
j12052 6.json & 1 & 0 & Solution & 60.04 & 208 & 181.00 & 12.98\\
j12052 7.json & 1 & 0 & Solution & 60.09 & 155 & 139.00 & 10.32\\
j12052 8.json & 1 & 0 & Solution & 60.02 & 166 & 147.00 & 11.45\\
j12052 9.json & 1 & 0 & Solution & 60.09 & 154 & 141.00 &  8.44\\
j12053 1.json & 1 & 0 & Solution & 60.04 & 149 & 137.00 &  8.05\\
j12053 10.json & 1 & 0 & Solution & 60.03 & 136 & 123.00 &  9.56\\
j12053 2.json & 1 & 0 & Solution & 60.05 & 119 & 108.00 &  9.24\\
j12053 3.json & 1 & 0 & Solution & 60.02 & 114 & 105.00 &  7.89\\
j12053 4.json & 1 & 0 & Solution & 60.03 & 148 & 136.00 &  8.11\\
j12053 5.json & 1 & 0 & Solution & 60.04 & 115 & 108.00 &  6.09\\
j12053 6.json & 1 & 0 & Solution & 60.09 & 109 & 100.00 &  8.26\\
j12053 7.json & 1 & 0 & Solution & 60.02 & 121 & 116.00 &  4.13\\
j12053 8.json & 1 & 0 & Solution & 60.05 & 142 & 134.00 &  5.63\\
j12053 9.json & 1 & 0 & Solution & 60.03 & 168 & 149.00 & 11.31\\
j12054 1.json & 1 & 0 & Solution & 60.09 & 106 & 101.00 &  4.72\\
j12054 10.json & 1 & 0 & Solution & 60.06 & 108 & 104.00 &  3.70\\
j12054 2.json & 1 & 0 & Optimal &  0.13 & 134 & 134.00 &  0.00\\
j12054 3.json & 1 & 0 & Optimal &  1.35 & 111 & 111.00 &  0.00\\
j12054 4.json & 1 & 0 & Solution & 60.06 & 120 & 119.00 &  0.83\\
j12054 5.json & 1 & 0 & Solution & 60.05 & 111 & 106.00 &  4.50\\
j12054 6.json & 1 & 0 & Solution & 60.05 & 111 & 103.00 &  7.21\\
j12054 7.json & 1 & 0 & Solution & 60.04 & 112 & 103.00 &  8.04\\
j12054 8.json & 1 & 0 & Solution & 60.03 & 104 & 99.00 &  4.81\\
j12054 9.json & 1 & 0 & Solution & 60.04 & 108 & 104.00 &  3.70\\
j12055 1.json & 1 & 0 & Solution & 60.03 & 101 & 99.00 &  1.98\\
j12055 10.json & 1 & 0 & Optimal &  0.15 & 100 & 100.00 &  0.00\\
j12055 2.json & 1 & 0 & Optimal &  0.08 & 83 & 83.00 &  0.00\\
j12055 3.json & 1 & 0 & Optimal &  0.10 & 126 & 126.00 &  0.00\\
j12055 4.json & 1 & 0 & Optimal &  0.14 & 90 & 90.00 &  0.00\\
j12055 5.json & 1 & 0 & Optimal &  0.09 & 106 & 106.00 &  0.00\\
j12055 6.json & 1 & 0 & Solution & 60.04 & 102 & 98.00 &  3.92\\
j12055 7.json & 1 & 0 & Optimal &  0.09 & 105 & 105.00 &  0.00\\
j12055 8.json & 1 & 0 & Optimal &  7.76 & 101 & 101.00 &  0.00\\
j12055 9.json & 1 & 0 & Optimal &  0.10 & 94 & 94.00 &  0.00\\
j12056 1.json & 1 & 0 & Solution & 60.04 & 255 & 214.00 & 16.08\\
j12056 10.json & 1 & 0 & Solution & 60.05 & 275 & 228.00 & 17.09\\
j12056 2.json & 1 & 0 & Solution & 60.15 & 221 & 183.00 & 17.19\\
j12056 3.json & 1 & 0 & Solution & 60.04 & 257 & 214.00 & 16.73\\
j12056 4.json & 1 & 0 & Solution & 60.05 & 240 & 199.00 & 17.08\\
j12056 5.json & 1 & 0 & Solution & 60.04 & 298 & 242.00 & 18.79\\
j12056 6.json & 1 & 0 & Solution & 60.03 & 230 & 194.00 & 15.65\\
j12056 7.json & 1 & 0 & Solution & 60.03 & 299 & 242.00 & 19.06\\
j12056 8.json & 1 & 0 & Solution & 60.05 & 304 & 244.00 & 19.74\\
j12056 9.json & 1 & 0 & Solution & 60.03 & 306 & 254.00 & 16.99\\
j12057 1.json & 1 & 0 & Solution & 60.07 & 198 & 169.00 & 14.65\\
j12057 10.json & 1 & 0 & Solution & 60.04 & 173 & 153.00 & 11.56\\
j12057 2.json & 1 & 0 & Solution & 60.04 & 169 & 147.00 & 13.02\\
j12057 3.json & 1 & 0 & Solution & 60.04 & 192 & 173.00 &  9.90\\
j12057 4.json & 1 & 0 & Solution & 60.03 & 208 & 183.00 & 12.02\\
j12057 5.json & 1 & 0 & Solution & 60.03 & 190 & 167.00 & 12.11\\
j12057 6.json & 1 & 0 & Solution & 60.04 & 200 & 171.00 & 14.50\\
j12057 7.json & 1 & 0 & Solution & 60.04 & 174 & 153.00 & 12.07\\
j12057 8.json & 1 & 0 & Solution & 60.03 & 172 & 149.00 & 13.37\\
j12057 9.json & 1 & 0 & Solution & 60.05 & 176 & 154.00 & 12.50\\
j12058 1.json & 1 & 0 & Solution & 60.03 & 146 & 130.00 & 10.96\\
j12058 10.json & 1 & 0 & Solution & 60.08 & 138 & 123.00 & 10.87\\
j12058 2.json & 1 & 0 & Solution & 60.03 & 130 & 113.00 & 13.08\\
j12058 3.json & 1 & 0 & Solution & 60.04 & 125 & 113.00 &  9.60\\
j12058 4.json & 1 & 0 & Solution & 60.03 & 152 & 128.00 & 15.79\\
j12058 5.json & 1 & 0 & Solution & 60.04 & 123 & 107.00 & 13.01\\
j12058 6.json & 1 & 0 & Solution & 60.03 & 146 & 127.00 & 13.01\\
j12058 7.json & 1 & 0 & Solution & 60.04 & 155 & 132.00 & 14.84\\
j12058 8.json & 1 & 0 & Solution & 60.04 & 138 & 124.00 & 10.14\\
j12058 9.json & 1 & 0 & Solution & 60.04 & 135 & 117.00 & 13.33\\
j12059 1.json & 1 & 0 & Solution & 60.03 & 116 & 110.00 &  5.17\\
j12059 10.json & 1 & 0 & Solution & 60.04 & 136 & 124.00 &  8.82\\
j12059 2.json & 1 & 0 & Solution & 60.03 & 108 & 93.00 & 13.89\\
j12059 3.json & 1 & 0 & Optimal & 60.02 & 108 & 108.00 &  0.00\\
j12059 4.json & 1 & 0 & Solution & 60.01 & 112 & 107.00 &  4.46\\
j12059 5.json & 1 & 0 & Solution & 60.03 & 108 & 100.00 &  7.41\\
j12059 6.json & 1 & 0 & Solution & 60.03 & 117 & 107.00 &  8.55\\
j12059 7.json & 1 & 0 & Solution & 60.03 & 115 & 108.00 &  6.09\\
j12059 8.json & 1 & 0 & Solution & 60.01 & 113 & 100.00 & 11.50\\
j12059 9.json & 1 & 0 & Solution & 60.07 & 120 & 115.00 &  4.17\\
j1205 1.json & 1 & 0 & Optimal &  0.04 & 92 & 92.00 &  0.00\\
j1205 10.json & 1 & 0 & Optimal &  0.06 & 92 & 92.00 &  0.00\\
j1205 2.json & 1 & 0 & Optimal &  0.07 & 80 & 80.00 &  0.00\\
j1205 3.json & 1 & 0 & Optimal &  0.09 & 72 & 72.00 &  0.00\\
j1205 4.json & 1 & 0 & Optimal &  0.06 & 97 & 97.00 &  0.00\\
j1205 5.json & 1 & 0 & Optimal &  0.05 & 77 & 77.00 &  0.00\\
j1205 6.json & 1 & 0 & Optimal &  0.11 & 88 & 88.00 &  0.00\\
j1205 7.json & 1 & 0 & Optimal &  0.06 & 84 & 84.00 &  0.00\\
j1205 8.json & 1 & 0 & Optimal &  0.11 & 78 & 78.00 &  0.00\\
j1205 9.json & 1 & 0 & Optimal &  0.12 & 106 & 106.00 &  0.00\\
j12060 1.json & 1 & 0 & Optimal &  0.09 & 101 & 101.00 &  0.00\\
j12060 10.json & 1 & 0 & Solution & 60.03 & 90 & 85.00 &  5.56\\
j12060 2.json & 1 & 0 & Solution & 60.03 & 84 & 81.00 &  3.57\\
j12060 3.json & 1 & 0 & Solution & 60.02 & 91 & 81.00 & 10.99\\
j12060 4.json & 1 & 0 & Solution & 60.02 & 105 & 101.00 &  3.81\\
j12060 5.json & 1 & 0 & Solution & 60.02 & 106 & 96.00 &  9.43\\
j12060 6.json & 1 & 0 & Optimal &  0.08 & 110 & 110.00 &  0.00\\
j12060 7.json & 1 & 0 & Solution & 60.04 & 97 & 88.00 &  9.28\\
j12060 8.json & 1 & 0 & Solution & 60.04 & 102 & 101.00 &  0.98\\
j12060 9.json & 1 & 0 & Optimal &  0.08 & 101 & 101.00 &  0.00\\
j1206 1.json & 1 & 0 & Solution & 60.03 & 152 & 133.00 & 12.50\\
j1206 10.json & 1 & 0 & Solution & 60.04 & 178 & 156.00 & 12.36\\
j1206 2.json & 1 & 0 & Solution & 60.05 & 141 & 125.00 & 11.35\\
j1206 3.json & 1 & 0 & Solution & 60.05 & 135 & 125.00 &  7.41\\
j1206 4.json & 1 & 0 & Solution & 60.14 & 156 & 144.00 &  7.69\\
j1206 5.json & 1 & 0 & Solution & 60.04 & 129 & 116.00 & 10.08\\
j1206 6.json & 1 & 0 & Solution & 60.04 & 158 & 140.00 & 11.39\\
j1206 7.json & 1 & 0 & Solution & 60.03 & 172 & 151.00 & 12.21\\
j1206 8.json & 1 & 0 & Solution & 60.05 & 150 & 140.00 &  6.67\\
j1206 9.json & 1 & 0 & Solution & 60.04 & 164 & 146.00 & 10.98\\
j1207 1.json & 1 & 0 & Solution & 60.03 & 105 & 98.00 &  6.67\\
j1207 10.json & 1 & 0 & Solution & 60.02 & 122 & 111.00 &  9.02\\
j1207 2.json & 1 & 0 & Solution & 60.03 & 114 & 106.00 &  7.02\\
j1207 3.json & 1 & 0 & Solution & 60.04 & 100 & 94.00 &  6.00\\
j1207 4.json & 1 & 0 & Solution & 60.04 & 116 & 105.00 &  9.48\\
j1207 5.json & 1 & 0 & Solution & 60.06 & 137 & 125.00 &  8.76\\
j1207 6.json & 1 & 0 & Solution & 60.03 & 126 & 115.00 &  8.73\\
j1207 7.json & 1 & 0 & Solution & 60.06 & 120 & 113.00 &  5.83\\
j1207 8.json & 1 & 0 & Solution & 60.03 & 100 & 92.00 &  8.00\\
j1207 9.json & 1 & 0 & Solution & 60.04 & 92 & 86.00 &  6.52\\
j1208 1.json & 1 & 0 & Optimal &  0.72 & 95 & 95.00 &  0.00\\
j1208 10.json & 1 & 0 & Solution & 60.02 & 94 & 91.00 &  3.19\\
j1208 2.json & 1 & 0 & Solution & 60.04 & 104 & 97.00 &  6.73\\
j1208 3.json & 1 & 0 & Solution & 60.05 & 95 & 93.00 &  2.11\\
j1208 4.json & 1 & 0 & Solution & 60.01 & 95 & 89.00 &  6.32\\
j1208 5.json & 1 & 0 & Solution & 60.05 & 107 & 99.00 &  7.48\\
j1208 6.json & 1 & 0 & Solution & 60.03 & 85 & 83.00 &  2.35\\
j1208 7.json & 1 & 0 & Solution & 60.02 & 88 & 87.00 &  1.14\\
j1208 8.json & 1 & 0 & Solution & 60.03 & 89 & 87.00 &  2.25\\
j1208 9.json & 1 & 0 & Solution & 60.02 & 96 & 88.00 &  8.33\\
j1209 1.json & 1 & 0 & Optimal &  0.12 & 88 & 88.00 &  0.00\\
j1209 10.json & 1 & 0 & Optimal &  0.13 & 84 & 84.00 &  0.00\\
j1209 2.json & 1 & 0 & Optimal &  0.10 & 94 & 94.00 &  0.00\\
j1209 3.json & 1 & 0 & Optimal &  0.18 & 87 & 87.00 &  0.00\\
j1209 4.json & 1 & 0 & Solution & 60.02 & 87 & 84.00 &  3.45\\
j1209 5.json & 1 & 0 & Optimal &  0.07 & 114 & 114.00 &  0.00\\
j1209 6.json & 1 & 0 & Optimal & 60.01 & 98 & 98.00 &  0.00\\
j1209 7.json & 1 & 0 & Optimal &  0.09 & 80 & 80.00 &  0.00\\
j1209 8.json & 1 & 0 & Optimal &  0.07 & 80 & 80.00 &  0.00\\
j1209 9.json & 1 & 0 & Optimal &  0.27 & 87 & 87.00 &  0.00\\
\end{longtable}



\clearpage
\chapter{Result Comparison for SALBP}

\begin{table}[htbp]
\caption{SALBP Results Summary Size 20 (525 Instances)}
{\tiny
\begin{tabular}{l*{11}{r}}\toprule
Type & base & Laborie & CPO & CPSat & Cplex & Chuffed & MCPSat & CPOA & CPSatA & ChuffedA & CplexA \\ \midrule
UniqueProvenOptimal& - & - & - & - & - & - & - & - & - & - & - \\
SharedProvenOptimal& 521 & - & 522 & 525 & 525 & 525 & 525 & 521 & 518 & 190 & 283 \\
UniqueUnprovenOptimal& - & - & - & - & - & - & - & - & - & - & - \\
SharedUnprovenOptimal& 4 & - & 3 & - & - & - & - & 4 & 7 & 6 & 242 \\
Optimal& 525 & - & 525 & 525 & 525 & 525 & 525 & 525 & 525 & 196 & 525 \\
UniqueBest& - & - & - & - & - & - & - & - & - & - & - \\
SharedBest& - & - & - & - & - & - & - & - & - & - & - \\
Best& - & - & - & - & - & - & - & - & - & - & - \\
BestOrOptimal& 525 & - & 525 & 525 & 525 & 525 & 525 & 525 & 525 & 196 & 525 \\
Gap1& - & - & - & - & - & - & - & - & - & 14 & - \\
Gap2& - & - & - & - & - & - & - & - & - & 5 & - \\
Gap3& - & - & - & - & - & - & - & - & - & 12 & - \\
Gap4Plus& - & - & - & - & - & - & - & - & - & 287 & - \\
NonBest& - & - & - & - & - & - & - & - & - & 318 & - \\
Solved& 525 & - & 525 & 525 & 525 & 525 & 525 & 525 & 525 & 514 & 525 \\
Unknown& - & - & - & - & - & - & - & - & - & 11 & - \\
Infeasible& - & - & - & - & - & - & - & - & - & - & - \\
NotPresent& - & 525 & - & - & - & - & - & - & - & - & - \\
N/A& - & 525 & - & - & - & - & - & - & - & 11 & - \\
Total& 525 & 525 & 525 & 525 & 525 & 525 & 525 & 525 & 525 & 525 & 525 \\
\bottomrule
\end{tabular}

}

\end{table}

\begin{table}[htbp]
\caption{SALBP Results Summary Size 20 (525 Instances)}
{\tiny
\begin{tabular}{l*{11}{r}}\toprule
Type & base & Laborie & CPO & CPSat & Cplex & Chuffed & MCPSat & CPOA & CPSatA & ChuffedA & CplexA \\ \midrule
UniqueProvenOptimal& - & - & - & - & - & - & - & - & - & - & - \\
SharedProvenOptimal& 99.24 & - & 99.43 & 100.00 & 100.00 & 100.00 & 100.00 & 99.24 & 98.67 & 36.19 & 53.90 \\
UniqueUnprovenOptimal& - & - & - & - & - & - & - & - & - & - & - \\
SharedUnprovenOptimal&  0.76 & - &  0.57 & - & - & - & - &  0.76 &  1.33 &  1.14 & 46.10 \\
Optimal& 100.00 & - & 100.00 & 100.00 & 100.00 & 100.00 & 100.00 & 100.00 & 100.00 & 37.33 & 100.00 \\
UniqueBest& - & - & - & - & - & - & - & - & - & - & - \\
SharedBest& - & - & - & - & - & - & - & - & - & - & - \\
Best& - & - & - & - & - & - & - & - & - & - & - \\
BestOrOptimal& 100.00 & - & 100.00 & 100.00 & 100.00 & 100.00 & 100.00 & 100.00 & 100.00 & 37.33 & 100.00 \\
Gap1& - & - & - & - & - & - & - & - & - &  2.67 & - \\
Gap2& - & - & - & - & - & - & - & - & - &  0.95 & - \\
Gap3& - & - & - & - & - & - & - & - & - &  2.29 & - \\
Gap4Plus& - & - & - & - & - & - & - & - & - & 54.67 & - \\
NonBest& - & - & - & - & - & - & - & - & - & 60.57 & - \\
Solved& 100.00 & - & 100.00 & 100.00 & 100.00 & 100.00 & 100.00 & 100.00 & 100.00 & 97.90 & 100.00 \\
Unknown& - & - & - & - & - & - & - & - & - &  2.10 & - \\
Infeasible& - & - & - & - & - & - & - & - & - & - & - \\
NotPresent& - & 100.00 & - & - & - & - & - & - & - & - & - \\
N/A& - & 100.00 & - & - & - & - & - & - & - &  2.10 & - \\
Total& 100.00 & 100.00 & 100.00 & 100.00 & 100.00 & 100.00 & 100.00 & 100.00 & 100.00 & 100.00 & 100.00 \\
\bottomrule
\end{tabular}

}

\end{table}

\begin{table}[htbp]
\caption{SALBP Results Summary Size 50 (525 Instances)}
{\tiny
\begin{tabular}{l*{11}{r}}\toprule
Type & base & Laborie & CPO & CPSat & Cplex & Chuffed & MCPSat & CPOA & CPSatA & ChuffedA & CplexA \\ \midrule
UniqueProvenOptimal& - & - & - & - & - & - & - & - & - & - & - \\
SharedProvenOptimal& 426 & - & 382 & 501 & 61 & 292 & 501 & 352 & 218 & - & 6 \\
UniqueUnprovenOptimal& - & - & - & - & - & - & - & 1 & - & - & - \\
SharedUnprovenOptimal& 44 & - & 113 & 3 & 232 & 171 & - & 152 & 258 & - & 105 \\
Optimal& 470 & - & 495 & 504 & 293 & 463 & 501 & 505 & 476 & - & 111 \\
UniqueBest& - & - & - & - & - & - & - & - & - & - & - \\
SharedBest& 12 & - & 16 & 16 & - & 12 & - & 16 & 16 & - & - \\
Best& 12 & - & 16 & 16 & - & 12 & - & 16 & 16 & - & - \\
BestOrOptimal& 482 & - & 511 & 520 & 293 & 475 & 501 & 521 & 492 & - & 111 \\
Gap1& 39 & - & 14 & 5 & 97 & 31 & - & 4 & 31 & - & 34 \\
Gap2& 4 & - & - & - & 39 & 5 & - & - & 2 & - & 22 \\
Gap3& - & - & - & - & 34 & 1 & - & - & - & - & 24 \\
Gap4Plus& - & - & - & - & 62 & 13 & - & - & - & - & 61 \\
NonBest& 43 & - & 14 & 5 & 232 & 50 & - & 4 & 33 & - & 141 \\
Solved& 525 & - & 525 & 525 & 525 & 525 & 501 & 525 & 525 & - & 252 \\
Unknown& - & - & - & - & - & - & 24 & - & - & - & 273 \\
Infeasible& - & - & - & - & - & - & - & - & - & - & - \\
NotPresent& - & 525 & - & - & - & - & - & - & - & 525 & - \\
N/A& - & 525 & - & - & - & - & 24 & - & - & 525 & 273 \\
Total& 525 & 525 & 525 & 525 & 525 & 525 & 525 & 525 & 525 & 525 & 525 \\
\bottomrule
\end{tabular}

}

\end{table}

\begin{table}[htbp]
\caption{SALBP Results Summary Size 50 (525 Instances)}
{\tiny
\begin{tabular}{l*{11}{r}}\toprule
Type & base & Laborie & CPO & CPSat & Cplex & Chuffed & MCPSat & CPOA & CPSatA & ChuffedA & CplexA \\ \midrule
UniqueProvenOptimal& - & - & - & - & - & - & - & - & - & - & - \\
SharedProvenOptimal& 81.14 & - & 72.76 & 95.43 & 11.62 & 55.62 & 95.43 & 67.05 & 41.52 & - &  1.14 \\
UniqueUnprovenOptimal& - & - & - & - & - & - & - &  0.19 & - & - & - \\
SharedUnprovenOptimal&  8.38 & - & 21.52 &  0.57 & 44.19 & 32.57 & - & 28.95 & 49.14 & - & 20.00 \\
Optimal& 89.52 & - & 94.29 & 96.00 & 55.81 & 88.19 & 95.43 & 96.19 & 90.67 & - & 21.14 \\
UniqueBest& - & - & - & - & - & - & - & - & - & - & - \\
SharedBest&  2.29 & - &  3.05 &  3.05 & - &  2.29 & - &  3.05 &  3.05 & - & - \\
Best&  2.29 & - &  3.05 &  3.05 & - &  2.29 & - &  3.05 &  3.05 & - & - \\
BestOrOptimal& 91.81 & - & 97.33 & 99.05 & 55.81 & 90.48 & 95.43 & 99.24 & 93.71 & - & 21.14 \\
Gap1&  7.43 & - &  2.67 &  0.95 & 18.48 &  5.90 & - &  0.76 &  5.90 & - &  6.48 \\
Gap2&  0.76 & - & - & - &  7.43 &  0.95 & - & - &  0.38 & - &  4.19 \\
Gap3& - & - & - & - &  6.48 &  0.19 & - & - & - & - &  4.57 \\
Gap4Plus& - & - & - & - & 11.81 &  2.48 & - & - & - & - & 11.62 \\
NonBest&  8.19 & - &  2.67 &  0.95 & 44.19 &  9.52 & - &  0.76 &  6.29 & - & 26.86 \\
Solved& 100.00 & - & 100.00 & 100.00 & 100.00 & 100.00 & 95.43 & 100.00 & 100.00 & - & 48.00 \\
Unknown& - & - & - & - & - & - &  4.57 & - & - & - & 52.00 \\
Infeasible& - & - & - & - & - & - & - & - & - & - & - \\
NotPresent& - & 100.00 & - & - & - & - & - & - & - & 100.00 & - \\
N/A& - & 100.00 & - & - & - & - &  4.57 & - & - & 100.00 & 52.00 \\
Total& 100.00 & 100.00 & 100.00 & 100.00 & 100.00 & 100.00 & 100.00 & 100.00 & 100.00 & 100.00 & 100.00 \\
\bottomrule
\end{tabular}

}

\end{table}

\begin{table}[htbp]
\caption{SALBP Results Summary Size 100 (525 Instances)}
{\tiny
\begin{tabular}{l*{11}{r}}\toprule
Type & base & Laborie & CPO & CPSat & Cplex & Chuffed & MCPSat & CPOA & CPSatA & ChuffedA & CplexA \\ \midrule
UniqueProvenOptimal& 11 & - & 1 & 1 & - & - & - & - & - & - & - \\
SharedProvenOptimal& 344 & - & 318 & 373 & - & 75 & 378 & 166 & 30 & - & - \\
UniqueUnprovenOptimal& - & - & - & - & - & - & - & 10 & - & - & - \\
SharedUnprovenOptimal& 7 & 25 & 37 & 1 & 11 & 52 & - & 220 & 298 & - & - \\
Optimal& 362 & 25 & 356 & 375 & 11 & 127 & 378 & 396 & 328 & - & - \\
UniqueBest& 1 & 2 & - & 8 & - & - & - & 8 & - & - & - \\
SharedBest& 8 & 51 & 50 & 82 & - & 2 & - & 87 & 16 & - & - \\
Best& 9 & 53 & 50 & 90 & - & 2 & - & 95 & 16 & - & - \\
BestOrOptimal& 371 & 78 & 406 & 465 & 11 & 129 & 378 & 491 & 344 & - & - \\
Gap1& 59 & 37 & 117 & 59 & 25 & 33 & - & 33 & 103 & - & - \\
Gap2& 50 & - & 2 & 1 & 19 & 11 & - & 1 & 58 & - & - \\
Gap3& 30 & - & - & - & 7 & 13 & - & - & 18 & - & - \\
Gap4Plus& 15 & - & - & - & 250 & 339 & - & - & 2 & - & - \\
NonBest& 154 & 37 & 119 & 60 & 301 & 396 & - & 34 & 181 & - & - \\
Solved& 525 & 115 & 525 & 525 & 312 & 525 & 378 & 525 & 525 & - & - \\
Unknown& - & - & - & - & 213 & - & 147 & - & - & - & - \\
Infeasible& - & - & - & - & - & - & - & - & - & - & - \\
NotPresent& - & 410 & - & - & - & - & - & - & - & 525 & 525 \\
N/A& - & 410 & - & - & 213 & - & 147 & - & - & 525 & 525 \\
Total& 525 & 525 & 525 & 525 & 525 & 525 & 525 & 525 & 525 & 525 & 525 \\
\bottomrule
\end{tabular}

}

\end{table}

\begin{table}[htbp]
\caption{SALBP Results Summary Size 100 (525 Instances)}
{\tiny
\begin{tabular}{l*{11}{r}}\toprule
Type & base & Laborie & CPO & CPSat & Cplex & Chuffed & MCPSat & CPOA & CPSatA & ChuffedA & CplexA \\ \midrule
UniqueProvenOptimal&  2.10 & - &  0.19 &  0.19 & - & - & - & - & - & - & - \\
SharedProvenOptimal& 65.52 & - & 60.57 & 71.05 & - & 14.29 & 72.00 & 31.62 &  5.71 & - & - \\
UniqueUnprovenOptimal& - & - & - & - & - & - & - &  1.90 & - & - & - \\
SharedUnprovenOptimal&  1.33 &  4.76 &  7.05 &  0.19 &  2.10 &  9.90 & - & 41.90 & 56.76 & - & - \\
Optimal& 68.95 &  4.76 & 67.81 & 71.43 &  2.10 & 24.19 & 72.00 & 75.43 & 62.48 & - & - \\
UniqueBest&  0.19 &  0.38 & - &  1.52 & - & - & - &  1.52 & - & - & - \\
SharedBest&  1.52 &  9.71 &  9.52 & 15.62 & - &  0.38 & - & 16.57 &  3.05 & - & - \\
Best&  1.71 & 10.10 &  9.52 & 17.14 & - &  0.38 & - & 18.10 &  3.05 & - & - \\
BestOrOptimal& 70.67 & 14.86 & 77.33 & 88.57 &  2.10 & 24.57 & 72.00 & 93.52 & 65.52 & - & - \\
Gap1& 11.24 &  7.05 & 22.29 & 11.24 &  4.76 &  6.29 & - &  6.29 & 19.62 & - & - \\
Gap2&  9.52 & - &  0.38 &  0.19 &  3.62 &  2.10 & - &  0.19 & 11.05 & - & - \\
Gap3&  5.71 & - & - & - &  1.33 &  2.48 & - & - &  3.43 & - & - \\
Gap4Plus&  2.86 & - & - & - & 47.62 & 64.57 & - & - &  0.38 & - & - \\
NonBest& 29.33 &  7.05 & 22.67 & 11.43 & 57.33 & 75.43 & - &  6.48 & 34.48 & - & - \\
Solved& 100.00 & 21.90 & 100.00 & 100.00 & 59.43 & 100.00 & 72.00 & 100.00 & 100.00 & - & - \\
Unknown& - & - & - & - & 40.57 & - & 28.00 & - & - & - & - \\
Infeasible& - & - & - & - & - & - & - & - & - & - & - \\
NotPresent& - & 78.10 & - & - & - & - & - & - & - & 100.00 & 100.00 \\
N/A& - & 78.10 & - & - & 40.57 & - & 28.00 & - & - & 100.00 & 100.00 \\
Total& 100.00 & 100.00 & 100.00 & 100.00 & 100.00 & 100.00 & 100.00 & 100.00 & 100.00 & 100.00 & 100.00 \\
\bottomrule
\end{tabular}

}

\end{table}

\begin{table}[htbp]
\caption{SALBP Results Summary Size 1000 (525 Instances)}
{\tiny
\begin{tabular}{l*{11}{r}}\toprule
Type & base & Laborie & CPO & CPSat & Cplex & Chuffed & MCPSat & CPOA & CPSatA & ChuffedA & CplexA \\ \midrule
UniqueProvenOptimal& 142 & - & - & - & - & - & - & - & - & - & - \\
SharedProvenOptimal& 44 & - & - & - & - & - & - & - & - & - & - \\
UniqueUnprovenOptimal& - & - & - & - & - & - & - & - & - & - & - \\
SharedUnprovenOptimal& - & - & - & - & - & - & - & 44 & - & - & - \\
Optimal& 186 & - & - & - & - & - & - & 44 & - & - & - \\
UniqueBest& 86 & 82 & 1 & 1 & - & - & - & 52 & - & - & - \\
SharedBest& 61 & 55 & 3 & 5 & - & - & - & 116 & 9 & - & - \\
Best& 147 & 137 & 4 & 6 & - & - & - & 168 & 9 & - & - \\
BestOrOptimal& 333 & 137 & 4 & 6 & - & - & - & 212 & 9 & - & - \\
Gap1& 59 & 18 & 83 & 98 & - & - & - & 181 & 172 & - & - \\
Gap2& 30 & 16 & 165 & 182 & - & - & - & 45 & 128 & - & - \\
Gap3& 14 & 3 & 80 & 49 & - & - & - & 21 & 34 & - & - \\
Gap4Plus& 89 & 6 & 193 & 190 & - & 112 & - & 65 & 182 & - & - \\
NonBest& 192 & 43 & 521 & 519 & - & 112 & - & 312 & 516 & - & - \\
Solved& 525 & 180 & 525 & 525 & - & 112 & - & 524 & 525 & - & - \\
Unknown& - & - & - & - & - & 413 & 525 & 1 & - & - & - \\
Infeasible& - & - & - & - & - & - & - & - & - & - & - \\
NotPresent& - & 345 & - & - & 525 & - & - & - & - & 525 & 525 \\
N/A& - & 345 & - & - & 525 & 413 & 525 & 1 & - & 525 & 525 \\
Total& 525 & 525 & 525 & 525 & 525 & 525 & 525 & 525 & 525 & 525 & 525 \\
\bottomrule
\end{tabular}

}

\end{table}

\begin{table}[htbp]
\caption{SALBP Results Summary Size 1000 (525 Instances)}
{\tiny
\begin{tabular}{l*{11}{r}}\toprule
Type & base & Laborie & CPO & CPSat & Cplex & Chuffed & MCPSat & CPOA & CPSatA & ChuffedA & CplexA \\ \midrule
UniqueProvenOptimal& 27.05 & - & - & - & - & - & - & - & - & - & - \\
SharedProvenOptimal&  8.38 & - & - & - & - & - & - & - & - & - & - \\
UniqueUnprovenOptimal& - & - & - & - & - & - & - & - & - & - & - \\
SharedUnprovenOptimal& - & - & - & - & - & - & - &  8.38 & - & - & - \\
Optimal& 35.43 & - & - & - & - & - & - &  8.38 & - & - & - \\
UniqueBest& 16.38 & 15.62 &  0.19 &  0.19 & - & - & - &  9.90 & - & - & - \\
SharedBest& 11.62 & 10.48 &  0.57 &  0.95 & - & - & - & 22.10 &  1.71 & - & - \\
Best& 28.00 & 26.10 &  0.76 &  1.14 & - & - & - & 32.00 &  1.71 & - & - \\
BestOrOptimal& 63.43 & 26.10 &  0.76 &  1.14 & - & - & - & 40.38 &  1.71 & - & - \\
Gap1& 11.24 &  3.43 & 15.81 & 18.67 & - & - & - & 34.48 & 32.76 & - & - \\
Gap2&  5.71 &  3.05 & 31.43 & 34.67 & - & - & - &  8.57 & 24.38 & - & - \\
Gap3&  2.67 &  0.57 & 15.24 &  9.33 & - & - & - &  4.00 &  6.48 & - & - \\
Gap4Plus& 16.95 &  1.14 & 36.76 & 36.19 & - & 21.33 & - & 12.38 & 34.67 & - & - \\
NonBest& 36.57 &  8.19 & 99.24 & 98.86 & - & 21.33 & - & 59.43 & 98.29 & - & - \\
Solved& 100.00 & 34.29 & 100.00 & 100.00 & - & 21.33 & - & 99.81 & 100.00 & - & - \\
Unknown& - & - & - & - & - & 78.67 & 100.00 &  0.19 & - & - & - \\
Infeasible& - & - & - & - & - & - & - & - & - & - & - \\
NotPresent& - & 65.71 & - & - & 100.00 & - & - & - & - & 100.00 & 100.00 \\
N/A& - & 65.71 & - & - & 100.00 & 78.67 & 100.00 &  0.19 & - & 100.00 & 100.00 \\
Total& 100.00 & 100.00 & 100.00 & 100.00 & 100.00 & 100.00 & 100.00 & 100.00 & 100.00 & 100.00 & 100.00 \\
\bottomrule
\end{tabular}

}

\end{table}


\clearpage

\begin{longtable}{lrrrrrrrr}
\caption{Result Comparisoon for SALBP}\\\toprule
& \multicolumn{2}{c}{SALOME} & \multicolumn{4}{c}{Direct} & \multicolumn{2}{c}{Alternative}\\Instance & LB & UB & CPO & CPSat & Cplex & Chuffed & CPO & CPSat \\\midrule
\endhead
\bottomrule
\endfoot
1000 1& 135 & \cellcolor{blue!40} \textbf{135} & 136 & 136 & \cellcolor{red!20} n/a & \cellcolor{red!20} n/a & 136 & 136 \\
1000 10& 140 & \cellcolor{blue!40} \textbf{140} & 141 & 141 & \cellcolor{red!20} n/a & \cellcolor{red!20} n/a & 141 & 141 \\
1000 100& 137 & \cellcolor{blue!40} \textbf{137} & 139 & 139 & \cellcolor{red!20} n/a & \cellcolor{red!20} n/a & 138 & \cellcolor{red!20} n/a \\
1000 101& 515 & \cellcolor{blue!40} 550 & 558 & 565 & \cellcolor{red!20} n/a & \cellcolor{red!20} n/a & 552 & \cellcolor{red!20} n/a \\
1000 102& 522 & \cellcolor{blue!20} 551 & 557 & 562 & \cellcolor{red!20} n/a & \cellcolor{red!20} n/a & \cellcolor{blue!20} 551 & \cellcolor{red!20} n/a \\
1000 103& 522 & \cellcolor{blue!40} 556 & 562 & 567 & \cellcolor{red!20} n/a & \cellcolor{red!20} n/a & 557 & \cellcolor{red!20} n/a \\
1000 104& 515 & 555 & 555 & 556 & \cellcolor{red!20} n/a & \cellcolor{red!20} n/a & \cellcolor{blue!40} 544 & \cellcolor{red!20} n/a \\
1000 105& 510 & 541 & 549 & 552 & \cellcolor{red!20} n/a & \cellcolor{red!20} n/a & \cellcolor{blue!40} 540 & \cellcolor{red!20} n/a \\
1000 106& 526 & 557 & 556 & 557 & \cellcolor{red!20} n/a & \cellcolor{red!20} n/a & \cellcolor{blue!40} 546 & \cellcolor{red!20} n/a \\
1000 107& 501 & 539 & 540 & 543 & \cellcolor{red!20} n/a & \cellcolor{red!20} n/a & \cellcolor{blue!40} 533 & \cellcolor{red!20} n/a \\
1000 108& 508 & 553 & 546 & 552 & \cellcolor{red!20} n/a & \cellcolor{red!20} n/a & \cellcolor{blue!40} 540 & \cellcolor{red!20} n/a \\
1000 109& 513 & 556 & 550 & 550 & \cellcolor{red!20} n/a & \cellcolor{red!20} n/a & \cellcolor{blue!40} 546 & \cellcolor{red!20} n/a \\
1000 11& 134 & \cellcolor{blue!40} \textbf{134} & 135 & 136 & \cellcolor{red!20} n/a & \cellcolor{red!20} n/a & 135 & 135 \\
1000 110& 516 & 555 & 557 & 566 & \cellcolor{red!20} n/a & \cellcolor{red!20} n/a & \cellcolor{blue!40} 553 & \cellcolor{red!20} n/a \\
1000 111& 509 & 541 & 550 & 552 & \cellcolor{red!20} n/a & \cellcolor{red!20} n/a & \cellcolor{blue!40} 539 & \cellcolor{red!20} n/a \\
1000 112& 509 & 550 & 550 & 548 & \cellcolor{red!20} n/a & \cellcolor{red!20} n/a & \cellcolor{blue!40} 545 & \cellcolor{red!20} n/a \\
1000 113& 496 & 549 & 543 & 546 & \cellcolor{red!20} n/a & \cellcolor{red!20} n/a & \cellcolor{blue!40} 539 & \cellcolor{red!20} n/a \\
1000 114& 512 & 553 & 552 & 555 & \cellcolor{red!20} n/a & \cellcolor{red!20} n/a & \cellcolor{blue!40} 544 & \cellcolor{red!20} n/a \\
1000 115& 503 & 550 & 542 & 545 & \cellcolor{red!20} n/a & \cellcolor{red!20} n/a & \cellcolor{blue!40} 537 & \cellcolor{red!20} n/a \\
1000 116& 499 & 542 & 546 & 554 & \cellcolor{red!20} n/a & \cellcolor{red!20} n/a & \cellcolor{blue!40} 536 & \cellcolor{red!20} n/a \\
1000 117& 512 & 550 & 552 & 556 & \cellcolor{red!20} n/a & \cellcolor{red!20} n/a & \cellcolor{blue!40} 546 & \cellcolor{red!20} n/a \\
1000 118& 524 & 570 & 566 & 566 & \cellcolor{red!20} n/a & \cellcolor{red!20} n/a & \cellcolor{blue!40} 557 & \cellcolor{red!20} n/a \\
1000 119& 501 & 538 & 532 & 536 & \cellcolor{red!20} n/a & \cellcolor{red!20} n/a & \cellcolor{blue!40} 527 & \cellcolor{red!20} n/a \\
1000 12& 134 & \cellcolor{blue!20} \textbf{134} & 135 & 135 & \cellcolor{red!20} n/a & \cellcolor{red!20} n/a & \cellcolor{blue!20} 134 & 135 \\
1000 120& 518 & 561 & 549 & 556 & \cellcolor{red!20} n/a & \cellcolor{red!20} n/a & \cellcolor{blue!40} 546 & \cellcolor{red!20} n/a \\
1000 121& 496 & \cellcolor{blue!40} 536 & 542 & 546 & \cellcolor{red!20} n/a & \cellcolor{red!20} n/a & 537 & \cellcolor{red!20} n/a \\
1000 122& 493 & 528 & 535 & 537 & \cellcolor{red!20} n/a & \cellcolor{red!20} n/a & \cellcolor{blue!40} 527 & \cellcolor{red!20} n/a \\
1000 123& 513 & \cellcolor{blue!40} 548 & 557 & 562 & \cellcolor{red!20} n/a & \cellcolor{red!20} n/a & 549 & \cellcolor{red!20} n/a \\
1000 124& 504 & 541 & 544 & 548 & \cellcolor{red!20} n/a & \cellcolor{red!20} n/a & \cellcolor{blue!40} 535 & \cellcolor{red!20} n/a \\
1000 125& 507 & 546 & 546 & 550 & \cellcolor{red!20} n/a & \cellcolor{red!20} n/a & \cellcolor{blue!40} 538 & \cellcolor{red!20} n/a \\
1000 126& 228 & \cellcolor{blue!20} 231 & 232 & 233 & \cellcolor{red!20} n/a & \cellcolor{red!20} n/a & \cellcolor{blue!20} 231 & \cellcolor{red!20} n/a \\
1000 127& 221 & \cellcolor{blue!20} 223 & 224 & 224 & \cellcolor{red!20} n/a & \cellcolor{red!20} n/a & \cellcolor{blue!20} 223 & \cellcolor{red!20} n/a \\
1000 128& 222 & \cellcolor{blue!10} 225 & \cellcolor{blue!10} 225 & 226 & \cellcolor{red!20} n/a & \cellcolor{red!20} n/a & \cellcolor{blue!10} 225 & \cellcolor{red!20} n/a \\
1000 129& 223 & \cellcolor{blue!40} \textbf{223} & 226 & 227 & \cellcolor{red!20} n/a & \cellcolor{red!20} n/a & 225 & 225 \\
1000 13& 131 & \cellcolor{blue!40} \textbf{131} & 132 & 133 & \cellcolor{red!20} n/a & \cellcolor{red!20} n/a & 132 & 132 \\
1000 130& 221 & \cellcolor{blue!40} 223 & 225 & 225 & \cellcolor{red!20} n/a & \cellcolor{red!20} n/a & 224 & \cellcolor{red!20} n/a \\
1000 131& 220 & \cellcolor{blue!40} \textbf{220} & 223 & 224 & \cellcolor{red!20} n/a & \cellcolor{red!20} n/a & 222 & \cellcolor{red!20} n/a \\
1000 132& 214 & \cellcolor{blue!20} 217 & 218 & 218 & \cellcolor{red!20} n/a & \cellcolor{red!20} n/a & \cellcolor{blue!20} 217 & \cellcolor{red!20} n/a \\
1000 133& 226 & \cellcolor{blue!40} 227 & 230 & 230 & \cellcolor{red!20} n/a & \cellcolor{red!20} n/a & 229 & \cellcolor{red!20} n/a \\
1000 134& 215 & \cellcolor{blue!40} \textbf{215} & 219 & 219 & \cellcolor{red!20} n/a & \cellcolor{red!20} n/a & 218 & \cellcolor{red!20} n/a \\
1000 135& 225 & \cellcolor{blue!20} 228 & 229 & 230 & \cellcolor{red!20} n/a & \cellcolor{red!20} n/a & \cellcolor{blue!20} 228 & \cellcolor{red!20} n/a \\
1000 136& 228 & \cellcolor{blue!20} 230 & 233 & 233 & \cellcolor{red!20} n/a & \cellcolor{red!20} n/a & \cellcolor{blue!20} 230 & \cellcolor{red!20} n/a \\
1000 137& 213 & \cellcolor{blue!20} 215 & 217 & 216 & \cellcolor{red!20} n/a & \cellcolor{red!20} n/a & \cellcolor{blue!20} 215 & \cellcolor{red!20} n/a \\
1000 138& 221 & 224 & 225 & 225 & \cellcolor{red!20} n/a & \cellcolor{red!20} n/a & \cellcolor{blue!40} 223 & \cellcolor{red!20} n/a \\
1000 139& 224 & 227 & 228 & 228 & \cellcolor{red!20} n/a & \cellcolor{red!20} n/a & \cellcolor{blue!40} 226 & \cellcolor{red!20} n/a \\
1000 14& 136 & \cellcolor{blue!40} \textbf{136} & 138 & 138 & \cellcolor{red!20} n/a & \cellcolor{red!20} n/a & 137 & 138 \\
1000 140& 226 & \cellcolor{blue!20} 228 & 230 & 230 & \cellcolor{red!20} n/a & \cellcolor{red!20} n/a & \cellcolor{blue!20} 228 & \cellcolor{red!20} n/a \\
1000 141& 215 & 219 & 219 & 219 & \cellcolor{red!20} n/a & \cellcolor{red!20} n/a & \cellcolor{blue!40} 217 & \cellcolor{red!20} n/a \\
1000 142& 220 & 223 & 223 & 224 & \cellcolor{red!20} n/a & \cellcolor{red!20} n/a & \cellcolor{blue!40} 222 & \cellcolor{red!20} n/a \\
1000 143& 213 & 216 & 217 & 217 & \cellcolor{red!20} n/a & \cellcolor{red!20} n/a & \cellcolor{blue!40} 215 & \cellcolor{red!20} n/a \\
1000 144& 217 & 220 & 221 & 221 & \cellcolor{red!20} n/a & \cellcolor{red!20} n/a & \cellcolor{blue!40} 219 & \cellcolor{red!20} n/a \\
1000 145& 220 & 223 & 224 & 224 & \cellcolor{red!20} n/a & \cellcolor{red!20} n/a & \cellcolor{blue!40} 222 & \cellcolor{red!20} n/a \\
1000 146& 219 & \cellcolor{blue!20} 222 & 224 & 223 & \cellcolor{red!20} n/a & \cellcolor{red!20} n/a & \cellcolor{blue!20} 222 & \cellcolor{red!20} n/a \\
1000 147& 229 & \cellcolor{blue!20} 232 & 234 & 234 & \cellcolor{red!20} n/a & \cellcolor{red!20} n/a & \cellcolor{blue!20} 232 & \cellcolor{red!20} n/a \\
1000 148& 219 & \cellcolor{blue!40} \textbf{219} & 223 & 223 & \cellcolor{red!20} n/a & \cellcolor{red!20} n/a & 221 & \cellcolor{red!20} n/a \\
1000 149& 237 & 240 & 241 & 241 & \cellcolor{red!20} n/a & \cellcolor{red!20} n/a & \cellcolor{blue!40} 239 & \cellcolor{red!20} n/a \\
1000 15& 136 & \cellcolor{blue!20} \textbf{136} & 137 & 137 & \cellcolor{red!20} n/a & \cellcolor{red!20} n/a & \cellcolor{blue!20} 136 & 137 \\
1000 150& 222 & 225 & 226 & 226 & \cellcolor{red!20} n/a & \cellcolor{red!20} n/a & \cellcolor{blue!40} 224 & \cellcolor{red!20} n/a \\
1000 151& 138 & \cellcolor{blue!40} \textbf{138} & 140 & 140 & \cellcolor{red!20} n/a & \cellcolor{red!20} n/a & 139 & 139 \\
1000 152& 136 & \cellcolor{blue!40} \textbf{136} & 138 & 138 & \cellcolor{red!20} n/a & \cellcolor{red!20} n/a & 137 & 138 \\
1000 153& 137 & \cellcolor{blue!40} \textbf{137} & 139 & 139 & \cellcolor{red!20} n/a & \cellcolor{red!20} n/a & 138 & 138 \\
1000 154& 140 & \cellcolor{blue!40} \textbf{140} & 142 & 142 & \cellcolor{red!20} n/a & \cellcolor{red!20} n/a & 141 & 141 \\
1000 155& 139 & \cellcolor{blue!40} \textbf{139} & 141 & 141 & \cellcolor{red!20} n/a & \cellcolor{red!20} n/a & 140 & 141 \\
1000 156& 141 & \cellcolor{blue!40} \textbf{141} & 143 & 143 & \cellcolor{red!20} n/a & \cellcolor{red!20} n/a & 142 & 143 \\
1000 157& 140 & \cellcolor{blue!40} \textbf{140} & 142 & 142 & \cellcolor{red!20} n/a & \cellcolor{red!20} n/a & 141 & 141 \\
1000 158& 136 & \cellcolor{blue!20} \textbf{136} & 137 & 137 & \cellcolor{red!20} n/a & \cellcolor{red!20} n/a & \cellcolor{blue!20} 136 & 137 \\
1000 159& 138 & \cellcolor{blue!40} \textbf{138} & 140 & 140 & \cellcolor{red!20} n/a & \cellcolor{red!20} n/a & 139 & 139 \\
1000 16& 137 & \cellcolor{blue!40} \textbf{137} & 138 & 138 & \cellcolor{red!20} n/a & \cellcolor{red!20} n/a & 138 & 138 \\
1000 160& 138 & \cellcolor{blue!40} \textbf{138} & 140 & 140 & \cellcolor{red!20} n/a & \cellcolor{red!20} n/a & 139 & 140 \\
1000 161& 133 & \cellcolor{blue!20} \textbf{133} & 134 & 134 & \cellcolor{red!20} n/a & \cellcolor{red!20} n/a & \cellcolor{blue!20} 133 & 134 \\
1000 162& 136 & \cellcolor{blue!20} \textbf{136} & 137 & 138 & \cellcolor{red!20} n/a & \cellcolor{red!20} n/a & \cellcolor{blue!20} 136 & 137 \\
1000 163& 139 & \cellcolor{blue!40} \textbf{139} & 141 & 141 & \cellcolor{red!20} n/a & \cellcolor{red!20} n/a & 140 & 141 \\
1000 164& 141 & \cellcolor{blue!40} \textbf{141} & 143 & 143 & \cellcolor{red!20} n/a & \cellcolor{red!20} n/a & 142 & 143 \\
1000 165& 135 & \cellcolor{blue!40} \textbf{135} & 137 & 137 & \cellcolor{red!20} n/a & \cellcolor{red!20} n/a & 136 & 137 \\
1000 166& 139 & \cellcolor{blue!40} \textbf{139} & 141 & 141 & \cellcolor{red!20} n/a & \cellcolor{red!20} n/a & 140 & 141 \\
1000 167& 139 & \cellcolor{blue!40} \textbf{139} & 141 & 141 & \cellcolor{red!20} n/a & \cellcolor{red!20} n/a & 140 & 141 \\
1000 168& 138 & \cellcolor{blue!40} \textbf{138} & 140 & 140 & \cellcolor{red!20} n/a & \cellcolor{red!20} n/a & 139 & 140 \\
1000 169& 134 & \cellcolor{blue!40} \textbf{134} & 136 & 136 & \cellcolor{red!20} n/a & \cellcolor{red!20} n/a & 135 & 136 \\
1000 17& 135 & \cellcolor{blue!20} \textbf{135} & 136 & 136 & \cellcolor{red!20} n/a & \cellcolor{red!20} n/a & \cellcolor{blue!20} 135 & 136 \\
1000 170& 134 & \cellcolor{blue!40} \textbf{134} & 136 & 136 & \cellcolor{red!20} n/a & \cellcolor{red!20} n/a & 135 & 136 \\
1000 171& 137 & \cellcolor{blue!40} \textbf{137} & 139 & 139 & \cellcolor{red!20} n/a & \cellcolor{red!20} n/a & 138 & 138 \\
1000 172& 135 & \cellcolor{blue!40} \textbf{135} & 136 & 136 & \cellcolor{red!20} n/a & \cellcolor{red!20} n/a & 136 & 136 \\
1000 173& 135 & \cellcolor{blue!40} \textbf{135} & 137 & 137 & \cellcolor{red!20} n/a & \cellcolor{red!20} n/a & 136 & 136 \\
1000 174& 136 & \cellcolor{blue!40} \textbf{136} & 138 & 138 & \cellcolor{red!20} n/a & \cellcolor{red!20} n/a & 137 & 137 \\
1000 175& 138 & \cellcolor{blue!40} \textbf{138} & 140 & 140 & \cellcolor{red!20} n/a & \cellcolor{red!20} n/a & 139 & 140 \\
1000 176& 507 & \cellcolor{blue!40} 529 & 559 & 577 & \cellcolor{red!20} n/a & \cellcolor{red!20} n/a & 538 & 561 \\
1000 177& 505 & \cellcolor{blue!40} 528 & 552 & 568 & \cellcolor{red!20} n/a & \cellcolor{red!20} n/a & 534 & 547 \\
1000 178& 521 & \cellcolor{blue!40} 547 & 568 & 573 & \cellcolor{red!20} n/a & \cellcolor{red!20} n/a & 555 & 564 \\
1000 179& 516 & \cellcolor{blue!40} 546 & 565 & 580 & \cellcolor{red!20} n/a & \cellcolor{red!20} n/a & 547 & 565 \\
1000 18& 134 & \cellcolor{blue!20} \textbf{134} & 135 & 135 & \cellcolor{red!20} n/a & \cellcolor{red!20} n/a & \cellcolor{blue!20} 134 & 135 \\
1000 180& 522 & \cellcolor{blue!40} 544 & 563 & 580 & \cellcolor{red!20} n/a & \cellcolor{red!20} n/a & 556 & 566 \\
1000 181& 515 & 553 & 567 & 584 & \cellcolor{red!20} n/a & \cellcolor{red!20} n/a & \cellcolor{blue!40} 552 & 566 \\
1000 182& 513 & \cellcolor{blue!40} 538 & 561 & 576 & \cellcolor{red!20} n/a & \cellcolor{red!20} n/a & 546 & 560 \\
1000 183& 510 & \cellcolor{blue!40} 534 & 552 & 564 & \cellcolor{red!20} n/a & \cellcolor{red!20} n/a & 543 & 554 \\
1000 184& 510 & \cellcolor{blue!40} 540 & 562 & 575 & \cellcolor{red!20} n/a & \cellcolor{red!20} n/a & 547 & 557 \\
1000 185& 512 & \cellcolor{blue!40} 540 & 565 & 567 & \cellcolor{red!20} n/a & \cellcolor{red!20} n/a & 547 & 557 \\
1000 186& 505 & \cellcolor{blue!40} 533 & 557 & 570 & \cellcolor{red!20} n/a & \cellcolor{red!20} n/a & 540 & 552 \\
1000 187& 520 & \cellcolor{blue!20} 556 & 569 & 584 & \cellcolor{red!20} n/a & \cellcolor{red!20} n/a & \cellcolor{blue!20} 556 & 565 \\
1000 188& 504 & \cellcolor{blue!40} 527 & 553 & 571 & \cellcolor{red!20} n/a & \cellcolor{red!20} n/a & 538 & 551 \\
1000 189& 501 & \cellcolor{blue!40} 534 & 556 & 567 & \cellcolor{red!20} n/a & \cellcolor{red!20} n/a & 540 & 552 \\
1000 19& 137 & \cellcolor{blue!20} \textbf{137} & 138 & 138 & \cellcolor{red!20} n/a & \cellcolor{red!20} n/a & \cellcolor{blue!20} 137 & 138 \\
1000 190& 512 & \cellcolor{blue!40} 535 & 556 & 573 & \cellcolor{red!20} n/a & \cellcolor{red!20} n/a & 547 & 556 \\
1000 191& 510 & \cellcolor{blue!40} 541 & 555 & 575 & \cellcolor{red!20} n/a & \cellcolor{red!20} n/a & 543 & 557 \\
1000 192& 507 & \cellcolor{blue!40} 530 & 559 & 568 & \cellcolor{red!20} n/a & \cellcolor{red!20} n/a & 544 & 556 \\
1000 193& 511 & \cellcolor{blue!40} 535 & 564 & 582 & \cellcolor{red!20} n/a & \cellcolor{red!20} n/a & 546 & 568 \\
1000 194& 508 & \cellcolor{blue!40} 532 & 561 & 580 & \cellcolor{red!20} n/a & \cellcolor{red!20} n/a & 542 & 558 \\
1000 195& 517 & \cellcolor{blue!40} 538 & 567 & 576 & \cellcolor{red!20} n/a & \cellcolor{red!20} n/a & 554 & 565 \\
1000 196& 515 & \cellcolor{blue!40} 537 & 561 & 574 & \cellcolor{red!20} n/a & \cellcolor{red!20} n/a & 547 & 559 \\
1000 197& 499 & \cellcolor{blue!40} 524 & 550 & 558 & \cellcolor{red!20} n/a & \cellcolor{red!20} n/a & 525 & 543 \\
1000 198& 515 & \cellcolor{blue!40} 544 & 566 & 580 & \cellcolor{red!20} n/a & \cellcolor{red!20} n/a & 549 & 565 \\
1000 199& 497 & \cellcolor{blue!40} 525 & 542 & 555 & \cellcolor{red!20} n/a & \cellcolor{red!20} n/a & 527 & 541 \\
1000 2& 137 & \cellcolor{blue!40} \textbf{137} & 138 & 138 & \cellcolor{red!20} n/a & \cellcolor{red!20} n/a & 138 & 138 \\
1000 20& 138 & \cellcolor{blue!20} \textbf{138} & 139 & 139 & \cellcolor{red!20} n/a & \cellcolor{red!20} n/a & \cellcolor{blue!20} 138 & 139 \\
1000 200& 500 & 537 & 554 & 566 & \cellcolor{red!20} n/a & \cellcolor{red!20} n/a & \cellcolor{blue!40} 534 & 546 \\
1000 201& 229 & \cellcolor{blue!20} 231 & 233 & 234 & \cellcolor{red!20} n/a & \cellcolor{red!20} n/a & \cellcolor{blue!20} 231 & 232 \\
1000 202& 225 & \cellcolor{blue!40} \textbf{225} & 230 & 230 & \cellcolor{red!20} n/a & \cellcolor{red!20} n/a & 228 & 229 \\
1000 203& 229 & \cellcolor{blue!20} 232 & 234 & 235 & \cellcolor{red!20} n/a & \cellcolor{red!20} n/a & \cellcolor{blue!20} 232 & 233 \\
1000 204& 228 & \cellcolor{blue!40} 230 & 233 & 233 & \cellcolor{red!20} n/a & \cellcolor{red!20} n/a & 231 & 232 \\
1000 205& 229 & \cellcolor{blue!20} 231 & 234 & 236 & \cellcolor{red!20} n/a & \cellcolor{red!20} n/a & \cellcolor{blue!20} 231 & 233 \\
1000 206& 229 & \cellcolor{blue!40} 230 & 233 & 233 & \cellcolor{red!20} n/a & \cellcolor{red!20} n/a & 231 & 233 \\
1000 207& 230 & \cellcolor{blue!40} \textbf{230} & 235 & 235 & \cellcolor{red!20} n/a & \cellcolor{red!20} n/a & 232 & 233 \\
1000 208& 229 & \cellcolor{blue!40} 231 & 234 & 234 & \cellcolor{red!20} n/a & \cellcolor{red!20} n/a & 232 & 233 \\
1000 209& 228 & \cellcolor{blue!40} \textbf{228} & 233 & 233 & \cellcolor{red!20} n/a & \cellcolor{red!20} n/a & 231 & 232 \\
1000 21& 138 & \cellcolor{blue!40} \textbf{138} & 139 & 139 & \cellcolor{red!20} n/a & \cellcolor{red!20} n/a & 139 & 139 \\
1000 210& 224 & \cellcolor{blue!40} \textbf{224} & 229 & 229 & \cellcolor{red!20} n/a & \cellcolor{red!20} n/a & 226 & 228 \\
1000 211& 219 & \cellcolor{blue!20} 221 & 223 & 224 & \cellcolor{red!20} n/a & \cellcolor{red!20} n/a & \cellcolor{blue!20} 221 & 223 \\
1000 212& 217 & \cellcolor{blue!20} 219 & 221 & 221 & \cellcolor{red!20} n/a & \cellcolor{red!20} n/a & \cellcolor{blue!20} 219 & 220 \\
1000 213& 233 & \cellcolor{blue!20} 236 & 238 & 238 & \cellcolor{red!20} n/a & \cellcolor{red!20} n/a & \cellcolor{blue!20} 236 & 238 \\
1000 214& 225 & \cellcolor{blue!40} 226 & 230 & 230 & \cellcolor{red!20} n/a & \cellcolor{red!20} n/a & 227 & 229 \\
1000 215& 223 & \cellcolor{blue!40} 224 & 227 & 229 & \cellcolor{red!20} n/a & \cellcolor{red!20} n/a & 225 & 227 \\
1000 216& 221 & \cellcolor{blue!40} 222 & 225 & 225 & \cellcolor{red!20} n/a & \cellcolor{red!20} n/a & 223 & 224 \\
1000 217& 225 & \cellcolor{blue!40} 227 & 229 & 230 & \cellcolor{red!20} n/a & \cellcolor{red!20} n/a & 228 & 229 \\
1000 218& 219 & \cellcolor{blue!20} 221 & 223 & 223 & \cellcolor{red!20} n/a & \cellcolor{red!20} n/a & \cellcolor{blue!20} 221 & 222 \\
1000 219& 232 & \cellcolor{blue!40} 233 & 236 & 238 & \cellcolor{red!20} n/a & \cellcolor{red!20} n/a & 235 & 235 \\
1000 22& 137 & \cellcolor{blue!40} \textbf{137} & 139 & 138 & \cellcolor{red!20} n/a & \cellcolor{red!20} n/a & 138 & 138 \\
1000 220& 225 & \cellcolor{blue!20} 227 & 229 & 230 & \cellcolor{red!20} n/a & \cellcolor{red!20} n/a & \cellcolor{blue!20} 227 & 228 \\
1000 221& 231 & \cellcolor{blue!40} 233 & 236 & 236 & \cellcolor{red!20} n/a & \cellcolor{red!20} n/a & 234 & 235 \\
1000 222& 221 & \cellcolor{blue!40} 222 & 226 & 227 & \cellcolor{red!20} n/a & \cellcolor{red!20} n/a & 224 & 225 \\
1000 223& 221 & \cellcolor{blue!40} 222 & 226 & 226 & \cellcolor{red!20} n/a & \cellcolor{red!20} n/a & 224 & 225 \\
1000 224& 226 & \cellcolor{blue!40} 227 & 231 & 232 & \cellcolor{red!20} n/a & \cellcolor{red!20} n/a & 229 & 230 \\
1000 225& 229 & \cellcolor{blue!40} 231 & 234 & 235 & \cellcolor{red!20} n/a & \cellcolor{red!20} n/a & 232 & 233 \\
1000 226& 136 & \cellcolor{blue!40} \textbf{136} & 138 & 138 & \cellcolor{red!20} n/a & \cellcolor{red!20} n/a & 137 & 138 \\
1000 227& 138 & \cellcolor{blue!40} \textbf{138} & 140 & 139 & \cellcolor{red!20} n/a & \cellcolor{red!20} n/a & 139 & \cellcolor{red!20} n/a \\
1000 228& 133 & \cellcolor{blue!40} \textbf{133} & 135 & 135 & \cellcolor{red!20} n/a & \cellcolor{red!20} n/a & 135 & 135 \\
1000 229& 134 & \cellcolor{blue!40} \textbf{134} & 136 & 136 & \cellcolor{red!20} n/a & \cellcolor{red!20} n/a & 135 & 136 \\
1000 23& 136 & \cellcolor{blue!40} \textbf{136} & 137 & 137 & \cellcolor{red!20} n/a & \cellcolor{red!20} n/a & 137 & 137 \\
1000 230& 131 & \cellcolor{blue!40} \textbf{131} & 134 & 133 & \cellcolor{red!20} n/a & \cellcolor{red!20} n/a & 132 & 133 \\
1000 231& 138 & \cellcolor{blue!40} \textbf{138} & 141 & 140 & \cellcolor{red!20} n/a & \cellcolor{red!20} n/a & 139 & 140 \\
1000 232& 133 & \cellcolor{blue!40} \textbf{133} & 136 & 135 & \cellcolor{red!20} n/a & \cellcolor{red!20} n/a & 134 & 135 \\
1000 233& 135 & \cellcolor{blue!40} \textbf{135} & 138 & 137 & \cellcolor{red!20} n/a & \cellcolor{red!20} n/a & 137 & 137 \\
1000 234& 137 & \cellcolor{blue!40} \textbf{137} & 139 & 139 & \cellcolor{red!20} n/a & \cellcolor{red!20} n/a & 138 & 139 \\
1000 235& 133 & \cellcolor{blue!40} \textbf{133} & 134 & 135 & \cellcolor{red!20} n/a & \cellcolor{red!20} n/a & 134 & 135 \\
1000 236& 136 & \cellcolor{blue!40} \textbf{136} & 138 & 138 & \cellcolor{red!20} n/a & \cellcolor{red!20} n/a & 137 & 138 \\
1000 237& 138 & \cellcolor{blue!40} \textbf{138} & 141 & 140 & \cellcolor{red!20} n/a & \cellcolor{red!20} n/a & 140 & 140 \\
1000 238& 138 & \cellcolor{blue!40} \textbf{138} & 140 & 140 & \cellcolor{red!20} n/a & \cellcolor{red!20} n/a & 139 & 140 \\
1000 239& 135 & \cellcolor{blue!40} \textbf{135} & 137 & 137 & \cellcolor{red!20} n/a & \cellcolor{red!20} n/a & 136 & 136 \\
1000 24& 140 & \cellcolor{blue!40} \textbf{140} & 141 & 141 & \cellcolor{red!20} n/a & \cellcolor{red!20} n/a & 141 & 141 \\
1000 240& 135 & \cellcolor{blue!40} \textbf{135} & 138 & 137 & \cellcolor{red!20} n/a & \cellcolor{red!20} n/a & 137 & 137 \\
1000 241& 138 & \cellcolor{blue!40} \textbf{138} & 140 & 140 & \cellcolor{red!20} n/a & \cellcolor{red!20} n/a & 139 & 140 \\
1000 242& 135 & \cellcolor{blue!40} \textbf{135} & 137 & 137 & \cellcolor{red!20} n/a & \cellcolor{red!20} n/a & 136 & 137 \\
1000 243& 137 & \cellcolor{blue!40} \textbf{137} & 139 & 139 & \cellcolor{red!20} n/a & \cellcolor{red!20} n/a & 138 & 139 \\
1000 244& 137 & \cellcolor{blue!40} \textbf{137} & 139 & 139 & \cellcolor{red!20} n/a & \cellcolor{red!20} n/a & 138 & 139 \\
1000 245& 135 & \cellcolor{blue!40} \textbf{135} & 137 & 137 & \cellcolor{red!20} n/a & \cellcolor{red!20} n/a & 136 & 137 \\
1000 246& 135 & \cellcolor{blue!40} \textbf{135} & 138 & 137 & \cellcolor{red!20} n/a & \cellcolor{red!20} n/a & 136 & 137 \\
1000 247& 138 & \cellcolor{blue!40} \textbf{138} & 141 & 141 & \cellcolor{red!20} n/a & \cellcolor{red!20} n/a & 140 & 140 \\
1000 248& 138 & \cellcolor{blue!40} \textbf{138} & 141 & 141 & \cellcolor{red!20} n/a & \cellcolor{red!20} n/a & 140 & 141 \\
1000 249& 138 & \cellcolor{blue!40} \textbf{138} & 141 & 140 & \cellcolor{red!20} n/a & \cellcolor{red!20} n/a & 139 & 140 \\
1000 25& 136 & \cellcolor{blue!20} \textbf{136} & 137 & 137 & \cellcolor{red!20} n/a & \cellcolor{red!20} n/a & \cellcolor{blue!20} 136 & 137 \\
1000 250& 140 & \cellcolor{blue!40} \textbf{140} & 142 & 142 & \cellcolor{red!20} n/a & \cellcolor{red!20} n/a & 141 & 142 \\
1000 251& 516 & \cellcolor{blue!40} 557 & 575 & 587 & \cellcolor{red!20} n/a & \cellcolor{red!20} n/a & 561 & \cellcolor{red!20} n/a \\
1000 252& 512 & \cellcolor{blue!40} 558 & 574 & 583 & \cellcolor{red!20} n/a & \cellcolor{red!20} n/a & 563 & \cellcolor{red!20} n/a \\
1000 253& 511 & \cellcolor{blue!40} 557 & 568 & 580 & \cellcolor{red!20} n/a & \cellcolor{red!20} n/a & 561 & \cellcolor{red!20} n/a \\
1000 254& 511 & \cellcolor{blue!40} 555 & 569 & 582 & \cellcolor{red!20} n/a & \cellcolor{red!20} n/a & 559 & \cellcolor{red!20} n/a \\
1000 255& 504 & \cellcolor{blue!40} 546 & 561 & 569 & \cellcolor{red!20} n/a & \cellcolor{red!20} n/a & 550 & \cellcolor{red!20} n/a \\
1000 256& 496 & \cellcolor{blue!40} 546 & 562 & 570 & \cellcolor{red!20} n/a & \cellcolor{red!20} n/a & 551 & \cellcolor{red!20} n/a \\
1000 257& 517 & \cellcolor{blue!40} 558 & 575 & 584 & \cellcolor{red!20} n/a & \cellcolor{red!20} n/a & 564 & \cellcolor{red!20} n/a \\
1000 258& 502 & \cellcolor{blue!40} 549 & 565 & 579 & \cellcolor{red!20} n/a & \cellcolor{red!20} n/a & 557 & \cellcolor{red!20} n/a \\
1000 259& 498 & \cellcolor{blue!40} 544 & 563 & 571 & \cellcolor{red!20} n/a & \cellcolor{red!20} n/a & 547 & \cellcolor{red!20} n/a \\
1000 26& 515 & \cellcolor{blue!40} 532 & 555 & 570 & \cellcolor{red!20} n/a & \cellcolor{red!20} n/a & 541 & 553 \\
1000 260& 495 & \cellcolor{blue!40} 542 & 562 & 568 & \cellcolor{red!20} n/a & \cellcolor{red!20} n/a & 549 & \cellcolor{red!20} n/a \\
1000 261& 509 & \cellcolor{blue!40} 548 & 570 & 581 & \cellcolor{red!20} n/a & \cellcolor{red!20} n/a & 558 & \cellcolor{red!20} n/a \\
1000 262& 501 & \cellcolor{blue!40} 538 & 551 & 563 & \cellcolor{red!20} n/a & \cellcolor{red!20} n/a & 540 & 557 \\
1000 263& 514 & \cellcolor{blue!40} 545 & 567 & 572 & \cellcolor{red!20} n/a & \cellcolor{red!20} n/a & 558 & \cellcolor{red!20} n/a \\
1000 264& 500 & \cellcolor{blue!40} 551 & 562 & 578 & \cellcolor{red!20} n/a & \cellcolor{red!20} n/a & 556 & \cellcolor{red!20} n/a \\
1000 265& 527 & \cellcolor{blue!40} 568 & 586 & 594 & \cellcolor{red!20} n/a & \cellcolor{red!20} n/a & 574 & \cellcolor{red!20} n/a \\
1000 266& 507 & \cellcolor{blue!40} 543 & 566 & 574 & \cellcolor{red!20} n/a & \cellcolor{red!20} n/a & 556 & \cellcolor{red!20} n/a \\
1000 267& 519 & 566 & 576 & 590 & \cellcolor{red!20} n/a & \cellcolor{red!20} n/a & \cellcolor{blue!40} 565 & \cellcolor{red!20} n/a \\
1000 268& 504 & \cellcolor{blue!40} 537 & 562 & 567 & \cellcolor{red!20} n/a & \cellcolor{red!20} n/a & 547 & \cellcolor{red!20} n/a \\
1000 269& 503 & \cellcolor{blue!20} 553 & 561 & 570 & \cellcolor{red!20} n/a & \cellcolor{red!20} n/a & \cellcolor{blue!20} 553 & \cellcolor{red!20} n/a \\
1000 27& 516 & \cellcolor{blue!40} 536 & 554 & 569 & \cellcolor{red!20} n/a & \cellcolor{red!20} n/a & 542 & 550 \\
1000 270& 532 & \cellcolor{blue!40} 570 & 584 & 603 & \cellcolor{red!20} n/a & \cellcolor{red!20} n/a & 581 & \cellcolor{red!20} n/a \\
1000 271& 501 & \cellcolor{blue!40} 545 & 557 & 567 & \cellcolor{red!20} n/a & \cellcolor{red!20} n/a & 546 & \cellcolor{red!20} n/a \\
1000 272& 516 & \cellcolor{blue!40} 554 & 572 & 584 & \cellcolor{red!20} n/a & \cellcolor{red!20} n/a & 562 & \cellcolor{red!20} n/a \\
1000 273& 512 & \cellcolor{blue!40} 550 & 569 & 578 & \cellcolor{red!20} n/a & \cellcolor{red!20} n/a & 561 & \cellcolor{red!20} n/a \\
1000 274& 507 & \cellcolor{blue!40} 551 & 565 & 583 & \cellcolor{red!20} n/a & \cellcolor{red!20} n/a & 555 & \cellcolor{red!20} n/a \\
1000 275& 516 & \cellcolor{blue!40} 563 & 574 & 586 & \cellcolor{red!20} n/a & \cellcolor{red!20} n/a & 568 & \cellcolor{red!20} n/a \\
1000 276& 217 & \cellcolor{blue!20} 220 & 223 & 223 & \cellcolor{red!20} n/a & \cellcolor{red!20} n/a & \cellcolor{blue!20} 220 & 222 \\
1000 277& 225 & \cellcolor{blue!40} 227 & 231 & 231 & \cellcolor{red!20} n/a & \cellcolor{red!20} n/a & 229 & 230 \\
1000 278& 220 & \cellcolor{blue!40} \textbf{220} & 226 & 226 & \cellcolor{red!20} n/a & \cellcolor{red!20} n/a & 224 & 226 \\
1000 279& 215 & \cellcolor{blue!40} 218 & 220 & 220 & \cellcolor{red!20} n/a & \cellcolor{red!20} n/a & 219 & 220 \\
1000 28& 510 & \cellcolor{blue!40} 523 & 538 & 557 & \cellcolor{red!20} n/a & \cellcolor{red!20} n/a & 531 & 534 \\
1000 280& 226 & \cellcolor{blue!20} 229 & 231 & 232 & \cellcolor{red!20} n/a & \cellcolor{red!20} n/a & \cellcolor{blue!20} 229 & \cellcolor{red!20} n/a \\
1000 281& 219 & \cellcolor{blue!20} 223 & 225 & 225 & \cellcolor{red!20} n/a & \cellcolor{red!20} n/a & \cellcolor{blue!20} 223 & 226 \\
1000 282& 214 & \cellcolor{blue!40} 215 & 220 & 220 & \cellcolor{red!20} n/a & \cellcolor{red!20} n/a & 218 & 219 \\
1000 283& 224 & \cellcolor{blue!40} \textbf{224} & 230 & 231 & \cellcolor{red!20} n/a & \cellcolor{red!20} n/a & 228 & 230 \\
1000 284& 217 & \cellcolor{blue!20} 220 & 222 & 222 & \cellcolor{red!20} n/a & \cellcolor{red!20} n/a & \cellcolor{blue!20} 220 & 222 \\
1000 285& 221 & \cellcolor{blue!20} 225 & 227 & 228 & \cellcolor{red!20} n/a & \cellcolor{red!20} n/a & \cellcolor{blue!20} 225 & 227 \\
1000 286& 221 & \cellcolor{blue!40} 224 & 227 & 227 & \cellcolor{red!20} n/a & \cellcolor{red!20} n/a & 225 & 226 \\
1000 287& 224 & \cellcolor{blue!20} 227 & 230 & 230 & \cellcolor{red!20} n/a & \cellcolor{red!20} n/a & \cellcolor{blue!20} 227 & 229 \\
1000 288& 219 & \cellcolor{blue!40} 222 & 225 & 225 & \cellcolor{red!20} n/a & \cellcolor{red!20} n/a & 223 & 224 \\
1000 289& 220 & \cellcolor{blue!40} 223 & 225 & 225 & \cellcolor{red!20} n/a & \cellcolor{red!20} n/a & 224 & 226 \\
1000 29& 507 & \cellcolor{blue!40} 524 & 545 & 560 & \cellcolor{red!20} n/a & \cellcolor{red!20} n/a & 530 & 540 \\
1000 290& 222 & \cellcolor{blue!40} 224 & 228 & 229 & \cellcolor{red!20} n/a & \cellcolor{red!20} n/a & 226 & 228 \\
1000 291& 225 & \cellcolor{blue!40} 227 & 231 & 230 & \cellcolor{red!20} n/a & \cellcolor{red!20} n/a & 228 & 230 \\
1000 292& 226 & \cellcolor{blue!40} 228 & 232 & 232 & \cellcolor{red!20} n/a & \cellcolor{red!20} n/a & 230 & 231 \\
1000 293& 225 & \cellcolor{blue!40} \textbf{225} & 231 & 232 & \cellcolor{red!20} n/a & \cellcolor{red!20} n/a & 228 & 231 \\
1000 294& 230 & \cellcolor{blue!20} 234 & 236 & 236 & \cellcolor{red!20} n/a & \cellcolor{red!20} n/a & \cellcolor{blue!20} 234 & \cellcolor{red!20} n/a \\
1000 295& 227 & \cellcolor{blue!40} 228 & 233 & 233 & \cellcolor{red!20} n/a & \cellcolor{red!20} n/a & 231 & 232 \\
1000 296& 208 & \cellcolor{blue!20} 211 & 213 & 213 & \cellcolor{red!20} n/a & \cellcolor{red!20} n/a & \cellcolor{blue!20} 211 & 212 \\
1000 297& 217 & \cellcolor{blue!40} 218 & 222 & 222 & \cellcolor{red!20} n/a & \cellcolor{red!20} n/a & 220 & 221 \\
1000 298& 214 & \cellcolor{blue!40} 217 & 220 & 220 & \cellcolor{red!20} n/a & \cellcolor{red!20} n/a & 218 & 220 \\
1000 299& 226 & \cellcolor{blue!20} 230 & 232 & 233 & \cellcolor{red!20} n/a & \cellcolor{red!20} n/a & \cellcolor{blue!20} 230 & \cellcolor{red!20} n/a \\
1000 3& 136 & \cellcolor{blue!40} \textbf{136} & 138 & 138 & \cellcolor{red!20} n/a & \cellcolor{red!20} n/a & 137 & 138 \\
1000 30& 528 & \cellcolor{blue!20} 548 & 559 & 573 & \cellcolor{red!20} n/a & \cellcolor{red!20} n/a & \cellcolor{blue!20} 548 & 556 \\
1000 300& 228 & \cellcolor{blue!20} 232 & 234 & 235 & \cellcolor{red!20} n/a & \cellcolor{red!20} n/a & \cellcolor{blue!20} 232 & 233 \\
1000 301& 137 & \cellcolor{blue!40} \textbf{137} & 138 & 138 & \cellcolor{red!20} n/a & \cellcolor{red!20} n/a & 138 & 138 \\
1000 302& 139 & \cellcolor{blue!20} \textbf{139} & 140 & 140 & \cellcolor{red!20} n/a & \cellcolor{red!20} n/a & \cellcolor{blue!20} 139 & 140 \\
1000 303& 138 & \cellcolor{blue!40} \textbf{138} & 140 & 140 & \cellcolor{red!20} n/a & \cellcolor{red!20} n/a & 139 & 140 \\
1000 304& 136 & \cellcolor{blue!40} \textbf{136} & 138 & 138 & \cellcolor{red!20} n/a & \cellcolor{red!20} n/a & 137 & 137 \\
1000 305& 140 & \cellcolor{blue!20} \textbf{140} & 141 & 141 & \cellcolor{red!20} n/a & \cellcolor{red!20} n/a & \cellcolor{blue!20} 140 & 141 \\
1000 306& 135 & \cellcolor{blue!40} \textbf{135} & 136 & 136 & \cellcolor{red!20} n/a & \cellcolor{red!20} n/a & 136 & 136 \\
1000 307& 136 & \cellcolor{blue!20} \textbf{136} & 137 & 137 & \cellcolor{red!20} n/a & \cellcolor{red!20} n/a & \cellcolor{blue!20} 136 & 137 \\
1000 308& 137 & \cellcolor{blue!40} \textbf{137} & 138 & 139 & \cellcolor{red!20} n/a & \cellcolor{red!20} n/a & 138 & 138 \\
1000 309& 135 & \cellcolor{blue!20} \textbf{135} & 136 & 136 & \cellcolor{red!20} n/a & \cellcolor{red!20} n/a & \cellcolor{blue!20} 135 & 136 \\
1000 31& 520 & \cellcolor{blue!40} 536 & 558 & 571 & \cellcolor{red!20} n/a & \cellcolor{red!20} n/a & 543 & 548 \\
1000 310& 141 & \cellcolor{blue!40} \textbf{141} & 143 & 143 & \cellcolor{red!20} n/a & \cellcolor{red!20} n/a & 142 & 143 \\
1000 311& 139 & \cellcolor{blue!40} \textbf{139} & 141 & 141 & \cellcolor{red!20} n/a & \cellcolor{red!20} n/a & 140 & 140 \\
1000 312& 135 & \cellcolor{blue!20} \textbf{135} & 136 & 136 & \cellcolor{red!20} n/a & \cellcolor{red!20} n/a & \cellcolor{blue!20} 135 & 136 \\
1000 313& 138 & \cellcolor{blue!40} \textbf{138} & 139 & 139 & \cellcolor{red!20} n/a & \cellcolor{red!20} n/a & 139 & 139 \\
1000 314& 142 & \cellcolor{blue!20} \textbf{142} & 143 & 143 & \cellcolor{red!20} n/a & \cellcolor{red!20} n/a & \cellcolor{blue!20} 142 & 143 \\
1000 315& 136 & \cellcolor{blue!40} \textbf{136} & 138 & 138 & \cellcolor{red!20} n/a & \cellcolor{red!20} n/a & 137 & 138 \\
1000 316& 137 & \cellcolor{blue!40} \textbf{137} & 139 & 139 & \cellcolor{red!20} n/a & \cellcolor{red!20} n/a & 138 & 138 \\
1000 317& 136 & \cellcolor{blue!40} \textbf{136} & 137 & 137 & \cellcolor{red!20} n/a & \cellcolor{red!20} n/a & 137 & 137 \\
1000 318& 138 & \cellcolor{blue!40} \textbf{138} & 139 & 139 & \cellcolor{red!20} n/a & \cellcolor{red!20} n/a & 139 & 139 \\
1000 319& 140 & \cellcolor{blue!40} \textbf{140} & 142 & 142 & \cellcolor{red!20} n/a & \cellcolor{red!20} n/a & 141 & 141 \\
1000 32& 507 & \cellcolor{blue!40} 527 & 543 & 567 & \cellcolor{red!20} n/a & \cellcolor{red!20} n/a & 531 & 536 \\
1000 320& 141 & \cellcolor{blue!40} \textbf{141} & 142 & 142 & \cellcolor{red!20} n/a & \cellcolor{red!20} n/a & 142 & 142 \\
1000 321& 140 & \cellcolor{blue!40} \textbf{140} & 141 & 141 & \cellcolor{red!20} n/a & \cellcolor{red!20} n/a & 141 & 141 \\
1000 322& 139 & \cellcolor{blue!20} \textbf{139} & 140 & 140 & \cellcolor{red!20} n/a & \cellcolor{red!20} n/a & \cellcolor{blue!20} 139 & 140 \\
1000 323& 138 & \cellcolor{blue!40} \textbf{138} & 140 & 140 & \cellcolor{red!20} n/a & \cellcolor{red!20} n/a & 139 & 139 \\
1000 324& 140 & \cellcolor{blue!40} \textbf{140} & 141 & 142 & \cellcolor{red!20} n/a & \cellcolor{red!20} n/a & 141 & 141 \\
1000 325& 138 & \cellcolor{blue!40} \textbf{138} & 140 & 140 & \cellcolor{red!20} n/a & \cellcolor{red!20} n/a & 139 & 139 \\
1000 326& 504 & \cellcolor{blue!40} 523 & 542 & 561 & \cellcolor{red!20} n/a & \cellcolor{red!20} n/a & 529 & 537 \\
1000 327& 509 & \cellcolor{blue!40} 536 & 554 & 573 & \cellcolor{red!20} n/a & \cellcolor{red!20} n/a & 540 & 548 \\
1000 328& 506 & 532 & 546 & 558 & \cellcolor{red!20} n/a & \cellcolor{red!20} n/a & \cellcolor{blue!40} 528 & 538 \\
1000 329& 509 & \cellcolor{blue!40} 535 & 555 & 569 & \cellcolor{red!20} n/a & \cellcolor{red!20} n/a & 539 & 548 \\
1000 33& 509 & \cellcolor{blue!40} 531 & 548 & 561 & \cellcolor{red!20} n/a & \cellcolor{red!20} n/a & 537 & 544 \\
1000 330& 509 & \cellcolor{blue!40} 522 & 538 & 553 & \cellcolor{red!20} n/a & \cellcolor{red!20} n/a & 527 & 535 \\
1000 331& 502 & 531 & 547 & 559 & \cellcolor{red!20} n/a & \cellcolor{red!20} n/a & \cellcolor{blue!40} 530 & 537 \\
1000 332& 497 & 524 & 535 & 549 & \cellcolor{red!20} n/a & \cellcolor{red!20} n/a & \cellcolor{blue!40} 522 & 533 \\
1000 333& 515 & \cellcolor{blue!40} 539 & 554 & 569 & \cellcolor{red!20} n/a & \cellcolor{red!20} n/a & 542 & 549 \\
1000 334& 501 & \cellcolor{blue!40} 520 & 540 & 554 & \cellcolor{red!20} n/a & \cellcolor{red!20} n/a & 522 & 536 \\
1000 335& 511 & \cellcolor{blue!40} 526 & 544 & 555 & \cellcolor{red!20} n/a & \cellcolor{red!20} n/a & 532 & 544 \\
1000 336& 502 & 528 & 534 & 552 & \cellcolor{red!20} n/a & \cellcolor{red!20} n/a & \cellcolor{blue!40} 523 & 533 \\
1000 337& 515 & \cellcolor{blue!40} 535 & 551 & 570 & \cellcolor{red!20} n/a & \cellcolor{red!20} n/a & 539 & 548 \\
1000 338& 512 & 540 & 553 & 563 & \cellcolor{red!20} n/a & \cellcolor{red!20} n/a & \cellcolor{blue!40} 535 & 549 \\
1000 339& 526 & 542 & 556 & 566 & \cellcolor{red!20} n/a & \cellcolor{red!20} n/a & \cellcolor{blue!40} 540 & 550 \\
1000 34& 534 & \cellcolor{blue!40} 551 & 563 & 579 & \cellcolor{red!20} n/a & \cellcolor{red!20} n/a & 555 & 560 \\
1000 340& 531 & \cellcolor{blue!40} 545 & 566 & 575 & \cellcolor{red!20} n/a & \cellcolor{red!20} n/a & 551 & 558 \\
1000 341& 513 & \cellcolor{blue!40} 539 & 552 & 572 & \cellcolor{red!20} n/a & \cellcolor{red!20} n/a & 543 & 550 \\
1000 342& 506 & \cellcolor{blue!40} 530 & 549 & 567 & \cellcolor{red!20} n/a & \cellcolor{red!20} n/a & 535 & 547 \\
1000 343& 510 & \cellcolor{blue!40} 540 & 556 & 568 & \cellcolor{red!20} n/a & \cellcolor{red!20} n/a & 541 & 548 \\
1000 344& 510 & \cellcolor{blue!40} 531 & 546 & 563 & \cellcolor{red!20} n/a & \cellcolor{red!20} n/a & 533 & 545 \\
1000 345& 510 & \cellcolor{blue!40} 538 & 554 & 567 & \cellcolor{red!20} n/a & \cellcolor{red!20} n/a & 540 & 548 \\
1000 346& 505 & \cellcolor{blue!40} 528 & 552 & 562 & \cellcolor{red!20} n/a & \cellcolor{red!20} n/a & 531 & 543 \\
1000 347& 505 & \cellcolor{blue!40} 531 & 547 & 561 & \cellcolor{red!20} n/a & \cellcolor{red!20} n/a & 538 & 543 \\
1000 348& 539 & \cellcolor{blue!40} 553 & 566 & 578 & \cellcolor{red!20} n/a & \cellcolor{red!20} n/a & 557 & 563 \\
1000 349& 512 & \cellcolor{blue!40} 535 & 559 & 570 & \cellcolor{red!20} n/a & \cellcolor{red!20} n/a & 542 & 550 \\
1000 35& 506 & \cellcolor{blue!40} 529 & 544 & 563 & \cellcolor{red!20} n/a & \cellcolor{red!20} n/a & 531 & 536 \\
1000 350& 498 & 526 & 534 & 555 & \cellcolor{red!20} n/a & \cellcolor{red!20} n/a & \cellcolor{blue!40} 525 & 529 \\
1000 351& 227 & \cellcolor{blue!40} 229 & 231 & 232 & \cellcolor{red!20} n/a & \cellcolor{red!20} n/a & 230 & 230 \\
1000 352& 227 & \cellcolor{blue!10} 229 & 231 & 232 & \cellcolor{red!20} n/a & \cellcolor{red!20} n/a & \cellcolor{blue!10} 229 & \cellcolor{blue!10} 229 \\
1000 353& 217 & \cellcolor{blue!40} 218 & 221 & 221 & \cellcolor{red!20} n/a & \cellcolor{red!20} n/a & 219 & 220 \\
1000 354& 222 & \cellcolor{blue!20} 224 & 226 & 227 & \cellcolor{red!20} n/a & \cellcolor{red!20} n/a & \cellcolor{blue!20} 224 & 225 \\
1000 355& 220 & \cellcolor{blue!20} 222 & 224 & 224 & \cellcolor{red!20} n/a & \cellcolor{red!20} n/a & \cellcolor{blue!20} 222 & 223 \\
1000 356& 226 & \cellcolor{blue!40} 227 & 230 & 231 & \cellcolor{red!20} n/a & \cellcolor{red!20} n/a & 228 & 229 \\
1000 357& 213 & \cellcolor{blue!40} 214 & 217 & 217 & \cellcolor{red!20} n/a & \cellcolor{red!20} n/a & 215 & 216 \\
1000 358& 219 & \cellcolor{blue!40} 220 & 223 & 223 & \cellcolor{red!20} n/a & \cellcolor{red!20} n/a & 221 & 222 \\
1000 359& 222 & \cellcolor{blue!20} 224 & 226 & 226 & \cellcolor{red!20} n/a & \cellcolor{red!20} n/a & \cellcolor{blue!20} 224 & 225 \\
1000 36& 502 & \cellcolor{blue!40} 527 & 538 & 556 & \cellcolor{red!20} n/a & \cellcolor{red!20} n/a & 529 & 530 \\
1000 360& 229 & \cellcolor{blue!40} 230 & 233 & 234 & \cellcolor{red!20} n/a & \cellcolor{red!20} n/a & 232 & 232 \\
1000 361& 215 & \cellcolor{blue!40} \textbf{215} & 219 & 219 & \cellcolor{red!20} n/a & \cellcolor{red!20} n/a & 217 & 218 \\
1000 362& 223 & \cellcolor{blue!40} 224 & 226 & 227 & \cellcolor{red!20} n/a & \cellcolor{red!20} n/a & 225 & 225 \\
1000 363& 215 & \cellcolor{blue!20} 217 & 219 & 219 & \cellcolor{red!20} n/a & \cellcolor{red!20} n/a & \cellcolor{blue!20} 217 & 218 \\
1000 364& 221 & \cellcolor{blue!40} 222 & 225 & 226 & \cellcolor{red!20} n/a & \cellcolor{red!20} n/a & 223 & 224 \\
1000 365& 227 & \cellcolor{blue!40} 229 & 231 & 233 & \cellcolor{red!20} n/a & \cellcolor{red!20} n/a & 230 & 230 \\
1000 366& 228 & \cellcolor{blue!10} 230 & 232 & 232 & \cellcolor{red!20} n/a & \cellcolor{red!20} n/a & \cellcolor{blue!10} 230 & \cellcolor{blue!10} 230 \\
1000 367& 227 & \cellcolor{blue!40} 228 & 231 & 232 & \cellcolor{red!20} n/a & \cellcolor{red!20} n/a & 229 & 230 \\
1000 368& 226 & \cellcolor{blue!20} 228 & 230 & 230 & \cellcolor{red!20} n/a & \cellcolor{red!20} n/a & \cellcolor{blue!20} 228 & 229 \\
1000 369& 220 & \cellcolor{blue!40} 221 & 224 & 224 & \cellcolor{red!20} n/a & \cellcolor{red!20} n/a & 223 & 223 \\
1000 37& 529 & \cellcolor{blue!40} 547 & 562 & 581 & \cellcolor{red!20} n/a & \cellcolor{red!20} n/a & 550 & 558 \\
1000 370& 223 & \cellcolor{blue!40} 224 & 227 & 227 & \cellcolor{red!20} n/a & \cellcolor{red!20} n/a & 226 & 226 \\
1000 371& 220 & \cellcolor{blue!40} 221 & 223 & 223 & \cellcolor{red!20} n/a & \cellcolor{red!20} n/a & 222 & 222 \\
1000 372& 230 & \cellcolor{blue!40} 232 & 234 & 236 & \cellcolor{red!20} n/a & \cellcolor{red!20} n/a & 233 & 233 \\
1000 373& 219 & \cellcolor{blue!40} 220 & 223 & 222 & \cellcolor{red!20} n/a & \cellcolor{red!20} n/a & 221 & 221 \\
1000 374& 219 & \cellcolor{blue!40} 220 & 222 & 222 & \cellcolor{red!20} n/a & \cellcolor{red!20} n/a & 221 & 221 \\
1000 375& 227 & \cellcolor{blue!10} 229 & 231 & 230 & \cellcolor{red!20} n/a & \cellcolor{red!20} n/a & \cellcolor{blue!10} 229 & \cellcolor{blue!10} 229 \\
1000 376& 132 & \cellcolor{blue!40} \textbf{132} & 134 & 134 & \cellcolor{red!20} n/a & \cellcolor{red!20} n/a & 133 & 134 \\
1000 377& 137 & \cellcolor{blue!40} \textbf{137} & 139 & 139 & \cellcolor{red!20} n/a & \cellcolor{red!20} n/a & 138 & 138 \\
1000 378& 134 & \cellcolor{blue!40} \textbf{134} & 136 & 136 & \cellcolor{red!20} n/a & \cellcolor{red!20} n/a & 135 & 136 \\
1000 379& 137 & \cellcolor{blue!40} \textbf{137} & 139 & 139 & \cellcolor{red!20} n/a & \cellcolor{red!20} n/a & 138 & 139 \\
1000 38& 519 & 546 & 557 & 574 & \cellcolor{red!20} n/a & \cellcolor{red!20} n/a & \cellcolor{blue!40} 545 & 552 \\
1000 380& 134 & \cellcolor{blue!40} \textbf{134} & 136 & 136 & \cellcolor{red!20} n/a & \cellcolor{red!20} n/a & 135 & 136 \\
1000 381& 138 & \cellcolor{blue!40} \textbf{138} & 140 & 140 & \cellcolor{red!20} n/a & \cellcolor{red!20} n/a & 139 & 139 \\
1000 382& 131 & \cellcolor{blue!40} \textbf{131} & 133 & 133 & \cellcolor{red!20} n/a & \cellcolor{red!20} n/a & 132 & 132 \\
1000 383& 138 & \cellcolor{blue!40} \textbf{138} & 141 & 140 & \cellcolor{red!20} n/a & \cellcolor{red!20} n/a & 139 & 140 \\
1000 384& 139 & \cellcolor{blue!40} \textbf{139} & 141 & 141 & \cellcolor{red!20} n/a & \cellcolor{red!20} n/a & 140 & 141 \\
1000 385& 135 & \cellcolor{blue!40} \textbf{135} & 137 & 137 & \cellcolor{red!20} n/a & \cellcolor{red!20} n/a & 137 & 138 \\
1000 386& 139 & \cellcolor{blue!40} \textbf{139} & 141 & 140 & \cellcolor{red!20} n/a & \cellcolor{red!20} n/a & 140 & 140 \\
1000 387& 137 & \cellcolor{blue!40} \textbf{137} & 139 & 139 & \cellcolor{red!20} n/a & \cellcolor{red!20} n/a & 138 & 139 \\
1000 388& 137 & \cellcolor{blue!40} \textbf{137} & 139 & 138 & \cellcolor{red!20} n/a & \cellcolor{red!20} n/a & 138 & 138 \\
1000 389& 136 & \cellcolor{blue!40} \textbf{136} & 138 & 137 & \cellcolor{red!20} n/a & \cellcolor{red!20} n/a & 137 & 137 \\
1000 39& 520 & \cellcolor{blue!40} 539 & 561 & 575 & \cellcolor{red!20} n/a & \cellcolor{red!20} n/a & 545 & 550 \\
1000 390& 136 & \cellcolor{blue!40} \textbf{136} & 138 & 138 & \cellcolor{red!20} n/a & \cellcolor{red!20} n/a & 137 & 137 \\
1000 391& 135 & \cellcolor{blue!40} \textbf{135} & 137 & 137 & \cellcolor{red!20} n/a & \cellcolor{red!20} n/a & 136 & 137 \\
1000 392& 136 & \cellcolor{blue!40} \textbf{136} & 137 & 137 & \cellcolor{red!20} n/a & \cellcolor{red!20} n/a & 137 & 137 \\
1000 393& 136 & \cellcolor{blue!40} \textbf{136} & 138 & 138 & \cellcolor{red!20} n/a & \cellcolor{red!20} n/a & 137 & 137 \\
1000 394& 138 & \cellcolor{blue!40} \textbf{138} & 140 & 140 & \cellcolor{red!20} n/a & \cellcolor{red!20} n/a & 140 & 140 \\
1000 395& 139 & \cellcolor{blue!40} \textbf{139} & 141 & 141 & \cellcolor{red!20} n/a & \cellcolor{red!20} n/a & 140 & 141 \\
1000 396& 136 & \cellcolor{blue!40} \textbf{136} & 138 & 138 & \cellcolor{red!20} n/a & \cellcolor{red!20} n/a & 137 & 138 \\
1000 397& 140 & \cellcolor{blue!40} \textbf{140} & 142 & 142 & \cellcolor{red!20} n/a & \cellcolor{red!20} n/a & 141 & 141 \\
1000 398& 134 & \cellcolor{blue!40} \textbf{134} & 136 & 136 & \cellcolor{red!20} n/a & \cellcolor{red!20} n/a & 135 & 136 \\
1000 399& 139 & \cellcolor{blue!40} \textbf{139} & 140 & 140 & \cellcolor{red!20} n/a & \cellcolor{red!20} n/a & 140 & 140 \\
1000 4& 138 & \cellcolor{blue!40} \textbf{138} & 139 & 140 & \cellcolor{red!20} n/a & \cellcolor{red!20} n/a & 139 & 139 \\
1000 40& 504 & \cellcolor{blue!40} 520 & 531 & 543 & \cellcolor{red!20} n/a & \cellcolor{red!20} n/a & 521 & 523 \\
1000 400& 140 & \cellcolor{blue!40} \textbf{140} & 142 & 142 & \cellcolor{red!20} n/a & \cellcolor{red!20} n/a & 141 & 142 \\
1000 401& 500 & 547 & 556 & 561 & \cellcolor{red!20} n/a & \cellcolor{red!20} n/a & \cellcolor{blue!40} 543 & \cellcolor{red!20} n/a \\
1000 402& 518 & 565 & 567 & 569 & \cellcolor{red!20} n/a & \cellcolor{red!20} n/a & \cellcolor{blue!40} 555 & \cellcolor{red!20} n/a \\
1000 403& 511 & 557 & 562 & 567 & \cellcolor{red!20} n/a & \cellcolor{red!20} n/a & \cellcolor{blue!40} 553 & \cellcolor{red!20} n/a \\
1000 404& 505 & \cellcolor{blue!20} 550 & 557 & 566 & \cellcolor{red!20} n/a & \cellcolor{red!20} n/a & \cellcolor{blue!20} 550 & \cellcolor{red!20} n/a \\
1000 405& 514 & \cellcolor{blue!40} 557 & 567 & 571 & \cellcolor{red!20} n/a & \cellcolor{red!20} n/a & 563 & \cellcolor{red!20} n/a \\
1000 406& 497 & 542 & 552 & 552 & \cellcolor{red!20} n/a & \cellcolor{red!20} n/a & \cellcolor{blue!40} 538 & \cellcolor{red!20} n/a \\
1000 407& 500 & \cellcolor{blue!40} 548 & 564 & 567 & \cellcolor{red!20} n/a & \cellcolor{red!20} n/a & 551 & \cellcolor{red!20} n/a \\
1000 408& 520 & 567 & 568 & 571 & \cellcolor{red!20} n/a & \cellcolor{red!20} n/a & \cellcolor{blue!40} 558 & \cellcolor{red!20} n/a \\
1000 409& 512 & 559 & 568 & 574 & \cellcolor{red!20} n/a & \cellcolor{red!20} n/a & \cellcolor{blue!40} 551 & \cellcolor{red!20} n/a \\
1000 41& 505 & \cellcolor{blue!40} 525 & 544 & 568 & \cellcolor{red!20} n/a & \cellcolor{red!20} n/a & 529 & 535 \\
1000 410& 523 & \cellcolor{blue!40} 561 & 580 & 593 & \cellcolor{red!20} n/a & \cellcolor{red!20} n/a & 570 & \cellcolor{red!20} n/a \\
1000 411& 504 & \cellcolor{blue!40} 541 & 563 & 569 & \cellcolor{red!20} n/a & \cellcolor{red!20} n/a & 549 & \cellcolor{red!20} n/a \\
1000 412& 509 & \cellcolor{blue!40} 547 & 565 & 571 & \cellcolor{red!20} n/a & \cellcolor{red!20} n/a & 558 & \cellcolor{red!20} n/a \\
1000 413& 515 & \cellcolor{blue!40} 547 & 567 & 568 & \cellcolor{red!20} n/a & \cellcolor{red!20} n/a & 553 & \cellcolor{red!20} n/a \\
1000 414& 505 & 557 & 563 & 566 & \cellcolor{red!20} n/a & \cellcolor{red!20} n/a & \cellcolor{blue!40} 550 & \cellcolor{red!20} n/a \\
1000 415& 509 & 553 & 564 & 571 & \cellcolor{red!20} n/a & \cellcolor{red!20} n/a & \cellcolor{blue!40} 550 & \cellcolor{red!20} n/a \\
1000 416& 520 & 561 & 567 & 578 & \cellcolor{red!20} n/a & \cellcolor{red!20} n/a & \cellcolor{blue!40} 558 & \cellcolor{red!20} n/a \\
1000 417& 550 & 585 & 592 & 594 & \cellcolor{red!20} n/a & \cellcolor{red!20} n/a & \cellcolor{blue!40} 583 & \cellcolor{red!20} n/a \\
1000 418& 507 & \cellcolor{blue!40} 548 & 560 & 566 & \cellcolor{red!20} n/a & \cellcolor{red!20} n/a & 550 & \cellcolor{red!20} n/a \\
1000 419& 537 & \cellcolor{blue!40} 574 & 583 & 592 & \cellcolor{red!20} n/a & \cellcolor{red!20} n/a & 575 & \cellcolor{red!20} n/a \\
1000 42& 499 & 524 & 533 & 549 & \cellcolor{red!20} n/a & \cellcolor{red!20} n/a & \cellcolor{blue!40} 520 & 527 \\
1000 420& 512 & \cellcolor{blue!40} 551 & 565 & 565 & \cellcolor{red!20} n/a & \cellcolor{red!20} n/a & 555 & \cellcolor{red!20} n/a \\
1000 421& 506 & \cellcolor{blue!40} 546 & 561 & 565 & \cellcolor{red!20} n/a & \cellcolor{red!20} n/a & 554 & \cellcolor{red!20} n/a \\
1000 422& 499 & \cellcolor{blue!20} 546 & 560 & 562 & \cellcolor{red!20} n/a & \cellcolor{red!20} n/a & \cellcolor{blue!20} 546 & \cellcolor{red!20} n/a \\
1000 423& 515 & \cellcolor{blue!40} 557 & 565 & 576 & \cellcolor{red!20} n/a & \cellcolor{red!20} n/a & 559 & \cellcolor{red!20} n/a \\
1000 424& 496 & 545 & 553 & 555 & \cellcolor{red!20} n/a & \cellcolor{red!20} n/a & \cellcolor{blue!40} 542 & \cellcolor{red!20} n/a \\
1000 425& 527 & \cellcolor{blue!20} 563 & 572 & 577 & \cellcolor{red!20} n/a & \cellcolor{red!20} n/a & \cellcolor{blue!20} 563 & \cellcolor{red!20} n/a \\
1000 426& 224 & \cellcolor{blue!20} 227 & 229 & 229 & \cellcolor{red!20} n/a & \cellcolor{red!20} n/a & \cellcolor{blue!20} 227 & 228 \\
1000 427& 229 & \cellcolor{blue!40} 232 & 235 & 235 & \cellcolor{red!20} n/a & \cellcolor{red!20} n/a & 233 & 234 \\
1000 428& 224 & \cellcolor{blue!20} 227 & 228 & 228 & \cellcolor{red!20} n/a & \cellcolor{red!20} n/a & \cellcolor{blue!20} 227 & 228 \\
1000 429& 235 & \cellcolor{blue!40} 237 & 240 & 240 & \cellcolor{red!20} n/a & \cellcolor{red!20} n/a & 238 & 239 \\
1000 43& 504 & 533 & 534 & 552 & \cellcolor{red!20} n/a & \cellcolor{red!20} n/a & \cellcolor{blue!40} 524 & 527 \\
1000 430& 220 & \cellcolor{blue!20} 223 & 224 & 224 & \cellcolor{red!20} n/a & \cellcolor{red!20} n/a & \cellcolor{blue!20} 223 & 224 \\
1000 431& 230 & \cellcolor{blue!40} 232 & 235 & 235 & \cellcolor{red!20} n/a & \cellcolor{red!20} n/a & 233 & 234 \\
1000 432& 227 & \cellcolor{blue!40} 230 & 232 & 232 & \cellcolor{red!20} n/a & \cellcolor{red!20} n/a & 231 & 232 \\
1000 433& 229 & \cellcolor{blue!10} 233 & 234 & 234 & \cellcolor{red!20} n/a & \cellcolor{red!20} n/a & \cellcolor{blue!10} 233 & \cellcolor{blue!10} 233 \\
1000 434& 212 & \cellcolor{blue!20} 214 & 215 & 215 & \cellcolor{red!20} n/a & \cellcolor{red!20} n/a & \cellcolor{blue!20} 214 & 215 \\
1000 435& 227 & \cellcolor{blue!20} 230 & 232 & 232 & \cellcolor{red!20} n/a & \cellcolor{red!20} n/a & \cellcolor{blue!20} 230 & 231 \\
1000 436& 226 & \cellcolor{blue!20} 230 & 231 & 231 & \cellcolor{red!20} n/a & \cellcolor{red!20} n/a & \cellcolor{blue!20} 230 & 231 \\
1000 437& 222 & \cellcolor{blue!10} 225 & 227 & 227 & \cellcolor{red!20} n/a & \cellcolor{red!20} n/a & \cellcolor{blue!10} 225 & \cellcolor{blue!10} 225 \\
1000 438& 221 & \cellcolor{blue!20} 224 & 226 & 226 & \cellcolor{red!20} n/a & \cellcolor{red!20} n/a & \cellcolor{blue!20} 224 & 225 \\
1000 439& 225 & \cellcolor{blue!20} 228 & 230 & 230 & \cellcolor{red!20} n/a & \cellcolor{red!20} n/a & \cellcolor{blue!20} 228 & 229 \\
1000 44& 508 & \cellcolor{blue!40} 525 & 552 & 565 & \cellcolor{red!20} n/a & \cellcolor{red!20} n/a & 543 & 548 \\
1000 440& 225 & \cellcolor{blue!40} \textbf{225} & 230 & 230 & \cellcolor{red!20} n/a & \cellcolor{red!20} n/a & 228 & 229 \\
1000 441& 221 & \cellcolor{blue!10} 225 & 226 & 226 & \cellcolor{red!20} n/a & \cellcolor{red!20} n/a & \cellcolor{blue!10} 225 & \cellcolor{blue!10} 225 \\
1000 442& 230 & 234 & 235 & 235 & \cellcolor{red!20} n/a & \cellcolor{red!20} n/a & \cellcolor{blue!40} 233 & 235 \\
1000 443& 217 & \cellcolor{blue!10} 220 & 222 & 222 & \cellcolor{red!20} n/a & \cellcolor{red!20} n/a & \cellcolor{blue!10} 220 & \cellcolor{blue!10} 220 \\
1000 444& 222 & \cellcolor{blue!20} 225 & 227 & 226 & \cellcolor{red!20} n/a & \cellcolor{red!20} n/a & \cellcolor{blue!20} 225 & 226 \\
1000 445& 229 & \cellcolor{blue!20} 233 & 235 & 236 & \cellcolor{red!20} n/a & \cellcolor{red!20} n/a & \cellcolor{blue!20} 233 & 234 \\
1000 446& 228 & 231 & 233 & 232 & \cellcolor{red!20} n/a & \cellcolor{red!20} n/a & \cellcolor{blue!40} 230 & 232 \\
1000 447& 221 & \cellcolor{blue!10} 225 & 227 & 227 & \cellcolor{red!20} n/a & \cellcolor{red!20} n/a & \cellcolor{blue!10} 225 & \cellcolor{blue!10} 225 \\
1000 448& 222 & \cellcolor{blue!40} \textbf{222} & 226 & 226 & \cellcolor{red!20} n/a & \cellcolor{red!20} n/a & 225 & 226 \\
1000 449& 232 & \cellcolor{blue!20} 236 & 238 & 238 & \cellcolor{red!20} n/a & \cellcolor{red!20} n/a & \cellcolor{blue!20} 236 & 237 \\
1000 45& 493 & \cellcolor{blue!40} 512 & 524 & 541 & \cellcolor{red!20} n/a & \cellcolor{red!20} n/a & 513 & 517 \\
1000 450& 220 & \cellcolor{blue!40} \textbf{220} & 225 & 224 & \cellcolor{red!20} n/a & \cellcolor{red!20} n/a & 223 & 224 \\
1000 451& 136 & \cellcolor{blue!40} \textbf{136} & 140 & 139 & \cellcolor{red!20} n/a & \cellcolor{red!20} n/a & 139 & 139 \\
1000 452& 132 & \cellcolor{blue!40} 133 & 135 & 135 & \cellcolor{red!20} n/a & \cellcolor{red!20} n/a & 134 & 134 \\
1000 453& 138 & \cellcolor{blue!40} \textbf{138} & 141 & 141 & \cellcolor{red!20} n/a & \cellcolor{red!20} n/a & 140 & 141 \\
1000 454& 139 & \cellcolor{blue!40} 140 & 142 & 142 & \cellcolor{red!20} n/a & \cellcolor{red!20} n/a & 141 & 142 \\
1000 455& 136 & \cellcolor{blue!40} 137 & 140 & 139 & \cellcolor{red!20} n/a & \cellcolor{red!20} n/a & 138 & 139 \\
1000 456& 135 & \cellcolor{blue!40} \textbf{135} & 138 & 137 & \cellcolor{red!20} n/a & \cellcolor{red!20} n/a & 137 & 137 \\
1000 457& 137 & \cellcolor{blue!40} \textbf{137} & 140 & 140 & \cellcolor{red!20} n/a & \cellcolor{red!20} n/a & 139 & 139 \\
1000 458& 135 & \cellcolor{blue!40} \textbf{135} & 138 & 137 & \cellcolor{red!20} n/a & \cellcolor{red!20} n/a & 136 & 137 \\
1000 459& 137 & \cellcolor{blue!40} 138 & 140 & 140 & \cellcolor{red!20} n/a & \cellcolor{red!20} n/a & 139 & 140 \\
1000 46& 499 & \cellcolor{blue!40} 521 & 538 & 563 & \cellcolor{red!20} n/a & \cellcolor{red!20} n/a & 527 & 533 \\
1000 460& 138 & \cellcolor{blue!40} \textbf{138} & 141 & 140 & \cellcolor{red!20} n/a & \cellcolor{red!20} n/a & 140 & 140 \\
1000 461& 137 & \cellcolor{blue!40} \textbf{137} & 140 & 139 & \cellcolor{red!20} n/a & \cellcolor{red!20} n/a & 139 & 139 \\
1000 462& 136 & \cellcolor{blue!40} \textbf{136} & 139 & 139 & \cellcolor{red!20} n/a & \cellcolor{red!20} n/a & 138 & 138 \\
1000 463& 136 & \cellcolor{blue!40} \textbf{136} & 138 & 138 & \cellcolor{red!20} n/a & \cellcolor{red!20} n/a & 138 & 138 \\
1000 464& 138 & \cellcolor{blue!40} 139 & 141 & 141 & \cellcolor{red!20} n/a & \cellcolor{red!20} n/a & 140 & 141 \\
1000 465& 138 & \cellcolor{blue!40} \textbf{138} & 141 & 141 & \cellcolor{red!20} n/a & \cellcolor{red!20} n/a & 140 & 141 \\
1000 466& 133 & \cellcolor{blue!40} \textbf{133} & 137 & 136 & \cellcolor{red!20} n/a & \cellcolor{red!20} n/a & 135 & 136 \\
1000 467& 138 & \cellcolor{blue!40} \textbf{138} & 140 & 140 & \cellcolor{red!20} n/a & \cellcolor{red!20} n/a & 140 & 140 \\
1000 468& 137 & \cellcolor{blue!20} 138 & 140 & 139 & \cellcolor{red!20} n/a & \cellcolor{red!20} n/a & \cellcolor{blue!20} 138 & 140 \\
1000 469& 137 & \cellcolor{blue!40} \textbf{137} & 140 & 140 & \cellcolor{red!20} n/a & \cellcolor{red!20} n/a & 139 & 140 \\
1000 47& 507 & 528 & 542 & 553 & \cellcolor{red!20} n/a & \cellcolor{red!20} n/a & \cellcolor{blue!40} 527 & 532 \\
1000 470& 135 & \cellcolor{blue!40} \textbf{135} & 138 & 138 & \cellcolor{red!20} n/a & \cellcolor{red!20} n/a & 136 & 137 \\
1000 471& 135 & \cellcolor{blue!40} \textbf{135} & 138 & 138 & \cellcolor{red!20} n/a & \cellcolor{red!20} n/a & 137 & 138 \\
1000 472& 140 & \cellcolor{blue!40} \textbf{140} & 143 & 143 & \cellcolor{red!20} n/a & \cellcolor{red!20} n/a & 142 & 143 \\
1000 473& 135 & \cellcolor{blue!40} \textbf{135} & 138 & 138 & \cellcolor{red!20} n/a & \cellcolor{red!20} n/a & 137 & 137 \\
1000 474& 136 & \cellcolor{blue!40} 137 & 139 & 139 & \cellcolor{red!20} n/a & \cellcolor{red!20} n/a & 138 & 139 \\
1000 475& 136 & \cellcolor{blue!40} \textbf{136} & 139 & 139 & \cellcolor{red!20} n/a & \cellcolor{red!20} n/a & 138 & 139 \\
1000 476& 519 & 585 & 588 & 590 & \cellcolor{red!20} n/a & \cellcolor{red!20} n/a & \cellcolor{blue!40} 581 & \cellcolor{red!20} n/a \\
1000 477& 526 & 597 & 594 & 596 & \cellcolor{red!20} n/a & \cellcolor{red!20} n/a & \cellcolor{blue!40} 586 & \cellcolor{red!20} n/a \\
1000 478& 545 & 607 & \cellcolor{blue!40} 604 & 605 & \cellcolor{red!20} n/a & \cellcolor{red!20} n/a & \cellcolor{red!20} n/a & \cellcolor{red!20} n/a \\
1000 479& 513 & 588 & 590 & 587 & \cellcolor{red!20} n/a & \cellcolor{red!20} n/a & \cellcolor{blue!40} 577 & \cellcolor{red!20} n/a \\
1000 48& 530 & \cellcolor{blue!40} 549 & 565 & 584 & \cellcolor{red!20} n/a & \cellcolor{red!20} n/a & 559 & 562 \\
1000 480& 508 & 582 & 578 & 574 & \cellcolor{red!20} n/a & \cellcolor{red!20} n/a & \cellcolor{blue!40} 568 & \cellcolor{red!20} n/a \\
1000 481& 519 & 594 & 585 & 587 & \cellcolor{red!20} n/a & \cellcolor{red!20} n/a & \cellcolor{blue!40} 582 & \cellcolor{red!20} n/a \\
1000 482& 541 & \cellcolor{blue!40} 604 & 605 & 610 & \cellcolor{red!20} n/a & \cellcolor{red!20} n/a & \cellcolor{red!20} n/a & \cellcolor{red!20} n/a \\
1000 483& 506 & 588 & 577 & 582 & \cellcolor{red!20} n/a & \cellcolor{red!20} n/a & \cellcolor{blue!40} 566 & \cellcolor{red!20} n/a \\
1000 484& 535 & 598 & 596 & 598 & \cellcolor{red!20} n/a & \cellcolor{red!20} n/a & \cellcolor{blue!40} 594 & \cellcolor{red!20} n/a \\
1000 485& 521 & 594 & 591 & 596 & \cellcolor{red!20} n/a & \cellcolor{red!20} n/a & \cellcolor{blue!40} 582 & \cellcolor{red!20} n/a \\
1000 486& 510 & 582 & 583 & 584 & \cellcolor{red!20} n/a & \cellcolor{red!20} n/a & \cellcolor{blue!40} 574 & \cellcolor{red!20} n/a \\
1000 487& 524 & 597 & 592 & 597 & \cellcolor{red!20} n/a & \cellcolor{red!20} n/a & \cellcolor{blue!40} 587 & \cellcolor{red!20} n/a \\
1000 488& 511 & 581 & 581 & 582 & \cellcolor{red!20} n/a & \cellcolor{red!20} n/a & \cellcolor{blue!40} 576 & \cellcolor{red!20} n/a \\
1000 489& 502 & 578 & 577 & 577 & \cellcolor{red!20} n/a & \cellcolor{red!20} n/a & \cellcolor{blue!40} 568 & \cellcolor{red!20} n/a \\
1000 49& 507 & \cellcolor{blue!40} 525 & 544 & 559 & \cellcolor{red!20} n/a & \cellcolor{red!20} n/a & 530 & 538 \\
1000 490& 514 & 594 & 587 & 583 & \cellcolor{red!20} n/a & \cellcolor{red!20} n/a & \cellcolor{blue!40} 576 & \cellcolor{red!20} n/a \\
1000 491& 507 & 585 & 582 & 584 & \cellcolor{red!20} n/a & \cellcolor{red!20} n/a & \cellcolor{blue!40} 571 & \cellcolor{red!20} n/a \\
1000 492& 527 & 606 & 596 & 604 & \cellcolor{red!20} n/a & \cellcolor{red!20} n/a & \cellcolor{blue!40} 591 & \cellcolor{red!20} n/a \\
1000 493& 498 & 571 & 568 & 573 & \cellcolor{red!20} n/a & \cellcolor{red!20} n/a & \cellcolor{blue!40} 561 & \cellcolor{red!20} n/a \\
1000 494& 514 & 583 & 580 & 587 & \cellcolor{red!20} n/a & \cellcolor{red!20} n/a & \cellcolor{blue!40} 575 & \cellcolor{red!20} n/a \\
1000 495& 530 & 606 & 603 & 609 & \cellcolor{red!20} n/a & \cellcolor{red!20} n/a & \cellcolor{blue!40} 596 & \cellcolor{red!20} n/a \\
1000 496& 505 & 569 & 571 & 572 & \cellcolor{red!20} n/a & \cellcolor{red!20} n/a & \cellcolor{blue!40} 565 & \cellcolor{red!20} n/a \\
1000 497& 504 & 580 & 578 & 574 & \cellcolor{red!20} n/a & \cellcolor{red!20} n/a & \cellcolor{blue!40} 568 & \cellcolor{red!20} n/a \\
1000 498& 523 & 593 & 594 & 592 & \cellcolor{red!20} n/a & \cellcolor{red!20} n/a & \cellcolor{blue!40} 587 & \cellcolor{red!20} n/a \\
1000 499& 505 & 579 & 580 & 579 & \cellcolor{red!20} n/a & \cellcolor{red!20} n/a & \cellcolor{blue!40} 570 & \cellcolor{red!20} n/a \\
1000 5& 135 & \cellcolor{blue!40} \textbf{135} & 136 & 136 & \cellcolor{red!20} n/a & \cellcolor{red!20} n/a & 136 & 136 \\
1000 50& 493 & \cellcolor{blue!40} 513 & 528 & 548 & \cellcolor{red!20} n/a & \cellcolor{red!20} n/a & 515 & 519 \\
1000 500& 512 & 591 & 586 & 593 & \cellcolor{red!20} n/a & \cellcolor{red!20} n/a & \cellcolor{blue!40} 575 & \cellcolor{red!20} n/a \\
1000 501& 227 & 234 & 237 & 235 & \cellcolor{red!20} n/a & \cellcolor{red!20} n/a & \cellcolor{blue!40} 232 & 234 \\
1000 502& 224 & \cellcolor{blue!10} 230 & 232 & 231 & \cellcolor{red!20} n/a & \cellcolor{red!20} n/a & \cellcolor{blue!10} 230 & \cellcolor{blue!10} 230 \\
1000 503& 224 & \cellcolor{blue!10} 231 & 233 & 233 & \cellcolor{red!20} n/a & \cellcolor{red!20} n/a & \cellcolor{blue!10} 231 & \cellcolor{blue!10} 231 \\
1000 504& 227 & 234 & 236 & 235 & \cellcolor{red!20} n/a & \cellcolor{red!20} n/a & \cellcolor{blue!40} 233 & 235 \\
1000 505& 213 & \cellcolor{blue!20} 218 & 222 & 221 & \cellcolor{red!20} n/a & \cellcolor{red!20} n/a & \cellcolor{blue!20} 218 & 220 \\
1000 506& 223 & 230 & 230 & 230 & \cellcolor{red!20} n/a & \cellcolor{red!20} n/a & \cellcolor{blue!40} 229 & 230 \\
1000 507& 220 & 227 & 230 & 228 & \cellcolor{red!20} n/a & \cellcolor{red!20} n/a & \cellcolor{blue!40} 226 & 228 \\
1000 508& 219 & 225 & 227 & 226 & \cellcolor{red!20} n/a & \cellcolor{red!20} n/a & \cellcolor{blue!40} 223 & 225 \\
1000 509& 225 & 232 & 233 & 232 & \cellcolor{red!20} n/a & \cellcolor{red!20} n/a & \cellcolor{blue!40} 230 & 231 \\
1000 51& 226 & \cellcolor{blue!20} 228 & 231 & 231 & \cellcolor{red!20} n/a & \cellcolor{red!20} n/a & \cellcolor{blue!20} 228 & 229 \\
1000 510& 226 & 234 & 235 & 234 & \cellcolor{red!20} n/a & \cellcolor{red!20} n/a & \cellcolor{blue!40} 232 & 234 \\
1000 511& 230 & 237 & 238 & 239 & \cellcolor{red!20} n/a & \cellcolor{red!20} n/a & \cellcolor{blue!40} 236 & 238 \\
1000 512& 219 & 227 & 227 & 226 & \cellcolor{red!20} n/a & \cellcolor{red!20} n/a & \cellcolor{blue!40} 224 & 226 \\
1000 513& 219 & 226 & 228 & 226 & \cellcolor{red!20} n/a & \cellcolor{red!20} n/a & \cellcolor{blue!40} 225 & 226 \\
1000 514& 226 & \cellcolor{blue!40} 232 & 234 & 234 & \cellcolor{red!20} n/a & \cellcolor{red!20} n/a & 233 & 234 \\
1000 515& 221 & 227 & 229 & 229 & \cellcolor{red!20} n/a & \cellcolor{red!20} n/a & \cellcolor{blue!40} 226 & 227 \\
1000 516& 229 & 236 & 238 & 237 & \cellcolor{red!20} n/a & \cellcolor{red!20} n/a & \cellcolor{blue!40} 235 & 237 \\
1000 517& 221 & 228 & 229 & 228 & \cellcolor{red!20} n/a & \cellcolor{red!20} n/a & \cellcolor{blue!40} 227 & 228 \\
1000 518& 220 & 226 & 229 & 227 & \cellcolor{red!20} n/a & \cellcolor{red!20} n/a & \cellcolor{blue!40} 225 & 226 \\
1000 519& 221 & \cellcolor{blue!40} 226 & 229 & 229 & \cellcolor{red!20} n/a & \cellcolor{red!20} n/a & 227 & 232 \\
1000 52& 228 & \cellcolor{blue!40} 229 & 232 & 233 & \cellcolor{red!20} n/a & \cellcolor{red!20} n/a & 230 & 230 \\
1000 520& 226 & \cellcolor{blue!20} 231 & 235 & 232 & \cellcolor{red!20} n/a & \cellcolor{red!20} n/a & \cellcolor{blue!20} 231 & 232 \\
1000 521& 229 & 236 & 239 & 238 & \cellcolor{red!20} n/a & \cellcolor{red!20} n/a & \cellcolor{blue!40} 235 & 236 \\
1000 522& 215 & 221 & 222 & 222 & \cellcolor{red!20} n/a & \cellcolor{red!20} n/a & \cellcolor{blue!40} 220 & 222 \\
1000 523& 220 & 226 & 228 & 228 & \cellcolor{red!20} n/a & \cellcolor{red!20} n/a & \cellcolor{blue!40} 225 & 227 \\
1000 524& 225 & 233 & 233 & 234 & \cellcolor{red!20} n/a & \cellcolor{red!20} n/a & \cellcolor{blue!40} 232 & 233 \\
1000 525& 221 & 227 & 230 & 228 & \cellcolor{red!20} n/a & \cellcolor{red!20} n/a & \cellcolor{blue!40} 226 & 229 \\
1000 53& 227 & \cellcolor{blue!10} 229 & 230 & 231 & \cellcolor{red!20} n/a & \cellcolor{red!20} n/a & \cellcolor{blue!10} 229 & \cellcolor{blue!10} 229 \\
1000 54& 219 & \cellcolor{blue!20} 221 & 223 & 224 & \cellcolor{red!20} n/a & \cellcolor{red!20} n/a & \cellcolor{blue!20} 221 & 222 \\
1000 55& 217 & \cellcolor{blue!40} 218 & 220 & 222 & \cellcolor{red!20} n/a & \cellcolor{red!20} n/a & 219 & 219 \\
1000 56& 228 & \cellcolor{blue!40} 229 & 232 & 232 & \cellcolor{red!20} n/a & \cellcolor{red!20} n/a & 230 & 230 \\
1000 57& 224 & \cellcolor{blue!10} 226 & 227 & 228 & \cellcolor{red!20} n/a & \cellcolor{red!20} n/a & \cellcolor{blue!10} 226 & \cellcolor{blue!10} 226 \\
1000 58& 224 & \cellcolor{blue!20} 225 & 227 & 227 & \cellcolor{red!20} n/a & \cellcolor{red!20} n/a & \cellcolor{blue!20} 225 & 226 \\
1000 59& 223 & \cellcolor{blue!40} 224 & 226 & 226 & \cellcolor{red!20} n/a & \cellcolor{red!20} n/a & 225 & 225 \\
1000 6& 141 & \cellcolor{blue!40} \textbf{141} & 143 & 143 & \cellcolor{red!20} n/a & \cellcolor{red!20} n/a & 142 & 143 \\
1000 60& 230 & \cellcolor{blue!10} 232 & 234 & 234 & \cellcolor{red!20} n/a & \cellcolor{red!20} n/a & \cellcolor{blue!10} 232 & \cellcolor{blue!10} 232 \\
1000 61& 229 & \cellcolor{blue!20} 231 & 233 & 234 & \cellcolor{red!20} n/a & \cellcolor{red!20} n/a & \cellcolor{blue!20} 231 & 232 \\
1000 62& 223 & \cellcolor{blue!10} 225 & 227 & 228 & \cellcolor{red!20} n/a & \cellcolor{red!20} n/a & \cellcolor{blue!10} 225 & \cellcolor{blue!10} 225 \\
1000 63& 227 & 229 & 230 & 230 & \cellcolor{red!20} n/a & \cellcolor{red!20} n/a & \cellcolor{blue!40} 228 & 229 \\
1000 64& 229 & \cellcolor{blue!40} \textbf{229} & 233 & 234 & \cellcolor{red!20} n/a & \cellcolor{red!20} n/a & 231 & 231 \\
1000 65& 225 & \cellcolor{blue!20} 226 & 227 & 229 & \cellcolor{red!20} n/a & \cellcolor{red!20} n/a & \cellcolor{blue!20} 226 & 227 \\
1000 66& 227 & \cellcolor{blue!40} \textbf{227} & 230 & 231 & \cellcolor{red!20} n/a & \cellcolor{red!20} n/a & 229 & 229 \\
1000 67& 223 & \cellcolor{blue!20} 224 & 227 & 227 & \cellcolor{red!20} n/a & \cellcolor{red!20} n/a & \cellcolor{blue!20} 224 & 225 \\
1000 68& 226 & \cellcolor{blue!10} 228 & 231 & 231 & \cellcolor{red!20} n/a & \cellcolor{red!20} n/a & \cellcolor{blue!10} 228 & \cellcolor{blue!10} 228 \\
1000 69& 224 & \cellcolor{blue!40} \textbf{224} & 227 & 227 & \cellcolor{red!20} n/a & \cellcolor{red!20} n/a & 226 & 226 \\
1000 7& 136 & \cellcolor{blue!40} \textbf{136} & 138 & 138 & \cellcolor{red!20} n/a & \cellcolor{red!20} n/a & 137 & 137 \\
1000 70& 228 & \cellcolor{blue!10} 230 & 231 & 232 & \cellcolor{red!20} n/a & \cellcolor{red!20} n/a & \cellcolor{blue!10} 230 & \cellcolor{blue!10} 230 \\
1000 71& 230 & \cellcolor{blue!10} 232 & 233 & 234 & \cellcolor{red!20} n/a & \cellcolor{red!20} n/a & \cellcolor{blue!10} 232 & \cellcolor{blue!10} 232 \\
1000 72& 222 & \cellcolor{blue!40} 223 & 226 & 226 & \cellcolor{red!20} n/a & \cellcolor{red!20} n/a & 224 & 224 \\
1000 73& 221 & \cellcolor{blue!10} 223 & 224 & 225 & \cellcolor{red!20} n/a & \cellcolor{red!20} n/a & \cellcolor{blue!10} 223 & \cellcolor{blue!10} 223 \\
1000 74& 227 & \cellcolor{blue!10} 229 & 231 & 231 & \cellcolor{red!20} n/a & \cellcolor{red!20} n/a & \cellcolor{blue!10} 229 & \cellcolor{blue!10} 229 \\
1000 75& 227 & \cellcolor{blue!10} 229 & 231 & 231 & \cellcolor{red!20} n/a & \cellcolor{red!20} n/a & \cellcolor{blue!10} 229 & \cellcolor{blue!10} 229 \\
1000 76& 136 & \cellcolor{blue!40} \textbf{136} & 137 & 137 & \cellcolor{red!20} n/a & \cellcolor{red!20} n/a & 137 & \cellcolor{red!20} n/a \\
1000 77& 136 & \cellcolor{blue!40} \textbf{136} & 137 & 137 & \cellcolor{red!20} n/a & \cellcolor{red!20} n/a & 137 & 137 \\
1000 78& 138 & \cellcolor{blue!40} \textbf{138} & 140 & 140 & \cellcolor{red!20} n/a & \cellcolor{red!20} n/a & 140 & \cellcolor{red!20} n/a \\
1000 79& 142 & \cellcolor{blue!40} \textbf{142} & 143 & 143 & \cellcolor{red!20} n/a & \cellcolor{red!20} n/a & 143 & 143 \\
1000 8& 138 & \cellcolor{blue!40} \textbf{138} & 140 & 140 & \cellcolor{red!20} n/a & \cellcolor{red!20} n/a & 139 & 139 \\
1000 80& 140 & \cellcolor{blue!40} \textbf{140} & 141 & 142 & \cellcolor{red!20} n/a & \cellcolor{red!20} n/a & 141 & 141 \\
1000 81& 136 & \cellcolor{blue!40} \textbf{136} & 138 & 138 & \cellcolor{red!20} n/a & \cellcolor{red!20} n/a & 137 & 137 \\
1000 82& 136 & \cellcolor{blue!40} \textbf{136} & 137 & 137 & \cellcolor{red!20} n/a & \cellcolor{red!20} n/a & 137 & 137 \\
1000 83& 140 & \cellcolor{blue!20} \textbf{140} & 141 & 141 & \cellcolor{red!20} n/a & \cellcolor{red!20} n/a & \cellcolor{blue!20} 140 & 141 \\
1000 84& 135 & \cellcolor{blue!40} \textbf{135} & 136 & 136 & \cellcolor{red!20} n/a & \cellcolor{red!20} n/a & 136 & 136 \\
1000 85& 136 & \cellcolor{blue!40} \textbf{136} & 138 & 137 & \cellcolor{red!20} n/a & \cellcolor{red!20} n/a & 137 & 137 \\
1000 86& 138 & \cellcolor{blue!40} \textbf{138} & 139 & 139 & \cellcolor{red!20} n/a & \cellcolor{red!20} n/a & 139 & 140 \\
1000 87& 140 & \cellcolor{blue!40} \textbf{140} & 142 & 142 & \cellcolor{red!20} n/a & \cellcolor{red!20} n/a & 141 & \cellcolor{red!20} n/a \\
1000 88& 140 & \cellcolor{blue!40} \textbf{140} & 142 & 142 & \cellcolor{red!20} n/a & \cellcolor{red!20} n/a & 141 & 141 \\
1000 89& 140 & \cellcolor{blue!40} \textbf{140} & 142 & 142 & \cellcolor{red!20} n/a & \cellcolor{red!20} n/a & 141 & 141 \\
1000 9& 134 & \cellcolor{blue!40} \textbf{134} & 136 & 136 & \cellcolor{red!20} n/a & \cellcolor{red!20} n/a & 135 & 135 \\
1000 90& 138 & \cellcolor{blue!20} \textbf{138} & 139 & 139 & \cellcolor{red!20} n/a & \cellcolor{red!20} n/a & \cellcolor{blue!20} 138 & \cellcolor{red!20} n/a \\
1000 91& 141 & \cellcolor{blue!40} \textbf{141} & 142 & 142 & \cellcolor{red!20} n/a & \cellcolor{red!20} n/a & 142 & \cellcolor{red!20} n/a \\
1000 92& 136 & \cellcolor{blue!40} \textbf{136} & 137 & 137 & \cellcolor{red!20} n/a & \cellcolor{red!20} n/a & 137 & \cellcolor{red!20} n/a \\
1000 93& 137 & \cellcolor{blue!40} \textbf{137} & 138 & 138 & \cellcolor{red!20} n/a & \cellcolor{red!20} n/a & 138 & \cellcolor{red!20} n/a \\
1000 94& 137 & \cellcolor{blue!40} \textbf{137} & 139 & 139 & \cellcolor{red!20} n/a & \cellcolor{red!20} n/a & 138 & \cellcolor{red!20} n/a \\
1000 95& 136 & \cellcolor{blue!40} \textbf{136} & 137 & 137 & \cellcolor{red!20} n/a & \cellcolor{red!20} n/a & 137 & \cellcolor{red!20} n/a \\
1000 96& 137 & \cellcolor{blue!40} \textbf{137} & 139 & 139 & \cellcolor{red!20} n/a & \cellcolor{red!20} n/a & 138 & 138 \\
1000 97& 138 & \cellcolor{blue!40} \textbf{138} & 140 & 140 & \cellcolor{red!20} n/a & \cellcolor{red!20} n/a & 139 & 139 \\
1000 98& 136 & \cellcolor{blue!20} \textbf{136} & 137 & 137 & \cellcolor{red!20} n/a & \cellcolor{red!20} n/a & \cellcolor{blue!20} 136 & \cellcolor{red!20} n/a \\
1000 99& 136 & \cellcolor{blue!40} \textbf{136} & 137 & 137 & \cellcolor{red!20} n/a & \cellcolor{red!20} n/a & 137 & \cellcolor{red!20} n/a \\
100 1& 23 & \cellcolor{blue!10} \textbf{23} & 24 & \cellcolor{blue!10} \textbf{23} & 24 & 78 & \cellcolor{blue!10} 23 & 24 \\
100 10& 22 & \cellcolor{blue!5} \textbf{22} & \cellcolor{blue!5} \textbf{22} & \cellcolor{blue!5} \textbf{22} & \cellcolor{red!20} n/a & 56 & \cellcolor{blue!5} 22 & \cellcolor{blue!5} 22 \\
100 100& 25 & \cellcolor{blue!5} \textbf{25} & \cellcolor{blue!5} \textbf{25} & \cellcolor{blue!5} \textbf{25} & \cellcolor{red!20} n/a & 65 & \cellcolor{blue!5} 25 & \cellcolor{blue!5} 25 \\
100 101& 15 & \cellcolor{blue!5} \textbf{15} & \cellcolor{blue!5} \textbf{15} & \cellcolor{blue!5} \textbf{15} & \cellcolor{red!20} n/a & 70 & \cellcolor{blue!5} 15 & \cellcolor{blue!5} 15 \\
100 102& 14 & \cellcolor{blue!5} \textbf{14} & \cellcolor{blue!5} \textbf{14} & \cellcolor{blue!5} \textbf{14} & \cellcolor{red!20} n/a & 15 & \cellcolor{blue!5} \textbf{14} & \cellcolor{blue!5} 14 \\
100 103& 14 & \cellcolor{blue!5} \textbf{14} & \cellcolor{blue!5} \textbf{14} & \cellcolor{blue!5} \textbf{14} & \cellcolor{red!20} n/a & \cellcolor{blue!5} 14 & \cellcolor{blue!5} \textbf{14} & \cellcolor{blue!5} 14 \\
100 104& 14 & \cellcolor{blue!5} \textbf{14} & \cellcolor{blue!5} \textbf{14} & \cellcolor{blue!5} \textbf{14} & \cellcolor{red!20} n/a & 83 & \cellcolor{blue!5} 14 & \cellcolor{blue!5} 14 \\
100 105& 13 & \cellcolor{blue!5} \textbf{13} & \cellcolor{blue!5} \textbf{13} & \cellcolor{blue!5} \textbf{13} & \cellcolor{red!20} n/a & \cellcolor{blue!5} 13 & \cellcolor{blue!5} 13 & \cellcolor{blue!5} 13 \\
100 106& 14 & \cellcolor{blue!5} \textbf{14} & \cellcolor{blue!5} \textbf{14} & \cellcolor{blue!5} \textbf{14} & 41 & \cellcolor{blue!5} 14 & \cellcolor{blue!5} \textbf{14} & \cellcolor{blue!5} \textbf{14} \\
100 107& 14 & \cellcolor{blue!5} \textbf{14} & \cellcolor{blue!5} \textbf{14} & \cellcolor{blue!5} \textbf{14} & \cellcolor{red!20} n/a & \cellcolor{blue!5} 14 & \cellcolor{blue!5} 14 & \cellcolor{blue!5} 14 \\
100 108& 14 & \cellcolor{blue!10} \textbf{14} & \cellcolor{blue!10} \textbf{14} & \cellcolor{blue!10} \textbf{14} & \cellcolor{red!20} n/a & 15 & 15 & 15 \\
100 109& 15 & \cellcolor{blue!5} \textbf{15} & \cellcolor{blue!5} \textbf{15} & \cellcolor{blue!5} \textbf{15} & \cellcolor{red!20} n/a & 92 & \cellcolor{blue!5} 15 & \cellcolor{blue!5} 15 \\
100 11& 24 & \cellcolor{blue!5} \textbf{24} & \cellcolor{blue!5} \textbf{24} & \cellcolor{blue!5} \textbf{24} & 53 & 88 & \cellcolor{blue!5} 24 & \cellcolor{blue!5} 24 \\
100 110& 13 & \cellcolor{blue!5} \textbf{13} & \cellcolor{blue!5} \textbf{13} & \cellcolor{blue!5} \textbf{13} & 45 & 84 & \cellcolor{blue!5} 13 & \cellcolor{blue!5} 13 \\
100 111& 16 & \cellcolor{blue!5} \textbf{16} & \cellcolor{blue!5} \textbf{16} & \cellcolor{blue!5} \textbf{16} & \cellcolor{red!20} n/a & 98 & \cellcolor{blue!5} 16 & \cellcolor{blue!5} 16 \\
100 112& 13 & \cellcolor{blue!5} \textbf{13} & \cellcolor{blue!5} \textbf{13} & \cellcolor{blue!5} \textbf{13} & \cellcolor{red!20} n/a & 14 & \cellcolor{blue!5} 13 & \cellcolor{blue!5} 13 \\
100 113& 14 & \cellcolor{blue!5} \textbf{14} & \cellcolor{blue!5} \textbf{14} & \cellcolor{blue!5} \textbf{14} & 34 & 49 & \cellcolor{blue!5} \textbf{14} & \cellcolor{blue!5} 14 \\
100 114& 13 & \cellcolor{blue!5} \textbf{13} & \cellcolor{blue!5} \textbf{13} & \cellcolor{blue!5} \textbf{13} & \cellcolor{red!20} n/a & 14 & \cellcolor{blue!5} 13 & \cellcolor{blue!5} 13 \\
100 115& 14 & \cellcolor{blue!5} \textbf{14} & \cellcolor{blue!5} \textbf{14} & \cellcolor{blue!5} \textbf{14} & 20 & 17 & \cellcolor{blue!5} 14 & \cellcolor{blue!5} 14 \\
100 116& 16 & \cellcolor{blue!5} \textbf{16} & \cellcolor{blue!5} \textbf{16} & \cellcolor{blue!5} \textbf{16} & \cellcolor{red!20} n/a & 71 & \cellcolor{blue!5} 16 & \cellcolor{blue!5} 16 \\
100 117& 15 & \cellcolor{blue!5} \textbf{15} & \cellcolor{blue!5} \textbf{15} & \cellcolor{blue!5} \textbf{15} & \cellcolor{red!20} n/a & 78 & \cellcolor{blue!5} 15 & 16 \\
100 118& 15 & \cellcolor{blue!5} \textbf{15} & \cellcolor{blue!5} \textbf{15} & \cellcolor{blue!5} \textbf{15} & 43 & \cellcolor{blue!5} 15 & \cellcolor{blue!5} 15 & \cellcolor{blue!5} 15 \\
100 119& 14 & \cellcolor{blue!5} \textbf{14} & \cellcolor{blue!5} \textbf{14} & \cellcolor{blue!5} \textbf{14} & \cellcolor{red!20} n/a & 90 & \cellcolor{blue!5} 14 & \cellcolor{blue!5} 14 \\
100 12& 25 & \cellcolor{blue!5} \textbf{25} & \cellcolor{blue!5} \textbf{25} & \cellcolor{blue!5} \textbf{25} & 27 & 79 & \cellcolor{blue!5} 25 & \cellcolor{blue!5} 25 \\
100 120& 14 & \cellcolor{blue!5} \textbf{14} & \cellcolor{blue!5} \textbf{14} & \cellcolor{blue!5} \textbf{14} & \cellcolor{red!20} n/a & \cellcolor{blue!5} 14 & \cellcolor{blue!5} 14 & \cellcolor{blue!5} 14 \\
100 121& 15 & \cellcolor{blue!5} \textbf{15} & \cellcolor{blue!5} \textbf{15} & \cellcolor{blue!5} \textbf{15} & \cellcolor{red!20} n/a & \cellcolor{blue!5} 15 & \cellcolor{blue!5} 15 & \cellcolor{blue!5} 15 \\
100 122& 13 & \cellcolor{blue!5} \textbf{13} & \cellcolor{blue!5} \textbf{13} & \cellcolor{blue!5} \textbf{13} & 41 & 19 & \cellcolor{blue!5} 13 & \cellcolor{blue!5} 13 \\
100 123& 15 & \cellcolor{blue!5} \textbf{15} & \cellcolor{blue!5} \textbf{15} & \cellcolor{blue!5} \textbf{15} & \cellcolor{red!20} n/a & 71 & \cellcolor{blue!5} 15 & \cellcolor{blue!5} 15 \\
100 124& 15 & \cellcolor{blue!5} \textbf{15} & \cellcolor{blue!5} \textbf{15} & \cellcolor{blue!5} \textbf{15} & \cellcolor{red!20} n/a & 16 & \cellcolor{blue!5} 15 & 16 \\
100 125& 14 & \cellcolor{blue!5} \textbf{14} & \cellcolor{blue!5} \textbf{14} & \cellcolor{blue!5} \textbf{14} & 48 & \cellcolor{blue!5} 14 & \cellcolor{blue!5} 14 & \cellcolor{blue!5} 14 \\
100 126& 49 & 54 & 53 & \cellcolor{blue!20} 51 & 77 & 63 & \cellcolor{blue!20} 51 & 52 \\
100 127& 50 & 54 & 54 & \cellcolor{blue!40} 52 & 78 & 53 & 53 & 55 \\
100 128& 56 & 59 & 58 & \cellcolor{blue!20} 57 & 70 & 83 & \cellcolor{blue!20} 57 & 58 \\
100 129& 53 & 56 & 55 & \cellcolor{blue!40} \textbf{54} & 67 & 55 & 55 & 56 \\
100 13& 24 & \cellcolor{blue!5} \textbf{24} & \cellcolor{blue!5} \textbf{24} & \cellcolor{blue!5} \textbf{24} & 25 & 84 & \cellcolor{blue!5} 24 & \cellcolor{blue!5} 24 \\
100 130& 52 & 56 & 56 & \cellcolor{blue!20} 55 & 79 & 56 & 56 & \cellcolor{blue!20} 55 \\
100 131& 50 & 54 & \cellcolor{blue!10} 53 & \cellcolor{blue!10} 53 & 69 & 71 & \cellcolor{blue!10} 53 & 54 \\
100 132& 56 & 59 & 59 & \cellcolor{blue!20} 58 & 70 & 77 & \cellcolor{blue!20} 58 & 59 \\
100 133& 53 & 56 & 57 & \cellcolor{blue!40} 55 & 86 & 56 & 57 & 57 \\
100 134& 52 & 58 & 57 & \cellcolor{blue!20} 55 & 72 & 56 & \cellcolor{blue!20} 55 & 56 \\
100 135& 53 & 58 & 56 & \cellcolor{blue!20} 55 & 71 & 57 & \cellcolor{blue!20} 55 & 58 \\
100 136& 49 & 54 & 53 & \cellcolor{blue!40} 52 & 72 & 76 & 54 & 56 \\
100 137& 50 & 57 & \cellcolor{blue!10} 54 & \cellcolor{blue!10} 54 & 75 & 66 & \cellcolor{blue!10} 54 & 57 \\
100 138& 56 & 57 & 57 & \cellcolor{blue!40} \textbf{56} & 70 & 76 & 57 & 58 \\
100 139& 50 & \cellcolor{blue!5} 52 & \cellcolor{blue!5} 52 & \cellcolor{blue!5} 52 & 70 & 84 & \cellcolor{blue!5} 52 & \cellcolor{blue!5} 52 \\
100 14& 20 & \cellcolor{blue!5} \textbf{20} & \cellcolor{blue!5} \textbf{20} & \cellcolor{blue!5} \textbf{20} & \cellcolor{red!20} n/a & 66 & \cellcolor{blue!5} 20 & \cellcolor{blue!5} 20 \\
100 140& 54 & 57 & \cellcolor{blue!10} 55 & \cellcolor{blue!10} 55 & 65 & 69 & \cellcolor{blue!10} 55 & 56 \\
100 141& 49 & \cellcolor{blue!10} 51 & 52 & \cellcolor{blue!10} 51 & 70 & 53 & \cellcolor{blue!10} 51 & 52 \\
100 142& 52 & 57 & 56 & \cellcolor{blue!10} 55 & 68 & 91 & \cellcolor{blue!10} 55 & \cellcolor{blue!10} 55 \\
100 143& 51 & \cellcolor{blue!40} 52 & 54 & 53 & 84 & 64 & 54 & 55 \\
100 144& 47 & \cellcolor{blue!5} 49 & \cellcolor{blue!5} 49 & \cellcolor{blue!5} 49 & 59 & 76 & \cellcolor{blue!5} 49 & \cellcolor{blue!5} 49 \\
100 145& 53 & 58 & 57 & \cellcolor{blue!40} 56 & \cellcolor{red!20} n/a & 82 & 57 & 59 \\
100 146& 53 & \cellcolor{blue!5} \textbf{53} & \cellcolor{blue!5} 53 & \cellcolor{blue!5} \textbf{53} & 63 & \cellcolor{blue!5} 53 & \cellcolor{blue!5} 53 & \cellcolor{blue!5} 53 \\
100 147& 58 & 62 & 60 & \cellcolor{blue!40} 59 & 73 & 71 & 60 & 61 \\
100 148& 50 & 54 & 53 & \cellcolor{blue!40} 52 & 65 & 80 & 54 & 55 \\
100 149& 54 & 57 & \cellcolor{blue!10} 55 & \cellcolor{blue!10} 55 & 69 & 76 & \cellcolor{blue!10} 55 & 56 \\
100 15& 24 & \cellcolor{blue!5} \textbf{24} & \cellcolor{blue!5} \textbf{24} & \cellcolor{blue!5} \textbf{24} & 93 & 63 & \cellcolor{blue!5} 24 & \cellcolor{blue!5} 24 \\
100 150& 54 & 58 & 58 & \cellcolor{blue!20} 57 & 72 & 59 & \cellcolor{blue!20} 57 & 59 \\
100 151& 21 & \cellcolor{blue!40} \textbf{21} & 22 & 22 & \cellcolor{red!20} n/a & 36 & 22 & 22 \\
100 152& 22 & \cellcolor{blue!5} \textbf{22} & \cellcolor{blue!5} \textbf{22} & \cellcolor{blue!5} \textbf{22} & \cellcolor{red!20} n/a & 75 & \cellcolor{blue!5} 22 & \cellcolor{blue!5} 22 \\
100 153& 21 & \cellcolor{blue!5} \textbf{21} & \cellcolor{blue!5} \textbf{21} & \cellcolor{blue!5} \textbf{21} & \cellcolor{red!20} n/a & \cellcolor{blue!5} 21 & \cellcolor{blue!5} 21 & \cellcolor{blue!5} 21 \\
100 154& 25 & \cellcolor{blue!5} \textbf{25} & \cellcolor{blue!5} \textbf{25} & \cellcolor{blue!5} \textbf{25} & \cellcolor{red!20} n/a & 76 & \cellcolor{blue!5} 25 & \cellcolor{blue!5} 25 \\
100 155& 22 & \cellcolor{blue!5} \textbf{22} & \cellcolor{blue!5} \textbf{22} & \cellcolor{blue!5} \textbf{22} & \cellcolor{red!20} n/a & 98 & \cellcolor{blue!5} 22 & \cellcolor{blue!5} 22 \\
100 156& 23 & \cellcolor{blue!5} \textbf{23} & \cellcolor{blue!5} \textbf{23} & \cellcolor{blue!5} \textbf{23} & 97 & 69 & \cellcolor{blue!5} 23 & \cellcolor{blue!5} 23 \\
100 157& 26 & \cellcolor{blue!5} \textbf{26} & \cellcolor{blue!5} \textbf{26} & \cellcolor{blue!5} \textbf{26} & 60 & 42 & \cellcolor{blue!5} 26 & \cellcolor{blue!5} 26 \\
100 158& 23 & \cellcolor{blue!5} \textbf{23} & \cellcolor{blue!5} \textbf{23} & \cellcolor{blue!5} \textbf{23} & \cellcolor{red!20} n/a & 86 & \cellcolor{blue!5} 23 & \cellcolor{blue!5} 23 \\
100 159& 19 & \cellcolor{blue!5} \textbf{19} & \cellcolor{blue!5} \textbf{19} & \cellcolor{blue!5} \textbf{19} & \cellcolor{red!20} n/a & 78 & \cellcolor{blue!5} 19 & \cellcolor{blue!5} 19 \\
100 16& 23 & \cellcolor{blue!5} \textbf{23} & \cellcolor{blue!5} \textbf{23} & \cellcolor{blue!5} \textbf{23} & 43 & 91 & \cellcolor{blue!5} 23 & \cellcolor{blue!5} 23 \\
100 160& 22 & \cellcolor{blue!5} \textbf{22} & \cellcolor{blue!5} \textbf{22} & \cellcolor{blue!5} \textbf{22} & \cellcolor{red!20} n/a & 89 & \cellcolor{blue!5} 22 & \cellcolor{blue!5} 22 \\
100 161& 22 & \cellcolor{blue!40} \textbf{22} & 23 & 23 & \cellcolor{red!20} n/a & 56 & 23 & 23 \\
100 162& 22 & \cellcolor{blue!40} \textbf{22} & 23 & 23 & \cellcolor{red!20} n/a & 32 & 23 & 23 \\
100 163& 25 & \cellcolor{blue!5} \textbf{25} & \cellcolor{blue!5} \textbf{25} & \cellcolor{blue!5} \textbf{25} & \cellcolor{red!20} n/a & 76 & \cellcolor{blue!5} 25 & \cellcolor{blue!5} 25 \\
100 164& 23 & \cellcolor{blue!5} \textbf{23} & \cellcolor{blue!5} \textbf{23} & \cellcolor{blue!5} \textbf{23} & 53 & 53 & \cellcolor{blue!5} 23 & \cellcolor{blue!5} 23 \\
100 165& 24 & \cellcolor{blue!40} \textbf{24} & 25 & 25 & \cellcolor{red!20} n/a & 70 & 25 & 25 \\
100 166& 24 & \cellcolor{blue!5} \textbf{24} & \cellcolor{blue!5} \textbf{24} & \cellcolor{blue!5} \textbf{24} & \cellcolor{red!20} n/a & 32 & \cellcolor{blue!5} 24 & \cellcolor{blue!5} 24 \\
100 167& 22 & \cellcolor{blue!5} \textbf{22} & \cellcolor{blue!5} \textbf{22} & \cellcolor{blue!5} \textbf{22} & \cellcolor{red!20} n/a & 51 & \cellcolor{blue!5} 22 & \cellcolor{blue!5} 22 \\
100 168& 21 & \cellcolor{blue!20} \textbf{21} & 22 & 22 & \cellcolor{red!20} n/a & 75 & \cellcolor{blue!20} 21 & 22 \\
100 169& 21 & \cellcolor{blue!5} \textbf{21} & \cellcolor{blue!5} \textbf{21} & \cellcolor{blue!5} \textbf{21} & \cellcolor{red!20} n/a & 94 & \cellcolor{blue!5} 21 & \cellcolor{blue!5} 21 \\
100 17& 21 & \cellcolor{blue!40} \textbf{21} & 22 & 22 & \cellcolor{red!20} n/a & 68 & 22 & 22 \\
100 170& 24 & \cellcolor{blue!5} \textbf{24} & \cellcolor{blue!5} \textbf{24} & \cellcolor{blue!5} \textbf{24} & 52 & 38 & \cellcolor{blue!5} 24 & \cellcolor{blue!5} 24 \\
100 171& 24 & \cellcolor{blue!20} \textbf{24} & 25 & 25 & 25 & 25 & \cellcolor{blue!20} 24 & 25 \\
100 172& 24 & \cellcolor{blue!5} \textbf{24} & \cellcolor{blue!5} \textbf{24} & \cellcolor{blue!5} \textbf{24} & \cellcolor{red!20} n/a & 91 & \cellcolor{blue!5} 24 & \cellcolor{blue!5} 24 \\
100 173& 24 & \cellcolor{blue!5} 25 & \cellcolor{blue!5} 25 & \cellcolor{blue!5} 25 & 26 & 91 & \cellcolor{blue!5} 25 & \cellcolor{blue!5} 25 \\
100 174& 22 & \cellcolor{blue!10} \textbf{22} & \cellcolor{blue!10} \textbf{22} & 23 & \cellcolor{red!20} n/a & 47 & \cellcolor{blue!10} 22 & 23 \\
100 175& 26 & \cellcolor{blue!5} 27 & \cellcolor{blue!5} 27 & \cellcolor{blue!5} 27 & 100 & 71 & \cellcolor{blue!5} 27 & \cellcolor{blue!5} 27 \\
100 176& 13 & \cellcolor{blue!5} \textbf{13} & \cellcolor{blue!5} \textbf{13} & \cellcolor{blue!5} \textbf{13} & 14 & 80 & \cellcolor{blue!5} 13 & \cellcolor{blue!5} 13 \\
100 177& 14 & \cellcolor{blue!5} \textbf{14} & \cellcolor{blue!5} \textbf{14} & \cellcolor{blue!5} \textbf{14} & 19 & 95 & \cellcolor{blue!5} 14 & \cellcolor{blue!5} 14 \\
100 178& 15 & \cellcolor{blue!5} \textbf{15} & \cellcolor{blue!5} \textbf{15} & \cellcolor{blue!5} \textbf{15} & \cellcolor{red!20} n/a & 81 & \cellcolor{blue!5} 15 & \cellcolor{blue!5} 15 \\
100 179& 15 & \cellcolor{blue!5} \textbf{15} & \cellcolor{blue!5} \textbf{15} & \cellcolor{blue!5} \textbf{15} & 16 & 74 & \cellcolor{blue!5} 15 & \cellcolor{blue!5} 15 \\
100 18& 19 & \cellcolor{blue!40} \textbf{19} & 20 & 20 & \cellcolor{red!20} n/a & 95 & 20 & 20 \\
100 180& 15 & \cellcolor{blue!5} \textbf{15} & \cellcolor{blue!5} \textbf{15} & \cellcolor{blue!5} \textbf{15} & \cellcolor{blue!5} 15 & 91 & \cellcolor{blue!5} 15 & \cellcolor{blue!5} 15 \\
100 181& 13 & \cellcolor{blue!5} \textbf{13} & \cellcolor{blue!5} \textbf{13} & \cellcolor{blue!5} \textbf{13} & 14 & 60 & \cellcolor{blue!5} 13 & \cellcolor{blue!5} 13 \\
100 182& 15 & \cellcolor{blue!5} \textbf{15} & \cellcolor{blue!5} \textbf{15} & \cellcolor{blue!5} \textbf{15} & \cellcolor{blue!5} 15 & 47 & \cellcolor{blue!5} 15 & \cellcolor{blue!5} 15 \\
100 183& 14 & \cellcolor{blue!5} \textbf{14} & \cellcolor{blue!5} \textbf{14} & \cellcolor{blue!5} \textbf{14} & 31 & 72 & \cellcolor{blue!5} 14 & \cellcolor{blue!5} 14 \\
100 184& 14 & \cellcolor{blue!5} \textbf{14} & \cellcolor{blue!5} \textbf{14} & \cellcolor{blue!5} \textbf{14} & \cellcolor{red!20} n/a & 96 & \cellcolor{blue!5} 14 & \cellcolor{blue!5} 14 \\
100 185& 15 & \cellcolor{blue!5} \textbf{15} & \cellcolor{blue!5} \textbf{15} & \cellcolor{blue!5} \textbf{15} & \cellcolor{red!20} n/a & 54 & \cellcolor{blue!5} 15 & \cellcolor{blue!5} 15 \\
100 186& 14 & \cellcolor{blue!5} \textbf{14} & \cellcolor{blue!5} \textbf{14} & \cellcolor{blue!5} \textbf{14} & 15 & 64 & \cellcolor{blue!5} 14 & \cellcolor{blue!5} 14 \\
100 187& 13 & \cellcolor{blue!20} \textbf{13} & 14 & 14 & 14 & 48 & \cellcolor{blue!20} 13 & 14 \\
100 188& 16 & \cellcolor{blue!5} \textbf{16} & \cellcolor{blue!5} \textbf{16} & \cellcolor{blue!5} \textbf{16} & \cellcolor{red!20} n/a & 69 & \cellcolor{blue!5} 16 & \cellcolor{blue!5} 16 \\
100 189& 14 & \cellcolor{blue!5} \textbf{14} & \cellcolor{blue!5} \textbf{14} & \cellcolor{blue!5} \textbf{14} & \cellcolor{red!20} n/a & 85 & \cellcolor{blue!5} 14 & \cellcolor{blue!5} 14 \\
100 19& 23 & \cellcolor{blue!5} \textbf{23} & \cellcolor{blue!5} \textbf{23} & \cellcolor{blue!5} \textbf{23} & \cellcolor{red!20} n/a & 91 & \cellcolor{blue!5} 23 & \cellcolor{blue!5} 23 \\
100 190& 13 & \cellcolor{blue!5} \textbf{13} & \cellcolor{blue!5} \textbf{13} & \cellcolor{blue!5} \textbf{13} & 14 & 89 & \cellcolor{blue!5} 13 & \cellcolor{blue!5} 13 \\
100 191& 14 & \cellcolor{blue!5} \textbf{14} & \cellcolor{blue!5} \textbf{14} & \cellcolor{blue!5} \textbf{14} & \cellcolor{blue!5} 14 & 78 & \cellcolor{blue!5} 14 & \cellcolor{blue!5} 14 \\
100 192& 13 & \cellcolor{blue!5} \textbf{13} & \cellcolor{blue!5} \textbf{13} & \cellcolor{blue!5} \textbf{13} & 14 & 78 & \cellcolor{blue!5} 13 & \cellcolor{blue!5} 13 \\
100 193& 15 & \cellcolor{blue!5} \textbf{15} & \cellcolor{blue!5} \textbf{15} & \cellcolor{blue!5} \textbf{15} & 62 & 98 & \cellcolor{blue!5} 15 & \cellcolor{blue!5} 15 \\
100 194& 15 & \cellcolor{blue!5} \textbf{15} & \cellcolor{blue!5} \textbf{15} & \cellcolor{blue!5} \textbf{15} & \cellcolor{red!20} n/a & 80 & \cellcolor{blue!5} 15 & \cellcolor{blue!5} 15 \\
100 195& 15 & \cellcolor{blue!5} \textbf{15} & \cellcolor{blue!5} \textbf{15} & \cellcolor{blue!5} \textbf{15} & 36 & 85 & \cellcolor{blue!5} 15 & \cellcolor{blue!5} 15 \\
100 196& 15 & \cellcolor{blue!5} \textbf{15} & \cellcolor{blue!5} \textbf{15} & \cellcolor{blue!5} \textbf{15} & \cellcolor{blue!5} 15 & 97 & \cellcolor{blue!5} 15 & \cellcolor{blue!5} 15 \\
100 197& 15 & \cellcolor{blue!5} \textbf{15} & \cellcolor{blue!5} \textbf{15} & \cellcolor{blue!5} \textbf{15} & 100 & 24 & \cellcolor{blue!5} 15 & \cellcolor{blue!5} 15 \\
100 198& 13 & \cellcolor{blue!5} \textbf{13} & \cellcolor{blue!5} \textbf{13} & \cellcolor{blue!5} \textbf{13} & 37 & 79 & \cellcolor{blue!5} 13 & \cellcolor{blue!5} 13 \\
100 199& 14 & \cellcolor{blue!5} \textbf{14} & \cellcolor{blue!5} \textbf{14} & \cellcolor{blue!5} \textbf{14} & 20 & 19 & \cellcolor{blue!5} 14 & \cellcolor{blue!5} 14 \\
100 2& 21 & \cellcolor{blue!5} \textbf{21} & \cellcolor{blue!5} \textbf{21} & \cellcolor{blue!5} \textbf{21} & \cellcolor{red!20} n/a & 51 & \cellcolor{blue!5} 21 & \cellcolor{blue!5} 21 \\
100 20& 21 & \cellcolor{blue!5} \textbf{21} & \cellcolor{blue!5} \textbf{21} & \cellcolor{blue!5} \textbf{21} & \cellcolor{red!20} n/a & 76 & \cellcolor{blue!5} 21 & \cellcolor{blue!5} 21 \\
100 200& 15 & \cellcolor{blue!5} \textbf{15} & \cellcolor{blue!5} \textbf{15} & \cellcolor{blue!5} \textbf{15} & \cellcolor{red!20} n/a & 96 & \cellcolor{blue!5} 15 & \cellcolor{blue!5} 15 \\
100 201& 52 & 54 & 55 & 54 & 72 & 84 & \cellcolor{blue!40} 53 & 55 \\
100 202& 61 & 62 & 62 & \cellcolor{blue!20} \textbf{61} & 100 & 92 & \cellcolor{blue!20} 61 & 65 \\
100 203& 52 & 53 & 53 & 53 & \cellcolor{red!20} n/a & 64 & \cellcolor{blue!40} 52 & 53 \\
100 204& 49 & 51 & 52 & 51 & \cellcolor{red!20} n/a & 80 & \cellcolor{blue!40} 50 & 51 \\
100 205& 56 & 58 & 58 & \cellcolor{blue!40} \textbf{56} & 75 & 91 & 57 & 58 \\
100 206& 50 & 54 & 53 & \cellcolor{blue!20} 52 & \cellcolor{red!20} n/a & 57 & \cellcolor{blue!20} 52 & 53 \\
100 207& 50 & 52 & 52 & 52 & \cellcolor{red!20} n/a & 70 & \cellcolor{blue!40} 51 & 53 \\
100 208& 56 & 60 & \cellcolor{blue!10} 57 & \cellcolor{blue!10} 57 & \cellcolor{red!20} n/a & 87 & \cellcolor{blue!10} 57 & 59 \\
100 209& 54 & \cellcolor{blue!5} 56 & \cellcolor{blue!5} 56 & \cellcolor{blue!5} 56 & 73 & 78 & \cellcolor{blue!5} 56 & 58 \\
100 21& 21 & \cellcolor{blue!5} \textbf{21} & \cellcolor{blue!5} \textbf{21} & \cellcolor{blue!5} \textbf{21} & 22 & 79 & \cellcolor{blue!5} 21 & \cellcolor{blue!5} 21 \\
100 210& 51 & 53 & 53 & \cellcolor{blue!20} 52 & 65 & 65 & \cellcolor{blue!20} 52 & 54 \\
100 211& 51 & 53 & 53 & \cellcolor{blue!10} 52 & 79 & 81 & \cellcolor{blue!10} 52 & \cellcolor{blue!10} 52 \\
100 212& 51 & 53 & 54 & \cellcolor{blue!20} 52 & 69 & 55 & \cellcolor{blue!20} 52 & 54 \\
100 213& 51 & 54 & 54 & \cellcolor{blue!20} 53 & 74 & 96 & \cellcolor{blue!20} 53 & 55 \\
100 214& 53 & 56 & \cellcolor{blue!10} 55 & \cellcolor{blue!10} 55 & 71 & 87 & \cellcolor{blue!10} 55 & 56 \\
100 215& 47 & \cellcolor{blue!20} 49 & 51 & 50 & \cellcolor{red!20} n/a & 76 & \cellcolor{blue!20} 49 & 51 \\
100 216& 51 & 53 & 54 & \cellcolor{blue!40} 52 & \cellcolor{red!20} n/a & 94 & 53 & 55 \\
100 217& 51 & 53 & 53 & \cellcolor{blue!20} 52 & 70 & 84 & \cellcolor{blue!20} 52 & 54 \\
100 218& 52 & 56 & 54 & \cellcolor{blue!40} 53 & 65 & 57 & 54 & 56 \\
100 219& 51 & 53 & 53 & \cellcolor{blue!20} 52 & 77 & 88 & \cellcolor{blue!20} 52 & 54 \\
100 22& 24 & \cellcolor{blue!40} \textbf{24} & 25 & 25 & \cellcolor{red!20} n/a & 48 & 25 & 25 \\
100 220& 52 & 54 & 54 & \cellcolor{blue!20} 53 & 88 & 87 & \cellcolor{blue!20} 53 & 55 \\
100 221& 56 & \cellcolor{blue!10} 57 & 58 & \cellcolor{blue!10} 57 & 97 & 88 & \cellcolor{blue!10} 57 & 59 \\
100 222& 51 & 55 & 54 & \cellcolor{blue!20} 53 & 82 & 66 & \cellcolor{blue!20} 53 & 55 \\
100 223& 50 & 53 & 52 & 52 & \cellcolor{red!20} n/a & 64 & \cellcolor{blue!40} 51 & 52 \\
100 224& 55 & 56 & 56 & 56 & \cellcolor{red!20} n/a & 63 & \cellcolor{blue!40} 55 & 57 \\
100 225& 51 & 55 & 54 & 54 & 74 & 58 & \cellcolor{blue!40} 53 & 55 \\
100 226& 24 & \cellcolor{blue!40} \textbf{24} & 25 & 25 & \cellcolor{red!20} n/a & 68 & 25 & 25 \\
100 227& 26 & \cellcolor{blue!5} 27 & \cellcolor{blue!5} 27 & \cellcolor{blue!5} 27 & \cellcolor{red!20} n/a & \cellcolor{blue!5} 27 & \cellcolor{blue!5} 27 & \cellcolor{blue!5} 27 \\
100 228& 22 & \cellcolor{blue!5} \textbf{22} & \cellcolor{blue!5} \textbf{22} & \cellcolor{blue!5} \textbf{22} & \cellcolor{red!20} n/a & \cellcolor{blue!5} 22 & \cellcolor{blue!5} 22 & \cellcolor{blue!5} 22 \\
100 229& 24 & \cellcolor{blue!5} \textbf{24} & \cellcolor{blue!5} \textbf{24} & \cellcolor{blue!5} \textbf{24} & \cellcolor{red!20} n/a & \cellcolor{blue!5} 24 & \cellcolor{blue!5} 24 & \cellcolor{blue!5} 24 \\
100 23& 24 & \cellcolor{blue!5} \textbf{24} & \cellcolor{blue!5} \textbf{24} & \cellcolor{blue!5} \textbf{24} & \cellcolor{red!20} n/a & 98 & \cellcolor{blue!5} 24 & \cellcolor{blue!5} 24 \\
100 230& 23 & \cellcolor{blue!40} \textbf{23} & 24 & 24 & \cellcolor{red!20} n/a & 38 & 24 & 24 \\
100 231& 22 & \cellcolor{blue!5} \textbf{22} & 23 & \cellcolor{blue!5} \textbf{22} & \cellcolor{red!20} n/a & 23 & \cellcolor{blue!5} 22 & \cellcolor{blue!5} 22 \\
100 232& 22 & \cellcolor{blue!5} \textbf{22} & \cellcolor{blue!5} \textbf{22} & \cellcolor{blue!5} \textbf{22} & 99 & 35 & \cellcolor{blue!5} 22 & \cellcolor{blue!5} 22 \\
100 233& 22 & \cellcolor{blue!40} \textbf{22} & 23 & 23 & \cellcolor{red!20} n/a & 36 & 23 & 23 \\
100 234& 23 & \cellcolor{blue!5} \textbf{23} & \cellcolor{blue!5} \textbf{23} & \cellcolor{blue!5} \textbf{23} & \cellcolor{red!20} n/a & \cellcolor{blue!5} 23 & \cellcolor{blue!5} 23 & \cellcolor{blue!5} 23 \\
100 235& 26 & \cellcolor{blue!5} \textbf{26} & \cellcolor{blue!5} \textbf{26} & \cellcolor{blue!5} \textbf{26} & 54 & \cellcolor{blue!5} 26 & \cellcolor{blue!5} 26 & \cellcolor{blue!5} 26 \\
100 236& 22 & \cellcolor{blue!40} \textbf{22} & 23 & 23 & 25 & 23 & 23 & 23 \\
100 237& 23 & \cellcolor{blue!5} \textbf{23} & \cellcolor{blue!5} \textbf{23} & \cellcolor{blue!5} \textbf{23} & \cellcolor{red!20} n/a & \cellcolor{blue!5} 23 & \cellcolor{blue!5} 23 & \cellcolor{blue!5} 23 \\
100 238& 23 & \cellcolor{blue!5} \textbf{23} & \cellcolor{blue!5} \textbf{23} & \cellcolor{blue!5} \textbf{23} & \cellcolor{red!20} n/a & 49 & \cellcolor{blue!5} 23 & \cellcolor{blue!5} 23 \\
100 239& 21 & \cellcolor{blue!5} \textbf{21} & \cellcolor{blue!5} \textbf{21} & \cellcolor{blue!5} \textbf{21} & 36 & 92 & \cellcolor{blue!5} 21 & \cellcolor{blue!5} 21 \\
100 24& 24 & \cellcolor{blue!5} \textbf{24} & \cellcolor{blue!5} \textbf{24} & \cellcolor{blue!5} \textbf{24} & 42 & 69 & \cellcolor{blue!5} 24 & \cellcolor{blue!5} 24 \\
100 240& 22 & \cellcolor{blue!5} \textbf{22} & \cellcolor{blue!5} \textbf{22} & \cellcolor{blue!5} \textbf{22} & \cellcolor{red!20} n/a & \cellcolor{blue!5} 22 & \cellcolor{blue!5} 22 & \cellcolor{blue!5} 22 \\
100 241& 22 & \cellcolor{blue!5} \textbf{22} & \cellcolor{blue!5} \textbf{22} & \cellcolor{blue!5} \textbf{22} & \cellcolor{red!20} n/a & 67 & \cellcolor{blue!5} 22 & 23 \\
100 242& 23 & \cellcolor{blue!5} \textbf{23} & \cellcolor{blue!5} \textbf{23} & \cellcolor{blue!5} \textbf{23} & \cellcolor{red!20} n/a & 87 & \cellcolor{blue!5} 23 & \cellcolor{blue!5} 23 \\
100 243& 23 & \cellcolor{blue!20} \textbf{23} & 24 & 24 & \cellcolor{red!20} n/a & 28 & \cellcolor{blue!20} 23 & 24 \\
100 244& 21 & \cellcolor{blue!5} \textbf{21} & \cellcolor{blue!5} \textbf{21} & \cellcolor{blue!5} \textbf{21} & \cellcolor{red!20} n/a & 51 & \cellcolor{blue!5} 21 & \cellcolor{blue!5} 21 \\
100 245& 23 & \cellcolor{blue!5} 24 & \cellcolor{blue!5} 24 & \cellcolor{blue!5} 24 & \cellcolor{red!20} n/a & \cellcolor{blue!5} 24 & \cellcolor{blue!5} 24 & \cellcolor{blue!5} 24 \\
100 246& 26 & \cellcolor{blue!5} \textbf{26} & \cellcolor{blue!5} \textbf{26} & \cellcolor{blue!5} \textbf{26} & 89 & \cellcolor{blue!5} 26 & \cellcolor{blue!5} 26 & 27 \\
100 247& 22 & \cellcolor{blue!5} \textbf{22} & \cellcolor{blue!5} \textbf{22} & \cellcolor{blue!5} \textbf{22} & \cellcolor{red!20} n/a & 24 & \cellcolor{blue!5} 22 & 23 \\
100 248& 19 & \cellcolor{blue!5} \textbf{19} & \cellcolor{blue!5} \textbf{19} & \cellcolor{blue!5} \textbf{19} & 36 & 20 & \cellcolor{blue!5} 19 & 20 \\
100 249& 21 & \cellcolor{blue!5} \textbf{21} & \cellcolor{blue!5} \textbf{21} & \cellcolor{blue!5} \textbf{21} & \cellcolor{red!20} n/a & 69 & \cellcolor{blue!5} 21 & \cellcolor{blue!5} 21 \\
100 25& 22 & \cellcolor{blue!5} \textbf{22} & \cellcolor{blue!5} \textbf{22} & \cellcolor{blue!5} \textbf{22} & 25 & 95 & \cellcolor{blue!5} 22 & \cellcolor{blue!5} 22 \\
100 250& 24 & \cellcolor{blue!5} \textbf{24} & \cellcolor{blue!5} \textbf{24} & \cellcolor{blue!5} \textbf{24} & \cellcolor{red!20} n/a & \cellcolor{blue!5} 24 & \cellcolor{blue!5} 24 & \cellcolor{blue!5} 24 \\
100 251& 15 & \cellcolor{blue!5} \textbf{15} & \cellcolor{blue!5} \textbf{15} & \cellcolor{blue!5} \textbf{15} & 25 & \cellcolor{blue!5} 15 & \cellcolor{blue!5} 15 & \cellcolor{blue!5} 15 \\
100 252& 14 & \cellcolor{blue!5} \textbf{14} & \cellcolor{blue!5} \textbf{14} & \cellcolor{blue!5} \textbf{14} & \cellcolor{red!20} n/a & \cellcolor{blue!5} 14 & \cellcolor{blue!5} 14 & \cellcolor{blue!5} 14 \\
100 253& 14 & \cellcolor{blue!5} \textbf{14} & \cellcolor{blue!5} \textbf{14} & \cellcolor{blue!5} \textbf{14} & 26 & 93 & \cellcolor{blue!5} 14 & \cellcolor{blue!5} 14 \\
100 254& 14 & \cellcolor{blue!5} \textbf{14} & \cellcolor{blue!5} \textbf{14} & \cellcolor{blue!5} \textbf{14} & 42 & \cellcolor{blue!5} 14 & \cellcolor{blue!5} 14 & \cellcolor{blue!5} 14 \\
100 255& 14 & \cellcolor{blue!5} \textbf{14} & \cellcolor{blue!5} \textbf{14} & \cellcolor{blue!5} \textbf{14} & \cellcolor{blue!5} 14 & \cellcolor{blue!5} 14 & \cellcolor{blue!5} \textbf{14} & \cellcolor{blue!5} \textbf{14} \\
100 256& 15 & \cellcolor{blue!5} \textbf{15} & \cellcolor{blue!5} \textbf{15} & \cellcolor{blue!5} \textbf{15} & \cellcolor{red!20} n/a & \cellcolor{blue!5} 15 & \cellcolor{blue!5} 15 & \cellcolor{blue!5} 15 \\
100 257& 12 & \cellcolor{blue!5} \textbf{12} & \cellcolor{blue!5} \textbf{12} & \cellcolor{blue!5} \textbf{12} & 100 & 13 & \cellcolor{blue!5} 12 & \cellcolor{blue!5} 12 \\
100 258& 14 & \cellcolor{blue!5} \textbf{14} & \cellcolor{blue!5} \textbf{14} & \cellcolor{blue!5} \textbf{14} & \cellcolor{red!20} n/a & 15 & \cellcolor{blue!5} 14 & 15 \\
100 259& 15 & \cellcolor{blue!5} \textbf{15} & \cellcolor{blue!5} \textbf{15} & \cellcolor{blue!5} \textbf{15} & \cellcolor{red!20} n/a & 35 & \cellcolor{blue!5} 15 & \cellcolor{blue!5} 15 \\
100 26& 14 & \cellcolor{blue!5} \textbf{14} & \cellcolor{blue!5} \textbf{14} & \cellcolor{blue!5} \textbf{14} & \cellcolor{red!20} n/a & 81 & \cellcolor{blue!5} 14 & \cellcolor{blue!5} 14 \\
100 260& 15 & \cellcolor{blue!5} \textbf{15} & \cellcolor{blue!5} \textbf{15} & \cellcolor{blue!5} \textbf{15} & \cellcolor{red!20} n/a & \cellcolor{blue!5} 15 & \cellcolor{blue!5} 15 & \cellcolor{blue!5} 15 \\
100 261& 14 & \cellcolor{blue!5} \textbf{14} & \cellcolor{blue!5} \textbf{14} & \cellcolor{blue!5} \textbf{14} & 99 & 49 & \cellcolor{blue!5} \textbf{14} & \cellcolor{blue!5} 14 \\
100 262& 14 & \cellcolor{blue!5} \textbf{14} & \cellcolor{blue!5} \textbf{14} & \cellcolor{blue!5} \textbf{14} & 100 & \cellcolor{blue!5} 14 & \cellcolor{blue!5} 14 & \cellcolor{blue!5} 14 \\
100 263& 14 & \cellcolor{blue!5} \textbf{14} & \cellcolor{blue!5} \textbf{14} & \cellcolor{blue!5} \textbf{14} & \cellcolor{red!20} n/a & \cellcolor{blue!5} 14 & \cellcolor{blue!5} 14 & \cellcolor{blue!5} 14 \\
100 264& 15 & \cellcolor{blue!5} \textbf{15} & \cellcolor{blue!5} \textbf{15} & \cellcolor{blue!5} \textbf{15} & 41 & 69 & \cellcolor{blue!5} 15 & \cellcolor{blue!5} 15 \\
100 265& 14 & \cellcolor{blue!5} \textbf{14} & \cellcolor{blue!5} \textbf{14} & \cellcolor{blue!5} \textbf{14} & 100 & 96 & \cellcolor{blue!5} \textbf{14} & \cellcolor{blue!5} 14 \\
100 266& 13 & \cellcolor{blue!5} \textbf{13} & \cellcolor{blue!5} \textbf{13} & \cellcolor{blue!5} \textbf{13} & 32 & \cellcolor{blue!5} 13 & \cellcolor{blue!5} 13 & \cellcolor{blue!5} 13 \\
100 267& 13 & \cellcolor{blue!5} \textbf{13} & \cellcolor{blue!5} \textbf{13} & \cellcolor{blue!5} \textbf{13} & 25 & \cellcolor{blue!5} 13 & \cellcolor{blue!5} 13 & \cellcolor{blue!5} 13 \\
100 268& 15 & \cellcolor{blue!5} \textbf{15} & \cellcolor{blue!5} \textbf{15} & \cellcolor{blue!5} \textbf{15} & \cellcolor{blue!5} 15 & \cellcolor{blue!5} 15 & \cellcolor{blue!5} 15 & \cellcolor{blue!5} 15 \\
100 269& 15 & \cellcolor{blue!5} \textbf{15} & \cellcolor{blue!5} \textbf{15} & \cellcolor{blue!5} \textbf{15} & \cellcolor{red!20} n/a & \cellcolor{blue!5} 15 & \cellcolor{blue!5} 15 & \cellcolor{blue!5} 15 \\
100 27& 13 & \cellcolor{blue!5} \textbf{13} & \cellcolor{blue!5} \textbf{13} & \cellcolor{blue!5} \textbf{13} & 35 & 56 & \cellcolor{blue!5} 13 & \cellcolor{blue!5} 13 \\
100 270& 13 & \cellcolor{blue!5} \textbf{13} & \cellcolor{blue!5} \textbf{13} & \cellcolor{blue!5} \textbf{13} & 99 & \cellcolor{blue!5} 13 & \cellcolor{blue!5} 13 & \cellcolor{blue!5} 13 \\
100 271& 13 & \cellcolor{blue!10} \textbf{13} & 14 & \cellcolor{blue!10} \textbf{13} & 39 & 95 & \cellcolor{blue!10} 13 & 14 \\
100 272& 14 & \cellcolor{blue!5} \textbf{14} & \cellcolor{blue!5} \textbf{14} & \cellcolor{blue!5} \textbf{14} & \cellcolor{red!20} n/a & \cellcolor{blue!5} 14 & \cellcolor{blue!5} 14 & \cellcolor{blue!5} 14 \\
100 273& 13 & \cellcolor{blue!5} \textbf{13} & \cellcolor{blue!5} \textbf{13} & \cellcolor{blue!5} \textbf{13} & \cellcolor{red!20} n/a & \cellcolor{blue!5} 13 & \cellcolor{blue!5} 13 & \cellcolor{blue!5} 13 \\
100 274& 13 & \cellcolor{blue!5} \textbf{13} & \cellcolor{blue!5} \textbf{13} & \cellcolor{blue!5} \textbf{13} & \cellcolor{red!20} n/a & 14 & \cellcolor{blue!5} 13 & 14 \\
100 275& 13 & \cellcolor{blue!5} \textbf{13} & \cellcolor{blue!5} \textbf{13} & \cellcolor{blue!5} \textbf{13} & 100 & \cellcolor{blue!5} 13 & \cellcolor{blue!5} 13 & \cellcolor{blue!5} 13 \\
100 276& 58 & 63 & \cellcolor{blue!10} 61 & \cellcolor{blue!10} 61 & 69 & 63 & \cellcolor{blue!10} 61 & 62 \\
100 277& 53 & 61 & 58 & 58 & 70 & 78 & \cellcolor{blue!40} 57 & 61 \\
100 278& 55 & 60 & 58 & 58 & 67 & 61 & \cellcolor{blue!40} 57 & 59 \\
100 279& 52 & 58 & 55 & \cellcolor{blue!40} 54 & 73 & 58 & 55 & 57 \\
100 28& 14 & \cellcolor{blue!5} \textbf{14} & \cellcolor{blue!5} \textbf{14} & \cellcolor{blue!5} \textbf{14} & 15 & 74 & \cellcolor{blue!5} 14 & \cellcolor{blue!5} 14 \\
100 280& 51 & 57 & 56 & \cellcolor{blue!40} 55 & 67 & 68 & 56 & 57 \\
100 281& 59 & 64 & \cellcolor{blue!10} 62 & \cellcolor{blue!10} 62 & 85 & 83 & \cellcolor{blue!10} 62 & 64 \\
100 282& 57 & 63 & 61 & \cellcolor{blue!40} 60 & 71 & 90 & 61 & 63 \\
100 283& 53 & 57 & 56 & \cellcolor{blue!20} 55 & 65 & 56 & \cellcolor{blue!20} 55 & 57 \\
100 284& 54 & 56 & 56 & \cellcolor{blue!40} 55 & 72 & 59 & 56 & 56 \\
100 285& 52 & 57 & \cellcolor{blue!10} 55 & \cellcolor{blue!10} 55 & 69 & 57 & \cellcolor{blue!10} 55 & 57 \\
100 286& 55 & 60 & 58 & \cellcolor{blue!20} 57 & 77 & 84 & \cellcolor{blue!20} 57 & 59 \\
100 287& 53 & 56 & 55 & \cellcolor{blue!40} \textbf{54} & 78 & 55 & 55 & 57 \\
100 288& 53 & 59 & \cellcolor{blue!10} 56 & \cellcolor{blue!10} 56 & 67 & 58 & \cellcolor{blue!10} 56 & 60 \\
100 289& 61 & 65 & \cellcolor{blue!10} 62 & \cellcolor{blue!10} 62 & 73 & 64 & \cellcolor{blue!10} 62 & 64 \\
100 29& 14 & \cellcolor{blue!5} \textbf{14} & \cellcolor{blue!5} \textbf{14} & \cellcolor{blue!5} \textbf{14} & 16 & 83 & \cellcolor{blue!5} 14 & \cellcolor{blue!5} 14 \\
100 290& 52 & 59 & \cellcolor{blue!10} 55 & \cellcolor{blue!10} 55 & 67 & 60 & \cellcolor{blue!10} 55 & 57 \\
100 291& 49 & 53 & 53 & \cellcolor{blue!40} 52 & 72 & 54 & 53 & 55 \\
100 292& 55 & 58 & 59 & \cellcolor{blue!40} 57 & 72 & 60 & 59 & 60 \\
100 293& 50 & 56 & \cellcolor{blue!10} 53 & \cellcolor{blue!10} 53 & 63 & 84 & \cellcolor{blue!10} 53 & 55 \\
100 294& 54 & 58 & 58 & 58 & 73 & \cellcolor{blue!20} 57 & \cellcolor{blue!20} 57 & 61 \\
100 295& 55 & 60 & \cellcolor{blue!10} 57 & \cellcolor{blue!10} 57 & 73 & 62 & \cellcolor{blue!10} 57 & 59 \\
100 296& 53 & 58 & 56 & \cellcolor{blue!40} 55 & 69 & 57 & 56 & 59 \\
100 297& 54 & 61 & 59 & \cellcolor{blue!40} \textbf{58} & 68 & 79 & 59 & 61 \\
100 298& 57 & \cellcolor{blue!5} 59 & \cellcolor{blue!5} 59 & \cellcolor{blue!5} 59 & 68 & 61 & \cellcolor{blue!5} 59 & 61 \\
100 299& 54 & 56 & 56 & \cellcolor{blue!40} \textbf{54} & 66 & 69 & 55 & 55 \\
100 3& 20 & \cellcolor{blue!5} \textbf{20} & \cellcolor{blue!5} \textbf{20} & \cellcolor{blue!5} \textbf{20} & 62 & 55 & \cellcolor{blue!5} 20 & \cellcolor{blue!5} 20 \\
100 30& 15 & \cellcolor{blue!5} \textbf{15} & \cellcolor{blue!5} \textbf{15} & \cellcolor{blue!5} \textbf{15} & 71 & 73 & \cellcolor{blue!5} 15 & \cellcolor{blue!5} 15 \\
100 300& 51 & 57 & 55 & 55 & 97 & 55 & \cellcolor{blue!40} 54 & 55 \\
100 301& 23 & \cellcolor{blue!5} \textbf{23} & \cellcolor{blue!5} \textbf{23} & \cellcolor{blue!5} \textbf{23} & 25 & 31 & \cellcolor{blue!5} 23 & \cellcolor{blue!5} 23 \\
100 302& 24 & \cellcolor{blue!5} \textbf{24} & \cellcolor{blue!5} \textbf{24} & \cellcolor{blue!5} \textbf{24} & 68 & 35 & \cellcolor{blue!5} 24 & \cellcolor{blue!5} 24 \\
100 303& 24 & \cellcolor{blue!5} \textbf{24} & \cellcolor{blue!5} \textbf{24} & \cellcolor{blue!5} \textbf{24} & 56 & 65 & \cellcolor{blue!5} 24 & \cellcolor{blue!5} 24 \\
100 304& 21 & \cellcolor{blue!5} \textbf{21} & \cellcolor{blue!5} \textbf{21} & \cellcolor{blue!5} \textbf{21} & 54 & 81 & \cellcolor{blue!5} 21 & \cellcolor{blue!5} 21 \\
100 305& 22 & \cellcolor{blue!5} \textbf{22} & \cellcolor{blue!5} \textbf{22} & \cellcolor{blue!5} \textbf{22} & \cellcolor{red!20} n/a & 82 & \cellcolor{blue!5} 22 & \cellcolor{blue!5} 22 \\
100 306& 24 & \cellcolor{blue!5} \textbf{24} & \cellcolor{blue!5} \textbf{24} & \cellcolor{blue!5} \textbf{24} & \cellcolor{red!20} n/a & 81 & \cellcolor{blue!5} 24 & \cellcolor{blue!5} 24 \\
100 307& 23 & \cellcolor{blue!40} \textbf{23} & 24 & 24 & \cellcolor{red!20} n/a & 44 & 24 & 24 \\
100 308& 20 & \cellcolor{blue!40} \textbf{20} & 21 & 21 & 22 & 75 & 21 & 21 \\
100 309& 21 & \cellcolor{blue!40} \textbf{21} & 22 & 22 & \cellcolor{red!20} n/a & 80 & 22 & 22 \\
100 31& 14 & \cellcolor{blue!5} \textbf{14} & \cellcolor{blue!5} \textbf{14} & \cellcolor{blue!5} \textbf{14} & \cellcolor{red!20} n/a & 38 & \cellcolor{blue!5} 14 & \cellcolor{blue!5} 14 \\
100 310& 23 & \cellcolor{blue!5} \textbf{23} & \cellcolor{blue!5} \textbf{23} & \cellcolor{blue!5} \textbf{23} & \cellcolor{red!20} n/a & 57 & \cellcolor{blue!5} 23 & \cellcolor{blue!5} 23 \\
100 311& 21 & \cellcolor{blue!5} \textbf{21} & \cellcolor{blue!5} \textbf{21} & \cellcolor{blue!5} \textbf{21} & \cellcolor{red!20} n/a & 94 & \cellcolor{blue!5} 21 & \cellcolor{blue!5} 21 \\
100 312& 22 & \cellcolor{blue!5} \textbf{22} & \cellcolor{blue!5} \textbf{22} & \cellcolor{blue!5} \textbf{22} & 44 & 68 & \cellcolor{blue!5} 22 & \cellcolor{blue!5} 22 \\
100 313& 23 & \cellcolor{blue!5} \textbf{23} & \cellcolor{blue!5} \textbf{23} & \cellcolor{blue!5} \textbf{23} & \cellcolor{red!20} n/a & 88 & \cellcolor{blue!5} 23 & \cellcolor{blue!5} 23 \\
100 314& 19 & \cellcolor{blue!5} \textbf{19} & \cellcolor{blue!5} \textbf{19} & \cellcolor{blue!5} \textbf{19} & 100 & 43 & \cellcolor{blue!5} 19 & \cellcolor{blue!5} 19 \\
100 315& 22 & \cellcolor{blue!10} \textbf{22} & 23 & \cellcolor{blue!10} \textbf{22} & \cellcolor{red!20} n/a & 51 & \cellcolor{blue!10} 22 & 23 \\
100 316& 24 & \cellcolor{blue!5} \textbf{24} & \cellcolor{blue!5} \textbf{24} & \cellcolor{blue!5} \textbf{24} & \cellcolor{red!20} n/a & 82 & \cellcolor{blue!5} 24 & \cellcolor{blue!5} 24 \\
100 317& 26 & \cellcolor{blue!5} \textbf{26} & \cellcolor{blue!5} \textbf{26} & \cellcolor{blue!5} \textbf{26} & \cellcolor{red!20} n/a & 97 & \cellcolor{blue!5} 26 & \cellcolor{blue!5} 26 \\
100 318& 21 & \cellcolor{blue!5} \textbf{21} & \cellcolor{blue!5} \textbf{21} & \cellcolor{blue!5} \textbf{21} & \cellcolor{red!20} n/a & 95 & \cellcolor{blue!5} 21 & \cellcolor{blue!5} 21 \\
100 319& 23 & \cellcolor{blue!5} \textbf{23} & \cellcolor{blue!5} \textbf{23} & \cellcolor{blue!5} \textbf{23} & \cellcolor{red!20} n/a & 85 & \cellcolor{blue!5} 23 & \cellcolor{blue!5} 23 \\
100 32& 14 & \cellcolor{blue!5} \textbf{14} & \cellcolor{blue!5} \textbf{14} & \cellcolor{blue!5} \textbf{14} & 100 & 82 & \cellcolor{blue!5} 14 & \cellcolor{blue!5} 14 \\
100 320& 22 & \cellcolor{blue!5} \textbf{22} & \cellcolor{blue!5} \textbf{22} & \cellcolor{blue!5} \textbf{22} & \cellcolor{red!20} n/a & 95 & \cellcolor{blue!5} 22 & \cellcolor{blue!5} 22 \\
100 321& 26 & \cellcolor{blue!5} \textbf{26} & \cellcolor{blue!5} \textbf{26} & \cellcolor{blue!5} \textbf{26} & 53 & 70 & \cellcolor{blue!5} 26 & \cellcolor{blue!5} 26 \\
100 322& 23 & \cellcolor{blue!20} \textbf{23} & 24 & 24 & 46 & 59 & \cellcolor{blue!20} 23 & 24 \\
100 323& 24 & \cellcolor{blue!5} \textbf{24} & \cellcolor{blue!5} \textbf{24} & \cellcolor{blue!5} \textbf{24} & 26 & 70 & \cellcolor{blue!5} 24 & \cellcolor{blue!5} 24 \\
100 324& 23 & \cellcolor{blue!5} \textbf{23} & \cellcolor{blue!5} \textbf{23} & \cellcolor{blue!5} \textbf{23} & \cellcolor{red!20} n/a & 66 & \cellcolor{blue!5} 23 & \cellcolor{blue!5} 23 \\
100 325& 25 & \cellcolor{blue!10} \textbf{25} & 26 & \cellcolor{blue!10} \textbf{25} & \cellcolor{red!20} n/a & 58 & \cellcolor{blue!10} 25 & 26 \\
100 326& 13 & \cellcolor{blue!5} \textbf{13} & \cellcolor{blue!5} \textbf{13} & \cellcolor{blue!5} \textbf{13} & 14 & 92 & \cellcolor{blue!5} 13 & \cellcolor{blue!5} 13 \\
100 327& 14 & \cellcolor{blue!5} \textbf{14} & \cellcolor{blue!5} \textbf{14} & \cellcolor{blue!5} \textbf{14} & 15 & 85 & \cellcolor{blue!5} 14 & \cellcolor{blue!5} 14 \\
100 328& 14 & \cellcolor{blue!10} \textbf{14} & 15 & \cellcolor{blue!10} \textbf{14} & 100 & 82 & \cellcolor{blue!10} 14 & 15 \\
100 329& 14 & \cellcolor{blue!5} \textbf{14} & \cellcolor{blue!5} \textbf{14} & \cellcolor{blue!5} \textbf{14} & 65 & 83 & \cellcolor{blue!5} 14 & \cellcolor{blue!5} 14 \\
100 33& 15 & \cellcolor{blue!5} \textbf{15} & \cellcolor{blue!5} \textbf{15} & \cellcolor{blue!5} \textbf{15} & 93 & 50 & \cellcolor{blue!5} 15 & \cellcolor{blue!5} 15 \\
100 330& 14 & \cellcolor{blue!5} \textbf{14} & \cellcolor{blue!5} \textbf{14} & \cellcolor{blue!5} \textbf{14} & \cellcolor{red!20} n/a & 66 & \cellcolor{blue!5} 14 & 15 \\
100 331& 14 & \cellcolor{blue!5} \textbf{14} & \cellcolor{blue!5} \textbf{14} & \cellcolor{blue!5} \textbf{14} & 92 & 86 & \cellcolor{blue!5} 14 & \cellcolor{blue!5} 14 \\
100 332& 14 & \cellcolor{blue!5} \textbf{14} & \cellcolor{blue!5} \textbf{14} & \cellcolor{blue!5} \textbf{14} & \cellcolor{blue!5} 14 & 81 & \cellcolor{blue!5} 14 & \cellcolor{blue!5} 14 \\
100 333& 15 & \cellcolor{blue!5} \textbf{15} & \cellcolor{blue!5} \textbf{15} & \cellcolor{blue!5} \textbf{15} & 49 & 86 & \cellcolor{blue!5} 15 & \cellcolor{blue!5} 15 \\
100 334& 14 & \cellcolor{blue!5} \textbf{14} & \cellcolor{blue!5} \textbf{14} & \cellcolor{blue!5} \textbf{14} & \cellcolor{red!20} n/a & 70 & \cellcolor{blue!5} 14 & \cellcolor{blue!5} 14 \\
100 335& 13 & \cellcolor{blue!5} \textbf{13} & \cellcolor{blue!5} \textbf{13} & \cellcolor{blue!5} \textbf{13} & 14 & 71 & \cellcolor{blue!5} \textbf{13} & \cellcolor{blue!5} 13 \\
100 336& 15 & \cellcolor{blue!5} \textbf{15} & \cellcolor{blue!5} \textbf{15} & \cellcolor{blue!5} \textbf{15} & 55 & 61 & \cellcolor{blue!5} 15 & \cellcolor{blue!5} 15 \\
100 337& 13 & \cellcolor{blue!5} \textbf{13} & \cellcolor{blue!5} \textbf{13} & \cellcolor{blue!5} \textbf{13} & 14 & 85 & \cellcolor{blue!5} 13 & \cellcolor{blue!5} 13 \\
100 338& 14 & \cellcolor{blue!20} \textbf{14} & 15 & 15 & 16 & 60 & \cellcolor{blue!20} 14 & 15 \\
100 339& 14 & \cellcolor{blue!5} \textbf{14} & \cellcolor{blue!5} \textbf{14} & \cellcolor{blue!5} \textbf{14} & 16 & 88 & \cellcolor{blue!5} 14 & \cellcolor{blue!5} 14 \\
100 34& 15 & \cellcolor{blue!5} \textbf{15} & \cellcolor{blue!5} \textbf{15} & \cellcolor{blue!5} \textbf{15} & \cellcolor{red!20} n/a & \cellcolor{blue!5} 15 & \cellcolor{blue!5} 15 & \cellcolor{blue!5} 15 \\
100 340& 14 & \cellcolor{blue!5} \textbf{14} & \cellcolor{blue!5} \textbf{14} & \cellcolor{blue!5} \textbf{14} & 15 & 72 & \cellcolor{blue!5} 14 & \cellcolor{blue!5} 14 \\
100 341& 16 & \cellcolor{blue!5} \textbf{16} & \cellcolor{blue!5} \textbf{16} & \cellcolor{blue!5} \textbf{16} & \cellcolor{red!20} n/a & 75 & \cellcolor{blue!5} 16 & \cellcolor{blue!5} 16 \\
100 342& 14 & \cellcolor{blue!5} \textbf{14} & \cellcolor{blue!5} \textbf{14} & \cellcolor{blue!5} \textbf{14} & 100 & 71 & \cellcolor{blue!5} 14 & \cellcolor{blue!5} 14 \\
100 343& 16 & \cellcolor{blue!5} \textbf{16} & \cellcolor{blue!5} \textbf{16} & \cellcolor{blue!5} \textbf{16} & \cellcolor{red!20} n/a & 81 & \cellcolor{blue!5} 16 & \cellcolor{blue!5} 16 \\
100 344& 15 & \cellcolor{blue!5} \textbf{15} & \cellcolor{blue!5} \textbf{15} & \cellcolor{blue!5} \textbf{15} & 100 & 96 & \cellcolor{blue!5} 15 & \cellcolor{blue!5} 15 \\
100 345& 14 & \cellcolor{blue!5} \textbf{14} & \cellcolor{blue!5} \textbf{14} & \cellcolor{blue!5} \textbf{14} & 98 & \cellcolor{blue!5} 14 & \cellcolor{blue!5} 14 & \cellcolor{blue!5} 14 \\
100 346& 14 & \cellcolor{blue!5} \textbf{14} & \cellcolor{blue!5} \textbf{14} & \cellcolor{blue!5} \textbf{14} & 99 & 95 & \cellcolor{blue!5} 14 & \cellcolor{blue!5} 14 \\
100 347& 14 & \cellcolor{blue!5} \textbf{14} & \cellcolor{blue!5} \textbf{14} & \cellcolor{blue!5} \textbf{14} & 15 & 73 & \cellcolor{blue!5} 14 & \cellcolor{blue!5} 14 \\
100 348& 14 & \cellcolor{blue!5} \textbf{14} & \cellcolor{blue!5} \textbf{14} & \cellcolor{blue!5} \textbf{14} & \cellcolor{red!20} n/a & 99 & \cellcolor{blue!5} 14 & \cellcolor{blue!5} 14 \\
100 349& 13 & \cellcolor{blue!5} \textbf{13} & \cellcolor{blue!5} \textbf{13} & \cellcolor{blue!5} \textbf{13} & 14 & 79 & \cellcolor{blue!5} 13 & \cellcolor{blue!5} 13 \\
100 35& 15 & \cellcolor{blue!5} \textbf{15} & \cellcolor{blue!5} \textbf{15} & \cellcolor{blue!5} \textbf{15} & 16 & 89 & \cellcolor{blue!5} 15 & \cellcolor{blue!5} 15 \\
100 350& 14 & \cellcolor{blue!5} \textbf{14} & \cellcolor{blue!5} \textbf{14} & \cellcolor{blue!5} \textbf{14} & 36 & 78 & \cellcolor{blue!5} 14 & \cellcolor{blue!5} 14 \\
100 351& 58 & 60 & 60 & \cellcolor{blue!20} 59 & 98 & 89 & \cellcolor{blue!20} 59 & 61 \\
100 352& 63 & \cellcolor{blue!5} \textbf{63} & \cellcolor{blue!5} 63 & \cellcolor{blue!5} \textbf{63} & 82 & 92 & \cellcolor{blue!5} 63 & 64 \\
100 353& 50 & 54 & 53 & \cellcolor{blue!20} 51 & \cellcolor{red!20} n/a & 83 & \cellcolor{blue!20} 51 & 52 \\
100 354& 51 & 54 & 53 & \cellcolor{blue!20} 52 & 67 & 76 & \cellcolor{blue!20} 52 & 53 \\
100 355& 53 & \cellcolor{blue!10} 55 & 56 & \cellcolor{blue!10} 55 & 68 & 88 & \cellcolor{blue!10} 55 & 57 \\
100 356& 59 & 61 & 61 & \cellcolor{blue!40} \textbf{59} & \cellcolor{red!20} n/a & 97 & 60 & 61 \\
100 357& 53 & \cellcolor{blue!10} \textbf{53} & 54 & \cellcolor{blue!10} \textbf{53} & 79 & 98 & \cellcolor{blue!10} 53 & 54 \\
100 358& 51 & 54 & 53 & \cellcolor{blue!20} 52 & 100 & 87 & \cellcolor{blue!20} 52 & 53 \\
100 359& 52 & 54 & 54 & \cellcolor{blue!20} 53 & 99 & 94 & 54 & \cellcolor{blue!20} 53 \\
100 36& 14 & \cellcolor{blue!10} \textbf{14} & \cellcolor{blue!10} \textbf{14} & 15 & \cellcolor{red!20} n/a & 70 & \cellcolor{blue!10} 14 & 15 \\
100 360& 54 & 57 & 56 & \cellcolor{blue!40} \textbf{54} & \cellcolor{red!20} n/a & 81 & 56 & 57 \\
100 361& 50 & 52 & 53 & 52 & \cellcolor{red!20} n/a & 66 & \cellcolor{blue!40} 51 & 52 \\
100 362& 57 & \cellcolor{blue!10} \textbf{57} & \cellcolor{blue!10} 57 & \cellcolor{blue!10} \textbf{57} & 98 & 75 & 58 & 59 \\
100 363& 51 & 53 & 54 & \cellcolor{blue!20} 52 & 65 & 92 & \cellcolor{blue!20} 52 & 53 \\
100 364& 51 & 53 & 53 & \cellcolor{blue!40} 52 & 79 & 86 & 53 & 53 \\
100 365& 52 & 54 & 55 & \cellcolor{blue!40} 53 & 76 & 86 & 54 & 56 \\
100 366& 61 & \cellcolor{blue!20} \textbf{61} & 62 & \cellcolor{blue!20} \textbf{61} & 88 & 95 & 62 & 62 \\
100 367& 55 & 57 & 57 & \cellcolor{blue!40} \textbf{55} & 82 & 87 & 56 & 58 \\
100 368& 58 & \cellcolor{blue!10} 59 & 60 & \cellcolor{blue!10} 59 & 100 & 99 & \cellcolor{blue!10} 59 & 60 \\
100 369& 49 & 52 & 52 & \cellcolor{blue!40} 51 & \cellcolor{red!20} n/a & 61 & 52 & 53 \\
100 37& 14 & \cellcolor{blue!5} \textbf{14} & \cellcolor{blue!5} \textbf{14} & \cellcolor{blue!5} \textbf{14} & \cellcolor{red!20} n/a & 95 & \cellcolor{blue!5} 14 & \cellcolor{blue!5} 14 \\
100 370& 56 & \cellcolor{blue!5} 57 & \cellcolor{blue!5} 57 & \cellcolor{blue!5} 57 & 91 & 90 & \cellcolor{blue!5} 57 & 60 \\
100 371& 51 & 54 & \cellcolor{blue!20} 53 & \cellcolor{blue!20} 53 & 100 & 81 & 54 & 55 \\
100 372& 47 & \cellcolor{blue!5} 49 & \cellcolor{blue!5} 49 & \cellcolor{blue!5} 49 & \cellcolor{red!20} n/a & 83 & \cellcolor{blue!5} 49 & 50 \\
100 373& 49 & 52 & 52 & \cellcolor{blue!20} 51 & \cellcolor{red!20} n/a & 67 & \cellcolor{blue!20} 51 & 53 \\
100 374& 51 & 53 & 53 & 52 & \cellcolor{red!20} n/a & 99 & \cellcolor{blue!40} 51 & 52 \\
100 375& 57 & \cellcolor{blue!10} \textbf{57} & 58 & \cellcolor{blue!10} \textbf{57} & 81 & 92 & \cellcolor{blue!10} 57 & 60 \\
100 376& 23 & \cellcolor{blue!5} \textbf{23} & \cellcolor{blue!5} \textbf{23} & \cellcolor{blue!5} \textbf{23} & \cellcolor{red!20} n/a & 32 & \cellcolor{blue!5} 23 & \cellcolor{blue!5} 23 \\
100 377& 20 & \cellcolor{blue!40} \textbf{20} & 21 & 21 & \cellcolor{red!20} n/a & 28 & 21 & 21 \\
100 378& 22 & \cellcolor{blue!5} \textbf{22} & \cellcolor{blue!5} \textbf{22} & \cellcolor{blue!5} \textbf{22} & \cellcolor{red!20} n/a & 86 & \cellcolor{blue!5} 22 & \cellcolor{blue!5} 22 \\
100 379& 23 & \cellcolor{blue!10} \textbf{23} & 24 & \cellcolor{blue!10} \textbf{23} & \cellcolor{red!20} n/a & 90 & \cellcolor{blue!10} 23 & 24 \\
100 38& 14 & \cellcolor{blue!5} \textbf{14} & \cellcolor{blue!5} \textbf{14} & \cellcolor{blue!5} \textbf{14} & 16 & 80 & \cellcolor{blue!5} 14 & \cellcolor{blue!5} 14 \\
100 380& 22 & \cellcolor{blue!40} \textbf{22} & 23 & 23 & \cellcolor{red!20} n/a & 46 & 23 & 23 \\
100 381& 24 & \cellcolor{blue!5} \textbf{24} & \cellcolor{blue!5} \textbf{24} & \cellcolor{blue!5} \textbf{24} & \cellcolor{red!20} n/a & 98 & \cellcolor{blue!5} 24 & \cellcolor{blue!5} 24 \\
100 382& 25 & \cellcolor{blue!5} \textbf{25} & \cellcolor{blue!5} \textbf{25} & \cellcolor{blue!5} \textbf{25} & 61 & 43 & \cellcolor{blue!5} 25 & 26 \\
100 383& 25 & \cellcolor{blue!5} \textbf{25} & \cellcolor{blue!5} \textbf{25} & \cellcolor{blue!5} \textbf{25} & 54 & 67 & \cellcolor{blue!5} 25 & \cellcolor{blue!5} 25 \\
100 384& 25 & \cellcolor{blue!5} \textbf{25} & \cellcolor{blue!5} \textbf{25} & \cellcolor{blue!5} \textbf{25} & \cellcolor{red!20} n/a & 96 & \cellcolor{blue!5} 25 & \cellcolor{blue!5} 25 \\
100 385& 22 & \cellcolor{blue!5} \textbf{22} & \cellcolor{blue!5} \textbf{22} & \cellcolor{blue!5} \textbf{22} & \cellcolor{red!20} n/a & \cellcolor{blue!5} 22 & \cellcolor{blue!5} 22 & \cellcolor{blue!5} 22 \\
100 386& 23 & \cellcolor{blue!40} \textbf{23} & 24 & 24 & \cellcolor{red!20} n/a & 72 & 24 & 24 \\
100 387& 22 & \cellcolor{blue!5} \textbf{22} & \cellcolor{blue!5} \textbf{22} & \cellcolor{blue!5} \textbf{22} & \cellcolor{red!20} n/a & 65 & \cellcolor{blue!5} 22 & \cellcolor{blue!5} 22 \\
100 388& 25 & \cellcolor{blue!40} \textbf{25} & 26 & 26 & \cellcolor{red!20} n/a & 40 & 26 & 26 \\
100 389& 23 & \cellcolor{blue!5} \textbf{23} & \cellcolor{blue!5} \textbf{23} & \cellcolor{blue!5} \textbf{23} & \cellcolor{red!20} n/a & 45 & \cellcolor{blue!5} 23 & \cellcolor{blue!5} 23 \\
100 39& 14 & \cellcolor{blue!5} \textbf{14} & \cellcolor{blue!5} \textbf{14} & \cellcolor{blue!5} \textbf{14} & 30 & 94 & \cellcolor{blue!5} 14 & \cellcolor{blue!5} 14 \\
100 390& 22 & \cellcolor{blue!40} \textbf{22} & 23 & 23 & 67 & 63 & 23 & 23 \\
100 391& 20 & \cellcolor{blue!5} \textbf{20} & \cellcolor{blue!5} \textbf{20} & \cellcolor{blue!5} \textbf{20} & \cellcolor{red!20} n/a & 86 & \cellcolor{blue!5} 20 & \cellcolor{blue!5} 20 \\
100 392& 22 & \cellcolor{blue!5} \textbf{22} & \cellcolor{blue!5} \textbf{22} & \cellcolor{blue!5} \textbf{22} & 47 & 77 & \cellcolor{blue!5} 22 & \cellcolor{blue!5} 22 \\
100 393& 23 & \cellcolor{blue!40} \textbf{23} & 24 & 24 & \cellcolor{red!20} n/a & 24 & 24 & 24 \\
100 394& 22 & \cellcolor{blue!5} \textbf{22} & \cellcolor{blue!5} \textbf{22} & \cellcolor{blue!5} \textbf{22} & \cellcolor{red!20} n/a & \cellcolor{blue!5} 22 & \cellcolor{blue!5} 22 & \cellcolor{blue!5} 22 \\
100 395& 24 & \cellcolor{blue!5} \textbf{24} & \cellcolor{blue!5} \textbf{24} & \cellcolor{blue!5} \textbf{24} & \cellcolor{red!20} n/a & 82 & \cellcolor{blue!5} 24 & \cellcolor{blue!5} 24 \\
100 396& 20 & \cellcolor{blue!5} \textbf{20} & \cellcolor{blue!5} \textbf{20} & \cellcolor{blue!5} \textbf{20} & \cellcolor{red!20} n/a & 61 & \cellcolor{blue!5} 20 & \cellcolor{blue!5} 20 \\
100 397& 25 & \cellcolor{blue!5} 26 & \cellcolor{blue!5} 26 & \cellcolor{blue!5} 26 & 57 & 27 & \cellcolor{blue!5} 26 & \cellcolor{blue!5} 26 \\
100 398& 25 & \cellcolor{blue!5} \textbf{25} & \cellcolor{blue!5} 25 & \cellcolor{blue!5} \textbf{25} & \cellcolor{red!20} n/a & 78 & \cellcolor{blue!5} 25 & \cellcolor{blue!5} 25 \\
100 399& 23 & \cellcolor{blue!5} \textbf{23} & \cellcolor{blue!5} \textbf{23} & \cellcolor{blue!5} \textbf{23} & \cellcolor{red!20} n/a & 43 & \cellcolor{blue!5} 23 & \cellcolor{blue!5} 23 \\
100 4& 24 & \cellcolor{blue!5} \textbf{24} & \cellcolor{blue!5} \textbf{24} & \cellcolor{blue!5} \textbf{24} & \cellcolor{red!20} n/a & 57 & \cellcolor{blue!5} 24 & \cellcolor{blue!5} 24 \\
100 40& 14 & \cellcolor{blue!5} \textbf{14} & \cellcolor{blue!5} \textbf{14} & \cellcolor{blue!5} \textbf{14} & 57 & 73 & \cellcolor{blue!5} 14 & \cellcolor{blue!5} 14 \\
100 400& 24 & \cellcolor{blue!5} \textbf{24} & \cellcolor{blue!5} \textbf{24} & \cellcolor{blue!5} \textbf{24} & 31 & \cellcolor{blue!5} 24 & \cellcolor{blue!5} 24 & \cellcolor{blue!5} 24 \\
100 401& 15 & \cellcolor{blue!5} \textbf{15} & \cellcolor{blue!5} \textbf{15} & \cellcolor{blue!5} \textbf{15} & \cellcolor{red!20} n/a & 70 & \cellcolor{blue!5} 15 & \cellcolor{blue!5} 15 \\
100 402& 15 & \cellcolor{blue!5} \textbf{15} & \cellcolor{blue!5} \textbf{15} & \cellcolor{blue!5} \textbf{15} & \cellcolor{red!20} n/a & 87 & \cellcolor{blue!5} 15 & \cellcolor{blue!5} 15 \\
100 403& 14 & \cellcolor{blue!5} \textbf{14} & \cellcolor{blue!5} \textbf{14} & \cellcolor{blue!5} \textbf{14} & 99 & 53 & \cellcolor{blue!5} 14 & \cellcolor{blue!5} 14 \\
100 404& 15 & \cellcolor{blue!5} \textbf{15} & \cellcolor{blue!5} \textbf{15} & \cellcolor{blue!5} \textbf{15} & 56 & \cellcolor{blue!5} 15 & \cellcolor{blue!5} \textbf{15} & \cellcolor{blue!5} 15 \\
100 405& 13 & \cellcolor{blue!5} \textbf{13} & \cellcolor{blue!5} \textbf{13} & \cellcolor{blue!5} \textbf{13} & 14 & 87 & \cellcolor{blue!5} 13 & \cellcolor{blue!5} 13 \\
100 406& 14 & \cellcolor{blue!5} \textbf{14} & \cellcolor{blue!5} \textbf{14} & \cellcolor{blue!5} \textbf{14} & \cellcolor{blue!5} 14 & \cellcolor{blue!5} 14 & \cellcolor{blue!5} 14 & \cellcolor{blue!5} 14 \\
100 407& 15 & \cellcolor{blue!5} \textbf{15} & \cellcolor{blue!5} \textbf{15} & \cellcolor{blue!5} \textbf{15} & \cellcolor{red!20} n/a & \cellcolor{blue!5} 15 & \cellcolor{blue!5} 15 & \cellcolor{blue!5} 15 \\
100 408& 14 & \cellcolor{blue!5} \textbf{14} & \cellcolor{blue!5} \textbf{14} & \cellcolor{blue!5} \textbf{14} & 90 & 94 & \cellcolor{blue!5} 14 & \cellcolor{blue!5} 14 \\
100 409& 15 & \cellcolor{blue!5} \textbf{15} & \cellcolor{blue!5} \textbf{15} & \cellcolor{blue!5} \textbf{15} & \cellcolor{red!20} n/a & 26 & \cellcolor{blue!5} 15 & \cellcolor{blue!5} 15 \\
100 41& 13 & \cellcolor{blue!5} \textbf{13} & \cellcolor{blue!5} \textbf{13} & \cellcolor{blue!5} \textbf{13} & \cellcolor{red!20} n/a & 49 & \cellcolor{blue!5} 13 & \cellcolor{blue!5} 13 \\
100 410& 14 & \cellcolor{blue!5} \textbf{14} & \cellcolor{blue!5} \textbf{14} & \cellcolor{blue!5} \textbf{14} & \cellcolor{red!20} n/a & \cellcolor{blue!5} 14 & \cellcolor{blue!5} 14 & \cellcolor{blue!5} 14 \\
100 411& 14 & \cellcolor{blue!5} \textbf{14} & \cellcolor{blue!5} \textbf{14} & \cellcolor{blue!5} \textbf{14} & \cellcolor{red!20} n/a & 15 & \cellcolor{blue!5} 14 & \cellcolor{blue!5} 14 \\
100 412& 14 & \cellcolor{blue!5} \textbf{14} & \cellcolor{blue!5} \textbf{14} & \cellcolor{blue!5} \textbf{14} & \cellcolor{red!20} n/a & 98 & \cellcolor{blue!5} 14 & \cellcolor{blue!5} 14 \\
100 413& 14 & \cellcolor{blue!5} \textbf{14} & \cellcolor{blue!5} \textbf{14} & \cellcolor{blue!5} \textbf{14} & \cellcolor{red!20} n/a & \cellcolor{blue!5} 14 & \cellcolor{blue!5} 14 & \cellcolor{blue!5} 14 \\
100 414& 14 & \cellcolor{blue!20} \textbf{14} & 15 & 15 & 47 & 34 & \cellcolor{blue!20} 14 & 15 \\
100 415& 13 & \cellcolor{blue!5} \textbf{13} & \cellcolor{blue!5} \textbf{13} & \cellcolor{blue!5} \textbf{13} & 99 & \cellcolor{blue!5} 13 & \cellcolor{blue!5} 13 & \cellcolor{blue!5} 13 \\
100 416& 14 & \cellcolor{blue!5} \textbf{14} & \cellcolor{blue!5} \textbf{14} & \cellcolor{blue!5} \textbf{14} & 52 & \cellcolor{blue!5} 14 & \cellcolor{blue!5} 14 & \cellcolor{blue!5} 14 \\
100 417& 15 & \cellcolor{blue!5} \textbf{15} & \cellcolor{blue!5} \textbf{15} & \cellcolor{blue!5} \textbf{15} & \cellcolor{red!20} n/a & \cellcolor{blue!5} 15 & \cellcolor{blue!5} 15 & \cellcolor{blue!5} 15 \\
100 418& 16 & \cellcolor{blue!5} \textbf{16} & \cellcolor{blue!5} \textbf{16} & \cellcolor{blue!5} \textbf{16} & \cellcolor{red!20} n/a & \cellcolor{blue!5} 16 & \cellcolor{blue!5} 16 & \cellcolor{blue!5} 16 \\
100 419& 14 & \cellcolor{blue!5} \textbf{14} & \cellcolor{blue!5} \textbf{14} & \cellcolor{blue!5} \textbf{14} & \cellcolor{red!20} n/a & 15 & \cellcolor{blue!5} 14 & 15 \\
100 42& 14 & \cellcolor{blue!5} \textbf{14} & \cellcolor{blue!5} \textbf{14} & \cellcolor{blue!5} \textbf{14} & \cellcolor{red!20} n/a & 87 & \cellcolor{blue!5} 14 & \cellcolor{blue!5} 14 \\
100 420& 14 & \cellcolor{blue!5} \textbf{14} & \cellcolor{blue!5} \textbf{14} & \cellcolor{blue!5} \textbf{14} & 37 & \cellcolor{blue!5} 14 & \cellcolor{blue!5} 14 & \cellcolor{blue!5} 14 \\
100 421& 14 & \cellcolor{blue!5} \textbf{14} & \cellcolor{blue!5} \textbf{14} & \cellcolor{blue!5} \textbf{14} & \cellcolor{blue!5} 14 & 87 & \cellcolor{blue!5} 14 & \cellcolor{blue!5} 14 \\
100 422& 15 & \cellcolor{blue!5} \textbf{15} & \cellcolor{blue!5} \textbf{15} & \cellcolor{blue!5} \textbf{15} & 39 & 72 & \cellcolor{blue!5} 15 & \cellcolor{blue!5} 15 \\
100 423& 14 & \cellcolor{blue!5} \textbf{14} & \cellcolor{blue!5} \textbf{14} & \cellcolor{blue!5} \textbf{14} & 42 & 91 & \cellcolor{blue!5} 14 & \cellcolor{blue!5} 14 \\
100 424& 14 & \cellcolor{blue!5} \textbf{14} & \cellcolor{blue!5} \textbf{14} & \cellcolor{blue!5} \textbf{14} & 57 & 71 & \cellcolor{blue!5} 14 & \cellcolor{blue!5} 14 \\
100 425& 15 & \cellcolor{blue!5} \textbf{15} & \cellcolor{blue!5} \textbf{15} & \cellcolor{blue!5} \textbf{15} & 98 & 57 & \cellcolor{blue!5} 15 & \cellcolor{blue!5} 15 \\
100 426& 58 & 63 & 61 & 61 & 74 & 64 & \cellcolor{blue!40} 60 & 62 \\
100 427& 54 & 57 & \cellcolor{blue!10} 56 & \cellcolor{blue!10} 56 & 78 & 59 & \cellcolor{blue!10} 56 & 59 \\
100 428& 54 & 57 & \cellcolor{blue!10} 55 & \cellcolor{blue!10} 55 & 69 & 81 & \cellcolor{blue!10} 55 & 57 \\
100 429& 57 & 61 & 59 & \cellcolor{blue!20} 58 & 73 & 61 & \cellcolor{blue!20} 58 & 60 \\
100 43& 14 & \cellcolor{blue!5} \textbf{14} & \cellcolor{blue!5} \textbf{14} & \cellcolor{blue!5} \textbf{14} & 15 & 95 & \cellcolor{blue!5} 14 & \cellcolor{blue!5} 14 \\
100 430& 52 & 54 & 55 & \cellcolor{blue!40} 53 & 68 & 70 & 55 & 56 \\
100 431& 52 & 55 & \cellcolor{blue!20} 54 & 55 & 68 & 84 & \cellcolor{blue!20} 54 & 55 \\
100 432& 54 & 59 & \cellcolor{blue!10} 56 & \cellcolor{blue!10} 56 & 74 & 66 & \cellcolor{blue!10} 56 & 58 \\
100 433& 51 & 54 & 54 & \cellcolor{blue!40} \textbf{52} & 65 & 54 & 53 & 54 \\
100 434& 55 & 59 & \cellcolor{blue!10} 57 & \cellcolor{blue!10} 57 & 70 & 80 & \cellcolor{blue!10} 57 & 59 \\
100 435& 52 & 58 & 57 & \cellcolor{blue!40} 56 & 69 & 58 & 57 & 58 \\
100 436& 49 & 55 & \cellcolor{blue!10} 52 & \cellcolor{blue!10} 52 & 66 & 91 & \cellcolor{blue!10} 52 & 53 \\
100 437& 51 & 56 & 54 & \cellcolor{blue!20} 53 & 66 & 56 & \cellcolor{blue!20} 53 & 55 \\
100 438& 52 & 57 & 56 & \cellcolor{blue!20} 55 & 66 & 72 & \cellcolor{blue!20} 55 & 57 \\
100 439& 54 & 58 & 56 & \cellcolor{blue!20} 55 & 79 & 58 & \cellcolor{blue!20} 55 & 57 \\
100 44& 14 & \cellcolor{blue!5} \textbf{14} & \cellcolor{blue!5} \textbf{14} & \cellcolor{blue!5} \textbf{14} & 37 & 83 & \cellcolor{blue!5} 14 & \cellcolor{blue!5} 14 \\
100 440& 51 & 55 & \cellcolor{blue!10} 54 & \cellcolor{blue!10} 54 & 63 & 80 & \cellcolor{blue!10} 54 & 55 \\
100 441& 51 & 54 & \cellcolor{blue!10} 53 & \cellcolor{blue!10} 53 & 66 & 61 & \cellcolor{blue!10} 53 & 56 \\
100 442& 48 & 56 & 53 & \cellcolor{blue!40} 52 & 68 & 53 & 53 & 55 \\
100 443& 53 & 58 & 56 & \cellcolor{blue!40} 55 & 66 & 56 & 56 & 57 \\
100 444& 51 & 56 & \cellcolor{blue!10} 54 & \cellcolor{blue!10} 54 & 99 & 56 & \cellcolor{blue!10} 54 & 55 \\
100 445& 54 & 57 & \cellcolor{blue!10} 56 & \cellcolor{blue!10} 56 & 65 & \cellcolor{blue!10} 56 & 57 & 57 \\
100 446& 54 & \cellcolor{blue!10} 57 & 58 & \cellcolor{blue!10} 57 & 73 & 58 & \cellcolor{blue!10} 57 & 59 \\
100 447& 52 & 55 & 55 & \cellcolor{blue!20} 54 & 64 & 63 & \cellcolor{blue!20} 54 & 56 \\
100 448& 54 & 60 & \cellcolor{blue!10} 56 & \cellcolor{blue!10} 56 & 84 & 58 & \cellcolor{blue!10} 56 & 58 \\
100 449& 52 & 57 & 56 & \cellcolor{blue!20} 55 & 71 & 56 & \cellcolor{blue!20} 55 & 56 \\
100 45& 14 & \cellcolor{blue!5} \textbf{14} & \cellcolor{blue!5} \textbf{14} & \cellcolor{blue!5} \textbf{14} & 95 & 89 & \cellcolor{blue!5} 14 & \cellcolor{blue!5} 14 \\
100 450& 53 & \cellcolor{blue!40} \textbf{53} & 55 & 54 & 68 & 56 & 54 & 56 \\
100 451& 26 & \cellcolor{blue!5} \textbf{26} & \cellcolor{blue!5} \textbf{26} & \cellcolor{blue!5} \textbf{26} & 30 & \cellcolor{blue!5} \textbf{26} & \cellcolor{blue!5} \textbf{26} & \cellcolor{blue!5} 26 \\
100 452& 22 & \cellcolor{blue!5} \textbf{22} & \cellcolor{blue!5} \textbf{22} & \cellcolor{blue!5} \textbf{22} & 24 & \cellcolor{blue!5} \textbf{22} & \cellcolor{blue!5} 22 & \cellcolor{blue!5} 22 \\
100 453& 24 & \cellcolor{blue!5} \textbf{24} & \cellcolor{blue!5} \textbf{24} & \cellcolor{blue!5} \textbf{24} & \cellcolor{red!20} n/a & \cellcolor{blue!5} \textbf{24} & \cellcolor{blue!5} \textbf{24} & \cellcolor{blue!5} 24 \\
100 454& 23 & \cellcolor{blue!5} \textbf{23} & \cellcolor{blue!5} \textbf{23} & \cellcolor{blue!5} \textbf{23} & \cellcolor{red!20} n/a & \cellcolor{blue!5} \textbf{23} & \cellcolor{blue!5} \textbf{23} & \cellcolor{blue!5} 23 \\
100 455& 23 & \cellcolor{blue!5} \textbf{23} & \cellcolor{blue!5} \textbf{23} & \cellcolor{blue!5} \textbf{23} & \cellcolor{red!20} n/a & \cellcolor{blue!5} \textbf{23} & \cellcolor{blue!5} \textbf{23} & \cellcolor{blue!5} 23 \\
100 456& 26 & \cellcolor{blue!5} \textbf{26} & \cellcolor{blue!5} \textbf{26} & \cellcolor{blue!5} \textbf{26} & 29 & \cellcolor{blue!5} \textbf{26} & \cellcolor{blue!5} \textbf{26} & \cellcolor{blue!5} 26 \\
100 457& 23 & \cellcolor{blue!5} \textbf{23} & \cellcolor{blue!5} \textbf{23} & \cellcolor{blue!5} \textbf{23} & \cellcolor{red!20} n/a & \cellcolor{blue!5} \textbf{23} & \cellcolor{blue!5} \textbf{23} & \cellcolor{blue!5} 23 \\
100 458& 24 & \cellcolor{blue!5} \textbf{24} & \cellcolor{blue!5} \textbf{24} & \cellcolor{blue!5} \textbf{24} & \cellcolor{red!20} n/a & \cellcolor{blue!5} \textbf{24} & \cellcolor{blue!5} \textbf{24} & \cellcolor{blue!5} 24 \\
100 459& 23 & \cellcolor{blue!5} \textbf{23} & \cellcolor{blue!5} \textbf{23} & \cellcolor{blue!5} \textbf{23} & \cellcolor{red!20} n/a & \cellcolor{blue!5} \textbf{23} & \cellcolor{blue!5} 23 & \cellcolor{blue!5} 23 \\
100 46& 14 & \cellcolor{blue!5} \textbf{14} & \cellcolor{blue!5} \textbf{14} & \cellcolor{blue!5} \textbf{14} & \cellcolor{blue!5} 14 & 73 & \cellcolor{blue!5} 14 & \cellcolor{blue!5} 14 \\
100 460& 23 & \cellcolor{blue!5} \textbf{23} & \cellcolor{blue!5} \textbf{23} & \cellcolor{blue!5} \textbf{23} & 26 & \cellcolor{blue!5} \textbf{23} & \cellcolor{blue!5} \textbf{23} & \cellcolor{blue!5} 23 \\
100 461& 23 & \cellcolor{blue!5} \textbf{23} & \cellcolor{blue!5} \textbf{23} & \cellcolor{blue!5} \textbf{23} & 31 & \cellcolor{blue!5} \textbf{23} & \cellcolor{blue!5} 23 & 24 \\
100 462& 23 & \cellcolor{blue!5} \textbf{23} & \cellcolor{blue!5} \textbf{23} & \cellcolor{blue!5} \textbf{23} & \cellcolor{red!20} n/a & \cellcolor{blue!5} \textbf{23} & \cellcolor{blue!5} \textbf{23} & \cellcolor{blue!5} 23 \\
100 463& 26 & \cellcolor{blue!5} \textbf{26} & \cellcolor{blue!5} \textbf{26} & \cellcolor{blue!5} \textbf{26} & 28 & \cellcolor{blue!5} \textbf{26} & \cellcolor{blue!5} \textbf{26} & \cellcolor{blue!5} 26 \\
100 464& 25 & \cellcolor{blue!5} \textbf{25} & \cellcolor{blue!5} \textbf{25} & \cellcolor{blue!5} \textbf{25} & 31 & \cellcolor{blue!5} \textbf{25} & \cellcolor{blue!5} \textbf{25} & \cellcolor{blue!5} 25 \\
100 465& 22 & \cellcolor{blue!5} \textbf{22} & \cellcolor{blue!5} \textbf{22} & \cellcolor{blue!5} \textbf{22} & 30 & \cellcolor{blue!5} \textbf{22} & \cellcolor{blue!5} \textbf{22} & \cellcolor{blue!5} 22 \\
100 466& 26 & \cellcolor{blue!5} \textbf{26} & \cellcolor{blue!5} \textbf{26} & \cellcolor{blue!5} \textbf{26} & 39 & \cellcolor{blue!5} \textbf{26} & \cellcolor{blue!5} \textbf{26} & \cellcolor{blue!5} 26 \\
100 467& 21 & \cellcolor{blue!5} \textbf{21} & \cellcolor{blue!5} \textbf{21} & \cellcolor{blue!5} \textbf{21} & \cellcolor{red!20} n/a & \cellcolor{blue!5} \textbf{21} & \cellcolor{blue!5} 21 & \cellcolor{blue!5} 21 \\
100 468& 25 & \cellcolor{blue!5} \textbf{25} & \cellcolor{blue!5} \textbf{25} & \cellcolor{blue!5} \textbf{25} & 32 & \cellcolor{blue!5} \textbf{25} & \cellcolor{blue!5} \textbf{25} & \cellcolor{blue!5} 25 \\
100 469& 22 & \cellcolor{blue!5} \textbf{22} & \cellcolor{blue!5} \textbf{22} & \cellcolor{blue!5} \textbf{22} & 24 & \cellcolor{blue!5} \textbf{22} & \cellcolor{blue!5} \textbf{22} & \cellcolor{blue!5} 22 \\
100 47& 14 & \cellcolor{blue!5} \textbf{14} & \cellcolor{blue!5} \textbf{14} & \cellcolor{blue!5} \textbf{14} & \cellcolor{red!20} n/a & 87 & \cellcolor{blue!5} 14 & \cellcolor{blue!5} 14 \\
100 470& 26 & \cellcolor{blue!5} \textbf{26} & \cellcolor{blue!5} 26 & \cellcolor{blue!5} \textbf{26} & \cellcolor{red!20} n/a & \cellcolor{blue!5} \textbf{26} & \cellcolor{blue!5} \textbf{26} & \cellcolor{blue!5} 26 \\
100 471& 26 & \cellcolor{blue!5} \textbf{26} & \cellcolor{blue!5} \textbf{26} & \cellcolor{blue!5} \textbf{26} & 88 & \cellcolor{blue!5} \textbf{26} & \cellcolor{blue!5} 26 & \cellcolor{blue!5} 26 \\
100 472& 23 & \cellcolor{blue!5} \textbf{23} & \cellcolor{blue!5} \textbf{23} & \cellcolor{blue!5} \textbf{23} & 32 & \cellcolor{blue!5} \textbf{23} & \cellcolor{blue!5} \textbf{23} & \cellcolor{blue!5} 23 \\
100 473& 28 & \cellcolor{blue!5} \textbf{28} & \cellcolor{blue!5} \textbf{28} & \cellcolor{blue!5} \textbf{28} & \cellcolor{red!20} n/a & \cellcolor{blue!5} \textbf{28} & \cellcolor{blue!5} 28 & \cellcolor{blue!5} 28 \\
100 474& 23 & \cellcolor{blue!5} \textbf{23} & \cellcolor{blue!5} \textbf{23} & \cellcolor{blue!5} \textbf{23} & 25 & \cellcolor{blue!5} \textbf{23} & \cellcolor{blue!5} \textbf{23} & \cellcolor{blue!5} 23 \\
100 475& 24 & \cellcolor{blue!5} \textbf{24} & \cellcolor{blue!5} 24 & \cellcolor{blue!5} \textbf{24} & \cellcolor{red!20} n/a & \cellcolor{blue!5} \textbf{24} & \cellcolor{blue!5} \textbf{24} & \cellcolor{blue!5} 24 \\
100 476& 14 & \cellcolor{blue!5} \textbf{14} & \cellcolor{blue!5} \textbf{14} & \cellcolor{blue!5} \textbf{14} & 15 & \cellcolor{blue!5} \textbf{14} & \cellcolor{blue!5} \textbf{14} & \cellcolor{blue!5} 14 \\
100 477& 14 & \cellcolor{blue!5} \textbf{14} & \cellcolor{blue!5} \textbf{14} & \cellcolor{blue!5} \textbf{14} & \cellcolor{red!20} n/a & \cellcolor{blue!5} \textbf{14} & \cellcolor{blue!5} \textbf{14} & \cellcolor{blue!5} 14 \\
100 478& 14 & \cellcolor{blue!5} \textbf{14} & \cellcolor{blue!5} \textbf{14} & \cellcolor{blue!5} \textbf{14} & \cellcolor{red!20} n/a & \cellcolor{blue!5} \textbf{14} & \cellcolor{blue!5} \textbf{14} & \cellcolor{blue!5} \textbf{14} \\
100 479& 16 & \cellcolor{blue!5} \textbf{16} & \cellcolor{blue!5} \textbf{16} & \cellcolor{blue!5} \textbf{16} & \cellcolor{red!20} n/a & \cellcolor{blue!5} \textbf{16} & \cellcolor{blue!5} \textbf{16} & \cellcolor{blue!5} \textbf{16} \\
100 48& 15 & \cellcolor{blue!5} \textbf{15} & \cellcolor{blue!5} \textbf{15} & \cellcolor{blue!5} \textbf{15} & 16 & 73 & \cellcolor{blue!5} 15 & \cellcolor{blue!5} 15 \\
100 480& 15 & \cellcolor{blue!5} \textbf{15} & \cellcolor{blue!5} \textbf{15} & \cellcolor{blue!5} \textbf{15} & \cellcolor{red!20} n/a & \cellcolor{blue!5} \textbf{15} & \cellcolor{blue!5} \textbf{15} & \cellcolor{blue!5} \textbf{15} \\
100 481& 15 & \cellcolor{blue!5} \textbf{15} & \cellcolor{blue!5} \textbf{15} & \cellcolor{blue!5} \textbf{15} & \cellcolor{red!20} n/a & \cellcolor{blue!5} \textbf{15} & \cellcolor{blue!5} \textbf{15} & \cellcolor{blue!5} 15 \\
100 482& 15 & \cellcolor{blue!5} \textbf{15} & \cellcolor{blue!5} \textbf{15} & \cellcolor{blue!5} \textbf{15} & \cellcolor{red!20} n/a & \cellcolor{blue!5} \textbf{15} & \cellcolor{blue!5} \textbf{15} & \cellcolor{blue!5} \textbf{15} \\
100 483& 14 & \cellcolor{blue!5} \textbf{14} & \cellcolor{blue!5} \textbf{14} & \cellcolor{blue!5} \textbf{14} & 26 & \cellcolor{blue!5} \textbf{14} & \cellcolor{blue!5} \textbf{14} & \cellcolor{blue!5} 14 \\
100 484& 14 & \cellcolor{blue!5} \textbf{14} & \cellcolor{blue!5} \textbf{14} & \cellcolor{blue!5} \textbf{14} & \cellcolor{red!20} n/a & \cellcolor{blue!5} \textbf{14} & \cellcolor{blue!5} \textbf{14} & \cellcolor{blue!5} \textbf{14} \\
100 485& 16 & \cellcolor{blue!5} \textbf{16} & \cellcolor{blue!5} \textbf{16} & \cellcolor{blue!5} \textbf{16} & 20 & \cellcolor{blue!5} \textbf{16} & \cellcolor{blue!5} \textbf{16} & \cellcolor{blue!5} \textbf{16} \\
100 486& 15 & \cellcolor{blue!5} \textbf{15} & \cellcolor{blue!5} \textbf{15} & \cellcolor{blue!5} \textbf{15} & \cellcolor{red!20} n/a & \cellcolor{blue!5} \textbf{15} & \cellcolor{blue!5} \textbf{15} & \cellcolor{blue!5} 15 \\
100 487& 15 & \cellcolor{blue!5} \textbf{15} & \cellcolor{blue!5} \textbf{15} & \cellcolor{blue!5} \textbf{15} & \cellcolor{red!20} n/a & \cellcolor{blue!5} \textbf{15} & \cellcolor{blue!5} \textbf{15} & \cellcolor{blue!5} \textbf{15} \\
100 488& 16 & \cellcolor{blue!5} \textbf{16} & \cellcolor{blue!5} \textbf{16} & \cellcolor{blue!5} \textbf{16} & \cellcolor{red!20} n/a & \cellcolor{blue!5} \textbf{16} & \cellcolor{blue!5} \textbf{16} & \cellcolor{blue!5} 16 \\
100 489& 13 & \cellcolor{blue!5} \textbf{13} & \cellcolor{blue!5} \textbf{13} & \cellcolor{blue!5} \textbf{13} & 33 & \cellcolor{blue!5} \textbf{13} & \cellcolor{blue!5} \textbf{13} & \cellcolor{blue!5} 13 \\
100 49& 14 & \cellcolor{blue!5} \textbf{14} & \cellcolor{blue!5} \textbf{14} & \cellcolor{blue!5} \textbf{14} & 15 & 97 & \cellcolor{blue!5} 14 & \cellcolor{blue!5} 14 \\
100 490& 15 & \cellcolor{blue!5} \textbf{15} & \cellcolor{blue!5} \textbf{15} & \cellcolor{blue!5} \textbf{15} & \cellcolor{red!20} n/a & \cellcolor{blue!5} \textbf{15} & \cellcolor{blue!5} \textbf{15} & \cellcolor{blue!5} 15 \\
100 491& 16 & \cellcolor{blue!5} \textbf{16} & \cellcolor{blue!5} \textbf{16} & \cellcolor{blue!5} \textbf{16} & 28 & \cellcolor{blue!5} \textbf{16} & \cellcolor{blue!5} \textbf{16} & \cellcolor{blue!5} \textbf{16} \\
100 492& 14 & \cellcolor{blue!5} \textbf{14} & \cellcolor{blue!5} \textbf{14} & \cellcolor{blue!5} \textbf{14} & 95 & \cellcolor{blue!5} \textbf{14} & \cellcolor{blue!5} \textbf{14} & \cellcolor{blue!5} \textbf{14} \\
100 493& 14 & \cellcolor{blue!5} \textbf{14} & \cellcolor{blue!5} \textbf{14} & \cellcolor{blue!5} \textbf{14} & \cellcolor{red!20} n/a & \cellcolor{blue!5} \textbf{14} & \cellcolor{blue!5} \textbf{14} & \cellcolor{blue!5} 14 \\
100 494& 14 & \cellcolor{blue!5} \textbf{14} & \cellcolor{blue!5} \textbf{14} & \cellcolor{blue!5} \textbf{14} & \cellcolor{red!20} n/a & \cellcolor{blue!5} \textbf{14} & \cellcolor{blue!5} \textbf{14} & \cellcolor{blue!5} 14 \\
100 495& 15 & \cellcolor{blue!5} \textbf{15} & \cellcolor{blue!5} \textbf{15} & \cellcolor{blue!5} \textbf{15} & 18 & \cellcolor{blue!5} \textbf{15} & \cellcolor{blue!5} \textbf{15} & \cellcolor{blue!5} 15 \\
100 496& 14 & \cellcolor{blue!5} \textbf{14} & \cellcolor{blue!5} \textbf{14} & \cellcolor{blue!5} \textbf{14} & 24 & \cellcolor{blue!5} \textbf{14} & \cellcolor{blue!5} \textbf{14} & \cellcolor{blue!5} 14 \\
100 497& 13 & \cellcolor{blue!5} \textbf{13} & \cellcolor{blue!5} \textbf{13} & \cellcolor{blue!5} \textbf{13} & 64 & \cellcolor{blue!5} \textbf{13} & \cellcolor{blue!5} \textbf{13} & \cellcolor{blue!5} 13 \\
100 498& 14 & \cellcolor{blue!5} \textbf{14} & \cellcolor{blue!5} \textbf{14} & \cellcolor{blue!5} \textbf{14} & \cellcolor{red!20} n/a & \cellcolor{blue!5} \textbf{14} & \cellcolor{blue!5} \textbf{14} & \cellcolor{blue!5} 14 \\
100 499& 14 & \cellcolor{blue!5} \textbf{14} & \cellcolor{blue!5} \textbf{14} & \cellcolor{blue!5} \textbf{14} & 25 & \cellcolor{blue!5} \textbf{14} & \cellcolor{blue!5} \textbf{14} & \cellcolor{blue!5} 14 \\
100 5& 22 & \cellcolor{blue!5} \textbf{22} & \cellcolor{blue!5} \textbf{22} & \cellcolor{blue!5} \textbf{22} & 24 & 92 & \cellcolor{blue!5} 22 & \cellcolor{blue!5} 22 \\
100 50& 14 & \cellcolor{blue!5} \textbf{14} & \cellcolor{blue!5} \textbf{14} & \cellcolor{blue!5} \textbf{14} & \cellcolor{blue!5} 14 & 57 & \cellcolor{blue!5} 14 & \cellcolor{blue!5} 14 \\
100 500& 14 & \cellcolor{blue!5} \textbf{14} & \cellcolor{blue!5} \textbf{14} & \cellcolor{blue!5} \textbf{14} & 17 & \cellcolor{blue!5} \textbf{14} & \cellcolor{blue!5} \textbf{14} & \cellcolor{blue!5} 14 \\
100 501& 58 & 63 & 63 & \cellcolor{blue!20} \textbf{62} & 67 & \cellcolor{blue!20} \textbf{62} & 63 & 64 \\
100 502& 60 & 67 & \cellcolor{blue!5} 64 & \cellcolor{blue!5} \textbf{64} & 69 & \cellcolor{blue!5} \textbf{64} & \cellcolor{blue!5} 64 & 66 \\
100 503& 55 & 62 & \cellcolor{blue!10} 60 & \cellcolor{blue!10} \textbf{60} & 65 & \cellcolor{blue!10} \textbf{60} & 61 & 62 \\
100 504& 55 & 62 & \cellcolor{blue!5} 60 & \cellcolor{blue!5} \textbf{60} & 64 & \cellcolor{blue!5} \textbf{60} & \cellcolor{blue!5} 60 & \cellcolor{blue!5} 60 \\
100 505& 55 & 62 & \cellcolor{blue!5} 61 & \cellcolor{blue!5} \textbf{61} & 63 & \cellcolor{blue!5} \textbf{61} & \cellcolor{blue!5} 61 & 63 \\
100 506& 54 & 58 & 59 & \cellcolor{blue!20} \textbf{57} & 64 & \cellcolor{blue!20} \textbf{57} & 58 & 60 \\
100 507& 55 & 60 & \cellcolor{blue!5} 59 & \cellcolor{blue!5} \textbf{59} & 62 & \cellcolor{blue!5} \textbf{59} & \cellcolor{blue!5} 59 & 60 \\
100 508& 53 & \cellcolor{blue!5} 56 & \cellcolor{blue!5} 56 & \cellcolor{blue!5} \textbf{56} & 61 & \cellcolor{blue!5} \textbf{56} & \cellcolor{blue!5} 56 & \cellcolor{blue!5} 56 \\
100 509& 53 & 60 & \cellcolor{blue!10} 57 & \cellcolor{blue!10} \textbf{57} & 63 & \cellcolor{blue!10} \textbf{57} & 58 & 60 \\
100 51& 48 & 51 & 51 & \cellcolor{blue!20} 50 & \cellcolor{red!20} n/a & 83 & \cellcolor{blue!20} 50 & 51 \\
100 510& 54 & 60 & \cellcolor{blue!5} 58 & \cellcolor{blue!5} \textbf{58} & 63 & \cellcolor{blue!5} \textbf{58} & \cellcolor{blue!5} 58 & \cellcolor{blue!5} 58 \\
100 511& 56 & \cellcolor{blue!10} 59 & 60 & \cellcolor{blue!10} \textbf{59} & 61 & \cellcolor{blue!10} \textbf{59} & 60 & 60 \\
100 512& 57 & 61 & \cellcolor{blue!5} 60 & \cellcolor{blue!5} \textbf{60} & 66 & \cellcolor{blue!5} \textbf{60} & \cellcolor{blue!5} 60 & 62 \\
100 513& 55 & 63 & \cellcolor{blue!5} 62 & \cellcolor{blue!5} \textbf{62} & 64 & \cellcolor{blue!5} \textbf{62} & \cellcolor{blue!5} 62 & 64 \\
100 514& 54 & 60 & \cellcolor{blue!5} 58 & \cellcolor{blue!5} \textbf{58} & 66 & \cellcolor{blue!5} \textbf{58} & \cellcolor{blue!5} 58 & 59 \\
100 515& 56 & 63 & \cellcolor{blue!10} 61 & \cellcolor{blue!10} \textbf{61} & 65 & \cellcolor{blue!10} \textbf{61} & 62 & 64 \\
100 516& 67 & \cellcolor{blue!5} 70 & \cellcolor{blue!5} 70 & \cellcolor{blue!5} \textbf{70} & 78 & \cellcolor{blue!5} \textbf{70} & \cellcolor{blue!5} 70 & \cellcolor{blue!5} 70 \\
100 517& 57 & \cellcolor{blue!5} 62 & \cellcolor{blue!5} 62 & \cellcolor{blue!5} \textbf{62} & 64 & \cellcolor{blue!5} \textbf{62} & \cellcolor{blue!5} 62 & \cellcolor{blue!5} 62 \\
100 518& 51 & 58 & \cellcolor{blue!5} 57 & \cellcolor{blue!5} \textbf{57} & 67 & \cellcolor{blue!5} \textbf{57} & \cellcolor{blue!5} 57 & 59 \\
100 519& 57 & 63 & \cellcolor{blue!5} 61 & \cellcolor{blue!5} \textbf{61} & 69 & \cellcolor{blue!5} \textbf{61} & \cellcolor{blue!5} 61 & 63 \\
100 52& 52 & 54 & \cellcolor{blue!10} 53 & \cellcolor{blue!10} 53 & 71 & 85 & \cellcolor{blue!10} 53 & 54 \\
100 520& 54 & 62 & \cellcolor{blue!5} 60 & \cellcolor{blue!5} \textbf{60} & 64 & \cellcolor{blue!5} \textbf{60} & \cellcolor{blue!5} 60 & 61 \\
100 521& 65 & \cellcolor{blue!5} 70 & \cellcolor{blue!5} 70 & \cellcolor{blue!5} \textbf{70} & 75 & \cellcolor{blue!5} \textbf{70} & \cellcolor{blue!5} 70 & \cellcolor{blue!5} 70 \\
100 522& 54 & 60 & \cellcolor{blue!5} 59 & \cellcolor{blue!5} \textbf{59} & 65 & \cellcolor{blue!5} \textbf{59} & \cellcolor{blue!5} 59 & 60 \\
100 523& 52 & 58 & \cellcolor{blue!5} 55 & \cellcolor{blue!5} \textbf{55} & 61 & \cellcolor{blue!5} \textbf{55} & \cellcolor{blue!5} 55 & 57 \\
100 524& 53 & \cellcolor{blue!5} 59 & \cellcolor{blue!5} 59 & \cellcolor{blue!5} \textbf{59} & 68 & \cellcolor{blue!5} \textbf{59} & \cellcolor{blue!5} 59 & 60 \\
100 525& 55 & \cellcolor{blue!5} 62 & \cellcolor{blue!5} 62 & \cellcolor{blue!5} \textbf{62} & 66 & \cellcolor{blue!5} \textbf{62} & \cellcolor{blue!5} 62 & \cellcolor{blue!5} 62 \\
100 53& 52 & 53 & 53 & \cellcolor{blue!20} \textbf{52} & 88 & 89 & \cellcolor{blue!20} 52 & 53 \\
100 54& 51 & 52 & 52 & 52 & \cellcolor{red!20} n/a & 95 & \cellcolor{blue!40} 51 & 52 \\
100 55& 52 & 54 & 54 & \cellcolor{blue!40} \textbf{52} & \cellcolor{red!20} n/a & 89 & 53 & 55 \\
100 56& 51 & 53 & 53 & 53 & 68 & 75 & \cellcolor{blue!40} 52 & 53 \\
100 57& 53 & 58 & \cellcolor{blue!10} 55 & \cellcolor{blue!10} 55 & 100 & 89 & \cellcolor{blue!10} 55 & 56 \\
100 58& 56 & 58 & 58 & \cellcolor{blue!20} 57 & 86 & 92 & \cellcolor{blue!20} 57 & 58 \\
100 59& 57 & \cellcolor{blue!20} \textbf{57} & 58 & \cellcolor{blue!20} \textbf{57} & 84 & 91 & 58 & 59 \\
100 6& 22 & \cellcolor{blue!5} \textbf{22} & \cellcolor{blue!5} \textbf{22} & \cellcolor{blue!5} \textbf{22} & \cellcolor{red!20} n/a & 83 & \cellcolor{blue!5} 22 & \cellcolor{blue!5} 22 \\
100 60& 53 & 57 & \cellcolor{blue!10} 54 & \cellcolor{blue!10} 54 & \cellcolor{red!20} n/a & 76 & \cellcolor{blue!10} 54 & 55 \\
100 61& 54 & \cellcolor{blue!20} \textbf{54} & 56 & \cellcolor{blue!20} \textbf{54} & 76 & 79 & 56 & 57 \\
100 62& 50 & 53 & 53 & \cellcolor{blue!20} 52 & 100 & 57 & \cellcolor{blue!20} 52 & 53 \\
100 63& 61 & \cellcolor{blue!5} \textbf{61} & \cellcolor{blue!5} 61 & \cellcolor{blue!5} \textbf{61} & \cellcolor{red!20} n/a & 96 & \cellcolor{blue!5} 61 & \cellcolor{blue!5} 61 \\
100 64& 55 & \cellcolor{blue!10} 56 & 57 & \cellcolor{blue!10} 56 & 71 & 88 & \cellcolor{blue!10} 56 & 57 \\
100 65& 61 & \cellcolor{blue!20} \textbf{61} & 62 & \cellcolor{blue!20} \textbf{61} & 66 & 83 & 62 & 63 \\
100 66& 50 & 52 & 52 & 52 & 78 & 98 & \cellcolor{blue!40} 51 & 52 \\
100 67& 54 & 56 & 56 & \cellcolor{blue!20} 55 & \cellcolor{red!20} n/a & 87 & \cellcolor{blue!20} 55 & 57 \\
100 68& 57 & \cellcolor{blue!5} \textbf{57} & \cellcolor{blue!5} 57 & \cellcolor{blue!5} \textbf{57} & 99 & 86 & \cellcolor{blue!5} 57 & \cellcolor{blue!5} 57 \\
100 69& 53 & 55 & 54 & \cellcolor{blue!20} \textbf{53} & 81 & 99 & \cellcolor{blue!20} 53 & 54 \\
100 7& 26 & \cellcolor{blue!5} \textbf{26} & \cellcolor{blue!5} \textbf{26} & \cellcolor{blue!5} \textbf{26} & \cellcolor{red!20} n/a & 53 & \cellcolor{blue!5} 26 & \cellcolor{blue!5} 26 \\
100 70& 51 & 56 & 55 & \cellcolor{blue!40} 53 & \cellcolor{red!20} n/a & 88 & 54 & 57 \\
100 71& 52 & \cellcolor{blue!10} 53 & 54 & \cellcolor{blue!10} 53 & 95 & 91 & \cellcolor{blue!10} 53 & 54 \\
100 72& 52 & 56 & 55 & \cellcolor{blue!20} 54 & 77 & 82 & \cellcolor{blue!20} 54 & 56 \\
100 73& 55 & 58 & \cellcolor{blue!10} 56 & \cellcolor{blue!10} 56 & 71 & 89 & \cellcolor{blue!10} 56 & 58 \\
100 74& 50 & \cellcolor{blue!10} 52 & 53 & \cellcolor{blue!10} 52 & \cellcolor{red!20} n/a & 93 & \cellcolor{blue!10} 52 & 53 \\
100 75& 54 & \cellcolor{blue!10} 55 & 56 & \cellcolor{blue!10} 55 & 85 & 87 & \cellcolor{blue!10} 55 & 56 \\
100 76& 23 & \cellcolor{blue!5} \textbf{23} & \cellcolor{blue!5} \textbf{23} & \cellcolor{blue!5} \textbf{23} & \cellcolor{red!20} n/a & 86 & \cellcolor{blue!5} 23 & \cellcolor{blue!5} 23 \\
100 77& 20 & \cellcolor{blue!5} \textbf{20} & \cellcolor{blue!5} \textbf{20} & \cellcolor{blue!5} \textbf{20} & \cellcolor{red!20} n/a & 71 & \cellcolor{blue!5} 20 & \cellcolor{blue!5} 20 \\
100 78& 21 & \cellcolor{blue!5} \textbf{21} & \cellcolor{blue!5} \textbf{21} & \cellcolor{blue!5} \textbf{21} & 65 & 43 & \cellcolor{blue!5} 21 & 22 \\
100 79& 21 & \cellcolor{blue!5} \textbf{21} & \cellcolor{blue!5} \textbf{21} & \cellcolor{blue!5} \textbf{21} & \cellcolor{red!20} n/a & 73 & \cellcolor{blue!5} 21 & \cellcolor{blue!5} 21 \\
100 8& 24 & \cellcolor{blue!5} \textbf{24} & \cellcolor{blue!5} \textbf{24} & \cellcolor{blue!5} \textbf{24} & \cellcolor{red!20} n/a & 69 & \cellcolor{blue!5} 24 & \cellcolor{blue!5} 24 \\
100 80& 22 & \cellcolor{blue!10} \textbf{22} & \cellcolor{blue!10} \textbf{22} & 23 & 99 & 23 & \cellcolor{blue!10} 22 & 23 \\
100 81& 20 & \cellcolor{blue!5} \textbf{20} & \cellcolor{blue!5} \textbf{20} & \cellcolor{blue!5} \textbf{20} & \cellcolor{red!20} n/a & 54 & \cellcolor{blue!5} 20 & \cellcolor{blue!5} 20 \\
100 82& 21 & \cellcolor{blue!5} \textbf{21} & \cellcolor{blue!5} \textbf{21} & \cellcolor{blue!5} \textbf{21} & \cellcolor{red!20} n/a & 90 & \cellcolor{blue!5} 21 & \cellcolor{blue!5} 21 \\
100 83& 22 & \cellcolor{blue!5} \textbf{22} & \cellcolor{blue!5} \textbf{22} & \cellcolor{blue!5} \textbf{22} & \cellcolor{red!20} n/a & 90 & \cellcolor{blue!5} 22 & \cellcolor{blue!5} 22 \\
100 84& 26 & \cellcolor{blue!5} 27 & \cellcolor{blue!5} 27 & \cellcolor{blue!5} 27 & 45 & 86 & \cellcolor{blue!5} 27 & \cellcolor{blue!5} 27 \\
100 85& 24 & \cellcolor{blue!40} \textbf{24} & 25 & 25 & \cellcolor{red!20} n/a & 25 & 25 & 25 \\
100 86& 23 & \cellcolor{blue!5} \textbf{23} & \cellcolor{blue!5} \textbf{23} & \cellcolor{blue!5} \textbf{23} & \cellcolor{red!20} n/a & \cellcolor{blue!5} 23 & \cellcolor{blue!5} 23 & \cellcolor{blue!5} 23 \\
100 87& 22 & \cellcolor{blue!5} \textbf{22} & \cellcolor{blue!5} \textbf{22} & \cellcolor{blue!5} \textbf{22} & \cellcolor{red!20} n/a & 74 & \cellcolor{blue!5} 22 & \cellcolor{blue!5} 22 \\
100 88& 23 & \cellcolor{blue!40} \textbf{23} & 24 & 24 & \cellcolor{red!20} n/a & 43 & 24 & 24 \\
100 89& 24 & \cellcolor{blue!5} \textbf{24} & \cellcolor{blue!5} \textbf{24} & \cellcolor{blue!5} \textbf{24} & 73 & 45 & \cellcolor{blue!5} 24 & \cellcolor{blue!5} 24 \\
100 9& 23 & \cellcolor{blue!20} \textbf{23} & 24 & 24 & 25 & 80 & \cellcolor{blue!20} 23 & 24 \\
100 90& 20 & \cellcolor{blue!40} \textbf{20} & 21 & 21 & \cellcolor{red!20} n/a & 75 & 21 & 21 \\
100 91& 25 & \cellcolor{blue!5} \textbf{25} & \cellcolor{blue!5} \textbf{25} & \cellcolor{blue!5} \textbf{25} & \cellcolor{red!20} n/a & 77 & \cellcolor{blue!5} 25 & \cellcolor{blue!5} 25 \\
100 92& 24 & \cellcolor{blue!5} \textbf{24} & \cellcolor{blue!5} \textbf{24} & \cellcolor{blue!5} \textbf{24} & \cellcolor{red!20} n/a & \cellcolor{blue!5} 24 & \cellcolor{blue!5} 24 & \cellcolor{blue!5} 24 \\
100 93& 27 & 28 & \cellcolor{blue!10} \textbf{27} & \cellcolor{blue!10} \textbf{27} & \cellcolor{red!20} n/a & 28 & \cellcolor{blue!10} 27 & 28 \\
100 94& 22 & \cellcolor{blue!10} \textbf{22} & \cellcolor{blue!10} \textbf{22} & \cellcolor{blue!10} \textbf{22} & 57 & 91 & 23 & 23 \\
100 95& 21 & \cellcolor{blue!5} \textbf{21} & \cellcolor{blue!5} \textbf{21} & \cellcolor{blue!5} \textbf{21} & \cellcolor{red!20} n/a & 22 & \cellcolor{blue!5} 21 & \cellcolor{blue!5} 21 \\
100 96& 21 & \cellcolor{blue!5} \textbf{21} & \cellcolor{blue!5} \textbf{21} & \cellcolor{blue!5} \textbf{21} & 100 & 22 & \cellcolor{blue!5} 21 & 22 \\
100 97& 22 & \cellcolor{blue!5} \textbf{22} & \cellcolor{blue!5} \textbf{22} & \cellcolor{blue!5} \textbf{22} & \cellcolor{red!20} n/a & \cellcolor{blue!5} 22 & \cellcolor{blue!5} 22 & \cellcolor{blue!5} 22 \\
100 98& 22 & \cellcolor{blue!5} \textbf{22} & \cellcolor{blue!5} \textbf{22} & \cellcolor{blue!5} \textbf{22} & \cellcolor{red!20} n/a & 63 & \cellcolor{blue!5} 22 & \cellcolor{blue!5} 22 \\
100 99& 22 & \cellcolor{blue!5} \textbf{22} & \cellcolor{blue!5} \textbf{22} & \cellcolor{blue!5} \textbf{22} & \cellcolor{red!20} n/a & 32 & \cellcolor{blue!5} 22 & \cellcolor{blue!5} 22 \\
20 1& 3 & \cellcolor{blue!5} \textbf{3} & \cellcolor{blue!5} \textbf{3} & \cellcolor{blue!5} \textbf{3} & \cellcolor{blue!5} \textbf{3} & \cellcolor{blue!5} \textbf{3} & \cellcolor{blue!5} \textbf{3} & \cellcolor{blue!5} \textbf{3} \\
20 10& 3 & \cellcolor{blue!5} \textbf{3} & \cellcolor{blue!5} \textbf{3} & \cellcolor{blue!5} \textbf{3} & \cellcolor{blue!5} \textbf{3} & \cellcolor{blue!5} \textbf{3} & \cellcolor{blue!5} \textbf{3} & \cellcolor{blue!5} \textbf{3} \\
20 100& 11 & \cellcolor{blue!5} \textbf{11} & \cellcolor{blue!5} \textbf{11} & \cellcolor{blue!5} \textbf{11} & \cellcolor{blue!5} \textbf{11} & \cellcolor{blue!5} \textbf{11} & \cellcolor{blue!5} \textbf{11} & \cellcolor{blue!5} 11 \\
20 101& 13 & \cellcolor{blue!5} \textbf{13} & \cellcolor{blue!5} \textbf{13} & \cellcolor{blue!5} \textbf{13} & \cellcolor{blue!5} \textbf{13} & \cellcolor{blue!5} \textbf{13} & \cellcolor{blue!5} \textbf{13} & \cellcolor{blue!5} 13 \\
20 102& 11 & \cellcolor{blue!5} 13 & \cellcolor{blue!5} \textbf{13} & \cellcolor{blue!5} \textbf{13} & \cellcolor{blue!5} \textbf{13} & \cellcolor{blue!5} \textbf{13} & \cellcolor{blue!5} \textbf{13} & \cellcolor{blue!5} 13 \\
20 103& 12 & \cellcolor{blue!5} \textbf{12} & \cellcolor{blue!5} \textbf{12} & \cellcolor{blue!5} \textbf{12} & \cellcolor{blue!5} \textbf{12} & \cellcolor{blue!5} \textbf{12} & \cellcolor{blue!5} \textbf{12} & \cellcolor{blue!5} \textbf{12} \\
20 104& 11 & \cellcolor{blue!5} \textbf{11} & \cellcolor{blue!5} \textbf{11} & \cellcolor{blue!5} \textbf{11} & \cellcolor{blue!5} \textbf{11} & \cellcolor{blue!5} \textbf{11} & \cellcolor{blue!5} \textbf{11} & \cellcolor{blue!5} \textbf{11} \\
20 105& 12 & \cellcolor{blue!5} \textbf{12} & \cellcolor{blue!5} \textbf{12} & \cellcolor{blue!5} \textbf{12} & \cellcolor{blue!5} \textbf{12} & \cellcolor{blue!5} \textbf{12} & \cellcolor{blue!5} \textbf{12} & \cellcolor{blue!5} 12 \\
20 106& 10 & \cellcolor{blue!5} \textbf{10} & \cellcolor{blue!5} \textbf{10} & \cellcolor{blue!5} \textbf{10} & \cellcolor{blue!5} \textbf{10} & \cellcolor{blue!5} \textbf{10} & \cellcolor{blue!5} \textbf{10} & \cellcolor{blue!5} \textbf{10} \\
20 107& 14 & \cellcolor{blue!5} \textbf{14} & \cellcolor{blue!5} \textbf{14} & \cellcolor{blue!5} \textbf{14} & \cellcolor{blue!5} \textbf{14} & \cellcolor{blue!5} \textbf{14} & \cellcolor{blue!5} \textbf{14} & \cellcolor{blue!5} 14 \\
20 108& 15 & \cellcolor{blue!5} \textbf{15} & \cellcolor{blue!5} \textbf{15} & \cellcolor{blue!5} \textbf{15} & \cellcolor{blue!5} \textbf{15} & \cellcolor{blue!5} \textbf{15} & \cellcolor{blue!5} \textbf{15} & \cellcolor{blue!5} 15 \\
20 109& 12 & \cellcolor{blue!5} \textbf{12} & \cellcolor{blue!5} \textbf{12} & \cellcolor{blue!5} \textbf{12} & \cellcolor{blue!5} \textbf{12} & \cellcolor{blue!5} \textbf{12} & \cellcolor{blue!5} \textbf{12} & \cellcolor{blue!5} \textbf{12} \\
20 11& 3 & \cellcolor{blue!5} \textbf{3} & \cellcolor{blue!5} \textbf{3} & \cellcolor{blue!5} \textbf{3} & \cellcolor{blue!5} \textbf{3} & \cellcolor{blue!5} \textbf{3} & \cellcolor{blue!5} \textbf{3} & \cellcolor{blue!5} \textbf{3} \\
20 110& 11 & \cellcolor{blue!5} \textbf{11} & \cellcolor{blue!5} \textbf{11} & \cellcolor{blue!5} \textbf{11} & \cellcolor{blue!5} \textbf{11} & \cellcolor{blue!5} \textbf{11} & \cellcolor{blue!5} \textbf{11} & \cellcolor{blue!5} \textbf{11} \\
20 111& 13 & \cellcolor{blue!5} \textbf{13} & \cellcolor{blue!5} \textbf{13} & \cellcolor{blue!5} \textbf{13} & \cellcolor{blue!5} \textbf{13} & \cellcolor{blue!5} \textbf{13} & \cellcolor{blue!5} \textbf{13} & \cellcolor{blue!5} 13 \\
20 112& 11 & \cellcolor{blue!5} \textbf{11} & \cellcolor{blue!5} \textbf{11} & \cellcolor{blue!5} \textbf{11} & \cellcolor{blue!5} \textbf{11} & \cellcolor{blue!5} \textbf{11} & \cellcolor{blue!5} \textbf{11} & \cellcolor{blue!5} \textbf{11} \\
20 113& 12 & \cellcolor{blue!5} \textbf{12} & \cellcolor{blue!5} \textbf{12} & \cellcolor{blue!5} \textbf{12} & \cellcolor{blue!5} \textbf{12} & \cellcolor{blue!5} \textbf{12} & \cellcolor{blue!5} \textbf{12} & \cellcolor{blue!5} \textbf{12} \\
20 114& 13 & \cellcolor{blue!5} \textbf{13} & \cellcolor{blue!5} \textbf{13} & \cellcolor{blue!5} \textbf{13} & \cellcolor{blue!5} \textbf{13} & \cellcolor{blue!5} \textbf{13} & \cellcolor{blue!5} \textbf{13} & \cellcolor{blue!5} \textbf{13} \\
20 115& 11 & \cellcolor{blue!5} \textbf{11} & \cellcolor{blue!5} \textbf{11} & \cellcolor{blue!5} \textbf{11} & \cellcolor{blue!5} \textbf{11} & \cellcolor{blue!5} \textbf{11} & \cellcolor{blue!5} \textbf{11} & \cellcolor{blue!5} \textbf{11} \\
20 116& 5 & \cellcolor{blue!5} \textbf{5} & \cellcolor{blue!5} \textbf{5} & \cellcolor{blue!5} \textbf{5} & \cellcolor{blue!5} \textbf{5} & \cellcolor{blue!5} \textbf{5} & \cellcolor{blue!5} \textbf{5} & \cellcolor{blue!5} \textbf{5} \\
20 117& 5 & \cellcolor{blue!5} \textbf{5} & \cellcolor{blue!5} \textbf{5} & \cellcolor{blue!5} \textbf{5} & \cellcolor{blue!5} \textbf{5} & \cellcolor{blue!5} \textbf{5} & \cellcolor{blue!5} \textbf{5} & \cellcolor{blue!5} \textbf{5} \\
20 118& 5 & \cellcolor{blue!5} \textbf{5} & \cellcolor{blue!5} \textbf{5} & \cellcolor{blue!5} \textbf{5} & \cellcolor{blue!5} \textbf{5} & \cellcolor{blue!5} \textbf{5} & \cellcolor{blue!5} \textbf{5} & \cellcolor{blue!5} \textbf{5} \\
20 119& 6 & \cellcolor{blue!5} \textbf{6} & \cellcolor{blue!5} \textbf{6} & \cellcolor{blue!5} \textbf{6} & \cellcolor{blue!5} \textbf{6} & \cellcolor{blue!5} \textbf{6} & \cellcolor{blue!5} \textbf{6} & \cellcolor{blue!5} \textbf{6} \\
20 12& 3 & \cellcolor{blue!5} \textbf{3} & \cellcolor{blue!5} \textbf{3} & \cellcolor{blue!5} \textbf{3} & \cellcolor{blue!5} \textbf{3} & \cellcolor{blue!5} \textbf{3} & \cellcolor{blue!5} \textbf{3} & \cellcolor{blue!5} \textbf{3} \\
20 120& 6 & \cellcolor{blue!5} \textbf{6} & \cellcolor{blue!5} \textbf{6} & \cellcolor{blue!5} \textbf{6} & \cellcolor{blue!5} \textbf{6} & \cellcolor{blue!5} \textbf{6} & \cellcolor{blue!5} \textbf{6} & \cellcolor{blue!5} \textbf{6} \\
20 121& 5 & \cellcolor{blue!5} \textbf{5} & \cellcolor{blue!5} \textbf{5} & \cellcolor{blue!5} \textbf{5} & \cellcolor{blue!5} \textbf{5} & \cellcolor{blue!5} \textbf{5} & \cellcolor{blue!5} \textbf{5} & \cellcolor{blue!5} \textbf{5} \\
20 122& 6 & \cellcolor{blue!5} \textbf{6} & \cellcolor{blue!5} \textbf{6} & \cellcolor{blue!5} \textbf{6} & \cellcolor{blue!5} \textbf{6} & \cellcolor{blue!5} \textbf{6} & \cellcolor{blue!5} \textbf{6} & \cellcolor{blue!5} \textbf{6} \\
20 123& 5 & \cellcolor{blue!5} \textbf{5} & \cellcolor{blue!5} \textbf{5} & \cellcolor{blue!5} \textbf{5} & \cellcolor{blue!5} \textbf{5} & \cellcolor{blue!5} \textbf{5} & \cellcolor{blue!5} \textbf{5} & \cellcolor{blue!5} \textbf{5} \\
20 124& 5 & \cellcolor{blue!5} \textbf{5} & \cellcolor{blue!5} \textbf{5} & \cellcolor{blue!5} \textbf{5} & \cellcolor{blue!5} \textbf{5} & \cellcolor{blue!5} \textbf{5} & \cellcolor{blue!5} \textbf{5} & \cellcolor{blue!5} \textbf{5} \\
20 125& 5 & \cellcolor{blue!5} \textbf{5} & \cellcolor{blue!5} \textbf{5} & \cellcolor{blue!5} \textbf{5} & \cellcolor{blue!5} \textbf{5} & \cellcolor{blue!5} \textbf{5} & \cellcolor{blue!5} \textbf{5} & \cellcolor{blue!5} \textbf{5} \\
20 126& 5 & \cellcolor{blue!5} \textbf{5} & \cellcolor{blue!5} \textbf{5} & \cellcolor{blue!5} \textbf{5} & \cellcolor{blue!5} \textbf{5} & \cellcolor{blue!5} \textbf{5} & \cellcolor{blue!5} \textbf{5} & \cellcolor{blue!5} \textbf{5} \\
20 127& 4 & \cellcolor{blue!5} \textbf{4} & \cellcolor{blue!5} \textbf{4} & \cellcolor{blue!5} \textbf{4} & \cellcolor{blue!5} \textbf{4} & \cellcolor{blue!5} \textbf{4} & \cellcolor{blue!5} \textbf{4} & \cellcolor{blue!5} \textbf{4} \\
20 128& 5 & \cellcolor{blue!5} \textbf{5} & \cellcolor{blue!5} \textbf{5} & \cellcolor{blue!5} \textbf{5} & \cellcolor{blue!5} \textbf{5} & \cellcolor{blue!5} \textbf{5} & \cellcolor{blue!5} \textbf{5} & \cellcolor{blue!5} \textbf{5} \\
20 129& 5 & \cellcolor{blue!5} \textbf{5} & \cellcolor{blue!5} \textbf{5} & \cellcolor{blue!5} \textbf{5} & \cellcolor{blue!5} \textbf{5} & \cellcolor{blue!5} \textbf{5} & \cellcolor{blue!5} \textbf{5} & \cellcolor{blue!5} \textbf{5} \\
20 13& 3 & \cellcolor{blue!5} \textbf{3} & \cellcolor{blue!5} \textbf{3} & \cellcolor{blue!5} \textbf{3} & \cellcolor{blue!5} \textbf{3} & \cellcolor{blue!5} \textbf{3} & \cellcolor{blue!5} \textbf{3} & \cellcolor{blue!5} \textbf{3} \\
20 130& 6 & \cellcolor{blue!5} \textbf{6} & \cellcolor{blue!5} \textbf{6} & \cellcolor{blue!5} \textbf{6} & \cellcolor{blue!5} \textbf{6} & \cellcolor{blue!5} \textbf{6} & \cellcolor{blue!5} \textbf{6} & \cellcolor{blue!5} \textbf{6} \\
20 131& 7 & \cellcolor{blue!5} \textbf{7} & \cellcolor{blue!5} \textbf{7} & \cellcolor{blue!5} \textbf{7} & \cellcolor{blue!5} \textbf{7} & \cellcolor{blue!5} \textbf{7} & \cellcolor{blue!5} \textbf{7} & \cellcolor{blue!5} \textbf{7} \\
20 132& 4 & \cellcolor{blue!5} \textbf{4} & \cellcolor{blue!5} \textbf{4} & \cellcolor{blue!5} \textbf{4} & \cellcolor{blue!5} \textbf{4} & \cellcolor{blue!5} \textbf{4} & \cellcolor{blue!5} \textbf{4} & \cellcolor{blue!5} \textbf{4} \\
20 133& 5 & \cellcolor{blue!5} \textbf{5} & \cellcolor{blue!5} \textbf{5} & \cellcolor{blue!5} \textbf{5} & \cellcolor{blue!5} \textbf{5} & \cellcolor{blue!5} \textbf{5} & \cellcolor{blue!5} \textbf{5} & \cellcolor{blue!5} \textbf{5} \\
20 134& 6 & \cellcolor{blue!5} \textbf{6} & \cellcolor{blue!5} \textbf{6} & \cellcolor{blue!5} \textbf{6} & \cellcolor{blue!5} \textbf{6} & \cellcolor{blue!5} \textbf{6} & \cellcolor{blue!5} \textbf{6} & \cellcolor{blue!5} \textbf{6} \\
20 135& 6 & \cellcolor{blue!5} \textbf{6} & \cellcolor{blue!5} \textbf{6} & \cellcolor{blue!5} \textbf{6} & \cellcolor{blue!5} \textbf{6} & \cellcolor{blue!5} \textbf{6} & \cellcolor{blue!5} \textbf{6} & \cellcolor{blue!5} \textbf{6} \\
20 136& 6 & \cellcolor{blue!5} \textbf{6} & \cellcolor{blue!5} \textbf{6} & \cellcolor{blue!5} \textbf{6} & \cellcolor{blue!5} \textbf{6} & \cellcolor{blue!5} \textbf{6} & \cellcolor{blue!5} \textbf{6} & \cellcolor{blue!5} \textbf{6} \\
20 137& 5 & \cellcolor{blue!5} \textbf{5} & \cellcolor{blue!5} \textbf{5} & \cellcolor{blue!5} \textbf{5} & \cellcolor{blue!5} \textbf{5} & \cellcolor{blue!5} \textbf{5} & \cellcolor{blue!5} \textbf{5} & \cellcolor{blue!5} \textbf{5} \\
20 138& 5 & \cellcolor{blue!5} \textbf{5} & \cellcolor{blue!5} \textbf{5} & \cellcolor{blue!5} \textbf{5} & \cellcolor{blue!5} \textbf{5} & \cellcolor{blue!5} \textbf{5} & \cellcolor{blue!5} \textbf{5} & \cellcolor{blue!5} \textbf{5} \\
20 139& 5 & \cellcolor{blue!5} \textbf{5} & \cellcolor{blue!5} \textbf{5} & \cellcolor{blue!5} \textbf{5} & \cellcolor{blue!5} \textbf{5} & \cellcolor{blue!5} \textbf{5} & \cellcolor{blue!5} \textbf{5} & \cellcolor{blue!5} \textbf{5} \\
20 14& 3 & \cellcolor{blue!5} \textbf{3} & \cellcolor{blue!5} \textbf{3} & \cellcolor{blue!5} \textbf{3} & \cellcolor{blue!5} \textbf{3} & \cellcolor{blue!5} \textbf{3} & \cellcolor{blue!5} \textbf{3} & \cellcolor{blue!5} \textbf{3} \\
20 140& 5 & \cellcolor{blue!5} \textbf{5} & \cellcolor{blue!5} \textbf{5} & \cellcolor{blue!5} \textbf{5} & \cellcolor{blue!5} \textbf{5} & \cellcolor{blue!5} \textbf{5} & \cellcolor{blue!5} \textbf{5} & \cellcolor{blue!5} \textbf{5} \\
20 141& 3 & \cellcolor{blue!5} \textbf{3} & \cellcolor{blue!5} \textbf{3} & \cellcolor{blue!5} \textbf{3} & \cellcolor{blue!5} \textbf{3} & \cellcolor{blue!5} \textbf{3} & \cellcolor{blue!5} \textbf{3} & \cellcolor{blue!5} \textbf{3} \\
20 142& 3 & \cellcolor{blue!5} \textbf{3} & \cellcolor{blue!5} \textbf{3} & \cellcolor{blue!5} \textbf{3} & \cellcolor{blue!5} \textbf{3} & \cellcolor{blue!5} \textbf{3} & \cellcolor{blue!5} \textbf{3} & \cellcolor{blue!5} \textbf{3} \\
20 143& 3 & \cellcolor{blue!5} \textbf{3} & \cellcolor{blue!5} \textbf{3} & \cellcolor{blue!5} \textbf{3} & \cellcolor{blue!5} \textbf{3} & \cellcolor{blue!5} \textbf{3} & \cellcolor{blue!5} \textbf{3} & \cellcolor{blue!5} \textbf{3} \\
20 144& 4 & \cellcolor{blue!5} \textbf{4} & \cellcolor{blue!5} \textbf{4} & \cellcolor{blue!5} \textbf{4} & \cellcolor{blue!5} \textbf{4} & \cellcolor{blue!5} \textbf{4} & \cellcolor{blue!5} \textbf{4} & \cellcolor{blue!5} \textbf{4} \\
20 145& 3 & \cellcolor{blue!5} \textbf{3} & \cellcolor{blue!5} \textbf{3} & \cellcolor{blue!5} \textbf{3} & \cellcolor{blue!5} \textbf{3} & \cellcolor{blue!5} \textbf{3} & \cellcolor{blue!5} \textbf{3} & \cellcolor{blue!5} \textbf{3} \\
20 146& 3 & \cellcolor{blue!5} \textbf{3} & \cellcolor{blue!5} \textbf{3} & \cellcolor{blue!5} \textbf{3} & \cellcolor{blue!5} \textbf{3} & \cellcolor{blue!5} \textbf{3} & \cellcolor{blue!5} \textbf{3} & \cellcolor{blue!5} \textbf{3} \\
20 147& 3 & \cellcolor{blue!5} \textbf{3} & \cellcolor{blue!5} \textbf{3} & \cellcolor{blue!5} \textbf{3} & \cellcolor{blue!5} \textbf{3} & \cellcolor{blue!5} \textbf{3} & \cellcolor{blue!5} \textbf{3} & \cellcolor{blue!5} \textbf{3} \\
20 148& 3 & \cellcolor{blue!5} \textbf{3} & \cellcolor{blue!5} \textbf{3} & \cellcolor{blue!5} \textbf{3} & \cellcolor{blue!5} \textbf{3} & \cellcolor{blue!5} \textbf{3} & \cellcolor{blue!5} \textbf{3} & \cellcolor{blue!5} \textbf{3} \\
20 149& 3 & \cellcolor{blue!5} \textbf{3} & \cellcolor{blue!5} \textbf{3} & \cellcolor{blue!5} \textbf{3} & \cellcolor{blue!5} \textbf{3} & \cellcolor{blue!5} \textbf{3} & \cellcolor{blue!5} \textbf{3} & \cellcolor{blue!5} \textbf{3} \\
20 15& 3 & \cellcolor{blue!5} \textbf{3} & \cellcolor{blue!5} \textbf{3} & \cellcolor{blue!5} \textbf{3} & \cellcolor{blue!5} \textbf{3} & \cellcolor{blue!5} \textbf{3} & \cellcolor{blue!5} \textbf{3} & \cellcolor{blue!5} \textbf{3} \\
20 150& 3 & \cellcolor{blue!5} \textbf{3} & \cellcolor{blue!5} \textbf{3} & \cellcolor{blue!5} \textbf{3} & \cellcolor{blue!5} \textbf{3} & \cellcolor{blue!5} \textbf{3} & \cellcolor{blue!5} \textbf{3} & \cellcolor{blue!5} \textbf{3} \\
20 151& 3 & \cellcolor{blue!5} \textbf{3} & \cellcolor{blue!5} \textbf{3} & \cellcolor{blue!5} \textbf{3} & \cellcolor{blue!5} \textbf{3} & \cellcolor{blue!5} \textbf{3} & \cellcolor{blue!5} \textbf{3} & \cellcolor{blue!5} \textbf{3} \\
20 152& 3 & \cellcolor{blue!5} \textbf{3} & \cellcolor{blue!5} \textbf{3} & \cellcolor{blue!5} \textbf{3} & \cellcolor{blue!5} \textbf{3} & \cellcolor{blue!5} \textbf{3} & \cellcolor{blue!5} \textbf{3} & \cellcolor{blue!5} \textbf{3} \\
20 153& 3 & \cellcolor{blue!5} \textbf{3} & \cellcolor{blue!5} \textbf{3} & \cellcolor{blue!5} \textbf{3} & \cellcolor{blue!5} \textbf{3} & \cellcolor{blue!5} \textbf{3} & \cellcolor{blue!5} \textbf{3} & \cellcolor{blue!5} \textbf{3} \\
20 154& 3 & \cellcolor{blue!5} \textbf{3} & \cellcolor{blue!5} \textbf{3} & \cellcolor{blue!5} \textbf{3} & \cellcolor{blue!5} \textbf{3} & \cellcolor{blue!5} \textbf{3} & \cellcolor{blue!5} \textbf{3} & \cellcolor{blue!5} \textbf{3} \\
20 155& 3 & \cellcolor{blue!5} \textbf{3} & \cellcolor{blue!5} \textbf{3} & \cellcolor{blue!5} \textbf{3} & \cellcolor{blue!5} \textbf{3} & \cellcolor{blue!5} \textbf{3} & \cellcolor{blue!5} \textbf{3} & \cellcolor{blue!5} \textbf{3} \\
20 156& 3 & \cellcolor{blue!5} \textbf{3} & \cellcolor{blue!5} \textbf{3} & \cellcolor{blue!5} \textbf{3} & \cellcolor{blue!5} \textbf{3} & \cellcolor{blue!5} \textbf{3} & \cellcolor{blue!5} \textbf{3} & \cellcolor{blue!5} \textbf{3} \\
20 157& 3 & \cellcolor{blue!5} \textbf{3} & \cellcolor{blue!5} \textbf{3} & \cellcolor{blue!5} \textbf{3} & \cellcolor{blue!5} \textbf{3} & \cellcolor{blue!5} \textbf{3} & \cellcolor{blue!5} \textbf{3} & \cellcolor{blue!5} \textbf{3} \\
20 158& 3 & \cellcolor{blue!5} \textbf{3} & \cellcolor{blue!5} \textbf{3} & \cellcolor{blue!5} \textbf{3} & \cellcolor{blue!5} \textbf{3} & \cellcolor{blue!5} \textbf{3} & \cellcolor{blue!5} \textbf{3} & \cellcolor{blue!5} \textbf{3} \\
20 159& 3 & \cellcolor{blue!5} \textbf{3} & \cellcolor{blue!5} \textbf{3} & \cellcolor{blue!5} \textbf{3} & \cellcolor{blue!5} \textbf{3} & \cellcolor{blue!5} \textbf{3} & \cellcolor{blue!5} \textbf{3} & \cellcolor{blue!5} \textbf{3} \\
20 16& 12 & \cellcolor{blue!5} \textbf{12} & \cellcolor{blue!5} \textbf{12} & \cellcolor{blue!5} \textbf{12} & \cellcolor{blue!5} \textbf{12} & \cellcolor{blue!5} \textbf{12} & \cellcolor{blue!5} \textbf{12} & \cellcolor{blue!5} 12 \\
20 160& 3 & \cellcolor{blue!5} \textbf{3} & \cellcolor{blue!5} \textbf{3} & \cellcolor{blue!5} \textbf{3} & \cellcolor{blue!5} \textbf{3} & \cellcolor{blue!5} \textbf{3} & \cellcolor{blue!5} \textbf{3} & \cellcolor{blue!5} \textbf{3} \\
20 161& 3 & \cellcolor{blue!5} \textbf{3} & \cellcolor{blue!5} \textbf{3} & \cellcolor{blue!5} \textbf{3} & \cellcolor{blue!5} \textbf{3} & \cellcolor{blue!5} \textbf{3} & \cellcolor{blue!5} \textbf{3} & \cellcolor{blue!5} \textbf{3} \\
20 162& 3 & \cellcolor{blue!5} \textbf{3} & \cellcolor{blue!5} \textbf{3} & \cellcolor{blue!5} \textbf{3} & \cellcolor{blue!5} \textbf{3} & \cellcolor{blue!5} \textbf{3} & \cellcolor{blue!5} \textbf{3} & \cellcolor{blue!5} \textbf{3} \\
20 163& 3 & \cellcolor{blue!5} \textbf{3} & \cellcolor{blue!5} \textbf{3} & \cellcolor{blue!5} \textbf{3} & \cellcolor{blue!5} \textbf{3} & \cellcolor{blue!5} \textbf{3} & \cellcolor{blue!5} \textbf{3} & \cellcolor{blue!5} \textbf{3} \\
20 164& 4 & \cellcolor{blue!5} \textbf{4} & \cellcolor{blue!5} \textbf{4} & \cellcolor{blue!5} \textbf{4} & \cellcolor{blue!5} \textbf{4} & \cellcolor{blue!5} \textbf{4} & \cellcolor{blue!5} \textbf{4} & \cellcolor{blue!5} \textbf{4} \\
20 165& 3 & \cellcolor{blue!5} \textbf{3} & \cellcolor{blue!5} \textbf{3} & \cellcolor{blue!5} \textbf{3} & \cellcolor{blue!5} \textbf{3} & \cellcolor{blue!5} \textbf{3} & \cellcolor{blue!5} \textbf{3} & \cellcolor{blue!5} \textbf{3} \\
20 166& 12 & \cellcolor{blue!5} \textbf{12} & \cellcolor{blue!5} \textbf{12} & \cellcolor{blue!5} \textbf{12} & \cellcolor{blue!5} \textbf{12} & \cellcolor{blue!5} \textbf{12} & \cellcolor{blue!5} \textbf{12} & \cellcolor{blue!5} 12 \\
20 167& 11 & \cellcolor{blue!5} \textbf{11} & \cellcolor{blue!5} \textbf{11} & \cellcolor{blue!5} \textbf{11} & \cellcolor{blue!5} \textbf{11} & \cellcolor{blue!5} \textbf{11} & \cellcolor{blue!5} \textbf{11} & \cellcolor{blue!5} 11 \\
20 168& 10 & \cellcolor{blue!5} \textbf{10} & \cellcolor{blue!5} \textbf{10} & \cellcolor{blue!5} \textbf{10} & \cellcolor{blue!5} \textbf{10} & \cellcolor{blue!5} \textbf{10} & \cellcolor{blue!5} \textbf{10} & \cellcolor{blue!5} 10 \\
20 169& 11 & \cellcolor{blue!5} \textbf{11} & \cellcolor{blue!5} \textbf{11} & \cellcolor{blue!5} \textbf{11} & \cellcolor{blue!5} \textbf{11} & \cellcolor{blue!5} \textbf{11} & \cellcolor{blue!5} \textbf{11} & \cellcolor{blue!5} 11 \\
20 17& 10 & \cellcolor{blue!5} \textbf{10} & \cellcolor{blue!5} \textbf{10} & \cellcolor{blue!5} \textbf{10} & \cellcolor{blue!5} \textbf{10} & \cellcolor{blue!5} \textbf{10} & \cellcolor{blue!5} \textbf{10} & \cellcolor{blue!5} 10 \\
20 170& 11 & \cellcolor{blue!5} \textbf{11} & \cellcolor{blue!5} \textbf{11} & \cellcolor{blue!5} \textbf{11} & \cellcolor{blue!5} \textbf{11} & \cellcolor{blue!5} \textbf{11} & \cellcolor{blue!5} \textbf{11} & \cellcolor{blue!5} 11 \\
20 171& 13 & \cellcolor{blue!5} \textbf{13} & \cellcolor{blue!5} \textbf{13} & \cellcolor{blue!5} \textbf{13} & \cellcolor{blue!5} \textbf{13} & \cellcolor{blue!5} \textbf{13} & \cellcolor{blue!5} 13 & \cellcolor{blue!5} 13 \\
20 172& 11 & \cellcolor{blue!5} \textbf{11} & \cellcolor{blue!5} \textbf{11} & \cellcolor{blue!5} \textbf{11} & \cellcolor{blue!5} \textbf{11} & \cellcolor{blue!5} \textbf{11} & \cellcolor{blue!5} \textbf{11} & \cellcolor{blue!5} 11 \\
20 173& 11 & \cellcolor{blue!5} \textbf{11} & \cellcolor{blue!5} \textbf{11} & \cellcolor{blue!5} \textbf{11} & \cellcolor{blue!5} \textbf{11} & \cellcolor{blue!5} \textbf{11} & \cellcolor{blue!5} \textbf{11} & \cellcolor{blue!5} 11 \\
20 174& 12 & \cellcolor{blue!5} \textbf{12} & \cellcolor{blue!5} \textbf{12} & \cellcolor{blue!5} \textbf{12} & \cellcolor{blue!5} \textbf{12} & \cellcolor{blue!5} \textbf{12} & \cellcolor{blue!5} \textbf{12} & \cellcolor{blue!5} 12 \\
20 175& 10 & \cellcolor{blue!5} \textbf{10} & \cellcolor{blue!5} \textbf{10} & \cellcolor{blue!5} \textbf{10} & \cellcolor{blue!5} \textbf{10} & \cellcolor{blue!5} \textbf{10} & \cellcolor{blue!5} \textbf{10} & \cellcolor{blue!5} \textbf{10} \\
20 176& 11 & \cellcolor{blue!5} \textbf{11} & \cellcolor{blue!5} \textbf{11} & \cellcolor{blue!5} \textbf{11} & \cellcolor{blue!5} \textbf{11} & \cellcolor{blue!5} \textbf{11} & \cellcolor{blue!5} \textbf{11} & \cellcolor{blue!5} 11 \\
20 177& 10 & \cellcolor{blue!5} \textbf{10} & \cellcolor{blue!5} \textbf{10} & \cellcolor{blue!5} \textbf{10} & \cellcolor{blue!5} \textbf{10} & \cellcolor{blue!5} \textbf{10} & \cellcolor{blue!5} \textbf{10} & \cellcolor{blue!5} 10 \\
20 178& 11 & \cellcolor{blue!5} \textbf{11} & \cellcolor{blue!5} \textbf{11} & \cellcolor{blue!5} \textbf{11} & \cellcolor{blue!5} \textbf{11} & \cellcolor{blue!5} \textbf{11} & \cellcolor{blue!5} \textbf{11} & \cellcolor{blue!5} 11 \\
20 179& 11 & \cellcolor{blue!5} \textbf{11} & \cellcolor{blue!5} \textbf{11} & \cellcolor{blue!5} \textbf{11} & \cellcolor{blue!5} \textbf{11} & \cellcolor{blue!5} \textbf{11} & \cellcolor{blue!5} \textbf{11} & \cellcolor{blue!5} 11 \\
20 18& 11 & \cellcolor{blue!5} \textbf{11} & \cellcolor{blue!5} \textbf{11} & \cellcolor{blue!5} \textbf{11} & \cellcolor{blue!5} \textbf{11} & \cellcolor{blue!5} \textbf{11} & \cellcolor{blue!5} \textbf{11} & \cellcolor{blue!5} 11 \\
20 180& 13 & \cellcolor{blue!5} \textbf{13} & \cellcolor{blue!5} \textbf{13} & \cellcolor{blue!5} \textbf{13} & \cellcolor{blue!5} \textbf{13} & \cellcolor{blue!5} \textbf{13} & \cellcolor{blue!5} \textbf{13} & \cellcolor{blue!5} 13 \\
20 181& 11 & \cellcolor{blue!5} \textbf{11} & \cellcolor{blue!5} \textbf{11} & \cellcolor{blue!5} \textbf{11} & \cellcolor{blue!5} \textbf{11} & \cellcolor{blue!5} \textbf{11} & \cellcolor{blue!5} \textbf{11} & \cellcolor{blue!5} \textbf{11} \\
20 182& 11 & \cellcolor{blue!5} \textbf{11} & \cellcolor{blue!5} \textbf{11} & \cellcolor{blue!5} \textbf{11} & \cellcolor{blue!5} \textbf{11} & \cellcolor{blue!5} \textbf{11} & \cellcolor{blue!5} \textbf{11} & \cellcolor{blue!5} 11 \\
20 183& 13 & \cellcolor{blue!5} \textbf{13} & \cellcolor{blue!5} \textbf{13} & \cellcolor{blue!5} \textbf{13} & \cellcolor{blue!5} \textbf{13} & \cellcolor{blue!5} \textbf{13} & \cellcolor{blue!5} \textbf{13} & \cellcolor{blue!5} 13 \\
20 184& 12 & \cellcolor{blue!5} \textbf{12} & \cellcolor{blue!5} \textbf{12} & \cellcolor{blue!5} \textbf{12} & \cellcolor{blue!5} \textbf{12} & \cellcolor{blue!5} \textbf{12} & \cellcolor{blue!5} \textbf{12} & \cellcolor{blue!5} 12 \\
20 185& 15 & \cellcolor{blue!5} \textbf{15} & \cellcolor{blue!5} \textbf{15} & \cellcolor{blue!5} \textbf{15} & \cellcolor{blue!5} \textbf{15} & \cellcolor{blue!5} \textbf{15} & \cellcolor{blue!5} 15 & \cellcolor{blue!5} 15 \\
20 186& 14 & \cellcolor{blue!5} \textbf{14} & \cellcolor{blue!5} \textbf{14} & \cellcolor{blue!5} \textbf{14} & \cellcolor{blue!5} \textbf{14} & \cellcolor{blue!5} \textbf{14} & \cellcolor{blue!5} \textbf{14} & \cellcolor{blue!5} 14 \\
20 187& 10 & \cellcolor{blue!5} \textbf{10} & \cellcolor{blue!5} \textbf{10} & \cellcolor{blue!5} \textbf{10} & \cellcolor{blue!5} \textbf{10} & \cellcolor{blue!5} \textbf{10} & \cellcolor{blue!5} \textbf{10} & \cellcolor{blue!5} \textbf{10} \\
20 188& 11 & \cellcolor{blue!5} \textbf{11} & \cellcolor{blue!5} \textbf{11} & \cellcolor{blue!5} \textbf{11} & \cellcolor{blue!5} \textbf{11} & \cellcolor{blue!5} \textbf{11} & \cellcolor{blue!5} \textbf{11} & \cellcolor{blue!5} 11 \\
20 189& 13 & \cellcolor{blue!5} \textbf{13} & \cellcolor{blue!5} \textbf{13} & \cellcolor{blue!5} \textbf{13} & \cellcolor{blue!5} \textbf{13} & \cellcolor{blue!5} \textbf{13} & \cellcolor{blue!5} \textbf{13} & \cellcolor{blue!5} 13 \\
20 19& 14 & \cellcolor{blue!5} \textbf{14} & \cellcolor{blue!5} \textbf{14} & \cellcolor{blue!5} \textbf{14} & \cellcolor{blue!5} \textbf{14} & \cellcolor{blue!5} \textbf{14} & \cellcolor{blue!5} 14 & \cellcolor{blue!5} 14 \\
20 190& 15 & \cellcolor{blue!5} \textbf{15} & \cellcolor{blue!5} \textbf{15} & \cellcolor{blue!5} \textbf{15} & \cellcolor{blue!5} \textbf{15} & \cellcolor{blue!5} \textbf{15} & \cellcolor{blue!5} 15 & \cellcolor{blue!5} 15 \\
20 191& 4 & \cellcolor{blue!5} \textbf{4} & \cellcolor{blue!5} \textbf{4} & \cellcolor{blue!5} \textbf{4} & \cellcolor{blue!5} \textbf{4} & \cellcolor{blue!5} \textbf{4} & \cellcolor{blue!5} \textbf{4} & \cellcolor{blue!5} \textbf{4} \\
20 192& 5 & \cellcolor{blue!5} \textbf{5} & \cellcolor{blue!5} \textbf{5} & \cellcolor{blue!5} \textbf{5} & \cellcolor{blue!5} \textbf{5} & \cellcolor{blue!5} \textbf{5} & \cellcolor{blue!5} \textbf{5} & \cellcolor{blue!5} \textbf{5} \\
20 193& 5 & \cellcolor{blue!5} \textbf{5} & \cellcolor{blue!5} \textbf{5} & \cellcolor{blue!5} \textbf{5} & \cellcolor{blue!5} \textbf{5} & \cellcolor{blue!5} \textbf{5} & \cellcolor{blue!5} \textbf{5} & \cellcolor{blue!5} \textbf{5} \\
20 194& 6 & \cellcolor{blue!5} \textbf{6} & \cellcolor{blue!5} \textbf{6} & \cellcolor{blue!5} \textbf{6} & \cellcolor{blue!5} \textbf{6} & \cellcolor{blue!5} \textbf{6} & \cellcolor{blue!5} \textbf{6} & \cellcolor{blue!5} \textbf{6} \\
20 195& 6 & \cellcolor{blue!5} \textbf{6} & \cellcolor{blue!5} \textbf{6} & \cellcolor{blue!5} \textbf{6} & \cellcolor{blue!5} \textbf{6} & \cellcolor{blue!5} \textbf{6} & \cellcolor{blue!5} \textbf{6} & \cellcolor{blue!5} \textbf{6} \\
20 196& 5 & \cellcolor{blue!5} \textbf{5} & \cellcolor{blue!5} \textbf{5} & \cellcolor{blue!5} \textbf{5} & \cellcolor{blue!5} \textbf{5} & \cellcolor{blue!5} \textbf{5} & \cellcolor{blue!5} \textbf{5} & \cellcolor{blue!5} \textbf{5} \\
20 197& 4 & \cellcolor{blue!5} \textbf{4} & \cellcolor{blue!5} \textbf{4} & \cellcolor{blue!5} \textbf{4} & \cellcolor{blue!5} \textbf{4} & \cellcolor{blue!5} \textbf{4} & \cellcolor{blue!5} \textbf{4} & \cellcolor{blue!5} \textbf{4} \\
20 198& 6 & \cellcolor{blue!5} \textbf{6} & \cellcolor{blue!5} \textbf{6} & \cellcolor{blue!5} \textbf{6} & \cellcolor{blue!5} \textbf{6} & \cellcolor{blue!5} \textbf{6} & \cellcolor{blue!5} \textbf{6} & \cellcolor{blue!5} \textbf{6} \\
20 199& 5 & \cellcolor{blue!5} \textbf{5} & \cellcolor{blue!5} \textbf{5} & \cellcolor{blue!5} \textbf{5} & \cellcolor{blue!5} \textbf{5} & \cellcolor{blue!5} \textbf{5} & \cellcolor{blue!5} \textbf{5} & \cellcolor{blue!5} \textbf{5} \\
20 2& 3 & \cellcolor{blue!5} \textbf{3} & \cellcolor{blue!5} \textbf{3} & \cellcolor{blue!5} \textbf{3} & \cellcolor{blue!5} \textbf{3} & \cellcolor{blue!5} \textbf{3} & \cellcolor{blue!5} \textbf{3} & \cellcolor{blue!5} \textbf{3} \\
20 20& 11 & \cellcolor{blue!5} \textbf{11} & \cellcolor{blue!5} \textbf{11} & \cellcolor{blue!5} \textbf{11} & \cellcolor{blue!5} \textbf{11} & \cellcolor{blue!5} \textbf{11} & \cellcolor{blue!5} \textbf{11} & \cellcolor{blue!5} 11 \\
20 200& 6 & \cellcolor{blue!5} \textbf{6} & \cellcolor{blue!5} \textbf{6} & \cellcolor{blue!5} \textbf{6} & \cellcolor{blue!5} \textbf{6} & \cellcolor{blue!5} \textbf{6} & \cellcolor{blue!5} \textbf{6} & \cellcolor{blue!5} \textbf{6} \\
20 201& 6 & \cellcolor{blue!5} \textbf{6} & \cellcolor{blue!5} \textbf{6} & \cellcolor{blue!5} \textbf{6} & \cellcolor{blue!5} \textbf{6} & \cellcolor{blue!5} \textbf{6} & \cellcolor{blue!5} \textbf{6} & \cellcolor{blue!5} \textbf{6} \\
20 202& 4 & \cellcolor{blue!5} \textbf{4} & \cellcolor{blue!5} \textbf{4} & \cellcolor{blue!5} \textbf{4} & \cellcolor{blue!5} \textbf{4} & \cellcolor{blue!5} \textbf{4} & \cellcolor{blue!5} \textbf{4} & \cellcolor{blue!5} \textbf{4} \\
20 203& 4 & \cellcolor{blue!5} \textbf{4} & \cellcolor{blue!5} \textbf{4} & \cellcolor{blue!5} \textbf{4} & \cellcolor{blue!5} \textbf{4} & \cellcolor{blue!5} \textbf{4} & \cellcolor{blue!5} \textbf{4} & \cellcolor{blue!5} \textbf{4} \\
20 204& 5 & \cellcolor{blue!5} \textbf{5} & \cellcolor{blue!5} \textbf{5} & \cellcolor{blue!5} \textbf{5} & \cellcolor{blue!5} \textbf{5} & \cellcolor{blue!5} \textbf{5} & \cellcolor{blue!5} \textbf{5} & \cellcolor{blue!5} \textbf{5} \\
20 205& 6 & \cellcolor{blue!5} \textbf{6} & \cellcolor{blue!5} \textbf{6} & \cellcolor{blue!5} \textbf{6} & \cellcolor{blue!5} \textbf{6} & \cellcolor{blue!5} \textbf{6} & \cellcolor{blue!5} \textbf{6} & \cellcolor{blue!5} \textbf{6} \\
20 206& 5 & \cellcolor{blue!5} \textbf{5} & \cellcolor{blue!5} \textbf{5} & \cellcolor{blue!5} \textbf{5} & \cellcolor{blue!5} \textbf{5} & \cellcolor{blue!5} \textbf{5} & \cellcolor{blue!5} \textbf{5} & \cellcolor{blue!5} \textbf{5} \\
20 207& 6 & \cellcolor{blue!5} \textbf{6} & \cellcolor{blue!5} \textbf{6} & \cellcolor{blue!5} \textbf{6} & \cellcolor{blue!5} \textbf{6} & \cellcolor{blue!5} \textbf{6} & \cellcolor{blue!5} \textbf{6} & \cellcolor{blue!5} \textbf{6} \\
20 208& 5 & \cellcolor{blue!5} \textbf{5} & \cellcolor{blue!5} \textbf{5} & \cellcolor{blue!5} \textbf{5} & \cellcolor{blue!5} \textbf{5} & \cellcolor{blue!5} \textbf{5} & \cellcolor{blue!5} \textbf{5} & \cellcolor{blue!5} \textbf{5} \\
20 209& 4 & \cellcolor{blue!5} \textbf{4} & \cellcolor{blue!5} \textbf{4} & \cellcolor{blue!5} \textbf{4} & \cellcolor{blue!5} \textbf{4} & \cellcolor{blue!5} \textbf{4} & \cellcolor{blue!5} \textbf{4} & \cellcolor{blue!5} \textbf{4} \\
20 21& 14 & \cellcolor{blue!5} \textbf{14} & \cellcolor{blue!5} \textbf{14} & \cellcolor{blue!5} \textbf{14} & \cellcolor{blue!5} \textbf{14} & \cellcolor{blue!5} \textbf{14} & \cellcolor{blue!5} \textbf{14} & \cellcolor{blue!5} 14 \\
20 210& 5 & \cellcolor{blue!5} \textbf{5} & \cellcolor{blue!5} \textbf{5} & \cellcolor{blue!5} \textbf{5} & \cellcolor{blue!5} \textbf{5} & \cellcolor{blue!5} \textbf{5} & \cellcolor{blue!5} \textbf{5} & \cellcolor{blue!5} \textbf{5} \\
20 211& 5 & \cellcolor{blue!5} \textbf{5} & \cellcolor{blue!5} \textbf{5} & \cellcolor{blue!5} \textbf{5} & \cellcolor{blue!5} \textbf{5} & \cellcolor{blue!5} \textbf{5} & \cellcolor{blue!5} \textbf{5} & \cellcolor{blue!5} \textbf{5} \\
20 212& 5 & \cellcolor{blue!5} \textbf{5} & \cellcolor{blue!5} \textbf{5} & \cellcolor{blue!5} \textbf{5} & \cellcolor{blue!5} \textbf{5} & \cellcolor{blue!5} \textbf{5} & \cellcolor{blue!5} \textbf{5} & \cellcolor{blue!5} \textbf{5} \\
20 213& 5 & \cellcolor{blue!5} \textbf{5} & \cellcolor{blue!5} \textbf{5} & \cellcolor{blue!5} \textbf{5} & \cellcolor{blue!5} \textbf{5} & \cellcolor{blue!5} \textbf{5} & \cellcolor{blue!5} \textbf{5} & \cellcolor{blue!5} \textbf{5} \\
20 214& 5 & \cellcolor{blue!5} \textbf{5} & \cellcolor{blue!5} \textbf{5} & \cellcolor{blue!5} \textbf{5} & \cellcolor{blue!5} \textbf{5} & \cellcolor{blue!5} \textbf{5} & \cellcolor{blue!5} \textbf{5} & \cellcolor{blue!5} \textbf{5} \\
20 215& 5 & \cellcolor{blue!5} \textbf{5} & \cellcolor{blue!5} \textbf{5} & \cellcolor{blue!5} \textbf{5} & \cellcolor{blue!5} \textbf{5} & \cellcolor{blue!5} \textbf{5} & \cellcolor{blue!5} \textbf{5} & \cellcolor{blue!5} \textbf{5} \\
20 216& 3 & \cellcolor{blue!5} \textbf{3} & \cellcolor{blue!5} \textbf{3} & \cellcolor{blue!5} \textbf{3} & \cellcolor{blue!5} \textbf{3} & \cellcolor{blue!5} \textbf{3} & \cellcolor{blue!5} \textbf{3} & \cellcolor{blue!5} \textbf{3} \\
20 217& 4 & \cellcolor{blue!5} \textbf{4} & \cellcolor{blue!5} \textbf{4} & \cellcolor{blue!5} \textbf{4} & \cellcolor{blue!5} \textbf{4} & \cellcolor{blue!5} \textbf{4} & \cellcolor{blue!5} \textbf{4} & \cellcolor{blue!5} \textbf{4} \\
20 218& 3 & \cellcolor{blue!5} \textbf{3} & \cellcolor{blue!5} \textbf{3} & \cellcolor{blue!5} \textbf{3} & \cellcolor{blue!5} \textbf{3} & \cellcolor{blue!5} \textbf{3} & \cellcolor{blue!5} \textbf{3} & \cellcolor{blue!5} \textbf{3} \\
20 219& 3 & \cellcolor{blue!5} \textbf{3} & \cellcolor{blue!5} \textbf{3} & \cellcolor{blue!5} \textbf{3} & \cellcolor{blue!5} \textbf{3} & \cellcolor{blue!5} \textbf{3} & \cellcolor{blue!5} \textbf{3} & \cellcolor{blue!5} \textbf{3} \\
20 22& 12 & \cellcolor{blue!5} \textbf{12} & \cellcolor{blue!5} \textbf{12} & \cellcolor{blue!5} \textbf{12} & \cellcolor{blue!5} \textbf{12} & \cellcolor{blue!5} \textbf{12} & \cellcolor{blue!5} \textbf{12} & \cellcolor{blue!5} 12 \\
20 220& 3 & \cellcolor{blue!5} \textbf{3} & \cellcolor{blue!5} \textbf{3} & \cellcolor{blue!5} \textbf{3} & \cellcolor{blue!5} \textbf{3} & \cellcolor{blue!5} \textbf{3} & \cellcolor{blue!5} \textbf{3} & \cellcolor{blue!5} \textbf{3} \\
20 221& 3 & \cellcolor{blue!5} \textbf{3} & \cellcolor{blue!5} \textbf{3} & \cellcolor{blue!5} \textbf{3} & \cellcolor{blue!5} \textbf{3} & \cellcolor{blue!5} \textbf{3} & \cellcolor{blue!5} \textbf{3} & \cellcolor{blue!5} \textbf{3} \\
20 222& 3 & \cellcolor{blue!5} \textbf{3} & \cellcolor{blue!5} \textbf{3} & \cellcolor{blue!5} \textbf{3} & \cellcolor{blue!5} \textbf{3} & \cellcolor{blue!5} \textbf{3} & \cellcolor{blue!5} \textbf{3} & \cellcolor{blue!5} \textbf{3} \\
20 223& 3 & \cellcolor{blue!5} \textbf{3} & \cellcolor{blue!5} \textbf{3} & \cellcolor{blue!5} \textbf{3} & \cellcolor{blue!5} \textbf{3} & \cellcolor{blue!5} \textbf{3} & \cellcolor{blue!5} \textbf{3} & \cellcolor{blue!5} \textbf{3} \\
20 224& 3 & \cellcolor{blue!5} \textbf{3} & \cellcolor{blue!5} \textbf{3} & \cellcolor{blue!5} \textbf{3} & \cellcolor{blue!5} \textbf{3} & \cellcolor{blue!5} \textbf{3} & \cellcolor{blue!5} \textbf{3} & \cellcolor{blue!5} \textbf{3} \\
20 225& 3 & \cellcolor{blue!5} \textbf{3} & \cellcolor{blue!5} \textbf{3} & \cellcolor{blue!5} \textbf{3} & \cellcolor{blue!5} \textbf{3} & \cellcolor{blue!5} \textbf{3} & \cellcolor{blue!5} \textbf{3} & \cellcolor{blue!5} \textbf{3} \\
20 226& 3 & \cellcolor{blue!5} \textbf{3} & \cellcolor{blue!5} \textbf{3} & \cellcolor{blue!5} \textbf{3} & \cellcolor{blue!5} \textbf{3} & \cellcolor{blue!5} \textbf{3} & \cellcolor{blue!5} \textbf{3} & \cellcolor{blue!5} \textbf{3} \\
20 227& 3 & \cellcolor{blue!5} \textbf{3} & \cellcolor{blue!5} \textbf{3} & \cellcolor{blue!5} \textbf{3} & \cellcolor{blue!5} \textbf{3} & \cellcolor{blue!5} \textbf{3} & \cellcolor{blue!5} \textbf{3} & \cellcolor{blue!5} \textbf{3} \\
20 228& 2 & \cellcolor{blue!5} \textbf{2} & \cellcolor{blue!5} \textbf{2} & \cellcolor{blue!5} \textbf{2} & \cellcolor{blue!5} \textbf{2} & \cellcolor{blue!5} \textbf{2} & \cellcolor{blue!5} \textbf{2} & \cellcolor{blue!5} \textbf{2} \\
20 229& 3 & \cellcolor{blue!5} \textbf{3} & \cellcolor{blue!5} \textbf{3} & \cellcolor{blue!5} \textbf{3} & \cellcolor{blue!5} \textbf{3} & \cellcolor{blue!5} \textbf{3} & \cellcolor{blue!5} \textbf{3} & \cellcolor{blue!5} \textbf{3} \\
20 23& 13 & \cellcolor{blue!5} \textbf{13} & \cellcolor{blue!5} \textbf{13} & \cellcolor{blue!5} \textbf{13} & \cellcolor{blue!5} \textbf{13} & \cellcolor{blue!5} \textbf{13} & \cellcolor{blue!5} \textbf{13} & \cellcolor{blue!5} 13 \\
20 230& 3 & \cellcolor{blue!5} \textbf{3} & \cellcolor{blue!5} \textbf{3} & \cellcolor{blue!5} \textbf{3} & \cellcolor{blue!5} \textbf{3} & \cellcolor{blue!5} \textbf{3} & \cellcolor{blue!5} \textbf{3} & \cellcolor{blue!5} \textbf{3} \\
20 231& 3 & \cellcolor{blue!5} \textbf{3} & \cellcolor{blue!5} \textbf{3} & \cellcolor{blue!5} \textbf{3} & \cellcolor{blue!5} \textbf{3} & \cellcolor{blue!5} \textbf{3} & \cellcolor{blue!5} \textbf{3} & \cellcolor{blue!5} \textbf{3} \\
20 232& 3 & \cellcolor{blue!5} \textbf{3} & \cellcolor{blue!5} \textbf{3} & \cellcolor{blue!5} \textbf{3} & \cellcolor{blue!5} \textbf{3} & \cellcolor{blue!5} \textbf{3} & \cellcolor{blue!5} \textbf{3} & \cellcolor{blue!5} \textbf{3} \\
20 233& 3 & \cellcolor{blue!5} \textbf{3} & \cellcolor{blue!5} \textbf{3} & \cellcolor{blue!5} \textbf{3} & \cellcolor{blue!5} \textbf{3} & \cellcolor{blue!5} \textbf{3} & \cellcolor{blue!5} \textbf{3} & \cellcolor{blue!5} \textbf{3} \\
20 234& 3 & \cellcolor{blue!5} \textbf{3} & \cellcolor{blue!5} \textbf{3} & \cellcolor{blue!5} \textbf{3} & \cellcolor{blue!5} \textbf{3} & \cellcolor{blue!5} \textbf{3} & \cellcolor{blue!5} \textbf{3} & \cellcolor{blue!5} \textbf{3} \\
20 235& 3 & \cellcolor{blue!5} \textbf{3} & \cellcolor{blue!5} \textbf{3} & \cellcolor{blue!5} \textbf{3} & \cellcolor{blue!5} \textbf{3} & \cellcolor{blue!5} \textbf{3} & \cellcolor{blue!5} \textbf{3} & \cellcolor{blue!5} \textbf{3} \\
20 236& 3 & \cellcolor{blue!5} \textbf{3} & \cellcolor{blue!5} \textbf{3} & \cellcolor{blue!5} \textbf{3} & \cellcolor{blue!5} \textbf{3} & \cellcolor{blue!5} \textbf{3} & \cellcolor{blue!5} \textbf{3} & \cellcolor{blue!5} \textbf{3} \\
20 237& 3 & \cellcolor{blue!5} \textbf{3} & \cellcolor{blue!5} \textbf{3} & \cellcolor{blue!5} \textbf{3} & \cellcolor{blue!5} \textbf{3} & \cellcolor{blue!5} \textbf{3} & \cellcolor{blue!5} \textbf{3} & \cellcolor{blue!5} \textbf{3} \\
20 238& 3 & \cellcolor{blue!5} \textbf{3} & \cellcolor{blue!5} \textbf{3} & \cellcolor{blue!5} \textbf{3} & \cellcolor{blue!5} \textbf{3} & \cellcolor{blue!5} \textbf{3} & \cellcolor{blue!5} \textbf{3} & \cellcolor{blue!5} \textbf{3} \\
20 239& 3 & \cellcolor{blue!5} \textbf{3} & \cellcolor{blue!5} \textbf{3} & \cellcolor{blue!5} \textbf{3} & \cellcolor{blue!5} \textbf{3} & \cellcolor{blue!5} \textbf{3} & \cellcolor{blue!5} \textbf{3} & \cellcolor{blue!5} \textbf{3} \\
20 24& 11 & \cellcolor{blue!5} \textbf{11} & \cellcolor{blue!5} \textbf{11} & \cellcolor{blue!5} \textbf{11} & \cellcolor{blue!5} \textbf{11} & \cellcolor{blue!5} \textbf{11} & \cellcolor{blue!5} \textbf{11} & \cellcolor{blue!5} 11 \\
20 240& 3 & \cellcolor{blue!5} \textbf{3} & \cellcolor{blue!5} \textbf{3} & \cellcolor{blue!5} \textbf{3} & \cellcolor{blue!5} \textbf{3} & \cellcolor{blue!5} \textbf{3} & \cellcolor{blue!5} \textbf{3} & \cellcolor{blue!5} \textbf{3} \\
20 241& 13 & \cellcolor{blue!5} \textbf{13} & \cellcolor{blue!5} \textbf{13} & \cellcolor{blue!5} \textbf{13} & \cellcolor{blue!5} \textbf{13} & \cellcolor{blue!5} \textbf{13} & \cellcolor{blue!5} \textbf{13} & \cellcolor{blue!5} \textbf{13} \\
20 242& 12 & \cellcolor{blue!5} \textbf{12} & \cellcolor{blue!5} \textbf{12} & \cellcolor{blue!5} \textbf{12} & \cellcolor{blue!5} \textbf{12} & \cellcolor{blue!5} \textbf{12} & \cellcolor{blue!5} \textbf{12} & \cellcolor{blue!5} \textbf{12} \\
20 243& 10 & \cellcolor{blue!5} \textbf{10} & \cellcolor{blue!5} \textbf{10} & \cellcolor{blue!5} \textbf{10} & \cellcolor{blue!5} \textbf{10} & \cellcolor{blue!5} \textbf{10} & \cellcolor{blue!5} \textbf{10} & \cellcolor{blue!5} \textbf{10} \\
20 244& 11 & \cellcolor{blue!5} \textbf{11} & \cellcolor{blue!5} \textbf{11} & \cellcolor{blue!5} \textbf{11} & \cellcolor{blue!5} \textbf{11} & \cellcolor{blue!5} \textbf{11} & \cellcolor{blue!5} \textbf{11} & \cellcolor{blue!5} \textbf{11} \\
20 245& 13 & \cellcolor{blue!5} \textbf{13} & \cellcolor{blue!5} \textbf{13} & \cellcolor{blue!5} \textbf{13} & \cellcolor{blue!5} \textbf{13} & \cellcolor{blue!5} \textbf{13} & \cellcolor{blue!5} \textbf{13} & \cellcolor{blue!5} \textbf{13} \\
20 246& 13 & \cellcolor{blue!5} \textbf{13} & \cellcolor{blue!5} \textbf{13} & \cellcolor{blue!5} \textbf{13} & \cellcolor{blue!5} \textbf{13} & \cellcolor{blue!5} \textbf{13} & \cellcolor{blue!5} \textbf{13} & \cellcolor{blue!5} 13 \\
20 247& 11 & \cellcolor{blue!5} \textbf{11} & \cellcolor{blue!5} \textbf{11} & \cellcolor{blue!5} \textbf{11} & \cellcolor{blue!5} \textbf{11} & \cellcolor{blue!5} \textbf{11} & \cellcolor{blue!5} \textbf{11} & \cellcolor{blue!5} \textbf{11} \\
20 248& 11 & \cellcolor{blue!5} \textbf{11} & \cellcolor{blue!5} \textbf{11} & \cellcolor{blue!5} \textbf{11} & \cellcolor{blue!5} \textbf{11} & \cellcolor{blue!5} \textbf{11} & \cellcolor{blue!5} \textbf{11} & \cellcolor{blue!5} \textbf{11} \\
20 249& 13 & \cellcolor{blue!5} \textbf{13} & \cellcolor{blue!5} \textbf{13} & \cellcolor{blue!5} \textbf{13} & \cellcolor{blue!5} \textbf{13} & \cellcolor{blue!5} \textbf{13} & \cellcolor{blue!5} \textbf{13} & \cellcolor{blue!5} 13 \\
20 25& 11 & \cellcolor{blue!5} \textbf{11} & \cellcolor{blue!5} \textbf{11} & \cellcolor{blue!5} \textbf{11} & \cellcolor{blue!5} \textbf{11} & \cellcolor{blue!5} \textbf{11} & \cellcolor{blue!5} \textbf{11} & \cellcolor{blue!5} \textbf{11} \\
20 250& 10 & \cellcolor{blue!5} \textbf{10} & \cellcolor{blue!5} \textbf{10} & \cellcolor{blue!5} \textbf{10} & \cellcolor{blue!5} \textbf{10} & \cellcolor{blue!5} \textbf{10} & \cellcolor{blue!5} \textbf{10} & \cellcolor{blue!5} \textbf{10} \\
20 251& 12 & \cellcolor{blue!5} \textbf{12} & \cellcolor{blue!5} \textbf{12} & \cellcolor{blue!5} \textbf{12} & \cellcolor{blue!5} \textbf{12} & \cellcolor{blue!5} \textbf{12} & \cellcolor{blue!5} \textbf{12} & \cellcolor{blue!5} \textbf{12} \\
20 252& 11 & \cellcolor{blue!5} \textbf{11} & \cellcolor{blue!5} \textbf{11} & \cellcolor{blue!5} \textbf{11} & \cellcolor{blue!5} \textbf{11} & \cellcolor{blue!5} \textbf{11} & \cellcolor{blue!5} \textbf{11} & \cellcolor{blue!5} \textbf{11} \\
20 253& 11 & \cellcolor{blue!5} 13 & \cellcolor{blue!5} \textbf{13} & \cellcolor{blue!5} \textbf{13} & \cellcolor{blue!5} \textbf{13} & \cellcolor{blue!5} \textbf{13} & \cellcolor{blue!5} \textbf{13} & \cellcolor{blue!5} 13 \\
20 254& 12 & \cellcolor{blue!5} \textbf{12} & \cellcolor{blue!5} \textbf{12} & \cellcolor{blue!5} \textbf{12} & \cellcolor{blue!5} \textbf{12} & \cellcolor{blue!5} \textbf{12} & \cellcolor{blue!5} \textbf{12} & \cellcolor{blue!5} \textbf{12} \\
20 255& 13 & \cellcolor{blue!5} \textbf{13} & \cellcolor{blue!5} \textbf{13} & \cellcolor{blue!5} \textbf{13} & \cellcolor{blue!5} \textbf{13} & \cellcolor{blue!5} \textbf{13} & \cellcolor{blue!5} \textbf{13} & \cellcolor{blue!5} \textbf{13} \\
20 256& 14 & \cellcolor{blue!5} \textbf{14} & \cellcolor{blue!5} \textbf{14} & \cellcolor{blue!5} \textbf{14} & \cellcolor{blue!5} \textbf{14} & \cellcolor{blue!5} \textbf{14} & \cellcolor{blue!5} \textbf{14} & \cellcolor{blue!5} \textbf{14} \\
20 257& 10 & \cellcolor{blue!5} \textbf{10} & \cellcolor{blue!5} \textbf{10} & \cellcolor{blue!5} \textbf{10} & \cellcolor{blue!5} \textbf{10} & \cellcolor{blue!5} \textbf{10} & \cellcolor{blue!5} \textbf{10} & \cellcolor{blue!5} \textbf{10} \\
20 258& 13 & \cellcolor{blue!5} \textbf{13} & \cellcolor{blue!5} \textbf{13} & \cellcolor{blue!5} \textbf{13} & \cellcolor{blue!5} \textbf{13} & \cellcolor{blue!5} \textbf{13} & \cellcolor{blue!5} \textbf{13} & \cellcolor{blue!5} \textbf{13} \\
20 259& 13 & \cellcolor{blue!5} \textbf{13} & \cellcolor{blue!5} \textbf{13} & \cellcolor{blue!5} \textbf{13} & \cellcolor{blue!5} \textbf{13} & \cellcolor{blue!5} \textbf{13} & \cellcolor{blue!5} \textbf{13} & \cellcolor{blue!5} \textbf{13} \\
20 26& 12 & \cellcolor{blue!5} \textbf{12} & \cellcolor{blue!5} \textbf{12} & \cellcolor{blue!5} \textbf{12} & \cellcolor{blue!5} \textbf{12} & \cellcolor{blue!5} \textbf{12} & \cellcolor{blue!5} \textbf{12} & \cellcolor{blue!5} \textbf{12} \\
20 260& 12 & \cellcolor{blue!5} \textbf{12} & \cellcolor{blue!5} \textbf{12} & \cellcolor{blue!5} \textbf{12} & \cellcolor{blue!5} \textbf{12} & \cellcolor{blue!5} \textbf{12} & \cellcolor{blue!5} \textbf{12} & \cellcolor{blue!5} \textbf{12} \\
20 261& 12 & \cellcolor{blue!5} \textbf{12} & \cellcolor{blue!5} \textbf{12} & \cellcolor{blue!5} \textbf{12} & \cellcolor{blue!5} \textbf{12} & \cellcolor{blue!5} \textbf{12} & \cellcolor{blue!5} \textbf{12} & \cellcolor{blue!5} \textbf{12} \\
20 262& 11 & \cellcolor{blue!5} \textbf{11} & \cellcolor{blue!5} \textbf{11} & \cellcolor{blue!5} \textbf{11} & \cellcolor{blue!5} \textbf{11} & \cellcolor{blue!5} \textbf{11} & \cellcolor{blue!5} \textbf{11} & \cellcolor{blue!5} \textbf{11} \\
20 263& 12 & \cellcolor{blue!5} \textbf{12} & \cellcolor{blue!5} \textbf{12} & \cellcolor{blue!5} \textbf{12} & \cellcolor{blue!5} \textbf{12} & \cellcolor{blue!5} \textbf{12} & \cellcolor{blue!5} \textbf{12} & \cellcolor{blue!5} \textbf{12} \\
20 264& 12 & \cellcolor{blue!5} \textbf{12} & \cellcolor{blue!5} \textbf{12} & \cellcolor{blue!5} \textbf{12} & \cellcolor{blue!5} \textbf{12} & \cellcolor{blue!5} \textbf{12} & \cellcolor{blue!5} \textbf{12} & \cellcolor{blue!5} \textbf{12} \\
20 265& 12 & \cellcolor{blue!5} \textbf{12} & \cellcolor{blue!5} \textbf{12} & \cellcolor{blue!5} \textbf{12} & \cellcolor{blue!5} \textbf{12} & \cellcolor{blue!5} \textbf{12} & \cellcolor{blue!5} \textbf{12} & \cellcolor{blue!5} \textbf{12} \\
20 266& 5 & \cellcolor{blue!5} \textbf{5} & \cellcolor{blue!5} \textbf{5} & \cellcolor{blue!5} \textbf{5} & \cellcolor{blue!5} \textbf{5} & \cellcolor{blue!5} \textbf{5} & \cellcolor{blue!5} \textbf{5} & \cellcolor{blue!5} \textbf{5} \\
20 267& 6 & \cellcolor{blue!5} \textbf{6} & \cellcolor{blue!5} \textbf{6} & \cellcolor{blue!5} \textbf{6} & \cellcolor{blue!5} \textbf{6} & \cellcolor{blue!5} \textbf{6} & \cellcolor{blue!5} \textbf{6} & \cellcolor{blue!5} \textbf{6} \\
20 268& 6 & \cellcolor{blue!5} \textbf{6} & \cellcolor{blue!5} \textbf{6} & \cellcolor{blue!5} \textbf{6} & \cellcolor{blue!5} \textbf{6} & \cellcolor{blue!5} \textbf{6} & \cellcolor{blue!5} \textbf{6} & \cellcolor{blue!5} \textbf{6} \\
20 269& 7 & \cellcolor{blue!5} \textbf{7} & \cellcolor{blue!5} \textbf{7} & \cellcolor{blue!5} \textbf{7} & \cellcolor{blue!5} \textbf{7} & \cellcolor{blue!5} \textbf{7} & \cellcolor{blue!5} \textbf{7} & \cellcolor{blue!5} \textbf{7} \\
20 27& 13 & \cellcolor{blue!5} \textbf{13} & \cellcolor{blue!5} \textbf{13} & \cellcolor{blue!5} \textbf{13} & \cellcolor{blue!5} \textbf{13} & \cellcolor{blue!5} \textbf{13} & \cellcolor{blue!5} \textbf{13} & \cellcolor{blue!5} 13 \\
20 270& 7 & \cellcolor{blue!5} \textbf{7} & \cellcolor{blue!5} \textbf{7} & \cellcolor{blue!5} \textbf{7} & \cellcolor{blue!5} \textbf{7} & \cellcolor{blue!5} \textbf{7} & \cellcolor{blue!5} \textbf{7} & \cellcolor{blue!5} \textbf{7} \\
20 271& 6 & \cellcolor{blue!5} \textbf{6} & \cellcolor{blue!5} \textbf{6} & \cellcolor{blue!5} \textbf{6} & \cellcolor{blue!5} \textbf{6} & \cellcolor{blue!5} \textbf{6} & \cellcolor{blue!5} \textbf{6} & \cellcolor{blue!5} \textbf{6} \\
20 272& 5 & \cellcolor{blue!5} \textbf{5} & \cellcolor{blue!5} \textbf{5} & \cellcolor{blue!5} \textbf{5} & \cellcolor{blue!5} \textbf{5} & \cellcolor{blue!5} \textbf{5} & \cellcolor{blue!5} \textbf{5} & \cellcolor{blue!5} \textbf{5} \\
20 273& 5 & \cellcolor{blue!5} \textbf{5} & \cellcolor{blue!5} \textbf{5} & \cellcolor{blue!5} \textbf{5} & \cellcolor{blue!5} \textbf{5} & \cellcolor{blue!5} \textbf{5} & \cellcolor{blue!5} \textbf{5} & \cellcolor{blue!5} \textbf{5} \\
20 274& 6 & \cellcolor{blue!5} \textbf{6} & \cellcolor{blue!5} \textbf{6} & \cellcolor{blue!5} \textbf{6} & \cellcolor{blue!5} \textbf{6} & \cellcolor{blue!5} \textbf{6} & \cellcolor{blue!5} \textbf{6} & \cellcolor{blue!5} \textbf{6} \\
20 275& 5 & \cellcolor{blue!5} \textbf{5} & \cellcolor{blue!5} \textbf{5} & \cellcolor{blue!5} \textbf{5} & \cellcolor{blue!5} \textbf{5} & \cellcolor{blue!5} \textbf{5} & \cellcolor{blue!5} \textbf{5} & \cellcolor{blue!5} \textbf{5} \\
20 276& 4 & \cellcolor{blue!5} \textbf{4} & \cellcolor{blue!5} \textbf{4} & \cellcolor{blue!5} \textbf{4} & \cellcolor{blue!5} \textbf{4} & \cellcolor{blue!5} \textbf{4} & \cellcolor{blue!5} \textbf{4} & \cellcolor{blue!5} \textbf{4} \\
20 277& 4 & \cellcolor{blue!5} \textbf{4} & \cellcolor{blue!5} \textbf{4} & \cellcolor{blue!5} \textbf{4} & \cellcolor{blue!5} \textbf{4} & \cellcolor{blue!5} \textbf{4} & \cellcolor{blue!5} \textbf{4} & \cellcolor{blue!5} \textbf{4} \\
20 278& 6 & \cellcolor{blue!5} \textbf{6} & \cellcolor{blue!5} \textbf{6} & \cellcolor{blue!5} \textbf{6} & \cellcolor{blue!5} \textbf{6} & \cellcolor{blue!5} \textbf{6} & \cellcolor{blue!5} \textbf{6} & \cellcolor{blue!5} \textbf{6} \\
20 279& 6 & \cellcolor{blue!5} \textbf{6} & \cellcolor{blue!5} \textbf{6} & \cellcolor{blue!5} \textbf{6} & \cellcolor{blue!5} \textbf{6} & \cellcolor{blue!5} \textbf{6} & \cellcolor{blue!5} \textbf{6} & \cellcolor{blue!5} \textbf{6} \\
20 28& 12 & \cellcolor{blue!5} \textbf{12} & \cellcolor{blue!5} \textbf{12} & \cellcolor{blue!5} \textbf{12} & \cellcolor{blue!5} \textbf{12} & \cellcolor{blue!5} \textbf{12} & \cellcolor{blue!5} \textbf{12} & \cellcolor{blue!5} \textbf{12} \\
20 280& 5 & \cellcolor{blue!5} \textbf{5} & \cellcolor{blue!5} \textbf{5} & \cellcolor{blue!5} \textbf{5} & \cellcolor{blue!5} \textbf{5} & \cellcolor{blue!5} \textbf{5} & \cellcolor{blue!5} \textbf{5} & \cellcolor{blue!5} \textbf{5} \\
20 281& 4 & \cellcolor{blue!5} \textbf{4} & \cellcolor{blue!5} \textbf{4} & \cellcolor{blue!5} \textbf{4} & \cellcolor{blue!5} \textbf{4} & \cellcolor{blue!5} \textbf{4} & \cellcolor{blue!5} \textbf{4} & \cellcolor{blue!5} \textbf{4} \\
20 282& 4 & \cellcolor{blue!5} \textbf{4} & \cellcolor{blue!5} \textbf{4} & \cellcolor{blue!5} \textbf{4} & \cellcolor{blue!5} \textbf{4} & \cellcolor{blue!5} \textbf{4} & \cellcolor{blue!5} \textbf{4} & \cellcolor{blue!5} \textbf{4} \\
20 283& 5 & \cellcolor{blue!5} \textbf{5} & \cellcolor{blue!5} \textbf{5} & \cellcolor{blue!5} \textbf{5} & \cellcolor{blue!5} \textbf{5} & \cellcolor{blue!5} \textbf{5} & \cellcolor{blue!5} \textbf{5} & \cellcolor{blue!5} \textbf{5} \\
20 284& 5 & \cellcolor{blue!5} \textbf{5} & \cellcolor{blue!5} \textbf{5} & \cellcolor{blue!5} \textbf{5} & \cellcolor{blue!5} \textbf{5} & \cellcolor{blue!5} \textbf{5} & \cellcolor{blue!5} \textbf{5} & \cellcolor{blue!5} \textbf{5} \\
20 285& 5 & \cellcolor{blue!5} \textbf{5} & \cellcolor{blue!5} \textbf{5} & \cellcolor{blue!5} \textbf{5} & \cellcolor{blue!5} \textbf{5} & \cellcolor{blue!5} \textbf{5} & \cellcolor{blue!5} \textbf{5} & \cellcolor{blue!5} \textbf{5} \\
20 286& 5 & \cellcolor{blue!5} \textbf{5} & \cellcolor{blue!5} \textbf{5} & \cellcolor{blue!5} \textbf{5} & \cellcolor{blue!5} \textbf{5} & \cellcolor{blue!5} \textbf{5} & \cellcolor{blue!5} \textbf{5} & \cellcolor{blue!5} \textbf{5} \\
20 287& 5 & \cellcolor{blue!5} \textbf{5} & \cellcolor{blue!5} \textbf{5} & \cellcolor{blue!5} \textbf{5} & \cellcolor{blue!5} \textbf{5} & \cellcolor{blue!5} \textbf{5} & \cellcolor{blue!5} \textbf{5} & \cellcolor{blue!5} \textbf{5} \\
20 288& 6 & \cellcolor{blue!5} \textbf{6} & \cellcolor{blue!5} \textbf{6} & \cellcolor{blue!5} \textbf{6} & \cellcolor{blue!5} \textbf{6} & \cellcolor{blue!5} \textbf{6} & \cellcolor{blue!5} \textbf{6} & \cellcolor{blue!5} \textbf{6} \\
20 289& 5 & \cellcolor{blue!5} \textbf{5} & \cellcolor{blue!5} \textbf{5} & \cellcolor{blue!5} \textbf{5} & \cellcolor{blue!5} \textbf{5} & \cellcolor{blue!5} \textbf{5} & \cellcolor{blue!5} \textbf{5} & \cellcolor{blue!5} \textbf{5} \\
20 29& 10 & \cellcolor{blue!5} \textbf{10} & \cellcolor{blue!5} \textbf{10} & \cellcolor{blue!5} \textbf{10} & \cellcolor{blue!5} \textbf{10} & \cellcolor{blue!5} \textbf{10} & \cellcolor{blue!5} \textbf{10} & \cellcolor{blue!5} \textbf{10} \\
20 290& 5 & \cellcolor{blue!5} \textbf{5} & \cellcolor{blue!5} \textbf{5} & \cellcolor{blue!5} \textbf{5} & \cellcolor{blue!5} \textbf{5} & \cellcolor{blue!5} \textbf{5} & \cellcolor{blue!5} \textbf{5} & \cellcolor{blue!5} \textbf{5} \\
20 291& 3 & \cellcolor{blue!5} \textbf{3} & \cellcolor{blue!5} \textbf{3} & \cellcolor{blue!5} \textbf{3} & \cellcolor{blue!5} \textbf{3} & \cellcolor{blue!5} \textbf{3} & \cellcolor{blue!5} \textbf{3} & \cellcolor{blue!5} \textbf{3} \\
20 292& 3 & \cellcolor{blue!5} \textbf{3} & \cellcolor{blue!5} \textbf{3} & \cellcolor{blue!5} \textbf{3} & \cellcolor{blue!5} \textbf{3} & \cellcolor{blue!5} \textbf{3} & \cellcolor{blue!5} \textbf{3} & \cellcolor{blue!5} \textbf{3} \\
20 293& 3 & \cellcolor{blue!5} \textbf{3} & \cellcolor{blue!5} \textbf{3} & \cellcolor{blue!5} \textbf{3} & \cellcolor{blue!5} \textbf{3} & \cellcolor{blue!5} \textbf{3} & \cellcolor{blue!5} \textbf{3} & \cellcolor{blue!5} \textbf{3} \\
20 294& 3 & \cellcolor{blue!5} \textbf{3} & \cellcolor{blue!5} \textbf{3} & \cellcolor{blue!5} \textbf{3} & \cellcolor{blue!5} \textbf{3} & \cellcolor{blue!5} \textbf{3} & \cellcolor{blue!5} \textbf{3} & \cellcolor{blue!5} \textbf{3} \\
20 295& 3 & \cellcolor{blue!5} \textbf{3} & \cellcolor{blue!5} \textbf{3} & \cellcolor{blue!5} \textbf{3} & \cellcolor{blue!5} \textbf{3} & \cellcolor{blue!5} \textbf{3} & \cellcolor{blue!5} \textbf{3} & \cellcolor{blue!5} \textbf{3} \\
20 296& 3 & \cellcolor{blue!5} \textbf{3} & \cellcolor{blue!5} \textbf{3} & \cellcolor{blue!5} \textbf{3} & \cellcolor{blue!5} \textbf{3} & \cellcolor{blue!5} \textbf{3} & \cellcolor{blue!5} \textbf{3} & \cellcolor{blue!5} \textbf{3} \\
20 297& 3 & \cellcolor{blue!5} \textbf{3} & \cellcolor{blue!5} \textbf{3} & \cellcolor{blue!5} \textbf{3} & \cellcolor{blue!5} \textbf{3} & \cellcolor{blue!5} \textbf{3} & \cellcolor{blue!5} \textbf{3} & \cellcolor{blue!5} \textbf{3} \\
20 298& 3 & \cellcolor{blue!5} \textbf{3} & \cellcolor{blue!5} \textbf{3} & \cellcolor{blue!5} \textbf{3} & \cellcolor{blue!5} \textbf{3} & \cellcolor{blue!5} \textbf{3} & \cellcolor{blue!5} \textbf{3} & \cellcolor{blue!5} \textbf{3} \\
20 299& 3 & \cellcolor{blue!5} \textbf{3} & \cellcolor{blue!5} \textbf{3} & \cellcolor{blue!5} \textbf{3} & \cellcolor{blue!5} \textbf{3} & \cellcolor{blue!5} \textbf{3} & \cellcolor{blue!5} \textbf{3} & \cellcolor{blue!5} \textbf{3} \\
20 3& 3 & \cellcolor{blue!5} \textbf{3} & \cellcolor{blue!5} \textbf{3} & \cellcolor{blue!5} \textbf{3} & \cellcolor{blue!5} \textbf{3} & \cellcolor{blue!5} \textbf{3} & \cellcolor{blue!5} \textbf{3} & \cellcolor{blue!5} \textbf{3} \\
20 30& 16 & \cellcolor{blue!5} \textbf{16} & \cellcolor{blue!5} \textbf{16} & \cellcolor{blue!5} \textbf{16} & \cellcolor{blue!5} \textbf{16} & \cellcolor{blue!5} \textbf{16} & \cellcolor{blue!5} 16 & \cellcolor{blue!5} 16 \\
20 300& 4 & \cellcolor{blue!5} \textbf{4} & \cellcolor{blue!5} \textbf{4} & \cellcolor{blue!5} \textbf{4} & \cellcolor{blue!5} \textbf{4} & \cellcolor{blue!5} \textbf{4} & \cellcolor{blue!5} \textbf{4} & \cellcolor{blue!5} \textbf{4} \\
20 301& 3 & \cellcolor{blue!5} \textbf{3} & \cellcolor{blue!5} \textbf{3} & \cellcolor{blue!5} \textbf{3} & \cellcolor{blue!5} \textbf{3} & \cellcolor{blue!5} \textbf{3} & \cellcolor{blue!5} \textbf{3} & \cellcolor{blue!5} \textbf{3} \\
20 302& 3 & \cellcolor{blue!5} \textbf{3} & \cellcolor{blue!5} \textbf{3} & \cellcolor{blue!5} \textbf{3} & \cellcolor{blue!5} \textbf{3} & \cellcolor{blue!5} \textbf{3} & \cellcolor{blue!5} \textbf{3} & \cellcolor{blue!5} \textbf{3} \\
20 303& 3 & \cellcolor{blue!5} \textbf{3} & \cellcolor{blue!5} \textbf{3} & \cellcolor{blue!5} \textbf{3} & \cellcolor{blue!5} \textbf{3} & \cellcolor{blue!5} \textbf{3} & \cellcolor{blue!5} \textbf{3} & \cellcolor{blue!5} \textbf{3} \\
20 304& 3 & \cellcolor{blue!5} \textbf{3} & \cellcolor{blue!5} \textbf{3} & \cellcolor{blue!5} \textbf{3} & \cellcolor{blue!5} \textbf{3} & \cellcolor{blue!5} \textbf{3} & \cellcolor{blue!5} \textbf{3} & \cellcolor{blue!5} \textbf{3} \\
20 305& 3 & \cellcolor{blue!5} \textbf{3} & \cellcolor{blue!5} \textbf{3} & \cellcolor{blue!5} \textbf{3} & \cellcolor{blue!5} \textbf{3} & \cellcolor{blue!5} \textbf{3} & \cellcolor{blue!5} \textbf{3} & \cellcolor{blue!5} \textbf{3} \\
20 306& 3 & \cellcolor{blue!5} \textbf{3} & \cellcolor{blue!5} \textbf{3} & \cellcolor{blue!5} \textbf{3} & \cellcolor{blue!5} \textbf{3} & \cellcolor{blue!5} \textbf{3} & \cellcolor{blue!5} \textbf{3} & \cellcolor{blue!5} \textbf{3} \\
20 307& 3 & \cellcolor{blue!5} \textbf{3} & \cellcolor{blue!5} \textbf{3} & \cellcolor{blue!5} \textbf{3} & \cellcolor{blue!5} \textbf{3} & \cellcolor{blue!5} \textbf{3} & \cellcolor{blue!5} \textbf{3} & \cellcolor{blue!5} \textbf{3} \\
20 308& 3 & \cellcolor{blue!5} \textbf{3} & \cellcolor{blue!5} \textbf{3} & \cellcolor{blue!5} \textbf{3} & \cellcolor{blue!5} \textbf{3} & \cellcolor{blue!5} \textbf{3} & \cellcolor{blue!5} \textbf{3} & \cellcolor{blue!5} \textbf{3} \\
20 309& 3 & \cellcolor{blue!5} \textbf{3} & \cellcolor{blue!5} \textbf{3} & \cellcolor{blue!5} \textbf{3} & \cellcolor{blue!5} \textbf{3} & \cellcolor{blue!5} \textbf{3} & \cellcolor{blue!5} \textbf{3} & \cellcolor{blue!5} \textbf{3} \\
20 31& 12 & \cellcolor{blue!5} \textbf{12} & \cellcolor{blue!5} \textbf{12} & \cellcolor{blue!5} \textbf{12} & \cellcolor{blue!5} \textbf{12} & \cellcolor{blue!5} \textbf{12} & \cellcolor{blue!5} \textbf{12} & \cellcolor{blue!5} \textbf{12} \\
20 310& 3 & \cellcolor{blue!5} \textbf{3} & \cellcolor{blue!5} \textbf{3} & \cellcolor{blue!5} \textbf{3} & \cellcolor{blue!5} \textbf{3} & \cellcolor{blue!5} \textbf{3} & \cellcolor{blue!5} \textbf{3} & \cellcolor{blue!5} \textbf{3} \\
20 311& 3 & \cellcolor{blue!5} \textbf{3} & \cellcolor{blue!5} \textbf{3} & \cellcolor{blue!5} \textbf{3} & \cellcolor{blue!5} \textbf{3} & \cellcolor{blue!5} \textbf{3} & \cellcolor{blue!5} \textbf{3} & \cellcolor{blue!5} \textbf{3} \\
20 312& 4 & \cellcolor{blue!5} \textbf{4} & \cellcolor{blue!5} \textbf{4} & \cellcolor{blue!5} \textbf{4} & \cellcolor{blue!5} \textbf{4} & \cellcolor{blue!5} \textbf{4} & \cellcolor{blue!5} \textbf{4} & \cellcolor{blue!5} \textbf{4} \\
20 313& 3 & \cellcolor{blue!5} \textbf{3} & \cellcolor{blue!5} \textbf{3} & \cellcolor{blue!5} \textbf{3} & \cellcolor{blue!5} \textbf{3} & \cellcolor{blue!5} \textbf{3} & \cellcolor{blue!5} \textbf{3} & \cellcolor{blue!5} \textbf{3} \\
20 314& 3 & \cellcolor{blue!5} \textbf{3} & \cellcolor{blue!5} \textbf{3} & \cellcolor{blue!5} \textbf{3} & \cellcolor{blue!5} \textbf{3} & \cellcolor{blue!5} \textbf{3} & \cellcolor{blue!5} \textbf{3} & \cellcolor{blue!5} \textbf{3} \\
20 315& 3 & \cellcolor{blue!5} \textbf{3} & \cellcolor{blue!5} \textbf{3} & \cellcolor{blue!5} \textbf{3} & \cellcolor{blue!5} \textbf{3} & \cellcolor{blue!5} \textbf{3} & \cellcolor{blue!5} \textbf{3} & \cellcolor{blue!5} \textbf{3} \\
20 316& 10 & \cellcolor{blue!5} \textbf{10} & \cellcolor{blue!5} \textbf{10} & \cellcolor{blue!5} \textbf{10} & \cellcolor{blue!5} \textbf{10} & \cellcolor{blue!5} \textbf{10} & \cellcolor{blue!5} \textbf{10} & \cellcolor{blue!5} 10 \\
20 317& 10 & \cellcolor{blue!5} \textbf{10} & \cellcolor{blue!5} \textbf{10} & \cellcolor{blue!5} \textbf{10} & \cellcolor{blue!5} \textbf{10} & \cellcolor{blue!5} \textbf{10} & \cellcolor{blue!5} \textbf{10} & \cellcolor{blue!5} 10 \\
20 318& 10 & \cellcolor{blue!5} \textbf{10} & \cellcolor{blue!5} \textbf{10} & \cellcolor{blue!5} \textbf{10} & \cellcolor{blue!5} \textbf{10} & \cellcolor{blue!5} \textbf{10} & \cellcolor{blue!5} \textbf{10} & \cellcolor{blue!5} \textbf{10} \\
20 319& 14 & \cellcolor{blue!5} \textbf{14} & \cellcolor{blue!5} \textbf{14} & \cellcolor{blue!5} \textbf{14} & \cellcolor{blue!5} \textbf{14} & \cellcolor{blue!5} \textbf{14} & \cellcolor{blue!5} \textbf{14} & \cellcolor{blue!5} 14 \\
20 32& 13 & \cellcolor{blue!5} \textbf{13} & \cellcolor{blue!5} \textbf{13} & \cellcolor{blue!5} \textbf{13} & \cellcolor{blue!5} \textbf{13} & \cellcolor{blue!5} \textbf{13} & \cellcolor{blue!5} \textbf{13} & \cellcolor{blue!5} 13 \\
20 320& 12 & \cellcolor{blue!5} \textbf{12} & \cellcolor{blue!5} \textbf{12} & \cellcolor{blue!5} \textbf{12} & \cellcolor{blue!5} \textbf{12} & \cellcolor{blue!5} \textbf{12} & \cellcolor{blue!5} \textbf{12} & \cellcolor{blue!5} \textbf{12} \\
20 321& 14 & \cellcolor{blue!5} \textbf{14} & \cellcolor{blue!5} 14 & \cellcolor{blue!5} \textbf{14} & \cellcolor{blue!5} \textbf{14} & \cellcolor{blue!5} \textbf{14} & \cellcolor{blue!5} 14 & \cellcolor{blue!5} 14 \\
20 322& 12 & \cellcolor{blue!5} \textbf{12} & \cellcolor{blue!5} \textbf{12} & \cellcolor{blue!5} \textbf{12} & \cellcolor{blue!5} \textbf{12} & \cellcolor{blue!5} \textbf{12} & \cellcolor{blue!5} \textbf{12} & \cellcolor{blue!5} 12 \\
20 323& 13 & \cellcolor{blue!5} \textbf{13} & \cellcolor{blue!5} \textbf{13} & \cellcolor{blue!5} \textbf{13} & \cellcolor{blue!5} \textbf{13} & \cellcolor{blue!5} \textbf{13} & \cellcolor{blue!5} \textbf{13} & \cellcolor{blue!5} 13 \\
20 324& 9 & \cellcolor{blue!5} \textbf{9} & \cellcolor{blue!5} \textbf{9} & \cellcolor{blue!5} \textbf{9} & \cellcolor{blue!5} \textbf{9} & \cellcolor{blue!5} \textbf{9} & \cellcolor{blue!5} \textbf{9} & \cellcolor{blue!5} \textbf{9} \\
20 325& 14 & \cellcolor{blue!5} \textbf{14} & \cellcolor{blue!5} \textbf{14} & \cellcolor{blue!5} \textbf{14} & \cellcolor{blue!5} \textbf{14} & \cellcolor{blue!5} \textbf{14} & \cellcolor{blue!5} 14 & \cellcolor{blue!5} 14 \\
20 326& 14 & \cellcolor{blue!5} \textbf{14} & \cellcolor{blue!5} \textbf{14} & \cellcolor{blue!5} \textbf{14} & \cellcolor{blue!5} \textbf{14} & \cellcolor{blue!5} \textbf{14} & \cellcolor{blue!5} 14 & \cellcolor{blue!5} 14 \\
20 327& 13 & \cellcolor{blue!5} \textbf{13} & \cellcolor{blue!5} \textbf{13} & \cellcolor{blue!5} \textbf{13} & \cellcolor{blue!5} \textbf{13} & \cellcolor{blue!5} \textbf{13} & \cellcolor{blue!5} \textbf{13} & \cellcolor{blue!5} 13 \\
20 328& 13 & \cellcolor{blue!5} \textbf{13} & \cellcolor{blue!5} \textbf{13} & \cellcolor{blue!5} \textbf{13} & \cellcolor{blue!5} \textbf{13} & \cellcolor{blue!5} \textbf{13} & \cellcolor{blue!5} \textbf{13} & \cellcolor{blue!5} 13 \\
20 329& 10 & \cellcolor{blue!5} \textbf{10} & \cellcolor{blue!5} \textbf{10} & \cellcolor{blue!5} \textbf{10} & \cellcolor{blue!5} \textbf{10} & \cellcolor{blue!5} \textbf{10} & \cellcolor{blue!5} \textbf{10} & \cellcolor{blue!5} \textbf{10} \\
20 33& 11 & \cellcolor{blue!5} \textbf{11} & \cellcolor{blue!5} \textbf{11} & \cellcolor{blue!5} \textbf{11} & \cellcolor{blue!5} \textbf{11} & \cellcolor{blue!5} \textbf{11} & \cellcolor{blue!5} \textbf{11} & \cellcolor{blue!5} \textbf{11} \\
20 330& 12 & \cellcolor{blue!5} \textbf{12} & \cellcolor{blue!5} \textbf{12} & \cellcolor{blue!5} \textbf{12} & \cellcolor{blue!5} \textbf{12} & \cellcolor{blue!5} \textbf{12} & \cellcolor{blue!5} 12 & \cellcolor{blue!5} 12 \\
20 331& 13 & \cellcolor{blue!5} \textbf{13} & \cellcolor{blue!5} \textbf{13} & \cellcolor{blue!5} \textbf{13} & \cellcolor{blue!5} \textbf{13} & \cellcolor{blue!5} \textbf{13} & \cellcolor{blue!5} 13 & \cellcolor{blue!5} 13 \\
20 332& 13 & \cellcolor{blue!5} \textbf{13} & \cellcolor{blue!5} \textbf{13} & \cellcolor{blue!5} \textbf{13} & \cellcolor{blue!5} \textbf{13} & \cellcolor{blue!5} \textbf{13} & \cellcolor{blue!5} \textbf{13} & \cellcolor{blue!5} 13 \\
20 333& 11 & \cellcolor{blue!5} \textbf{11} & \cellcolor{blue!5} \textbf{11} & \cellcolor{blue!5} \textbf{11} & \cellcolor{blue!5} \textbf{11} & \cellcolor{blue!5} \textbf{11} & \cellcolor{blue!5} \textbf{11} & \cellcolor{blue!5} \textbf{11} \\
20 334& 10 & \cellcolor{blue!5} \textbf{10} & \cellcolor{blue!5} \textbf{10} & \cellcolor{blue!5} \textbf{10} & \cellcolor{blue!5} \textbf{10} & \cellcolor{blue!5} \textbf{10} & \cellcolor{blue!5} \textbf{10} & \cellcolor{blue!5} \textbf{10} \\
20 335& 14 & \cellcolor{blue!5} \textbf{14} & \cellcolor{blue!5} 14 & \cellcolor{blue!5} \textbf{14} & \cellcolor{blue!5} \textbf{14} & \cellcolor{blue!5} \textbf{14} & \cellcolor{blue!5} 14 & \cellcolor{blue!5} 14 \\
20 336& 11 & \cellcolor{blue!5} \textbf{11} & \cellcolor{blue!5} \textbf{11} & \cellcolor{blue!5} \textbf{11} & \cellcolor{blue!5} \textbf{11} & \cellcolor{blue!5} \textbf{11} & \cellcolor{blue!5} \textbf{11} & \cellcolor{blue!5} \textbf{11} \\
20 337& 10 & \cellcolor{blue!5} \textbf{10} & \cellcolor{blue!5} \textbf{10} & \cellcolor{blue!5} \textbf{10} & \cellcolor{blue!5} \textbf{10} & \cellcolor{blue!5} \textbf{10} & \cellcolor{blue!5} \textbf{10} & \cellcolor{blue!5} \textbf{10} \\
20 338& 14 & \cellcolor{blue!5} \textbf{14} & \cellcolor{blue!5} \textbf{14} & \cellcolor{blue!5} \textbf{14} & \cellcolor{blue!5} \textbf{14} & \cellcolor{blue!5} \textbf{14} & \cellcolor{blue!5} \textbf{14} & \cellcolor{blue!5} 14 \\
20 339& 13 & \cellcolor{blue!5} \textbf{13} & \cellcolor{blue!5} \textbf{13} & \cellcolor{blue!5} \textbf{13} & \cellcolor{blue!5} \textbf{13} & \cellcolor{blue!5} \textbf{13} & \cellcolor{blue!5} \textbf{13} & \cellcolor{blue!5} 13 \\
20 34& 12 & \cellcolor{blue!5} \textbf{12} & \cellcolor{blue!5} \textbf{12} & \cellcolor{blue!5} \textbf{12} & \cellcolor{blue!5} \textbf{12} & \cellcolor{blue!5} \textbf{12} & \cellcolor{blue!5} \textbf{12} & \cellcolor{blue!5} \textbf{12} \\
20 340& 11 & \cellcolor{blue!5} \textbf{11} & \cellcolor{blue!5} \textbf{11} & \cellcolor{blue!5} \textbf{11} & \cellcolor{blue!5} \textbf{11} & \cellcolor{blue!5} \textbf{11} & \cellcolor{blue!5} \textbf{11} & \cellcolor{blue!5} \textbf{11} \\
20 341& 6 & \cellcolor{blue!5} \textbf{6} & \cellcolor{blue!5} \textbf{6} & \cellcolor{blue!5} \textbf{6} & \cellcolor{blue!5} \textbf{6} & \cellcolor{blue!5} \textbf{6} & \cellcolor{blue!5} \textbf{6} & \cellcolor{blue!5} \textbf{6} \\
20 342& 6 & \cellcolor{blue!5} \textbf{6} & \cellcolor{blue!5} \textbf{6} & \cellcolor{blue!5} \textbf{6} & \cellcolor{blue!5} \textbf{6} & \cellcolor{blue!5} \textbf{6} & \cellcolor{blue!5} \textbf{6} & \cellcolor{blue!5} \textbf{6} \\
20 343& 6 & \cellcolor{blue!5} \textbf{6} & \cellcolor{blue!5} \textbf{6} & \cellcolor{blue!5} \textbf{6} & \cellcolor{blue!5} \textbf{6} & \cellcolor{blue!5} \textbf{6} & \cellcolor{blue!5} \textbf{6} & \cellcolor{blue!5} \textbf{6} \\
20 344& 6 & \cellcolor{blue!5} \textbf{6} & \cellcolor{blue!5} \textbf{6} & \cellcolor{blue!5} \textbf{6} & \cellcolor{blue!5} \textbf{6} & \cellcolor{blue!5} \textbf{6} & \cellcolor{blue!5} \textbf{6} & \cellcolor{blue!5} \textbf{6} \\
20 345& 4 & \cellcolor{blue!5} \textbf{4} & \cellcolor{blue!5} \textbf{4} & \cellcolor{blue!5} \textbf{4} & \cellcolor{blue!5} \textbf{4} & \cellcolor{blue!5} \textbf{4} & \cellcolor{blue!5} \textbf{4} & \cellcolor{blue!5} \textbf{4} \\
20 346& 5 & \cellcolor{blue!5} \textbf{5} & \cellcolor{blue!5} \textbf{5} & \cellcolor{blue!5} \textbf{5} & \cellcolor{blue!5} \textbf{5} & \cellcolor{blue!5} \textbf{5} & \cellcolor{blue!5} \textbf{5} & \cellcolor{blue!5} \textbf{5} \\
20 347& 6 & \cellcolor{blue!5} \textbf{6} & \cellcolor{blue!5} \textbf{6} & \cellcolor{blue!5} \textbf{6} & \cellcolor{blue!5} \textbf{6} & \cellcolor{blue!5} \textbf{6} & \cellcolor{blue!5} \textbf{6} & \cellcolor{blue!5} \textbf{6} \\
20 348& 5 & \cellcolor{blue!5} \textbf{5} & \cellcolor{blue!5} \textbf{5} & \cellcolor{blue!5} \textbf{5} & \cellcolor{blue!5} \textbf{5} & \cellcolor{blue!5} \textbf{5} & \cellcolor{blue!5} \textbf{5} & \cellcolor{blue!5} \textbf{5} \\
20 349& 5 & \cellcolor{blue!5} \textbf{5} & \cellcolor{blue!5} \textbf{5} & \cellcolor{blue!5} \textbf{5} & \cellcolor{blue!5} \textbf{5} & \cellcolor{blue!5} \textbf{5} & \cellcolor{blue!5} \textbf{5} & \cellcolor{blue!5} \textbf{5} \\
20 35& 12 & \cellcolor{blue!5} \textbf{12} & \cellcolor{blue!5} \textbf{12} & \cellcolor{blue!5} \textbf{12} & \cellcolor{blue!5} \textbf{12} & \cellcolor{blue!5} \textbf{12} & \cellcolor{blue!5} \textbf{12} & \cellcolor{blue!5} \textbf{12} \\
20 350& 5 & \cellcolor{blue!5} \textbf{5} & \cellcolor{blue!5} \textbf{5} & \cellcolor{blue!5} \textbf{5} & \cellcolor{blue!5} \textbf{5} & \cellcolor{blue!5} \textbf{5} & \cellcolor{blue!5} \textbf{5} & \cellcolor{blue!5} \textbf{5} \\
20 351& 5 & \cellcolor{blue!5} \textbf{5} & \cellcolor{blue!5} \textbf{5} & \cellcolor{blue!5} \textbf{5} & \cellcolor{blue!5} \textbf{5} & \cellcolor{blue!5} \textbf{5} & \cellcolor{blue!5} \textbf{5} & \cellcolor{blue!5} \textbf{5} \\
20 352& 4 & \cellcolor{blue!5} \textbf{4} & \cellcolor{blue!5} \textbf{4} & \cellcolor{blue!5} \textbf{4} & \cellcolor{blue!5} \textbf{4} & \cellcolor{blue!5} \textbf{4} & \cellcolor{blue!5} \textbf{4} & \cellcolor{blue!5} \textbf{4} \\
20 353& 6 & \cellcolor{blue!5} \textbf{6} & \cellcolor{blue!5} \textbf{6} & \cellcolor{blue!5} \textbf{6} & \cellcolor{blue!5} \textbf{6} & \cellcolor{blue!5} \textbf{6} & \cellcolor{blue!5} \textbf{6} & \cellcolor{blue!5} \textbf{6} \\
20 354& 6 & \cellcolor{blue!5} \textbf{6} & \cellcolor{blue!5} \textbf{6} & \cellcolor{blue!5} \textbf{6} & \cellcolor{blue!5} \textbf{6} & \cellcolor{blue!5} \textbf{6} & \cellcolor{blue!5} \textbf{6} & \cellcolor{blue!5} \textbf{6} \\
20 355& 5 & \cellcolor{blue!5} \textbf{5} & \cellcolor{blue!5} \textbf{5} & \cellcolor{blue!5} \textbf{5} & \cellcolor{blue!5} \textbf{5} & \cellcolor{blue!5} \textbf{5} & \cellcolor{blue!5} \textbf{5} & \cellcolor{blue!5} \textbf{5} \\
20 356& 5 & \cellcolor{blue!5} \textbf{5} & \cellcolor{blue!5} \textbf{5} & \cellcolor{blue!5} \textbf{5} & \cellcolor{blue!5} \textbf{5} & \cellcolor{blue!5} \textbf{5} & \cellcolor{blue!5} \textbf{5} & \cellcolor{blue!5} \textbf{5} \\
20 357& 5 & \cellcolor{blue!5} \textbf{5} & \cellcolor{blue!5} \textbf{5} & \cellcolor{blue!5} \textbf{5} & \cellcolor{blue!5} \textbf{5} & \cellcolor{blue!5} \textbf{5} & \cellcolor{blue!5} \textbf{5} & \cellcolor{blue!5} \textbf{5} \\
20 358& 4 & \cellcolor{blue!5} \textbf{4} & \cellcolor{blue!5} \textbf{4} & \cellcolor{blue!5} \textbf{4} & \cellcolor{blue!5} \textbf{4} & \cellcolor{blue!5} \textbf{4} & \cellcolor{blue!5} \textbf{4} & \cellcolor{blue!5} \textbf{4} \\
20 359& 4 & \cellcolor{blue!5} \textbf{4} & \cellcolor{blue!5} \textbf{4} & \cellcolor{blue!5} \textbf{4} & \cellcolor{blue!5} \textbf{4} & \cellcolor{blue!5} \textbf{4} & \cellcolor{blue!5} \textbf{4} & \cellcolor{blue!5} \textbf{4} \\
20 36& 13 & \cellcolor{blue!5} \textbf{13} & \cellcolor{blue!5} \textbf{13} & \cellcolor{blue!5} \textbf{13} & \cellcolor{blue!5} \textbf{13} & \cellcolor{blue!5} \textbf{13} & \cellcolor{blue!5} \textbf{13} & \cellcolor{blue!5} \textbf{13} \\
20 360& 6 & \cellcolor{blue!5} \textbf{6} & \cellcolor{blue!5} \textbf{6} & \cellcolor{blue!5} \textbf{6} & \cellcolor{blue!5} \textbf{6} & \cellcolor{blue!5} \textbf{6} & \cellcolor{blue!5} \textbf{6} & \cellcolor{blue!5} \textbf{6} \\
20 361& 5 & \cellcolor{blue!5} \textbf{5} & \cellcolor{blue!5} \textbf{5} & \cellcolor{blue!5} \textbf{5} & \cellcolor{blue!5} \textbf{5} & \cellcolor{blue!5} \textbf{5} & \cellcolor{blue!5} \textbf{5} & \cellcolor{blue!5} \textbf{5} \\
20 362& 5 & \cellcolor{blue!5} \textbf{5} & \cellcolor{blue!5} \textbf{5} & \cellcolor{blue!5} \textbf{5} & \cellcolor{blue!5} \textbf{5} & \cellcolor{blue!5} \textbf{5} & \cellcolor{blue!5} \textbf{5} & \cellcolor{blue!5} \textbf{5} \\
20 363& 7 & \cellcolor{blue!5} \textbf{7} & \cellcolor{blue!5} \textbf{7} & \cellcolor{blue!5} \textbf{7} & \cellcolor{blue!5} \textbf{7} & \cellcolor{blue!5} \textbf{7} & \cellcolor{blue!5} \textbf{7} & \cellcolor{blue!5} \textbf{7} \\
20 364& 4 & \cellcolor{blue!5} \textbf{4} & \cellcolor{blue!5} \textbf{4} & \cellcolor{blue!5} \textbf{4} & \cellcolor{blue!5} \textbf{4} & \cellcolor{blue!5} \textbf{4} & \cellcolor{blue!5} \textbf{4} & \cellcolor{blue!5} \textbf{4} \\
20 365& 5 & \cellcolor{blue!5} \textbf{5} & \cellcolor{blue!5} \textbf{5} & \cellcolor{blue!5} \textbf{5} & \cellcolor{blue!5} \textbf{5} & \cellcolor{blue!5} \textbf{5} & \cellcolor{blue!5} \textbf{5} & \cellcolor{blue!5} \textbf{5} \\
20 366& 3 & \cellcolor{blue!5} \textbf{3} & \cellcolor{blue!5} \textbf{3} & \cellcolor{blue!5} \textbf{3} & \cellcolor{blue!5} \textbf{3} & \cellcolor{blue!5} \textbf{3} & \cellcolor{blue!5} \textbf{3} & \cellcolor{blue!5} \textbf{3} \\
20 367& 3 & \cellcolor{blue!5} \textbf{3} & \cellcolor{blue!5} \textbf{3} & \cellcolor{blue!5} \textbf{3} & \cellcolor{blue!5} \textbf{3} & \cellcolor{blue!5} \textbf{3} & \cellcolor{blue!5} \textbf{3} & \cellcolor{blue!5} \textbf{3} \\
20 368& 3 & \cellcolor{blue!5} \textbf{3} & \cellcolor{blue!5} \textbf{3} & \cellcolor{blue!5} \textbf{3} & \cellcolor{blue!5} \textbf{3} & \cellcolor{blue!5} \textbf{3} & \cellcolor{blue!5} \textbf{3} & \cellcolor{blue!5} \textbf{3} \\
20 369& 3 & \cellcolor{blue!5} \textbf{3} & \cellcolor{blue!5} \textbf{3} & \cellcolor{blue!5} \textbf{3} & \cellcolor{blue!5} \textbf{3} & \cellcolor{blue!5} \textbf{3} & \cellcolor{blue!5} \textbf{3} & \cellcolor{blue!5} \textbf{3} \\
20 37& 12 & \cellcolor{blue!5} \textbf{12} & \cellcolor{blue!5} \textbf{12} & \cellcolor{blue!5} \textbf{12} & \cellcolor{blue!5} \textbf{12} & \cellcolor{blue!5} \textbf{12} & \cellcolor{blue!5} \textbf{12} & \cellcolor{blue!5} 12 \\
20 370& 3 & \cellcolor{blue!5} \textbf{3} & \cellcolor{blue!5} \textbf{3} & \cellcolor{blue!5} \textbf{3} & \cellcolor{blue!5} \textbf{3} & \cellcolor{blue!5} \textbf{3} & \cellcolor{blue!5} \textbf{3} & \cellcolor{blue!5} \textbf{3} \\
20 371& 3 & \cellcolor{blue!5} \textbf{3} & \cellcolor{blue!5} \textbf{3} & \cellcolor{blue!5} \textbf{3} & \cellcolor{blue!5} \textbf{3} & \cellcolor{blue!5} \textbf{3} & \cellcolor{blue!5} \textbf{3} & \cellcolor{blue!5} \textbf{3} \\
20 372& 3 & \cellcolor{blue!5} \textbf{3} & \cellcolor{blue!5} \textbf{3} & \cellcolor{blue!5} \textbf{3} & \cellcolor{blue!5} \textbf{3} & \cellcolor{blue!5} \textbf{3} & \cellcolor{blue!5} \textbf{3} & \cellcolor{blue!5} \textbf{3} \\
20 373& 3 & \cellcolor{blue!5} \textbf{3} & \cellcolor{blue!5} \textbf{3} & \cellcolor{blue!5} \textbf{3} & \cellcolor{blue!5} \textbf{3} & \cellcolor{blue!5} \textbf{3} & \cellcolor{blue!5} \textbf{3} & \cellcolor{blue!5} \textbf{3} \\
20 374& 3 & \cellcolor{blue!5} \textbf{3} & \cellcolor{blue!5} \textbf{3} & \cellcolor{blue!5} \textbf{3} & \cellcolor{blue!5} \textbf{3} & \cellcolor{blue!5} \textbf{3} & \cellcolor{blue!5} \textbf{3} & \cellcolor{blue!5} \textbf{3} \\
20 375& 3 & \cellcolor{blue!5} \textbf{3} & \cellcolor{blue!5} \textbf{3} & \cellcolor{blue!5} \textbf{3} & \cellcolor{blue!5} \textbf{3} & \cellcolor{blue!5} \textbf{3} & \cellcolor{blue!5} \textbf{3} & \cellcolor{blue!5} \textbf{3} \\
20 376& 3 & \cellcolor{blue!5} \textbf{3} & \cellcolor{blue!5} \textbf{3} & \cellcolor{blue!5} \textbf{3} & \cellcolor{blue!5} \textbf{3} & \cellcolor{blue!5} \textbf{3} & \cellcolor{blue!5} \textbf{3} & \cellcolor{blue!5} \textbf{3} \\
20 377& 3 & \cellcolor{blue!5} \textbf{3} & \cellcolor{blue!5} \textbf{3} & \cellcolor{blue!5} \textbf{3} & \cellcolor{blue!5} \textbf{3} & \cellcolor{blue!5} \textbf{3} & \cellcolor{blue!5} \textbf{3} & \cellcolor{blue!5} \textbf{3} \\
20 378& 3 & \cellcolor{blue!5} \textbf{3} & \cellcolor{blue!5} \textbf{3} & \cellcolor{blue!5} \textbf{3} & \cellcolor{blue!5} \textbf{3} & \cellcolor{blue!5} \textbf{3} & \cellcolor{blue!5} \textbf{3} & \cellcolor{blue!5} \textbf{3} \\
20 379& 4 & \cellcolor{blue!5} \textbf{4} & \cellcolor{blue!5} \textbf{4} & \cellcolor{blue!5} \textbf{4} & \cellcolor{blue!5} \textbf{4} & \cellcolor{blue!5} \textbf{4} & \cellcolor{blue!5} \textbf{4} & \cellcolor{blue!5} \textbf{4} \\
20 38& 12 & \cellcolor{blue!5} \textbf{12} & \cellcolor{blue!5} \textbf{12} & \cellcolor{blue!5} \textbf{12} & \cellcolor{blue!5} \textbf{12} & \cellcolor{blue!5} \textbf{12} & \cellcolor{blue!5} \textbf{12} & \cellcolor{blue!5} \textbf{12} \\
20 380& 3 & \cellcolor{blue!5} \textbf{3} & \cellcolor{blue!5} \textbf{3} & \cellcolor{blue!5} \textbf{3} & \cellcolor{blue!5} \textbf{3} & \cellcolor{blue!5} \textbf{3} & \cellcolor{blue!5} \textbf{3} & \cellcolor{blue!5} \textbf{3} \\
20 381& 3 & \cellcolor{blue!5} \textbf{3} & \cellcolor{blue!5} \textbf{3} & \cellcolor{blue!5} \textbf{3} & \cellcolor{blue!5} \textbf{3} & \cellcolor{blue!5} \textbf{3} & \cellcolor{blue!5} \textbf{3} & \cellcolor{blue!5} \textbf{3} \\
20 382& 4 & \cellcolor{blue!5} \textbf{4} & \cellcolor{blue!5} \textbf{4} & \cellcolor{blue!5} \textbf{4} & \cellcolor{blue!5} \textbf{4} & \cellcolor{blue!5} \textbf{4} & \cellcolor{blue!5} \textbf{4} & \cellcolor{blue!5} \textbf{4} \\
20 383& 3 & \cellcolor{blue!5} \textbf{3} & \cellcolor{blue!5} \textbf{3} & \cellcolor{blue!5} \textbf{3} & \cellcolor{blue!5} \textbf{3} & \cellcolor{blue!5} \textbf{3} & \cellcolor{blue!5} \textbf{3} & \cellcolor{blue!5} \textbf{3} \\
20 384& 3 & \cellcolor{blue!5} \textbf{3} & \cellcolor{blue!5} \textbf{3} & \cellcolor{blue!5} \textbf{3} & \cellcolor{blue!5} \textbf{3} & \cellcolor{blue!5} \textbf{3} & \cellcolor{blue!5} \textbf{3} & \cellcolor{blue!5} \textbf{3} \\
20 385& 3 & \cellcolor{blue!5} \textbf{3} & \cellcolor{blue!5} \textbf{3} & \cellcolor{blue!5} \textbf{3} & \cellcolor{blue!5} \textbf{3} & \cellcolor{blue!5} \textbf{3} & \cellcolor{blue!5} \textbf{3} & \cellcolor{blue!5} \textbf{3} \\
20 386& 3 & \cellcolor{blue!5} \textbf{3} & \cellcolor{blue!5} \textbf{3} & \cellcolor{blue!5} \textbf{3} & \cellcolor{blue!5} \textbf{3} & \cellcolor{blue!5} \textbf{3} & \cellcolor{blue!5} \textbf{3} & \cellcolor{blue!5} \textbf{3} \\
20 387& 3 & \cellcolor{blue!5} \textbf{3} & \cellcolor{blue!5} \textbf{3} & \cellcolor{blue!5} \textbf{3} & \cellcolor{blue!5} \textbf{3} & \cellcolor{blue!5} \textbf{3} & \cellcolor{blue!5} \textbf{3} & \cellcolor{blue!5} \textbf{3} \\
20 388& 3 & \cellcolor{blue!5} \textbf{3} & \cellcolor{blue!5} \textbf{3} & \cellcolor{blue!5} \textbf{3} & \cellcolor{blue!5} \textbf{3} & \cellcolor{blue!5} \textbf{3} & \cellcolor{blue!5} \textbf{3} & \cellcolor{blue!5} \textbf{3} \\
20 389& 3 & \cellcolor{blue!5} \textbf{3} & \cellcolor{blue!5} \textbf{3} & \cellcolor{blue!5} \textbf{3} & \cellcolor{blue!5} \textbf{3} & \cellcolor{blue!5} \textbf{3} & \cellcolor{blue!5} \textbf{3} & \cellcolor{blue!5} \textbf{3} \\
20 39& 13 & \cellcolor{blue!5} \textbf{13} & \cellcolor{blue!5} \textbf{13} & \cellcolor{blue!5} \textbf{13} & \cellcolor{blue!5} \textbf{13} & \cellcolor{blue!5} \textbf{13} & \cellcolor{blue!5} \textbf{13} & \cellcolor{blue!5} 13 \\
20 390& 3 & \cellcolor{blue!5} \textbf{3} & \cellcolor{blue!5} \textbf{3} & \cellcolor{blue!5} \textbf{3} & \cellcolor{blue!5} \textbf{3} & \cellcolor{blue!5} \textbf{3} & \cellcolor{blue!5} \textbf{3} & \cellcolor{blue!5} \textbf{3} \\
20 391& 11 & \cellcolor{blue!5} \textbf{11} & \cellcolor{blue!5} \textbf{11} & \cellcolor{blue!5} \textbf{11} & \cellcolor{blue!5} \textbf{11} & \cellcolor{blue!5} \textbf{11} & \cellcolor{blue!5} \textbf{11} & \cellcolor{blue!5} \textbf{11} \\
20 392& 14 & \cellcolor{blue!5} \textbf{14} & \cellcolor{blue!5} \textbf{14} & \cellcolor{blue!5} \textbf{14} & \cellcolor{blue!5} \textbf{14} & \cellcolor{blue!5} \textbf{14} & \cellcolor{blue!5} \textbf{14} & \cellcolor{blue!5} \textbf{14} \\
20 393& 11 & \cellcolor{blue!5} \textbf{11} & \cellcolor{blue!5} \textbf{11} & \cellcolor{blue!5} \textbf{11} & \cellcolor{blue!5} \textbf{11} & \cellcolor{blue!5} \textbf{11} & \cellcolor{blue!5} \textbf{11} & \cellcolor{blue!5} \textbf{11} \\
20 394& 12 & \cellcolor{blue!5} \textbf{12} & \cellcolor{blue!5} \textbf{12} & \cellcolor{blue!5} \textbf{12} & \cellcolor{blue!5} \textbf{12} & \cellcolor{blue!5} \textbf{12} & \cellcolor{blue!5} \textbf{12} & \cellcolor{blue!5} \textbf{12} \\
20 395& 12 & \cellcolor{blue!5} \textbf{12} & \cellcolor{blue!5} \textbf{12} & \cellcolor{blue!5} \textbf{12} & \cellcolor{blue!5} \textbf{12} & \cellcolor{blue!5} \textbf{12} & \cellcolor{blue!5} \textbf{12} & \cellcolor{blue!5} \textbf{12} \\
20 396& 13 & \cellcolor{blue!5} \textbf{13} & \cellcolor{blue!5} \textbf{13} & \cellcolor{blue!5} \textbf{13} & \cellcolor{blue!5} \textbf{13} & \cellcolor{blue!5} \textbf{13} & \cellcolor{blue!5} \textbf{13} & \cellcolor{blue!5} \textbf{13} \\
20 397& 10 & \cellcolor{blue!5} \textbf{10} & \cellcolor{blue!5} \textbf{10} & \cellcolor{blue!5} \textbf{10} & \cellcolor{blue!5} \textbf{10} & \cellcolor{blue!5} \textbf{10} & \cellcolor{blue!5} \textbf{10} & \cellcolor{blue!5} \textbf{10} \\
20 398& 11 & \cellcolor{blue!5} \textbf{11} & \cellcolor{blue!5} \textbf{11} & \cellcolor{blue!5} \textbf{11} & \cellcolor{blue!5} \textbf{11} & \cellcolor{blue!5} \textbf{11} & \cellcolor{blue!5} \textbf{11} & \cellcolor{blue!5} \textbf{11} \\
20 399& 13 & \cellcolor{blue!5} \textbf{13} & \cellcolor{blue!5} \textbf{13} & \cellcolor{blue!5} \textbf{13} & \cellcolor{blue!5} \textbf{13} & \cellcolor{blue!5} \textbf{13} & \cellcolor{blue!5} \textbf{13} & \cellcolor{blue!5} \textbf{13} \\
20 4& 3 & \cellcolor{blue!5} \textbf{3} & \cellcolor{blue!5} \textbf{3} & \cellcolor{blue!5} \textbf{3} & \cellcolor{blue!5} \textbf{3} & \cellcolor{blue!5} \textbf{3} & \cellcolor{blue!5} \textbf{3} & \cellcolor{blue!5} \textbf{3} \\
20 40& 12 & \cellcolor{blue!5} \textbf{12} & \cellcolor{blue!5} \textbf{12} & \cellcolor{blue!5} \textbf{12} & \cellcolor{blue!5} \textbf{12} & \cellcolor{blue!5} \textbf{12} & \cellcolor{blue!5} \textbf{12} & \cellcolor{blue!5} \textbf{12} \\
20 400& 12 & \cellcolor{blue!5} \textbf{12} & \cellcolor{blue!5} \textbf{12} & \cellcolor{blue!5} \textbf{12} & \cellcolor{blue!5} \textbf{12} & \cellcolor{blue!5} \textbf{12} & \cellcolor{blue!5} \textbf{12} & \cellcolor{blue!5} \textbf{12} \\
20 401& 12 & \cellcolor{blue!5} \textbf{12} & \cellcolor{blue!5} \textbf{12} & \cellcolor{blue!5} \textbf{12} & \cellcolor{blue!5} \textbf{12} & \cellcolor{blue!5} \textbf{12} & \cellcolor{blue!5} \textbf{12} & \cellcolor{blue!5} \textbf{12} \\
20 402& 12 & \cellcolor{blue!5} \textbf{12} & \cellcolor{blue!5} \textbf{12} & \cellcolor{blue!5} \textbf{12} & \cellcolor{blue!5} \textbf{12} & \cellcolor{blue!5} \textbf{12} & \cellcolor{blue!5} \textbf{12} & \cellcolor{blue!5} \textbf{12} \\
20 403& 12 & \cellcolor{blue!5} \textbf{12} & \cellcolor{blue!5} \textbf{12} & \cellcolor{blue!5} \textbf{12} & \cellcolor{blue!5} \textbf{12} & \cellcolor{blue!5} \textbf{12} & \cellcolor{blue!5} \textbf{12} & \cellcolor{blue!5} \textbf{12} \\
20 404& 10 & \cellcolor{blue!5} \textbf{10} & \cellcolor{blue!5} \textbf{10} & \cellcolor{blue!5} \textbf{10} & \cellcolor{blue!5} \textbf{10} & \cellcolor{blue!5} \textbf{10} & \cellcolor{blue!5} \textbf{10} & \cellcolor{blue!5} \textbf{10} \\
20 405& 12 & \cellcolor{blue!5} \textbf{12} & \cellcolor{blue!5} \textbf{12} & \cellcolor{blue!5} \textbf{12} & \cellcolor{blue!5} \textbf{12} & \cellcolor{blue!5} \textbf{12} & \cellcolor{blue!5} \textbf{12} & \cellcolor{blue!5} \textbf{12} \\
20 406& 14 & \cellcolor{blue!5} \textbf{14} & \cellcolor{blue!5} \textbf{14} & \cellcolor{blue!5} \textbf{14} & \cellcolor{blue!5} \textbf{14} & \cellcolor{blue!5} \textbf{14} & \cellcolor{blue!5} \textbf{14} & \cellcolor{blue!5} \textbf{14} \\
20 407& 10 & \cellcolor{blue!5} \textbf{10} & \cellcolor{blue!5} \textbf{10} & \cellcolor{blue!5} \textbf{10} & \cellcolor{blue!5} \textbf{10} & \cellcolor{blue!5} \textbf{10} & \cellcolor{blue!5} \textbf{10} & \cellcolor{blue!5} \textbf{10} \\
20 408& 14 & \cellcolor{blue!5} \textbf{14} & \cellcolor{blue!5} \textbf{14} & \cellcolor{blue!5} \textbf{14} & \cellcolor{blue!5} \textbf{14} & \cellcolor{blue!5} \textbf{14} & \cellcolor{blue!5} \textbf{14} & \cellcolor{blue!5} \textbf{14} \\
20 409& 12 & \cellcolor{blue!5} \textbf{12} & \cellcolor{blue!5} \textbf{12} & \cellcolor{blue!5} \textbf{12} & \cellcolor{blue!5} \textbf{12} & \cellcolor{blue!5} \textbf{12} & \cellcolor{blue!5} \textbf{12} & \cellcolor{blue!5} \textbf{12} \\
20 41& 6 & \cellcolor{blue!5} \textbf{6} & \cellcolor{blue!5} \textbf{6} & \cellcolor{blue!5} \textbf{6} & \cellcolor{blue!5} \textbf{6} & \cellcolor{blue!5} \textbf{6} & \cellcolor{blue!5} \textbf{6} & \cellcolor{blue!5} \textbf{6} \\
20 410& 11 & \cellcolor{blue!5} \textbf{11} & \cellcolor{blue!5} \textbf{11} & \cellcolor{blue!5} \textbf{11} & \cellcolor{blue!5} \textbf{11} & \cellcolor{blue!5} \textbf{11} & \cellcolor{blue!5} \textbf{11} & \cellcolor{blue!5} \textbf{11} \\
20 411& 15 & \cellcolor{blue!5} \textbf{15} & \cellcolor{blue!5} \textbf{15} & \cellcolor{blue!5} \textbf{15} & \cellcolor{blue!5} \textbf{15} & \cellcolor{blue!5} \textbf{15} & \cellcolor{blue!5} \textbf{15} & \cellcolor{blue!5} \textbf{15} \\
20 412& 11 & \cellcolor{blue!5} \textbf{11} & \cellcolor{blue!5} \textbf{11} & \cellcolor{blue!5} \textbf{11} & \cellcolor{blue!5} \textbf{11} & \cellcolor{blue!5} \textbf{11} & \cellcolor{blue!5} \textbf{11} & \cellcolor{blue!5} \textbf{11} \\
20 413& 10 & \cellcolor{blue!5} \textbf{10} & \cellcolor{blue!5} \textbf{10} & \cellcolor{blue!5} \textbf{10} & \cellcolor{blue!5} \textbf{10} & \cellcolor{blue!5} \textbf{10} & \cellcolor{blue!5} \textbf{10} & \cellcolor{blue!5} \textbf{10} \\
20 414& 12 & \cellcolor{blue!5} \textbf{12} & \cellcolor{blue!5} \textbf{12} & \cellcolor{blue!5} \textbf{12} & \cellcolor{blue!5} \textbf{12} & \cellcolor{blue!5} \textbf{12} & \cellcolor{blue!5} \textbf{12} & \cellcolor{blue!5} \textbf{12} \\
20 415& 10 & \cellcolor{blue!5} \textbf{10} & \cellcolor{blue!5} \textbf{10} & \cellcolor{blue!5} \textbf{10} & \cellcolor{blue!5} \textbf{10} & \cellcolor{blue!5} \textbf{10} & \cellcolor{blue!5} \textbf{10} & \cellcolor{blue!5} \textbf{10} \\
20 416& 6 & \cellcolor{blue!5} \textbf{6} & \cellcolor{blue!5} \textbf{6} & \cellcolor{blue!5} \textbf{6} & \cellcolor{blue!5} \textbf{6} & \cellcolor{blue!5} \textbf{6} & \cellcolor{blue!5} \textbf{6} & \cellcolor{blue!5} \textbf{6} \\
20 417& 5 & \cellcolor{blue!5} \textbf{5} & \cellcolor{blue!5} \textbf{5} & \cellcolor{blue!5} \textbf{5} & \cellcolor{blue!5} \textbf{5} & \cellcolor{blue!5} \textbf{5} & \cellcolor{blue!5} \textbf{5} & \cellcolor{blue!5} \textbf{5} \\
20 418& 6 & \cellcolor{blue!5} \textbf{6} & \cellcolor{blue!5} \textbf{6} & \cellcolor{blue!5} \textbf{6} & \cellcolor{blue!5} \textbf{6} & \cellcolor{blue!5} \textbf{6} & \cellcolor{blue!5} \textbf{6} & \cellcolor{blue!5} \textbf{6} \\
20 419& 4 & \cellcolor{blue!5} \textbf{4} & \cellcolor{blue!5} \textbf{4} & \cellcolor{blue!5} \textbf{4} & \cellcolor{blue!5} \textbf{4} & \cellcolor{blue!5} \textbf{4} & \cellcolor{blue!5} \textbf{4} & \cellcolor{blue!5} \textbf{4} \\
20 42& 5 & \cellcolor{blue!5} \textbf{5} & \cellcolor{blue!5} \textbf{5} & \cellcolor{blue!5} \textbf{5} & \cellcolor{blue!5} \textbf{5} & \cellcolor{blue!5} \textbf{5} & \cellcolor{blue!5} \textbf{5} & \cellcolor{blue!5} \textbf{5} \\
20 420& 5 & \cellcolor{blue!5} \textbf{5} & \cellcolor{blue!5} \textbf{5} & \cellcolor{blue!5} \textbf{5} & \cellcolor{blue!5} \textbf{5} & \cellcolor{blue!5} \textbf{5} & \cellcolor{blue!5} \textbf{5} & \cellcolor{blue!5} \textbf{5} \\
20 421& 6 & \cellcolor{blue!5} \textbf{6} & \cellcolor{blue!5} \textbf{6} & \cellcolor{blue!5} \textbf{6} & \cellcolor{blue!5} \textbf{6} & \cellcolor{blue!5} \textbf{6} & \cellcolor{blue!5} \textbf{6} & \cellcolor{blue!5} \textbf{6} \\
20 422& 4 & \cellcolor{blue!5} \textbf{4} & \cellcolor{blue!5} \textbf{4} & \cellcolor{blue!5} \textbf{4} & \cellcolor{blue!5} \textbf{4} & \cellcolor{blue!5} \textbf{4} & \cellcolor{blue!5} \textbf{4} & \cellcolor{blue!5} \textbf{4} \\
20 423& 6 & \cellcolor{blue!5} \textbf{6} & \cellcolor{blue!5} \textbf{6} & \cellcolor{blue!5} \textbf{6} & \cellcolor{blue!5} \textbf{6} & \cellcolor{blue!5} \textbf{6} & \cellcolor{blue!5} \textbf{6} & \cellcolor{blue!5} \textbf{6} \\
20 424& 5 & \cellcolor{blue!5} \textbf{5} & \cellcolor{blue!5} \textbf{5} & \cellcolor{blue!5} \textbf{5} & \cellcolor{blue!5} \textbf{5} & \cellcolor{blue!5} \textbf{5} & \cellcolor{blue!5} \textbf{5} & \cellcolor{blue!5} \textbf{5} \\
20 425& 6 & \cellcolor{blue!5} \textbf{6} & \cellcolor{blue!5} \textbf{6} & \cellcolor{blue!5} \textbf{6} & \cellcolor{blue!5} \textbf{6} & \cellcolor{blue!5} \textbf{6} & \cellcolor{blue!5} \textbf{6} & \cellcolor{blue!5} \textbf{6} \\
20 426& 5 & \cellcolor{blue!5} \textbf{5} & \cellcolor{blue!5} \textbf{5} & \cellcolor{blue!5} \textbf{5} & \cellcolor{blue!5} \textbf{5} & \cellcolor{blue!5} \textbf{5} & \cellcolor{blue!5} \textbf{5} & \cellcolor{blue!5} \textbf{5} \\
20 427& 6 & \cellcolor{blue!5} \textbf{6} & \cellcolor{blue!5} \textbf{6} & \cellcolor{blue!5} \textbf{6} & \cellcolor{blue!5} \textbf{6} & \cellcolor{blue!5} \textbf{6} & \cellcolor{blue!5} \textbf{6} & \cellcolor{blue!5} \textbf{6} \\
20 428& 5 & \cellcolor{blue!5} \textbf{5} & \cellcolor{blue!5} \textbf{5} & \cellcolor{blue!5} \textbf{5} & \cellcolor{blue!5} \textbf{5} & \cellcolor{blue!5} \textbf{5} & \cellcolor{blue!5} \textbf{5} & \cellcolor{blue!5} \textbf{5} \\
20 429& 4 & \cellcolor{blue!5} \textbf{4} & \cellcolor{blue!5} \textbf{4} & \cellcolor{blue!5} \textbf{4} & \cellcolor{blue!5} \textbf{4} & \cellcolor{blue!5} \textbf{4} & \cellcolor{blue!5} \textbf{4} & \cellcolor{blue!5} \textbf{4} \\
20 43& 5 & \cellcolor{blue!5} \textbf{5} & \cellcolor{blue!5} \textbf{5} & \cellcolor{blue!5} \textbf{5} & \cellcolor{blue!5} \textbf{5} & \cellcolor{blue!5} \textbf{5} & \cellcolor{blue!5} \textbf{5} & \cellcolor{blue!5} \textbf{5} \\
20 430& 5 & \cellcolor{blue!5} \textbf{5} & \cellcolor{blue!5} \textbf{5} & \cellcolor{blue!5} \textbf{5} & \cellcolor{blue!5} \textbf{5} & \cellcolor{blue!5} \textbf{5} & \cellcolor{blue!5} \textbf{5} & \cellcolor{blue!5} \textbf{5} \\
20 431& 6 & \cellcolor{blue!5} \textbf{6} & \cellcolor{blue!5} \textbf{6} & \cellcolor{blue!5} \textbf{6} & \cellcolor{blue!5} \textbf{6} & \cellcolor{blue!5} \textbf{6} & \cellcolor{blue!5} \textbf{6} & \cellcolor{blue!5} \textbf{6} \\
20 432& 5 & \cellcolor{blue!5} \textbf{5} & \cellcolor{blue!5} \textbf{5} & \cellcolor{blue!5} \textbf{5} & \cellcolor{blue!5} \textbf{5} & \cellcolor{blue!5} \textbf{5} & \cellcolor{blue!5} \textbf{5} & \cellcolor{blue!5} \textbf{5} \\
20 433& 5 & \cellcolor{blue!5} \textbf{5} & \cellcolor{blue!5} \textbf{5} & \cellcolor{blue!5} \textbf{5} & \cellcolor{blue!5} \textbf{5} & \cellcolor{blue!5} \textbf{5} & \cellcolor{blue!5} \textbf{5} & \cellcolor{blue!5} \textbf{5} \\
20 434& 5 & \cellcolor{blue!5} \textbf{5} & \cellcolor{blue!5} \textbf{5} & \cellcolor{blue!5} \textbf{5} & \cellcolor{blue!5} \textbf{5} & \cellcolor{blue!5} \textbf{5} & \cellcolor{blue!5} \textbf{5} & \cellcolor{blue!5} \textbf{5} \\
20 435& 7 & \cellcolor{blue!5} \textbf{7} & \cellcolor{blue!5} \textbf{7} & \cellcolor{blue!5} \textbf{7} & \cellcolor{blue!5} \textbf{7} & \cellcolor{blue!5} \textbf{7} & \cellcolor{blue!5} \textbf{7} & \cellcolor{blue!5} \textbf{7} \\
20 436& 5 & \cellcolor{blue!5} \textbf{5} & \cellcolor{blue!5} \textbf{5} & \cellcolor{blue!5} \textbf{5} & \cellcolor{blue!5} \textbf{5} & \cellcolor{blue!5} \textbf{5} & \cellcolor{blue!5} \textbf{5} & \cellcolor{blue!5} \textbf{5} \\
20 437& 5 & \cellcolor{blue!5} \textbf{5} & \cellcolor{blue!5} \textbf{5} & \cellcolor{blue!5} \textbf{5} & \cellcolor{blue!5} \textbf{5} & \cellcolor{blue!5} \textbf{5} & \cellcolor{blue!5} \textbf{5} & \cellcolor{blue!5} \textbf{5} \\
20 438& 6 & \cellcolor{blue!5} \textbf{6} & \cellcolor{blue!5} \textbf{6} & \cellcolor{blue!5} \textbf{6} & \cellcolor{blue!5} \textbf{6} & \cellcolor{blue!5} \textbf{6} & \cellcolor{blue!5} \textbf{6} & \cellcolor{blue!5} \textbf{6} \\
20 439& 5 & \cellcolor{blue!5} \textbf{5} & \cellcolor{blue!5} \textbf{5} & \cellcolor{blue!5} \textbf{5} & \cellcolor{blue!5} \textbf{5} & \cellcolor{blue!5} \textbf{5} & \cellcolor{blue!5} \textbf{5} & \cellcolor{blue!5} \textbf{5} \\
20 44& 5 & \cellcolor{blue!5} \textbf{5} & \cellcolor{blue!5} \textbf{5} & \cellcolor{blue!5} \textbf{5} & \cellcolor{blue!5} \textbf{5} & \cellcolor{blue!5} \textbf{5} & \cellcolor{blue!5} \textbf{5} & \cellcolor{blue!5} \textbf{5} \\
20 440& 5 & \cellcolor{blue!5} \textbf{5} & \cellcolor{blue!5} \textbf{5} & \cellcolor{blue!5} \textbf{5} & \cellcolor{blue!5} \textbf{5} & \cellcolor{blue!5} \textbf{5} & \cellcolor{blue!5} \textbf{5} & \cellcolor{blue!5} \textbf{5} \\
20 441& 3 & \cellcolor{blue!5} \textbf{3} & \cellcolor{blue!5} \textbf{3} & \cellcolor{blue!5} \textbf{3} & \cellcolor{blue!5} \textbf{3} & \cellcolor{blue!5} \textbf{3} & \cellcolor{blue!5} \textbf{3} & \cellcolor{blue!5} \textbf{3} \\
20 442& 3 & \cellcolor{blue!5} \textbf{3} & \cellcolor{blue!5} \textbf{3} & \cellcolor{blue!5} \textbf{3} & \cellcolor{blue!5} \textbf{3} & \cellcolor{blue!5} \textbf{3} & \cellcolor{blue!5} \textbf{3} & \cellcolor{blue!5} \textbf{3} \\
20 443& 3 & \cellcolor{blue!5} \textbf{3} & \cellcolor{blue!5} \textbf{3} & \cellcolor{blue!5} \textbf{3} & \cellcolor{blue!5} \textbf{3} & \cellcolor{blue!5} \textbf{3} & \cellcolor{blue!5} \textbf{3} & \cellcolor{blue!5} \textbf{3} \\
20 444& 3 & \cellcolor{blue!5} \textbf{3} & \cellcolor{blue!5} \textbf{3} & \cellcolor{blue!5} \textbf{3} & \cellcolor{blue!5} \textbf{3} & \cellcolor{blue!5} \textbf{3} & \cellcolor{blue!5} \textbf{3} & \cellcolor{blue!5} \textbf{3} \\
20 445& 3 & \cellcolor{blue!5} \textbf{3} & \cellcolor{blue!5} \textbf{3} & \cellcolor{blue!5} \textbf{3} & \cellcolor{blue!5} \textbf{3} & \cellcolor{blue!5} \textbf{3} & \cellcolor{blue!5} \textbf{3} & \cellcolor{blue!5} \textbf{3} \\
20 446& 3 & \cellcolor{blue!5} \textbf{3} & \cellcolor{blue!5} \textbf{3} & \cellcolor{blue!5} \textbf{3} & \cellcolor{blue!5} \textbf{3} & \cellcolor{blue!5} \textbf{3} & \cellcolor{blue!5} \textbf{3} & \cellcolor{blue!5} \textbf{3} \\
20 447& 3 & \cellcolor{blue!5} \textbf{3} & \cellcolor{blue!5} \textbf{3} & \cellcolor{blue!5} \textbf{3} & \cellcolor{blue!5} \textbf{3} & \cellcolor{blue!5} \textbf{3} & \cellcolor{blue!5} \textbf{3} & \cellcolor{blue!5} \textbf{3} \\
20 448& 3 & \cellcolor{blue!5} \textbf{3} & \cellcolor{blue!5} \textbf{3} & \cellcolor{blue!5} \textbf{3} & \cellcolor{blue!5} \textbf{3} & \cellcolor{blue!5} \textbf{3} & \cellcolor{blue!5} \textbf{3} & \cellcolor{blue!5} \textbf{3} \\
20 449& 3 & \cellcolor{blue!5} \textbf{3} & \cellcolor{blue!5} \textbf{3} & \cellcolor{blue!5} \textbf{3} & \cellcolor{blue!5} \textbf{3} & \cellcolor{blue!5} \textbf{3} & \cellcolor{blue!5} \textbf{3} & \cellcolor{blue!5} \textbf{3} \\
20 45& 6 & \cellcolor{blue!5} \textbf{6} & \cellcolor{blue!5} \textbf{6} & \cellcolor{blue!5} \textbf{6} & \cellcolor{blue!5} \textbf{6} & \cellcolor{blue!5} \textbf{6} & \cellcolor{blue!5} \textbf{6} & \cellcolor{blue!5} \textbf{6} \\
20 450& 3 & \cellcolor{blue!5} \textbf{3} & \cellcolor{blue!5} \textbf{3} & \cellcolor{blue!5} \textbf{3} & \cellcolor{blue!5} \textbf{3} & \cellcolor{blue!5} \textbf{3} & \cellcolor{blue!5} \textbf{3} & \cellcolor{blue!5} \textbf{3} \\
20 451& 3 & \cellcolor{blue!5} \textbf{3} & \cellcolor{blue!5} \textbf{3} & \cellcolor{blue!5} \textbf{3} & \cellcolor{blue!5} \textbf{3} & \cellcolor{blue!5} \textbf{3} & \cellcolor{blue!5} \textbf{3} & \cellcolor{blue!5} \textbf{3} \\
20 452& 3 & \cellcolor{blue!5} \textbf{3} & \cellcolor{blue!5} \textbf{3} & \cellcolor{blue!5} \textbf{3} & \cellcolor{blue!5} \textbf{3} & \cellcolor{blue!5} \textbf{3} & \cellcolor{blue!5} \textbf{3} & \cellcolor{blue!5} \textbf{3} \\
20 453& 3 & \cellcolor{blue!5} \textbf{3} & \cellcolor{blue!5} \textbf{3} & \cellcolor{blue!5} \textbf{3} & \cellcolor{blue!5} \textbf{3} & \cellcolor{blue!5} \textbf{3} & \cellcolor{blue!5} \textbf{3} & \cellcolor{blue!5} \textbf{3} \\
20 454& 3 & \cellcolor{blue!5} \textbf{3} & \cellcolor{blue!5} \textbf{3} & \cellcolor{blue!5} \textbf{3} & \cellcolor{blue!5} \textbf{3} & \cellcolor{blue!5} \textbf{3} & \cellcolor{blue!5} \textbf{3} & \cellcolor{blue!5} \textbf{3} \\
20 455& 3 & \cellcolor{blue!5} \textbf{3} & \cellcolor{blue!5} \textbf{3} & \cellcolor{blue!5} \textbf{3} & \cellcolor{blue!5} \textbf{3} & \cellcolor{blue!5} \textbf{3} & \cellcolor{blue!5} \textbf{3} & \cellcolor{blue!5} \textbf{3} \\
20 456& 4 & \cellcolor{blue!5} \textbf{4} & \cellcolor{blue!5} \textbf{4} & \cellcolor{blue!5} \textbf{4} & \cellcolor{blue!5} \textbf{4} & \cellcolor{blue!5} \textbf{4} & \cellcolor{blue!5} \textbf{4} & \cellcolor{blue!5} \textbf{4} \\
20 457& 3 & \cellcolor{blue!5} \textbf{3} & \cellcolor{blue!5} \textbf{3} & \cellcolor{blue!5} \textbf{3} & \cellcolor{blue!5} \textbf{3} & \cellcolor{blue!5} \textbf{3} & \cellcolor{blue!5} \textbf{3} & \cellcolor{blue!5} \textbf{3} \\
20 458& 3 & \cellcolor{blue!5} \textbf{3} & \cellcolor{blue!5} \textbf{3} & \cellcolor{blue!5} \textbf{3} & \cellcolor{blue!5} \textbf{3} & \cellcolor{blue!5} \textbf{3} & \cellcolor{blue!5} \textbf{3} & \cellcolor{blue!5} \textbf{3} \\
20 459& 3 & \cellcolor{blue!5} \textbf{3} & \cellcolor{blue!5} \textbf{3} & \cellcolor{blue!5} \textbf{3} & \cellcolor{blue!5} \textbf{3} & \cellcolor{blue!5} \textbf{3} & \cellcolor{blue!5} \textbf{3} & \cellcolor{blue!5} \textbf{3} \\
20 46& 4 & \cellcolor{blue!5} \textbf{4} & \cellcolor{blue!5} \textbf{4} & \cellcolor{blue!5} \textbf{4} & \cellcolor{blue!5} \textbf{4} & \cellcolor{blue!5} \textbf{4} & \cellcolor{blue!5} \textbf{4} & \cellcolor{blue!5} \textbf{4} \\
20 460& 3 & \cellcolor{blue!5} \textbf{3} & \cellcolor{blue!5} \textbf{3} & \cellcolor{blue!5} \textbf{3} & \cellcolor{blue!5} \textbf{3} & \cellcolor{blue!5} \textbf{3} & \cellcolor{blue!5} \textbf{3} & \cellcolor{blue!5} \textbf{3} \\
20 461& 3 & \cellcolor{blue!5} \textbf{3} & \cellcolor{blue!5} \textbf{3} & \cellcolor{blue!5} \textbf{3} & \cellcolor{blue!5} \textbf{3} & \cellcolor{blue!5} \textbf{3} & \cellcolor{blue!5} \textbf{3} & \cellcolor{blue!5} \textbf{3} \\
20 462& 3 & \cellcolor{blue!5} \textbf{3} & \cellcolor{blue!5} \textbf{3} & \cellcolor{blue!5} \textbf{3} & \cellcolor{blue!5} \textbf{3} & \cellcolor{blue!5} \textbf{3} & \cellcolor{blue!5} \textbf{3} & \cellcolor{blue!5} \textbf{3} \\
20 463& 3 & \cellcolor{blue!5} \textbf{3} & \cellcolor{blue!5} \textbf{3} & \cellcolor{blue!5} \textbf{3} & \cellcolor{blue!5} \textbf{3} & \cellcolor{blue!5} \textbf{3} & \cellcolor{blue!5} \textbf{3} & \cellcolor{blue!5} \textbf{3} \\
20 464& 3 & \cellcolor{blue!5} \textbf{3} & \cellcolor{blue!5} \textbf{3} & \cellcolor{blue!5} \textbf{3} & \cellcolor{blue!5} \textbf{3} & \cellcolor{blue!5} \textbf{3} & \cellcolor{blue!5} \textbf{3} & \cellcolor{blue!5} \textbf{3} \\
20 465& 3 & \cellcolor{blue!5} \textbf{3} & \cellcolor{blue!5} \textbf{3} & \cellcolor{blue!5} \textbf{3} & \cellcolor{blue!5} \textbf{3} & \cellcolor{blue!5} \textbf{3} & \cellcolor{blue!5} \textbf{3} & \cellcolor{blue!5} \textbf{3} \\
20 466& 13 & \cellcolor{blue!5} \textbf{13} & \cellcolor{blue!5} \textbf{13} & \cellcolor{blue!5} \textbf{13} & \cellcolor{blue!5} \textbf{13} & \cellcolor{blue!5} \textbf{13} & \cellcolor{blue!5} \textbf{13} & \cellcolor{blue!5} \textbf{13} \\
20 467& 12 & \cellcolor{blue!5} 14 & \cellcolor{blue!5} \textbf{14} & \cellcolor{blue!5} \textbf{14} & \cellcolor{blue!5} \textbf{14} & \cellcolor{blue!5} \textbf{14} & \cellcolor{blue!5} \textbf{14} & \cellcolor{blue!5} \textbf{14} \\
20 468& 13 & \cellcolor{blue!5} \textbf{13} & \cellcolor{blue!5} \textbf{13} & \cellcolor{blue!5} \textbf{13} & \cellcolor{blue!5} \textbf{13} & \cellcolor{blue!5} \textbf{13} & \cellcolor{blue!5} \textbf{13} & \cellcolor{blue!5} \textbf{13} \\
20 469& 14 & \cellcolor{blue!5} \textbf{14} & \cellcolor{blue!5} \textbf{14} & \cellcolor{blue!5} \textbf{14} & \cellcolor{blue!5} \textbf{14} & \cellcolor{blue!5} \textbf{14} & \cellcolor{blue!5} \textbf{14} & \cellcolor{blue!5} \textbf{14} \\
20 47& 4 & \cellcolor{blue!5} \textbf{4} & \cellcolor{blue!5} \textbf{4} & \cellcolor{blue!5} \textbf{4} & \cellcolor{blue!5} \textbf{4} & \cellcolor{blue!5} \textbf{4} & \cellcolor{blue!5} \textbf{4} & \cellcolor{blue!5} \textbf{4} \\
20 470& 12 & \cellcolor{blue!5} \textbf{12} & \cellcolor{blue!5} \textbf{12} & \cellcolor{blue!5} \textbf{12} & \cellcolor{blue!5} \textbf{12} & \cellcolor{blue!5} \textbf{12} & \cellcolor{blue!5} \textbf{12} & \cellcolor{blue!5} \textbf{12} \\
20 471& 12 & \cellcolor{blue!5} \textbf{12} & \cellcolor{blue!5} \textbf{12} & \cellcolor{blue!5} \textbf{12} & \cellcolor{blue!5} \textbf{12} & \cellcolor{blue!5} \textbf{12} & \cellcolor{blue!5} \textbf{12} & \cellcolor{blue!5} \textbf{12} \\
20 472& 13 & \cellcolor{blue!5} \textbf{13} & \cellcolor{blue!5} \textbf{13} & \cellcolor{blue!5} \textbf{13} & \cellcolor{blue!5} \textbf{13} & \cellcolor{blue!5} \textbf{13} & \cellcolor{blue!5} \textbf{13} & \cellcolor{blue!5} \textbf{13} \\
20 473& 10 & \cellcolor{blue!5} \textbf{10} & \cellcolor{blue!5} \textbf{10} & \cellcolor{blue!5} \textbf{10} & \cellcolor{blue!5} \textbf{10} & \cellcolor{blue!5} \textbf{10} & \cellcolor{blue!5} \textbf{10} & \cellcolor{blue!5} \textbf{10} \\
20 474& 14 & \cellcolor{blue!5} \textbf{14} & \cellcolor{blue!5} \textbf{14} & \cellcolor{blue!5} \textbf{14} & \cellcolor{blue!5} \textbf{14} & \cellcolor{blue!5} \textbf{14} & \cellcolor{blue!5} \textbf{14} & \cellcolor{blue!5} \textbf{14} \\
20 475& 11 & \cellcolor{blue!5} \textbf{11} & \cellcolor{blue!5} \textbf{11} & \cellcolor{blue!5} \textbf{11} & \cellcolor{blue!5} \textbf{11} & \cellcolor{blue!5} \textbf{11} & \cellcolor{blue!5} \textbf{11} & \cellcolor{blue!5} \textbf{11} \\
20 476& 11 & \cellcolor{blue!5} \textbf{11} & \cellcolor{blue!5} \textbf{11} & \cellcolor{blue!5} \textbf{11} & \cellcolor{blue!5} \textbf{11} & \cellcolor{blue!5} \textbf{11} & \cellcolor{blue!5} \textbf{11} & \cellcolor{blue!5} \textbf{11} \\
20 477& 11 & \cellcolor{blue!5} \textbf{11} & \cellcolor{blue!5} \textbf{11} & \cellcolor{blue!5} \textbf{11} & \cellcolor{blue!5} \textbf{11} & \cellcolor{blue!5} \textbf{11} & \cellcolor{blue!5} \textbf{11} & \cellcolor{blue!5} \textbf{11} \\
20 478& 12 & \cellcolor{blue!5} \textbf{12} & \cellcolor{blue!5} \textbf{12} & \cellcolor{blue!5} \textbf{12} & \cellcolor{blue!5} \textbf{12} & \cellcolor{blue!5} \textbf{12} & \cellcolor{blue!5} \textbf{12} & \cellcolor{blue!5} \textbf{12} \\
20 479& 13 & \cellcolor{blue!5} \textbf{13} & \cellcolor{blue!5} \textbf{13} & \cellcolor{blue!5} \textbf{13} & \cellcolor{blue!5} \textbf{13} & \cellcolor{blue!5} \textbf{13} & \cellcolor{blue!5} \textbf{13} & \cellcolor{blue!5} \textbf{13} \\
20 48& 5 & \cellcolor{blue!5} \textbf{5} & \cellcolor{blue!5} \textbf{5} & \cellcolor{blue!5} \textbf{5} & \cellcolor{blue!5} \textbf{5} & \cellcolor{blue!5} \textbf{5} & \cellcolor{blue!5} \textbf{5} & \cellcolor{blue!5} \textbf{5} \\
20 480& 13 & \cellcolor{blue!5} \textbf{13} & \cellcolor{blue!5} \textbf{13} & \cellcolor{blue!5} \textbf{13} & \cellcolor{blue!5} \textbf{13} & \cellcolor{blue!5} \textbf{13} & \cellcolor{blue!5} \textbf{13} & \cellcolor{blue!5} \textbf{13} \\
20 481& 13 & \cellcolor{blue!5} \textbf{13} & \cellcolor{blue!5} \textbf{13} & \cellcolor{blue!5} \textbf{13} & \cellcolor{blue!5} \textbf{13} & \cellcolor{blue!5} \textbf{13} & \cellcolor{blue!5} \textbf{13} & \cellcolor{blue!5} \textbf{13} \\
20 482& 13 & \cellcolor{blue!5} \textbf{13} & \cellcolor{blue!5} \textbf{13} & \cellcolor{blue!5} \textbf{13} & \cellcolor{blue!5} \textbf{13} & \cellcolor{blue!5} \textbf{13} & \cellcolor{blue!5} \textbf{13} & \cellcolor{blue!5} \textbf{13} \\
20 483& 12 & \cellcolor{blue!5} \textbf{12} & \cellcolor{blue!5} \textbf{12} & \cellcolor{blue!5} \textbf{12} & \cellcolor{blue!5} \textbf{12} & \cellcolor{blue!5} \textbf{12} & \cellcolor{blue!5} \textbf{12} & \cellcolor{blue!5} \textbf{12} \\
20 484& 13 & \cellcolor{blue!5} \textbf{13} & \cellcolor{blue!5} \textbf{13} & \cellcolor{blue!5} \textbf{13} & \cellcolor{blue!5} \textbf{13} & \cellcolor{blue!5} \textbf{13} & \cellcolor{blue!5} \textbf{13} & \cellcolor{blue!5} \textbf{13} \\
20 485& 15 & \cellcolor{blue!5} \textbf{15} & \cellcolor{blue!5} \textbf{15} & \cellcolor{blue!5} \textbf{15} & \cellcolor{blue!5} \textbf{15} & \cellcolor{blue!5} \textbf{15} & \cellcolor{blue!5} \textbf{15} & \cellcolor{blue!5} \textbf{15} \\
20 486& 11 & \cellcolor{blue!5} \textbf{11} & \cellcolor{blue!5} \textbf{11} & \cellcolor{blue!5} \textbf{11} & \cellcolor{blue!5} \textbf{11} & \cellcolor{blue!5} \textbf{11} & \cellcolor{blue!5} \textbf{11} & \cellcolor{blue!5} \textbf{11} \\
20 487& 12 & \cellcolor{blue!5} \textbf{12} & \cellcolor{blue!5} \textbf{12} & \cellcolor{blue!5} \textbf{12} & \cellcolor{blue!5} \textbf{12} & \cellcolor{blue!5} \textbf{12} & \cellcolor{blue!5} \textbf{12} & \cellcolor{blue!5} \textbf{12} \\
20 488& 15 & \cellcolor{blue!5} \textbf{15} & \cellcolor{blue!5} \textbf{15} & \cellcolor{blue!5} \textbf{15} & \cellcolor{blue!5} \textbf{15} & \cellcolor{blue!5} \textbf{15} & \cellcolor{blue!5} \textbf{15} & \cellcolor{blue!5} \textbf{15} \\
20 489& 12 & \cellcolor{blue!5} \textbf{12} & \cellcolor{blue!5} \textbf{12} & \cellcolor{blue!5} \textbf{12} & \cellcolor{blue!5} \textbf{12} & \cellcolor{blue!5} \textbf{12} & \cellcolor{blue!5} \textbf{12} & \cellcolor{blue!5} \textbf{12} \\
20 49& 4 & \cellcolor{blue!5} \textbf{4} & \cellcolor{blue!5} \textbf{4} & \cellcolor{blue!5} \textbf{4} & \cellcolor{blue!5} \textbf{4} & \cellcolor{blue!5} \textbf{4} & \cellcolor{blue!5} \textbf{4} & \cellcolor{blue!5} \textbf{4} \\
20 490& 12 & \cellcolor{blue!5} \textbf{12} & \cellcolor{blue!5} \textbf{12} & \cellcolor{blue!5} \textbf{12} & \cellcolor{blue!5} \textbf{12} & \cellcolor{blue!5} \textbf{12} & \cellcolor{blue!5} \textbf{12} & \cellcolor{blue!5} \textbf{12} \\
20 491& 6 & \cellcolor{blue!5} \textbf{6} & \cellcolor{blue!5} \textbf{6} & \cellcolor{blue!5} \textbf{6} & \cellcolor{blue!5} \textbf{6} & \cellcolor{blue!5} \textbf{6} & \cellcolor{blue!5} \textbf{6} & \cellcolor{blue!5} \textbf{6} \\
20 492& 5 & \cellcolor{blue!5} \textbf{5} & \cellcolor{blue!5} \textbf{5} & \cellcolor{blue!5} \textbf{5} & \cellcolor{blue!5} \textbf{5} & \cellcolor{blue!5} \textbf{5} & \cellcolor{blue!5} \textbf{5} & \cellcolor{blue!5} \textbf{5} \\
20 493& 5 & \cellcolor{blue!5} \textbf{5} & \cellcolor{blue!5} \textbf{5} & \cellcolor{blue!5} \textbf{5} & \cellcolor{blue!5} \textbf{5} & \cellcolor{blue!5} \textbf{5} & \cellcolor{blue!5} \textbf{5} & \cellcolor{blue!5} \textbf{5} \\
20 494& 6 & \cellcolor{blue!5} \textbf{6} & \cellcolor{blue!5} \textbf{6} & \cellcolor{blue!5} \textbf{6} & \cellcolor{blue!5} \textbf{6} & \cellcolor{blue!5} \textbf{6} & \cellcolor{blue!5} \textbf{6} & \cellcolor{blue!5} \textbf{6} \\
20 495& 6 & \cellcolor{blue!5} \textbf{6} & \cellcolor{blue!5} \textbf{6} & \cellcolor{blue!5} \textbf{6} & \cellcolor{blue!5} \textbf{6} & \cellcolor{blue!5} \textbf{6} & \cellcolor{blue!5} \textbf{6} & \cellcolor{blue!5} \textbf{6} \\
20 496& 5 & \cellcolor{blue!5} \textbf{5} & \cellcolor{blue!5} \textbf{5} & \cellcolor{blue!5} \textbf{5} & \cellcolor{blue!5} \textbf{5} & \cellcolor{blue!5} \textbf{5} & \cellcolor{blue!5} \textbf{5} & \cellcolor{blue!5} \textbf{5} \\
20 497& 6 & \cellcolor{blue!5} \textbf{6} & \cellcolor{blue!5} \textbf{6} & \cellcolor{blue!5} \textbf{6} & \cellcolor{blue!5} \textbf{6} & \cellcolor{blue!5} \textbf{6} & \cellcolor{blue!5} \textbf{6} & \cellcolor{blue!5} \textbf{6} \\
20 498& 6 & \cellcolor{blue!5} \textbf{6} & \cellcolor{blue!5} \textbf{6} & \cellcolor{blue!5} \textbf{6} & \cellcolor{blue!5} \textbf{6} & \cellcolor{blue!5} \textbf{6} & \cellcolor{blue!5} \textbf{6} & \cellcolor{blue!5} \textbf{6} \\
20 499& 5 & \cellcolor{blue!5} \textbf{5} & \cellcolor{blue!5} \textbf{5} & \cellcolor{blue!5} \textbf{5} & \cellcolor{blue!5} \textbf{5} & \cellcolor{blue!5} \textbf{5} & \cellcolor{blue!5} \textbf{5} & \cellcolor{blue!5} \textbf{5} \\
20 5& 3 & \cellcolor{blue!5} \textbf{3} & \cellcolor{blue!5} \textbf{3} & \cellcolor{blue!5} \textbf{3} & \cellcolor{blue!5} \textbf{3} & \cellcolor{blue!5} \textbf{3} & \cellcolor{blue!5} \textbf{3} & \cellcolor{blue!5} \textbf{3} \\
20 50& 4 & \cellcolor{blue!5} \textbf{4} & \cellcolor{blue!5} \textbf{4} & \cellcolor{blue!5} \textbf{4} & \cellcolor{blue!5} \textbf{4} & \cellcolor{blue!5} \textbf{4} & \cellcolor{blue!5} \textbf{4} & \cellcolor{blue!5} \textbf{4} \\
20 500& 8 & \cellcolor{blue!5} \textbf{8} & \cellcolor{blue!5} \textbf{8} & \cellcolor{blue!5} \textbf{8} & \cellcolor{blue!5} \textbf{8} & \cellcolor{blue!5} \textbf{8} & \cellcolor{blue!5} \textbf{8} & \cellcolor{blue!5} \textbf{8} \\
20 501& 5 & \cellcolor{blue!5} \textbf{5} & \cellcolor{blue!5} \textbf{5} & \cellcolor{blue!5} \textbf{5} & \cellcolor{blue!5} \textbf{5} & \cellcolor{blue!5} \textbf{5} & \cellcolor{blue!5} \textbf{5} & \cellcolor{blue!5} \textbf{5} \\
20 502& 4 & \cellcolor{blue!5} \textbf{4} & \cellcolor{blue!5} \textbf{4} & \cellcolor{blue!5} \textbf{4} & \cellcolor{blue!5} \textbf{4} & \cellcolor{blue!5} \textbf{4} & \cellcolor{blue!5} \textbf{4} & \cellcolor{blue!5} \textbf{4} \\
20 503& 6 & \cellcolor{blue!5} \textbf{6} & \cellcolor{blue!5} \textbf{6} & \cellcolor{blue!5} \textbf{6} & \cellcolor{blue!5} \textbf{6} & \cellcolor{blue!5} \textbf{6} & \cellcolor{blue!5} \textbf{6} & \cellcolor{blue!5} \textbf{6} \\
20 504& 6 & \cellcolor{blue!5} \textbf{6} & \cellcolor{blue!5} \textbf{6} & \cellcolor{blue!5} \textbf{6} & \cellcolor{blue!5} \textbf{6} & \cellcolor{blue!5} \textbf{6} & \cellcolor{blue!5} \textbf{6} & \cellcolor{blue!5} \textbf{6} \\
20 505& 6 & \cellcolor{blue!5} \textbf{6} & \cellcolor{blue!5} \textbf{6} & \cellcolor{blue!5} \textbf{6} & \cellcolor{blue!5} \textbf{6} & \cellcolor{blue!5} \textbf{6} & \cellcolor{blue!5} \textbf{6} & \cellcolor{blue!5} \textbf{6} \\
20 506& 5 & \cellcolor{blue!5} \textbf{5} & \cellcolor{blue!5} \textbf{5} & \cellcolor{blue!5} \textbf{5} & \cellcolor{blue!5} \textbf{5} & \cellcolor{blue!5} \textbf{5} & \cellcolor{blue!5} \textbf{5} & \cellcolor{blue!5} \textbf{5} \\
20 507& 5 & \cellcolor{blue!5} \textbf{5} & \cellcolor{blue!5} \textbf{5} & \cellcolor{blue!5} \textbf{5} & \cellcolor{blue!5} \textbf{5} & \cellcolor{blue!5} \textbf{5} & \cellcolor{blue!5} \textbf{5} & \cellcolor{blue!5} \textbf{5} \\
20 508& 5 & \cellcolor{blue!5} \textbf{5} & \cellcolor{blue!5} \textbf{5} & \cellcolor{blue!5} \textbf{5} & \cellcolor{blue!5} \textbf{5} & \cellcolor{blue!5} \textbf{5} & \cellcolor{blue!5} \textbf{5} & \cellcolor{blue!5} \textbf{5} \\
20 509& 4 & \cellcolor{blue!5} \textbf{4} & \cellcolor{blue!5} \textbf{4} & \cellcolor{blue!5} \textbf{4} & \cellcolor{blue!5} \textbf{4} & \cellcolor{blue!5} \textbf{4} & \cellcolor{blue!5} \textbf{4} & \cellcolor{blue!5} \textbf{4} \\
20 51& 4 & \cellcolor{blue!5} \textbf{4} & \cellcolor{blue!5} \textbf{4} & \cellcolor{blue!5} \textbf{4} & \cellcolor{blue!5} \textbf{4} & \cellcolor{blue!5} \textbf{4} & \cellcolor{blue!5} \textbf{4} & \cellcolor{blue!5} \textbf{4} \\
20 510& 5 & \cellcolor{blue!5} \textbf{5} & \cellcolor{blue!5} \textbf{5} & \cellcolor{blue!5} \textbf{5} & \cellcolor{blue!5} \textbf{5} & \cellcolor{blue!5} \textbf{5} & \cellcolor{blue!5} \textbf{5} & \cellcolor{blue!5} \textbf{5} \\
20 511& 5 & \cellcolor{blue!5} \textbf{5} & \cellcolor{blue!5} \textbf{5} & \cellcolor{blue!5} \textbf{5} & \cellcolor{blue!5} \textbf{5} & \cellcolor{blue!5} \textbf{5} & \cellcolor{blue!5} \textbf{5} & \cellcolor{blue!5} \textbf{5} \\
20 512& 5 & \cellcolor{blue!5} \textbf{5} & \cellcolor{blue!5} \textbf{5} & \cellcolor{blue!5} \textbf{5} & \cellcolor{blue!5} \textbf{5} & \cellcolor{blue!5} \textbf{5} & \cellcolor{blue!5} \textbf{5} & \cellcolor{blue!5} \textbf{5} \\
20 513& 5 & \cellcolor{blue!5} \textbf{5} & \cellcolor{blue!5} \textbf{5} & \cellcolor{blue!5} \textbf{5} & \cellcolor{blue!5} \textbf{5} & \cellcolor{blue!5} \textbf{5} & \cellcolor{blue!5} \textbf{5} & \cellcolor{blue!5} \textbf{5} \\
20 514& 5 & \cellcolor{blue!5} \textbf{5} & \cellcolor{blue!5} \textbf{5} & \cellcolor{blue!5} \textbf{5} & \cellcolor{blue!5} \textbf{5} & \cellcolor{blue!5} \textbf{5} & \cellcolor{blue!5} \textbf{5} & \cellcolor{blue!5} \textbf{5} \\
20 515& 6 & \cellcolor{blue!5} \textbf{6} & \cellcolor{blue!5} \textbf{6} & \cellcolor{blue!5} \textbf{6} & \cellcolor{blue!5} \textbf{6} & \cellcolor{blue!5} \textbf{6} & \cellcolor{blue!5} \textbf{6} & \cellcolor{blue!5} \textbf{6} \\
20 516& 3 & \cellcolor{blue!5} \textbf{3} & \cellcolor{blue!5} \textbf{3} & \cellcolor{blue!5} \textbf{3} & \cellcolor{blue!5} \textbf{3} & \cellcolor{blue!5} \textbf{3} & \cellcolor{blue!5} \textbf{3} & \cellcolor{blue!5} \textbf{3} \\
20 517& 3 & \cellcolor{blue!5} \textbf{3} & \cellcolor{blue!5} \textbf{3} & \cellcolor{blue!5} \textbf{3} & \cellcolor{blue!5} \textbf{3} & \cellcolor{blue!5} \textbf{3} & \cellcolor{blue!5} \textbf{3} & \cellcolor{blue!5} \textbf{3} \\
20 518& 3 & \cellcolor{blue!5} \textbf{3} & \cellcolor{blue!5} \textbf{3} & \cellcolor{blue!5} \textbf{3} & \cellcolor{blue!5} \textbf{3} & \cellcolor{blue!5} \textbf{3} & \cellcolor{blue!5} \textbf{3} & \cellcolor{blue!5} \textbf{3} \\
20 519& 3 & \cellcolor{blue!5} \textbf{3} & \cellcolor{blue!5} \textbf{3} & \cellcolor{blue!5} \textbf{3} & \cellcolor{blue!5} \textbf{3} & \cellcolor{blue!5} \textbf{3} & \cellcolor{blue!5} \textbf{3} & \cellcolor{blue!5} \textbf{3} \\
20 52& 4 & \cellcolor{blue!5} \textbf{4} & \cellcolor{blue!5} \textbf{4} & \cellcolor{blue!5} \textbf{4} & \cellcolor{blue!5} \textbf{4} & \cellcolor{blue!5} \textbf{4} & \cellcolor{blue!5} \textbf{4} & \cellcolor{blue!5} \textbf{4} \\
20 520& 3 & \cellcolor{blue!5} \textbf{3} & \cellcolor{blue!5} \textbf{3} & \cellcolor{blue!5} \textbf{3} & \cellcolor{blue!5} \textbf{3} & \cellcolor{blue!5} \textbf{3} & \cellcolor{blue!5} \textbf{3} & \cellcolor{blue!5} \textbf{3} \\
20 521& 3 & \cellcolor{blue!5} \textbf{3} & \cellcolor{blue!5} \textbf{3} & \cellcolor{blue!5} \textbf{3} & \cellcolor{blue!5} \textbf{3} & \cellcolor{blue!5} \textbf{3} & \cellcolor{blue!5} \textbf{3} & \cellcolor{blue!5} \textbf{3} \\
20 522& 3 & \cellcolor{blue!5} \textbf{3} & \cellcolor{blue!5} \textbf{3} & \cellcolor{blue!5} \textbf{3} & \cellcolor{blue!5} \textbf{3} & \cellcolor{blue!5} \textbf{3} & \cellcolor{blue!5} \textbf{3} & \cellcolor{blue!5} \textbf{3} \\
20 523& 3 & \cellcolor{blue!5} \textbf{3} & \cellcolor{blue!5} \textbf{3} & \cellcolor{blue!5} \textbf{3} & \cellcolor{blue!5} \textbf{3} & \cellcolor{blue!5} \textbf{3} & \cellcolor{blue!5} \textbf{3} & \cellcolor{blue!5} \textbf{3} \\
20 524& 3 & \cellcolor{blue!5} \textbf{3} & \cellcolor{blue!5} \textbf{3} & \cellcolor{blue!5} \textbf{3} & \cellcolor{blue!5} \textbf{3} & \cellcolor{blue!5} \textbf{3} & \cellcolor{blue!5} \textbf{3} & \cellcolor{blue!5} \textbf{3} \\
20 525& 3 & \cellcolor{blue!5} \textbf{3} & \cellcolor{blue!5} \textbf{3} & \cellcolor{blue!5} \textbf{3} & \cellcolor{blue!5} \textbf{3} & \cellcolor{blue!5} \textbf{3} & \cellcolor{blue!5} \textbf{3} & \cellcolor{blue!5} \textbf{3} \\
20 53& 5 & \cellcolor{blue!5} \textbf{5} & \cellcolor{blue!5} \textbf{5} & \cellcolor{blue!5} \textbf{5} & \cellcolor{blue!5} \textbf{5} & \cellcolor{blue!5} \textbf{5} & \cellcolor{blue!5} \textbf{5} & \cellcolor{blue!5} \textbf{5} \\
20 54& 5 & \cellcolor{blue!5} \textbf{5} & \cellcolor{blue!5} \textbf{5} & \cellcolor{blue!5} \textbf{5} & \cellcolor{blue!5} \textbf{5} & \cellcolor{blue!5} \textbf{5} & \cellcolor{blue!5} \textbf{5} & \cellcolor{blue!5} \textbf{5} \\
20 55& 5 & \cellcolor{blue!5} \textbf{5} & \cellcolor{blue!5} \textbf{5} & \cellcolor{blue!5} \textbf{5} & \cellcolor{blue!5} \textbf{5} & \cellcolor{blue!5} \textbf{5} & \cellcolor{blue!5} \textbf{5} & \cellcolor{blue!5} \textbf{5} \\
20 56& 4 & \cellcolor{blue!5} \textbf{4} & \cellcolor{blue!5} \textbf{4} & \cellcolor{blue!5} \textbf{4} & \cellcolor{blue!5} \textbf{4} & \cellcolor{blue!5} \textbf{4} & \cellcolor{blue!5} \textbf{4} & \cellcolor{blue!5} \textbf{4} \\
20 57& 4 & \cellcolor{blue!5} \textbf{4} & \cellcolor{blue!5} \textbf{4} & \cellcolor{blue!5} \textbf{4} & \cellcolor{blue!5} \textbf{4} & \cellcolor{blue!5} \textbf{4} & \cellcolor{blue!5} \textbf{4} & \cellcolor{blue!5} \textbf{4} \\
20 58& 5 & \cellcolor{blue!5} \textbf{5} & \cellcolor{blue!5} \textbf{5} & \cellcolor{blue!5} \textbf{5} & \cellcolor{blue!5} \textbf{5} & \cellcolor{blue!5} \textbf{5} & \cellcolor{blue!5} \textbf{5} & \cellcolor{blue!5} \textbf{5} \\
20 59& 4 & \cellcolor{blue!5} \textbf{4} & \cellcolor{blue!5} \textbf{4} & \cellcolor{blue!5} \textbf{4} & \cellcolor{blue!5} \textbf{4} & \cellcolor{blue!5} \textbf{4} & \cellcolor{blue!5} \textbf{4} & \cellcolor{blue!5} \textbf{4} \\
20 6& 3 & \cellcolor{blue!5} \textbf{3} & \cellcolor{blue!5} \textbf{3} & \cellcolor{blue!5} \textbf{3} & \cellcolor{blue!5} \textbf{3} & \cellcolor{blue!5} \textbf{3} & \cellcolor{blue!5} \textbf{3} & \cellcolor{blue!5} \textbf{3} \\
20 60& 6 & \cellcolor{blue!5} \textbf{6} & \cellcolor{blue!5} \textbf{6} & \cellcolor{blue!5} \textbf{6} & \cellcolor{blue!5} \textbf{6} & \cellcolor{blue!5} \textbf{6} & \cellcolor{blue!5} \textbf{6} & \cellcolor{blue!5} \textbf{6} \\
20 61& 7 & \cellcolor{blue!5} \textbf{7} & \cellcolor{blue!5} \textbf{7} & \cellcolor{blue!5} \textbf{7} & \cellcolor{blue!5} \textbf{7} & \cellcolor{blue!5} \textbf{7} & \cellcolor{blue!5} \textbf{7} & \cellcolor{blue!5} \textbf{7} \\
20 62& 5 & \cellcolor{blue!5} \textbf{5} & \cellcolor{blue!5} \textbf{5} & \cellcolor{blue!5} \textbf{5} & \cellcolor{blue!5} \textbf{5} & \cellcolor{blue!5} \textbf{5} & \cellcolor{blue!5} \textbf{5} & \cellcolor{blue!5} \textbf{5} \\
20 63& 5 & \cellcolor{blue!5} \textbf{5} & \cellcolor{blue!5} \textbf{5} & \cellcolor{blue!5} \textbf{5} & \cellcolor{blue!5} \textbf{5} & \cellcolor{blue!5} \textbf{5} & \cellcolor{blue!5} \textbf{5} & \cellcolor{blue!5} \textbf{5} \\
20 64& 5 & \cellcolor{blue!5} \textbf{5} & \cellcolor{blue!5} \textbf{5} & \cellcolor{blue!5} \textbf{5} & \cellcolor{blue!5} \textbf{5} & \cellcolor{blue!5} \textbf{5} & \cellcolor{blue!5} \textbf{5} & \cellcolor{blue!5} \textbf{5} \\
20 65& 5 & \cellcolor{blue!5} \textbf{5} & \cellcolor{blue!5} \textbf{5} & \cellcolor{blue!5} \textbf{5} & \cellcolor{blue!5} \textbf{5} & \cellcolor{blue!5} \textbf{5} & \cellcolor{blue!5} \textbf{5} & \cellcolor{blue!5} \textbf{5} \\
20 66& 3 & \cellcolor{blue!5} \textbf{3} & \cellcolor{blue!5} \textbf{3} & \cellcolor{blue!5} \textbf{3} & \cellcolor{blue!5} \textbf{3} & \cellcolor{blue!5} \textbf{3} & \cellcolor{blue!5} \textbf{3} & \cellcolor{blue!5} \textbf{3} \\
20 67& 3 & \cellcolor{blue!5} \textbf{3} & \cellcolor{blue!5} \textbf{3} & \cellcolor{blue!5} \textbf{3} & \cellcolor{blue!5} \textbf{3} & \cellcolor{blue!5} \textbf{3} & \cellcolor{blue!5} \textbf{3} & \cellcolor{blue!5} \textbf{3} \\
20 68& 3 & \cellcolor{blue!5} \textbf{3} & \cellcolor{blue!5} \textbf{3} & \cellcolor{blue!5} \textbf{3} & \cellcolor{blue!5} \textbf{3} & \cellcolor{blue!5} \textbf{3} & \cellcolor{blue!5} \textbf{3} & \cellcolor{blue!5} \textbf{3} \\
20 69& 2 & \cellcolor{blue!5} \textbf{2} & \cellcolor{blue!5} \textbf{2} & \cellcolor{blue!5} \textbf{2} & \cellcolor{blue!5} \textbf{2} & \cellcolor{blue!5} \textbf{2} & \cellcolor{blue!5} \textbf{2} & \cellcolor{blue!5} \textbf{2} \\
20 7& 3 & \cellcolor{blue!5} \textbf{3} & \cellcolor{blue!5} \textbf{3} & \cellcolor{blue!5} \textbf{3} & \cellcolor{blue!5} \textbf{3} & \cellcolor{blue!5} \textbf{3} & \cellcolor{blue!5} \textbf{3} & \cellcolor{blue!5} \textbf{3} \\
20 70& 3 & \cellcolor{blue!5} \textbf{3} & \cellcolor{blue!5} \textbf{3} & \cellcolor{blue!5} \textbf{3} & \cellcolor{blue!5} \textbf{3} & \cellcolor{blue!5} \textbf{3} & \cellcolor{blue!5} \textbf{3} & \cellcolor{blue!5} \textbf{3} \\
20 71& 3 & \cellcolor{blue!5} \textbf{3} & \cellcolor{blue!5} \textbf{3} & \cellcolor{blue!5} \textbf{3} & \cellcolor{blue!5} \textbf{3} & \cellcolor{blue!5} \textbf{3} & \cellcolor{blue!5} \textbf{3} & \cellcolor{blue!5} \textbf{3} \\
20 72& 3 & \cellcolor{blue!5} \textbf{3} & \cellcolor{blue!5} \textbf{3} & \cellcolor{blue!5} \textbf{3} & \cellcolor{blue!5} \textbf{3} & \cellcolor{blue!5} \textbf{3} & \cellcolor{blue!5} \textbf{3} & \cellcolor{blue!5} \textbf{3} \\
20 73& 2 & \cellcolor{blue!5} \textbf{2} & \cellcolor{blue!5} \textbf{2} & \cellcolor{blue!5} \textbf{2} & \cellcolor{blue!5} \textbf{2} & \cellcolor{blue!5} \textbf{2} & \cellcolor{blue!5} \textbf{2} & \cellcolor{blue!5} \textbf{2} \\
20 74& 3 & \cellcolor{blue!5} \textbf{3} & \cellcolor{blue!5} \textbf{3} & \cellcolor{blue!5} \textbf{3} & \cellcolor{blue!5} \textbf{3} & \cellcolor{blue!5} \textbf{3} & \cellcolor{blue!5} \textbf{3} & \cellcolor{blue!5} \textbf{3} \\
20 75& 3 & \cellcolor{blue!5} \textbf{3} & \cellcolor{blue!5} \textbf{3} & \cellcolor{blue!5} \textbf{3} & \cellcolor{blue!5} \textbf{3} & \cellcolor{blue!5} \textbf{3} & \cellcolor{blue!5} \textbf{3} & \cellcolor{blue!5} \textbf{3} \\
20 76& 3 & \cellcolor{blue!5} \textbf{3} & \cellcolor{blue!5} \textbf{3} & \cellcolor{blue!5} \textbf{3} & \cellcolor{blue!5} \textbf{3} & \cellcolor{blue!5} \textbf{3} & \cellcolor{blue!5} \textbf{3} & \cellcolor{blue!5} \textbf{3} \\
20 77& 3 & \cellcolor{blue!5} \textbf{3} & \cellcolor{blue!5} \textbf{3} & \cellcolor{blue!5} \textbf{3} & \cellcolor{blue!5} \textbf{3} & \cellcolor{blue!5} \textbf{3} & \cellcolor{blue!5} \textbf{3} & \cellcolor{blue!5} \textbf{3} \\
20 78& 3 & \cellcolor{blue!5} \textbf{3} & \cellcolor{blue!5} \textbf{3} & \cellcolor{blue!5} \textbf{3} & \cellcolor{blue!5} \textbf{3} & \cellcolor{blue!5} \textbf{3} & \cellcolor{blue!5} \textbf{3} & \cellcolor{blue!5} \textbf{3} \\
20 79& 3 & \cellcolor{blue!5} \textbf{3} & \cellcolor{blue!5} \textbf{3} & \cellcolor{blue!5} \textbf{3} & \cellcolor{blue!5} \textbf{3} & \cellcolor{blue!5} \textbf{3} & \cellcolor{blue!5} \textbf{3} & \cellcolor{blue!5} \textbf{3} \\
20 8& 3 & \cellcolor{blue!5} \textbf{3} & \cellcolor{blue!5} \textbf{3} & \cellcolor{blue!5} \textbf{3} & \cellcolor{blue!5} \textbf{3} & \cellcolor{blue!5} \textbf{3} & \cellcolor{blue!5} \textbf{3} & \cellcolor{blue!5} \textbf{3} \\
20 80& 3 & \cellcolor{blue!5} \textbf{3} & \cellcolor{blue!5} \textbf{3} & \cellcolor{blue!5} \textbf{3} & \cellcolor{blue!5} \textbf{3} & \cellcolor{blue!5} \textbf{3} & \cellcolor{blue!5} \textbf{3} & \cellcolor{blue!5} \textbf{3} \\
20 81& 3 & \cellcolor{blue!5} \textbf{3} & \cellcolor{blue!5} \textbf{3} & \cellcolor{blue!5} \textbf{3} & \cellcolor{blue!5} \textbf{3} & \cellcolor{blue!5} \textbf{3} & \cellcolor{blue!5} \textbf{3} & \cellcolor{blue!5} \textbf{3} \\
20 82& 4 & \cellcolor{blue!5} \textbf{4} & \cellcolor{blue!5} \textbf{4} & \cellcolor{blue!5} \textbf{4} & \cellcolor{blue!5} \textbf{4} & \cellcolor{blue!5} \textbf{4} & \cellcolor{blue!5} \textbf{4} & \cellcolor{blue!5} \textbf{4} \\
20 83& 3 & \cellcolor{blue!5} \textbf{3} & \cellcolor{blue!5} \textbf{3} & \cellcolor{blue!5} \textbf{3} & \cellcolor{blue!5} \textbf{3} & \cellcolor{blue!5} \textbf{3} & \cellcolor{blue!5} \textbf{3} & \cellcolor{blue!5} \textbf{3} \\
20 84& 3 & \cellcolor{blue!5} \textbf{3} & \cellcolor{blue!5} \textbf{3} & \cellcolor{blue!5} \textbf{3} & \cellcolor{blue!5} \textbf{3} & \cellcolor{blue!5} \textbf{3} & \cellcolor{blue!5} \textbf{3} & \cellcolor{blue!5} \textbf{3} \\
20 85& 3 & \cellcolor{blue!5} \textbf{3} & \cellcolor{blue!5} \textbf{3} & \cellcolor{blue!5} \textbf{3} & \cellcolor{blue!5} \textbf{3} & \cellcolor{blue!5} \textbf{3} & \cellcolor{blue!5} \textbf{3} & \cellcolor{blue!5} \textbf{3} \\
20 86& 3 & \cellcolor{blue!5} \textbf{3} & \cellcolor{blue!5} \textbf{3} & \cellcolor{blue!5} \textbf{3} & \cellcolor{blue!5} \textbf{3} & \cellcolor{blue!5} \textbf{3} & \cellcolor{blue!5} \textbf{3} & \cellcolor{blue!5} \textbf{3} \\
20 87& 3 & \cellcolor{blue!5} \textbf{3} & \cellcolor{blue!5} \textbf{3} & \cellcolor{blue!5} \textbf{3} & \cellcolor{blue!5} \textbf{3} & \cellcolor{blue!5} \textbf{3} & \cellcolor{blue!5} \textbf{3} & \cellcolor{blue!5} \textbf{3} \\
20 88& 3 & \cellcolor{blue!5} \textbf{3} & \cellcolor{blue!5} \textbf{3} & \cellcolor{blue!5} \textbf{3} & \cellcolor{blue!5} \textbf{3} & \cellcolor{blue!5} \textbf{3} & \cellcolor{blue!5} \textbf{3} & \cellcolor{blue!5} \textbf{3} \\
20 89& 3 & \cellcolor{blue!5} \textbf{3} & \cellcolor{blue!5} \textbf{3} & \cellcolor{blue!5} \textbf{3} & \cellcolor{blue!5} \textbf{3} & \cellcolor{blue!5} \textbf{3} & \cellcolor{blue!5} \textbf{3} & \cellcolor{blue!5} \textbf{3} \\
20 9& 3 & \cellcolor{blue!5} \textbf{3} & \cellcolor{blue!5} \textbf{3} & \cellcolor{blue!5} \textbf{3} & \cellcolor{blue!5} \textbf{3} & \cellcolor{blue!5} \textbf{3} & \cellcolor{blue!5} \textbf{3} & \cellcolor{blue!5} \textbf{3} \\
20 90& 3 & \cellcolor{blue!5} \textbf{3} & \cellcolor{blue!5} \textbf{3} & \cellcolor{blue!5} \textbf{3} & \cellcolor{blue!5} \textbf{3} & \cellcolor{blue!5} \textbf{3} & \cellcolor{blue!5} \textbf{3} & \cellcolor{blue!5} \textbf{3} \\
20 91& 11 & \cellcolor{blue!5} \textbf{11} & \cellcolor{blue!5} \textbf{11} & \cellcolor{blue!5} \textbf{11} & \cellcolor{blue!5} \textbf{11} & \cellcolor{blue!5} \textbf{11} & \cellcolor{blue!5} \textbf{11} & \cellcolor{blue!5} \textbf{11} \\
20 92& 11 & \cellcolor{blue!5} \textbf{11} & \cellcolor{blue!5} \textbf{11} & \cellcolor{blue!5} \textbf{11} & \cellcolor{blue!5} \textbf{11} & \cellcolor{blue!5} \textbf{11} & \cellcolor{blue!5} \textbf{11} & \cellcolor{blue!5} \textbf{11} \\
20 93& 13 & \cellcolor{blue!5} \textbf{13} & \cellcolor{blue!5} \textbf{13} & \cellcolor{blue!5} \textbf{13} & \cellcolor{blue!5} \textbf{13} & \cellcolor{blue!5} \textbf{13} & \cellcolor{blue!5} \textbf{13} & \cellcolor{blue!5} \textbf{13} \\
20 94& 10 & \cellcolor{blue!5} \textbf{10} & \cellcolor{blue!5} \textbf{10} & \cellcolor{blue!5} \textbf{10} & \cellcolor{blue!5} \textbf{10} & \cellcolor{blue!5} \textbf{10} & \cellcolor{blue!5} \textbf{10} & \cellcolor{blue!5} \textbf{10} \\
20 95& 12 & \cellcolor{blue!5} \textbf{12} & \cellcolor{blue!5} \textbf{12} & \cellcolor{blue!5} \textbf{12} & \cellcolor{blue!5} \textbf{12} & \cellcolor{blue!5} \textbf{12} & \cellcolor{blue!5} \textbf{12} & \cellcolor{blue!5} \textbf{12} \\
20 96& 10 & \cellcolor{blue!5} \textbf{10} & \cellcolor{blue!5} \textbf{10} & \cellcolor{blue!5} \textbf{10} & \cellcolor{blue!5} \textbf{10} & \cellcolor{blue!5} \textbf{10} & \cellcolor{blue!5} \textbf{10} & \cellcolor{blue!5} \textbf{10} \\
20 97& 13 & \cellcolor{blue!5} 15 & \cellcolor{blue!5} \textbf{15} & \cellcolor{blue!5} \textbf{15} & \cellcolor{blue!5} \textbf{15} & \cellcolor{blue!5} \textbf{15} & \cellcolor{blue!5} \textbf{15} & \cellcolor{blue!5} \textbf{15} \\
20 98& 13 & \cellcolor{blue!5} \textbf{13} & \cellcolor{blue!5} \textbf{13} & \cellcolor{blue!5} \textbf{13} & \cellcolor{blue!5} \textbf{13} & \cellcolor{blue!5} \textbf{13} & \cellcolor{blue!5} \textbf{13} & \cellcolor{blue!5} \textbf{13} \\
20 99& 12 & \cellcolor{blue!5} \textbf{12} & \cellcolor{blue!5} \textbf{12} & \cellcolor{blue!5} \textbf{12} & \cellcolor{blue!5} \textbf{12} & \cellcolor{blue!5} \textbf{12} & \cellcolor{blue!5} \textbf{12} & \cellcolor{blue!5} \textbf{12} \\
50 1& 8 & \cellcolor{blue!5} \textbf{8} & \cellcolor{blue!5} \textbf{8} & \cellcolor{blue!5} \textbf{8} & \cellcolor{blue!5} 8 & \cellcolor{blue!5} 8 & \cellcolor{blue!5} \textbf{8} & \cellcolor{blue!5} 8 \\
50 10& 7 & \cellcolor{blue!5} \textbf{7} & \cellcolor{blue!5} \textbf{7} & \cellcolor{blue!5} \textbf{7} & \cellcolor{blue!5} 7 & \cellcolor{blue!5} 7 & \cellcolor{blue!5} \textbf{7} & \cellcolor{blue!5} \textbf{7} \\
50 100& 7 & \cellcolor{blue!5} \textbf{7} & \cellcolor{blue!5} \textbf{7} & \cellcolor{blue!5} \textbf{7} & \cellcolor{blue!5} 7 & \cellcolor{blue!5} \textbf{7} & \cellcolor{blue!5} \textbf{7} & \cellcolor{blue!5} 7 \\
50 101& 29 & \cellcolor{blue!5} 30 & \cellcolor{blue!5} 30 & \cellcolor{blue!5} \textbf{30} & 33 & \cellcolor{blue!5} \textbf{30} & \cellcolor{blue!5} 30 & \cellcolor{blue!5} 30 \\
50 102& 30 & \cellcolor{blue!5} 32 & \cellcolor{blue!5} 32 & \cellcolor{blue!5} 32 & 34 & \cellcolor{blue!5} \textbf{32} & \cellcolor{blue!5} 32 & 33 \\
50 103& 29 & \cellcolor{blue!5} \textbf{29} & \cellcolor{blue!5} 29 & \cellcolor{blue!5} \textbf{29} & 30 & \cellcolor{blue!5} \textbf{29} & \cellcolor{blue!5} 29 & \cellcolor{blue!5} 29 \\
50 104& 25 & 28 & \cellcolor{blue!5} 27 & \cellcolor{blue!5} \textbf{27} & 29 & \cellcolor{blue!5} \textbf{27} & \cellcolor{blue!5} 27 & 28 \\
50 105& 23 & 25 & \cellcolor{blue!5} 24 & \cellcolor{blue!5} \textbf{24} & 27 & \cellcolor{blue!5} 24 & \cellcolor{blue!5} 24 & \cellcolor{blue!5} 24 \\
50 106& 27 & \cellcolor{blue!5} 28 & \cellcolor{blue!5} 28 & \cellcolor{blue!5} \textbf{28} & 29 & \cellcolor{blue!5} \textbf{28} & \cellcolor{blue!5} 28 & \cellcolor{blue!5} 28 \\
50 107& 28 & \cellcolor{blue!5} \textbf{28} & \cellcolor{blue!5} 28 & \cellcolor{blue!5} \textbf{28} & 31 & \cellcolor{blue!5} \textbf{28} & \cellcolor{blue!5} 28 & \cellcolor{blue!5} 28 \\
50 108& 30 & \cellcolor{blue!5} \textbf{30} & \cellcolor{blue!5} 30 & \cellcolor{blue!5} \textbf{30} & 33 & \cellcolor{blue!5} \textbf{30} & \cellcolor{blue!5} 30 & 31 \\
50 109& 30 & \cellcolor{blue!5} \textbf{30} & \cellcolor{blue!5} 30 & \cellcolor{blue!5} \textbf{30} & 31 & \cellcolor{blue!5} \textbf{30} & \cellcolor{blue!5} 30 & \cellcolor{blue!5} 30 \\
50 11& 7 & \cellcolor{blue!5} \textbf{7} & \cellcolor{blue!5} \textbf{7} & \cellcolor{blue!5} \textbf{7} & \cellcolor{blue!5} 7 & \cellcolor{blue!5} 7 & \cellcolor{blue!5} \textbf{7} & \cellcolor{blue!5} 7 \\
50 110& 26 & 27 & \cellcolor{blue!5} 26 & \cellcolor{blue!5} \textbf{26} & 28 & 27 & \cellcolor{blue!5} 26 & \cellcolor{blue!5} 26 \\
50 111& 28 & 29 & \cellcolor{blue!5} 28 & \cellcolor{blue!5} \textbf{28} & 29 & \cellcolor{blue!5} \textbf{28} & \cellcolor{blue!5} 28 & \cellcolor{blue!5} 28 \\
50 112& 27 & \cellcolor{blue!5} \textbf{27} & \cellcolor{blue!5} 27 & \cellcolor{blue!5} \textbf{27} & 29 & \cellcolor{blue!5} \textbf{27} & \cellcolor{blue!5} 27 & \cellcolor{blue!5} 27 \\
50 113& 28 & \cellcolor{blue!5} \textbf{28} & \cellcolor{blue!5} 28 & \cellcolor{blue!5} \textbf{28} & 31 & \cellcolor{blue!5} 28 & \cellcolor{blue!5} 28 & \cellcolor{blue!5} 28 \\
50 114& 27 & \cellcolor{blue!5} \textbf{27} & \cellcolor{blue!5} 27 & \cellcolor{blue!5} \textbf{27} & 30 & \cellcolor{blue!5} \textbf{27} & \cellcolor{blue!5} 27 & 28 \\
50 115& 26 & \cellcolor{blue!5} 28 & \cellcolor{blue!5} 28 & \cellcolor{blue!5} 28 & 31 & 29 & \cellcolor{blue!5} 28 & 29 \\
50 116& 31 & 33 & \cellcolor{blue!5} 32 & \cellcolor{blue!5} \textbf{32} & 34 & \cellcolor{blue!5} \textbf{32} & \cellcolor{blue!5} 32 & \cellcolor{blue!5} 32 \\
50 117& 26 & \cellcolor{blue!5} 27 & \cellcolor{blue!5} 27 & \cellcolor{blue!5} \textbf{27} & \cellcolor{blue!5} 27 & \cellcolor{blue!5} \textbf{27} & \cellcolor{blue!5} 27 & \cellcolor{blue!5} 27 \\
50 118& 29 & \cellcolor{blue!5} \textbf{29} & \cellcolor{blue!5} 29 & \cellcolor{blue!5} \textbf{29} & 32 & \cellcolor{blue!5} \textbf{29} & \cellcolor{blue!5} 29 & \cellcolor{blue!5} 29 \\
50 119& 25 & \cellcolor{blue!5} \textbf{25} & \cellcolor{blue!5} \textbf{25} & \cellcolor{blue!5} \textbf{25} & 27 & \cellcolor{blue!5} \textbf{25} & \cellcolor{blue!5} 25 & \cellcolor{blue!5} 25 \\
50 12& 6 & \cellcolor{blue!5} \textbf{6} & \cellcolor{blue!5} \textbf{6} & \cellcolor{blue!5} \textbf{6} & 7 & 7 & \cellcolor{blue!5} \textbf{6} & \cellcolor{blue!5} \textbf{6} \\
50 120& 27 & \cellcolor{blue!5} \textbf{27} & \cellcolor{blue!5} 27 & \cellcolor{blue!5} \textbf{27} & 29 & \cellcolor{blue!5} \textbf{27} & \cellcolor{blue!5} 27 & 28 \\
50 121& 31 & \cellcolor{blue!5} 32 & \cellcolor{blue!5} 32 & \cellcolor{blue!5} \textbf{32} & \cellcolor{blue!5} 32 & \cellcolor{blue!5} \textbf{32} & \cellcolor{blue!5} 32 & \cellcolor{blue!5} 32 \\
50 122& 28 & 30 & \cellcolor{blue!5} 29 & \cellcolor{blue!5} 29 & 32 & \cellcolor{blue!5} \textbf{29} & \cellcolor{blue!5} 29 & 30 \\
50 123& 32 & \cellcolor{blue!5} \textbf{32} & \cellcolor{blue!5} 32 & \cellcolor{blue!5} \textbf{32} & 33 & \cellcolor{blue!5} \textbf{32} & \cellcolor{blue!5} 32 & \cellcolor{blue!5} 32 \\
50 124& 29 & 30 & \cellcolor{blue!5} 29 & \cellcolor{blue!5} \textbf{29} & 31 & \cellcolor{blue!5} \textbf{29} & \cellcolor{blue!5} 29 & 30 \\
50 125& 32 & \cellcolor{blue!5} 33 & \cellcolor{blue!5} 33 & \cellcolor{blue!5} \textbf{33} & 34 & \cellcolor{blue!5} 33 & \cellcolor{blue!5} 33 & \cellcolor{blue!5} 33 \\
50 126& 12 & \cellcolor{blue!5} \textbf{12} & \cellcolor{blue!5} \textbf{12} & \cellcolor{blue!5} \textbf{12} & \cellcolor{blue!5} 12 & \cellcolor{blue!5} \textbf{12} & \cellcolor{blue!5} \textbf{12} & \cellcolor{blue!5} \textbf{12} \\
50 127& 14 & \cellcolor{blue!5} \textbf{14} & \cellcolor{blue!5} \textbf{14} & \cellcolor{blue!5} \textbf{14} & \cellcolor{blue!5} 14 & \cellcolor{blue!5} \textbf{14} & \cellcolor{blue!5} \textbf{14} & \cellcolor{blue!5} 14 \\
50 128& 12 & \cellcolor{blue!5} \textbf{12} & \cellcolor{blue!5} \textbf{12} & \cellcolor{blue!5} \textbf{12} & 13 & \cellcolor{blue!5} \textbf{12} & \cellcolor{blue!5} \textbf{12} & \cellcolor{blue!5} 12 \\
50 129& 13 & \cellcolor{blue!5} \textbf{13} & \cellcolor{blue!5} \textbf{13} & \cellcolor{blue!5} \textbf{13} & \cellcolor{blue!5} 13 & \cellcolor{blue!5} \textbf{13} & \cellcolor{blue!5} \textbf{13} & \cellcolor{blue!5} 13 \\
50 13& 6 & \cellcolor{blue!5} \textbf{6} & \cellcolor{blue!5} \textbf{6} & \cellcolor{blue!5} \textbf{6} & \cellcolor{blue!5} 6 & \cellcolor{blue!5} 6 & \cellcolor{blue!5} \textbf{6} & \cellcolor{blue!5} \textbf{6} \\
50 130& 13 & \cellcolor{blue!5} \textbf{13} & \cellcolor{blue!5} \textbf{13} & \cellcolor{blue!5} \textbf{13} & \cellcolor{blue!5} 13 & \cellcolor{blue!5} \textbf{13} & \cellcolor{blue!5} \textbf{13} & \cellcolor{blue!5} 13 \\
50 131& 12 & \cellcolor{blue!5} \textbf{12} & \cellcolor{blue!5} \textbf{12} & \cellcolor{blue!5} \textbf{12} & \cellcolor{blue!5} 12 & \cellcolor{blue!5} \textbf{12} & \cellcolor{blue!5} \textbf{12} & \cellcolor{blue!5} 12 \\
50 132& 12 & \cellcolor{blue!5} \textbf{12} & \cellcolor{blue!5} \textbf{12} & \cellcolor{blue!5} \textbf{12} & 13 & \cellcolor{blue!5} \textbf{12} & \cellcolor{blue!5} \textbf{12} & \cellcolor{blue!5} 12 \\
50 133& 12 & \cellcolor{blue!5} \textbf{12} & \cellcolor{blue!5} \textbf{12} & \cellcolor{blue!5} \textbf{12} & \cellcolor{blue!5} 12 & \cellcolor{blue!5} \textbf{12} & \cellcolor{blue!5} \textbf{12} & \cellcolor{blue!5} 12 \\
50 134& 14 & \cellcolor{blue!5} \textbf{14} & \cellcolor{blue!5} \textbf{14} & \cellcolor{blue!5} \textbf{14} & 15 & \cellcolor{blue!5} \textbf{14} & \cellcolor{blue!5} \textbf{14} & \cellcolor{blue!5} 14 \\
50 135& 13 & \cellcolor{blue!5} \textbf{13} & \cellcolor{blue!5} \textbf{13} & \cellcolor{blue!5} \textbf{13} & 14 & \cellcolor{blue!5} \textbf{13} & \cellcolor{blue!5} \textbf{13} & \cellcolor{blue!5} 13 \\
50 136& 11 & \cellcolor{blue!5} \textbf{11} & \cellcolor{blue!5} \textbf{11} & \cellcolor{blue!5} \textbf{11} & \cellcolor{blue!5} 11 & \cellcolor{blue!5} \textbf{11} & \cellcolor{blue!5} \textbf{11} & \cellcolor{blue!5} 11 \\
50 137& 11 & \cellcolor{blue!5} \textbf{11} & \cellcolor{blue!5} \textbf{11} & \cellcolor{blue!5} \textbf{11} & \cellcolor{blue!5} 11 & \cellcolor{blue!5} \textbf{11} & \cellcolor{blue!5} \textbf{11} & \cellcolor{blue!5} 11 \\
50 138& 12 & \cellcolor{blue!5} \textbf{12} & \cellcolor{blue!5} \textbf{12} & \cellcolor{blue!5} \textbf{12} & \cellcolor{blue!5} 12 & \cellcolor{blue!5} \textbf{12} & \cellcolor{blue!5} \textbf{12} & \cellcolor{blue!5} 12 \\
50 139& 11 & \cellcolor{blue!5} \textbf{11} & \cellcolor{blue!5} \textbf{11} & \cellcolor{blue!5} \textbf{11} & 12 & \cellcolor{blue!5} \textbf{11} & \cellcolor{blue!5} 11 & 12 \\
50 14& 7 & \cellcolor{blue!5} \textbf{7} & \cellcolor{blue!5} \textbf{7} & \cellcolor{blue!5} \textbf{7} & \cellcolor{blue!5} 7 & \cellcolor{blue!5} 7 & \cellcolor{blue!5} \textbf{7} & \cellcolor{blue!5} 7 \\
50 140& 12 & \cellcolor{blue!5} \textbf{12} & \cellcolor{blue!5} \textbf{12} & \cellcolor{blue!5} \textbf{12} & \cellcolor{blue!5} 12 & \cellcolor{blue!5} \textbf{12} & \cellcolor{blue!5} \textbf{12} & \cellcolor{blue!5} 12 \\
50 141& 13 & \cellcolor{blue!5} \textbf{13} & \cellcolor{blue!5} \textbf{13} & \cellcolor{blue!5} \textbf{13} & \cellcolor{blue!5} 13 & \cellcolor{blue!5} \textbf{13} & \cellcolor{blue!5} \textbf{13} & \cellcolor{blue!5} 13 \\
50 142& 11 & \cellcolor{blue!5} \textbf{11} & \cellcolor{blue!5} \textbf{11} & \cellcolor{blue!5} \textbf{11} & \cellcolor{blue!5} 11 & \cellcolor{blue!5} \textbf{11} & \cellcolor{blue!5} \textbf{11} & \cellcolor{blue!5} 11 \\
50 143& 12 & \cellcolor{blue!5} \textbf{12} & \cellcolor{blue!5} \textbf{12} & \cellcolor{blue!5} \textbf{12} & \cellcolor{blue!5} 12 & \cellcolor{blue!5} \textbf{12} & \cellcolor{blue!5} \textbf{12} & \cellcolor{blue!5} 12 \\
50 144& 13 & \cellcolor{blue!5} \textbf{13} & \cellcolor{blue!5} \textbf{13} & \cellcolor{blue!5} \textbf{13} & \cellcolor{blue!5} 13 & \cellcolor{blue!5} \textbf{13} & \cellcolor{blue!5} \textbf{13} & \cellcolor{blue!5} 13 \\
50 145& 10 & \cellcolor{blue!5} \textbf{10} & \cellcolor{blue!5} \textbf{10} & \cellcolor{blue!5} \textbf{10} & \cellcolor{blue!5} 10 & \cellcolor{blue!5} \textbf{10} & \cellcolor{blue!5} \textbf{10} & \cellcolor{blue!5} 10 \\
50 146& 13 & \cellcolor{blue!5} \textbf{13} & \cellcolor{blue!5} \textbf{13} & \cellcolor{blue!5} \textbf{13} & \cellcolor{blue!5} 13 & \cellcolor{blue!5} \textbf{13} & \cellcolor{blue!5} \textbf{13} & \cellcolor{blue!5} 13 \\
50 147& 13 & \cellcolor{blue!5} \textbf{13} & \cellcolor{blue!5} \textbf{13} & \cellcolor{blue!5} \textbf{13} & \cellcolor{blue!5} 13 & \cellcolor{blue!5} \textbf{13} & \cellcolor{blue!5} \textbf{13} & \cellcolor{blue!5} 13 \\
50 148& 10 & \cellcolor{blue!5} \textbf{10} & \cellcolor{blue!5} \textbf{10} & \cellcolor{blue!5} \textbf{10} & \cellcolor{blue!5} 10 & \cellcolor{blue!5} \textbf{10} & \cellcolor{blue!5} \textbf{10} & \cellcolor{blue!5} 10 \\
50 149& 12 & \cellcolor{blue!5} \textbf{12} & \cellcolor{blue!5} \textbf{12} & \cellcolor{blue!5} \textbf{12} & \cellcolor{blue!5} 12 & \cellcolor{blue!5} \textbf{12} & \cellcolor{blue!5} \textbf{12} & \cellcolor{blue!5} 12 \\
50 15& 8 & \cellcolor{blue!5} \textbf{8} & \cellcolor{blue!5} \textbf{8} & \cellcolor{blue!5} \textbf{8} & \cellcolor{blue!5} 8 & \cellcolor{blue!5} 8 & \cellcolor{blue!5} \textbf{8} & \cellcolor{blue!5} 8 \\
50 150& 11 & \cellcolor{blue!5} \textbf{11} & \cellcolor{blue!5} \textbf{11} & \cellcolor{blue!5} \textbf{11} & \cellcolor{blue!5} 11 & \cellcolor{blue!5} \textbf{11} & \cellcolor{blue!5} \textbf{11} & \cellcolor{blue!5} 11 \\
50 151& 7 & \cellcolor{blue!5} \textbf{7} & \cellcolor{blue!5} \textbf{7} & \cellcolor{blue!5} \textbf{7} & \cellcolor{blue!5} 7 & \cellcolor{blue!5} 7 & \cellcolor{blue!5} \textbf{7} & \cellcolor{blue!5} 7 \\
50 152& 7 & \cellcolor{blue!5} \textbf{7} & \cellcolor{blue!5} \textbf{7} & \cellcolor{blue!5} \textbf{7} & \cellcolor{blue!5} 7 & \cellcolor{blue!5} 7 & \cellcolor{blue!5} \textbf{7} & \cellcolor{blue!5} 7 \\
50 153& 7 & \cellcolor{blue!5} \textbf{7} & \cellcolor{blue!5} \textbf{7} & \cellcolor{blue!5} \textbf{7} & 8 & 8 & \cellcolor{blue!5} \textbf{7} & \cellcolor{blue!5} 7 \\
50 154& 8 & \cellcolor{blue!5} \textbf{8} & \cellcolor{blue!5} \textbf{8} & \cellcolor{blue!5} \textbf{8} & \cellcolor{blue!5} 8 & \cellcolor{blue!5} 8 & \cellcolor{blue!5} \textbf{8} & \cellcolor{blue!5} 8 \\
50 155& 7 & \cellcolor{blue!5} \textbf{7} & \cellcolor{blue!5} \textbf{7} & \cellcolor{blue!5} \textbf{7} & \cellcolor{blue!5} 7 & \cellcolor{blue!5} 7 & \cellcolor{blue!5} \textbf{7} & \cellcolor{blue!5} 7 \\
50 156& 7 & \cellcolor{blue!5} \textbf{7} & \cellcolor{blue!5} \textbf{7} & \cellcolor{blue!5} \textbf{7} & \cellcolor{blue!5} 7 & \cellcolor{blue!5} 7 & \cellcolor{blue!5} \textbf{7} & \cellcolor{blue!5} 7 \\
50 157& 8 & \cellcolor{blue!5} \textbf{8} & \cellcolor{blue!5} \textbf{8} & \cellcolor{blue!5} \textbf{8} & \cellcolor{blue!5} 8 & \cellcolor{blue!5} 8 & \cellcolor{blue!5} \textbf{8} & \cellcolor{blue!5} \textbf{8} \\
50 158& 7 & \cellcolor{blue!5} \textbf{7} & \cellcolor{blue!5} \textbf{7} & \cellcolor{blue!5} \textbf{7} & \cellcolor{blue!5} 7 & 46 & \cellcolor{blue!5} \textbf{7} & \cellcolor{blue!5} 7 \\
50 159& 7 & \cellcolor{blue!5} \textbf{7} & \cellcolor{blue!5} \textbf{7} & \cellcolor{blue!5} \textbf{7} & \cellcolor{blue!5} 7 & \cellcolor{blue!5} 7 & \cellcolor{blue!5} \textbf{7} & \cellcolor{blue!5} \textbf{7} \\
50 16& 8 & \cellcolor{blue!5} \textbf{8} & \cellcolor{blue!5} \textbf{8} & \cellcolor{blue!5} \textbf{8} & \cellcolor{blue!5} 8 & \cellcolor{blue!5} 8 & \cellcolor{blue!5} \textbf{8} & \cellcolor{blue!5} 8 \\
50 160& 8 & \cellcolor{blue!5} \textbf{8} & \cellcolor{blue!5} \textbf{8} & \cellcolor{blue!5} \textbf{8} & \cellcolor{blue!5} 8 & \cellcolor{blue!5} 8 & \cellcolor{blue!5} \textbf{8} & \cellcolor{blue!5} 8 \\
50 161& 7 & \cellcolor{blue!5} \textbf{7} & \cellcolor{blue!5} \textbf{7} & \cellcolor{blue!5} \textbf{7} & \cellcolor{blue!5} 7 & \cellcolor{blue!5} 7 & \cellcolor{blue!5} \textbf{7} & \cellcolor{blue!5} 7 \\
50 162& 8 & \cellcolor{blue!5} \textbf{8} & \cellcolor{blue!5} \textbf{8} & \cellcolor{blue!5} \textbf{8} & \cellcolor{blue!5} 8 & \cellcolor{blue!5} 8 & \cellcolor{blue!5} \textbf{8} & \cellcolor{blue!5} 8 \\
50 163& 7 & \cellcolor{blue!5} \textbf{7} & \cellcolor{blue!5} \textbf{7} & \cellcolor{blue!5} \textbf{7} & \cellcolor{blue!5} 7 & \cellcolor{blue!5} 7 & \cellcolor{blue!5} \textbf{7} & \cellcolor{blue!5} 7 \\
50 164& 7 & \cellcolor{blue!5} \textbf{7} & \cellcolor{blue!5} \textbf{7} & \cellcolor{blue!5} \textbf{7} & \cellcolor{blue!5} 7 & \cellcolor{blue!5} 7 & \cellcolor{blue!5} \textbf{7} & \cellcolor{blue!5} 7 \\
50 165& 8 & \cellcolor{blue!5} \textbf{8} & \cellcolor{blue!5} \textbf{8} & \cellcolor{blue!5} \textbf{8} & \cellcolor{blue!5} 8 & \cellcolor{blue!5} 8 & \cellcolor{blue!5} \textbf{8} & \cellcolor{blue!5} 8 \\
50 166& 8 & \cellcolor{blue!5} \textbf{8} & \cellcolor{blue!5} \textbf{8} & \cellcolor{blue!5} \textbf{8} & \cellcolor{blue!5} 8 & \cellcolor{blue!5} 8 & \cellcolor{blue!5} \textbf{8} & \cellcolor{blue!5} 8 \\
50 167& 7 & \cellcolor{blue!5} \textbf{7} & \cellcolor{blue!5} \textbf{7} & \cellcolor{blue!5} \textbf{7} & 8 & 8 & \cellcolor{blue!5} \textbf{7} & \cellcolor{blue!5} 7 \\
50 168& 8 & \cellcolor{blue!5} \textbf{8} & \cellcolor{blue!5} \textbf{8} & \cellcolor{blue!5} \textbf{8} & 9 & \cellcolor{blue!5} 8 & \cellcolor{blue!5} \textbf{8} & \cellcolor{blue!5} 8 \\
50 169& 8 & \cellcolor{blue!5} \textbf{8} & \cellcolor{blue!5} \textbf{8} & \cellcolor{blue!5} \textbf{8} & \cellcolor{blue!5} 8 & \cellcolor{blue!5} 8 & \cellcolor{blue!5} \textbf{8} & \cellcolor{blue!5} 8 \\
50 17& 7 & \cellcolor{blue!5} \textbf{7} & \cellcolor{blue!5} \textbf{7} & \cellcolor{blue!5} \textbf{7} & \cellcolor{blue!5} 7 & \cellcolor{blue!5} 7 & \cellcolor{blue!5} \textbf{7} & \cellcolor{blue!5} \textbf{7} \\
50 170& 7 & \cellcolor{blue!5} \textbf{7} & \cellcolor{blue!5} \textbf{7} & \cellcolor{blue!5} \textbf{7} & 8 & \cellcolor{blue!5} 7 & \cellcolor{blue!5} \textbf{7} & \cellcolor{blue!5} 7 \\
50 171& 8 & \cellcolor{blue!5} \textbf{8} & \cellcolor{blue!5} \textbf{8} & \cellcolor{blue!5} \textbf{8} & \cellcolor{blue!5} 8 & \cellcolor{blue!5} 8 & \cellcolor{blue!5} \textbf{8} & \cellcolor{blue!5} 8 \\
50 172& 7 & \cellcolor{blue!5} \textbf{7} & \cellcolor{blue!5} \textbf{7} & \cellcolor{blue!5} \textbf{7} & \cellcolor{blue!5} 7 & \cellcolor{blue!5} 7 & \cellcolor{blue!5} \textbf{7} & \cellcolor{blue!5} \textbf{7} \\
50 173& 7 & \cellcolor{blue!5} \textbf{7} & \cellcolor{blue!5} \textbf{7} & \cellcolor{blue!5} \textbf{7} & 8 & 8 & \cellcolor{blue!5} \textbf{7} & \cellcolor{blue!5} 7 \\
50 174& 7 & \cellcolor{blue!5} \textbf{7} & \cellcolor{blue!5} \textbf{7} & \cellcolor{blue!5} \textbf{7} & \cellcolor{blue!5} 7 & \cellcolor{blue!5} 7 & \cellcolor{blue!5} \textbf{7} & \cellcolor{blue!5} 7 \\
50 175& 7 & \cellcolor{blue!5} \textbf{7} & \cellcolor{blue!5} \textbf{7} & \cellcolor{blue!5} \textbf{7} & 8 & \cellcolor{blue!5} 7 & \cellcolor{blue!5} \textbf{7} & \cellcolor{blue!5} 7 \\
50 176& 27 & \cellcolor{blue!5} \textbf{27} & \cellcolor{blue!5} 27 & \cellcolor{blue!5} \textbf{27} & 33 & 28 & \cellcolor{blue!5} 27 & \cellcolor{blue!5} 27 \\
50 177& 27 & \cellcolor{blue!5} 28 & \cellcolor{blue!5} 28 & \cellcolor{blue!5} 28 & 33 & \cellcolor{blue!5} 28 & \cellcolor{blue!5} 28 & \cellcolor{blue!5} 28 \\
50 178& 27 & \cellcolor{blue!5} 28 & \cellcolor{blue!5} 28 & \cellcolor{blue!5} 28 & 32 & \cellcolor{blue!5} 28 & \cellcolor{blue!5} 28 & \cellcolor{blue!5} 28 \\
50 179& 26 & \cellcolor{blue!20} \textbf{26} & 27 & \cellcolor{blue!20} \textbf{26} & 32 & 28 & 27 & 27 \\
50 18& 7 & \cellcolor{blue!5} \textbf{7} & \cellcolor{blue!5} \textbf{7} & \cellcolor{blue!5} \textbf{7} & \cellcolor{blue!5} 7 & \cellcolor{blue!5} 7 & \cellcolor{blue!5} \textbf{7} & \cellcolor{blue!5} 7 \\
50 180& 26 & \cellcolor{blue!5} \textbf{26} & \cellcolor{blue!5} 26 & \cellcolor{blue!5} \textbf{26} & 30 & \cellcolor{blue!5} 26 & \cellcolor{blue!5} 26 & \cellcolor{blue!5} 26 \\
50 181& 29 & \cellcolor{blue!10} \textbf{29} & \cellcolor{blue!10} 29 & \cellcolor{blue!10} \textbf{29} & 33 & 31 & 30 & 30 \\
50 182& 26 & 27 & 27 & 27 & 30 & 27 & \cellcolor{blue!40} 26 & 27 \\
50 183& 28 & \cellcolor{blue!10} \textbf{28} & 29 & \cellcolor{blue!10} \textbf{28} & 33 & \cellcolor{blue!10} 28 & 29 & 29 \\
50 184& 38 & \cellcolor{blue!5} \textbf{38} & \cellcolor{blue!5} 38 & \cellcolor{blue!5} \textbf{38} & 39 & 40 & \cellcolor{blue!5} 38 & \cellcolor{blue!5} 38 \\
50 185& 26 & \cellcolor{blue!10} \textbf{26} & 27 & \cellcolor{blue!10} \textbf{26} & 32 & \cellcolor{blue!10} 26 & 27 & 27 \\
50 186& 26 & \cellcolor{blue!5} \textbf{26} & \cellcolor{blue!5} 26 & \cellcolor{blue!5} \textbf{26} & 32 & 27 & \cellcolor{blue!5} 26 & 27 \\
50 187& 25 & 27 & \cellcolor{blue!5} 26 & \cellcolor{blue!5} 26 & 31 & \cellcolor{blue!5} 26 & \cellcolor{blue!5} 26 & \cellcolor{blue!5} 26 \\
50 188& 24 & \cellcolor{blue!5} 25 & \cellcolor{blue!5} 25 & \cellcolor{blue!5} 25 & 27 & \cellcolor{blue!5} 25 & \cellcolor{blue!5} 25 & \cellcolor{blue!5} 25 \\
50 189& 26 & \cellcolor{blue!5} \textbf{26} & \cellcolor{blue!5} 26 & \cellcolor{blue!5} 26 & 31 & 28 & \cellcolor{blue!5} 26 & \cellcolor{blue!5} 26 \\
50 19& 8 & \cellcolor{blue!5} \textbf{8} & \cellcolor{blue!5} \textbf{8} & \cellcolor{blue!5} \textbf{8} & \cellcolor{blue!5} 8 & \cellcolor{blue!5} 8 & \cellcolor{blue!5} \textbf{8} & \cellcolor{blue!5} 8 \\
50 190& 30 & \cellcolor{blue!5} \textbf{30} & \cellcolor{blue!5} 30 & \cellcolor{blue!5} \textbf{30} & 34 & 31 & \cellcolor{blue!5} 30 & \cellcolor{blue!5} 30 \\
50 191& 27 & 28 & 28 & \cellcolor{blue!40} \textbf{27} & 33 & 30 & 28 & 28 \\
50 192& 27 & 28 & \cellcolor{blue!10} 27 & \cellcolor{blue!10} \textbf{27} & 31 & 28 & \cellcolor{blue!10} 27 & 28 \\
50 193& 28 & \cellcolor{blue!5} \textbf{28} & \cellcolor{blue!5} 28 & \cellcolor{blue!5} \textbf{28} & 35 & 29 & \cellcolor{blue!5} 28 & 29 \\
50 194& 28 & 29 & \cellcolor{blue!5} 28 & \cellcolor{blue!5} \textbf{28} & 32 & 39 & \cellcolor{blue!5} 28 & \cellcolor{blue!5} 28 \\
50 195& 28 & \cellcolor{blue!5} \textbf{28} & \cellcolor{blue!5} 28 & \cellcolor{blue!5} \textbf{28} & 33 & \cellcolor{blue!5} 28 & \cellcolor{blue!5} 28 & 29 \\
50 196& 27 & \cellcolor{blue!5} \textbf{27} & \cellcolor{blue!5} 27 & \cellcolor{blue!5} \textbf{27} & 33 & 28 & \cellcolor{blue!5} 27 & 28 \\
50 197& 28 & 29 & \cellcolor{blue!10} 28 & \cellcolor{blue!10} \textbf{28} & 32 & 29 & \cellcolor{blue!10} 28 & 29 \\
50 198& 28 & \cellcolor{blue!5} \textbf{28} & \cellcolor{blue!5} 28 & \cellcolor{blue!5} \textbf{28} & 32 & \cellcolor{blue!5} 28 & \cellcolor{blue!5} 28 & \cellcolor{blue!5} 28 \\
50 199& 29 & \cellcolor{blue!5} \textbf{29} & \cellcolor{blue!5} 29 & \cellcolor{blue!5} \textbf{29} & 34 & \cellcolor{blue!5} 29 & \cellcolor{blue!5} 29 & \cellcolor{blue!5} 29 \\
50 2& 6 & \cellcolor{blue!5} \textbf{6} & \cellcolor{blue!5} \textbf{6} & \cellcolor{blue!5} \textbf{6} & \cellcolor{blue!5} 6 & \cellcolor{blue!5} 6 & \cellcolor{blue!5} \textbf{6} & \cellcolor{blue!5} \textbf{6} \\
50 20& 8 & \cellcolor{blue!5} \textbf{8} & \cellcolor{blue!5} \textbf{8} & \cellcolor{blue!5} \textbf{8} & \cellcolor{blue!5} 8 & \cellcolor{blue!5} 8 & \cellcolor{blue!5} \textbf{8} & \cellcolor{blue!5} 8 \\
50 200& 24 & 26 & \cellcolor{blue!5} 25 & \cellcolor{blue!5} 25 & 30 & 37 & \cellcolor{blue!5} 25 & \cellcolor{blue!5} 25 \\
50 201& 13 & \cellcolor{blue!5} \textbf{13} & \cellcolor{blue!5} \textbf{13} & \cellcolor{blue!5} \textbf{13} & \cellcolor{blue!5} 13 & \cellcolor{blue!5} 13 & \cellcolor{blue!5} 13 & \cellcolor{blue!5} 13 \\
50 202& 9 & \cellcolor{blue!5} \textbf{9} & \cellcolor{blue!5} \textbf{9} & \cellcolor{blue!5} \textbf{9} & 10 & \cellcolor{blue!5} 9 & \cellcolor{blue!5} \textbf{9} & \cellcolor{blue!5} 9 \\
50 203& 11 & \cellcolor{blue!5} \textbf{11} & \cellcolor{blue!5} \textbf{11} & \cellcolor{blue!5} \textbf{11} & 12 & \cellcolor{blue!5} 11 & \cellcolor{blue!5} \textbf{11} & \cellcolor{blue!5} 11 \\
50 204& 10 & \cellcolor{blue!5} \textbf{10} & \cellcolor{blue!5} \textbf{10} & \cellcolor{blue!5} \textbf{10} & 11 & 11 & \cellcolor{blue!5} \textbf{10} & \cellcolor{blue!5} 10 \\
50 205& 13 & \cellcolor{blue!5} \textbf{13} & \cellcolor{blue!5} \textbf{13} & \cellcolor{blue!5} \textbf{13} & \cellcolor{blue!5} 13 & \cellcolor{blue!5} 13 & \cellcolor{blue!5} 13 & \cellcolor{blue!5} 13 \\
50 206& 11 & \cellcolor{blue!10} \textbf{11} & \cellcolor{blue!10} \textbf{11} & 12 & 13 & 12 & \cellcolor{blue!10} 11 & 12 \\
50 207& 10 & \cellcolor{blue!5} \textbf{10} & \cellcolor{blue!5} \textbf{10} & \cellcolor{blue!5} \textbf{10} & \cellcolor{blue!5} 10 & \cellcolor{blue!5} 10 & \cellcolor{blue!5} 10 & \cellcolor{blue!5} 10 \\
50 208& 13 & \cellcolor{blue!5} \textbf{13} & \cellcolor{blue!5} \textbf{13} & \cellcolor{blue!5} \textbf{13} & 14 & 50 & \cellcolor{blue!5} 13 & \cellcolor{blue!5} 13 \\
50 209& 11 & \cellcolor{blue!5} \textbf{11} & \cellcolor{blue!5} \textbf{11} & \cellcolor{blue!5} \textbf{11} & \cellcolor{blue!5} 11 & \cellcolor{blue!5} 11 & \cellcolor{blue!5} \textbf{11} & \cellcolor{blue!5} 11 \\
50 21& 6 & \cellcolor{blue!5} \textbf{6} & \cellcolor{blue!5} \textbf{6} & \cellcolor{blue!5} \textbf{6} & \cellcolor{blue!5} 6 & \cellcolor{blue!5} 6 & \cellcolor{blue!5} \textbf{6} & \cellcolor{blue!5} \textbf{6} \\
50 210& 13 & \cellcolor{blue!5} \textbf{13} & \cellcolor{blue!5} \textbf{13} & \cellcolor{blue!5} \textbf{13} & 14 & \cellcolor{blue!5} 13 & \cellcolor{blue!5} 13 & \cellcolor{blue!5} 13 \\
50 211& 12 & \cellcolor{blue!5} \textbf{12} & \cellcolor{blue!5} \textbf{12} & \cellcolor{blue!5} \textbf{12} & \cellcolor{blue!5} 12 & \cellcolor{blue!5} 12 & \cellcolor{blue!5} \textbf{12} & \cellcolor{blue!5} 12 \\
50 212& 10 & \cellcolor{blue!5} \textbf{10} & \cellcolor{blue!5} \textbf{10} & \cellcolor{blue!5} \textbf{10} & 11 & \cellcolor{blue!5} 10 & \cellcolor{blue!5} 10 & \cellcolor{blue!5} 10 \\
50 213& 13 & \cellcolor{blue!5} \textbf{13} & \cellcolor{blue!5} \textbf{13} & \cellcolor{blue!5} \textbf{13} & 14 & \cellcolor{blue!5} 13 & \cellcolor{blue!5} 13 & \cellcolor{blue!5} 13 \\
50 214& 11 & \cellcolor{blue!5} \textbf{11} & \cellcolor{blue!5} \textbf{11} & \cellcolor{blue!5} \textbf{11} & \cellcolor{blue!5} 11 & \cellcolor{blue!5} 11 & \cellcolor{blue!5} 11 & \cellcolor{blue!5} 11 \\
50 215& 11 & \cellcolor{blue!5} \textbf{11} & \cellcolor{blue!5} \textbf{11} & \cellcolor{blue!5} \textbf{11} & \cellcolor{blue!5} 11 & \cellcolor{blue!5} 11 & \cellcolor{blue!5} \textbf{11} & \cellcolor{blue!5} 11 \\
50 216& 12 & \cellcolor{blue!5} \textbf{12} & \cellcolor{blue!5} \textbf{12} & \cellcolor{blue!5} \textbf{12} & 13 & \cellcolor{blue!5} 12 & \cellcolor{blue!5} 12 & \cellcolor{blue!5} 12 \\
50 217& 13 & \cellcolor{blue!5} \textbf{13} & \cellcolor{blue!5} \textbf{13} & \cellcolor{blue!5} \textbf{13} & 14 & \cellcolor{blue!5} 13 & \cellcolor{blue!5} 13 & \cellcolor{blue!5} 13 \\
50 218& 12 & \cellcolor{blue!5} \textbf{12} & \cellcolor{blue!5} \textbf{12} & \cellcolor{blue!5} \textbf{12} & 13 & \cellcolor{blue!5} 12 & \cellcolor{blue!5} 12 & \cellcolor{blue!5} 12 \\
50 219& 11 & \cellcolor{blue!5} \textbf{11} & \cellcolor{blue!5} \textbf{11} & \cellcolor{blue!5} \textbf{11} & \cellcolor{blue!5} 11 & \cellcolor{blue!5} 11 & \cellcolor{blue!5} \textbf{11} & \cellcolor{blue!5} 11 \\
50 22& 7 & \cellcolor{blue!5} \textbf{7} & \cellcolor{blue!5} \textbf{7} & \cellcolor{blue!5} \textbf{7} & \cellcolor{blue!5} 7 & 44 & \cellcolor{blue!5} \textbf{7} & \cellcolor{blue!5} 7 \\
50 220& 11 & \cellcolor{blue!5} \textbf{11} & \cellcolor{blue!5} \textbf{11} & \cellcolor{blue!5} \textbf{11} & 12 & \cellcolor{blue!5} 11 & \cellcolor{blue!5} \textbf{11} & \cellcolor{blue!5} 11 \\
50 221& 11 & \cellcolor{blue!5} \textbf{11} & \cellcolor{blue!5} \textbf{11} & \cellcolor{blue!5} \textbf{11} & 12 & \cellcolor{blue!5} 11 & \cellcolor{blue!5} \textbf{11} & \cellcolor{blue!5} 11 \\
50 222& 14 & \cellcolor{blue!5} \textbf{14} & \cellcolor{blue!5} \textbf{14} & \cellcolor{blue!5} \textbf{14} & 16 & \cellcolor{blue!5} 14 & \cellcolor{blue!5} 14 & \cellcolor{blue!5} 14 \\
50 223& 11 & \cellcolor{blue!5} \textbf{11} & \cellcolor{blue!5} \textbf{11} & \cellcolor{blue!5} \textbf{11} & 12 & \cellcolor{blue!5} 11 & \cellcolor{blue!5} 11 & \cellcolor{blue!5} 11 \\
50 224& 11 & \cellcolor{blue!5} \textbf{11} & \cellcolor{blue!5} \textbf{11} & \cellcolor{blue!5} \textbf{11} & \cellcolor{blue!5} 11 & \cellcolor{blue!5} 11 & \cellcolor{blue!5} 11 & \cellcolor{blue!5} 11 \\
50 225& 12 & \cellcolor{blue!5} \textbf{12} & \cellcolor{blue!5} \textbf{12} & \cellcolor{blue!5} \textbf{12} & \cellcolor{blue!5} 12 & \cellcolor{blue!5} 12 & \cellcolor{blue!5} \textbf{12} & \cellcolor{blue!5} 12 \\
50 226& 7 & \cellcolor{blue!5} \textbf{7} & \cellcolor{blue!5} \textbf{7} & \cellcolor{blue!5} \textbf{7} & \cellcolor{blue!5} 7 & \cellcolor{blue!5} \textbf{7} & \cellcolor{blue!5} \textbf{7} & \cellcolor{blue!5} \textbf{7} \\
50 227& 6 & \cellcolor{blue!5} \textbf{6} & \cellcolor{blue!5} \textbf{6} & \cellcolor{blue!5} \textbf{6} & \cellcolor{blue!5} \textbf{6} & \cellcolor{blue!5} \textbf{6} & \cellcolor{blue!5} \textbf{6} & \cellcolor{blue!5} \textbf{6} \\
50 228& 6 & \cellcolor{blue!5} \textbf{6} & \cellcolor{blue!5} \textbf{6} & \cellcolor{blue!5} \textbf{6} & \cellcolor{blue!5} \textbf{6} & \cellcolor{blue!5} \textbf{6} & \cellcolor{blue!5} \textbf{6} & \cellcolor{blue!5} \textbf{6} \\
50 229& 6 & \cellcolor{blue!5} \textbf{6} & \cellcolor{blue!5} \textbf{6} & \cellcolor{blue!5} \textbf{6} & \cellcolor{blue!5} \textbf{6} & \cellcolor{blue!5} \textbf{6} & \cellcolor{blue!5} \textbf{6} & \cellcolor{blue!5} \textbf{6} \\
50 23& 7 & \cellcolor{blue!5} \textbf{7} & \cellcolor{blue!5} \textbf{7} & \cellcolor{blue!5} \textbf{7} & \cellcolor{blue!5} 7 & \cellcolor{blue!5} 7 & \cellcolor{blue!5} \textbf{7} & \cellcolor{blue!5} 7 \\
50 230& 7 & \cellcolor{blue!5} \textbf{7} & \cellcolor{blue!5} \textbf{7} & \cellcolor{blue!5} \textbf{7} & \cellcolor{blue!5} 7 & \cellcolor{blue!5} \textbf{7} & \cellcolor{blue!5} \textbf{7} & \cellcolor{blue!5} \textbf{7} \\
50 231& 7 & \cellcolor{blue!5} \textbf{7} & \cellcolor{blue!5} \textbf{7} & \cellcolor{blue!5} \textbf{7} & \cellcolor{blue!5} 7 & \cellcolor{blue!5} \textbf{7} & \cellcolor{blue!5} \textbf{7} & \cellcolor{blue!5} \textbf{7} \\
50 232& 7 & \cellcolor{blue!5} \textbf{7} & \cellcolor{blue!5} \textbf{7} & \cellcolor{blue!5} \textbf{7} & 8 & \cellcolor{blue!5} \textbf{7} & \cellcolor{blue!5} \textbf{7} & \cellcolor{blue!5} 7 \\
50 233& 6 & \cellcolor{blue!5} \textbf{6} & \cellcolor{blue!5} \textbf{6} & \cellcolor{blue!5} \textbf{6} & \cellcolor{blue!5} \textbf{6} & \cellcolor{blue!5} \textbf{6} & \cellcolor{blue!5} \textbf{6} & \cellcolor{blue!5} \textbf{6} \\
50 234& 8 & \cellcolor{blue!5} \textbf{8} & \cellcolor{blue!5} \textbf{8} & \cellcolor{blue!5} \textbf{8} & \cellcolor{blue!5} 8 & \cellcolor{blue!5} \textbf{8} & \cellcolor{blue!5} \textbf{8} & \cellcolor{blue!5} \textbf{8} \\
50 235& 7 & \cellcolor{blue!5} \textbf{7} & \cellcolor{blue!5} \textbf{7} & \cellcolor{blue!5} \textbf{7} & \cellcolor{blue!5} 7 & \cellcolor{blue!5} \textbf{7} & \cellcolor{blue!5} \textbf{7} & \cellcolor{blue!5} \textbf{7} \\
50 236& 7 & \cellcolor{blue!5} \textbf{7} & \cellcolor{blue!5} \textbf{7} & \cellcolor{blue!5} \textbf{7} & 8 & \cellcolor{blue!5} \textbf{7} & \cellcolor{blue!5} \textbf{7} & \cellcolor{blue!5} 7 \\
50 237& 8 & \cellcolor{blue!5} \textbf{8} & \cellcolor{blue!5} \textbf{8} & \cellcolor{blue!5} \textbf{8} & \cellcolor{blue!5} 8 & \cellcolor{blue!5} \textbf{8} & \cellcolor{blue!5} \textbf{8} & \cellcolor{blue!5} 8 \\
50 238& 7 & \cellcolor{blue!5} \textbf{7} & \cellcolor{blue!5} \textbf{7} & \cellcolor{blue!5} \textbf{7} & \cellcolor{blue!5} 7 & \cellcolor{blue!5} \textbf{7} & \cellcolor{blue!5} \textbf{7} & \cellcolor{blue!5} \textbf{7} \\
50 239& 7 & \cellcolor{blue!5} \textbf{7} & \cellcolor{blue!5} \textbf{7} & \cellcolor{blue!5} \textbf{7} & \cellcolor{blue!5} 7 & \cellcolor{blue!5} \textbf{7} & \cellcolor{blue!5} \textbf{7} & \cellcolor{blue!5} 7 \\
50 24& 7 & \cellcolor{blue!5} \textbf{7} & \cellcolor{blue!5} \textbf{7} & \cellcolor{blue!5} \textbf{7} & \cellcolor{blue!5} 7 & \cellcolor{blue!5} 7 & \cellcolor{blue!5} \textbf{7} & \cellcolor{blue!5} \textbf{7} \\
50 240& 7 & \cellcolor{blue!5} \textbf{7} & \cellcolor{blue!5} \textbf{7} & \cellcolor{blue!5} \textbf{7} & \cellcolor{blue!5} 7 & \cellcolor{blue!5} \textbf{7} & \cellcolor{blue!5} \textbf{7} & \cellcolor{blue!5} 7 \\
50 241& 7 & \cellcolor{blue!5} \textbf{7} & \cellcolor{blue!5} \textbf{7} & \cellcolor{blue!5} \textbf{7} & \cellcolor{blue!5} 7 & \cellcolor{blue!5} \textbf{7} & \cellcolor{blue!5} \textbf{7} & \cellcolor{blue!5} 7 \\
50 242& 8 & \cellcolor{blue!5} \textbf{8} & \cellcolor{blue!5} \textbf{8} & \cellcolor{blue!5} \textbf{8} & \cellcolor{blue!5} 8 & \cellcolor{blue!5} \textbf{8} & \cellcolor{blue!5} \textbf{8} & \cellcolor{blue!5} 8 \\
50 243& 7 & \cellcolor{blue!5} \textbf{7} & \cellcolor{blue!5} \textbf{7} & \cellcolor{blue!5} \textbf{7} & \cellcolor{blue!5} 7 & \cellcolor{blue!5} \textbf{7} & \cellcolor{blue!5} \textbf{7} & \cellcolor{blue!5} \textbf{7} \\
50 244& 7 & \cellcolor{blue!5} \textbf{7} & \cellcolor{blue!5} \textbf{7} & \cellcolor{blue!5} \textbf{7} & \cellcolor{blue!5} \textbf{7} & \cellcolor{blue!5} \textbf{7} & \cellcolor{blue!5} \textbf{7} & \cellcolor{blue!5} \textbf{7} \\
50 245& 7 & \cellcolor{blue!5} \textbf{7} & \cellcolor{blue!5} \textbf{7} & \cellcolor{blue!5} \textbf{7} & \cellcolor{blue!5} 7 & \cellcolor{blue!5} \textbf{7} & \cellcolor{blue!5} \textbf{7} & \cellcolor{blue!5} 7 \\
50 246& 8 & \cellcolor{blue!5} \textbf{8} & \cellcolor{blue!5} \textbf{8} & \cellcolor{blue!5} \textbf{8} & \cellcolor{blue!5} 8 & \cellcolor{blue!5} \textbf{8} & \cellcolor{blue!5} \textbf{8} & \cellcolor{blue!5} 8 \\
50 247& 7 & \cellcolor{blue!5} \textbf{7} & \cellcolor{blue!5} \textbf{7} & \cellcolor{blue!5} \textbf{7} & \cellcolor{blue!5} 7 & \cellcolor{blue!5} \textbf{7} & \cellcolor{blue!5} \textbf{7} & \cellcolor{blue!5} \textbf{7} \\
50 248& 7 & \cellcolor{blue!5} \textbf{7} & \cellcolor{blue!5} \textbf{7} & \cellcolor{blue!5} \textbf{7} & \cellcolor{blue!5} 7 & \cellcolor{blue!5} \textbf{7} & \cellcolor{blue!5} \textbf{7} & \cellcolor{blue!5} 7 \\
50 249& 7 & \cellcolor{blue!5} \textbf{7} & \cellcolor{blue!5} \textbf{7} & \cellcolor{blue!5} \textbf{7} & \cellcolor{blue!5} 7 & \cellcolor{blue!5} \textbf{7} & \cellcolor{blue!5} \textbf{7} & \cellcolor{blue!5} 7 \\
50 25& 6 & \cellcolor{blue!5} \textbf{6} & \cellcolor{blue!5} \textbf{6} & \cellcolor{blue!5} \textbf{6} & \cellcolor{blue!5} 6 & \cellcolor{blue!5} 6 & \cellcolor{blue!5} \textbf{6} & \cellcolor{blue!5} \textbf{6} \\
50 250& 7 & \cellcolor{blue!5} \textbf{7} & \cellcolor{blue!5} \textbf{7} & \cellcolor{blue!5} \textbf{7} & \cellcolor{blue!5} 7 & \cellcolor{blue!5} \textbf{7} & \cellcolor{blue!5} \textbf{7} & \cellcolor{blue!5} \textbf{7} \\
50 251& 26 & 28 & \cellcolor{blue!5} 27 & \cellcolor{blue!5} \textbf{27} & 29 & \cellcolor{blue!5} \textbf{27} & \cellcolor{blue!5} 27 & \cellcolor{blue!5} 27 \\
50 252& 31 & 33 & \cellcolor{blue!5} 32 & \cellcolor{blue!5} \textbf{32} & 35 & \cellcolor{blue!5} \textbf{32} & \cellcolor{blue!5} 32 & \cellcolor{blue!5} 32 \\
50 253& 27 & 29 & \cellcolor{blue!5} 28 & \cellcolor{blue!5} \textbf{28} & 31 & \cellcolor{blue!5} \textbf{28} & \cellcolor{blue!5} 28 & 29 \\
50 254& 30 & \cellcolor{blue!5} \textbf{30} & \cellcolor{blue!5} 30 & \cellcolor{blue!5} \textbf{30} & 33 & \cellcolor{blue!5} \textbf{30} & \cellcolor{blue!5} 30 & \cellcolor{blue!5} 30 \\
50 255& 27 & 31 & \cellcolor{blue!5} 29 & \cellcolor{blue!5} \textbf{29} & 32 & \cellcolor{blue!5} \textbf{29} & \cellcolor{blue!5} 29 & 30 \\
50 256& 30 & \cellcolor{blue!5} \textbf{30} & \cellcolor{blue!5} 30 & \cellcolor{blue!5} \textbf{30} & 32 & \cellcolor{blue!5} \textbf{30} & \cellcolor{blue!5} 30 & 31 \\
50 257& 31 & \cellcolor{blue!5} 33 & \cellcolor{blue!5} 33 & \cellcolor{blue!5} \textbf{33} & 35 & \cellcolor{blue!5} \textbf{33} & \cellcolor{blue!5} 33 & \cellcolor{blue!5} 33 \\
50 258& 26 & \cellcolor{blue!5} 28 & \cellcolor{blue!5} 28 & \cellcolor{blue!5} \textbf{28} & 30 & \cellcolor{blue!5} \textbf{28} & \cellcolor{blue!5} 28 & \cellcolor{blue!5} 28 \\
50 259& 29 & \cellcolor{blue!5} 31 & \cellcolor{blue!5} 31 & \cellcolor{blue!5} \textbf{31} & 32 & \cellcolor{blue!5} \textbf{31} & \cellcolor{blue!5} 31 & \cellcolor{blue!5} 31 \\
50 26& 27 & \cellcolor{blue!5} \textbf{27} & \cellcolor{blue!5} 27 & \cellcolor{blue!5} \textbf{27} & 30 & \cellcolor{blue!5} 27 & \cellcolor{blue!5} 27 & \cellcolor{blue!5} 27 \\
50 260& 28 & \cellcolor{blue!5} 29 & \cellcolor{blue!5} 29 & \cellcolor{blue!5} \textbf{29} & 30 & \cellcolor{blue!5} \textbf{29} & \cellcolor{blue!5} 29 & \cellcolor{blue!5} 29 \\
50 261& 25 & \cellcolor{blue!5} 28 & \cellcolor{blue!5} 28 & \cellcolor{blue!5} \textbf{28} & 30 & \cellcolor{blue!5} \textbf{28} & \cellcolor{blue!5} 28 & \cellcolor{blue!5} 28 \\
50 262& 30 & \cellcolor{blue!5} 31 & \cellcolor{blue!5} 31 & \cellcolor{blue!5} \textbf{31} & \cellcolor{blue!5} 31 & \cellcolor{blue!5} \textbf{31} & \cellcolor{blue!5} 31 & \cellcolor{blue!5} 31 \\
50 263& 28 & 31 & \cellcolor{blue!10} 29 & \cellcolor{blue!10} \textbf{29} & 31 & \cellcolor{blue!10} \textbf{29} & 30 & 30 \\
50 264& 27 & \cellcolor{blue!5} \textbf{27} & \cellcolor{blue!5} 27 & \cellcolor{blue!5} \textbf{27} & 31 & \cellcolor{blue!5} \textbf{27} & \cellcolor{blue!5} 27 & 28 \\
50 265& 27 & \cellcolor{blue!5} \textbf{27} & \cellcolor{blue!5} 27 & \cellcolor{blue!5} \textbf{27} & 30 & \cellcolor{blue!5} \textbf{27} & \cellcolor{blue!5} 27 & \cellcolor{blue!5} 27 \\
50 266& 29 & \cellcolor{blue!5} \textbf{29} & \cellcolor{blue!5} \textbf{29} & \cellcolor{blue!5} \textbf{29} & 31 & \cellcolor{blue!5} \textbf{29} & \cellcolor{blue!5} \textbf{29} & \cellcolor{blue!5} 29 \\
50 267& 27 & 29 & \cellcolor{blue!5} 28 & \cellcolor{blue!5} \textbf{28} & 30 & \cellcolor{blue!5} \textbf{28} & \cellcolor{blue!5} 28 & 29 \\
50 268& 28 & \cellcolor{blue!5} 29 & \cellcolor{blue!5} 29 & \cellcolor{blue!5} \textbf{29} & 31 & \cellcolor{blue!5} \textbf{29} & \cellcolor{blue!5} 29 & \cellcolor{blue!5} 29 \\
50 269& 26 & \cellcolor{blue!5} \textbf{26} & \cellcolor{blue!5} \textbf{26} & \cellcolor{blue!5} \textbf{26} & 29 & \cellcolor{blue!5} \textbf{26} & \cellcolor{blue!5} 26 & 27 \\
50 27& 30 & \cellcolor{blue!5} \textbf{30} & \cellcolor{blue!5} 30 & \cellcolor{blue!5} \textbf{30} & 35 & \cellcolor{blue!5} 30 & \cellcolor{blue!5} 30 & \cellcolor{blue!5} 30 \\
50 270& 28 & \cellcolor{blue!5} \textbf{28} & \cellcolor{blue!5} 28 & \cellcolor{blue!5} \textbf{28} & 29 & \cellcolor{blue!5} \textbf{28} & \cellcolor{blue!5} 28 & \cellcolor{blue!5} 28 \\
50 271& 28 & \cellcolor{blue!5} 31 & \cellcolor{blue!5} 31 & \cellcolor{blue!5} \textbf{31} & 33 & \cellcolor{blue!5} \textbf{31} & \cellcolor{blue!5} 31 & 32 \\
50 272& 27 & \cellcolor{blue!5} \textbf{27} & \cellcolor{blue!5} 27 & \cellcolor{blue!5} \textbf{27} & 30 & \cellcolor{blue!5} \textbf{27} & \cellcolor{blue!5} 27 & \cellcolor{blue!5} 27 \\
50 273& 25 & 28 & \cellcolor{blue!5} \textbf{27} & \cellcolor{blue!5} \textbf{27} & 30 & \cellcolor{blue!5} \textbf{27} & \cellcolor{blue!5} \textbf{27} & 28 \\
50 274& 29 & 30 & \cellcolor{blue!5} 29 & \cellcolor{blue!5} \textbf{29} & 32 & \cellcolor{blue!5} \textbf{29} & \cellcolor{blue!5} 29 & 30 \\
50 275& 27 & \cellcolor{blue!5} \textbf{27} & \cellcolor{blue!5} \textbf{27} & \cellcolor{blue!5} \textbf{27} & 29 & \cellcolor{blue!5} \textbf{27} & \cellcolor{blue!5} 27 & 28 \\
50 276& 12 & \cellcolor{blue!5} \textbf{12} & \cellcolor{blue!5} \textbf{12} & \cellcolor{blue!5} \textbf{12} & 13 & \cellcolor{blue!5} \textbf{12} & \cellcolor{blue!5} \textbf{12} & \cellcolor{blue!5} 12 \\
50 277& 13 & \cellcolor{blue!5} \textbf{13} & \cellcolor{blue!5} \textbf{13} & \cellcolor{blue!5} \textbf{13} & \cellcolor{blue!5} 13 & \cellcolor{blue!5} \textbf{13} & \cellcolor{blue!5} \textbf{13} & \cellcolor{blue!5} \textbf{13} \\
50 278& 12 & \cellcolor{blue!5} \textbf{12} & \cellcolor{blue!5} \textbf{12} & \cellcolor{blue!5} \textbf{12} & 13 & \cellcolor{blue!5} \textbf{12} & \cellcolor{blue!5} \textbf{12} & \cellcolor{blue!5} 12 \\
50 279& 11 & \cellcolor{blue!5} \textbf{11} & \cellcolor{blue!5} \textbf{11} & \cellcolor{blue!5} \textbf{11} & \cellcolor{blue!5} 11 & \cellcolor{blue!5} \textbf{11} & \cellcolor{blue!5} \textbf{11} & \cellcolor{blue!5} 11 \\
50 28& 28 & 29 & \cellcolor{blue!5} 28 & \cellcolor{blue!5} \textbf{28} & 34 & \cellcolor{blue!5} 28 & \cellcolor{blue!5} 28 & \cellcolor{blue!5} 28 \\
50 280& 13 & \cellcolor{blue!5} \textbf{13} & \cellcolor{blue!5} \textbf{13} & \cellcolor{blue!5} \textbf{13} & \cellcolor{blue!5} 13 & \cellcolor{blue!5} \textbf{13} & \cellcolor{blue!5} \textbf{13} & \cellcolor{blue!5} 13 \\
50 281& 11 & \cellcolor{blue!5} \textbf{11} & \cellcolor{blue!5} \textbf{11} & \cellcolor{blue!5} \textbf{11} & \cellcolor{blue!5} 11 & \cellcolor{blue!5} \textbf{11} & \cellcolor{blue!5} \textbf{11} & \cellcolor{blue!5} \textbf{11} \\
50 282& 12 & \cellcolor{blue!5} \textbf{12} & \cellcolor{blue!5} \textbf{12} & \cellcolor{blue!5} \textbf{12} & \cellcolor{blue!5} 12 & \cellcolor{blue!5} \textbf{12} & \cellcolor{blue!5} \textbf{12} & \cellcolor{blue!5} 12 \\
50 283& 12 & \cellcolor{blue!5} \textbf{12} & \cellcolor{blue!5} \textbf{12} & \cellcolor{blue!5} \textbf{12} & 13 & \cellcolor{blue!5} \textbf{12} & \cellcolor{blue!5} \textbf{12} & \cellcolor{blue!5} 12 \\
50 284& 11 & \cellcolor{blue!5} \textbf{11} & \cellcolor{blue!5} \textbf{11} & \cellcolor{blue!5} \textbf{11} & \cellcolor{blue!5} 11 & \cellcolor{blue!5} \textbf{11} & \cellcolor{blue!5} \textbf{11} & \cellcolor{blue!5} \textbf{11} \\
50 285& 13 & \cellcolor{blue!5} \textbf{13} & \cellcolor{blue!5} \textbf{13} & \cellcolor{blue!5} \textbf{13} & 14 & \cellcolor{blue!5} \textbf{13} & \cellcolor{blue!5} \textbf{13} & \cellcolor{blue!5} \textbf{13} \\
50 286& 11 & \cellcolor{blue!5} \textbf{11} & \cellcolor{blue!5} \textbf{11} & \cellcolor{blue!5} \textbf{11} & 12 & \cellcolor{blue!5} \textbf{11} & \cellcolor{blue!5} \textbf{11} & \cellcolor{blue!5} 11 \\
50 287& 12 & \cellcolor{blue!5} \textbf{12} & \cellcolor{blue!5} \textbf{12} & \cellcolor{blue!5} \textbf{12} & 13 & \cellcolor{blue!5} \textbf{12} & \cellcolor{blue!5} \textbf{12} & \cellcolor{blue!5} 12 \\
50 288& 10 & \cellcolor{blue!5} \textbf{10} & \cellcolor{blue!5} \textbf{10} & \cellcolor{blue!5} \textbf{10} & 11 & \cellcolor{blue!5} \textbf{10} & \cellcolor{blue!5} \textbf{10} & \cellcolor{blue!5} \textbf{10} \\
50 289& 11 & \cellcolor{blue!5} \textbf{11} & \cellcolor{blue!5} \textbf{11} & \cellcolor{blue!5} \textbf{11} & 12 & \cellcolor{blue!5} \textbf{11} & \cellcolor{blue!5} \textbf{11} & \cellcolor{blue!5} 11 \\
50 29& 29 & \cellcolor{blue!5} \textbf{29} & \cellcolor{blue!5} 29 & \cellcolor{blue!5} \textbf{29} & 32 & \cellcolor{blue!5} 29 & \cellcolor{blue!5} 29 & \cellcolor{blue!5} 29 \\
50 290& 14 & \cellcolor{blue!5} \textbf{14} & \cellcolor{blue!5} \textbf{14} & \cellcolor{blue!5} \textbf{14} & \cellcolor{blue!5} 14 & \cellcolor{blue!5} \textbf{14} & \cellcolor{blue!5} \textbf{14} & \cellcolor{blue!5} 14 \\
50 291& 12 & \cellcolor{blue!5} \textbf{12} & \cellcolor{blue!5} \textbf{12} & \cellcolor{blue!5} \textbf{12} & \cellcolor{blue!5} 12 & \cellcolor{blue!5} \textbf{12} & \cellcolor{blue!5} \textbf{12} & \cellcolor{blue!5} 12 \\
50 292& 13 & \cellcolor{blue!5} \textbf{13} & \cellcolor{blue!5} \textbf{13} & \cellcolor{blue!5} \textbf{13} & \cellcolor{blue!5} 13 & \cellcolor{blue!5} \textbf{13} & \cellcolor{blue!5} \textbf{13} & \cellcolor{blue!5} 13 \\
50 293& 12 & \cellcolor{blue!5} \textbf{12} & \cellcolor{blue!5} \textbf{12} & \cellcolor{blue!5} \textbf{12} & \cellcolor{blue!5} 12 & \cellcolor{blue!5} \textbf{12} & \cellcolor{blue!5} \textbf{12} & \cellcolor{blue!5} 12 \\
50 294& 13 & \cellcolor{blue!5} \textbf{13} & \cellcolor{blue!5} \textbf{13} & \cellcolor{blue!5} \textbf{13} & \cellcolor{blue!5} 13 & \cellcolor{blue!5} \textbf{13} & \cellcolor{blue!5} \textbf{13} & \cellcolor{blue!5} 13 \\
50 295& 16 & \cellcolor{blue!5} \textbf{16} & \cellcolor{blue!5} \textbf{16} & \cellcolor{blue!5} \textbf{16} & 17 & \cellcolor{blue!5} \textbf{16} & \cellcolor{blue!5} \textbf{16} & \cellcolor{blue!5} 16 \\
50 296& 13 & \cellcolor{blue!5} \textbf{13} & \cellcolor{blue!5} \textbf{13} & \cellcolor{blue!5} \textbf{13} & \cellcolor{blue!5} 13 & \cellcolor{blue!5} 13 & \cellcolor{blue!5} \textbf{13} & \cellcolor{blue!5} 13 \\
50 297& 13 & \cellcolor{blue!5} \textbf{13} & \cellcolor{blue!5} \textbf{13} & \cellcolor{blue!5} \textbf{13} & \cellcolor{blue!5} 13 & \cellcolor{blue!5} \textbf{13} & \cellcolor{blue!5} \textbf{13} & \cellcolor{blue!5} 13 \\
50 298& 11 & \cellcolor{blue!5} \textbf{11} & \cellcolor{blue!5} \textbf{11} & \cellcolor{blue!5} \textbf{11} & \cellcolor{blue!5} 11 & \cellcolor{blue!5} \textbf{11} & \cellcolor{blue!5} \textbf{11} & \cellcolor{blue!5} 11 \\
50 299& 12 & \cellcolor{blue!5} \textbf{12} & \cellcolor{blue!5} \textbf{12} & \cellcolor{blue!5} \textbf{12} & \cellcolor{blue!5} 12 & \cellcolor{blue!5} \textbf{12} & \cellcolor{blue!5} \textbf{12} & \cellcolor{blue!5} 12 \\
50 3& 8 & \cellcolor{blue!5} \textbf{8} & \cellcolor{blue!5} \textbf{8} & \cellcolor{blue!5} \textbf{8} & \cellcolor{blue!5} 8 & \cellcolor{blue!5} 8 & \cellcolor{blue!5} \textbf{8} & \cellcolor{blue!5} 8 \\
50 30& 26 & 28 & 27 & \cellcolor{blue!40} \textbf{26} & 30 & 27 & 27 & 27 \\
50 300& 12 & \cellcolor{blue!5} \textbf{12} & \cellcolor{blue!5} \textbf{12} & \cellcolor{blue!5} \textbf{12} & \cellcolor{blue!5} 12 & \cellcolor{blue!5} \textbf{12} & \cellcolor{blue!5} \textbf{12} & \cellcolor{blue!5} \textbf{12} \\
50 301& 6 & \cellcolor{blue!5} \textbf{6} & \cellcolor{blue!5} \textbf{6} & \cellcolor{blue!5} \textbf{6} & 7 & \cellcolor{blue!5} 6 & \cellcolor{blue!5} \textbf{6} & \cellcolor{blue!5} \textbf{6} \\
50 302& 7 & \cellcolor{blue!5} \textbf{7} & \cellcolor{blue!5} \textbf{7} & \cellcolor{blue!5} \textbf{7} & \cellcolor{blue!5} 7 & \cellcolor{blue!5} 7 & \cellcolor{blue!5} \textbf{7} & \cellcolor{blue!5} \textbf{7} \\
50 303& 8 & \cellcolor{blue!5} \textbf{8} & \cellcolor{blue!5} \textbf{8} & \cellcolor{blue!5} \textbf{8} & \cellcolor{blue!5} 8 & \cellcolor{blue!5} 8 & \cellcolor{blue!5} \textbf{8} & \cellcolor{blue!5} 8 \\
50 304& 7 & \cellcolor{blue!5} \textbf{7} & \cellcolor{blue!5} \textbf{7} & \cellcolor{blue!5} \textbf{7} & \cellcolor{blue!5} 7 & \cellcolor{blue!5} 7 & \cellcolor{blue!5} \textbf{7} & \cellcolor{blue!5} \textbf{7} \\
50 305& 8 & \cellcolor{blue!5} \textbf{8} & \cellcolor{blue!5} \textbf{8} & \cellcolor{blue!5} \textbf{8} & \cellcolor{blue!5} 8 & \cellcolor{blue!5} 8 & \cellcolor{blue!5} \textbf{8} & \cellcolor{blue!5} 8 \\
50 306& 7 & \cellcolor{blue!5} \textbf{7} & \cellcolor{blue!5} \textbf{7} & \cellcolor{blue!5} \textbf{7} & \cellcolor{blue!5} 7 & \cellcolor{blue!5} 7 & \cellcolor{blue!5} \textbf{7} & \cellcolor{blue!5} \textbf{7} \\
50 307& 7 & \cellcolor{blue!5} \textbf{7} & \cellcolor{blue!5} \textbf{7} & \cellcolor{blue!5} \textbf{7} & \cellcolor{blue!5} 7 & \cellcolor{blue!5} 7 & \cellcolor{blue!5} \textbf{7} & \cellcolor{blue!5} 7 \\
50 308& 8 & \cellcolor{blue!5} \textbf{8} & \cellcolor{blue!5} \textbf{8} & \cellcolor{blue!5} \textbf{8} & \cellcolor{blue!5} 8 & \cellcolor{blue!5} 8 & \cellcolor{blue!5} \textbf{8} & \cellcolor{blue!5} \textbf{8} \\
50 309& 7 & \cellcolor{blue!5} \textbf{7} & \cellcolor{blue!5} \textbf{7} & \cellcolor{blue!5} \textbf{7} & 8 & 29 & \cellcolor{blue!5} \textbf{7} & \cellcolor{blue!5} 7 \\
50 31& 27 & \cellcolor{blue!5} 28 & \cellcolor{blue!5} 28 & \cellcolor{blue!5} 28 & 31 & \cellcolor{blue!5} 28 & \cellcolor{blue!5} 28 & \cellcolor{blue!5} 28 \\
50 310& 8 & \cellcolor{blue!5} \textbf{8} & \cellcolor{blue!5} \textbf{8} & \cellcolor{blue!5} \textbf{8} & \cellcolor{blue!5} 8 & \cellcolor{blue!5} 8 & \cellcolor{blue!5} \textbf{8} & \cellcolor{blue!5} 8 \\
50 311& 8 & \cellcolor{blue!5} \textbf{8} & \cellcolor{blue!5} \textbf{8} & \cellcolor{blue!5} \textbf{8} & \cellcolor{blue!5} 8 & \cellcolor{blue!5} 8 & \cellcolor{blue!5} \textbf{8} & \cellcolor{blue!5} 8 \\
50 312& 6 & \cellcolor{blue!5} \textbf{6} & \cellcolor{blue!5} \textbf{6} & \cellcolor{blue!5} \textbf{6} & 7 & \cellcolor{blue!5} 6 & \cellcolor{blue!5} \textbf{6} & \cellcolor{blue!5} \textbf{6} \\
50 313& 8 & \cellcolor{blue!5} \textbf{8} & \cellcolor{blue!5} \textbf{8} & \cellcolor{blue!5} \textbf{8} & \cellcolor{blue!5} 8 & \cellcolor{blue!5} 8 & \cellcolor{blue!5} \textbf{8} & \cellcolor{blue!5} 8 \\
50 314& 7 & \cellcolor{blue!5} \textbf{7} & \cellcolor{blue!5} \textbf{7} & \cellcolor{blue!5} \textbf{7} & \cellcolor{blue!5} 7 & 22 & \cellcolor{blue!5} \textbf{7} & \cellcolor{blue!5} \textbf{7} \\
50 315& 8 & \cellcolor{blue!5} \textbf{8} & \cellcolor{blue!5} \textbf{8} & \cellcolor{blue!5} \textbf{8} & \cellcolor{blue!5} 8 & \cellcolor{blue!5} 8 & \cellcolor{blue!5} \textbf{8} & \cellcolor{blue!5} 8 \\
50 316& 8 & \cellcolor{blue!5} \textbf{8} & \cellcolor{blue!5} \textbf{8} & \cellcolor{blue!5} \textbf{8} & \cellcolor{blue!5} 8 & \cellcolor{blue!5} 8 & \cellcolor{blue!5} \textbf{8} & \cellcolor{blue!5} 8 \\
50 317& 6 & \cellcolor{blue!5} \textbf{6} & \cellcolor{blue!5} \textbf{6} & \cellcolor{blue!5} \textbf{6} & \cellcolor{blue!5} 6 & \cellcolor{blue!5} 6 & \cellcolor{blue!5} \textbf{6} & \cellcolor{blue!5} \textbf{6} \\
50 318& 8 & \cellcolor{blue!5} \textbf{8} & \cellcolor{blue!5} \textbf{8} & \cellcolor{blue!5} \textbf{8} & \cellcolor{blue!5} 8 & \cellcolor{blue!5} 8 & \cellcolor{blue!5} \textbf{8} & \cellcolor{blue!5} 8 \\
50 319& 7 & \cellcolor{blue!5} \textbf{7} & \cellcolor{blue!5} \textbf{7} & \cellcolor{blue!5} \textbf{7} & \cellcolor{blue!5} 7 & \cellcolor{blue!5} 7 & \cellcolor{blue!5} \textbf{7} & \cellcolor{blue!5} 7 \\
50 32& 25 & 26 & \cellcolor{blue!10} \textbf{25} & \cellcolor{blue!10} \textbf{25} & 31 & 26 & \cellcolor{blue!10} 25 & 26 \\
50 320& 8 & \cellcolor{blue!5} \textbf{8} & \cellcolor{blue!5} \textbf{8} & \cellcolor{blue!5} \textbf{8} & \cellcolor{blue!5} 8 & \cellcolor{blue!5} 8 & \cellcolor{blue!5} \textbf{8} & \cellcolor{blue!5} 8 \\
50 321& 6 & \cellcolor{blue!5} \textbf{6} & \cellcolor{blue!5} \textbf{6} & \cellcolor{blue!5} \textbf{6} & \cellcolor{blue!5} 6 & \cellcolor{blue!5} 6 & \cellcolor{blue!5} \textbf{6} & \cellcolor{blue!5} \textbf{6} \\
50 322& 7 & \cellcolor{blue!5} \textbf{7} & \cellcolor{blue!5} \textbf{7} & \cellcolor{blue!5} \textbf{7} & \cellcolor{blue!5} 7 & \cellcolor{blue!5} 7 & \cellcolor{blue!5} \textbf{7} & \cellcolor{blue!5} 7 \\
50 323& 7 & \cellcolor{blue!5} \textbf{7} & \cellcolor{blue!5} \textbf{7} & \cellcolor{blue!5} \textbf{7} & \cellcolor{blue!5} 7 & \cellcolor{blue!5} 7 & \cellcolor{blue!5} \textbf{7} & \cellcolor{blue!5} 7 \\
50 324& 7 & \cellcolor{blue!5} \textbf{7} & \cellcolor{blue!5} \textbf{7} & \cellcolor{blue!5} \textbf{7} & \cellcolor{blue!5} 7 & \cellcolor{blue!5} 7 & \cellcolor{blue!5} \textbf{7} & \cellcolor{blue!5} 7 \\
50 325& 7 & \cellcolor{blue!5} \textbf{7} & \cellcolor{blue!5} \textbf{7} & \cellcolor{blue!5} \textbf{7} & \cellcolor{blue!5} 7 & \cellcolor{blue!5} 7 & \cellcolor{blue!5} \textbf{7} & \cellcolor{blue!5} \textbf{7} \\
50 326& 33 & \cellcolor{blue!5} \textbf{33} & \cellcolor{blue!5} 33 & \cellcolor{blue!5} \textbf{33} & 36 & \cellcolor{blue!5} 33 & \cellcolor{blue!5} 33 & \cellcolor{blue!5} 33 \\
50 327& 28 & \cellcolor{blue!5} \textbf{28} & \cellcolor{blue!5} 28 & \cellcolor{blue!5} \textbf{28} & 31 & \cellcolor{blue!5} 28 & \cellcolor{blue!5} 28 & \cellcolor{blue!5} 28 \\
50 328& 32 & \cellcolor{blue!5} \textbf{32} & \cellcolor{blue!5} 32 & \cellcolor{blue!5} \textbf{32} & 34 & \cellcolor{blue!5} 32 & \cellcolor{blue!5} 32 & \cellcolor{blue!5} 32 \\
50 329& 24 & 25 & 25 & 25 & 30 & 25 & \cellcolor{blue!40} 24 & 25 \\
50 33& 24 & \cellcolor{blue!5} 25 & \cellcolor{blue!5} 25 & \cellcolor{blue!5} 25 & 28 & \cellcolor{blue!5} 25 & \cellcolor{blue!5} 25 & \cellcolor{blue!5} 25 \\
50 330& 29 & \cellcolor{blue!5} \textbf{29} & \cellcolor{blue!5} 29 & \cellcolor{blue!5} \textbf{29} & 33 & 30 & \cellcolor{blue!5} 29 & \cellcolor{blue!5} 29 \\
50 331& 29 & \cellcolor{blue!5} \textbf{29} & \cellcolor{blue!5} 29 & \cellcolor{blue!5} \textbf{29} & 36 & 40 & \cellcolor{blue!5} 29 & \cellcolor{blue!5} 29 \\
50 332& 24 & \cellcolor{blue!5} 25 & \cellcolor{blue!5} 25 & \cellcolor{blue!5} 25 & 30 & \cellcolor{blue!5} 25 & \cellcolor{blue!5} 25 & \cellcolor{blue!5} 25 \\
50 333& 28 & \cellcolor{blue!5} \textbf{28} & \cellcolor{blue!5} 28 & \cellcolor{blue!5} \textbf{28} & 32 & \cellcolor{blue!5} 28 & \cellcolor{blue!5} 28 & \cellcolor{blue!5} 28 \\
50 334& 29 & \cellcolor{blue!5} \textbf{29} & \cellcolor{blue!5} 29 & \cellcolor{blue!5} \textbf{29} & 32 & \cellcolor{blue!5} 29 & \cellcolor{blue!5} 29 & \cellcolor{blue!5} 29 \\
50 335& 27 & \cellcolor{blue!5} \textbf{27} & \cellcolor{blue!5} 27 & \cellcolor{blue!5} \textbf{27} & 33 & \cellcolor{blue!5} 27 & \cellcolor{blue!5} 27 & \cellcolor{blue!5} 27 \\
50 336& 25 & \cellcolor{blue!5} 26 & \cellcolor{blue!5} 26 & \cellcolor{blue!5} 26 & 31 & \cellcolor{blue!5} 26 & \cellcolor{blue!5} 26 & \cellcolor{blue!5} 26 \\
50 337& 26 & \cellcolor{blue!5} \textbf{26} & \cellcolor{blue!5} 26 & \cellcolor{blue!5} \textbf{26} & 31 & \cellcolor{blue!5} 26 & \cellcolor{blue!5} 26 & \cellcolor{blue!5} 26 \\
50 338& 26 & 27 & 27 & \cellcolor{blue!20} \textbf{26} & 34 & 36 & \cellcolor{blue!20} 26 & 27 \\
50 339& 27 & 28 & \cellcolor{blue!5} 27 & \cellcolor{blue!5} \textbf{27} & 32 & 29 & \cellcolor{blue!5} 27 & \cellcolor{blue!5} 27 \\
50 34& 30 & \cellcolor{blue!5} \textbf{30} & \cellcolor{blue!5} 30 & \cellcolor{blue!5} \textbf{30} & 32 & \cellcolor{blue!5} 30 & \cellcolor{blue!5} 30 & \cellcolor{blue!5} 30 \\
50 340& 27 & 29 & \cellcolor{blue!5} 28 & \cellcolor{blue!5} 28 & 33 & 32 & \cellcolor{blue!5} 28 & \cellcolor{blue!5} 28 \\
50 341& 27 & \cellcolor{blue!5} \textbf{27} & \cellcolor{blue!5} 27 & \cellcolor{blue!5} \textbf{27} & 33 & \cellcolor{blue!5} 27 & \cellcolor{blue!5} 27 & \cellcolor{blue!5} 27 \\
50 342& 27 & \cellcolor{blue!5} 28 & \cellcolor{blue!5} 28 & \cellcolor{blue!5} 28 & 33 & 29 & \cellcolor{blue!5} 28 & \cellcolor{blue!5} 28 \\
50 343& 26 & \cellcolor{blue!5} 27 & \cellcolor{blue!5} 27 & \cellcolor{blue!5} \textbf{27} & 31 & 28 & \cellcolor{blue!5} 27 & \cellcolor{blue!5} 27 \\
50 344& 30 & \cellcolor{blue!5} \textbf{30} & \cellcolor{blue!5} 30 & \cellcolor{blue!5} \textbf{30} & 33 & \cellcolor{blue!5} 30 & \cellcolor{blue!5} 30 & \cellcolor{blue!5} 30 \\
50 345& 29 & 30 & \cellcolor{blue!5} 29 & \cellcolor{blue!5} \textbf{29} & 35 & \cellcolor{blue!5} 29 & \cellcolor{blue!5} 29 & 30 \\
50 346& 27 & \cellcolor{blue!5} \textbf{27} & \cellcolor{blue!5} 27 & \cellcolor{blue!5} \textbf{27} & 30 & \cellcolor{blue!5} 27 & \cellcolor{blue!5} 27 & \cellcolor{blue!5} 27 \\
50 347& 25 & 26 & 26 & \cellcolor{blue!20} \textbf{25} & 33 & 31 & \cellcolor{blue!20} 25 & 26 \\
50 348& 30 & \cellcolor{blue!5} \textbf{30} & \cellcolor{blue!5} 30 & \cellcolor{blue!5} \textbf{30} & 33 & \cellcolor{blue!5} 30 & \cellcolor{blue!5} 30 & \cellcolor{blue!5} 30 \\
50 349& 28 & 29 & \cellcolor{blue!10} 28 & \cellcolor{blue!10} \textbf{28} & 33 & 29 & \cellcolor{blue!10} 28 & 29 \\
50 35& 31 & \cellcolor{blue!20} \textbf{31} & 32 & \cellcolor{blue!20} \textbf{31} & 33 & 32 & 32 & 33 \\
50 350& 23 & 25 & \cellcolor{blue!5} 24 & \cellcolor{blue!5} 24 & 28 & 32 & \cellcolor{blue!5} 24 & \cellcolor{blue!5} 24 \\
50 351& 12 & \cellcolor{blue!5} \textbf{12} & \cellcolor{blue!5} \textbf{12} & \cellcolor{blue!5} \textbf{12} & \cellcolor{blue!5} 12 & \cellcolor{blue!5} 12 & \cellcolor{blue!5} 12 & \cellcolor{blue!5} 12 \\
50 352& 10 & \cellcolor{blue!5} \textbf{10} & \cellcolor{blue!5} \textbf{10} & \cellcolor{blue!5} \textbf{10} & 11 & \cellcolor{blue!5} 10 & \cellcolor{blue!5} \textbf{10} & \cellcolor{blue!5} 10 \\
50 353& 13 & \cellcolor{blue!5} \textbf{13} & \cellcolor{blue!5} \textbf{13} & \cellcolor{blue!5} \textbf{13} & 14 & \cellcolor{blue!5} 13 & \cellcolor{blue!5} 13 & \cellcolor{blue!5} 13 \\
50 354& 13 & \cellcolor{blue!10} \textbf{13} & \cellcolor{blue!10} \textbf{13} & 14 & 14 & 14 & \cellcolor{blue!10} 13 & 14 \\
50 355& 11 & \cellcolor{blue!5} \textbf{11} & \cellcolor{blue!5} \textbf{11} & \cellcolor{blue!5} \textbf{11} & \cellcolor{blue!5} 11 & \cellcolor{blue!5} 11 & \cellcolor{blue!5} \textbf{11} & \cellcolor{blue!5} 11 \\
50 356& 15 & \cellcolor{blue!5} \textbf{15} & \cellcolor{blue!5} \textbf{15} & \cellcolor{blue!5} \textbf{15} & 16 & \cellcolor{blue!5} 15 & \cellcolor{blue!5} 15 & \cellcolor{blue!5} 15 \\
50 357& 12 & \cellcolor{blue!5} \textbf{12} & \cellcolor{blue!5} \textbf{12} & \cellcolor{blue!5} \textbf{12} & 13 & \cellcolor{blue!5} 12 & \cellcolor{blue!5} 12 & \cellcolor{blue!5} 12 \\
50 358& 11 & \cellcolor{blue!5} \textbf{11} & \cellcolor{blue!5} \textbf{11} & \cellcolor{blue!5} \textbf{11} & \cellcolor{blue!5} 11 & \cellcolor{blue!5} 11 & \cellcolor{blue!5} \textbf{11} & \cellcolor{blue!5} 11 \\
50 359& 10 & \cellcolor{blue!5} \textbf{10} & \cellcolor{blue!5} \textbf{10} & \cellcolor{blue!5} \textbf{10} & \cellcolor{blue!5} 10 & \cellcolor{blue!5} 10 & \cellcolor{blue!5} \textbf{10} & \cellcolor{blue!5} 10 \\
50 36& 31 & \cellcolor{blue!5} \textbf{31} & \cellcolor{blue!5} 31 & \cellcolor{blue!5} \textbf{31} & 35 & \cellcolor{blue!5} 31 & \cellcolor{blue!5} 31 & \cellcolor{blue!5} 31 \\
50 360& 12 & \cellcolor{blue!5} \textbf{12} & \cellcolor{blue!5} \textbf{12} & \cellcolor{blue!5} \textbf{12} & 13 & \cellcolor{blue!5} 12 & \cellcolor{blue!5} 12 & \cellcolor{blue!5} 12 \\
50 361& 11 & \cellcolor{blue!5} \textbf{11} & \cellcolor{blue!5} \textbf{11} & \cellcolor{blue!5} \textbf{11} & 12 & \cellcolor{blue!5} 11 & \cellcolor{blue!5} \textbf{11} & \cellcolor{blue!5} 11 \\
50 362& 10 & \cellcolor{blue!5} \textbf{10} & \cellcolor{blue!5} \textbf{10} & \cellcolor{blue!5} \textbf{10} & 11 & \cellcolor{blue!5} 10 & \cellcolor{blue!5} 10 & \cellcolor{blue!5} 10 \\
50 363& 11 & \cellcolor{blue!20} \textbf{11} & 12 & 12 & 12 & 12 & \cellcolor{blue!20} \textbf{11} & 12 \\
50 364& 13 & \cellcolor{blue!5} \textbf{13} & \cellcolor{blue!5} \textbf{13} & \cellcolor{blue!5} \textbf{13} & \cellcolor{blue!5} 13 & \cellcolor{blue!5} 13 & \cellcolor{blue!5} 13 & \cellcolor{blue!5} 13 \\
50 365& 11 & \cellcolor{blue!5} \textbf{11} & \cellcolor{blue!5} \textbf{11} & \cellcolor{blue!5} \textbf{11} & \cellcolor{blue!5} 11 & \cellcolor{blue!5} 11 & \cellcolor{blue!5} \textbf{11} & \cellcolor{blue!5} 11 \\
50 366& 13 & \cellcolor{blue!5} \textbf{13} & \cellcolor{blue!5} \textbf{13} & \cellcolor{blue!5} \textbf{13} & 14 & \cellcolor{blue!5} 13 & \cellcolor{blue!5} 13 & \cellcolor{blue!5} 13 \\
50 367& 12 & \cellcolor{blue!5} \textbf{12} & \cellcolor{blue!5} \textbf{12} & \cellcolor{blue!5} \textbf{12} & \cellcolor{blue!5} 12 & \cellcolor{blue!5} 12 & \cellcolor{blue!5} 12 & \cellcolor{blue!5} 12 \\
50 368& 12 & \cellcolor{blue!5} \textbf{12} & \cellcolor{blue!5} \textbf{12} & \cellcolor{blue!5} \textbf{12} & \cellcolor{blue!5} 12 & \cellcolor{blue!5} 12 & \cellcolor{blue!5} 12 & \cellcolor{blue!5} 12 \\
50 369& 12 & \cellcolor{blue!5} \textbf{12} & \cellcolor{blue!5} \textbf{12} & \cellcolor{blue!5} \textbf{12} & 13 & \cellcolor{blue!5} 12 & \cellcolor{blue!5} 12 & \cellcolor{blue!5} 12 \\
50 37& 32 & \cellcolor{blue!5} \textbf{32} & \cellcolor{blue!5} 32 & \cellcolor{blue!5} 32 & 36 & \cellcolor{blue!5} 32 & \cellcolor{blue!5} 32 & \cellcolor{blue!5} 32 \\
50 370& 12 & \cellcolor{blue!5} \textbf{12} & \cellcolor{blue!5} \textbf{12} & \cellcolor{blue!5} \textbf{12} & \cellcolor{blue!5} 12 & \cellcolor{blue!5} 12 & \cellcolor{blue!5} \textbf{12} & \cellcolor{blue!5} 12 \\
50 371& 11 & \cellcolor{blue!5} \textbf{11} & \cellcolor{blue!5} \textbf{11} & \cellcolor{blue!5} \textbf{11} & 12 & 12 & \cellcolor{blue!5} 11 & \cellcolor{blue!5} 11 \\
50 372& 10 & \cellcolor{blue!5} \textbf{10} & \cellcolor{blue!5} \textbf{10} & \cellcolor{blue!5} \textbf{10} & 11 & \cellcolor{blue!5} 10 & \cellcolor{blue!5} 10 & \cellcolor{blue!5} 10 \\
50 373& 12 & \cellcolor{blue!5} \textbf{12} & \cellcolor{blue!5} \textbf{12} & \cellcolor{blue!5} \textbf{12} & 13 & \cellcolor{blue!5} 12 & \cellcolor{blue!5} 12 & \cellcolor{blue!5} 12 \\
50 374& 11 & \cellcolor{blue!5} \textbf{11} & \cellcolor{blue!5} \textbf{11} & \cellcolor{blue!5} \textbf{11} & \cellcolor{blue!5} 11 & \cellcolor{blue!5} 11 & \cellcolor{blue!5} \textbf{11} & \cellcolor{blue!5} 11 \\
50 375& 13 & \cellcolor{blue!5} \textbf{13} & \cellcolor{blue!5} \textbf{13} & \cellcolor{blue!5} \textbf{13} & 14 & \cellcolor{blue!5} 13 & \cellcolor{blue!5} 13 & \cellcolor{blue!5} 13 \\
50 376& 7 & \cellcolor{blue!5} \textbf{7} & \cellcolor{blue!5} \textbf{7} & \cellcolor{blue!5} \textbf{7} & \cellcolor{blue!5} 7 & \cellcolor{blue!5} \textbf{7} & \cellcolor{blue!5} \textbf{7} & \cellcolor{blue!5} 7 \\
50 377& 7 & \cellcolor{blue!5} \textbf{7} & \cellcolor{blue!5} \textbf{7} & \cellcolor{blue!5} \textbf{7} & \cellcolor{blue!5} 7 & \cellcolor{blue!5} \textbf{7} & \cellcolor{blue!5} \textbf{7} & \cellcolor{blue!5} 7 \\
50 378& 8 & \cellcolor{blue!5} \textbf{8} & \cellcolor{blue!5} \textbf{8} & \cellcolor{blue!5} \textbf{8} & \cellcolor{blue!5} 8 & \cellcolor{blue!5} \textbf{8} & \cellcolor{blue!5} \textbf{8} & \cellcolor{blue!5} 8 \\
50 379& 7 & \cellcolor{blue!5} \textbf{7} & \cellcolor{blue!5} \textbf{7} & \cellcolor{blue!5} \textbf{7} & \cellcolor{blue!5} 7 & \cellcolor{blue!5} \textbf{7} & \cellcolor{blue!5} \textbf{7} & \cellcolor{blue!5} \textbf{7} \\
50 38& 31 & \cellcolor{blue!5} \textbf{31} & \cellcolor{blue!5} 31 & \cellcolor{blue!5} \textbf{31} & 35 & \cellcolor{blue!5} 31 & \cellcolor{blue!5} 31 & \cellcolor{blue!5} 31 \\
50 380& 7 & \cellcolor{blue!5} \textbf{7} & \cellcolor{blue!5} \textbf{7} & \cellcolor{blue!5} \textbf{7} & \cellcolor{blue!5} 7 & \cellcolor{blue!5} \textbf{7} & \cellcolor{blue!5} \textbf{7} & \cellcolor{blue!5} 7 \\
50 381& 8 & \cellcolor{blue!5} \textbf{8} & \cellcolor{blue!5} \textbf{8} & \cellcolor{blue!5} \textbf{8} & \cellcolor{blue!5} 8 & \cellcolor{blue!5} \textbf{8} & \cellcolor{blue!5} \textbf{8} & \cellcolor{blue!5} \textbf{8} \\
50 382& 6 & \cellcolor{blue!5} \textbf{6} & \cellcolor{blue!5} \textbf{6} & \cellcolor{blue!5} \textbf{6} & \cellcolor{blue!5} 6 & \cellcolor{blue!5} \textbf{6} & \cellcolor{blue!5} \textbf{6} & \cellcolor{blue!5} 6 \\
50 383& 7 & \cellcolor{blue!5} \textbf{7} & \cellcolor{blue!5} \textbf{7} & \cellcolor{blue!5} \textbf{7} & \cellcolor{blue!5} 7 & \cellcolor{blue!5} \textbf{7} & \cellcolor{blue!5} \textbf{7} & \cellcolor{blue!5} \textbf{7} \\
50 384& 8 & \cellcolor{blue!5} \textbf{8} & \cellcolor{blue!5} \textbf{8} & \cellcolor{blue!5} \textbf{8} & 9 & \cellcolor{blue!5} \textbf{8} & \cellcolor{blue!5} \textbf{8} & \cellcolor{blue!5} 8 \\
50 385& 7 & \cellcolor{blue!5} \textbf{7} & \cellcolor{blue!5} \textbf{7} & \cellcolor{blue!5} \textbf{7} & \cellcolor{blue!5} 7 & \cellcolor{blue!5} \textbf{7} & \cellcolor{blue!5} \textbf{7} & \cellcolor{blue!5} 7 \\
50 386& 7 & \cellcolor{blue!5} \textbf{7} & \cellcolor{blue!5} \textbf{7} & \cellcolor{blue!5} \textbf{7} & \cellcolor{blue!5} 7 & \cellcolor{blue!5} \textbf{7} & \cellcolor{blue!5} \textbf{7} & \cellcolor{blue!5} \textbf{7} \\
50 387& 8 & \cellcolor{blue!5} \textbf{8} & \cellcolor{blue!5} \textbf{8} & \cellcolor{blue!5} \textbf{8} & \cellcolor{blue!5} 8 & \cellcolor{blue!5} \textbf{8} & \cellcolor{blue!5} \textbf{8} & \cellcolor{blue!5} 8 \\
50 388& 7 & \cellcolor{blue!5} \textbf{7} & \cellcolor{blue!5} \textbf{7} & \cellcolor{blue!5} \textbf{7} & \cellcolor{blue!5} 7 & \cellcolor{blue!5} \textbf{7} & \cellcolor{blue!5} \textbf{7} & \cellcolor{blue!5} \textbf{7} \\
50 389& 8 & \cellcolor{blue!5} \textbf{8} & \cellcolor{blue!5} \textbf{8} & \cellcolor{blue!5} \textbf{8} & \cellcolor{blue!5} 8 & \cellcolor{blue!5} \textbf{8} & \cellcolor{blue!5} \textbf{8} & \cellcolor{blue!5} \textbf{8} \\
50 39& 29 & \cellcolor{blue!5} \textbf{29} & \cellcolor{blue!5} 29 & \cellcolor{blue!5} 29 & 35 & \cellcolor{blue!5} 29 & \cellcolor{blue!5} 29 & \cellcolor{blue!5} 29 \\
50 390& 7 & \cellcolor{blue!5} \textbf{7} & \cellcolor{blue!5} \textbf{7} & \cellcolor{blue!5} \textbf{7} & 8 & \cellcolor{blue!5} \textbf{7} & \cellcolor{blue!5} \textbf{7} & \cellcolor{blue!5} 7 \\
50 391& 7 & \cellcolor{blue!5} \textbf{7} & \cellcolor{blue!5} \textbf{7} & \cellcolor{blue!5} \textbf{7} & \cellcolor{blue!5} 7 & \cellcolor{blue!5} \textbf{7} & \cellcolor{blue!5} \textbf{7} & \cellcolor{blue!5} 7 \\
50 392& 8 & \cellcolor{blue!5} \textbf{8} & \cellcolor{blue!5} \textbf{8} & \cellcolor{blue!5} \textbf{8} & \cellcolor{blue!5} 8 & \cellcolor{blue!5} \textbf{8} & \cellcolor{blue!5} \textbf{8} & \cellcolor{blue!5} 8 \\
50 393& 7 & \cellcolor{blue!5} \textbf{7} & \cellcolor{blue!5} \textbf{7} & \cellcolor{blue!5} \textbf{7} & \cellcolor{blue!5} 7 & \cellcolor{blue!5} \textbf{7} & \cellcolor{blue!5} \textbf{7} & \cellcolor{blue!5} 7 \\
50 394& 8 & \cellcolor{blue!5} \textbf{8} & \cellcolor{blue!5} \textbf{8} & \cellcolor{blue!5} \textbf{8} & \cellcolor{blue!5} 8 & \cellcolor{blue!5} \textbf{8} & \cellcolor{blue!5} \textbf{8} & \cellcolor{blue!5} 8 \\
50 395& 7 & \cellcolor{blue!5} \textbf{7} & \cellcolor{blue!5} \textbf{7} & \cellcolor{blue!5} \textbf{7} & \cellcolor{blue!5} 7 & \cellcolor{blue!5} \textbf{7} & \cellcolor{blue!5} \textbf{7} & \cellcolor{blue!5} 7 \\
50 396& 8 & \cellcolor{blue!5} \textbf{8} & \cellcolor{blue!5} \textbf{8} & \cellcolor{blue!5} \textbf{8} & \cellcolor{blue!5} 8 & \cellcolor{blue!5} \textbf{8} & \cellcolor{blue!5} \textbf{8} & \cellcolor{blue!5} \textbf{8} \\
50 397& 7 & \cellcolor{blue!5} \textbf{7} & \cellcolor{blue!5} \textbf{7} & \cellcolor{blue!5} \textbf{7} & \cellcolor{blue!5} 7 & \cellcolor{blue!5} \textbf{7} & \cellcolor{blue!5} \textbf{7} & \cellcolor{blue!5} 7 \\
50 398& 6 & \cellcolor{blue!5} \textbf{6} & \cellcolor{blue!5} \textbf{6} & \cellcolor{blue!5} \textbf{6} & 7 & \cellcolor{blue!5} \textbf{6} & \cellcolor{blue!5} \textbf{6} & \cellcolor{blue!5} 6 \\
50 399& 7 & \cellcolor{blue!5} \textbf{7} & \cellcolor{blue!5} \textbf{7} & \cellcolor{blue!5} \textbf{7} & 8 & \cellcolor{blue!5} \textbf{7} & \cellcolor{blue!5} \textbf{7} & \cellcolor{blue!5} \textbf{7} \\
50 4& 7 & \cellcolor{blue!5} \textbf{7} & \cellcolor{blue!5} \textbf{7} & \cellcolor{blue!5} \textbf{7} & \cellcolor{blue!5} 7 & \cellcolor{blue!5} 7 & \cellcolor{blue!5} \textbf{7} & \cellcolor{blue!5} \textbf{7} \\
50 40& 26 & \cellcolor{blue!5} \textbf{26} & \cellcolor{blue!5} 26 & \cellcolor{blue!5} \textbf{26} & 32 & 27 & \cellcolor{blue!5} 26 & 27 \\
50 400& 8 & \cellcolor{blue!5} \textbf{8} & \cellcolor{blue!5} \textbf{8} & \cellcolor{blue!5} \textbf{8} & \cellcolor{blue!5} 8 & \cellcolor{blue!5} \textbf{8} & \cellcolor{blue!5} \textbf{8} & \cellcolor{blue!5} \textbf{8} \\
50 401& 28 & \cellcolor{blue!5} \textbf{28} & \cellcolor{blue!5} 28 & \cellcolor{blue!5} 28 & 31 & \cellcolor{blue!5} 28 & \cellcolor{blue!5} 28 & \cellcolor{blue!5} 28 \\
50 402& 27 & \cellcolor{blue!5} \textbf{27} & \cellcolor{blue!5} 27 & \cellcolor{blue!5} \textbf{27} & 30 & \cellcolor{blue!5} \textbf{27} & \cellcolor{blue!5} 27 & 28 \\
50 403& 33 & \cellcolor{blue!5} 34 & \cellcolor{blue!5} 34 & \cellcolor{blue!5} \textbf{34} & 36 & \cellcolor{blue!5} \textbf{34} & \cellcolor{blue!5} 34 & \cellcolor{blue!5} 34 \\
50 404& 29 & 32 & \cellcolor{blue!5} 31 & \cellcolor{blue!5} \textbf{31} & 32 & \cellcolor{blue!5} \textbf{31} & \cellcolor{blue!5} 31 & \cellcolor{blue!5} 31 \\
50 405& 27 & \cellcolor{blue!5} \textbf{27} & \cellcolor{blue!5} 27 & \cellcolor{blue!5} \textbf{27} & 29 & \cellcolor{blue!5} \textbf{27} & \cellcolor{blue!5} 27 & \cellcolor{blue!5} 27 \\
50 406& 32 & 34 & \cellcolor{blue!5} 32 & \cellcolor{blue!5} \textbf{32} & 36 & \cellcolor{blue!5} \textbf{32} & \cellcolor{blue!5} 32 & 33 \\
50 407& 27 & 30 & \cellcolor{blue!5} 29 & \cellcolor{blue!5} \textbf{29} & 30 & \cellcolor{blue!5} \textbf{29} & \cellcolor{blue!5} 29 & \cellcolor{blue!5} 29 \\
50 408& 26 & \cellcolor{blue!5} \textbf{26} & \cellcolor{blue!5} \textbf{26} & \cellcolor{blue!5} \textbf{26} & 28 & \cellcolor{blue!5} \textbf{26} & \cellcolor{blue!5} 26 & \cellcolor{blue!5} 26 \\
50 409& 30 & \cellcolor{blue!5} 33 & \cellcolor{blue!5} 33 & \cellcolor{blue!5} \textbf{33} & 34 & \cellcolor{blue!5} \textbf{33} & \cellcolor{blue!5} 33 & \cellcolor{blue!5} 33 \\
50 41& 25 & 26 & 26 & \cellcolor{blue!40} \textbf{25} & 32 & 26 & 26 & 26 \\
50 410& 28 & \cellcolor{blue!5} \textbf{28} & \cellcolor{blue!5} 28 & \cellcolor{blue!5} \textbf{28} & 30 & \cellcolor{blue!5} \textbf{28} & \cellcolor{blue!5} 28 & 29 \\
50 411& 29 & \cellcolor{blue!5} \textbf{29} & \cellcolor{blue!5} 29 & \cellcolor{blue!5} \textbf{29} & 32 & \cellcolor{blue!5} \textbf{29} & \cellcolor{blue!5} 29 & 30 \\
50 412& 26 & \cellcolor{blue!5} \textbf{26} & \cellcolor{blue!5} \textbf{26} & \cellcolor{blue!5} \textbf{26} & 29 & \cellcolor{blue!5} \textbf{26} & \cellcolor{blue!5} 26 & \cellcolor{blue!5} 26 \\
50 413& 28 & \cellcolor{blue!5} 30 & \cellcolor{blue!5} 30 & \cellcolor{blue!5} \textbf{30} & 32 & \cellcolor{blue!5} \textbf{30} & \cellcolor{blue!5} 30 & 31 \\
50 414& 27 & \cellcolor{blue!5} \textbf{27} & \cellcolor{blue!5} 27 & \cellcolor{blue!5} 27 & 28 & \cellcolor{blue!5} \textbf{27} & \cellcolor{blue!5} 27 & \cellcolor{blue!5} 27 \\
50 415& 26 & 29 & \cellcolor{blue!5} 28 & \cellcolor{blue!5} \textbf{28} & 32 & \cellcolor{blue!5} \textbf{28} & \cellcolor{blue!5} 28 & 29 \\
50 416& 27 & \cellcolor{blue!5} \textbf{27} & \cellcolor{blue!5} 27 & \cellcolor{blue!5} \textbf{27} & 29 & \cellcolor{blue!5} \textbf{27} & \cellcolor{blue!5} 27 & \cellcolor{blue!5} 27 \\
50 417& 30 & \cellcolor{blue!5} \textbf{30} & \cellcolor{blue!5} 30 & \cellcolor{blue!5} 30 & 32 & \cellcolor{blue!5} \textbf{30} & \cellcolor{blue!5} 30 & \cellcolor{blue!5} 30 \\
50 418& 26 & 28 & \cellcolor{blue!5} 27 & \cellcolor{blue!5} \textbf{27} & 29 & \cellcolor{blue!5} \textbf{27} & \cellcolor{blue!5} 27 & 28 \\
50 419& 32 & \cellcolor{blue!5} 33 & \cellcolor{blue!5} 33 & \cellcolor{blue!5} \textbf{33} & 34 & \cellcolor{blue!5} \textbf{33} & \cellcolor{blue!5} 33 & 34 \\
50 42& 23 & \cellcolor{blue!5} 24 & \cellcolor{blue!5} 24 & \cellcolor{blue!5} 24 & 31 & \cellcolor{blue!5} 24 & \cellcolor{blue!5} 24 & \cellcolor{blue!5} 24 \\
50 420& 27 & \cellcolor{blue!5} 28 & \cellcolor{blue!5} 28 & \cellcolor{blue!5} \textbf{28} & 30 & \cellcolor{blue!5} \textbf{28} & \cellcolor{blue!5} 28 & \cellcolor{blue!5} 28 \\
50 421& 34 & \cellcolor{blue!5} \textbf{34} & \cellcolor{blue!5} 34 & \cellcolor{blue!5} \textbf{34} & 35 & \cellcolor{blue!5} \textbf{34} & \cellcolor{blue!5} 34 & 35 \\
50 422& 26 & \cellcolor{blue!5} 29 & \cellcolor{blue!5} 29 & \cellcolor{blue!5} \textbf{29} & 31 & \cellcolor{blue!5} \textbf{29} & \cellcolor{blue!5} 29 & \cellcolor{blue!5} 29 \\
50 423& 27 & \cellcolor{blue!5} 29 & \cellcolor{blue!5} 29 & \cellcolor{blue!5} \textbf{29} & 31 & \cellcolor{blue!5} \textbf{29} & \cellcolor{blue!5} 29 & \cellcolor{blue!5} 29 \\
50 424& 27 & \cellcolor{blue!5} \textbf{27} & \cellcolor{blue!5} 27 & \cellcolor{blue!5} \textbf{27} & 30 & \cellcolor{blue!5} \textbf{27} & \cellcolor{blue!5} 27 & \cellcolor{blue!5} 27 \\
50 425& 34 & \cellcolor{blue!5} \textbf{34} & \cellcolor{blue!5} 34 & \cellcolor{blue!5} \textbf{34} & 35 & \cellcolor{blue!5} \textbf{34} & \cellcolor{blue!5} 34 & 35 \\
50 426& 11 & \cellcolor{blue!5} \textbf{11} & \cellcolor{blue!5} \textbf{11} & \cellcolor{blue!5} \textbf{11} & 12 & \cellcolor{blue!5} \textbf{11} & \cellcolor{blue!5} \textbf{11} & \cellcolor{blue!5} 11 \\
50 427& 12 & \cellcolor{blue!5} \textbf{12} & \cellcolor{blue!5} \textbf{12} & \cellcolor{blue!5} \textbf{12} & \cellcolor{blue!5} 12 & \cellcolor{blue!5} \textbf{12} & \cellcolor{blue!5} \textbf{12} & \cellcolor{blue!5} 12 \\
50 428& 13 & \cellcolor{blue!5} \textbf{13} & \cellcolor{blue!5} \textbf{13} & \cellcolor{blue!5} \textbf{13} & \cellcolor{blue!5} 13 & \cellcolor{blue!5} \textbf{13} & \cellcolor{blue!5} \textbf{13} & \cellcolor{blue!5} 13 \\
50 429& 11 & \cellcolor{blue!5} \textbf{11} & \cellcolor{blue!5} \textbf{11} & \cellcolor{blue!5} \textbf{11} & \cellcolor{blue!5} 11 & \cellcolor{blue!5} \textbf{11} & \cellcolor{blue!5} \textbf{11} & \cellcolor{blue!5} 11 \\
50 43& 25 & 26 & \cellcolor{blue!5} \textbf{25} & \cellcolor{blue!5} \textbf{25} & 31 & \cellcolor{blue!5} 25 & \cellcolor{blue!5} 25 & 26 \\
50 430& 14 & \cellcolor{blue!5} \textbf{14} & \cellcolor{blue!5} \textbf{14} & \cellcolor{blue!5} \textbf{14} & 15 & \cellcolor{blue!5} \textbf{14} & \cellcolor{blue!5} \textbf{14} & \cellcolor{blue!5} 14 \\
50 431& 11 & \cellcolor{blue!5} \textbf{11} & \cellcolor{blue!5} \textbf{11} & \cellcolor{blue!5} \textbf{11} & \cellcolor{blue!5} 11 & \cellcolor{blue!5} \textbf{11} & \cellcolor{blue!5} \textbf{11} & \cellcolor{blue!5} \textbf{11} \\
50 432& 12 & \cellcolor{blue!5} \textbf{12} & \cellcolor{blue!5} \textbf{12} & \cellcolor{blue!5} \textbf{12} & 13 & \cellcolor{blue!5} \textbf{12} & \cellcolor{blue!5} \textbf{12} & \cellcolor{blue!5} 12 \\
50 433& 12 & \cellcolor{blue!5} \textbf{12} & \cellcolor{blue!5} \textbf{12} & \cellcolor{blue!5} \textbf{12} & \cellcolor{blue!5} 12 & \cellcolor{blue!5} \textbf{12} & \cellcolor{blue!5} \textbf{12} & \cellcolor{blue!5} 12 \\
50 434& 11 & \cellcolor{blue!5} \textbf{11} & \cellcolor{blue!5} \textbf{11} & \cellcolor{blue!5} \textbf{11} & \cellcolor{blue!5} 11 & \cellcolor{blue!5} \textbf{11} & \cellcolor{blue!5} \textbf{11} & \cellcolor{blue!5} 11 \\
50 435& 11 & \cellcolor{blue!5} \textbf{11} & \cellcolor{blue!5} \textbf{11} & \cellcolor{blue!5} \textbf{11} & \cellcolor{blue!5} 11 & \cellcolor{blue!5} \textbf{11} & \cellcolor{blue!5} \textbf{11} & \cellcolor{blue!5} 11 \\
50 436& 11 & \cellcolor{blue!5} \textbf{11} & \cellcolor{blue!5} \textbf{11} & \cellcolor{blue!5} \textbf{11} & \cellcolor{blue!5} 11 & \cellcolor{blue!5} \textbf{11} & \cellcolor{blue!5} \textbf{11} & \cellcolor{blue!5} 11 \\
50 437& 12 & \cellcolor{blue!5} \textbf{12} & \cellcolor{blue!5} \textbf{12} & \cellcolor{blue!5} \textbf{12} & 13 & \cellcolor{blue!5} \textbf{12} & \cellcolor{blue!5} \textbf{12} & 13 \\
50 438& 10 & \cellcolor{blue!5} \textbf{10} & \cellcolor{blue!5} \textbf{10} & \cellcolor{blue!5} \textbf{10} & 11 & \cellcolor{blue!5} \textbf{10} & \cellcolor{blue!5} \textbf{10} & \cellcolor{blue!5} 10 \\
50 439& 12 & \cellcolor{blue!5} \textbf{12} & \cellcolor{blue!5} \textbf{12} & \cellcolor{blue!5} \textbf{12} & 13 & \cellcolor{blue!5} \textbf{12} & \cellcolor{blue!5} \textbf{12} & \cellcolor{blue!5} 12 \\
50 44& 24 & \cellcolor{blue!5} 25 & \cellcolor{blue!5} 25 & \cellcolor{blue!5} 25 & 31 & \cellcolor{blue!5} 25 & \cellcolor{blue!5} 25 & \cellcolor{blue!5} 25 \\
50 440& 13 & \cellcolor{blue!5} \textbf{13} & \cellcolor{blue!5} \textbf{13} & \cellcolor{blue!5} \textbf{13} & \cellcolor{blue!5} 13 & \cellcolor{blue!5} \textbf{13} & \cellcolor{blue!5} \textbf{13} & \cellcolor{blue!5} \textbf{13} \\
50 441& 11 & \cellcolor{blue!5} \textbf{11} & \cellcolor{blue!5} \textbf{11} & \cellcolor{blue!5} \textbf{11} & \cellcolor{blue!5} 11 & \cellcolor{blue!5} \textbf{11} & \cellcolor{blue!5} \textbf{11} & \cellcolor{blue!5} 11 \\
50 442& 12 & \cellcolor{blue!5} \textbf{12} & \cellcolor{blue!5} \textbf{12} & \cellcolor{blue!5} \textbf{12} & 13 & \cellcolor{blue!5} \textbf{12} & \cellcolor{blue!5} \textbf{12} & \cellcolor{blue!5} 12 \\
50 443& 11 & \cellcolor{blue!5} \textbf{11} & \cellcolor{blue!5} \textbf{11} & \cellcolor{blue!5} \textbf{11} & 12 & \cellcolor{blue!5} \textbf{11} & \cellcolor{blue!5} \textbf{11} & \cellcolor{blue!5} 11 \\
50 444& 12 & \cellcolor{blue!5} \textbf{12} & \cellcolor{blue!5} \textbf{12} & \cellcolor{blue!5} \textbf{12} & \cellcolor{blue!5} 12 & \cellcolor{blue!5} \textbf{12} & \cellcolor{blue!5} \textbf{12} & \cellcolor{blue!5} 12 \\
50 445& 12 & \cellcolor{blue!5} \textbf{12} & \cellcolor{blue!5} \textbf{12} & \cellcolor{blue!5} \textbf{12} & \cellcolor{blue!5} 12 & \cellcolor{blue!5} \textbf{12} & \cellcolor{blue!5} \textbf{12} & \cellcolor{blue!5} 12 \\
50 446& 12 & \cellcolor{blue!5} \textbf{12} & \cellcolor{blue!5} \textbf{12} & \cellcolor{blue!5} \textbf{12} & 13 & \cellcolor{blue!5} \textbf{12} & \cellcolor{blue!5} \textbf{12} & \cellcolor{blue!5} 12 \\
50 447& 13 & \cellcolor{blue!5} \textbf{13} & \cellcolor{blue!5} \textbf{13} & \cellcolor{blue!5} \textbf{13} & 14 & \cellcolor{blue!5} \textbf{13} & \cellcolor{blue!5} \textbf{13} & \cellcolor{blue!5} 13 \\
50 448& 12 & \cellcolor{blue!5} \textbf{12} & \cellcolor{blue!5} \textbf{12} & \cellcolor{blue!5} \textbf{12} & 13 & \cellcolor{blue!5} \textbf{12} & \cellcolor{blue!5} \textbf{12} & \cellcolor{blue!5} 12 \\
50 449& 11 & \cellcolor{blue!5} \textbf{11} & \cellcolor{blue!5} \textbf{11} & \cellcolor{blue!5} \textbf{11} & \cellcolor{blue!5} 11 & \cellcolor{blue!5} \textbf{11} & \cellcolor{blue!5} \textbf{11} & \cellcolor{blue!5} 11 \\
50 45& 24 & \cellcolor{blue!5} 25 & \cellcolor{blue!5} 25 & \cellcolor{blue!5} 25 & 28 & \cellcolor{blue!5} 25 & \cellcolor{blue!5} 25 & \cellcolor{blue!5} 25 \\
50 450& 11 & \cellcolor{blue!5} \textbf{11} & \cellcolor{blue!5} \textbf{11} & \cellcolor{blue!5} \textbf{11} & \cellcolor{blue!5} 11 & \cellcolor{blue!5} \textbf{11} & \cellcolor{blue!5} \textbf{11} & \cellcolor{blue!5} 11 \\
50 451& 8 & \cellcolor{blue!5} \textbf{8} & \cellcolor{blue!5} \textbf{8} & \cellcolor{blue!5} \textbf{8} & \cellcolor{blue!5} \textbf{8} & \cellcolor{blue!5} \textbf{8} & \cellcolor{blue!5} \textbf{8} & \cellcolor{blue!5} \textbf{8} \\
50 452& 8 & \cellcolor{blue!5} \textbf{8} & \cellcolor{blue!5} \textbf{8} & \cellcolor{blue!5} \textbf{8} & \cellcolor{blue!5} \textbf{8} & \cellcolor{blue!5} \textbf{8} & \cellcolor{blue!5} \textbf{8} & \cellcolor{blue!5} \textbf{8} \\
50 453& 7 & \cellcolor{blue!5} \textbf{7} & \cellcolor{blue!5} \textbf{7} & \cellcolor{blue!5} \textbf{7} & \cellcolor{blue!5} \textbf{7} & \cellcolor{blue!5} \textbf{7} & \cellcolor{blue!5} \textbf{7} & \cellcolor{blue!5} \textbf{7} \\
50 454& 8 & \cellcolor{blue!5} \textbf{8} & \cellcolor{blue!5} \textbf{8} & \cellcolor{blue!5} \textbf{8} & \cellcolor{blue!5} \textbf{8} & \cellcolor{blue!5} \textbf{8} & \cellcolor{blue!5} \textbf{8} & \cellcolor{blue!5} \textbf{8} \\
50 455& 6 & \cellcolor{blue!5} \textbf{6} & \cellcolor{blue!5} \textbf{6} & \cellcolor{blue!5} \textbf{6} & \cellcolor{blue!5} \textbf{6} & \cellcolor{blue!5} \textbf{6} & \cellcolor{blue!5} \textbf{6} & \cellcolor{blue!5} \textbf{6} \\
50 456& 8 & \cellcolor{blue!5} \textbf{8} & \cellcolor{blue!5} \textbf{8} & \cellcolor{blue!5} \textbf{8} & \cellcolor{blue!5} \textbf{8} & \cellcolor{blue!5} \textbf{8} & \cellcolor{blue!5} \textbf{8} & \cellcolor{blue!5} \textbf{8} \\
50 457& 8 & \cellcolor{blue!5} \textbf{8} & \cellcolor{blue!5} \textbf{8} & \cellcolor{blue!5} \textbf{8} & \cellcolor{blue!5} \textbf{8} & \cellcolor{blue!5} \textbf{8} & \cellcolor{blue!5} \textbf{8} & \cellcolor{blue!5} \textbf{8} \\
50 458& 7 & \cellcolor{blue!5} \textbf{7} & \cellcolor{blue!5} \textbf{7} & \cellcolor{blue!5} \textbf{7} & \cellcolor{blue!5} \textbf{7} & \cellcolor{blue!5} \textbf{7} & \cellcolor{blue!5} \textbf{7} & \cellcolor{blue!5} \textbf{7} \\
50 459& 7 & \cellcolor{blue!5} \textbf{7} & \cellcolor{blue!5} \textbf{7} & \cellcolor{blue!5} \textbf{7} & \cellcolor{blue!5} \textbf{7} & \cellcolor{blue!5} \textbf{7} & \cellcolor{blue!5} \textbf{7} & \cellcolor{blue!5} \textbf{7} \\
50 46& 28 & 29 & \cellcolor{blue!10} 28 & \cellcolor{blue!10} \textbf{28} & 33 & 29 & \cellcolor{blue!10} 28 & 29 \\
50 460& 7 & \cellcolor{blue!5} \textbf{7} & \cellcolor{blue!5} \textbf{7} & \cellcolor{blue!5} \textbf{7} & \cellcolor{blue!5} \textbf{7} & \cellcolor{blue!5} \textbf{7} & \cellcolor{blue!5} \textbf{7} & \cellcolor{blue!5} \textbf{7} \\
50 461& 6 & \cellcolor{blue!5} \textbf{6} & \cellcolor{blue!5} \textbf{6} & \cellcolor{blue!5} \textbf{6} & \cellcolor{blue!5} \textbf{6} & \cellcolor{blue!5} \textbf{6} & \cellcolor{blue!5} \textbf{6} & \cellcolor{blue!5} \textbf{6} \\
50 462& 7 & \cellcolor{blue!5} \textbf{7} & \cellcolor{blue!5} \textbf{7} & \cellcolor{blue!5} \textbf{7} & \cellcolor{blue!5} \textbf{7} & \cellcolor{blue!5} \textbf{7} & \cellcolor{blue!5} \textbf{7} & \cellcolor{blue!5} \textbf{7} \\
50 463& 8 & \cellcolor{blue!5} \textbf{8} & \cellcolor{blue!5} \textbf{8} & \cellcolor{blue!5} \textbf{8} & \cellcolor{blue!5} \textbf{8} & \cellcolor{blue!5} \textbf{8} & \cellcolor{blue!5} \textbf{8} & \cellcolor{blue!5} \textbf{8} \\
50 464& 6 & \cellcolor{blue!5} \textbf{6} & \cellcolor{blue!5} \textbf{6} & \cellcolor{blue!5} \textbf{6} & \cellcolor{blue!5} \textbf{6} & \cellcolor{blue!5} \textbf{6} & \cellcolor{blue!5} \textbf{6} & \cellcolor{blue!5} \textbf{6} \\
50 465& 8 & \cellcolor{blue!5} \textbf{8} & \cellcolor{blue!5} \textbf{8} & \cellcolor{blue!5} \textbf{8} & \cellcolor{blue!5} \textbf{8} & \cellcolor{blue!5} \textbf{8} & \cellcolor{blue!5} \textbf{8} & \cellcolor{blue!5} \textbf{8} \\
50 466& 7 & \cellcolor{blue!5} \textbf{7} & \cellcolor{blue!5} \textbf{7} & \cellcolor{blue!5} \textbf{7} & \cellcolor{blue!5} \textbf{7} & \cellcolor{blue!5} \textbf{7} & \cellcolor{blue!5} \textbf{7} & \cellcolor{blue!5} \textbf{7} \\
50 467& 9 & \cellcolor{blue!5} \textbf{9} & \cellcolor{blue!5} \textbf{9} & \cellcolor{blue!5} \textbf{9} & \cellcolor{blue!5} \textbf{9} & \cellcolor{blue!5} \textbf{9} & \cellcolor{blue!5} \textbf{9} & \cellcolor{blue!5} \textbf{9} \\
50 468& 7 & \cellcolor{blue!5} \textbf{7} & \cellcolor{blue!5} \textbf{7} & \cellcolor{blue!5} \textbf{7} & \cellcolor{blue!5} \textbf{7} & \cellcolor{blue!5} \textbf{7} & \cellcolor{blue!5} \textbf{7} & \cellcolor{blue!5} \textbf{7} \\
50 469& 8 & \cellcolor{blue!5} \textbf{8} & \cellcolor{blue!5} \textbf{8} & \cellcolor{blue!5} \textbf{8} & \cellcolor{blue!5} \textbf{8} & \cellcolor{blue!5} \textbf{8} & \cellcolor{blue!5} \textbf{8} & \cellcolor{blue!5} \textbf{8} \\
50 47& 28 & \cellcolor{blue!5} \textbf{28} & \cellcolor{blue!5} 28 & \cellcolor{blue!5} \textbf{28} & 33 & 45 & \cellcolor{blue!5} 28 & \cellcolor{blue!5} 28 \\
50 470& 8 & \cellcolor{blue!5} \textbf{8} & \cellcolor{blue!5} \textbf{8} & \cellcolor{blue!5} \textbf{8} & \cellcolor{blue!5} \textbf{8} & \cellcolor{blue!5} \textbf{8} & \cellcolor{blue!5} \textbf{8} & \cellcolor{blue!5} \textbf{8} \\
50 471& 7 & \cellcolor{blue!5} \textbf{7} & \cellcolor{blue!5} \textbf{7} & \cellcolor{blue!5} \textbf{7} & \cellcolor{blue!5} \textbf{7} & \cellcolor{blue!5} \textbf{7} & \cellcolor{blue!5} \textbf{7} & \cellcolor{blue!5} \textbf{7} \\
50 472& 8 & \cellcolor{blue!5} \textbf{8} & \cellcolor{blue!5} \textbf{8} & \cellcolor{blue!5} \textbf{8} & \cellcolor{blue!5} \textbf{8} & \cellcolor{blue!5} \textbf{8} & \cellcolor{blue!5} \textbf{8} & \cellcolor{blue!5} \textbf{8} \\
50 473& 7 & \cellcolor{blue!5} \textbf{7} & \cellcolor{blue!5} \textbf{7} & \cellcolor{blue!5} \textbf{7} & \cellcolor{blue!5} \textbf{7} & \cellcolor{blue!5} \textbf{7} & \cellcolor{blue!5} \textbf{7} & \cellcolor{blue!5} \textbf{7} \\
50 474& 7 & \cellcolor{blue!5} \textbf{7} & \cellcolor{blue!5} \textbf{7} & \cellcolor{blue!5} \textbf{7} & \cellcolor{blue!5} \textbf{7} & \cellcolor{blue!5} \textbf{7} & \cellcolor{blue!5} \textbf{7} & \cellcolor{blue!5} \textbf{7} \\
50 475& 6 & \cellcolor{blue!5} \textbf{6} & \cellcolor{blue!5} \textbf{6} & \cellcolor{blue!5} \textbf{6} & \cellcolor{blue!5} \textbf{6} & \cellcolor{blue!5} \textbf{6} & \cellcolor{blue!5} \textbf{6} & \cellcolor{blue!5} \textbf{6} \\
50 476& 26 & \cellcolor{blue!5} 28 & \cellcolor{blue!5} \textbf{28} & \cellcolor{blue!5} \textbf{28} & \cellcolor{blue!5} \textbf{28} & \cellcolor{blue!5} \textbf{28} & \cellcolor{blue!5} \textbf{28} & \cellcolor{blue!5} \textbf{28} \\
50 477& 27 & \cellcolor{blue!5} 29 & \cellcolor{blue!5} \textbf{29} & \cellcolor{blue!5} \textbf{29} & \cellcolor{blue!5} 29 & \cellcolor{blue!5} \textbf{29} & \cellcolor{blue!5} \textbf{29} & \cellcolor{blue!5} \textbf{29} \\
50 478& 29 & \cellcolor{blue!5} 32 & \cellcolor{blue!5} \textbf{32} & \cellcolor{blue!5} \textbf{32} & \cellcolor{blue!5} 32 & \cellcolor{blue!5} \textbf{32} & \cellcolor{blue!5} \textbf{32} & \cellcolor{blue!5} 32 \\
50 479& 28 & \cellcolor{blue!5} \textbf{28} & \cellcolor{blue!5} \textbf{28} & \cellcolor{blue!5} \textbf{28} & \cellcolor{blue!5} \textbf{28} & \cellcolor{blue!5} \textbf{28} & \cellcolor{blue!5} \textbf{28} & \cellcolor{blue!5} \textbf{28} \\
50 48& 27 & 28 & \cellcolor{blue!5} 27 & \cellcolor{blue!5} \textbf{27} & 32 & \cellcolor{blue!5} 27 & \cellcolor{blue!5} 27 & 28 \\
50 480& 34 & \cellcolor{blue!5} \textbf{34} & \cellcolor{blue!5} \textbf{34} & \cellcolor{blue!5} \textbf{34} & \cellcolor{blue!5} \textbf{34} & \cellcolor{blue!5} \textbf{34} & \cellcolor{blue!5} \textbf{34} & \cellcolor{blue!5} \textbf{34} \\
50 481& 26 & \cellcolor{blue!5} 28 & \cellcolor{blue!5} \textbf{28} & \cellcolor{blue!5} \textbf{28} & \cellcolor{blue!5} 28 & \cellcolor{blue!5} \textbf{28} & \cellcolor{blue!5} \textbf{28} & \cellcolor{blue!5} \textbf{28} \\
50 482& 26 & \cellcolor{blue!5} 27 & \cellcolor{blue!5} \textbf{27} & \cellcolor{blue!5} \textbf{27} & \cellcolor{blue!5} \textbf{27} & \cellcolor{blue!5} \textbf{27} & \cellcolor{blue!5} \textbf{27} & \cellcolor{blue!5} \textbf{27} \\
50 483& 28 & \cellcolor{blue!5} 30 & \cellcolor{blue!5} \textbf{30} & \cellcolor{blue!5} \textbf{30} & \cellcolor{blue!5} 30 & \cellcolor{blue!5} \textbf{30} & \cellcolor{blue!5} \textbf{30} & \cellcolor{blue!5} \textbf{30} \\
50 484& 30 & \cellcolor{blue!5} 32 & \cellcolor{blue!5} \textbf{32} & \cellcolor{blue!5} \textbf{32} & \cellcolor{blue!5} \textbf{32} & \cellcolor{blue!5} \textbf{32} & \cellcolor{blue!5} \textbf{32} & \cellcolor{blue!5} \textbf{32} \\
50 485& 29 & \cellcolor{blue!5} 31 & \cellcolor{blue!5} \textbf{31} & \cellcolor{blue!5} \textbf{31} & \cellcolor{blue!5} 31 & \cellcolor{blue!5} \textbf{31} & \cellcolor{blue!5} \textbf{31} & \cellcolor{blue!5} \textbf{31} \\
50 486& 30 & \cellcolor{blue!5} 32 & \cellcolor{blue!5} \textbf{32} & \cellcolor{blue!5} \textbf{32} & \cellcolor{blue!5} \textbf{32} & \cellcolor{blue!5} \textbf{32} & \cellcolor{blue!5} \textbf{32} & \cellcolor{blue!5} \textbf{32} \\
50 487& 30 & \cellcolor{blue!5} 31 & \cellcolor{blue!5} \textbf{31} & \cellcolor{blue!5} \textbf{31} & \cellcolor{blue!5} 31 & \cellcolor{blue!5} \textbf{31} & \cellcolor{blue!5} \textbf{31} & \cellcolor{blue!5} \textbf{31} \\
50 488& 28 & \cellcolor{blue!5} 31 & \cellcolor{blue!5} \textbf{31} & \cellcolor{blue!5} \textbf{31} & \cellcolor{blue!5} 31 & \cellcolor{blue!5} \textbf{31} & \cellcolor{blue!5} \textbf{31} & \cellcolor{blue!5} 31 \\
50 489& 33 & \cellcolor{blue!5} 35 & \cellcolor{blue!5} \textbf{35} & \cellcolor{blue!5} \textbf{35} & \cellcolor{blue!5} 35 & \cellcolor{blue!5} \textbf{35} & \cellcolor{blue!5} \textbf{35} & \cellcolor{blue!5} 35 \\
50 49& 24 & \cellcolor{blue!5} 25 & \cellcolor{blue!5} 25 & \cellcolor{blue!5} \textbf{25} & 31 & \cellcolor{blue!5} 25 & \cellcolor{blue!5} 25 & \cellcolor{blue!5} 25 \\
50 490& 29 & \cellcolor{blue!5} \textbf{29} & \cellcolor{blue!5} \textbf{29} & \cellcolor{blue!5} \textbf{29} & \cellcolor{blue!5} 29 & \cellcolor{blue!5} \textbf{29} & \cellcolor{blue!5} \textbf{29} & \cellcolor{blue!5} \textbf{29} \\
50 491& 33 & \cellcolor{blue!5} 35 & \cellcolor{blue!5} \textbf{35} & \cellcolor{blue!5} \textbf{35} & \cellcolor{blue!5} 35 & \cellcolor{blue!5} \textbf{35} & \cellcolor{blue!5} \textbf{35} & \cellcolor{blue!5} 35 \\
50 492& 28 & 30 & \cellcolor{blue!5} \textbf{29} & \cellcolor{blue!5} \textbf{29} & \cellcolor{blue!5} 29 & \cellcolor{blue!5} \textbf{29} & \cellcolor{blue!5} \textbf{29} & \cellcolor{blue!5} 29 \\
50 493& 27 & \cellcolor{blue!5} 30 & \cellcolor{blue!5} \textbf{30} & \cellcolor{blue!5} \textbf{30} & \cellcolor{blue!5} 30 & \cellcolor{blue!5} \textbf{30} & \cellcolor{blue!5} \textbf{30} & \cellcolor{blue!5} \textbf{30} \\
50 494& 30 & \cellcolor{blue!5} 32 & \cellcolor{blue!5} \textbf{32} & \cellcolor{blue!5} \textbf{32} & \cellcolor{blue!5} 32 & \cellcolor{blue!5} \textbf{32} & \cellcolor{blue!5} \textbf{32} & \cellcolor{blue!5} \textbf{32} \\
50 495& 34 & \cellcolor{blue!5} \textbf{34} & \cellcolor{blue!5} \textbf{34} & \cellcolor{blue!5} \textbf{34} & \cellcolor{blue!5} 34 & \cellcolor{blue!5} \textbf{34} & \cellcolor{blue!5} \textbf{34} & \cellcolor{blue!5} \textbf{34} \\
50 496& 27 & \cellcolor{blue!5} 29 & \cellcolor{blue!5} \textbf{29} & \cellcolor{blue!5} \textbf{29} & \cellcolor{blue!5} 29 & \cellcolor{blue!5} \textbf{29} & \cellcolor{blue!5} \textbf{29} & \cellcolor{blue!5} \textbf{29} \\
50 497& 28 & \cellcolor{blue!5} 30 & \cellcolor{blue!5} \textbf{30} & \cellcolor{blue!5} \textbf{30} & \cellcolor{blue!5} 30 & \cellcolor{blue!5} \textbf{30} & \cellcolor{blue!5} \textbf{30} & \cellcolor{blue!5} \textbf{30} \\
50 498& 28 & \cellcolor{blue!5} 30 & \cellcolor{blue!5} \textbf{30} & \cellcolor{blue!5} \textbf{30} & \cellcolor{blue!5} 30 & \cellcolor{blue!5} \textbf{30} & \cellcolor{blue!5} \textbf{30} & \cellcolor{blue!5} \textbf{30} \\
50 499& 31 & \cellcolor{blue!5} 33 & \cellcolor{blue!5} \textbf{33} & \cellcolor{blue!5} \textbf{33} & \cellcolor{blue!5} 33 & \cellcolor{blue!5} \textbf{33} & \cellcolor{blue!5} \textbf{33} & \cellcolor{blue!5} \textbf{33} \\
50 5& 7 & \cellcolor{blue!5} \textbf{7} & \cellcolor{blue!5} \textbf{7} & \cellcolor{blue!5} \textbf{7} & \cellcolor{blue!5} 7 & \cellcolor{blue!5} 7 & \cellcolor{blue!5} \textbf{7} & \cellcolor{blue!5} \textbf{7} \\
50 50& 26 & \cellcolor{blue!5} 27 & \cellcolor{blue!5} 27 & \cellcolor{blue!5} 27 & 32 & \cellcolor{blue!5} 27 & \cellcolor{blue!5} 27 & \cellcolor{blue!5} 27 \\
50 500& 30 & \cellcolor{blue!5} 34 & \cellcolor{blue!5} \textbf{34} & \cellcolor{blue!5} \textbf{34} & \cellcolor{blue!5} 34 & \cellcolor{blue!5} \textbf{34} & \cellcolor{blue!5} \textbf{34} & \cellcolor{blue!5} \textbf{34} \\
50 501& 12 & \cellcolor{blue!5} \textbf{12} & \cellcolor{blue!5} \textbf{12} & \cellcolor{blue!5} \textbf{12} & \cellcolor{blue!5} \textbf{12} & \cellcolor{blue!5} \textbf{12} & \cellcolor{blue!5} \textbf{12} & \cellcolor{blue!5} \textbf{12} \\
50 502& 10 & \cellcolor{blue!5} \textbf{10} & \cellcolor{blue!5} \textbf{10} & \cellcolor{blue!5} \textbf{10} & \cellcolor{blue!5} \textbf{10} & \cellcolor{blue!5} \textbf{10} & \cellcolor{blue!5} \textbf{10} & \cellcolor{blue!5} \textbf{10} \\
50 503& 13 & \cellcolor{blue!5} \textbf{13} & \cellcolor{blue!5} \textbf{13} & \cellcolor{blue!5} \textbf{13} & \cellcolor{blue!5} \textbf{13} & \cellcolor{blue!5} \textbf{13} & \cellcolor{blue!5} \textbf{13} & \cellcolor{blue!5} \textbf{13} \\
50 504& 11 & \cellcolor{blue!5} \textbf{11} & \cellcolor{blue!5} \textbf{11} & \cellcolor{blue!5} \textbf{11} & \cellcolor{blue!5} \textbf{11} & \cellcolor{blue!5} \textbf{11} & \cellcolor{blue!5} \textbf{11} & \cellcolor{blue!5} \textbf{11} \\
50 505& 12 & \cellcolor{blue!5} \textbf{12} & \cellcolor{blue!5} \textbf{12} & \cellcolor{blue!5} \textbf{12} & \cellcolor{blue!5} \textbf{12} & \cellcolor{blue!5} \textbf{12} & \cellcolor{blue!5} \textbf{12} & \cellcolor{blue!5} \textbf{12} \\
50 506& 11 & \cellcolor{blue!5} \textbf{11} & \cellcolor{blue!5} \textbf{11} & \cellcolor{blue!5} \textbf{11} & \cellcolor{blue!5} \textbf{11} & \cellcolor{blue!5} \textbf{11} & \cellcolor{blue!5} \textbf{11} & \cellcolor{blue!5} \textbf{11} \\
50 507& 13 & \cellcolor{blue!5} \textbf{13} & \cellcolor{blue!5} \textbf{13} & \cellcolor{blue!5} \textbf{13} & \cellcolor{blue!5} \textbf{13} & \cellcolor{blue!5} \textbf{13} & \cellcolor{blue!5} \textbf{13} & \cellcolor{blue!5} \textbf{13} \\
50 508& 14 & \cellcolor{blue!5} \textbf{14} & \cellcolor{blue!5} \textbf{14} & \cellcolor{blue!5} \textbf{14} & \cellcolor{blue!5} \textbf{14} & \cellcolor{blue!5} \textbf{14} & \cellcolor{blue!5} \textbf{14} & \cellcolor{blue!5} \textbf{14} \\
50 509& 13 & \cellcolor{blue!5} \textbf{13} & \cellcolor{blue!5} \textbf{13} & \cellcolor{blue!5} \textbf{13} & \cellcolor{blue!5} \textbf{13} & \cellcolor{blue!5} \textbf{13} & \cellcolor{blue!5} \textbf{13} & \cellcolor{blue!5} \textbf{13} \\
50 51& 12 & \cellcolor{blue!5} \textbf{12} & \cellcolor{blue!5} \textbf{12} & \cellcolor{blue!5} \textbf{12} & \cellcolor{blue!5} 12 & \cellcolor{blue!5} 12 & \cellcolor{blue!5} \textbf{12} & \cellcolor{blue!5} 12 \\
50 510& 11 & \cellcolor{blue!5} \textbf{11} & \cellcolor{blue!5} \textbf{11} & \cellcolor{blue!5} \textbf{11} & \cellcolor{blue!5} \textbf{11} & \cellcolor{blue!5} \textbf{11} & \cellcolor{blue!5} \textbf{11} & \cellcolor{blue!5} \textbf{11} \\
50 511& 13 & \cellcolor{blue!5} \textbf{13} & \cellcolor{blue!5} \textbf{13} & \cellcolor{blue!5} \textbf{13} & \cellcolor{blue!5} \textbf{13} & \cellcolor{blue!5} \textbf{13} & \cellcolor{blue!5} \textbf{13} & \cellcolor{blue!5} \textbf{13} \\
50 512& 13 & \cellcolor{blue!5} \textbf{13} & \cellcolor{blue!5} \textbf{13} & \cellcolor{blue!5} \textbf{13} & \cellcolor{blue!5} \textbf{13} & \cellcolor{blue!5} \textbf{13} & \cellcolor{blue!5} \textbf{13} & \cellcolor{blue!5} \textbf{13} \\
50 513& 12 & \cellcolor{blue!5} \textbf{12} & \cellcolor{blue!5} \textbf{12} & \cellcolor{blue!5} \textbf{12} & \cellcolor{blue!5} \textbf{12} & \cellcolor{blue!5} \textbf{12} & \cellcolor{blue!5} \textbf{12} & \cellcolor{blue!5} \textbf{12} \\
50 514& 12 & \cellcolor{blue!5} \textbf{12} & \cellcolor{blue!5} \textbf{12} & \cellcolor{blue!5} \textbf{12} & \cellcolor{blue!5} \textbf{12} & \cellcolor{blue!5} \textbf{12} & \cellcolor{blue!5} \textbf{12} & \cellcolor{blue!5} \textbf{12} \\
50 515& 11 & \cellcolor{blue!5} \textbf{11} & \cellcolor{blue!5} \textbf{11} & \cellcolor{blue!5} \textbf{11} & \cellcolor{blue!5} \textbf{11} & \cellcolor{blue!5} \textbf{11} & \cellcolor{blue!5} \textbf{11} & \cellcolor{blue!5} \textbf{11} \\
50 516& 13 & \cellcolor{blue!5} \textbf{13} & \cellcolor{blue!5} \textbf{13} & \cellcolor{blue!5} \textbf{13} & \cellcolor{blue!5} \textbf{13} & \cellcolor{blue!5} \textbf{13} & \cellcolor{blue!5} \textbf{13} & \cellcolor{blue!5} \textbf{13} \\
50 517& 14 & \cellcolor{blue!5} \textbf{14} & \cellcolor{blue!5} \textbf{14} & \cellcolor{blue!5} \textbf{14} & \cellcolor{blue!5} \textbf{14} & \cellcolor{blue!5} \textbf{14} & \cellcolor{blue!5} \textbf{14} & \cellcolor{blue!5} \textbf{14} \\
50 518& 11 & \cellcolor{blue!5} \textbf{11} & \cellcolor{blue!5} \textbf{11} & \cellcolor{blue!5} \textbf{11} & \cellcolor{blue!5} \textbf{11} & \cellcolor{blue!5} \textbf{11} & \cellcolor{blue!5} \textbf{11} & \cellcolor{blue!5} \textbf{11} \\
50 519& 12 & \cellcolor{blue!5} \textbf{12} & \cellcolor{blue!5} \textbf{12} & \cellcolor{blue!5} \textbf{12} & \cellcolor{blue!5} \textbf{12} & \cellcolor{blue!5} \textbf{12} & \cellcolor{blue!5} \textbf{12} & \cellcolor{blue!5} \textbf{12} \\
50 52& 11 & \cellcolor{blue!5} \textbf{11} & \cellcolor{blue!5} \textbf{11} & \cellcolor{blue!5} \textbf{11} & \cellcolor{blue!5} 11 & \cellcolor{blue!5} 11 & \cellcolor{blue!5} 11 & \cellcolor{blue!5} 11 \\
50 520& 11 & \cellcolor{blue!5} \textbf{11} & \cellcolor{blue!5} \textbf{11} & \cellcolor{blue!5} \textbf{11} & \cellcolor{blue!5} \textbf{11} & \cellcolor{blue!5} \textbf{11} & \cellcolor{blue!5} \textbf{11} & \cellcolor{blue!5} \textbf{11} \\
50 521& 10 & \cellcolor{blue!5} \textbf{10} & \cellcolor{blue!5} \textbf{10} & \cellcolor{blue!5} \textbf{10} & \cellcolor{blue!5} \textbf{10} & \cellcolor{blue!5} \textbf{10} & \cellcolor{blue!5} \textbf{10} & \cellcolor{blue!5} \textbf{10} \\
50 522& 11 & \cellcolor{blue!5} \textbf{11} & \cellcolor{blue!5} \textbf{11} & \cellcolor{blue!5} \textbf{11} & \cellcolor{blue!5} \textbf{11} & \cellcolor{blue!5} \textbf{11} & \cellcolor{blue!5} \textbf{11} & \cellcolor{blue!5} \textbf{11} \\
50 523& 11 & \cellcolor{blue!5} \textbf{11} & \cellcolor{blue!5} \textbf{11} & \cellcolor{blue!5} \textbf{11} & \cellcolor{blue!5} \textbf{11} & \cellcolor{blue!5} \textbf{11} & \cellcolor{blue!5} \textbf{11} & \cellcolor{blue!5} \textbf{11} \\
50 524& 14 & \cellcolor{blue!5} \textbf{14} & \cellcolor{blue!5} \textbf{14} & \cellcolor{blue!5} \textbf{14} & \cellcolor{blue!5} \textbf{14} & \cellcolor{blue!5} \textbf{14} & \cellcolor{blue!5} \textbf{14} & \cellcolor{blue!5} \textbf{14} \\
50 525& 11 & \cellcolor{blue!5} \textbf{11} & \cellcolor{blue!5} \textbf{11} & \cellcolor{blue!5} \textbf{11} & \cellcolor{blue!5} \textbf{11} & \cellcolor{blue!5} \textbf{11} & \cellcolor{blue!5} \textbf{11} & \cellcolor{blue!5} \textbf{11} \\
50 53& 12 & \cellcolor{blue!40} \textbf{12} & 13 & 13 & 13 & 13 & 13 & 13 \\
50 54& 11 & \cellcolor{blue!5} \textbf{11} & \cellcolor{blue!5} \textbf{11} & \cellcolor{blue!5} \textbf{11} & 12 & \cellcolor{blue!5} 11 & \cellcolor{blue!5} \textbf{11} & \cellcolor{blue!5} 11 \\
50 55& 13 & \cellcolor{blue!5} \textbf{13} & \cellcolor{blue!5} \textbf{13} & \cellcolor{blue!5} \textbf{13} & 14 & \cellcolor{blue!5} 13 & \cellcolor{blue!5} 13 & \cellcolor{blue!5} 13 \\
50 56& 11 & \cellcolor{blue!5} \textbf{11} & \cellcolor{blue!5} \textbf{11} & \cellcolor{blue!5} \textbf{11} & 12 & \cellcolor{blue!5} 11 & \cellcolor{blue!5} 11 & \cellcolor{blue!5} 11 \\
50 57& 13 & \cellcolor{blue!5} \textbf{13} & \cellcolor{blue!5} \textbf{13} & \cellcolor{blue!5} \textbf{13} & 15 & \cellcolor{blue!5} 13 & \cellcolor{blue!5} 13 & \cellcolor{blue!5} 13 \\
50 58& 11 & \cellcolor{blue!5} \textbf{11} & \cellcolor{blue!5} \textbf{11} & \cellcolor{blue!5} \textbf{11} & \cellcolor{blue!5} 11 & \cellcolor{blue!5} 11 & \cellcolor{blue!5} 11 & \cellcolor{blue!5} 11 \\
50 59& 11 & \cellcolor{blue!5} \textbf{11} & \cellcolor{blue!5} \textbf{11} & \cellcolor{blue!5} \textbf{11} & \cellcolor{blue!5} 11 & \cellcolor{blue!5} 11 & \cellcolor{blue!5} 11 & \cellcolor{blue!5} 11 \\
50 6& 6 & \cellcolor{blue!5} \textbf{6} & \cellcolor{blue!5} \textbf{6} & \cellcolor{blue!5} \textbf{6} & \cellcolor{blue!5} 6 & \cellcolor{blue!5} 6 & \cellcolor{blue!5} \textbf{6} & \cellcolor{blue!5} \textbf{6} \\
50 60& 12 & \cellcolor{blue!5} \textbf{12} & \cellcolor{blue!5} \textbf{12} & \cellcolor{blue!5} \textbf{12} & 13 & \cellcolor{blue!5} 12 & \cellcolor{blue!5} 12 & \cellcolor{blue!5} 12 \\
50 61& 13 & \cellcolor{blue!5} \textbf{13} & \cellcolor{blue!5} \textbf{13} & \cellcolor{blue!5} \textbf{13} & \cellcolor{blue!5} 13 & \cellcolor{blue!5} 13 & \cellcolor{blue!5} 13 & \cellcolor{blue!5} 13 \\
50 62& 13 & \cellcolor{blue!5} \textbf{13} & \cellcolor{blue!5} \textbf{13} & \cellcolor{blue!5} \textbf{13} & 14 & \cellcolor{blue!5} 13 & \cellcolor{blue!5} 13 & \cellcolor{blue!5} 13 \\
50 63& 12 & \cellcolor{blue!5} \textbf{12} & \cellcolor{blue!5} \textbf{12} & \cellcolor{blue!5} \textbf{12} & \cellcolor{blue!5} 12 & \cellcolor{blue!5} 12 & \cellcolor{blue!5} 12 & \cellcolor{blue!5} 12 \\
50 64& 13 & \cellcolor{blue!5} \textbf{13} & \cellcolor{blue!5} \textbf{13} & \cellcolor{blue!5} \textbf{13} & \cellcolor{blue!5} 13 & \cellcolor{blue!5} 13 & \cellcolor{blue!5} 13 & \cellcolor{blue!5} 13 \\
50 65& 12 & \cellcolor{blue!5} \textbf{12} & \cellcolor{blue!5} \textbf{12} & \cellcolor{blue!5} \textbf{12} & \cellcolor{blue!5} 12 & \cellcolor{blue!5} 12 & \cellcolor{blue!5} 12 & \cellcolor{blue!5} 12 \\
50 66& 12 & \cellcolor{blue!5} \textbf{12} & \cellcolor{blue!5} \textbf{12} & \cellcolor{blue!5} \textbf{12} & 14 & \cellcolor{blue!5} 12 & \cellcolor{blue!5} 12 & \cellcolor{blue!5} 12 \\
50 67& 12 & \cellcolor{blue!5} \textbf{12} & \cellcolor{blue!5} \textbf{12} & \cellcolor{blue!5} \textbf{12} & 13 & \cellcolor{blue!5} 12 & \cellcolor{blue!5} 12 & \cellcolor{blue!5} 12 \\
50 68& 12 & \cellcolor{blue!5} \textbf{12} & \cellcolor{blue!5} \textbf{12} & \cellcolor{blue!5} \textbf{12} & \cellcolor{blue!5} 12 & \cellcolor{blue!5} 12 & \cellcolor{blue!5} \textbf{12} & \cellcolor{blue!5} 12 \\
50 69& 12 & \cellcolor{blue!5} \textbf{12} & \cellcolor{blue!5} \textbf{12} & \cellcolor{blue!5} \textbf{12} & 13 & \cellcolor{blue!5} 12 & \cellcolor{blue!5} 12 & \cellcolor{blue!5} 12 \\
50 7& 7 & \cellcolor{blue!5} \textbf{7} & \cellcolor{blue!5} \textbf{7} & \cellcolor{blue!5} \textbf{7} & \cellcolor{blue!5} 7 & \cellcolor{blue!5} 7 & \cellcolor{blue!5} \textbf{7} & \cellcolor{blue!5} \textbf{7} \\
50 70& 10 & \cellcolor{blue!5} \textbf{10} & \cellcolor{blue!5} \textbf{10} & \cellcolor{blue!5} \textbf{10} & \cellcolor{blue!5} 10 & \cellcolor{blue!5} 10 & \cellcolor{blue!5} 10 & \cellcolor{blue!5} 10 \\
50 71& 13 & \cellcolor{blue!5} \textbf{13} & \cellcolor{blue!5} \textbf{13} & \cellcolor{blue!5} \textbf{13} & 15 & \cellcolor{blue!5} 13 & \cellcolor{blue!5} 13 & \cellcolor{blue!5} 13 \\
50 72& 11 & \cellcolor{blue!5} \textbf{11} & \cellcolor{blue!5} \textbf{11} & \cellcolor{blue!5} \textbf{11} & \cellcolor{blue!5} 11 & \cellcolor{blue!5} 11 & \cellcolor{blue!5} 11 & \cellcolor{blue!5} 11 \\
50 73& 11 & \cellcolor{blue!5} \textbf{11} & \cellcolor{blue!5} \textbf{11} & \cellcolor{blue!5} \textbf{11} & 12 & \cellcolor{blue!5} 11 & \cellcolor{blue!5} 11 & \cellcolor{blue!5} 11 \\
50 74& 12 & \cellcolor{blue!5} \textbf{12} & \cellcolor{blue!5} \textbf{12} & \cellcolor{blue!5} \textbf{12} & \cellcolor{blue!5} 12 & \cellcolor{blue!5} 12 & \cellcolor{blue!5} 12 & \cellcolor{blue!5} 12 \\
50 75& 11 & \cellcolor{blue!5} \textbf{11} & \cellcolor{blue!5} \textbf{11} & \cellcolor{blue!5} \textbf{11} & 12 & \cellcolor{blue!5} 11 & \cellcolor{blue!5} 11 & \cellcolor{blue!5} 11 \\
50 76& 7 & \cellcolor{blue!5} \textbf{7} & \cellcolor{blue!5} \textbf{7} & \cellcolor{blue!5} \textbf{7} & \cellcolor{blue!5} 7 & \cellcolor{blue!5} \textbf{7} & \cellcolor{blue!5} \textbf{7} & \cellcolor{blue!5} \textbf{7} \\
50 77& 7 & \cellcolor{blue!5} \textbf{7} & \cellcolor{blue!5} \textbf{7} & \cellcolor{blue!5} \textbf{7} & \cellcolor{blue!5} 7 & \cellcolor{blue!5} \textbf{7} & \cellcolor{blue!5} \textbf{7} & \cellcolor{blue!5} 7 \\
50 78& 7 & \cellcolor{blue!5} \textbf{7} & \cellcolor{blue!5} \textbf{7} & \cellcolor{blue!5} \textbf{7} & \cellcolor{blue!5} 7 & \cellcolor{blue!5} \textbf{7} & \cellcolor{blue!5} \textbf{7} & \cellcolor{blue!5} 7 \\
50 79& 8 & \cellcolor{blue!5} \textbf{8} & \cellcolor{blue!5} \textbf{8} & \cellcolor{blue!5} \textbf{8} & \cellcolor{blue!5} 8 & \cellcolor{blue!5} \textbf{8} & \cellcolor{blue!5} \textbf{8} & \cellcolor{blue!5} 8 \\
50 8& 7 & \cellcolor{blue!5} \textbf{7} & \cellcolor{blue!5} \textbf{7} & \cellcolor{blue!5} \textbf{7} & \cellcolor{blue!5} 7 & \cellcolor{blue!5} 7 & \cellcolor{blue!5} \textbf{7} & \cellcolor{blue!5} \textbf{7} \\
50 80& 7 & \cellcolor{blue!5} \textbf{7} & \cellcolor{blue!5} \textbf{7} & \cellcolor{blue!5} \textbf{7} & \cellcolor{blue!5} 7 & \cellcolor{blue!5} \textbf{7} & \cellcolor{blue!5} \textbf{7} & \cellcolor{blue!5} 7 \\
50 81& 7 & \cellcolor{blue!5} \textbf{7} & \cellcolor{blue!5} \textbf{7} & \cellcolor{blue!5} \textbf{7} & \cellcolor{blue!5} 7 & \cellcolor{blue!5} \textbf{7} & \cellcolor{blue!5} \textbf{7} & \cellcolor{blue!5} 7 \\
50 82& 6 & \cellcolor{blue!5} \textbf{6} & \cellcolor{blue!5} \textbf{6} & \cellcolor{blue!5} \textbf{6} & \cellcolor{blue!5} 6 & \cellcolor{blue!5} \textbf{6} & \cellcolor{blue!5} \textbf{6} & \cellcolor{blue!5} \textbf{6} \\
50 83& 8 & \cellcolor{blue!5} \textbf{8} & \cellcolor{blue!5} \textbf{8} & \cellcolor{blue!5} \textbf{8} & \cellcolor{blue!5} 8 & \cellcolor{blue!5} \textbf{8} & \cellcolor{blue!5} \textbf{8} & \cellcolor{blue!5} 8 \\
50 84& 7 & \cellcolor{blue!5} \textbf{7} & \cellcolor{blue!5} \textbf{7} & \cellcolor{blue!5} \textbf{7} & \cellcolor{blue!5} 7 & \cellcolor{blue!5} \textbf{7} & \cellcolor{blue!5} \textbf{7} & \cellcolor{blue!5} \textbf{7} \\
50 85& 8 & \cellcolor{blue!5} \textbf{8} & \cellcolor{blue!5} \textbf{8} & \cellcolor{blue!5} \textbf{8} & \cellcolor{blue!5} 8 & \cellcolor{blue!5} \textbf{8} & \cellcolor{blue!5} \textbf{8} & \cellcolor{blue!5} 8 \\
50 86& 7 & \cellcolor{blue!5} \textbf{7} & \cellcolor{blue!5} \textbf{7} & \cellcolor{blue!5} \textbf{7} & \cellcolor{blue!5} 7 & \cellcolor{blue!5} \textbf{7} & \cellcolor{blue!5} \textbf{7} & \cellcolor{blue!5} \textbf{7} \\
50 87& 8 & \cellcolor{blue!5} \textbf{8} & \cellcolor{blue!5} \textbf{8} & \cellcolor{blue!5} \textbf{8} & \cellcolor{blue!5} 8 & \cellcolor{blue!5} \textbf{8} & \cellcolor{blue!5} \textbf{8} & \cellcolor{blue!5} 8 \\
50 88& 8 & \cellcolor{blue!5} \textbf{8} & \cellcolor{blue!5} \textbf{8} & \cellcolor{blue!5} \textbf{8} & \cellcolor{blue!5} 8 & \cellcolor{blue!5} 8 & \cellcolor{blue!5} \textbf{8} & \cellcolor{blue!5} \textbf{8} \\
50 89& 7 & \cellcolor{blue!5} \textbf{7} & \cellcolor{blue!5} \textbf{7} & \cellcolor{blue!5} \textbf{7} & \cellcolor{blue!5} 7 & \cellcolor{blue!5} \textbf{7} & \cellcolor{blue!5} \textbf{7} & \cellcolor{blue!5} \textbf{7} \\
50 9& 9 & \cellcolor{blue!5} \textbf{9} & \cellcolor{blue!5} \textbf{9} & \cellcolor{blue!5} \textbf{9} & \cellcolor{blue!5} 9 & \cellcolor{blue!5} 9 & \cellcolor{blue!5} \textbf{9} & \cellcolor{blue!5} \textbf{9} \\
50 90& 7 & \cellcolor{blue!5} \textbf{7} & \cellcolor{blue!5} \textbf{7} & \cellcolor{blue!5} \textbf{7} & 8 & \cellcolor{blue!5} \textbf{7} & \cellcolor{blue!5} \textbf{7} & \cellcolor{blue!5} 7 \\
50 91& 7 & \cellcolor{blue!5} \textbf{7} & \cellcolor{blue!5} \textbf{7} & \cellcolor{blue!5} \textbf{7} & \cellcolor{blue!5} 7 & \cellcolor{blue!5} \textbf{7} & \cellcolor{blue!5} \textbf{7} & \cellcolor{blue!5} 7 \\
50 92& 7 & \cellcolor{blue!5} \textbf{7} & \cellcolor{blue!5} \textbf{7} & \cellcolor{blue!5} \textbf{7} & \cellcolor{blue!5} 7 & \cellcolor{blue!5} \textbf{7} & \cellcolor{blue!5} \textbf{7} & \cellcolor{blue!5} 7 \\
50 93& 7 & \cellcolor{blue!5} \textbf{7} & \cellcolor{blue!5} \textbf{7} & \cellcolor{blue!5} \textbf{7} & \cellcolor{blue!5} 7 & \cellcolor{blue!5} \textbf{7} & \cellcolor{blue!5} \textbf{7} & \cellcolor{blue!5} 7 \\
50 94& 7 & \cellcolor{blue!5} \textbf{7} & \cellcolor{blue!5} \textbf{7} & \cellcolor{blue!5} \textbf{7} & \cellcolor{blue!5} 7 & \cellcolor{blue!5} \textbf{7} & \cellcolor{blue!5} \textbf{7} & \cellcolor{blue!5} 7 \\
50 95& 7 & \cellcolor{blue!5} \textbf{7} & \cellcolor{blue!5} \textbf{7} & \cellcolor{blue!5} \textbf{7} & \cellcolor{blue!5} 7 & \cellcolor{blue!5} \textbf{7} & \cellcolor{blue!5} \textbf{7} & \cellcolor{blue!5} 7 \\
50 96& 7 & \cellcolor{blue!5} \textbf{7} & \cellcolor{blue!5} \textbf{7} & \cellcolor{blue!5} \textbf{7} & \cellcolor{blue!5} 7 & \cellcolor{blue!5} \textbf{7} & \cellcolor{blue!5} \textbf{7} & \cellcolor{blue!5} 7 \\
50 97& 7 & \cellcolor{blue!5} \textbf{7} & \cellcolor{blue!5} \textbf{7} & \cellcolor{blue!5} \textbf{7} & \cellcolor{blue!5} 7 & \cellcolor{blue!5} \textbf{7} & \cellcolor{blue!5} \textbf{7} & \cellcolor{blue!5} \textbf{7} \\
50 98& 8 & \cellcolor{blue!5} \textbf{8} & \cellcolor{blue!5} \textbf{8} & \cellcolor{blue!5} \textbf{8} & \cellcolor{blue!5} 8 & \cellcolor{blue!5} \textbf{8} & \cellcolor{blue!5} \textbf{8} & \cellcolor{blue!5} 8 \\
50 99& 7 & \cellcolor{blue!5} \textbf{7} & \cellcolor{blue!5} \textbf{7} & \cellcolor{blue!5} \textbf{7} & \cellcolor{blue!5} 7 & \cellcolor{blue!5} \textbf{7} & \cellcolor{blue!5} \textbf{7} & \cellcolor{blue!5} 7 \\
\end{longtable}



\end{document}