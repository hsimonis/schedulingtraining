\documentclass[a4paper]{report}
\usepackage{booktabs}
\usepackage{longtable}
\usepackage{hyperref}
\title{Results for Scheduling Benchmark Classes}
\author{Luis Quesada and Helmut Simonis}
\begin{document}
\maketitle
\begin{abstract}
this reposts lists results of the \emph{tbischeduling} tool for a number of existing benchmarks on scheduling related problems. THe results indicate that depending on the problem type, only a fraction of the benchmarks are solved to optimality, while good or reasonable results are obtained by CPOptimizer of IBM.
\end{abstract}

\tableofcontents

\chapter{Introduction}

The results are obtained by running the TestAll main routine for the different benchmark problems, selecting the necessary parameters and limits for each benchmark type.

The detailed execution time depends on many parameters that are not well controlled in the test environment, so the results should be considered with caution. Tests were run on a Windows 11 laptop using CPOptimizer 22.1.0.


\clearpage
\chapter{Taillard Open Shop Problems}
All problems are solved to optimality, possible due to their small to moderate size.

\section{Results for CPOptimizer}

\begin{longtable}{lrrlrrrr}
\caption{Results for Taillard OpenShop (CPOptimizer) (60 Instances)}\\\toprule
Name & \shortstack{Nr\\Jobs} & \shortstack{Nr\\Machines} & Status & Time & Makespan & Bound & \shortstack{Gap\\Percent}\\ \midrule
\endhead
\bottomrule
\endfoot
tai10 10 0.json & 10 & 10 & Optimal &  0.45 & 637 & 637.00 &  0.00\\
tai10 10 1.json & 10 & 10 & Optimal &  0.06 & 588 & 588.00 &  0.00\\
tai10 10 2.json & 10 & 10 & Optimal &  0.27 & 598 & 598.00 &  0.00\\
tai10 10 3.json & 10 & 10 & Optimal &  0.05 & 577 & 577.00 &  0.00\\
tai10 10 4.json & 10 & 10 & Optimal &  0.05 & 640 & 640.00 &  0.00\\
tai10 10 5.json & 10 & 10 & Optimal &  0.04 & 538 & 538.00 &  0.00\\
tai10 10 6.json & 10 & 10 & Optimal &  0.06 & 616 & 616.00 &  0.00\\
tai10 10 7.json & 10 & 10 & Optimal &  0.11 & 595 & 595.00 &  0.00\\
tai10 10 8.json & 10 & 10 & Optimal &  0.05 & 595 & 595.00 &  0.00\\
tai10 10 9.json & 10 & 10 & Optimal &  0.08 & 596 & 596.00 &  0.00\\
tai15 15 0.json & 15 & 15 & Optimal &  0.11 & 937 & 937.00 &  0.00\\
tai15 15 1.json & 15 & 15 & Optimal &  0.11 & 918 & 918.00 &  0.00\\
tai15 15 2.json & 15 & 15 & Optimal &  0.08 & 871 & 871.00 &  0.00\\
tai15 15 3.json & 15 & 15 & Optimal &  0.13 & 934 & 934.00 &  0.00\\
tai15 15 4.json & 15 & 15 & Optimal &  0.09 & 946 & 946.00 &  0.00\\
tai15 15 5.json & 15 & 15 & Optimal &  0.08 & 933 & 933.00 &  0.00\\
tai15 15 6.json & 15 & 15 & Optimal &  0.16 & 891 & 891.00 &  0.00\\
tai15 15 7.json & 15 & 15 & Optimal &  0.13 & 893 & 893.00 &  0.00\\
tai15 15 8.json & 15 & 15 & Optimal &  0.28 & 899 & 899.00 &  0.00\\
tai15 15 9.json & 15 & 15 & Optimal &  0.17 & 902 & 902.00 &  0.00\\
tai20 20 0.json & 20 & 20 & Optimal &  0.35 & 1155 & 1155.00 &  0.00\\
tai20 20 1.json & 20 & 20 & Optimal &  1.00 & 1241 & 1241.00 &  0.00\\
tai20 20 2.json & 20 & 20 & Optimal &  0.56 & 1257 & 1257.00 &  0.00\\
tai20 20 3.json & 20 & 20 & Optimal &  0.25 & 1248 & 1248.00 &  0.00\\
tai20 20 4.json & 20 & 20 & Optimal &  0.19 & 1256 & 1256.00 &  0.00\\
tai20 20 5.json & 20 & 20 & Optimal &  0.16 & 1204 & 1204.00 &  0.00\\
tai20 20 6.json & 20 & 20 & Optimal &  0.66 & 1294 & 1294.00 &  0.00\\
tai20 20 7.json & 20 & 20 & Optimal &  1.18 & 1169 & 1169.00 &  0.00\\
tai20 20 8.json & 20 & 20 & Optimal &  0.17 & 1289 & 1289.00 &  0.00\\
tai20 20 9.json & 20 & 20 & Optimal &  0.17 & 1241 & 1241.00 &  0.00\\
tai4 4 0.json & 4 & 4 & Optimal &  0.13 & 193 & 193.00 &  0.00\\
tai4 4 1.json & 4 & 4 & Optimal &  0.11 & 236 & 236.00 &  0.00\\
tai4 4 2.json & 4 & 4 & Optimal &  0.08 & 271 & 271.00 &  0.00\\
tai4 4 3.json & 4 & 4 & Optimal &  0.15 & 250 & 250.00 &  0.00\\
tai4 4 4.json & 4 & 4 & Optimal &  0.17 & 295 & 295.00 &  0.00\\
tai4 4 5.json & 4 & 4 & Optimal &  0.05 & 189 & 189.00 &  0.00\\
tai4 4 6.json & 4 & 4 & Optimal &  0.10 & 201 & 201.00 &  0.00\\
tai4 4 7.json & 4 & 4 & Optimal &  0.05 & 217 & 217.00 &  0.00\\
tai4 4 8.json & 4 & 4 & Optimal &  0.13 & 261 & 261.00 &  0.00\\
tai4 4 9.json & 4 & 4 & Optimal &  0.12 & 217 & 217.00 &  0.00\\
tai5 5 0.json & 5 & 5 & Optimal &  0.18 & 300 & 300.00 &  0.00\\
tai5 5 1.json & 5 & 5 & Optimal &  0.16 & 262 & 262.00 &  0.00\\
tai5 5 2.json & 5 & 5 & Optimal &  0.20 & 323 & 323.00 &  0.00\\
tai5 5 3.json & 5 & 5 & Optimal &  0.17 & 310 & 310.00 &  0.00\\
tai5 5 4.json & 5 & 5 & Optimal &  0.27 & 326 & 326.00 &  0.00\\
tai5 5 5.json & 5 & 5 & Optimal &  0.16 & 312 & 312.00 &  0.00\\
tai5 5 6.json & 5 & 5 & Optimal &  0.21 & 303 & 303.00 &  0.00\\
tai5 5 7.json & 5 & 5 & Optimal &  0.25 & 300 & 300.00 &  0.00\\
tai5 5 8.json & 5 & 5 & Optimal &  0.17 & 353 & 353.00 &  0.00\\
tai5 5 9.json & 5 & 5 & Optimal &  0.25 & 326 & 326.00 &  0.00\\
tai7 7 0.json & 7 & 7 & Optimal &  0.03 & 435 & 435.00 &  0.00\\
tai7 7 1.json & 7 & 7 & Optimal &  0.12 & 443 & 443.00 &  0.00\\
tai7 7 2.json & 7 & 7 & Optimal &  0.31 & 468 & 468.00 &  0.00\\
tai7 7 3.json & 7 & 7 & Optimal &  0.03 & 463 & 463.00 &  0.00\\
tai7 7 4.json & 7 & 7 & Optimal &  0.03 & 416 & 416.00 &  0.00\\
tai7 7 5.json & 7 & 7 & Optimal &  0.80 & 451 & 451.00 &  0.00\\
tai7 7 6.json & 7 & 7 & Optimal &  1.10 & 422 & 422.00 &  0.00\\
tai7 7 7.json & 7 & 7 & Optimal &  0.05 & 424 & 424.00 &  0.00\\
tai7 7 8.json & 7 & 7 & Optimal &  0.09 & 458 & 458.00 &  0.00\\
tai7 7 9.json & 7 & 7 & Optimal &  0.06 & 398 & 398.00 &  0.00\\
\end{longtable}



\section{Results for CPSat}

\begin{longtable}{lrrlrrrr}
\caption{Results for Taillard OpenShop (CPSat) (60 Instances)}\\\toprule
Name & \shortstack{Nr\\Jobs} & \shortstack{Nr\\Machines} & Status & Time & Makespan & Bound & \shortstack{Gap\\Percent}\\ \midrule
\endhead
\bottomrule
\endfoot
tai10 10 0.json & 10 & 10 & Optimal &  0.37 & 637 &  0.00 &  0.00\\
tai10 10 1.json & 10 & 10 & Optimal &  0.07 & 588 &  0.00 &  0.00\\
tai10 10 2.json & 10 & 10 & Optimal &  0.16 & 598 &  0.00 &  0.00\\
tai10 10 3.json & 10 & 10 & Optimal &  0.09 & 577 &  0.00 &  0.00\\
tai10 10 4.json & 10 & 10 & Optimal &  0.20 & 640 &  0.00 &  0.00\\
tai10 10 5.json & 10 & 10 & Optimal &  0.13 & 538 &  0.00 &  0.00\\
tai10 10 6.json & 10 & 10 & Optimal &  0.10 & 616 &  0.00 &  0.00\\
tai10 10 7.json & 10 & 10 & Optimal &  0.17 & 595 &  0.00 &  0.00\\
tai10 10 8.json & 10 & 10 & Optimal &  0.11 & 595 &  0.00 &  0.00\\
tai10 10 9.json & 10 & 10 & Optimal &  0.14 & 596 &  0.00 &  0.00\\
tai15 15 0.json & 15 & 15 & Optimal &  0.31 & 937 &  0.00 &  0.00\\
tai15 15 1.json & 15 & 15 & Optimal &  0.45 & 918 &  0.00 &  0.00\\
tai15 15 2.json & 15 & 15 & Optimal &  0.17 & 871 &  0.00 &  0.00\\
tai15 15 3.json & 15 & 15 & Optimal &  0.17 & 934 &  0.00 &  0.00\\
tai15 15 4.json & 15 & 15 & Optimal &  0.27 & 946 &  0.00 &  0.00\\
tai15 15 5.json & 15 & 15 & Optimal &  0.25 & 933 &  0.00 &  0.00\\
tai15 15 6.json & 15 & 15 & Optimal &  0.25 & 891 &  0.00 &  0.00\\
tai15 15 7.json & 15 & 15 & Optimal &  0.32 & 893 &  0.00 &  0.00\\
tai15 15 8.json & 15 & 15 & Optimal &  1.27 & 899 &  0.00 &  0.00\\
tai15 15 9.json & 15 & 15 & Optimal &  0.38 & 902 &  0.00 &  0.00\\
tai20 20 0.json & 20 & 20 & Optimal &  1.01 & 1155 &  0.00 &  0.00\\
tai20 20 1.json & 20 & 20 & Optimal &  2.44 & 1241 &  0.00 &  0.00\\
tai20 20 2.json & 20 & 20 & Optimal &  0.12 & 1257 &  0.00 &  0.00\\
tai20 20 3.json & 20 & 20 & Optimal &  0.35 & 1248 &  0.00 &  0.00\\
tai20 20 4.json & 20 & 20 & Optimal &  0.40 & 1256 &  0.00 &  0.00\\
tai20 20 5.json & 20 & 20 & Optimal &  0.62 & 1204 &  0.00 &  0.00\\
tai20 20 6.json & 20 & 20 & Optimal &  0.52 & 1294 &  0.00 &  0.00\\
tai20 20 7.json & 20 & 20 & Optimal &  2.13 & 1169 &  0.00 &  0.00\\
tai20 20 8.json & 20 & 20 & Optimal &  0.26 & 1289 &  0.00 &  0.00\\
tai20 20 9.json & 20 & 20 & Optimal &  0.65 & 1241 &  0.00 &  0.00\\
tai4 4 0.json & 4 & 4 & Optimal &  0.02 & 193 &  0.00 &  0.00\\
tai4 4 1.json & 4 & 4 & Optimal &  0.03 & 236 &  0.00 &  0.00\\
tai4 4 2.json & 4 & 4 & Optimal &  0.01 & 271 &  0.00 &  0.00\\
tai4 4 3.json & 4 & 4 & Optimal &  0.01 & 250 &  0.00 &  0.00\\
tai4 4 4.json & 4 & 4 & Optimal &  0.03 & 295 &  0.00 &  0.00\\
tai4 4 5.json & 4 & 4 & Optimal &  0.01 & 189 &  0.00 &  0.00\\
tai4 4 6.json & 4 & 4 & Optimal &  0.01 & 201 &  0.00 &  0.00\\
tai4 4 7.json & 4 & 4 & Optimal &  0.01 & 217 &  0.00 &  0.00\\
tai4 4 8.json & 4 & 4 & Optimal &  0.01 & 261 &  0.00 &  0.00\\
tai4 4 9.json & 4 & 4 & Optimal &  0.01 & 217 &  0.00 &  0.00\\
tai5 5 0.json & 5 & 5 & Optimal &  0.06 & 300 &  0.00 &  0.00\\
tai5 5 1.json & 5 & 5 & Optimal &  0.04 & 262 &  0.00 &  0.00\\
tai5 5 2.json & 5 & 5 & Optimal &  0.12 & 323 &  0.00 &  0.00\\
tai5 5 3.json & 5 & 5 & Optimal &  0.07 & 310 &  0.00 &  0.00\\
tai5 5 4.json & 5 & 5 & Optimal &  0.16 & 326 &  0.00 &  0.00\\
tai5 5 5.json & 5 & 5 & Optimal &  0.07 & 312 &  0.00 &  0.00\\
tai5 5 6.json & 5 & 5 & Optimal &  0.09 & 303 &  0.00 &  0.00\\
tai5 5 7.json & 5 & 5 & Optimal &  0.11 & 300 &  0.00 &  0.00\\
tai5 5 8.json & 5 & 5 & Optimal &  0.11 & 353 &  0.00 &  0.00\\
tai5 5 9.json & 5 & 5 & Optimal &  0.11 & 326 &  0.00 &  0.00\\
tai7 7 0.json & 7 & 7 & Optimal &  0.06 & 435 &  0.00 &  0.00\\
tai7 7 1.json & 7 & 7 & Optimal &  0.11 & 443 &  0.00 &  0.00\\
tai7 7 2.json & 7 & 7 & Optimal &  0.15 & 468 &  0.00 &  0.00\\
tai7 7 3.json & 7 & 7 & Optimal &  0.06 & 463 &  0.00 &  0.00\\
tai7 7 4.json & 7 & 7 & Optimal &  0.05 & 416 &  0.00 &  0.00\\
tai7 7 5.json & 7 & 7 & Optimal &  0.48 & 451 &  0.00 &  0.00\\
tai7 7 6.json & 7 & 7 & Optimal &  0.29 & 422 &  0.00 &  0.00\\
tai7 7 7.json & 7 & 7 & Optimal &  0.04 & 424 &  0.00 &  0.00\\
tai7 7 8.json & 7 & 7 & Optimal &  0.05 & 458 &  0.00 &  0.00\\
tai7 7 9.json & 7 & 7 & Optimal &  0.08 & 398 &  0.00 &  0.00\\
\end{longtable}



\clearpage
\chapter{Taillard Job Shop Problems}

The results are rather confusing, as some smaller problems cannot be solved to optimality, while complete groups of larger instances can. The number of jobs clearly is not the only indicator of difficulty of these problems.

\section{Results for CPOptimizer}

\begin{longtable}{lrrlrrrr}
\caption{Results for Taillard JobShop (80 Instances)}\\\toprule
Name & \shortstack{Nr\\Jobs} & \shortstack{Nr\\Machines} & Status & Time & Makespan & Bound & \shortstack{Gap\\Percent}\\ \midrule
\endhead
\bottomrule
\endfoot
tai100 20 0.json & 100 & 20 & Optimal & 266.17 & 5464 & 5464.00 &  0.00\\
tai100 20 1.json & 100 & 20 & Optimal & 31.20 & 5181 & 5181.00 &  0.00\\
tai100 20 2.json & 100 & 20 & Optimal & 38.67 & 5568 & 5568.00 &  0.00\\
tai100 20 3.json & 100 & 20 & Optimal & 177.17 & 5339 & 5339.00 &  0.00\\
tai100 20 4.json & 100 & 20 & Solution & 300.04 & 5412 & 5392.00 &  0.37\\
tai100 20 5.json & 100 & 20 & Optimal & 172.70 & 5342 & 5342.00 &  0.00\\
tai100 20 6.json & 100 & 20 & Optimal & 298.02 & 5436 & 5436.00 &  0.00\\
tai100 20 7.json & 100 & 20 & Optimal & 111.51 & 5394 & 5394.00 &  0.00\\
tai100 20 8.json & 100 & 20 & Optimal & 86.84 & 5358 & 5358.00 &  0.00\\
tai100 20 9.json & 100 & 20 & Optimal & 188.94 & 5183 & 5183.00 &  0.00\\
tai15 15 0.json & 15 & 15 & Optimal & 12.44 & 1231 & 1231.00 &  0.00\\
tai15 15 1.json & 15 & 15 & Optimal & 35.14 & 1244 & 1244.00 &  0.00\\
tai15 15 2.json & 15 & 15 & Optimal & 17.26 & 1218 & 1218.00 &  0.00\\
tai15 15 3.json & 15 & 15 & Optimal & 30.94 & 1175 & 1175.00 &  0.00\\
tai15 15 4.json & 15 & 15 & Optimal & 134.85 & 1224 & 1224.00 &  0.00\\
tai15 15 5.json & 15 & 15 & Solution & 300.02 & 1243 & 1183.00 &  4.83\\
tai15 15 6.json & 15 & 15 & Optimal & 104.11 & 1227 & 1227.00 &  0.00\\
tai15 15 7.json & 15 & 15 & Optimal & 102.79 & 1217 & 1217.00 &  0.00\\
tai15 15 8.json & 15 & 15 & Optimal & 133.01 & 1274 & 1274.00 &  0.00\\
tai15 15 9.json & 15 & 15 & Optimal & 32.78 & 1241 & 1241.00 &  0.00\\
tai20 15 0.json & 20 & 15 & Solution & 300.02 & 1424 & 1274.00 & 10.53\\
tai20 15 1.json & 20 & 15 & Solution & 300.01 & 1378 & 1328.00 &  3.63\\
tai20 15 2.json & 20 & 15 & Solution & 300.02 & 1398 & 1243.00 & 11.09\\
tai20 15 3.json & 20 & 15 & Optimal & 17.73 & 1345 & 1345.00 &  0.00\\
tai20 15 4.json & 20 & 15 & Solution & 300.02 & 1374 & 1270.00 &  7.57\\
tai20 15 5.json & 20 & 15 & Solution & 300.02 & 1389 & 1268.00 &  8.71\\
tai20 15 6.json & 20 & 15 & Optimal & 76.19 & 1462 & 1462.00 &  0.00\\
tai20 15 7.json & 20 & 15 & Solution & 300.02 & 1427 & 1358.00 &  4.84\\
tai20 15 8.json & 20 & 15 & Solution & 300.02 & 1369 & 1258.00 &  8.11\\
tai20 15 9.json & 20 & 15 & Solution & 300.02 & 1406 & 1289.00 &  8.32\\
tai20 20 0.json & 20 & 20 & Solution & 300.02 & 1688 & 1514.00 & 10.31\\
tai20 20 1.json & 20 & 20 & Solution & 300.02 & 1640 & 1454.00 & 11.34\\
tai20 20 2.json & 20 & 20 & Solution & 300.02 & 1585 & 1456.00 &  8.14\\
tai20 20 3.json & 20 & 20 & Solution & 300.01 & 1656 & 1583.00 &  4.41\\
tai20 20 4.json & 20 & 20 & Solution & 300.02 & 1642 & 1474.00 & 10.23\\
tai20 20 5.json & 20 & 20 & Solution & 300.02 & 1663 & 1490.00 & 10.40\\
tai20 20 6.json & 20 & 20 & Solution & 300.02 & 1724 & 1605.00 &  6.90\\
tai20 20 7.json & 20 & 20 & Solution & 300.02 & 1629 & 1564.00 &  3.99\\
tai20 20 8.json & 20 & 20 & Solution & 300.02 & 1675 & 1466.00 & 12.48\\
tai20 20 9.json & 20 & 20 & Solution & 300.01 & 1627 & 1424.00 & 12.48\\
tai30 15 0.json & 30 & 15 & Solution & 300.03 & 1766 & 1764.00 &  0.11\\
tai30 15 1.json & 30 & 15 & Solution & 300.03 & 1860 & 1774.00 &  4.62\\
tai30 15 2.json & 30 & 15 & Solution & 300.03 & 1828 & 1778.00 &  2.74\\
tai30 15 3.json & 30 & 15 & Solution & 300.03 & 1885 & 1828.00 &  3.02\\
tai30 15 4.json & 30 & 15 & Optimal & 15.87 & 2007 & 2007.00 &  0.00\\
tai30 15 5.json & 30 & 15 & Solution & 300.02 & 1852 & 1819.00 &  1.78\\
tai30 15 6.json & 30 & 15 & Solution & 300.03 & 1804 & 1771.00 &  1.83\\
tai30 15 7.json & 30 & 15 & Solution & 300.03 & 1701 & 1673.00 &  1.65\\
tai30 15 8.json & 30 & 15 & Solution & 300.02 & 1821 & 1795.00 &  1.43\\
tai30 15 9.json & 30 & 15 & Solution & 300.02 & 1706 & 1631.00 &  4.40\\
tai30 20 0.json & 30 & 20 & Solution & 300.03 & 2071 & 1857.00 & 10.33\\
tai30 20 1.json & 30 & 20 & Solution & 300.03 & 2006 & 1867.00 &  6.93\\
tai30 20 2.json & 30 & 20 & Solution & 300.04 & 1912 & 1809.00 &  5.39\\
tai30 20 3.json & 30 & 20 & Solution & 300.04 & 2086 & 1923.00 &  7.81\\
tai30 20 4.json & 30 & 20 & Solution & 300.03 & 2014 & 1997.00 &  0.84\\
tai30 20 5.json & 30 & 20 & Solution & 300.03 & 2070 & 1940.00 &  6.28\\
tai30 20 6.json & 30 & 20 & Solution & 300.02 & 1942 & 1783.00 &  8.19\\
tai30 20 7.json & 30 & 20 & Solution & 300.03 & 2027 & 1905.00 &  6.02\\
tai30 20 8.json & 30 & 20 & Solution & 300.03 & 2026 & 1903.00 &  6.07\\
tai30 20 9.json & 30 & 20 & Solution & 300.03 & 2006 & 1806.00 &  9.97\\
tai50 15 0.json & 50 & 15 & Optimal & 90.61 & 2760 & 2760.00 &  0.00\\
tai50 15 1.json & 50 & 15 & Optimal & 52.92 & 2756 & 2756.00 &  0.00\\
tai50 15 2.json & 50 & 15 & Optimal & 16.80 & 2717 & 2717.00 &  0.00\\
tai50 15 3.json & 50 & 15 & Optimal & 10.28 & 2839 & 2839.00 &  0.00\\
tai50 15 4.json & 50 & 15 & Optimal & 35.27 & 2679 & 2679.00 &  0.00\\
tai50 15 5.json & 50 & 15 & Optimal & 66.12 & 2781 & 2781.00 &  0.00\\
tai50 15 6.json & 50 & 15 & Optimal & 15.50 & 2943 & 2943.00 &  0.00\\
tai50 15 7.json & 50 & 15 & Optimal & 30.88 & 2885 & 2885.00 &  0.00\\
tai50 15 8.json & 50 & 15 & Optimal & 35.67 & 2655 & 2655.00 &  0.00\\
tai50 15 9.json & 50 & 15 & Optimal & 37.75 & 2723 & 2723.00 &  0.00\\
tai50 20 0.json & 50 & 20 & Optimal & 165.80 & 2868 & 2868.00 &  0.00\\
tai50 20 1.json & 50 & 20 & Solution & 300.11 & 2907 & 2869.00 &  1.31\\
tai50 20 2.json & 50 & 20 & Solution & 300.10 & 2784 & 2755.00 &  1.04\\
tai50 20 3.json & 50 & 20 & Solution & 300.10 & 2708 & 2702.00 &  0.22\\
tai50 20 4.json & 50 & 20 & Solution & 300.11 & 2738 & 2725.00 &  0.47\\
tai50 20 5.json & 50 & 20 & Optimal & 199.89 & 2845 & 2845.00 &  0.00\\
tai50 20 6.json & 50 & 20 & Solution & 300.12 & 2826 & 2825.00 &  0.04\\
tai50 20 7.json & 50 & 20 & Optimal & 144.63 & 2784 & 2784.00 &  0.00\\
tai50 20 8.json & 50 & 20 & Optimal & 87.12 & 3071 & 3071.00 &  0.00\\
tai50 20 9.json & 50 & 20 & Solution & 300.12 & 3036 & 2995.00 &  1.35\\
\end{longtable}



\section{Results for CPSat}

\begin{longtable}{lrrlrrrr}
\caption{Results for Taillard JobShop (CPSat) (30 Instances)}\\\toprule
Name & \shortstack{Nr\\Jobs} & \shortstack{Nr\\Machines} & Status & Time & Makespan & Bound & \shortstack{Gap\\Percent}\\ \midrule
\endhead
\bottomrule
\endfoot
tai100 20 0.json & 100 & 20 & Solution & 600.40 & 5620 &  0.00 &  0.00\\
tai100 20 1.json & 100 & 20 & Solution & 600.30 & 5280 &  0.00 &  0.00\\
tai100 20 2.json & 100 & 20 & Solution & 601.15 & 5638 &  0.00 &  0.00\\
tai100 20 3.json & 100 & 20 & Solution & 600.36 & 5355 & 5339.00 &  0.00\\
tai100 20 4.json & 100 & 20 & Solution & 600.16 & 5664 & 5392.00 &  0.00\\
tai100 20 5.json & 100 & 20 & Solution & 600.14 & 5433 & 5342.00 &  0.00\\
tai100 20 6.json & 100 & 20 & Solution & 600.47 & 5457 & 5436.00 &  0.00\\
tai100 20 7.json & 100 & 20 & Solution & 600.46 & 5435 & 5394.00 &  0.00\\
tai100 20 8.json & 100 & 20 & Solution & 600.34 & 5397 & 5358.00 &  0.00\\
tai100 20 9.json & 100 & 20 & Solution & 600.43 & 5267 & 5183.00 &  0.00\\
tai15 15 0.json & 15 & 15 & Optimal &  5.32 & 1231 & 1231.00 &  0.00\\
tai15 15 1.json & 15 & 15 & Optimal & 44.73 & 1244 & 1244.00 &  0.00\\
tai15 15 2.json & 15 & 15 & Optimal & 18.76 & 1218 & 1218.00 &  0.00\\
tai15 15 3.json & 15 & 15 & Optimal & 19.12 & 1175 & 1175.00 &  0.00\\
tai15 15 4.json & 15 & 15 & Optimal & 216.74 & 1224 & 1224.00 &  0.00\\
tai15 15 5.json & 15 & 15 & Solution & 600.10 & 1238 & 1202.00 &  2.91\\
tai15 15 6.json & 15 & 15 & Optimal & 246.22 & 1227 & 1227.00 &  0.00\\
tai15 15 7.json & 15 & 15 & Optimal & 186.11 & 1217 & 1217.00 &  0.00\\
tai15 15 8.json & 15 & 15 & Optimal & 134.40 & 1274 & 1274.00 &  0.00\\
tai15 15 9.json & 15 & 15 & Optimal & 20.74 & 1241 & 1241.00 &  0.00\\
tai50 15 0.json & 50 & 15 & Optimal & 186.33 & 2760 & 2760.00 &  0.00\\
tai50 15 1.json & 50 & 15 & Optimal & 155.63 & 2756 & 2756.00 &  0.00\\
tai50 15 2.json & 50 & 15 & Optimal & 68.58 & 2717 & 2717.00 &  0.00\\
tai50 15 3.json & 50 & 15 & Optimal & 26.60 & 2839 & 2839.00 &  0.00\\
tai50 15 4.json & 50 & 15 & Optimal & 362.73 & 2679 & 2679.00 &  0.00\\
tai50 15 5.json & 50 & 15 & Optimal & 249.56 & 2781 & 2781.00 &  0.00\\
tai50 15 6.json & 50 & 15 & Optimal & 120.38 & 2943 & 2943.00 &  0.00\\
tai50 15 7.json & 50 & 15 & Optimal & 216.50 & 2885 & 2885.00 &  0.00\\
tai50 15 8.json & 50 & 15 & Optimal & 435.42 & 2655 & 2655.00 &  0.00\\
tai50 15 9.json & 50 & 15 & Optimal & 217.29 & 2723 & 2723.00 &  0.00\\
\end{longtable}



\section{Sample Results on Mac (CPOptimizer)}
For a selected subset of the tests, we also tried running on a mac laptop, results show some good improvement of the m2 based laptop over the Intel based Windows machine, but the improvements are not consistent.
\begin{longtable}{lrrlrrrr}
\caption{Results for Taillard Jobshop (Selected Instances on Mac)}\\\toprule
Name & \shortstack{Nr\\Jobs} & \shortstack{Nr\\Machines} & Status & Time & Makespan & Bound & \shortstack{Gap\\Percent}\\ \midrule
\endhead
\bottomrule
\endfoot
tai100 20 0.json & 100 & 20 & Optimal & 143.93 & 5464 & 5464.00 &  0.00\\
tai100 20 1.json & 100 & 20 & Optimal & 86.52 & 5181 & 5181.00 &  0.00\\
tai100 20 2.json & 100 & 20 & Optimal & 63.63 & 5568 & 5568.00 &  0.00\\
tai100 20 3.json & 100 & 20 & Optimal & 19.51 & 5339 & 5339.00 &  0.00\\
tai100 20 4.json & 100 & 20 & Optimal & 174.11 & 5392 & 5392.00 &  0.00\\
tai100 20 5.json & 100 & 20 & Optimal & 80.95 & 5342 & 5342.00 &  0.00\\
tai100 20 6.json & 100 & 20 & Optimal & 139.30 & 5436 & 5436.00 &  0.00\\
tai100 20 7.json & 100 & 20 & Optimal & 48.86 & 5394 & 5394.00 &  0.00\\
tai100 20 8.json & 100 & 20 & Optimal & 82.22 & 5358 & 5358.00 &  0.00\\
tai100 20 9.json & 100 & 20 & Optimal & 143.55 & 5183 & 5183.00 &  0.00\\
\end{longtable}



\clearpage
\chapter{Taillard Flow Shop Problems}

These problems seem to be more difficult to solve to optimality. The number of stages seems to make a huge difference, we can solve the problems with five stages (machines) much more easily than the problems with 10 or twenty stages.

\section{Results for CPOptimizer}

\begin{longtable}{lrrlrrrr}
\caption{Results for Taillard Flowshop (120 Instances)}\\\toprule
Name & \shortstack{Nr\\Jobs} & \shortstack{Nr\\Machines} & Status & Time & Makespan & Bound & \shortstack{Gap\\Percent}\\ \midrule
\endhead
\bottomrule
\endfoot
tai100 10 0.json & 100 & 10 & Solution & 600.16 & 5979 & 5759.00 &  3.68\\
tai100 10 1.json & 100 & 10 & Solution & 600.06 & 5418 & 5345.00 &  1.35\\
tai100 10 2.json & 100 & 10 & Solution & 600.06 & 5798 & 5646.00 &  2.62\\
tai100 10 3.json & 100 & 10 & Solution & 600.03 & 6040 & 5737.00 &  5.02\\
tai100 10 4.json & 100 & 10 & Solution & 600.02 & 5663 & 5431.00 &  4.10\\
tai100 10 5.json & 100 & 10 & Solution & 600.05 & 5378 & 5274.00 &  1.93\\
tai100 10 6.json & 100 & 10 & Solution & 600.04 & 5697 & 5553.00 &  2.53\\
tai100 10 7.json & 100 & 10 & Solution & 600.03 & 5813 & 5575.00 &  4.09\\
tai100 10 8.json & 100 & 10 & Solution & 600.04 & 5983 & 5838.00 &  2.42\\
tai100 10 9.json & 100 & 10 & Solution & 600.02 & 5903 & 5835.00 &  1.15\\
tai100 20 0.json & 100 & 20 & Solution & 600.07 & 6731 & 5914.00 & 12.14\\
tai100 20 1.json & 100 & 20 & Solution & 600.05 & 6840 & 6115.00 & 10.60\\
tai100 20 2.json & 100 & 20 & Solution & 600.06 & 6778 & 6139.00 &  9.43\\
tai100 20 3.json & 100 & 20 & Solution & 600.06 & 6720 & 6117.00 &  8.97\\
tai100 20 4.json & 100 & 20 & Solution & 600.05 & 6853 & 6148.00 & 10.29\\
tai100 20 5.json & 100 & 20 & Solution & 600.07 & 6989 & 6192.00 & 11.40\\
tai100 20 6.json & 100 & 20 & Solution & 600.04 & 6772 & 6045.00 & 10.74\\
tai100 20 7.json & 100 & 20 & Solution & 600.05 & 6940 & 6113.00 & 11.92\\
tai100 20 8.json & 100 & 20 & Solution & 600.04 & 7092 & 6014.00 & 15.20\\
tai100 20 9.json & 100 & 20 & Solution & 600.06 & 6871 & 6359.00 &  7.45\\
tai100 5 0.json & 100 & 5 & Optimal & 73.90 & 5493 & 5493.00 &  0.00\\
tai100 5 1.json & 100 & 5 & Solution & 600.13 & 5276 & 5232.00 &  0.83\\
tai100 5 2.json & 100 & 5 & Solution & 600.11 & 5178 & 5170.00 &  0.15\\
tai100 5 3.json & 100 & 5 & Solution & 600.12 & 4996 & 4993.00 &  0.06\\
tai100 5 4.json & 100 & 5 & Optimal & 79.87 & 5247 & 5247.00 &  0.00\\
tai100 5 5.json & 100 & 5 & Optimal & 231.03 & 5135 & 5135.00 &  0.00\\
tai100 5 6.json & 100 & 5 & Optimal & 95.07 & 5232 & 5232.00 &  0.00\\
tai100 5 7.json & 100 & 5 & Optimal & 293.37 & 5083 & 5083.00 &  0.00\\
tai100 5 8.json & 100 & 5 & Solution & 600.12 & 5464 & 5438.00 &  0.48\\
tai100 5 9.json & 100 & 5 & Optimal & 70.58 & 5318 & 5318.00 &  0.00\\
tai200 10 0.json & 200 & 10 & Solution & 600.06 & 11136 & 10842.00 &  2.64\\
tai200 10 1.json & 200 & 10 & Solution & 600.04 & 10981 & 10429.00 &  5.03\\
tai200 10 2.json & 200 & 10 & Solution & 600.06 & 11276 & 10915.00 &  3.20\\
tai200 10 3.json & 200 & 10 & Solution & 600.06 & 11217 & 10826.00 &  3.49\\
tai200 10 4.json & 200 & 10 & Solution & 600.06 & 11139 & 10474.00 &  5.97\\
tai200 10 5.json & 200 & 10 & Solution & 600.06 & 10828 & 10311.00 &  4.77\\
tai200 10 6.json & 200 & 10 & Solution & 600.07 & 11202 & 10825.00 &  3.37\\
tai200 10 7.json & 200 & 10 & Solution & 600.07 & 11287 & 10709.00 &  5.12\\
tai200 10 8.json & 200 & 10 & Solution & 600.07 & 11004 & 10419.00 &  5.32\\
tai200 10 9.json & 200 & 10 & Solution & 600.05 & 11178 & 10664.00 &  4.60\\
tai200 20 0.json & 200 & 20 & Solution & 600.11 & 12439 & 11010.00 & 11.49\\
tai200 20 1.json & 200 & 20 & Solution & 600.12 & 12878 & 10976.00 & 14.77\\
tai200 20 2.json & 200 & 20 & Solution & 600.12 & 12506 & 11168.00 & 10.70\\
tai200 20 3.json & 200 & 20 & Solution & 600.09 & 12618 & 11131.00 & 11.78\\
tai200 20 4.json & 200 & 20 & Solution & 600.06 & 12649 & 11160.00 & 11.77\\
tai200 20 5.json & 200 & 20 & Solution & 600.13 & 12740 & 11114.00 & 12.76\\
tai200 20 6.json & 200 & 20 & Solution & 600.10 & 12985 & 11249.00 & 13.37\\
tai200 20 7.json & 200 & 20 & Solution & 600.08 & 12724 & 11149.00 & 12.38\\
tai200 20 8.json & 200 & 20 & Solution & 600.07 & 12693 & 11013.00 & 13.24\\
tai200 20 9.json & 200 & 20 & Solution & 600.07 & 12869 & 11167.00 & 13.23\\
tai20 10 0.json & 20 & 10 & Solution & 600.01 & 1575 & 1494.00 &  5.14\\
tai20 10 1.json & 20 & 10 & Solution & 600.02 & 1648 & 1555.00 &  5.64\\
tai20 10 2.json & 20 & 10 & Solution & 600.04 & 1477 & 1430.00 &  3.18\\
tai20 10 3.json & 20 & 10 & Optimal & 170.76 & 1356 & 1356.00 &  0.00\\
tai20 10 4.json & 20 & 10 & Solution & 600.04 & 1403 & 1353.00 &  3.56\\
tai20 10 5.json & 20 & 10 & Solution & 600.03 & 1374 & 1352.00 &  1.60\\
tai20 10 6.json & 20 & 10 & Solution & 600.03 & 1446 & 1388.00 &  4.01\\
tai20 10 7.json & 20 & 10 & Solution & 600.02 & 1579 & 1407.00 & 10.89\\
tai20 10 8.json & 20 & 10 & Optimal & 74.54 & 1586 & 1586.00 &  0.00\\
tai20 10 9.json & 20 & 10 & Solution & 600.02 & 1600 & 1473.00 &  7.94\\
tai20 20 0.json & 20 & 20 & Solution & 600.04 & 2284 & 1970.00 & 13.75\\
tai20 20 1.json & 20 & 20 & Solution & 600.05 & 2109 & 1784.00 & 15.41\\
tai20 20 2.json & 20 & 20 & Solution & 600.05 & 2341 & 1924.00 & 17.81\\
tai20 20 3.json & 20 & 20 & Solution & 600.04 & 2266 & 1892.00 & 16.50\\
tai20 20 4.json & 20 & 20 & Solution & 600.04 & 2311 & 1994.00 & 13.72\\
tai20 20 5.json & 20 & 20 & Solution & 600.03 & 2215 & 1899.00 & 14.27\\
tai20 20 6.json & 20 & 20 & Solution & 600.03 & 2265 & 1952.00 & 13.82\\
tai20 20 7.json & 20 & 20 & Solution & 600.04 & 2206 & 1931.00 & 12.47\\
tai20 20 8.json & 20 & 20 & Solution & 600.03 & 2233 & 1900.00 & 14.91\\
tai20 20 9.json & 20 & 20 & Solution & 600.03 & 2181 & 1939.00 & 11.10\\
tai20 5 0.json & 20 & 5 & Optimal &  3.03 & 1278 & 1278.00 &  0.00\\
tai20 5 1.json & 20 & 5 & Optimal &  2.20 & 1358 & 1358.00 &  0.00\\
tai20 5 2.json & 20 & 5 & Optimal &  2.72 & 1073 & 1073.00 &  0.00\\
tai20 5 3.json & 20 & 5 & Optimal &  3.28 & 1292 & 1292.00 &  0.00\\
tai20 5 4.json & 20 & 5 & Optimal &  5.46 & 1231 & 1231.00 &  0.00\\
tai20 5 5.json & 20 & 5 & Optimal &  2.45 & 1193 & 1193.00 &  0.00\\
tai20 5 6.json & 20 & 5 & Optimal &  1.89 & 1234 & 1234.00 &  0.00\\
tai20 5 7.json & 20 & 5 & Optimal &  3.99 & 1199 & 1199.00 &  0.00\\
tai20 5 8.json & 20 & 5 & Optimal &  1.72 & 1210 & 1210.00 &  0.00\\
tai20 5 9.json & 20 & 5 & Optimal &  1.76 & 1103 & 1103.00 &  0.00\\
tai500 20 0.json & 500 & 20 & Solution & 600.23 & 28669 & 25931.00 &  9.55\\
tai500 20 1.json & 500 & 20 & Solution & 600.18 & 29047 & 26390.00 &  9.15\\
tai500 20 2.json & 500 & 20 & Solution & 600.22 & 28796 & 26330.00 &  8.56\\
tai500 20 3.json & 500 & 20 & Solution & 600.18 & 28907 & 26456.00 &  8.48\\
tai500 20 4.json & 500 & 20 & Solution & 600.21 & 28841 & 26205.00 &  9.14\\
tai500 20 5.json & 500 & 20 & Solution & 600.19 & 28954 & 26436.00 &  8.70\\
tai500 20 6.json & 500 & 20 & Solution & 600.17 & 28750 & 26329.00 &  8.42\\
tai500 20 7.json & 500 & 20 & Solution & 600.17 & 28924 & 26451.00 &  8.55\\
tai500 20 8.json & 500 & 20 & Solution & 600.19 & 28038 & 25929.00 &  7.52\\
tai500 20 9.json & 500 & 20 & Solution & 600.22 & 28726 & 26355.00 &  8.25\\
tai50 10 0.json & 50 & 10 & Solution & 600.07 & 3055 & 2962.00 &  3.04\\
tai50 10 1.json & 50 & 10 & Solution & 600.10 & 2929 & 2829.00 &  3.41\\
tai50 10 2.json & 50 & 10 & Solution & 600.11 & 2908 & 2825.00 &  2.85\\
tai50 10 3.json & 50 & 10 & Solution & 600.06 & 3125 & 3038.00 &  2.78\\
tai50 10 4.json & 50 & 10 & Solution & 600.06 & 3100 & 2923.00 &  5.71\\
tai50 10 5.json & 50 & 10 & Solution & 600.11 & 3065 & 2966.00 &  3.23\\
tai50 10 6.json & 50 & 10 & Solution & 600.06 & 3133 & 3063.00 &  2.23\\
tai50 10 7.json & 50 & 10 & Solution & 600.11 & 3096 & 3000.00 &  3.10\\
tai50 10 8.json & 50 & 10 & Solution & 600.09 & 2959 & 2832.00 &  4.29\\
tai50 10 9.json & 50 & 10 & Solution & 600.08 & 3111 & 3046.00 &  2.09\\
tai50 20 0.json & 50 & 20 & Solution & 600.17 & 4003 & 3562.00 & 11.02\\
tai50 20 1.json & 50 & 20 & Solution & 600.20 & 4041 & 3533.00 & 12.57\\
tai50 20 2.json & 50 & 20 & Solution & 600.18 & 3972 & 3412.00 & 14.10\\
tai50 20 3.json & 50 & 20 & Solution & 600.16 & 3922 & 3383.00 & 13.74\\
tai50 20 4.json & 50 & 20 & Solution & 600.15 & 3830 & 3387.00 & 11.57\\
tai50 20 5.json & 50 & 20 & Solution & 600.19 & 3861 & 3499.00 &  9.38\\
tai50 20 6.json & 50 & 20 & Solution & 600.21 & 3930 & 3461.00 & 11.93\\
tai50 20 7.json & 50 & 20 & Solution & 60.25 & 4056 & 3411.00 & 15.90\\
tai50 20 8.json & 50 & 20 & Solution & 60.17 & 4115 & 3468.00 & 15.72\\
tai50 20 9.json & 50 & 20 & Solution & 60.19 & 3986 & 3474.00 & 12.84\\
tai50 5 0.json & 50 & 5 & Optimal & 17.70 & 2724 & 2724.00 &  0.00\\
tai50 5 1.json & 50 & 5 & Optimal & 57.29 & 2834 & 2834.00 &  0.00\\
tai50 5 2.json & 50 & 5 & Optimal & 47.53 & 2612 & 2612.00 &  0.00\\
tai50 5 3.json & 50 & 5 & Optimal & 11.58 & 2751 & 2751.00 &  0.00\\
tai50 5 4.json & 50 & 5 & Optimal & 23.50 & 2853 & 2853.00 &  0.00\\
tai50 5 5.json & 50 & 5 & Optimal & 28.67 & 2825 & 2825.00 &  0.00\\
tai50 5 6.json & 50 & 5 & Optimal & 42.14 & 2716 & 2716.00 &  0.00\\
tai50 5 7.json & 50 & 5 & Optimal & 19.26 & 2683 & 2683.00 &  0.00\\
tai50 5 8.json & 50 & 5 & Optimal & 56.32 & 2545 & 2545.00 &  0.00\\
tai50 5 9.json & 50 & 5 & Optimal &  5.82 & 2776 & 2776.00 &  0.00\\
\end{longtable}



\section{Results for CPSat}

\begin{longtable}{lrrlrrrr}
\caption{Results for Taillard Flowshop (CPSat) (120 Instances)}\\\toprule
Name & \shortstack{Nr\\Jobs} & \shortstack{Nr\\Machines} & Status & Time & Makespan & Bound & \shortstack{Gap\\Percent}\\ \midrule
\endhead
\bottomrule
\endfoot
tai100 10 0.json & 100 & 10 & Solution & 600.16 & 6170 & 5759.00 &  6.66\\
tai100 10 1.json & 100 & 10 & Solution & 600.16 & 5813 & 4707.00 & 19.03\\
tai100 10 2.json & 100 & 10 & Solution & 600.31 & 6133 & 5543.00 &  9.62\\
tai100 10 3.json & 100 & 10 & Solution & 600.14 & 6464 & 5716.00 & 11.57\\
tai100 10 4.json & 100 & 10 & Solution & 600.19 & 6143 & 5153.00 & 16.12\\
tai100 10 5.json & 100 & 10 & Solution & 600.21 & 5844 & 4823.00 & 17.47\\
tai100 10 6.json & 100 & 10 & Solution & 600.21 & 5949 & 5179.00 & 12.94\\
tai100 10 7.json & 100 & 10 & Solution & 600.17 & 6180 & 5237.00 & 15.26\\
tai100 10 8.json & 100 & 10 & Solution & 600.18 & 6341 & 5844.00 &  7.84\\
tai100 10 9.json & 100 & 10 & Solution & 600.35 & 6317 & 5254.00 & 16.83\\
tai100 20 0.json & 100 & 20 & Solution & 600.28 & 7389 & 4511.00 & 38.95\\
tai100 20 1.json & 100 & 20 & Solution & 600.23 & 7196 & 4343.00 & 39.65\\
tai100 20 2.json & 100 & 20 & Solution & 600.21 & 7408 & 4853.00 & 34.49\\
tai100 20 3.json & 100 & 20 & Solution & 600.22 & 7050 & 4667.00 & 33.80\\
tai100 20 4.json & 100 & 20 & Solution & 600.21 & 7220 & 4908.00 & 32.02\\
tai100 20 5.json & 100 & 20 & Solution & 600.28 & 7646 & 4680.00 & 38.79\\
tai100 20 6.json & 100 & 20 & Solution & 600.24 & 7164 & 4457.00 & 37.79\\
tai100 20 7.json & 100 & 20 & Solution & 600.22 & 7691 & 4792.00 & 37.69\\
tai100 20 8.json & 100 & 20 & Solution & 600.24 & 7445 & 4852.00 & 34.83\\
tai100 20 9.json & 100 & 20 & Solution & 600.22 & 7429 & 5405.00 & 27.24\\
tai100 5 0.json & 100 & 5 & Optimal & 600.07 & 5493 & 5493.00 &  0.00\\
tai100 5 1.json & 100 & 5 & Solution & 600.36 & 5294 & 5240.00 &  1.02\\
tai100 5 2.json & 100 & 5 & Solution & 600.16 & 5216 & 5173.00 &  0.82\\
tai100 5 3.json & 100 & 5 & Solution & 600.13 & 5001 & 4993.00 &  0.16\\
tai100 5 4.json & 100 & 5 & Solution & 600.36 & 5279 & 5247.00 &  0.61\\
tai100 5 5.json & 100 & 5 & Optimal & 600.07 & 5135 & 5135.00 &  0.00\\
tai100 5 6.json & 100 & 5 & Solution & 600.20 & 5281 & 5228.00 &  1.00\\
tai100 5 7.json & 100 & 5 & Solution & 600.22 & 5137 & 5083.00 &  1.05\\
tai100 5 8.json & 100 & 5 & Solution & 600.16 & 5481 & 5442.00 &  0.71\\
tai100 5 9.json & 100 & 5 & Solution & 600.17 & 5346 & 5305.00 &  0.77\\
tai200 10 0.json & 200 & 10 & Solution & 600.35 & 11993 & 6397.00 & 46.66\\
tai200 10 1.json & 200 & 10 & Solution & 600.27 & 12255 & 6257.00 & 48.94\\
tai200 10 2.json & 200 & 10 & Solution & 600.37 & 11987 & 6508.00 & 45.71\\
tai200 10 3.json & 200 & 10 & Solution & 600.35 & 11934 & 6160.00 & 48.38\\
tai200 10 4.json & 200 & 10 & Solution & 600.32 & 11982 & 6309.00 & 47.35\\
tai200 10 5.json & 200 & 10 & Solution & 600.35 & 11666 & 6156.00 & 47.23\\
tai200 10 6.json & 200 & 10 & Solution & 600.38 & 12236 & 6001.00 & 50.96\\
tai200 10 7.json & 200 & 10 & Solution & 600.32 & 12186 & 6214.00 & 49.01\\
tai200 10 8.json & 200 & 10 & Solution & 600.35 & 11818 & 6108.00 & 48.32\\
tai200 10 9.json & 200 & 10 & Solution & 600.29 & 11764 & 6496.00 & 44.78\\
tai200 20 0.json & 200 & 20 & Solution & 600.37 & 13270 & 11010.00 & 17.03\\
tai200 20 1.json & 200 & 20 & Solution & 600.36 & 13276 & 6170.00 & 53.53\\
tai200 20 2.json & 200 & 20 & Solution & 600.38 & 13303 & 6157.00 & 53.72\\
tai200 20 3.json & 200 & 20 & Solution & 600.37 & 13647 & 6224.00 & 54.39\\
tai200 20 4.json & 200 & 20 & Solution & 600.35 & 13271 & 6162.00 & 53.57\\
tai200 20 5.json & 200 & 20 & Solution & 602.34 & 13608 & 6257.00 & 54.02\\
tai200 20 6.json & 200 & 20 & Solution & 601.71 & 13515 & 6352.00 & 53.00\\
tai200 20 7.json & 200 & 20 & Solution & 600.40 & 13193 & 6242.00 & 52.69\\
tai200 20 8.json & 200 & 20 & Solution & 600.37 & 13298 & 6210.00 & 53.30\\
tai200 20 9.json & 200 & 20 & Solution & 600.39 & 13355 & 6194.00 & 53.62\\
tai20 10 0.json & 20 & 10 & Solution & 600.16 & 1579 & 1547.00 &  2.03\\
tai20 10 1.json & 20 & 10 & Solution & 600.16 & 1686 & 1587.00 &  5.87\\
tai20 10 2.json & 20 & 10 & Solution & 600.12 & 1481 & 1438.00 &  2.90\\
tai20 10 3.json & 20 & 10 & Solution & 600.16 & 1400 & 1356.00 &  3.14\\
tai20 10 4.json & 20 & 10 & Solution & 600.29 & 1411 & 1360.00 &  3.61\\
tai20 10 5.json & 20 & 10 & Solution & 600.45 & 1374 & 1356.00 &  1.31\\
tai20 10 6.json & 20 & 10 & Solution & 600.14 & 1446 & 1398.00 &  3.32\\
tai20 10 7.json & 20 & 10 & Solution & 600.15 & 1548 & 1448.00 &  6.46\\
tai20 10 8.json & 20 & 10 & Optimal & 600.04 & 1586 & 1586.00 &  0.00\\
tai20 10 9.json & 20 & 10 & Solution & 600.11 & 1590 & 1529.00 &  3.84\\
tai20 20 0.json & 20 & 20 & Solution & 600.11 & 2265 & 2047.00 &  9.62\\
tai20 20 1.json & 20 & 20 & Solution & 600.28 & 2125 & 1844.00 & 13.22\\
tai20 20 2.json & 20 & 20 & Solution & 600.11 & 2349 & 1993.00 & 15.16\\
tai20 20 3.json & 20 & 20 & Solution & 600.14 & 2281 & 1957.00 & 14.20\\
tai20 20 4.json & 20 & 20 & Solution & 600.12 & 2376 & 2058.00 & 13.38\\
tai20 20 5.json & 20 & 20 & Solution & 600.11 & 2176 & 1974.00 &  9.28\\
tai20 20 6.json & 20 & 20 & Solution & 600.12 & 2320 & 2001.00 & 13.75\\
tai20 20 7.json & 20 & 20 & Solution & 600.12 & 2247 & 1982.00 & 11.79\\
tai20 20 8.json & 20 & 20 & Solution & 600.11 & 2267 & 1960.00 & 13.54\\
tai20 20 9.json & 20 & 20 & Solution & 600.12 & 2176 & 1971.00 &  9.42\\
tai20 5 0.json & 20 & 5 & Optimal & 409.17 & 1278 & 1278.00 &  0.00\\
tai20 5 1.json & 20 & 5 & Optimal & 12.58 & 1358 & 1358.00 &  0.00\\
tai20 5 2.json & 20 & 5 & Optimal &  2.90 & 1073 & 1073.00 &  0.00\\
tai20 5 3.json & 20 & 5 & Optimal &  8.44 & 1292 & 1292.00 &  0.00\\
tai20 5 4.json & 20 & 5 & Optimal & 56.24 & 1231 & 1231.00 &  0.00\\
tai20 5 5.json & 20 & 5 & Optimal & 600.01 & 1193 & 1193.00 &  0.00\\
tai20 5 6.json & 20 & 5 & Optimal &  3.20 & 1234 & 1234.00 &  0.00\\
tai20 5 7.json & 20 & 5 & Optimal & 13.83 & 1199 & 1199.00 &  0.00\\
tai20 5 8.json & 20 & 5 & Optimal &  8.57 & 1210 & 1210.00 &  0.00\\
tai20 5 9.json & 20 & 5 & Optimal &  3.37 & 1103 & 1103.00 &  0.00\\
tai500 20 0.json & 500 & 20 & Solution & 601.11 & 30220 & 13561.00 & 55.13\\
tai500 20 1.json & 500 & 20 & Solution & 602.31 & 30765 & 13909.00 & 54.79\\
tai500 20 2.json & 500 & 20 & Solution & 609.69 & 30517 & 13847.00 & 54.63\\
tai500 20 3.json & 500 & 20 & Solution & 603.09 & 30572 & 13410.00 & 56.14\\
tai500 20 4.json & 500 & 20 & Solution & 601.39 & 30483 & 13439.00 & 55.91\\
tai500 20 5.json & 500 & 20 & Solution & 601.50 & 30843 & 13725.00 & 55.50\\
tai500 20 6.json & 500 & 20 & Solution & 603.24 & 30714 & 13837.00 & 54.95\\
tai500 20 7.json & 500 & 20 & Solution & 601.54 & 30625 & 13932.00 & 54.51\\
tai500 20 8.json & 500 & 20 & Solution & 609.39 & 30367 & 13646.00 & 55.06\\
tai500 20 9.json & 500 & 20 & Solution & 601.57 & 30643 & 13782.00 & 55.02\\
tai50 10 0.json & 50 & 10 & Solution & 600.22 & 3162 & 2976.00 &  5.88\\
tai50 10 1.json & 50 & 10 & Solution & 600.32 & 3048 & 2829.00 &  7.19\\
tai50 10 2.json & 50 & 10 & Solution & 600.19 & 2962 & 2830.00 &  4.46\\
tai50 10 3.json & 50 & 10 & Solution & 600.17 & 3166 & 3059.00 &  3.38\\
tai50 10 4.json & 50 & 10 & Solution & 600.20 & 3093 & 2933.00 &  5.17\\
tai50 10 5.json & 50 & 10 & Solution & 600.16 & 3155 & 2986.00 &  5.36\\
tai50 10 6.json & 50 & 10 & Solution & 600.44 & 3201 & 3093.00 &  3.37\\
tai50 10 7.json & 50 & 10 & Solution & 600.32 & 3184 & 3003.00 &  5.68\\
tai50 10 8.json & 50 & 10 & Solution & 600.20 & 3004 & 2864.00 &  4.66\\
tai50 10 9.json & 50 & 10 & Solution & 600.21 & 3192 & 3046.00 &  4.57\\
tai50 20 0.json & 50 & 20 & Solution & 600.19 & 4301 & 3591.00 & 16.51\\
tai50 20 1.json & 50 & 20 & Solution & 600.21 & 4085 & 3554.00 & 13.00\\
tai50 20 2.json & 50 & 20 & Solution & 600.24 & 4227 & 3431.00 & 18.83\\
tai50 20 3.json & 50 & 20 & Solution & 600.21 & 4203 & 3419.00 & 18.65\\
tai50 20 4.json & 50 & 20 & Solution & 600.18 & 4100 & 3415.00 & 16.71\\
tai50 20 5.json & 50 & 20 & Solution & 600.17 & 4109 & 3516.00 & 14.43\\
tai50 20 6.json & 50 & 20 & Solution & 600.22 & 4079 & 3494.00 & 14.34\\
tai50 20 7.json & 50 & 20 & Solution & 600.38 & 4129 & 3456.00 & 16.30\\
tai50 20 8.json & 50 & 20 & Solution & 600.43 & 4143 & 3489.00 & 15.79\\
tai50 20 9.json & 50 & 20 & Solution & 600.24 & 4300 & 3520.00 & 18.14\\
tai50 5 0.json & 50 & 5 & Optimal & 600.05 & 2724 & 2724.00 &  0.00\\
tai50 5 1.json & 50 & 5 & Optimal & 600.03 & 2834 & 2834.00 &  0.00\\
tai50 5 2.json & 50 & 5 & Optimal & 166.99 & 2612 & 2612.00 &  0.00\\
tai50 5 3.json & 50 & 5 & Optimal & 193.35 & 2751 & 2751.00 &  0.00\\
tai50 5 4.json & 50 & 5 & Optimal & 600.03 & 2853 & 2853.00 &  0.00\\
tai50 5 5.json & 50 & 5 & Optimal & 600.03 & 2825 & 2825.00 &  0.00\\
tai50 5 6.json & 50 & 5 & Optimal & 600.06 & 2716 & 2716.00 &  0.00\\
tai50 5 7.json & 50 & 5 & Optimal & 600.03 & 2683 & 2683.00 &  0.00\\
tai50 5 8.json & 50 & 5 & Solution & 600.20 & 2549 & 2545.00 &  0.16\\
tai50 5 9.json & 50 & 5 & Optimal & 600.03 & 2776 & 2776.00 &  0.00\\
\end{longtable}



\section{Permutation Flowshop Results for CPOptimizer}

We can run the flowshop benchmarks with an additional constraint to be solved as a permutation flowshop, which dramatically reduces the sets of feasible solutions, and the search tree to be searched. This might results in improved solutions found as a larger part of that search space can be explored, but solutions can be worse than for the original problem. In particular the optimal solution for the permutation flowshop can be worse than a good feasible solution for the unrestricted flowshop.

\begin{longtable}{lrrlrrrr}
\caption{Results for Taillard Permutation Flowshop (120 Instances)}\\\toprule
Name & \shortstack{Nr\\Jobs} & \shortstack{Nr\\Machines} & Status & Time & Makespan & Bound & \shortstack{Gap\\Percent}\\ \midrule
\endhead
\bottomrule
\endfoot
tai100 10 0.json & 100 & 10 & Solution & 600.34 & 5789 & 5766.00 &  0.40\\
tai100 10 1.json & 100 & 10 & Solution & 600.07 & 5391 & 5347.00 &  0.82\\
tai100 10 2.json & 100 & 10 & Solution & 600.08 & 5691 & 5659.00 &  0.56\\
tai100 10 3.json & 100 & 10 & Solution & 600.06 & 5860 & 5776.00 &  1.43\\
tai100 10 4.json & 100 & 10 & Solution & 600.05 & 5513 & 5450.00 &  1.14\\
tai100 10 5.json & 100 & 10 & Solution & 600.03 & 5308 & 5290.00 &  0.34\\
tai100 10 6.json & 100 & 10 & Solution & 600.03 & 5647 & 5556.00 &  1.61\\
tai100 10 7.json & 100 & 10 & Solution & 600.03 & 5689 & 5586.00 &  1.81\\
tai100 10 8.json & 100 & 10 & Solution & 600.05 & 5903 & 5865.00 &  0.64\\
tai100 10 9.json & 100 & 10 & Solution & 600.04 & 5860 & 5837.00 &  0.39\\
tai100 20 0.json & 100 & 20 & Solution & 600.05 & 6526 & 5936.00 &  9.04\\
tai100 20 1.json & 100 & 20 & Solution & 600.07 & 6390 & 6122.00 &  4.19\\
tai100 20 2.json & 100 & 20 & Solution & 600.07 & 6481 & 6162.00 &  4.92\\
tai100 20 3.json & 100 & 20 & Solution & 600.08 & 6463 & 6163.00 &  4.64\\
tai100 20 4.json & 100 & 20 & Solution & 600.05 & 6497 & 6161.00 &  5.17\\
tai100 20 5.json & 100 & 20 & Solution & 600.05 & 6554 & 6203.00 &  5.36\\
tai100 20 6.json & 100 & 20 & Solution & 600.07 & 6483 & 6061.00 &  6.51\\
tai100 20 7.json & 100 & 20 & Solution & 600.08 & 6670 & 6190.00 &  7.20\\
tai100 20 8.json & 100 & 20 & Solution & 600.05 & 6577 & 6063.00 &  7.82\\
tai100 20 9.json & 100 & 20 & Solution & 600.06 & 6684 & 6382.00 &  4.52\\
tai100 5 0.json & 100 & 5 & Optimal &  4.06 & 5493 & 5493.00 &  0.00\\
tai100 5 1.json & 100 & 5 & Optimal & 67.53 & 5268 & 5268.00 &  0.00\\
tai100 5 2.json & 100 & 5 & Optimal &  7.66 & 5175 & 5175.00 &  0.00\\
tai100 5 3.json & 100 & 5 & Optimal & 60.38 & 5014 & 5014.00 &  0.00\\
tai100 5 4.json & 100 & 5 & Optimal & 62.17 & 5250 & 5250.00 &  0.00\\
tai100 5 5.json & 100 & 5 & Optimal &  6.22 & 5135 & 5135.00 &  0.00\\
tai100 5 6.json & 100 & 5 & Optimal &  9.45 & 5246 & 5246.00 &  0.00\\
tai100 5 7.json & 100 & 5 & Optimal &  9.90 & 5094 & 5094.00 &  0.00\\
tai100 5 8.json & 100 & 5 & Optimal & 65.13 & 5448 & 5448.00 &  0.00\\
tai100 5 9.json & 100 & 5 & Optimal & 67.74 & 5322 & 5322.00 &  0.00\\
tai200 10 0.json & 200 & 10 & Solution & 600.05 & 10918 & 10861.00 &  0.52\\
tai200 10 1.json & 200 & 10 & Solution & 600.07 & 10718 & 10447.00 &  2.53\\
tai200 10 2.json & 200 & 10 & Solution & 600.05 & 11060 & 10920.00 &  1.27\\
tai200 10 3.json & 200 & 10 & Solution & 600.07 & 10934 & 10846.00 &  0.80\\
tai200 10 4.json & 200 & 10 & Solution & 600.08 & 10626 & 10494.00 &  1.24\\
tai200 10 5.json & 200 & 10 & Solution & 600.07 & 10453 & 10312.00 &  1.35\\
tai200 10 6.json & 200 & 10 & Solution & 600.07 & 10979 & 10853.00 &  1.15\\
tai200 10 7.json & 200 & 10 & Solution & 600.07 & 10856 & 10715.00 &  1.30\\
tai200 10 8.json & 200 & 10 & Solution & 600.06 & 10558 & 10422.00 &  1.29\\
tai200 10 9.json & 200 & 10 & Solution & 600.05 & 10761 & 10666.00 &  0.88\\
tai200 20 0.json & 200 & 20 & Solution & 600.13 & 11928 & 11048.00 &  7.38\\
tai200 20 1.json & 200 & 20 & Solution & 600.09 & 11991 & 11009.00 &  8.19\\
tai200 20 2.json & 200 & 20 & Solution & 600.09 & 12248 & 11217.00 &  8.42\\
tai200 20 3.json & 200 & 20 & Solution & 600.12 & 11967 & 11179.00 &  6.58\\
tai200 20 4.json & 200 & 20 & Solution & 600.13 & 11915 & 11168.00 &  6.27\\
tai200 20 5.json & 200 & 20 & Solution & 600.08 & 11923 & 11159.00 &  6.41\\
tai200 20 6.json & 200 & 20 & Solution & 600.10 & 12205 & 11269.00 &  7.67\\
tai200 20 7.json & 200 & 20 & Solution & 600.10 & 12221 & 11216.00 &  8.22\\
tai200 20 8.json & 200 & 20 & Solution & 600.12 & 11991 & 11054.00 &  7.81\\
tai200 20 9.json & 200 & 20 & Solution & 600.11 & 12022 & 11242.00 &  6.49\\
tai20 10 0.json & 20 & 10 & Optimal & 292.19 & 1582 & 1582.00 &  0.00\\
tai20 10 1.json & 20 & 10 & Solution & 600.02 & 1659 & 1580.00 &  4.76\\
tai20 10 2.json & 20 & 10 & Optimal & 587.59 & 1496 & 1496.00 &  0.00\\
tai20 10 3.json & 20 & 10 & Optimal & 62.06 & 1377 & 1377.00 &  0.00\\
tai20 10 4.json & 20 & 10 & Optimal & 101.03 & 1419 & 1419.00 &  0.00\\
tai20 10 5.json & 20 & 10 & Optimal & 119.12 & 1397 & 1397.00 &  0.00\\
tai20 10 6.json & 20 & 10 & Solution & 600.02 & 1484 & 1399.00 &  5.73\\
tai20 10 7.json & 20 & 10 & Optimal & 357.94 & 1538 & 1538.00 &  0.00\\
tai20 10 8.json & 20 & 10 & Optimal & 31.26 & 1593 & 1593.00 &  0.00\\
tai20 10 9.json & 20 & 10 & Solution & 600.04 & 1603 & 1492.00 &  6.92\\
tai20 20 0.json & 20 & 20 & Solution & 600.04 & 2340 & 2010.00 & 14.10\\
tai20 20 1.json & 20 & 20 & Solution & 600.03 & 2130 & 1823.00 & 14.41\\
tai20 20 2.json & 20 & 20 & Solution & 600.04 & 2329 & 1945.00 & 16.49\\
tai20 20 3.json & 20 & 20 & Solution & 600.04 & 2229 & 1933.00 & 13.28\\
tai20 20 4.json & 20 & 20 & Solution & 600.02 & 2324 & 2034.00 & 12.48\\
tai20 20 5.json & 20 & 20 & Solution & 600.04 & 2235 & 1967.00 & 11.99\\
tai20 20 6.json & 20 & 20 & Solution & 600.05 & 2291 & 1976.00 & 13.75\\
tai20 20 7.json & 20 & 20 & Solution & 600.04 & 2222 & 1936.00 & 12.87\\
tai20 20 8.json & 20 & 20 & Solution & 600.04 & 2250 & 1909.00 & 15.16\\
tai20 20 9.json & 20 & 20 & Solution & 600.02 & 2189 & 1954.00 & 10.74\\
tai20 5 0.json & 20 & 5 & Optimal &  0.79 & 1278 & 1278.00 &  0.00\\
tai20 5 1.json & 20 & 5 & Optimal &  0.39 & 1359 & 1359.00 &  0.00\\
tai20 5 2.json & 20 & 5 & Optimal &  0.76 & 1081 & 1081.00 &  0.00\\
tai20 5 3.json & 20 & 5 & Optimal &  1.38 & 1293 & 1293.00 &  0.00\\
tai20 5 4.json & 20 & 5 & Optimal &  4.98 & 1235 & 1235.00 &  0.00\\
tai20 5 5.json & 20 & 5 & Optimal &  0.45 & 1195 & 1195.00 &  0.00\\
tai20 5 6.json & 20 & 5 & Optimal &  0.37 & 1234 & 1234.00 &  0.00\\
tai20 5 7.json & 20 & 5 & Optimal &  1.22 & 1206 & 1206.00 &  0.00\\
tai20 5 8.json & 20 & 5 & Optimal &  0.65 & 1230 & 1230.00 &  0.00\\
tai20 5 9.json & 20 & 5 & Optimal &  0.58 & 1108 & 1108.00 &  0.00\\
tai500 20 0.json & 500 & 20 & Solution & 600.40 & 28935 & 25955.00 & 10.30\\
tai500 20 1.json & 500 & 20 & Solution & 600.21 & 29270 & 26432.00 &  9.70\\
tai500 20 2.json & 500 & 20 & Solution & 600.25 & 28956 & 26330.00 &  9.07\\
tai500 20 3.json & 500 & 20 & Solution & 600.21 & 28977 & 26456.00 &  8.70\\
tai500 20 4.json & 500 & 20 & Solution & 600.23 & 28999 & 26263.00 &  9.43\\
tai500 20 5.json & 500 & 20 & Solution & 600.28 & 28939 & 26440.00 &  8.64\\
tai500 20 6.json & 500 & 20 & Solution & 600.27 & 28709 & 26362.00 &  8.18\\
tai500 20 7.json & 500 & 20 & Solution & 600.29 & 29115 & 26514.00 &  8.93\\
tai500 20 8.json & 500 & 20 & Solution & 600.22 & 28659 & 25952.00 &  9.45\\
tai500 20 9.json & 500 & 20 & Solution & 600.25 & 28948 & 26429.00 &  8.70\\
tai50 10 0.json & 50 & 10 & Solution & 600.09 & 3039 & 2967.00 &  2.37\\
tai50 10 1.json & 50 & 10 & Solution & 600.09 & 2933 & 2829.00 &  3.55\\
tai50 10 2.json & 50 & 10 & Solution & 600.11 & 2921 & 2828.00 &  3.18\\
tai50 10 3.json & 50 & 10 & Optimal & 535.73 & 3063 & 3063.00 &  0.00\\
tai50 10 4.json & 50 & 10 & Solution & 600.10 & 3021 & 2928.00 &  3.08\\
tai50 10 5.json & 50 & 10 & Solution & 600.12 & 3050 & 2987.00 &  2.07\\
tai50 10 6.json & 50 & 10 & Solution & 600.10 & 3124 & 3065.00 &  1.89\\
tai50 10 7.json & 50 & 10 & Solution & 600.05 & 3040 & 3037.00 &  0.10\\
tai50 10 8.json & 50 & 10 & Solution & 600.12 & 2902 & 2883.00 &  0.65\\
tai50 10 9.json & 50 & 10 & Solution & 600.06 & 3121 & 3046.00 &  2.40\\
tai50 20 0.json & 50 & 20 & Solution & 600.21 & 3931 & 3591.00 &  8.65\\
tai50 20 1.json & 50 & 20 & Solution & 600.24 & 3812 & 3534.00 &  7.29\\
tai50 20 2.json & 50 & 20 & Solution & 600.24 & 3756 & 3428.00 &  8.73\\
tai50 20 3.json & 50 & 20 & Solution & 600.24 & 3817 & 3453.00 &  9.54\\
tai50 20 4.json & 50 & 20 & Solution & 600.20 & 3736 & 3389.00 &  9.29\\
tai50 20 5.json & 50 & 20 & Solution & 600.17 & 3784 & 3535.00 &  6.58\\
tai50 20 6.json & 50 & 20 & Solution & 600.18 & 3799 & 3495.00 &  8.00\\
tai50 20 7.json & 50 & 20 & Solution & 600.18 & 3836 & 3443.00 & 10.25\\
tai50 20 8.json & 50 & 20 & Solution & 600.22 & 3908 & 3482.00 & 10.90\\
tai50 20 9.json & 50 & 20 & Solution & 600.16 & 3857 & 3538.00 &  8.27\\
tai50 5 0.json & 50 & 5 & Optimal &  1.24 & 2724 & 2724.00 &  0.00\\
tai50 5 1.json & 50 & 5 & Optimal &  2.71 & 2834 & 2834.00 &  0.00\\
tai50 5 2.json & 50 & 5 & Optimal & 32.80 & 2621 & 2621.00 &  0.00\\
tai50 5 3.json & 50 & 5 & Optimal &  1.66 & 2751 & 2751.00 &  0.00\\
tai50 5 4.json & 50 & 5 & Optimal &  2.22 & 2863 & 2863.00 &  0.00\\
tai50 5 5.json & 50 & 5 & Optimal &  3.09 & 2829 & 2829.00 &  0.00\\
tai50 5 6.json & 50 & 5 & Optimal & 14.28 & 2725 & 2725.00 &  0.00\\
tai50 5 7.json & 50 & 5 & Optimal &  2.61 & 2683 & 2683.00 &  0.00\\
tai50 5 8.json & 50 & 5 & Optimal &  3.82 & 2552 & 2552.00 &  0.00\\
tai50 5 9.json & 50 & 5 & Optimal &  2.03 & 2782 & 2782.00 &  0.00\\
\end{longtable}



\clearpage
\chapter{SALBP-1 Assembly Line Balancing Problems}

The assembly line balancing problems have a single cumulative and no disjunctive constraints, so the indicated number of (disjunctive) machines is zero. 

The larger problem instances are still missing. For the small instances (20 tasks), only a few are not solved to optimality, for the medium sizes the number of optimal solutions found is reduced, and for larger instances, optimal solutions are rare.

\section{Results for CPOptimizer}

\begin{longtable}{lrrlrrrr}
\caption{Results for SALBP-1 Problems (CPO) (2100 Instances)}\\\toprule
Name & \shortstack{Nr\\Jobs} & \shortstack{Nr\\Machines} & Status & Time & Makespan & Bound & \shortstack{Gap\\Percent}\\ \midrule
\endhead
\bottomrule
\endfoot
instance n=1000 1.alb & 1 & 0 & Solution & 120.19 & 136 & 135.00 &  0.74\\
instance n=1000 10.alb & 1 & 0 & Solution & 120.07 & 141 & 140.00 &  0.71\\
instance n=1000 100.alb & 1 & 0 & Solution & 120.10 & 139 & 137.00 &  1.44\\
instance n=1000 101.alb & 1 & 0 & Solution & 120.18 & 558 & 505.00 &  9.50\\
instance n=1000 102.alb & 1 & 0 & Solution & 120.20 & 556 & 503.00 &  9.53\\
instance n=1000 103.alb & 1 & 0 & Solution & 120.25 & 562 & 503.00 & 10.50\\
instance n=1000 104.alb & 1 & 0 & Solution & 120.20 & 553 & 504.00 &  8.86\\
instance n=1000 105.alb & 1 & 0 & Solution & 120.19 & 548 & 499.00 &  8.94\\
instance n=1000 106.alb & 1 & 0 & Solution & 120.22 & 556 & 499.00 & 10.25\\
instance n=1000 107.alb & 1 & 0 & Solution & 120.20 & 540 & 496.00 &  8.15\\
instance n=1000 108.alb & 1 & 0 & Solution & 120.18 & 545 & 498.00 &  8.62\\
instance n=1000 109.alb & 1 & 0 & Solution & 120.21 & 549 & 500.00 &  8.93\\
instance n=1000 11.alb & 1 & 0 & Solution & 120.07 & 135 & 134.00 &  0.74\\
instance n=1000 110.alb & 1 & 0 & Solution & 120.21 & 555 & 501.00 &  9.73\\
instance n=1000 111.alb & 1 & 0 & Solution & 120.25 & 546 & 500.00 &  8.42\\
instance n=1000 112.alb & 1 & 0 & Solution & 120.22 & 548 & 499.00 &  8.94\\
instance n=1000 113.alb & 1 & 0 & Solution & 120.17 & 540 & 495.00 &  8.33\\
instance n=1000 114.alb & 1 & 0 & Solution & 120.20 & 550 & 502.00 &  8.73\\
instance n=1000 115.alb & 1 & 0 & Solution & 120.19 & 539 & 498.00 &  7.61\\
instance n=1000 116.alb & 1 & 0 & Solution & 120.19 & 545 & 496.00 &  8.99\\
instance n=1000 117.alb & 1 & 0 & Solution & 120.20 & 552 & 500.00 &  9.42\\
instance n=1000 118.alb & 1 & 0 & Solution & 120.30 & 563 & 509.00 &  9.59\\
instance n=1000 119.alb & 1 & 0 & Solution & 120.25 & 529 & 496.00 &  6.24\\
instance n=1000 12.alb & 1 & 0 & Solution & 120.05 & 135 & 134.00 &  0.74\\
instance n=1000 120.alb & 1 & 0 & Solution & 120.19 & 549 & 502.00 &  8.56\\
instance n=1000 121.alb & 1 & 0 & Solution & 120.24 & 541 & 496.00 &  8.32\\
instance n=1000 122.alb & 1 & 0 & Solution & 120.22 & 535 & 493.00 &  7.85\\
instance n=1000 123.alb & 1 & 0 & Solution & 120.19 & 555 & 504.00 &  9.19\\
instance n=1000 124.alb & 1 & 0 & Solution & 120.25 & 543 & 498.00 &  8.29\\
instance n=1000 125.alb & 1 & 0 & Solution & 120.21 & 545 & 499.00 &  8.44\\
instance n=1000 126.alb & 1 & 0 & Solution & 120.12 & 232 & 228.00 &  1.72\\
instance n=1000 127.alb & 1 & 0 & Solution & 120.12 & 224 & 221.00 &  1.34\\
instance n=1000 128.alb & 1 & 0 & Solution & 120.15 & 225 & 222.00 &  1.33\\
instance n=1000 129.alb & 1 & 0 & Solution & 120.10 & 226 & 223.00 &  1.33\\
instance n=1000 13.alb & 1 & 0 & Solution & 120.09 & 132 & 131.00 &  0.76\\
instance n=1000 130.alb & 1 & 0 & Solution & 120.18 & 225 & 221.00 &  1.78\\
instance n=1000 131.alb & 1 & 0 & Solution & 120.09 & 223 & 220.00 &  1.35\\
instance n=1000 132.alb & 1 & 0 & Solution & 120.10 & 218 & 214.00 &  1.83\\
instance n=1000 133.alb & 1 & 0 & Solution & 120.14 & 229 & 226.00 &  1.31\\
instance n=1000 134.alb & 1 & 0 & Solution & 120.14 & 219 & 215.00 &  1.83\\
instance n=1000 135.alb & 1 & 0 & Solution & 120.12 & 229 & 225.00 &  1.75\\
instance n=1000 136.alb & 1 & 0 & Solution & 120.25 & 232 & 228.00 &  1.72\\
instance n=1000 137.alb & 1 & 0 & Solution & 120.10 & 216 & 213.00 &  1.39\\
instance n=1000 138.alb & 1 & 0 & Solution & 120.11 & 225 & 221.00 &  1.78\\
instance n=1000 139.alb & 1 & 0 & Solution & 120.13 & 227 & 224.00 &  1.32\\
instance n=1000 14.alb & 1 & 0 & Solution & 120.05 & 138 & 136.00 &  1.45\\
instance n=1000 140.alb & 1 & 0 & Solution & 120.13 & 230 & 226.00 &  1.74\\
instance n=1000 141.alb & 1 & 0 & Solution & 120.16 & 218 & 215.00 &  1.38\\
instance n=1000 142.alb & 1 & 0 & Solution & 120.11 & 223 & 220.00 &  1.35\\
instance n=1000 143.alb & 1 & 0 & Solution & 120.11 & 216 & 213.00 &  1.39\\
instance n=1000 144.alb & 1 & 0 & Solution & 120.09 & 221 & 217.00 &  1.81\\
instance n=1000 145.alb & 1 & 0 & Solution & 120.15 & 223 & 220.00 &  1.35\\
instance n=1000 146.alb & 1 & 0 & Solution & 120.13 & 223 & 219.00 &  1.79\\
instance n=1000 147.alb & 1 & 0 & Solution & 120.13 & 234 & 229.00 &  2.14\\
instance n=1000 148.alb & 1 & 0 & Solution & 120.14 & 223 & 219.00 &  1.79\\
instance n=1000 149.alb & 1 & 0 & Solution & 120.11 & 241 & 237.00 &  1.66\\
instance n=1000 15.alb & 1 & 0 & Solution & 120.08 & 137 & 136.00 &  0.73\\
instance n=1000 150.alb & 1 & 0 & Solution & 120.10 & 225 & 222.00 &  1.33\\
instance n=1000 151.alb & 1 & 0 & Solution & 120.16 & 140 & 138.00 &  1.43\\
instance n=1000 152.alb & 1 & 0 & Solution & 120.13 & 138 & 136.00 &  1.45\\
instance n=1000 153.alb & 1 & 0 & Solution & 120.06 & 139 & 137.00 &  1.44\\
instance n=1000 154.alb & 1 & 0 & Solution & 120.22 & 142 & 140.00 &  1.41\\
instance n=1000 155.alb & 1 & 0 & Solution & 120.11 & 141 & 139.00 &  1.42\\
instance n=1000 156.alb & 1 & 0 & Solution & 120.22 & 143 & 141.00 &  1.40\\
instance n=1000 157.alb & 1 & 0 & Solution & 120.15 & 141 & 140.00 &  0.71\\
instance n=1000 158.alb & 1 & 0 & Solution & 120.12 & 137 & 136.00 &  0.73\\
instance n=1000 159.alb & 1 & 0 & Solution & 120.09 & 140 & 138.00 &  1.43\\
instance n=1000 16.alb & 1 & 0 & Solution & 120.06 & 138 & 137.00 &  0.72\\
instance n=1000 160.alb & 1 & 0 & Solution & 120.12 & 140 & 138.00 &  1.43\\
instance n=1000 161.alb & 1 & 0 & Solution & 120.10 & 134 & 133.00 &  0.75\\
instance n=1000 162.alb & 1 & 0 & Solution & 120.06 & 137 & 136.00 &  0.73\\
instance n=1000 163.alb & 1 & 0 & Solution & 120.16 & 141 & 139.00 &  1.42\\
instance n=1000 164.alb & 1 & 0 & Solution & 120.08 & 143 & 141.00 &  1.40\\
instance n=1000 165.alb & 1 & 0 & Solution & 120.15 & 137 & 135.00 &  1.46\\
instance n=1000 166.alb & 1 & 0 & Solution & 120.08 & 141 & 139.00 &  1.42\\
instance n=1000 167.alb & 1 & 0 & Solution & 120.06 & 141 & 139.00 &  1.42\\
instance n=1000 168.alb & 1 & 0 & Solution & 120.05 & 140 & 138.00 &  1.43\\
instance n=1000 169.alb & 1 & 0 & Solution & 120.13 & 136 & 134.00 &  1.47\\
instance n=1000 17.alb & 1 & 0 & Solution & 120.06 & 136 & 135.00 &  0.74\\
instance n=1000 170.alb & 1 & 0 & Solution & 120.11 & 136 & 134.00 &  1.47\\
instance n=1000 171.alb & 1 & 0 & Solution & 120.06 & 139 & 137.00 &  1.44\\
instance n=1000 172.alb & 1 & 0 & Solution & 120.13 & 136 & 135.00 &  0.74\\
instance n=1000 173.alb & 1 & 0 & Solution & 120.15 & 137 & 135.00 &  1.46\\
instance n=1000 174.alb & 1 & 0 & Solution & 120.17 & 138 & 136.00 &  1.45\\
instance n=1000 175.alb & 1 & 0 & Solution & 120.13 & 140 & 138.00 &  1.43\\
instance n=1000 176.alb & 1 & 0 & Solution & 120.22 & 557 & 500.00 & 10.23\\
instance n=1000 177.alb & 1 & 0 & Solution & 120.19 & 552 & 499.00 &  9.60\\
instance n=1000 178.alb & 1 & 0 & Solution & 120.22 & 566 & 506.00 & 10.60\\
instance n=1000 179.alb & 1 & 0 & Solution & 120.24 & 564 & 505.00 & 10.46\\
instance n=1000 18.alb & 1 & 0 & Solution & 120.07 & 135 & 134.00 &  0.74\\
instance n=1000 180.alb & 1 & 0 & Solution & 120.20 & 559 & 503.00 & 10.02\\
instance n=1000 181.alb & 1 & 0 & Solution & 120.24 & 561 & 505.00 &  9.98\\
instance n=1000 182.alb & 1 & 0 & Solution & 120.24 & 557 & 502.00 &  9.87\\
instance n=1000 183.alb & 1 & 0 & Solution & 120.21 & 552 & 500.00 &  9.42\\
instance n=1000 184.alb & 1 & 0 & Solution & 120.22 & 559 & 502.00 & 10.20\\
instance n=1000 185.alb & 1 & 0 & Solution & 120.24 & 560 & 503.00 & 10.18\\
instance n=1000 186.alb & 1 & 0 & Solution & 120.22 & 552 & 500.00 &  9.42\\
instance n=1000 187.alb & 1 & 0 & Solution & 120.21 & 565 & 505.00 & 10.62\\
instance n=1000 188.alb & 1 & 0 & Solution & 120.24 & 552 & 498.00 &  9.78\\
instance n=1000 189.alb & 1 & 0 & Solution & 120.22 & 552 & 498.00 &  9.78\\
instance n=1000 19.alb & 1 & 0 & Solution & 120.06 & 138 & 137.00 &  0.72\\
instance n=1000 190.alb & 1 & 0 & Solution & 120.23 & 556 & 501.00 &  9.89\\
instance n=1000 191.alb & 1 & 0 & Solution & 120.21 & 553 & 501.00 &  9.40\\
instance n=1000 192.alb & 1 & 0 & Solution & 120.19 & 556 & 501.00 &  9.89\\
instance n=1000 193.alb & 1 & 0 & Solution & 120.20 & 559 & 503.00 & 10.02\\
instance n=1000 194.alb & 1 & 0 & Solution & 120.23 & 560 & 502.00 & 10.36\\
instance n=1000 195.alb & 1 & 0 & Solution & 120.20 & 562 & 502.00 & 10.68\\
instance n=1000 196.alb & 1 & 0 & Solution & 120.21 & 559 & 500.00 & 10.55\\
instance n=1000 197.alb & 1 & 0 & Solution & 120.19 & 546 & 496.00 &  9.16\\
instance n=1000 198.alb & 1 & 0 & Solution & 120.19 & 562 & 503.00 & 10.50\\
instance n=1000 199.alb & 1 & 0 & Solution & 120.24 & 541 & 495.00 &  8.50\\
instance n=1000 2.alb & 1 & 0 & Solution & 120.07 & 138 & 137.00 &  0.72\\
instance n=1000 20.alb & 1 & 0 & Solution & 120.05 & 139 & 138.00 &  0.72\\
instance n=1000 200.alb & 1 & 0 & Solution & 120.23 & 550 & 498.00 &  9.45\\
instance n=1000 201.alb & 1 & 0 & Solution & 120.20 & 233 & 229.00 &  1.72\\
instance n=1000 202.alb & 1 & 0 & Solution & 120.13 & 230 & 225.00 &  2.17\\
instance n=1000 203.alb & 1 & 0 & Solution & 120.10 & 234 & 229.00 &  2.14\\
instance n=1000 204.alb & 1 & 0 & Solution & 120.15 & 233 & 228.00 &  2.15\\
instance n=1000 205.alb & 1 & 0 & Solution & 120.25 & 234 & 229.00 &  2.14\\
instance n=1000 206.alb & 1 & 0 & Solution & 120.11 & 233 & 229.00 &  1.72\\
instance n=1000 207.alb & 1 & 0 & Solution & 120.13 & 234 & 230.00 &  1.71\\
instance n=1000 208.alb & 1 & 0 & Solution & 120.24 & 234 & 229.00 &  2.14\\
instance n=1000 209.alb & 1 & 0 & Solution & 120.13 & 233 & 228.00 &  2.15\\
instance n=1000 21.alb & 1 & 0 & Solution & 120.07 & 139 & 138.00 &  0.72\\
instance n=1000 210.alb & 1 & 0 & Solution & 120.13 & 229 & 224.00 &  2.18\\
instance n=1000 211.alb & 1 & 0 & Solution & 120.14 & 223 & 219.00 &  1.79\\
instance n=1000 212.alb & 1 & 0 & Solution & 120.12 & 221 & 217.00 &  1.81\\
instance n=1000 213.alb & 1 & 0 & Solution & 120.17 & 238 & 233.00 &  2.10\\
instance n=1000 214.alb & 1 & 0 & Solution & 120.16 & 230 & 225.00 &  2.17\\
instance n=1000 215.alb & 1 & 0 & Solution & 120.14 & 227 & 223.00 &  1.76\\
instance n=1000 216.alb & 1 & 0 & Solution & 120.10 & 225 & 221.00 &  1.78\\
instance n=1000 217.alb & 1 & 0 & Solution & 120.31 & 229 & 225.00 &  1.75\\
instance n=1000 218.alb & 1 & 0 & Solution & 120.14 & 223 & 219.00 &  1.79\\
instance n=1000 219.alb & 1 & 0 & Solution & 120.09 & 236 & 232.00 &  1.69\\
instance n=1000 22.alb & 1 & 0 & Solution & 120.05 & 139 & 137.00 &  1.44\\
instance n=1000 220.alb & 1 & 0 & Solution & 120.11 & 229 & 225.00 &  1.75\\
instance n=1000 221.alb & 1 & 0 & Solution & 120.11 & 236 & 231.00 &  2.12\\
instance n=1000 222.alb & 1 & 0 & Solution & 120.11 & 226 & 221.00 &  2.21\\
instance n=1000 223.alb & 1 & 0 & Solution & 120.17 & 226 & 221.00 &  2.21\\
instance n=1000 224.alb & 1 & 0 & Solution & 120.11 & 231 & 226.00 &  2.16\\
instance n=1000 225.alb & 1 & 0 & Solution & 120.24 & 234 & 229.00 &  2.14\\
instance n=1000 226.alb & 1 & 0 & Solution & 120.15 & 138 & 136.00 &  1.45\\
instance n=1000 227.alb & 1 & 0 & Solution & 120.11 & 140 & 138.00 &  1.43\\
instance n=1000 228.alb & 1 & 0 & Solution & 120.16 & 135 & 133.00 &  1.48\\
instance n=1000 229.alb & 1 & 0 & Solution & 120.17 & 136 & 134.00 &  1.47\\
instance n=1000 23.alb & 1 & 0 & Solution & 120.06 & 137 & 136.00 &  0.73\\
instance n=1000 230.alb & 1 & 0 & Solution & 120.15 & 134 & 131.00 &  2.24\\
instance n=1000 231.alb & 1 & 0 & Solution & 120.17 & 141 & 138.00 &  2.13\\
instance n=1000 232.alb & 1 & 0 & Solution & 120.11 & 135 & 133.00 &  1.48\\
instance n=1000 233.alb & 1 & 0 & Solution & 120.30 & 138 & 135.00 &  2.17\\
instance n=1000 234.alb & 1 & 0 & Solution & 120.07 & 139 & 137.00 &  1.44\\
instance n=1000 235.alb & 1 & 0 & Solution & 120.29 & 134 & 133.00 &  0.75\\
instance n=1000 236.alb & 1 & 0 & Solution & 120.16 & 138 & 136.00 &  1.45\\
instance n=1000 237.alb & 1 & 0 & Solution & 120.18 & 141 & 138.00 &  2.13\\
instance n=1000 238.alb & 1 & 0 & Solution & 120.13 & 140 & 138.00 &  1.43\\
instance n=1000 239.alb & 1 & 0 & Solution & 120.11 & 137 & 135.00 &  1.46\\
instance n=1000 24.alb & 1 & 0 & Solution & 120.06 & 141 & 140.00 &  0.71\\
instance n=1000 240.alb & 1 & 0 & Solution & 120.19 & 137 & 135.00 &  1.46\\
instance n=1000 241.alb & 1 & 0 & Solution & 120.15 & 140 & 138.00 &  1.43\\
instance n=1000 242.alb & 1 & 0 & Solution & 120.11 & 137 & 135.00 &  1.46\\
instance n=1000 243.alb & 1 & 0 & Solution & 120.09 & 139 & 137.00 &  1.44\\
instance n=1000 244.alb & 1 & 0 & Solution & 120.20 & 139 & 137.00 &  1.44\\
instance n=1000 245.alb & 1 & 0 & Solution & 120.10 & 137 & 135.00 &  1.46\\
instance n=1000 246.alb & 1 & 0 & Solution & 120.12 & 137 & 135.00 &  1.46\\
instance n=1000 247.alb & 1 & 0 & Solution & 120.22 & 141 & 138.00 &  2.13\\
instance n=1000 248.alb & 1 & 0 & Solution & 120.07 & 141 & 138.00 &  2.13\\
instance n=1000 249.alb & 1 & 0 & Solution & 120.20 & 141 & 138.00 &  2.13\\
instance n=1000 25.alb & 1 & 0 & Solution & 120.06 & 137 & 136.00 &  0.73\\
instance n=1000 250.alb & 1 & 0 & Solution & 120.35 & 142 & 140.00 &  1.41\\
instance n=1000 251.alb & 1 & 0 & Solution & 120.30 & 568 & 502.00 & 11.62\\
instance n=1000 252.alb & 1 & 0 & Solution & 120.24 & 567 & 501.00 & 11.64\\
instance n=1000 253.alb & 1 & 0 & Solution & 120.25 & 560 & 502.00 & 10.36\\
instance n=1000 254.alb & 1 & 0 & Solution & 120.20 & 563 & 501.00 & 11.01\\
instance n=1000 255.alb & 1 & 0 & Solution & 120.33 & 551 & 498.00 &  9.62\\
instance n=1000 256.alb & 1 & 0 & Solution & 120.24 & 558 & 495.00 & 11.29\\
instance n=1000 257.alb & 1 & 0 & Solution & 120.24 & 566 & 502.00 & 11.31\\
instance n=1000 258.alb & 1 & 0 & Solution & 120.35 & 557 & 497.00 & 10.77\\
instance n=1000 259.alb & 1 & 0 & Solution & 120.33 & 557 & 496.00 & 10.95\\
instance n=1000 26.alb & 1 & 0 & Solution & 120.19 & 555 & 502.00 &  9.55\\
instance n=1000 260.alb & 1 & 0 & Solution & 120.20 & 556 & 495.00 & 10.97\\
instance n=1000 261.alb & 1 & 0 & Solution & 120.26 & 564 & 501.00 & 11.17\\
instance n=1000 262.alb & 1 & 0 & Solution & 120.23 & 544 & 495.00 &  9.01\\
instance n=1000 263.alb & 1 & 0 & Solution & 120.23 & 561 & 499.00 & 11.05\\
instance n=1000 264.alb & 1 & 0 & Solution & 120.31 & 557 & 499.00 & 10.41\\
instance n=1000 265.alb & 1 & 0 & Solution & 120.24 & 579 & 506.00 & 12.61\\
instance n=1000 266.alb & 1 & 0 & Solution & 120.23 & 562 & 500.00 & 11.03\\
instance n=1000 267.alb & 1 & 0 & Solution & 120.25 & 571 & 506.00 & 11.38\\
instance n=1000 268.alb & 1 & 0 & Solution & 120.24 & 554 & 497.00 & 10.29\\
instance n=1000 269.alb & 1 & 0 & Solution & 120.40 & 558 & 500.00 & 10.39\\
instance n=1000 27.alb & 1 & 0 & Solution & 120.19 & 551 & 502.00 &  8.89\\
instance n=1000 270.alb & 1 & 0 & Solution & 120.21 & 581 & 508.00 & 12.56\\
instance n=1000 271.alb & 1 & 0 & Solution & 120.38 & 553 & 497.00 & 10.13\\
instance n=1000 272.alb & 1 & 0 & Solution & 120.24 & 567 & 502.00 & 11.46\\
instance n=1000 273.alb & 1 & 0 & Solution & 120.19 & 563 & 500.00 & 11.19\\
instance n=1000 274.alb & 1 & 0 & Solution & 120.22 & 559 & 496.00 & 11.27\\
instance n=1000 275.alb & 1 & 0 & Solution & 120.21 & 565 & 504.00 & 10.80\\
instance n=1000 276.alb & 1 & 0 & Solution & 120.09 & 223 & 217.00 &  2.69\\
instance n=1000 277.alb & 1 & 0 & Solution & 120.38 & 230 & 225.00 &  2.17\\
instance n=1000 278.alb & 1 & 0 & Solution & 120.22 & 226 & 220.00 &  2.65\\
instance n=1000 279.alb & 1 & 0 & Solution & 120.21 & 220 & 215.00 &  2.27\\
instance n=1000 28.alb & 1 & 0 & Solution & 120.18 & 538 & 497.00 &  7.62\\
instance n=1000 280.alb & 1 & 0 & Solution & 120.10 & 231 & 226.00 &  2.16\\
instance n=1000 281.alb & 1 & 0 & Solution & 120.08 & 225 & 219.00 &  2.67\\
instance n=1000 282.alb & 1 & 0 & Solution & 120.10 & 220 & 214.00 &  2.73\\
instance n=1000 283.alb & 1 & 0 & Solution & 120.25 & 230 & 224.00 &  2.61\\
instance n=1000 284.alb & 1 & 0 & Solution & 120.11 & 222 & 217.00 &  2.25\\
instance n=1000 285.alb & 1 & 0 & Solution & 120.14 & 227 & 221.00 &  2.64\\
instance n=1000 286.alb & 1 & 0 & Solution & 120.14 & 227 & 221.00 &  2.64\\
instance n=1000 287.alb & 1 & 0 & Solution & 120.10 & 230 & 224.00 &  2.61\\
instance n=1000 288.alb & 1 & 0 & Solution & 120.16 & 225 & 219.00 &  2.67\\
instance n=1000 289.alb & 1 & 0 & Solution & 120.13 & 225 & 220.00 &  2.22\\
instance n=1000 29.alb & 1 & 0 & Solution & 120.19 & 542 & 498.00 &  8.12\\
instance n=1000 290.alb & 1 & 0 & Solution & 120.24 & 228 & 222.00 &  2.63\\
instance n=1000 291.alb & 1 & 0 & Solution & 120.16 & 231 & 225.00 &  2.60\\
instance n=1000 292.alb & 1 & 0 & Solution & 120.11 & 232 & 226.00 &  2.59\\
instance n=1000 293.alb & 1 & 0 & Solution & 120.14 & 231 & 225.00 &  2.60\\
instance n=1000 294.alb & 1 & 0 & Solution & 120.19 & 236 & 230.00 &  2.54\\
instance n=1000 295.alb & 1 & 0 & Solution & 120.16 & 233 & 227.00 &  2.58\\
instance n=1000 296.alb & 1 & 0 & Solution & 120.21 & 213 & 208.00 &  2.35\\
instance n=1000 297.alb & 1 & 0 & Solution & 120.13 & 222 & 217.00 &  2.25\\
instance n=1000 298.alb & 1 & 0 & Solution & 120.14 & 219 & 214.00 &  2.28\\
instance n=1000 299.alb & 1 & 0 & Solution & 120.25 & 232 & 226.00 &  2.59\\
instance n=1000 3.alb & 1 & 0 & Solution & 120.10 & 138 & 136.00 &  1.45\\
instance n=1000 30.alb & 1 & 0 & Solution & 120.22 & 559 & 506.00 &  9.48\\
instance n=1000 300.alb & 1 & 0 & Solution & 120.21 & 234 & 228.00 &  2.56\\
instance n=1000 301.alb & 1 & 0 & Solution & 120.10 & 138 & 137.00 &  0.72\\
instance n=1000 302.alb & 1 & 0 & Solution & 120.15 & 140 & 139.00 &  0.71\\
instance n=1000 303.alb & 1 & 0 & Solution & 120.14 & 140 & 138.00 &  1.43\\
instance n=1000 304.alb & 1 & 0 & Solution & 120.11 & 138 & 136.00 &  1.45\\
instance n=1000 305.alb & 1 & 0 & Solution & 120.20 & 141 & 140.00 &  0.71\\
instance n=1000 306.alb & 1 & 0 & Solution & 120.22 & 136 & 135.00 &  0.74\\
instance n=1000 307.alb & 1 & 0 & Solution & 120.23 & 137 & 136.00 &  0.73\\
instance n=1000 308.alb & 1 & 0 & Solution & 120.13 & 138 & 137.00 &  0.72\\
instance n=1000 309.alb & 1 & 0 & Solution & 120.24 & 136 & 135.00 &  0.74\\
instance n=1000 31.alb & 1 & 0 & Solution & 120.19 & 555 & 506.00 &  8.83\\
instance n=1000 310.alb & 1 & 0 & Solution & 120.16 & 143 & 141.00 &  1.40\\
instance n=1000 311.alb & 1 & 0 & Solution & 120.24 & 141 & 139.00 &  1.42\\
instance n=1000 312.alb & 1 & 0 & Solution & 120.13 & 136 & 135.00 &  0.74\\
instance n=1000 313.alb & 1 & 0 & Solution & 120.15 & 139 & 138.00 &  0.72\\
instance n=1000 314.alb & 1 & 0 & Solution & 120.20 & 143 & 142.00 &  0.70\\
instance n=1000 315.alb & 1 & 0 & Solution & 120.36 & 138 & 136.00 &  1.45\\
instance n=1000 316.alb & 1 & 0 & Solution & 120.23 & 139 & 137.00 &  1.44\\
instance n=1000 317.alb & 1 & 0 & Solution & 120.23 & 137 & 136.00 &  0.73\\
instance n=1000 318.alb & 1 & 0 & Solution & 120.09 & 139 & 138.00 &  0.72\\
instance n=1000 319.alb & 1 & 0 & Solution & 120.09 & 142 & 140.00 &  1.41\\
instance n=1000 32.alb & 1 & 0 & Solution & 120.20 & 542 & 502.00 &  7.38\\
instance n=1000 320.alb & 1 & 0 & Solution & 120.09 & 142 & 141.00 &  0.70\\
instance n=1000 321.alb & 1 & 0 & Solution & 120.19 & 141 & 140.00 &  0.71\\
instance n=1000 322.alb & 1 & 0 & Solution & 120.20 & 140 & 139.00 &  0.71\\
instance n=1000 323.alb & 1 & 0 & Solution & 120.10 & 140 & 138.00 &  1.43\\
instance n=1000 324.alb & 1 & 0 & Solution & 120.11 & 141 & 140.00 &  0.71\\
instance n=1000 325.alb & 1 & 0 & Solution & 120.11 & 140 & 138.00 &  1.43\\
instance n=1000 326.alb & 1 & 0 & Solution & 120.35 & 541 & 496.00 &  8.32\\
instance n=1000 327.alb & 1 & 0 & Solution & 120.22 & 552 & 503.00 &  8.88\\
instance n=1000 328.alb & 1 & 0 & Solution & 120.18 & 545 & 500.00 &  8.26\\
instance n=1000 329.alb & 1 & 0 & Solution & 120.27 & 554 & 502.00 &  9.39\\
instance n=1000 33.alb & 1 & 0 & Solution & 120.18 & 548 & 501.00 &  8.58\\
instance n=1000 330.alb & 1 & 0 & Solution & 120.23 & 538 & 498.00 &  7.43\\
instance n=1000 331.alb & 1 & 0 & Solution & 120.22 & 547 & 498.00 &  8.96\\
instance n=1000 332.alb & 1 & 0 & Solution & 120.25 & 535 & 495.00 &  7.48\\
instance n=1000 333.alb & 1 & 0 & Solution & 120.22 & 553 & 499.00 &  9.76\\
instance n=1000 334.alb & 1 & 0 & Solution & 120.27 & 540 & 498.00 &  7.78\\
instance n=1000 335.alb & 1 & 0 & Solution & 120.21 & 544 & 496.00 &  8.82\\
instance n=1000 336.alb & 1 & 0 & Solution & 120.47 & 534 & 497.00 &  6.93\\
instance n=1000 337.alb & 1 & 0 & Solution & 120.22 & 551 & 501.00 &  9.07\\
instance n=1000 338.alb & 1 & 0 & Solution & 120.23 & 553 & 502.00 &  9.22\\
instance n=1000 339.alb & 1 & 0 & Solution & 120.35 & 555 & 500.00 &  9.91\\
instance n=1000 34.alb & 1 & 0 & Solution & 120.19 & 563 & 507.00 &  9.95\\
instance n=1000 340.alb & 1 & 0 & Solution & 120.46 & 563 & 505.00 & 10.30\\
instance n=1000 341.alb & 1 & 0 & Solution & 120.25 & 552 & 503.00 &  8.88\\
instance n=1000 342.alb & 1 & 0 & Solution & 120.20 & 549 & 500.00 &  8.93\\
instance n=1000 343.alb & 1 & 0 & Solution & 120.29 & 554 & 500.00 &  9.75\\
instance n=1000 344.alb & 1 & 0 & Solution & 120.26 & 545 & 500.00 &  8.26\\
instance n=1000 345.alb & 1 & 0 & Solution & 120.27 & 552 & 502.00 &  9.06\\
instance n=1000 346.alb & 1 & 0 & Solution & 120.25 & 551 & 501.00 &  9.07\\
instance n=1000 347.alb & 1 & 0 & Solution & 120.37 & 547 & 498.00 &  8.96\\
instance n=1000 348.alb & 1 & 0 & Solution & 120.33 & 566 & 506.00 & 10.60\\
instance n=1000 349.alb & 1 & 0 & Solution & 120.42 & 558 & 503.00 &  9.86\\
instance n=1000 35.alb & 1 & 0 & Solution & 120.19 & 544 & 501.00 &  7.90\\
instance n=1000 350.alb & 1 & 0 & Solution & 120.18 & 534 & 496.00 &  7.12\\
instance n=1000 351.alb & 1 & 0 & Solution & 120.19 & 231 & 227.00 &  1.73\\
instance n=1000 352.alb & 1 & 0 & Solution & 120.13 & 231 & 227.00 &  1.73\\
instance n=1000 353.alb & 1 & 0 & Solution & 120.12 & 221 & 217.00 &  1.81\\
instance n=1000 354.alb & 1 & 0 & Solution & 120.24 & 226 & 222.00 &  1.77\\
instance n=1000 355.alb & 1 & 0 & Solution & 120.29 & 224 & 220.00 &  1.79\\
instance n=1000 356.alb & 1 & 0 & Solution & 120.13 & 230 & 226.00 &  1.74\\
instance n=1000 357.alb & 1 & 0 & Solution & 120.26 & 217 & 213.00 &  1.84\\
instance n=1000 358.alb & 1 & 0 & Solution & 120.12 & 223 & 219.00 &  1.79\\
instance n=1000 359.alb & 1 & 0 & Solution & 120.11 & 226 & 222.00 &  1.77\\
instance n=1000 36.alb & 1 & 0 & Solution & 120.19 & 538 & 497.00 &  7.62\\
instance n=1000 360.alb & 1 & 0 & Solution & 120.35 & 233 & 229.00 &  1.72\\
instance n=1000 361.alb & 1 & 0 & Solution & 120.11 & 219 & 215.00 &  1.83\\
instance n=1000 362.alb & 1 & 0 & Solution & 120.13 & 226 & 223.00 &  1.33\\
instance n=1000 363.alb & 1 & 0 & Solution & 120.10 & 218 & 215.00 &  1.38\\
instance n=1000 364.alb & 1 & 0 & Solution & 120.29 & 225 & 221.00 &  1.78\\
instance n=1000 365.alb & 1 & 0 & Solution & 120.30 & 231 & 227.00 &  1.73\\
instance n=1000 366.alb & 1 & 0 & Solution & 120.48 & 232 & 228.00 &  1.72\\
instance n=1000 367.alb & 1 & 0 & Solution & 120.11 & 231 & 227.00 &  1.73\\
instance n=1000 368.alb & 1 & 0 & Solution & 120.21 & 230 & 226.00 &  1.74\\
instance n=1000 369.alb & 1 & 0 & Solution & 120.29 & 224 & 220.00 &  1.79\\
instance n=1000 37.alb & 1 & 0 & Solution & 120.18 & 559 & 506.00 &  9.48\\
instance n=1000 370.alb & 1 & 0 & Solution & 120.20 & 227 & 223.00 &  1.76\\
instance n=1000 371.alb & 1 & 0 & Solution & 120.31 & 223 & 220.00 &  1.35\\
instance n=1000 372.alb & 1 & 0 & Solution & 120.35 & 234 & 230.00 &  1.71\\
instance n=1000 373.alb & 1 & 0 & Solution & 120.24 & 223 & 219.00 &  1.79\\
instance n=1000 374.alb & 1 & 0 & Solution & 120.19 & 222 & 219.00 &  1.35\\
instance n=1000 375.alb & 1 & 0 & Solution & 120.30 & 231 & 227.00 &  1.73\\
instance n=1000 376.alb & 1 & 0 & Solution & 120.19 & 134 & 132.00 &  1.49\\
instance n=1000 377.alb & 1 & 0 & Solution & 120.22 & 138 & 137.00 &  0.72\\
instance n=1000 378.alb & 1 & 0 & Solution & 120.24 & 136 & 134.00 &  1.47\\
instance n=1000 379.alb & 1 & 0 & Solution & 120.09 & 139 & 137.00 &  1.44\\
instance n=1000 38.alb & 1 & 0 & Solution & 120.20 & 557 & 504.00 &  9.52\\
instance n=1000 380.alb & 1 & 0 & Solution & 120.17 & 136 & 134.00 &  1.47\\
instance n=1000 381.alb & 1 & 0 & Solution & 120.21 & 140 & 138.00 &  1.43\\
instance n=1000 382.alb & 1 & 0 & Solution & 120.28 & 133 & 131.00 &  1.50\\
instance n=1000 383.alb & 1 & 0 & Solution & 120.19 & 141 & 138.00 &  2.13\\
instance n=1000 384.alb & 1 & 0 & Solution & 120.48 & 141 & 139.00 &  1.42\\
instance n=1000 385.alb & 1 & 0 & Solution & 120.15 & 137 & 135.00 &  1.46\\
instance n=1000 386.alb & 1 & 0 & Solution & 120.08 & 141 & 139.00 &  1.42\\
instance n=1000 387.alb & 1 & 0 & Solution & 120.07 & 139 & 137.00 &  1.44\\
instance n=1000 388.alb & 1 & 0 & Solution & 120.06 & 138 & 137.00 &  0.72\\
instance n=1000 389.alb & 1 & 0 & Solution & 120.34 & 138 & 136.00 &  1.45\\
instance n=1000 39.alb & 1 & 0 & Solution & 120.21 & 560 & 507.00 &  9.46\\
instance n=1000 390.alb & 1 & 0 & Solution & 120.27 & 138 & 136.00 &  1.45\\
instance n=1000 391.alb & 1 & 0 & Solution & 120.27 & 137 & 135.00 &  1.46\\
instance n=1000 392.alb & 1 & 0 & Solution & 120.15 & 137 & 136.00 &  0.73\\
instance n=1000 393.alb & 1 & 0 & Solution & 120.17 & 138 & 136.00 &  1.45\\
instance n=1000 394.alb & 1 & 0 & Solution & 120.52 & 140 & 138.00 &  1.43\\
instance n=1000 395.alb & 1 & 0 & Solution & 120.25 & 141 & 139.00 &  1.42\\
instance n=1000 396.alb & 1 & 0 & Solution & 120.21 & 138 & 136.00 &  1.45\\
instance n=1000 397.alb & 1 & 0 & Solution & 120.29 & 142 & 140.00 &  1.41\\
instance n=1000 398.alb & 1 & 0 & Solution & 120.19 & 136 & 134.00 &  1.47\\
instance n=1000 399.alb & 1 & 0 & Solution & 120.30 & 140 & 139.00 &  0.71\\
instance n=1000 4.alb & 1 & 0 & Solution & 120.09 & 139 & 138.00 &  0.72\\
instance n=1000 40.alb & 1 & 0 & Solution & 120.17 & 531 & 496.00 &  6.59\\
instance n=1000 400.alb & 1 & 0 & Solution & 120.45 & 142 & 140.00 &  1.41\\
instance n=1000 401.alb & 1 & 0 & Solution & 120.26 & 554 & 497.00 & 10.29\\
instance n=1000 402.alb & 1 & 0 & Solution & 120.35 & 559 & 500.00 & 10.55\\
instance n=1000 403.alb & 1 & 0 & Solution & 120.23 & 555 & 500.00 &  9.91\\
instance n=1000 404.alb & 1 & 0 & Solution & 120.25 & 555 & 500.00 &  9.91\\
instance n=1000 405.alb & 1 & 0 & Solution & 120.29 & 564 & 501.00 & 11.17\\
instance n=1000 406.alb & 1 & 0 & Solution & 120.40 & 547 & 495.00 &  9.51\\
instance n=1000 407.alb & 1 & 0 & Solution & 120.39 & 559 & 498.00 & 10.91\\
instance n=1000 408.alb & 1 & 0 & Solution & 120.32 & 564 & 501.00 & 11.17\\
instance n=1000 409.alb & 1 & 0 & Solution & 120.32 & 565 & 504.00 & 10.80\\
instance n=1000 41.alb & 1 & 0 & Solution & 120.20 & 543 & 500.00 &  7.92\\
instance n=1000 410.alb & 1 & 0 & Solution & 120.46 & 575 & 505.00 & 12.17\\
instance n=1000 411.alb & 1 & 0 & Solution & 120.31 & 559 & 498.00 & 10.91\\
instance n=1000 412.alb & 1 & 0 & Solution & 120.24 & 558 & 499.00 & 10.57\\
instance n=1000 413.alb & 1 & 0 & Solution & 120.25 & 564 & 503.00 & 10.82\\
instance n=1000 414.alb & 1 & 0 & Solution & 120.29 & 558 & 501.00 & 10.22\\
instance n=1000 415.alb & 1 & 0 & Solution & 120.23 & 559 & 501.00 & 10.38\\
instance n=1000 416.alb & 1 & 0 & Solution & 120.24 & 564 & 502.00 & 10.99\\
instance n=1000 417.alb & 1 & 0 & Solution & 120.33 & 585 & 512.00 & 12.48\\
instance n=1000 418.alb & 1 & 0 & Solution & 120.24 & 558 & 501.00 & 10.22\\
instance n=1000 419.alb & 1 & 0 & Solution & 120.27 & 579 & 510.00 & 11.92\\
instance n=1000 42.alb & 1 & 0 & Solution & 120.19 & 533 & 497.00 &  6.75\\
instance n=1000 420.alb & 1 & 0 & Solution & 120.18 & 561 & 501.00 & 10.70\\
instance n=1000 421.alb & 1 & 0 & Solution & 120.21 & 556 & 499.00 & 10.25\\
instance n=1000 422.alb & 1 & 0 & Solution & 120.25 & 552 & 495.00 & 10.33\\
instance n=1000 423.alb & 1 & 0 & Solution & 120.36 & 562 & 500.00 & 11.03\\
instance n=1000 424.alb & 1 & 0 & Solution & 120.34 & 550 & 495.00 & 10.00\\
instance n=1000 425.alb & 1 & 0 & Solution & 120.26 & 565 & 504.00 & 10.80\\
instance n=1000 426.alb & 1 & 0 & Solution & 120.10 & 229 & 224.00 &  2.18\\
instance n=1000 427.alb & 1 & 0 & Solution & 120.15 & 235 & 229.00 &  2.55\\
instance n=1000 428.alb & 1 & 0 & Solution & 120.14 & 228 & 224.00 &  1.75\\
instance n=1000 429.alb & 1 & 0 & Solution & 120.12 & 240 & 235.00 &  2.08\\
instance n=1000 43.alb & 1 & 0 & Solution & 120.17 & 534 & 496.00 &  7.12\\
instance n=1000 430.alb & 1 & 0 & Solution & 120.34 & 224 & 220.00 &  1.79\\
instance n=1000 431.alb & 1 & 0 & Solution & 120.21 & 234 & 230.00 &  1.71\\
instance n=1000 432.alb & 1 & 0 & Solution & 120.15 & 232 & 227.00 &  2.16\\
instance n=1000 433.alb & 1 & 0 & Solution & 120.16 & 234 & 229.00 &  2.14\\
instance n=1000 434.alb & 1 & 0 & Solution & 120.40 & 215 & 212.00 &  1.40\\
instance n=1000 435.alb & 1 & 0 & Solution & 120.40 & 232 & 227.00 &  2.16\\
instance n=1000 436.alb & 1 & 0 & Solution & 120.13 & 231 & 226.00 &  2.16\\
instance n=1000 437.alb & 1 & 0 & Solution & 120.44 & 226 & 222.00 &  1.77\\
instance n=1000 438.alb & 1 & 0 & Solution & 120.13 & 226 & 221.00 &  2.21\\
instance n=1000 439.alb & 1 & 0 & Solution & 120.20 & 229 & 225.00 &  1.75\\
instance n=1000 44.alb & 1 & 0 & Solution & 120.16 & 550 & 502.00 &  8.73\\
instance n=1000 440.alb & 1 & 0 & Solution & 120.56 & 230 & 225.00 &  2.17\\
instance n=1000 441.alb & 1 & 0 & Solution & 120.20 & 226 & 221.00 &  2.21\\
instance n=1000 442.alb & 1 & 0 & Solution & 120.12 & 235 & 230.00 &  2.13\\
instance n=1000 443.alb & 1 & 0 & Solution & 120.34 & 222 & 217.00 &  2.25\\
instance n=1000 444.alb & 1 & 0 & Solution & 120.23 & 227 & 222.00 &  2.20\\
instance n=1000 445.alb & 1 & 0 & Solution & 120.28 & 235 & 229.00 &  2.55\\
instance n=1000 446.alb & 1 & 0 & Solution & 120.35 & 232 & 228.00 &  1.72\\
instance n=1000 447.alb & 1 & 0 & Solution & 120.14 & 227 & 221.00 &  2.64\\
instance n=1000 448.alb & 1 & 0 & Solution & 120.22 & 226 & 222.00 &  1.77\\
instance n=1000 449.alb & 1 & 0 & Solution & 120.25 & 238 & 232.00 &  2.52\\
instance n=1000 45.alb & 1 & 0 & Solution & 120.18 & 524 & 492.00 &  6.11\\
instance n=1000 450.alb & 1 & 0 & Solution & 120.33 & 225 & 220.00 &  2.22\\
instance n=1000 451.alb & 1 & 0 & Solution & 120.23 & 140 & 136.00 &  2.86\\
instance n=1000 452.alb & 1 & 0 & Solution & 120.10 & 134 & 132.00 &  1.49\\
instance n=1000 453.alb & 1 & 0 & Solution & 120.41 & 141 & 138.00 &  2.13\\
instance n=1000 454.alb & 1 & 0 & Solution & 120.19 & 142 & 139.00 &  2.11\\
instance n=1000 455.alb & 1 & 0 & Solution & 120.68 & 139 & 136.00 &  2.16\\
instance n=1000 456.alb & 1 & 0 & Solution & 120.14 & 138 & 135.00 &  2.17\\
instance n=1000 457.alb & 1 & 0 & Solution & 120.11 & 140 & 137.00 &  2.14\\
instance n=1000 458.alb & 1 & 0 & Solution & 120.26 & 137 & 135.00 &  1.46\\
instance n=1000 459.alb & 1 & 0 & Solution & 120.33 & 140 & 137.00 &  2.14\\
instance n=1000 46.alb & 1 & 0 & Solution & 120.19 & 538 & 498.00 &  7.43\\
instance n=1000 460.alb & 1 & 0 & Solution & 120.13 & 141 & 138.00 &  2.13\\
instance n=1000 461.alb & 1 & 0 & Solution & 120.34 & 140 & 137.00 &  2.14\\
instance n=1000 462.alb & 1 & 0 & Solution & 120.59 & 139 & 136.00 &  2.16\\
instance n=1000 463.alb & 1 & 0 & Solution & 120.10 & 138 & 136.00 &  1.45\\
instance n=1000 464.alb & 1 & 0 & Solution & 120.30 & 141 & 138.00 &  2.13\\
instance n=1000 465.alb & 1 & 0 & Solution & 120.42 & 141 & 138.00 &  2.13\\
instance n=1000 466.alb & 1 & 0 & Solution & 120.39 & 137 & 133.00 &  2.92\\
instance n=1000 467.alb & 1 & 0 & Solution & 120.20 & 140 & 138.00 &  1.43\\
instance n=1000 468.alb & 1 & 0 & Solution & 120.27 & 139 & 137.00 &  1.44\\
instance n=1000 469.alb & 1 & 0 & Solution & 120.55 & 140 & 137.00 &  2.14\\
instance n=1000 47.alb & 1 & 0 & Solution & 120.19 & 542 & 499.00 &  7.93\\
instance n=1000 470.alb & 1 & 0 & Solution & 120.17 & 138 & 135.00 &  2.17\\
instance n=1000 471.alb & 1 & 0 & Solution & 120.36 & 138 & 135.00 &  2.17\\
instance n=1000 472.alb & 1 & 0 & Solution & 120.19 & 142 & 140.00 &  1.41\\
instance n=1000 473.alb & 1 & 0 & Solution & 120.14 & 138 & 135.00 &  2.17\\
instance n=1000 474.alb & 1 & 0 & Solution & 120.10 & 139 & 136.00 &  2.16\\
instance n=1000 475.alb & 1 & 0 & Solution & 120.41 & 139 & 136.00 &  2.16\\
instance n=1000 476.alb & 1 & 0 & Solution & 120.22 & 574 & 503.00 & 12.37\\
instance n=1000 477.alb & 1 & 0 & Solution & 120.34 & 586 & 507.00 & 13.48\\
instance n=1000 478.alb & 1 & 0 & Solution & 120.29 & 596 & 510.00 & 14.43\\
instance n=1000 479.alb & 1 & 0 & Solution & 120.20 & 579 & 503.00 & 13.13\\
instance n=1000 48.alb & 1 & 0 & Solution & 120.20 & 565 & 508.00 & 10.09\\
instance n=1000 480.alb & 1 & 0 & Solution & 120.19 & 566 & 498.00 & 12.01\\
instance n=1000 481.alb & 1 & 0 & Solution & 120.30 & 581 & 504.00 & 13.25\\
instance n=1000 482.alb & 1 & 0 & Solution & 120.21 & 596 & 505.00 & 15.27\\
instance n=1000 483.alb & 1 & 0 & Solution & 120.33 & 569 & 499.00 & 12.30\\
instance n=1000 484.alb & 1 & 0 & Solution & 120.21 & 589 & 508.00 & 13.75\\
instance n=1000 485.alb & 1 & 0 & Solution & 120.21 & 586 & 505.00 & 13.82\\
instance n=1000 486.alb & 1 & 0 & Solution & 120.21 & 571 & 500.00 & 12.43\\
instance n=1000 487.alb & 1 & 0 & Solution & 120.22 & 584 & 502.00 & 14.04\\
instance n=1000 488.alb & 1 & 0 & Solution & 120.28 & 572 & 502.00 & 12.24\\
instance n=1000 489.alb & 1 & 0 & Solution & 120.43 & 568 & 498.00 & 12.32\\
instance n=1000 49.alb & 1 & 0 & Solution & 120.19 & 544 & 500.00 &  8.09\\
instance n=1000 490.alb & 1 & 0 & Solution & 120.52 & 573 & 501.00 & 12.57\\
instance n=1000 491.alb & 1 & 0 & Solution & 120.36 & 575 & 500.00 & 13.04\\
instance n=1000 492.alb & 1 & 0 & Solution & 120.40 & 585 & 509.00 & 12.99\\
instance n=1000 493.alb & 1 & 0 & Solution & 120.27 & 561 & 495.00 & 11.76\\
instance n=1000 494.alb & 1 & 0 & Solution & 120.31 & 571 & 500.00 & 12.43\\
instance n=1000 495.alb & 1 & 0 & Solution & 120.23 & 587 & 507.00 & 13.63\\
instance n=1000 496.alb & 1 & 0 & Solution & 120.20 & 553 & 495.00 & 10.49\\
instance n=1000 497.alb & 1 & 0 & Solution & 120.25 & 566 & 499.00 & 11.84\\
instance n=1000 498.alb & 1 & 0 & Solution & 120.23 & 588 & 506.00 & 13.95\\
instance n=1000 499.alb & 1 & 0 & Solution & 120.27 & 566 & 499.00 & 11.84\\
instance n=1000 5.alb & 1 & 0 & Solution & 120.07 & 136 & 135.00 &  0.74\\
instance n=1000 50.alb & 1 & 0 & Solution & 120.15 & 526 & 493.00 &  6.27\\
instance n=1000 500.alb & 1 & 0 & Solution & 120.23 & 571 & 503.00 & 11.91\\
instance n=1000 501.alb & 1 & 0 & Solution & 120.23 & 234 & 227.00 &  2.99\\
instance n=1000 502.alb & 1 & 0 & Solution & 120.29 & 232 & 224.00 &  3.45\\
instance n=1000 503.alb & 1 & 0 & Solution & 120.69 & 233 & 224.00 &  3.86\\
instance n=1000 504.alb & 1 & 0 & Solution & 120.44 & 236 & 227.00 &  3.81\\
instance n=1000 505.alb & 1 & 0 & Solution & 120.24 & 219 & 213.00 &  2.74\\
instance n=1000 506.alb & 1 & 0 & Solution & 120.26 & 230 & 223.00 &  3.04\\
instance n=1000 507.alb & 1 & 0 & Solution & 120.18 & 228 & 220.00 &  3.51\\
instance n=1000 508.alb & 1 & 0 & Solution & 120.36 & 226 & 219.00 &  3.10\\
instance n=1000 509.alb & 1 & 0 & Solution & 120.13 & 232 & 225.00 &  3.02\\
instance n=1000 51.alb & 1 & 0 & Solution & 120.13 & 229 & 226.00 &  1.31\\
instance n=1000 510.alb & 1 & 0 & Solution & 120.43 & 235 & 226.00 &  3.83\\
instance n=1000 511.alb & 1 & 0 & Solution & 120.13 & 237 & 230.00 &  2.95\\
instance n=1000 512.alb & 1 & 0 & Solution & 120.27 & 226 & 219.00 &  3.10\\
instance n=1000 513.alb & 1 & 0 & Solution & 120.15 & 227 & 219.00 &  3.52\\
instance n=1000 514.alb & 1 & 0 & Solution & 120.14 & 233 & 226.00 &  3.00\\
instance n=1000 515.alb & 1 & 0 & Solution & 120.18 & 228 & 221.00 &  3.07\\
instance n=1000 516.alb & 1 & 0 & Solution & 120.47 & 237 & 229.00 &  3.38\\
instance n=1000 517.alb & 1 & 0 & Solution & 120.17 & 229 & 221.00 &  3.49\\
instance n=1000 518.alb & 1 & 0 & Solution & 120.48 & 226 & 220.00 &  2.65\\
instance n=1000 519.alb & 1 & 0 & Solution & 120.34 & 229 & 221.00 &  3.49\\
instance n=1000 52.alb & 1 & 0 & Solution & 120.09 & 231 & 228.00 &  1.30\\
instance n=1000 520.alb & 1 & 0 & Solution & 120.33 & 234 & 226.00 &  3.42\\
instance n=1000 521.alb & 1 & 0 & Solution & 120.25 & 236 & 229.00 &  2.97\\
instance n=1000 522.alb & 1 & 0 & Solution & 120.21 & 221 & 215.00 &  2.71\\
instance n=1000 523.alb & 1 & 0 & Solution & 120.23 & 228 & 220.00 &  3.51\\
instance n=1000 524.alb & 1 & 0 & Solution & 120.26 & 232 & 226.00 &  2.59\\
instance n=1000 525.alb & 1 & 0 & Solution & 120.36 & 229 & 221.00 &  3.49\\
instance n=1000 53.alb & 1 & 0 & Solution & 120.09 & 230 & 227.00 &  1.30\\
instance n=1000 54.alb & 1 & 0 & Solution & 120.10 & 223 & 219.00 &  1.79\\
instance n=1000 55.alb & 1 & 0 & Solution & 120.10 & 220 & 217.00 &  1.36\\
instance n=1000 56.alb & 1 & 0 & Solution & 120.08 & 232 & 228.00 &  1.72\\
instance n=1000 57.alb & 1 & 0 & Solution & 120.08 & 227 & 224.00 &  1.32\\
instance n=1000 58.alb & 1 & 0 & Solution & 120.11 & 227 & 224.00 &  1.32\\
instance n=1000 59.alb & 1 & 0 & Solution & 120.11 & 226 & 223.00 &  1.33\\
instance n=1000 6.alb & 1 & 0 & Solution & 120.08 & 143 & 141.00 &  1.40\\
instance n=1000 60.alb & 1 & 0 & Solution & 120.08 & 233 & 230.00 &  1.29\\
instance n=1000 61.alb & 1 & 0 & Solution & 120.11 & 233 & 229.00 &  1.72\\
instance n=1000 62.alb & 1 & 0 & Solution & 120.08 & 226 & 223.00 &  1.33\\
instance n=1000 63.alb & 1 & 0 & Solution & 120.10 & 230 & 227.00 &  1.30\\
instance n=1000 64.alb & 1 & 0 & Solution & 120.11 & 233 & 229.00 &  1.72\\
instance n=1000 65.alb & 1 & 0 & Solution & 120.11 & 227 & 225.00 &  0.88\\
instance n=1000 66.alb & 1 & 0 & Solution & 120.08 & 230 & 227.00 &  1.30\\
instance n=1000 67.alb & 1 & 0 & Solution & 120.08 & 226 & 223.00 &  1.33\\
instance n=1000 68.alb & 1 & 0 & Solution & 120.11 & 230 & 226.00 &  1.74\\
instance n=1000 69.alb & 1 & 0 & Solution & 120.11 & 227 & 224.00 &  1.32\\
instance n=1000 7.alb & 1 & 0 & Solution & 120.07 & 138 & 136.00 &  1.45\\
instance n=1000 70.alb & 1 & 0 & Solution & 120.11 & 231 & 228.00 &  1.30\\
instance n=1000 71.alb & 1 & 0 & Solution & 120.09 & 233 & 230.00 &  1.29\\
instance n=1000 72.alb & 1 & 0 & Solution & 120.08 & 225 & 222.00 &  1.33\\
instance n=1000 73.alb & 1 & 0 & Solution & 120.12 & 224 & 221.00 &  1.34\\
instance n=1000 74.alb & 1 & 0 & Solution & 120.14 & 231 & 227.00 &  1.73\\
instance n=1000 75.alb & 1 & 0 & Solution & 120.10 & 230 & 227.00 &  1.30\\
instance n=1000 76.alb & 1 & 0 & Solution & 120.10 & 137 & 136.00 &  0.73\\
instance n=1000 77.alb & 1 & 0 & Solution & 120.06 & 137 & 136.00 &  0.73\\
instance n=1000 78.alb & 1 & 0 & Solution & 120.11 & 140 & 138.00 &  1.43\\
instance n=1000 79.alb & 1 & 0 & Solution & 120.13 & 143 & 142.00 &  0.70\\
instance n=1000 8.alb & 1 & 0 & Solution & 120.08 & 140 & 138.00 &  1.43\\
instance n=1000 80.alb & 1 & 0 & Solution & 120.06 & 141 & 140.00 &  0.71\\
instance n=1000 81.alb & 1 & 0 & Solution & 120.14 & 138 & 136.00 &  1.45\\
instance n=1000 82.alb & 1 & 0 & Solution & 120.13 & 137 & 136.00 &  0.73\\
instance n=1000 83.alb & 1 & 0 & Solution & 120.13 & 141 & 140.00 &  0.71\\
instance n=1000 84.alb & 1 & 0 & Solution & 120.07 & 136 & 135.00 &  0.74\\
instance n=1000 85.alb & 1 & 0 & Solution & 120.08 & 137 & 136.00 &  0.73\\
instance n=1000 86.alb & 1 & 0 & Solution & 120.13 & 139 & 138.00 &  0.72\\
instance n=1000 87.alb & 1 & 0 & Solution & 120.10 & 142 & 140.00 &  1.41\\
instance n=1000 88.alb & 1 & 0 & Solution & 120.09 & 142 & 140.00 &  1.41\\
instance n=1000 89.alb & 1 & 0 & Solution & 120.07 & 141 & 140.00 &  0.71\\
instance n=1000 9.alb & 1 & 0 & Solution & 120.06 & 136 & 134.00 &  1.47\\
instance n=1000 90.alb & 1 & 0 & Solution & 120.11 & 139 & 138.00 &  0.72\\
instance n=1000 91.alb & 1 & 0 & Solution & 120.05 & 142 & 141.00 &  0.70\\
instance n=1000 92.alb & 1 & 0 & Solution & 120.06 & 137 & 136.00 &  0.73\\
instance n=1000 93.alb & 1 & 0 & Solution & 120.11 & 138 & 137.00 &  0.72\\
instance n=1000 94.alb & 1 & 0 & Solution & 120.10 & 139 & 137.00 &  1.44\\
instance n=1000 95.alb & 1 & 0 & Solution & 120.14 & 137 & 136.00 &  0.73\\
instance n=1000 96.alb & 1 & 0 & Solution & 120.09 & 139 & 137.00 &  1.44\\
instance n=1000 97.alb & 1 & 0 & Solution & 120.12 & 140 & 138.00 &  1.43\\
instance n=1000 98.alb & 1 & 0 & Solution & 120.08 & 137 & 136.00 &  0.73\\
instance n=1000 99.alb & 1 & 0 & Solution & 120.08 & 137 & 136.00 &  0.73\\
instance n=100 1.alb & 1 & 0 & Optimal &  5.53 & 23 & 23.00 &  0.00\\
instance n=100 10.alb & 1 & 0 & Optimal &  0.05 & 22 & 22.00 &  0.00\\
instance n=100 100.alb & 1 & 0 & Optimal &  2.72 & 25 & 25.00 &  0.00\\
instance n=100 101.alb & 1 & 0 & Optimal &  0.73 & 15 & 15.00 &  0.00\\
instance n=100 102.alb & 1 & 0 & Optimal &  0.15 & 14 & 14.00 &  0.00\\
instance n=100 103.alb & 1 & 0 & Optimal &  0.13 & 14 & 14.00 &  0.00\\
instance n=100 104.alb & 1 & 0 & Optimal &  0.13 & 14 & 14.00 &  0.00\\
instance n=100 105.alb & 1 & 0 & Optimal &  0.13 & 13 & 13.00 &  0.00\\
instance n=100 106.alb & 1 & 0 & Optimal &  0.16 & 14 & 14.00 &  0.00\\
instance n=100 107.alb & 1 & 0 & Optimal &  0.20 & 14 & 14.00 &  0.00\\
instance n=100 108.alb & 1 & 0 & Optimal &  3.54 & 14 & 14.00 &  0.00\\
instance n=100 109.alb & 1 & 0 & Optimal &  0.19 & 15 & 15.00 &  0.00\\
instance n=100 11.alb & 1 & 0 & Optimal &  0.06 & 24 & 24.00 &  0.00\\
instance n=100 110.alb & 1 & 0 & Optimal &  0.22 & 13 & 13.00 &  0.00\\
instance n=100 111.alb & 1 & 0 & Optimal &  0.19 & 16 & 16.00 &  0.00\\
instance n=100 112.alb & 1 & 0 & Optimal &  3.03 & 13 & 13.00 &  0.00\\
instance n=100 113.alb & 1 & 0 & Optimal &  0.12 & 14 & 14.00 &  0.00\\
instance n=100 114.alb & 1 & 0 & Optimal &  0.29 & 13 & 13.00 &  0.00\\
instance n=100 115.alb & 1 & 0 & Optimal &  0.22 & 14 & 14.00 &  0.00\\
instance n=100 116.alb & 1 & 0 & Optimal &  0.16 & 16 & 16.00 &  0.00\\
instance n=100 117.alb & 1 & 0 & Optimal &  4.30 & 15 & 15.00 &  0.00\\
instance n=100 118.alb & 1 & 0 & Optimal &  0.11 & 15 & 15.00 &  0.00\\
instance n=100 119.alb & 1 & 0 & Optimal &  0.18 & 14 & 14.00 &  0.00\\
instance n=100 12.alb & 1 & 0 & Optimal &  2.85 & 25 & 25.00 &  0.00\\
instance n=100 120.alb & 1 & 0 & Optimal &  0.22 & 14 & 14.00 &  0.00\\
instance n=100 121.alb & 1 & 0 & Optimal &  0.15 & 15 & 15.00 &  0.00\\
instance n=100 122.alb & 1 & 0 & Optimal &  0.23 & 13 & 13.00 &  0.00\\
instance n=100 123.alb & 1 & 0 & Optimal &  0.19 & 15 & 15.00 &  0.00\\
instance n=100 124.alb & 1 & 0 & Optimal &  3.34 & 15 & 15.00 &  0.00\\
instance n=100 125.alb & 1 & 0 & Optimal &  0.12 & 14 & 14.00 &  0.00\\
instance n=100 126.alb & 1 & 0 & Solution & 120.01 & 51 & 49.00 &  3.92\\
instance n=100 127.alb & 1 & 0 & Solution & 120.01 & 53 & 49.00 &  7.55\\
instance n=100 128.alb & 1 & 0 & Solution & 120.03 & 57 & 52.00 &  8.77\\
instance n=100 129.alb & 1 & 0 & Solution & 120.02 & 55 & 50.00 &  9.09\\
instance n=100 13.alb & 1 & 0 & Optimal &  0.48 & 24 & 24.00 &  0.00\\
instance n=100 130.alb & 1 & 0 & Solution & 120.02 & 55 & 51.00 &  7.27\\
instance n=100 131.alb & 1 & 0 & Solution & 120.03 & 53 & 50.00 &  5.66\\
instance n=100 132.alb & 1 & 0 & Solution & 120.02 & 57 & 53.00 &  7.02\\
instance n=100 133.alb & 1 & 0 & Solution & 120.03 & 55 & 51.00 &  7.27\\
instance n=100 134.alb & 1 & 0 & Solution & 120.03 & 54 & 51.00 &  5.56\\
instance n=100 135.alb & 1 & 0 & Solution & 120.03 & 56 & 51.00 &  8.93\\
instance n=100 136.alb & 1 & 0 & Solution & 120.05 & 52 & 49.00 &  5.77\\
instance n=100 137.alb & 1 & 0 & Solution & 120.03 & 54 & 50.00 &  7.41\\
instance n=100 138.alb & 1 & 0 & Solution & 120.02 & 56 & 52.00 &  7.14\\
instance n=100 139.alb & 1 & 0 & Solution & 120.03 & 52 & 49.00 &  5.77\\
instance n=100 14.alb & 1 & 0 & Optimal &  1.46 & 20 & 20.00 &  0.00\\
instance n=100 140.alb & 1 & 0 & Solution & 120.04 & 55 & 51.00 &  7.27\\
instance n=100 141.alb & 1 & 0 & Solution & 120.03 & 51 & 49.00 &  3.92\\
instance n=100 142.alb & 1 & 0 & Solution & 120.03 & 55 & 50.00 &  9.09\\
instance n=100 143.alb & 1 & 0 & Solution & 120.02 & 53 & 51.00 &  3.77\\
instance n=100 144.alb & 1 & 0 & Solution & 120.03 & 49 & 47.00 &  4.08\\
instance n=100 145.alb & 1 & 0 & Solution & 120.04 & 56 & 51.00 &  8.93\\
instance n=100 146.alb & 1 & 0 & Solution & 120.03 & 53 & 50.00 &  5.66\\
instance n=100 147.alb & 1 & 0 & Solution & 120.03 & 59 & 52.00 & 11.86\\
instance n=100 148.alb & 1 & 0 & Solution & 120.04 & 53 & 50.00 &  5.66\\
instance n=100 149.alb & 1 & 0 & Solution & 120.03 & 55 & 51.00 &  7.27\\
instance n=100 15.alb & 1 & 0 & Optimal &  0.06 & 24 & 24.00 &  0.00\\
instance n=100 150.alb & 1 & 0 & Solution & 120.03 & 58 & 51.00 & 12.07\\
instance n=100 151.alb & 1 & 0 & Solution & 120.04 & 22 & 21.00 &  4.55\\
instance n=100 152.alb & 1 & 0 & Optimal &  0.31 & 22 & 22.00 &  0.00\\
instance n=100 153.alb & 1 & 0 & Optimal &  0.17 & 21 & 21.00 &  0.00\\
instance n=100 154.alb & 1 & 0 & Optimal &  0.25 & 25 & 25.00 &  0.00\\
instance n=100 155.alb & 1 & 0 & Optimal &  0.23 & 22 & 22.00 &  0.00\\
instance n=100 156.alb & 1 & 0 & Optimal &  0.28 & 23 & 23.00 &  0.00\\
instance n=100 157.alb & 1 & 0 & Optimal &  1.29 & 26 & 26.00 &  0.00\\
instance n=100 158.alb & 1 & 0 & Optimal &  0.29 & 23 & 23.00 &  0.00\\
instance n=100 159.alb & 1 & 0 & Optimal &  0.14 & 19 & 19.00 &  0.00\\
instance n=100 16.alb & 1 & 0 & Optimal &  0.04 & 23 & 23.00 &  0.00\\
instance n=100 160.alb & 1 & 0 & Optimal &  0.30 & 22 & 22.00 &  0.00\\
instance n=100 161.alb & 1 & 0 & Optimal & 117.95 & 22 & 22.00 &  0.00\\
instance n=100 162.alb & 1 & 0 & Solution & 120.04 & 23 & 22.00 &  4.35\\
instance n=100 163.alb & 1 & 0 & Optimal &  0.20 & 25 & 25.00 &  0.00\\
instance n=100 164.alb & 1 & 0 & Optimal &  0.20 & 23 & 23.00 &  0.00\\
instance n=100 165.alb & 1 & 0 & Solution & 120.02 & 25 & 24.00 &  4.00\\
instance n=100 166.alb & 1 & 0 & Optimal &  2.08 & 24 & 24.00 &  0.00\\
instance n=100 167.alb & 1 & 0 & Optimal &  0.25 & 22 & 22.00 &  0.00\\
instance n=100 168.alb & 1 & 0 & Solution & 120.05 & 22 & 21.00 &  4.55\\
instance n=100 169.alb & 1 & 0 & Optimal &  0.38 & 21 & 21.00 &  0.00\\
instance n=100 17.alb & 1 & 0 & Solution & 120.02 & 22 & 21.00 &  4.55\\
instance n=100 170.alb & 1 & 0 & Optimal &  4.00 & 24 & 24.00 &  0.00\\
instance n=100 171.alb & 1 & 0 & Solution & 120.03 & 25 & 24.00 &  4.00\\
instance n=100 172.alb & 1 & 0 & Optimal &  0.25 & 24 & 24.00 &  0.00\\
instance n=100 173.alb & 1 & 0 & Solution & 120.03 & 25 & 24.00 &  4.00\\
instance n=100 174.alb & 1 & 0 & Optimal &  4.22 & 22 & 22.00 &  0.00\\
instance n=100 175.alb & 1 & 0 & Solution & 120.03 & 27 & 26.00 &  3.70\\
instance n=100 176.alb & 1 & 0 & Optimal &  0.20 & 13 & 13.00 &  0.00\\
instance n=100 177.alb & 1 & 0 & Optimal &  0.18 & 14 & 14.00 &  0.00\\
instance n=100 178.alb & 1 & 0 & Optimal &  0.23 & 15 & 15.00 &  0.00\\
instance n=100 179.alb & 1 & 0 & Optimal &  0.14 & 15 & 15.00 &  0.00\\
instance n=100 18.alb & 1 & 0 & Solution & 120.01 & 20 & 19.00 &  5.00\\
instance n=100 180.alb & 1 & 0 & Optimal &  0.22 & 15 & 15.00 &  0.00\\
instance n=100 181.alb & 1 & 0 & Optimal &  0.22 & 13 & 13.00 &  0.00\\
instance n=100 182.alb & 1 & 0 & Optimal &  0.23 & 15 & 15.00 &  0.00\\
instance n=100 183.alb & 1 & 0 & Optimal &  0.20 & 14 & 14.00 &  0.00\\
instance n=100 184.alb & 1 & 0 & Optimal &  0.28 & 14 & 14.00 &  0.00\\
instance n=100 185.alb & 1 & 0 & Optimal &  0.26 & 15 & 15.00 &  0.00\\
instance n=100 186.alb & 1 & 0 & Optimal &  2.05 & 14 & 14.00 &  0.00\\
instance n=100 187.alb & 1 & 0 & Optimal & 10.45 & 13 & 13.00 &  0.00\\
instance n=100 188.alb & 1 & 0 & Optimal &  0.26 & 16 & 16.00 &  0.00\\
instance n=100 189.alb & 1 & 0 & Optimal &  0.23 & 14 & 14.00 &  0.00\\
instance n=100 19.alb & 1 & 0 & Optimal &  0.43 & 23 & 23.00 &  0.00\\
instance n=100 190.alb & 1 & 0 & Optimal &  0.26 & 13 & 13.00 &  0.00\\
instance n=100 191.alb & 1 & 0 & Optimal &  0.23 & 14 & 14.00 &  0.00\\
instance n=100 192.alb & 1 & 0 & Optimal &  2.87 & 13 & 13.00 &  0.00\\
instance n=100 193.alb & 1 & 0 & Optimal &  0.38 & 15 & 15.00 &  0.00\\
instance n=100 194.alb & 1 & 0 & Optimal &  0.30 & 15 & 15.00 &  0.00\\
instance n=100 195.alb & 1 & 0 & Optimal &  0.23 & 15 & 15.00 &  0.00\\
instance n=100 196.alb & 1 & 0 & Optimal &  0.28 & 15 & 15.00 &  0.00\\
instance n=100 197.alb & 1 & 0 & Optimal &  0.17 & 15 & 15.00 &  0.00\\
instance n=100 198.alb & 1 & 0 & Optimal &  2.85 & 13 & 13.00 &  0.00\\
instance n=100 199.alb & 1 & 0 & Optimal &  0.25 & 14 & 14.00 &  0.00\\
instance n=100 2.alb & 1 & 0 & Optimal &  0.05 & 21 & 21.00 &  0.00\\
instance n=100 20.alb & 1 & 0 & Optimal &  0.04 & 21 & 21.00 &  0.00\\
instance n=100 200.alb & 1 & 0 & Optimal &  0.22 & 15 & 15.00 &  0.00\\
instance n=100 201.alb & 1 & 0 & Solution & 120.06 & 53 & 51.00 &  3.77\\
instance n=100 202.alb & 1 & 0 & Solution & 120.05 & 61 & 52.00 & 14.75\\
instance n=100 203.alb & 1 & 0 & Solution & 120.03 & 53 & 49.00 &  7.55\\
instance n=100 204.alb & 1 & 0 & Solution & 120.05 & 51 & 48.00 &  5.88\\
instance n=100 205.alb & 1 & 0 & Solution & 120.03 & 57 & 51.00 & 10.53\\
instance n=100 206.alb & 1 & 0 & Solution & 120.03 & 52 & 49.00 &  5.77\\
instance n=100 207.alb & 1 & 0 & Solution & 120.01 & 52 & 49.00 &  5.77\\
instance n=100 208.alb & 1 & 0 & Solution & 120.03 & 57 & 51.00 & 10.53\\
instance n=100 209.alb & 1 & 0 & Solution & 120.07 & 55 & 51.00 &  7.27\\
instance n=100 21.alb & 1 & 0 & Optimal &  0.45 & 21 & 21.00 &  0.00\\
instance n=100 210.alb & 1 & 0 & Solution & 120.04 & 53 & 49.00 &  7.55\\
instance n=100 211.alb & 1 & 0 & Solution & 120.04 & 52 & 49.00 &  5.77\\
instance n=100 212.alb & 1 & 0 & Solution & 120.04 & 53 & 50.00 &  5.66\\
instance n=100 213.alb & 1 & 0 & Solution & 120.02 & 53 & 50.00 &  5.66\\
instance n=100 214.alb & 1 & 0 & Solution & 120.04 & 55 & 50.00 &  9.09\\
instance n=100 215.alb & 1 & 0 & Solution & 120.05 & 49 & 47.00 &  4.08\\
instance n=100 216.alb & 1 & 0 & Solution & 120.01 & 53 & 50.00 &  5.66\\
instance n=100 217.alb & 1 & 0 & Solution & 120.06 & 52 & 49.00 &  5.77\\
instance n=100 218.alb & 1 & 0 & Solution & 120.03 & 54 & 50.00 &  7.41\\
instance n=100 219.alb & 1 & 0 & Solution & 120.05 & 52 & 49.00 &  5.77\\
instance n=100 22.alb & 1 & 0 & Solution & 120.01 & 25 & 24.00 &  4.00\\
instance n=100 220.alb & 1 & 0 & Solution & 120.04 & 54 & 50.00 &  7.41\\
instance n=100 221.alb & 1 & 0 & Solution & 120.03 & 57 & 51.00 & 10.53\\
instance n=100 222.alb & 1 & 0 & Solution & 120.05 & 53 & 50.00 &  5.66\\
instance n=100 223.alb & 1 & 0 & Solution & 120.04 & 51 & 49.00 &  3.92\\
instance n=100 224.alb & 1 & 0 & Solution & 120.05 & 56 & 51.00 &  8.93\\
instance n=100 225.alb & 1 & 0 & Solution & 120.03 & 53 & 51.00 &  3.77\\
instance n=100 226.alb & 1 & 0 & Solution & 120.05 & 25 & 24.00 &  4.00\\
instance n=100 227.alb & 1 & 0 & Optimal & 82.63 & 26 & 26.00 &  0.00\\
instance n=100 228.alb & 1 & 0 & Optimal &  8.12 & 22 & 22.00 &  0.00\\
instance n=100 229.alb & 1 & 0 & Optimal &  0.40 & 24 & 24.00 &  0.00\\
instance n=100 23.alb & 1 & 0 & Optimal &  0.07 & 24 & 24.00 &  0.00\\
instance n=100 230.alb & 1 & 0 & Optimal & 19.43 & 23 & 23.00 &  0.00\\
instance n=100 231.alb & 1 & 0 & Optimal & 28.64 & 22 & 22.00 &  0.00\\
instance n=100 232.alb & 1 & 0 & Optimal &  0.52 & 22 & 22.00 &  0.00\\
instance n=100 233.alb & 1 & 0 & Solution & 120.03 & 23 & 22.00 &  4.35\\
instance n=100 234.alb & 1 & 0 & Optimal &  0.36 & 23 & 23.00 &  0.00\\
instance n=100 235.alb & 1 & 0 & Optimal &  2.66 & 26 & 26.00 &  0.00\\
instance n=100 236.alb & 1 & 0 & Solution & 120.03 & 23 & 22.00 &  4.35\\
instance n=100 237.alb & 1 & 0 & Optimal &  5.82 & 23 & 23.00 &  0.00\\
instance n=100 238.alb & 1 & 0 & Optimal &  4.46 & 23 & 23.00 &  0.00\\
instance n=100 239.alb & 1 & 0 & Optimal &  0.31 & 21 & 21.00 &  0.00\\
instance n=100 24.alb & 1 & 0 & Optimal &  0.07 & 24 & 24.00 &  0.00\\
instance n=100 240.alb & 1 & 0 & Optimal &  3.22 & 22 & 22.00 &  0.00\\
instance n=100 241.alb & 1 & 0 & Optimal &  4.17 & 22 & 22.00 &  0.00\\
instance n=100 242.alb & 1 & 0 & Optimal &  4.05 & 23 & 23.00 &  0.00\\
instance n=100 243.alb & 1 & 0 & Optimal & 112.03 & 23 & 23.00 &  0.00\\
instance n=100 244.alb & 1 & 0 & Optimal &  0.38 & 21 & 21.00 &  0.00\\
instance n=100 245.alb & 1 & 0 & Solution & 120.05 & 24 & 23.00 &  4.17\\
instance n=100 246.alb & 1 & 0 & Optimal &  6.42 & 26 & 26.00 &  0.00\\
instance n=100 247.alb & 1 & 0 & Optimal &  4.47 & 22 & 22.00 &  0.00\\
instance n=100 248.alb & 1 & 0 & Optimal &  3.60 & 19 & 19.00 &  0.00\\
instance n=100 249.alb & 1 & 0 & Optimal &  3.14 & 21 & 21.00 &  0.00\\
instance n=100 25.alb & 1 & 0 & Optimal &  0.52 & 22 & 22.00 &  0.00\\
instance n=100 250.alb & 1 & 0 & Optimal &  2.48 & 24 & 24.00 &  0.00\\
instance n=100 251.alb & 1 & 0 & Optimal &  0.19 & 15 & 15.00 &  0.00\\
instance n=100 252.alb & 1 & 0 & Optimal &  0.48 & 14 & 14.00 &  0.00\\
instance n=100 253.alb & 1 & 0 & Optimal &  0.20 & 14 & 14.00 &  0.00\\
instance n=100 254.alb & 1 & 0 & Optimal &  0.23 & 14 & 14.00 &  0.00\\
instance n=100 255.alb & 1 & 0 & Optimal &  0.20 & 14 & 14.00 &  0.00\\
instance n=100 256.alb & 1 & 0 & Optimal &  0.30 & 15 & 15.00 &  0.00\\
instance n=100 257.alb & 1 & 0 & Optimal &  3.57 & 12 & 12.00 &  0.00\\
instance n=100 258.alb & 1 & 0 & Optimal &  3.55 & 14 & 14.00 &  0.00\\
instance n=100 259.alb & 1 & 0 & Optimal &  1.84 & 15 & 15.00 &  0.00\\
instance n=100 26.alb & 1 & 0 & Optimal &  0.55 & 14 & 14.00 &  0.00\\
instance n=100 260.alb & 1 & 0 & Optimal &  0.31 & 15 & 15.00 &  0.00\\
instance n=100 261.alb & 1 & 0 & Optimal &  0.35 & 14 & 14.00 &  0.00\\
instance n=100 262.alb & 1 & 0 & Optimal &  0.28 & 14 & 14.00 &  0.00\\
instance n=100 263.alb & 1 & 0 & Optimal &  0.39 & 14 & 14.00 &  0.00\\
instance n=100 264.alb & 1 & 0 & Optimal &  0.28 & 15 & 15.00 &  0.00\\
instance n=100 265.alb & 1 & 0 & Optimal &  0.39 & 14 & 14.00 &  0.00\\
instance n=100 266.alb & 1 & 0 & Optimal &  3.19 & 13 & 13.00 &  0.00\\
instance n=100 267.alb & 1 & 0 & Optimal &  0.35 & 13 & 13.00 &  0.00\\
instance n=100 268.alb & 1 & 0 & Optimal &  0.28 & 15 & 15.00 &  0.00\\
instance n=100 269.alb & 1 & 0 & Optimal &  0.33 & 15 & 15.00 &  0.00\\
instance n=100 27.alb & 1 & 0 & Optimal &  0.32 & 13 & 13.00 &  0.00\\
instance n=100 270.alb & 1 & 0 & Optimal &  0.46 & 13 & 13.00 &  0.00\\
instance n=100 271.alb & 1 & 0 & Optimal & 15.96 & 13 & 13.00 &  0.00\\
instance n=100 272.alb & 1 & 0 & Optimal &  0.26 & 14 & 14.00 &  0.00\\
instance n=100 273.alb & 1 & 0 & Optimal &  7.35 & 13 & 13.00 &  0.00\\
instance n=100 274.alb & 1 & 0 & Optimal &  4.30 & 13 & 13.00 &  0.00\\
instance n=100 275.alb & 1 & 0 & Optimal &  0.43 & 13 & 13.00 &  0.00\\
instance n=100 276.alb & 1 & 0 & Solution & 120.05 & 60 & 52.00 & 13.33\\
instance n=100 277.alb & 1 & 0 & Solution & 120.04 & 57 & 52.00 &  8.77\\
instance n=100 278.alb & 1 & 0 & Solution & 120.05 & 58 & 52.00 & 10.34\\
instance n=100 279.alb & 1 & 0 & Solution & 120.06 & 54 & 51.00 &  5.56\\
instance n=100 28.alb & 1 & 0 & Optimal &  0.40 & 14 & 14.00 &  0.00\\
instance n=100 280.alb & 1 & 0 & Solution & 120.06 & 56 & 51.00 &  8.93\\
instance n=100 281.alb & 1 & 0 & Solution & 120.05 & 62 & 52.00 & 16.13\\
instance n=100 282.alb & 1 & 0 & Solution & 120.04 & 60 & 53.00 & 11.67\\
instance n=100 283.alb & 1 & 0 & Solution & 120.07 & 55 & 51.00 &  7.27\\
instance n=100 284.alb & 1 & 0 & Solution & 120.08 & 55 & 51.00 &  7.27\\
instance n=100 285.alb & 1 & 0 & Solution & 120.05 & 55 & 51.00 &  7.27\\
instance n=100 286.alb & 1 & 0 & Solution & 120.04 & 57 & 51.00 & 10.53\\
instance n=100 287.alb & 1 & 0 & Solution & 120.04 & 54 & 50.00 &  7.41\\
instance n=100 288.alb & 1 & 0 & Solution & 120.07 & 56 & 51.00 &  8.93\\
instance n=100 289.alb & 1 & 0 & Solution & 120.05 & 62 & 52.00 & 16.13\\
instance n=100 29.alb & 1 & 0 & Optimal &  0.35 & 14 & 14.00 &  0.00\\
instance n=100 290.alb & 1 & 0 & Solution & 120.04 & 55 & 51.00 &  7.27\\
instance n=100 291.alb & 1 & 0 & Solution & 120.04 & 53 & 49.00 &  7.55\\
instance n=100 292.alb & 1 & 0 & Solution & 120.05 & 58 & 51.00 & 12.07\\
instance n=100 293.alb & 1 & 0 & Solution & 120.05 & 53 & 50.00 &  5.66\\
instance n=100 294.alb & 1 & 0 & Solution & 120.05 & 58 & 52.00 & 10.34\\
instance n=100 295.alb & 1 & 0 & Solution & 120.05 & 57 & 51.00 & 10.53\\
instance n=100 296.alb & 1 & 0 & Solution & 120.06 & 55 & 51.00 &  7.27\\
instance n=100 297.alb & 1 & 0 & Solution & 120.04 & 59 & 51.00 & 13.56\\
instance n=100 298.alb & 1 & 0 & Solution & 120.03 & 59 & 52.00 & 11.86\\
instance n=100 299.alb & 1 & 0 & Solution & 120.09 & 55 & 50.00 &  9.09\\
instance n=100 3.alb & 1 & 0 & Optimal &  0.05 & 20 & 20.00 &  0.00\\
instance n=100 30.alb & 1 & 0 & Optimal &  0.04 & 15 & 15.00 &  0.00\\
instance n=100 300.alb & 1 & 0 & Solution & 120.04 & 54 & 50.00 &  7.41\\
instance n=100 301.alb & 1 & 0 & Optimal &  0.49 & 23 & 23.00 &  0.00\\
instance n=100 302.alb & 1 & 0 & Optimal &  0.39 & 24 & 24.00 &  0.00\\
instance n=100 303.alb & 1 & 0 & Optimal &  5.07 & 24 & 24.00 &  0.00\\
instance n=100 304.alb & 1 & 0 & Optimal &  1.82 & 21 & 21.00 &  0.00\\
instance n=100 305.alb & 1 & 0 & Optimal &  0.31 & 22 & 22.00 &  0.00\\
instance n=100 306.alb & 1 & 0 & Optimal &  0.50 & 24 & 24.00 &  0.00\\
instance n=100 307.alb & 1 & 0 & Solution & 120.06 & 24 & 23.00 &  4.17\\
instance n=100 308.alb & 1 & 0 & Solution & 120.06 & 21 & 20.00 &  4.76\\
instance n=100 309.alb & 1 & 0 & Solution & 120.05 & 22 & 21.00 &  4.55\\
instance n=100 31.alb & 1 & 0 & Optimal &  0.05 & 14 & 14.00 &  0.00\\
instance n=100 310.alb & 1 & 0 & Optimal &  2.46 & 23 & 23.00 &  0.00\\
instance n=100 311.alb & 1 & 0 & Optimal &  0.41 & 21 & 21.00 &  0.00\\
instance n=100 312.alb & 1 & 0 & Optimal &  0.41 & 22 & 22.00 &  0.00\\
instance n=100 313.alb & 1 & 0 & Optimal &  0.61 & 23 & 23.00 &  0.00\\
instance n=100 314.alb & 1 & 0 & Optimal &  0.60 & 19 & 19.00 &  0.00\\
instance n=100 315.alb & 1 & 0 & Solution & 120.07 & 23 & 22.00 &  4.35\\
instance n=100 316.alb & 1 & 0 & Optimal &  0.38 & 24 & 24.00 &  0.00\\
instance n=100 317.alb & 1 & 0 & Optimal &  0.43 & 26 & 26.00 &  0.00\\
instance n=100 318.alb & 1 & 0 & Optimal &  0.41 & 21 & 21.00 &  0.00\\
instance n=100 319.alb & 1 & 0 & Optimal &  2.56 & 23 & 23.00 &  0.00\\
instance n=100 32.alb & 1 & 0 & Optimal &  0.04 & 14 & 14.00 &  0.00\\
instance n=100 320.alb & 1 & 0 & Optimal &  0.30 & 22 & 22.00 &  0.00\\
instance n=100 321.alb & 1 & 0 & Optimal &  0.42 & 26 & 26.00 &  0.00\\
instance n=100 322.alb & 1 & 0 & Solution & 120.07 & 24 & 23.00 &  4.17\\
instance n=100 323.alb & 1 & 0 & Optimal &  0.33 & 24 & 24.00 &  0.00\\
instance n=100 324.alb & 1 & 0 & Optimal &  0.50 & 23 & 23.00 &  0.00\\
instance n=100 325.alb & 1 & 0 & Solution & 120.09 & 26 & 25.00 &  3.85\\
instance n=100 326.alb & 1 & 0 & Optimal &  0.27 & 13 & 13.00 &  0.00\\
instance n=100 327.alb & 1 & 0 & Optimal &  0.57 & 14 & 14.00 &  0.00\\
instance n=100 328.alb & 1 & 0 & Optimal & 17.30 & 14 & 14.00 &  0.00\\
instance n=100 329.alb & 1 & 0 & Optimal &  0.45 & 14 & 14.00 &  0.00\\
instance n=100 33.alb & 1 & 0 & Optimal &  0.05 & 15 & 15.00 &  0.00\\
instance n=100 330.alb & 1 & 0 & Optimal &  9.76 & 14 & 14.00 &  0.00\\
instance n=100 331.alb & 1 & 0 & Optimal &  0.27 & 14 & 14.00 &  0.00\\
instance n=100 332.alb & 1 & 0 & Optimal &  0.35 & 14 & 14.00 &  0.00\\
instance n=100 333.alb & 1 & 0 & Optimal &  0.38 & 15 & 15.00 &  0.00\\
instance n=100 334.alb & 1 & 0 & Optimal &  3.56 & 14 & 14.00 &  0.00\\
instance n=100 335.alb & 1 & 0 & Optimal &  0.38 & 13 & 13.00 &  0.00\\
instance n=100 336.alb & 1 & 0 & Optimal &  0.43 & 15 & 15.00 &  0.00\\
instance n=100 337.alb & 1 & 0 & Optimal &  0.78 & 13 & 13.00 &  0.00\\
instance n=100 338.alb & 1 & 0 & Solution & 120.06 & 15 & 14.00 &  6.67\\
instance n=100 339.alb & 1 & 0 & Optimal &  0.31 & 14 & 14.00 &  0.00\\
instance n=100 34.alb & 1 & 0 & Optimal &  0.05 & 15 & 15.00 &  0.00\\
instance n=100 340.alb & 1 & 0 & Optimal &  0.41 & 14 & 14.00 &  0.00\\
instance n=100 341.alb & 1 & 0 & Optimal &  0.49 & 16 & 16.00 &  0.00\\
instance n=100 342.alb & 1 & 0 & Optimal &  2.14 & 14 & 14.00 &  0.00\\
instance n=100 343.alb & 1 & 0 & Optimal &  0.57 & 16 & 16.00 &  0.00\\
instance n=100 344.alb & 1 & 0 & Optimal &  0.45 & 15 & 15.00 &  0.00\\
instance n=100 345.alb & 1 & 0 & Optimal &  0.32 & 14 & 14.00 &  0.00\\
instance n=100 346.alb & 1 & 0 & Optimal &  0.37 & 14 & 14.00 &  0.00\\
instance n=100 347.alb & 1 & 0 & Optimal &  0.40 & 14 & 14.00 &  0.00\\
instance n=100 348.alb & 1 & 0 & Optimal &  0.36 & 14 & 14.00 &  0.00\\
instance n=100 349.alb & 1 & 0 & Optimal &  0.36 & 13 & 13.00 &  0.00\\
instance n=100 35.alb & 1 & 0 & Optimal &  0.05 & 15 & 15.00 &  0.00\\
instance n=100 350.alb & 1 & 0 & Optimal &  0.43 & 14 & 14.00 &  0.00\\
instance n=100 351.alb & 1 & 0 & Solution & 120.08 & 59 & 52.00 & 11.86\\
instance n=100 352.alb & 1 & 0 & Solution & 120.07 & 63 & 52.00 & 17.46\\
instance n=100 353.alb & 1 & 0 & Solution & 120.06 & 51 & 49.00 &  3.92\\
instance n=100 354.alb & 1 & 0 & Solution & 120.05 & 53 & 49.00 &  7.55\\
instance n=100 355.alb & 1 & 0 & Solution & 120.04 & 55 & 51.00 &  7.27\\
instance n=100 356.alb & 1 & 0 & Solution & 120.09 & 61 & 53.00 & 13.11\\
instance n=100 357.alb & 1 & 0 & Solution & 120.08 & 54 & 50.00 &  7.41\\
instance n=100 358.alb & 1 & 0 & Solution & 120.02 & 53 & 50.00 &  5.66\\
instance n=100 359.alb & 1 & 0 & Solution & 120.07 & 54 & 50.00 &  7.41\\
instance n=100 36.alb & 1 & 0 & Optimal &  1.95 & 14 & 14.00 &  0.00\\
instance n=100 360.alb & 1 & 0 & Solution & 120.08 & 55 & 51.00 &  7.27\\
instance n=100 361.alb & 1 & 0 & Solution & 120.05 & 52 & 49.00 &  5.77\\
instance n=100 362.alb & 1 & 0 & Solution & 120.07 & 57 & 51.00 & 10.53\\
instance n=100 363.alb & 1 & 0 & Solution & 120.09 & 53 & 50.00 &  5.66\\
instance n=100 364.alb & 1 & 0 & Solution & 120.06 & 53 & 50.00 &  5.66\\
instance n=100 365.alb & 1 & 0 & Solution & 120.06 & 53 & 50.00 &  5.66\\
instance n=100 366.alb & 1 & 0 & Solution & 120.08 & 61 & 53.00 & 13.11\\
instance n=100 367.alb & 1 & 0 & Solution & 120.05 & 56 & 51.00 &  8.93\\
instance n=100 368.alb & 1 & 0 & Solution & 120.04 & 59 & 52.00 & 11.86\\
instance n=100 369.alb & 1 & 0 & Solution & 120.05 & 51 & 49.00 &  3.92\\
instance n=100 37.alb & 1 & 0 & Optimal &  0.04 & 14 & 14.00 &  0.00\\
instance n=100 370.alb & 1 & 0 & Solution & 120.05 & 57 & 52.00 &  8.77\\
instance n=100 371.alb & 1 & 0 & Solution & 120.08 & 53 & 50.00 &  5.66\\
instance n=100 372.alb & 1 & 0 & Solution & 120.04 & 49 & 47.00 &  4.08\\
instance n=100 373.alb & 1 & 0 & Solution & 120.09 & 51 & 49.00 &  3.92\\
instance n=100 374.alb & 1 & 0 & Solution & 120.06 & 53 & 50.00 &  5.66\\
instance n=100 375.alb & 1 & 0 & Solution & 120.07 & 58 & 52.00 & 10.34\\
instance n=100 376.alb & 1 & 0 & Optimal &  0.59 & 23 & 23.00 &  0.00\\
instance n=100 377.alb & 1 & 0 & Solution & 120.10 & 21 & 20.00 &  4.76\\
instance n=100 378.alb & 1 & 0 & Optimal &  5.15 & 22 & 22.00 &  0.00\\
instance n=100 379.alb & 1 & 0 & Optimal & 50.33 & 23 & 23.00 &  0.00\\
instance n=100 38.alb & 1 & 0 & Optimal &  0.05 & 14 & 14.00 &  0.00\\
instance n=100 380.alb & 1 & 0 & Solution & 120.07 & 23 & 22.00 &  4.35\\
instance n=100 381.alb & 1 & 0 & Optimal &  2.24 & 24 & 24.00 &  0.00\\
instance n=100 382.alb & 1 & 0 & Optimal &  6.90 & 25 & 25.00 &  0.00\\
instance n=100 383.alb & 1 & 0 & Optimal &  0.50 & 25 & 25.00 &  0.00\\
instance n=100 384.alb & 1 & 0 & Optimal &  1.24 & 25 & 25.00 &  0.00\\
instance n=100 385.alb & 1 & 0 & Optimal &  0.44 & 22 & 22.00 &  0.00\\
instance n=100 386.alb & 1 & 0 & Optimal & 67.35 & 23 & 23.00 &  0.00\\
instance n=100 387.alb & 1 & 0 & Optimal &  0.91 & 22 & 22.00 &  0.00\\
instance n=100 388.alb & 1 & 0 & Solution & 120.06 & 26 & 25.00 &  3.85\\
instance n=100 389.alb & 1 & 0 & Optimal &  0.35 & 23 & 23.00 &  0.00\\
instance n=100 39.alb & 1 & 0 & Optimal &  0.03 & 14 & 14.00 &  0.00\\
instance n=100 390.alb & 1 & 0 & Solution & 120.06 & 23 & 22.00 &  4.35\\
instance n=100 391.alb & 1 & 0 & Optimal &  0.44 & 20 & 20.00 &  0.00\\
instance n=100 392.alb & 1 & 0 & Optimal &  0.46 & 22 & 22.00 &  0.00\\
instance n=100 393.alb & 1 & 0 & Solution & 120.06 & 24 & 23.00 &  4.17\\
instance n=100 394.alb & 1 & 0 & Optimal &  0.64 & 22 & 22.00 &  0.00\\
instance n=100 395.alb & 1 & 0 & Optimal &  9.99 & 24 & 24.00 &  0.00\\
instance n=100 396.alb & 1 & 0 & Optimal & 11.36 & 20 & 20.00 &  0.00\\
instance n=100 397.alb & 1 & 0 & Solution & 120.06 & 26 & 25.00 &  3.85\\
instance n=100 398.alb & 1 & 0 & Solution & 120.07 & 25 & 24.00 &  4.00\\
instance n=100 399.alb & 1 & 0 & Optimal &  1.01 & 23 & 23.00 &  0.00\\
instance n=100 4.alb & 1 & 0 & Optimal &  0.07 & 24 & 24.00 &  0.00\\
instance n=100 40.alb & 1 & 0 & Optimal &  0.09 & 14 & 14.00 &  0.00\\
instance n=100 400.alb & 1 & 0 & Optimal &  4.68 & 24 & 24.00 &  0.00\\
instance n=100 401.alb & 1 & 0 & Optimal &  0.46 & 15 & 15.00 &  0.00\\
instance n=100 402.alb & 1 & 0 & Optimal &  0.46 & 15 & 15.00 &  0.00\\
instance n=100 403.alb & 1 & 0 & Optimal &  0.60 & 14 & 14.00 &  0.00\\
instance n=100 404.alb & 1 & 0 & Optimal &  0.57 & 15 & 15.00 &  0.00\\
instance n=100 405.alb & 1 & 0 & Optimal &  0.45 & 13 & 13.00 &  0.00\\
instance n=100 406.alb & 1 & 0 & Optimal &  0.49 & 14 & 14.00 &  0.00\\
instance n=100 407.alb & 1 & 0 & Optimal &  0.69 & 15 & 15.00 &  0.00\\
instance n=100 408.alb & 1 & 0 & Optimal &  0.71 & 14 & 14.00 &  0.00\\
instance n=100 409.alb & 1 & 0 & Optimal &  0.36 & 15 & 15.00 &  0.00\\
instance n=100 41.alb & 1 & 0 & Optimal &  0.08 & 13 & 13.00 &  0.00\\
instance n=100 410.alb & 1 & 0 & Optimal &  0.33 & 14 & 14.00 &  0.00\\
instance n=100 411.alb & 1 & 0 & Optimal &  4.24 & 14 & 14.00 &  0.00\\
instance n=100 412.alb & 1 & 0 & Optimal &  0.55 & 14 & 14.00 &  0.00\\
instance n=100 413.alb & 1 & 0 & Optimal &  0.61 & 14 & 14.00 &  0.00\\
instance n=100 414.alb & 1 & 0 & Optimal & 34.16 & 14 & 14.00 &  0.00\\
instance n=100 415.alb & 1 & 0 & Optimal &  4.46 & 13 & 13.00 &  0.00\\
instance n=100 416.alb & 1 & 0 & Optimal &  0.53 & 14 & 14.00 &  0.00\\
instance n=100 417.alb & 1 & 0 & Optimal &  0.52 & 15 & 15.00 &  0.00\\
instance n=100 418.alb & 1 & 0 & Optimal &  0.60 & 16 & 16.00 &  0.00\\
instance n=100 419.alb & 1 & 0 & Optimal &  4.26 & 14 & 14.00 &  0.00\\
instance n=100 42.alb & 1 & 0 & Optimal &  0.05 & 14 & 14.00 &  0.00\\
instance n=100 420.alb & 1 & 0 & Optimal &  0.35 & 14 & 14.00 &  0.00\\
instance n=100 421.alb & 1 & 0 & Optimal &  0.35 & 14 & 14.00 &  0.00\\
instance n=100 422.alb & 1 & 0 & Optimal &  0.46 & 15 & 15.00 &  0.00\\
instance n=100 423.alb & 1 & 0 & Optimal &  3.71 & 14 & 14.00 &  0.00\\
instance n=100 424.alb & 1 & 0 & Optimal &  0.41 & 14 & 14.00 &  0.00\\
instance n=100 425.alb & 1 & 0 & Optimal &  0.58 & 15 & 15.00 &  0.00\\
instance n=100 426.alb & 1 & 0 & Solution & 120.09 & 60 & 53.00 & 11.67\\
instance n=100 427.alb & 1 & 0 & Solution & 120.06 & 56 & 50.00 & 10.71\\
instance n=100 428.alb & 1 & 0 & Solution & 120.07 & 55 & 51.00 &  7.27\\
instance n=100 429.alb & 1 & 0 & Solution & 120.06 & 59 & 52.00 & 11.86\\
instance n=100 43.alb & 1 & 0 & Optimal &  0.83 & 14 & 14.00 &  0.00\\
instance n=100 430.alb & 1 & 0 & Solution & 120.07 & 54 & 50.00 &  7.41\\
instance n=100 431.alb & 1 & 0 & Solution & 120.05 & 54 & 50.00 &  7.41\\
instance n=100 432.alb & 1 & 0 & Solution & 120.03 & 56 & 51.00 &  8.93\\
instance n=100 433.alb & 1 & 0 & Solution & 120.04 & 53 & 49.00 &  7.55\\
instance n=100 434.alb & 1 & 0 & Solution & 120.08 & 57 & 51.00 & 10.53\\
instance n=100 435.alb & 1 & 0 & Solution & 120.08 & 56 & 50.00 & 10.71\\
instance n=100 436.alb & 1 & 0 & Solution & 120.09 & 52 & 48.00 &  7.69\\
instance n=100 437.alb & 1 & 0 & Solution & 120.03 & 53 & 50.00 &  5.66\\
instance n=100 438.alb & 1 & 0 & Solution & 120.05 & 55 & 51.00 &  7.27\\
instance n=100 439.alb & 1 & 0 & Solution & 120.06 & 56 & 51.00 &  8.93\\
instance n=100 44.alb & 1 & 0 & Optimal &  0.05 & 14 & 14.00 &  0.00\\
instance n=100 440.alb & 1 & 0 & Solution & 120.08 & 53 & 49.00 &  7.55\\
instance n=100 441.alb & 1 & 0 & Solution & 120.06 & 53 & 50.00 &  5.66\\
instance n=100 442.alb & 1 & 0 & Solution & 120.03 & 53 & 48.00 &  9.43\\
instance n=100 443.alb & 1 & 0 & Solution & 120.11 & 56 & 50.00 & 10.71\\
instance n=100 444.alb & 1 & 0 & Solution & 120.07 & 54 & 50.00 &  7.41\\
instance n=100 445.alb & 1 & 0 & Solution & 120.04 & 55 & 51.00 &  7.27\\
instance n=100 446.alb & 1 & 0 & Solution & 120.02 & 57 & 52.00 &  8.77\\
instance n=100 447.alb & 1 & 0 & Solution & 120.06 & 54 & 50.00 &  7.41\\
instance n=100 448.alb & 1 & 0 & Solution & 120.09 & 56 & 51.00 &  8.93\\
instance n=100 449.alb & 1 & 0 & Solution & 120.05 & 56 & 50.00 & 10.71\\
instance n=100 45.alb & 1 & 0 & Optimal &  0.04 & 14 & 14.00 &  0.00\\
instance n=100 450.alb & 1 & 0 & Solution & 120.10 & 54 & 51.00 &  5.56\\
instance n=100 451.alb & 1 & 0 & Optimal &  3.90 & 26 & 26.00 &  0.00\\
instance n=100 452.alb & 1 & 0 & Optimal &  3.31 & 22 & 22.00 &  0.00\\
instance n=100 453.alb & 1 & 0 & Optimal &  2.72 & 24 & 24.00 &  0.00\\
instance n=100 454.alb & 1 & 0 & Optimal &  1.96 & 23 & 23.00 &  0.00\\
instance n=100 455.alb & 1 & 0 & Optimal &  3.85 & 23 & 23.00 &  0.00\\
instance n=100 456.alb & 1 & 0 & Optimal &  4.02 & 26 & 26.00 &  0.00\\
instance n=100 457.alb & 1 & 0 & Optimal &  2.42 & 23 & 23.00 &  0.00\\
instance n=100 458.alb & 1 & 0 & Optimal &  2.61 & 24 & 24.00 &  0.00\\
instance n=100 459.alb & 1 & 0 & Optimal &  3.47 & 23 & 23.00 &  0.00\\
instance n=100 46.alb & 1 & 0 & Optimal &  0.05 & 14 & 14.00 &  0.00\\
instance n=100 460.alb & 1 & 0 & Optimal &  5.71 & 23 & 23.00 &  0.00\\
instance n=100 461.alb & 1 & 0 & Optimal &  2.91 & 23 & 23.00 &  0.00\\
instance n=100 462.alb & 1 & 0 & Optimal &  4.90 & 23 & 23.00 &  0.00\\
instance n=100 463.alb & 1 & 0 & Optimal &  1.62 & 26 & 26.00 &  0.00\\
instance n=100 464.alb & 1 & 0 & Optimal &  4.91 & 25 & 25.00 &  0.00\\
instance n=100 465.alb & 1 & 0 & Optimal &  4.37 & 22 & 22.00 &  0.00\\
instance n=100 466.alb & 1 & 0 & Optimal &  3.47 & 26 & 25.00 &  3.85\\
instance n=100 467.alb & 1 & 0 & Optimal &  6.89 & 21 & 21.00 &  0.00\\
instance n=100 468.alb & 1 & 0 & Optimal &  8.72 & 25 & 25.00 &  0.00\\
instance n=100 469.alb & 1 & 0 & Optimal &  1.76 & 22 & 22.00 &  0.00\\
instance n=100 47.alb & 1 & 0 & Optimal &  0.08 & 14 & 14.00 &  0.00\\
instance n=100 470.alb & 1 & 0 & Optimal & 34.68 & 26 & 26.00 &  0.00\\
instance n=100 471.alb & 1 & 0 & Optimal &  7.15 & 26 & 26.00 &  0.00\\
instance n=100 472.alb & 1 & 0 & Optimal &  0.85 & 23 & 23.00 &  0.00\\
instance n=100 473.alb & 1 & 0 & Optimal &  2.90 & 28 & 28.00 &  0.00\\
instance n=100 474.alb & 1 & 0 & Optimal &  1.88 & 23 & 23.00 &  0.00\\
instance n=100 475.alb & 1 & 0 & Optimal & 33.97 & 24 & 24.00 &  0.00\\
instance n=100 476.alb & 1 & 0 & Optimal &  0.49 & 14 & 14.00 &  0.00\\
instance n=100 477.alb & 1 & 0 & Optimal &  0.52 & 14 & 14.00 &  0.00\\
instance n=100 478.alb & 1 & 0 & Optimal &  0.71 & 14 & 14.00 &  0.00\\
instance n=100 479.alb & 1 & 0 & Optimal &  0.97 & 16 & 16.00 &  0.00\\
instance n=100 48.alb & 1 & 0 & Optimal &  0.07 & 15 & 15.00 &  0.00\\
instance n=100 480.alb & 1 & 0 & Optimal &  1.16 & 15 & 15.00 &  0.00\\
instance n=100 481.alb & 1 & 0 & Optimal &  1.82 & 15 & 15.00 &  0.00\\
instance n=100 482.alb & 1 & 0 & Optimal &  2.27 & 15 & 15.00 &  0.00\\
instance n=100 483.alb & 1 & 0 & Optimal &  1.41 & 14 & 14.00 &  0.00\\
instance n=100 484.alb & 1 & 0 & Optimal &  0.54 & 14 & 14.00 &  0.00\\
instance n=100 485.alb & 1 & 0 & Optimal &  1.91 & 16 & 16.00 &  0.00\\
instance n=100 486.alb & 1 & 0 & Optimal &  0.85 & 15 & 15.00 &  0.00\\
instance n=100 487.alb & 1 & 0 & Optimal &  1.90 & 15 & 15.00 &  0.00\\
instance n=100 488.alb & 1 & 0 & Optimal &  1.12 & 16 & 16.00 &  0.00\\
instance n=100 489.alb & 1 & 0 & Optimal &  3.57 & 13 & 13.00 &  0.00\\
instance n=100 49.alb & 1 & 0 & Optimal &  0.08 & 14 & 14.00 &  0.00\\
instance n=100 490.alb & 1 & 0 & Optimal &  1.13 & 15 & 15.00 &  0.00\\
instance n=100 491.alb & 1 & 0 & Optimal &  1.38 & 16 & 16.00 &  0.00\\
instance n=100 492.alb & 1 & 0 & Optimal &  2.66 & 14 & 14.00 &  0.00\\
instance n=100 493.alb & 1 & 0 & Optimal &  1.67 & 14 & 14.00 &  0.00\\
instance n=100 494.alb & 1 & 0 & Optimal &  0.72 & 14 & 14.00 &  0.00\\
instance n=100 495.alb & 1 & 0 & Optimal &  1.54 & 15 & 15.00 &  0.00\\
instance n=100 496.alb & 1 & 0 & Optimal &  1.23 & 14 & 14.00 &  0.00\\
instance n=100 497.alb & 1 & 0 & Optimal &  0.52 & 13 & 13.00 &  0.00\\
instance n=100 498.alb & 1 & 0 & Optimal &  1.04 & 14 & 14.00 &  0.00\\
instance n=100 499.alb & 1 & 0 & Optimal &  1.26 & 14 & 14.00 &  0.00\\
instance n=100 5.alb & 1 & 0 & Optimal &  0.08 & 22 & 22.00 &  0.00\\
instance n=100 50.alb & 1 & 0 & Optimal &  0.06 & 14 & 14.00 &  0.00\\
instance n=100 500.alb & 1 & 0 & Optimal &  0.70 & 14 & 14.00 &  0.00\\
instance n=100 501.alb & 1 & 0 & Solution & 120.09 & 63 & 58.00 &  7.94\\
instance n=100 502.alb & 1 & 0 & Solution & 120.08 & 64 & 61.00 &  4.69\\
instance n=100 503.alb & 1 & 0 & Solution & 120.06 & 60 & 55.00 &  8.33\\
instance n=100 504.alb & 1 & 0 & Solution & 120.06 & 60 & 58.00 &  3.33\\
instance n=100 505.alb & 1 & 0 & Solution & 120.07 & 61 & 55.00 &  9.84\\
instance n=100 506.alb & 1 & 0 & Solution & 120.08 & 58 & 53.00 &  8.62\\
instance n=100 507.alb & 1 & 0 & Solution & 120.05 & 59 & 55.00 &  6.78\\
instance n=100 508.alb & 1 & 0 & Optimal & 118.93 & 56 & 56.00 &  0.00\\
instance n=100 509.alb & 1 & 0 & Solution & 120.11 & 57 & 54.00 &  5.26\\
instance n=100 51.alb & 1 & 0 & Solution & 120.03 & 51 & 48.00 &  5.88\\
instance n=100 510.alb & 1 & 0 & Solution & 120.05 & 58 & 55.00 &  5.17\\
instance n=100 511.alb & 1 & 0 & Solution & 120.08 & 60 & 57.00 &  5.00\\
instance n=100 512.alb & 1 & 0 & Solution & 120.09 & 60 & 58.00 &  3.33\\
instance n=100 513.alb & 1 & 0 & Solution & 120.10 & 62 & 56.00 &  9.68\\
instance n=100 514.alb & 1 & 0 & Solution & 120.04 & 58 & 55.00 &  5.17\\
instance n=100 515.alb & 1 & 0 & Solution & 120.10 & 61 & 56.00 &  8.20\\
instance n=100 516.alb & 1 & 0 & Solution & 120.09 & 70 & 60.00 & 14.29\\
instance n=100 517.alb & 1 & 0 & Solution & 120.10 & 62 & 57.00 &  8.06\\
instance n=100 518.alb & 1 & 0 & Solution & 120.05 & 57 & 53.00 &  7.02\\
instance n=100 519.alb & 1 & 0 & Solution & 120.07 & 61 & 58.00 &  4.92\\
instance n=100 52.alb & 1 & 0 & Solution & 120.01 & 53 & 50.00 &  5.66\\
instance n=100 520.alb & 1 & 0 & Solution & 120.03 & 60 & 56.00 &  6.67\\
instance n=100 521.alb & 1 & 0 & Solution & 120.06 & 70 & 61.00 & 12.86\\
instance n=100 522.alb & 1 & 0 & Solution & 120.11 & 59 & 55.00 &  6.78\\
instance n=100 523.alb & 1 & 0 & Solution & 120.10 & 55 & 53.00 &  3.64\\
instance n=100 524.alb & 1 & 0 & Solution & 120.10 & 59 & 55.00 &  6.78\\
instance n=100 525.alb & 1 & 0 & Solution & 120.08 & 62 & 56.00 &  9.68\\
instance n=100 53.alb & 1 & 0 & Solution & 120.00 & 52 & 50.00 &  3.85\\
instance n=100 54.alb & 1 & 0 & Solution & 120.01 & 51 & 49.00 &  3.92\\
instance n=100 55.alb & 1 & 0 & Solution & 120.02 & 53 & 50.00 &  5.66\\
instance n=100 56.alb & 1 & 0 & Solution & 120.01 & 52 & 50.00 &  3.85\\
instance n=100 57.alb & 1 & 0 & Solution & 120.01 & 55 & 51.00 &  7.27\\
instance n=100 58.alb & 1 & 0 & Solution & 120.01 & 57 & 52.00 &  8.77\\
instance n=100 59.alb & 1 & 0 & Solution & 120.02 & 57 & 51.00 & 10.53\\
instance n=100 6.alb & 1 & 0 & Optimal &  0.36 & 22 & 22.00 &  0.00\\
instance n=100 60.alb & 1 & 0 & Solution & 120.01 & 54 & 51.00 &  5.56\\
instance n=100 61.alb & 1 & 0 & Solution & 120.02 & 54 & 51.00 &  5.56\\
instance n=100 62.alb & 1 & 0 & Solution & 120.01 & 52 & 49.00 &  5.77\\
instance n=100 63.alb & 1 & 0 & Solution & 120.01 & 61 & 52.00 & 14.75\\
instance n=100 64.alb & 1 & 0 & Solution & 120.00 & 57 & 51.00 & 10.53\\
instance n=100 65.alb & 1 & 0 & Solution & 120.03 & 62 & 53.00 & 14.52\\
instance n=100 66.alb & 1 & 0 & Solution & 120.01 & 52 & 49.00 &  5.77\\
instance n=100 67.alb & 1 & 0 & Solution & 120.02 & 55 & 51.00 &  7.27\\
instance n=100 68.alb & 1 & 0 & Solution & 120.03 & 57 & 49.00 & 14.04\\
instance n=100 69.alb & 1 & 0 & Solution & 120.02 & 53 & 51.00 &  3.77\\
instance n=100 7.alb & 1 & 0 & Optimal &  0.05 & 26 & 26.00 &  0.00\\
instance n=100 70.alb & 1 & 0 & Solution & 120.02 & 53 & 50.00 &  5.66\\
instance n=100 71.alb & 1 & 0 & Solution & 120.02 & 53 & 50.00 &  5.66\\
instance n=100 72.alb & 1 & 0 & Solution & 120.03 & 54 & 50.00 &  7.41\\
instance n=100 73.alb & 1 & 0 & Solution & 120.03 & 56 & 52.00 &  7.14\\
instance n=100 74.alb & 1 & 0 & Solution & 120.03 & 52 & 49.00 &  5.77\\
instance n=100 75.alb & 1 & 0 & Solution & 120.03 & 55 & 51.00 &  7.27\\
instance n=100 76.alb & 1 & 0 & Optimal &  0.15 & 23 & 23.00 &  0.00\\
instance n=100 77.alb & 1 & 0 & Optimal &  0.13 & 20 & 20.00 &  0.00\\
instance n=100 78.alb & 1 & 0 & Optimal &  3.36 & 21 & 21.00 &  0.00\\
instance n=100 79.alb & 1 & 0 & Optimal &  0.14 & 21 & 21.00 &  0.00\\
instance n=100 8.alb & 1 & 0 & Optimal &  0.09 & 24 & 24.00 &  0.00\\
instance n=100 80.alb & 1 & 0 & Optimal &  2.20 & 22 & 22.00 &  0.00\\
instance n=100 81.alb & 1 & 0 & Optimal &  2.78 & 20 & 20.00 &  0.00\\
instance n=100 82.alb & 1 & 0 & Optimal &  0.87 & 21 & 21.00 &  0.00\\
instance n=100 83.alb & 1 & 0 & Optimal &  0.12 & 22 & 22.00 &  0.00\\
instance n=100 84.alb & 1 & 0 & Solution & 120.03 & 27 & 26.00 &  3.70\\
instance n=100 85.alb & 1 & 0 & Solution & 120.01 & 25 & 24.00 &  4.00\\
instance n=100 86.alb & 1 & 0 & Optimal &  0.34 & 23 & 23.00 &  0.00\\
instance n=100 87.alb & 1 & 0 & Optimal &  0.13 & 22 & 22.00 &  0.00\\
instance n=100 88.alb & 1 & 0 & Solution & 120.01 & 24 & 23.00 &  4.17\\
instance n=100 89.alb & 1 & 0 & Optimal &  1.33 & 24 & 24.00 &  0.00\\
instance n=100 9.alb & 1 & 0 & Optimal & 24.45 & 23 & 23.00 &  0.00\\
instance n=100 90.alb & 1 & 0 & Solution & 120.03 & 21 & 20.00 &  4.76\\
instance n=100 91.alb & 1 & 0 & Optimal &  0.17 & 25 & 25.00 &  0.00\\
instance n=100 92.alb & 1 & 0 & Optimal &  0.12 & 24 & 24.00 &  0.00\\
instance n=100 93.alb & 1 & 0 & Optimal &  5.69 & 27 & 27.00 &  0.00\\
instance n=100 94.alb & 1 & 0 & Optimal &  4.16 & 22 & 22.00 &  0.00\\
instance n=100 95.alb & 1 & 0 & Optimal &  1.53 & 21 & 21.00 &  0.00\\
instance n=100 96.alb & 1 & 0 & Optimal &  1.63 & 21 & 21.00 &  0.00\\
instance n=100 97.alb & 1 & 0 & Optimal &  0.99 & 22 & 22.00 &  0.00\\
instance n=100 98.alb & 1 & 0 & Optimal &  0.29 & 22 & 22.00 &  0.00\\
instance n=100 99.alb & 1 & 0 & Optimal &  0.15 & 22 & 22.00 &  0.00\\
instance n=20 1.alb & 1 & 0 & Optimal &  0.03 & 3 &  3.00 &  0.00\\
instance n=20 10.alb & 1 & 0 & Optimal &  0.03 & 3 &  3.00 &  0.00\\
instance n=20 100.alb & 1 & 0 & Optimal &  0.38 & 11 & 11.00 &  0.00\\
instance n=20 101.alb & 1 & 0 & Optimal &  4.28 & 13 & 13.00 &  0.00\\
instance n=20 102.alb & 1 & 0 & Optimal &  0.82 & 13 & 13.00 &  0.00\\
instance n=20 103.alb & 1 & 0 & Optimal &  0.25 & 12 & 12.00 &  0.00\\
instance n=20 104.alb & 1 & 0 & Optimal &  0.25 & 11 & 11.00 &  0.00\\
instance n=20 105.alb & 1 & 0 & Optimal &  0.24 & 12 & 12.00 &  0.00\\
instance n=20 106.alb & 1 & 0 & Optimal &  0.08 & 10 & 10.00 &  0.00\\
instance n=20 107.alb & 1 & 0 & Optimal &  2.06 & 14 & 14.00 &  0.00\\
instance n=20 108.alb & 1 & 0 & Optimal &  3.49 & 15 & 15.00 &  0.00\\
instance n=20 109.alb & 1 & 0 & Optimal &  0.56 & 12 & 12.00 &  0.00\\
instance n=20 11.alb & 1 & 0 & Optimal &  0.02 & 3 &  3.00 &  0.00\\
instance n=20 110.alb & 1 & 0 & Optimal &  0.24 & 11 & 11.00 &  0.00\\
instance n=20 111.alb & 1 & 0 & Optimal &  0.58 & 13 & 13.00 &  0.00\\
instance n=20 112.alb & 1 & 0 & Optimal &  0.27 & 11 & 11.00 &  0.00\\
instance n=20 113.alb & 1 & 0 & Optimal &  0.64 & 12 & 12.00 &  0.00\\
instance n=20 114.alb & 1 & 0 & Optimal &  0.88 & 13 & 13.00 &  0.00\\
instance n=20 115.alb & 1 & 0 & Optimal &  0.16 & 11 & 11.00 &  0.00\\
instance n=20 116.alb & 1 & 0 & Optimal &  0.09 & 5 &  5.00 &  0.00\\
instance n=20 117.alb & 1 & 0 & Optimal &  0.09 & 5 &  5.00 &  0.00\\
instance n=20 118.alb & 1 & 0 & Optimal &  0.06 & 5 &  5.00 &  0.00\\
instance n=20 119.alb & 1 & 0 & Optimal &  0.12 & 6 &  6.00 &  0.00\\
instance n=20 12.alb & 1 & 0 & Optimal &  0.02 & 3 &  3.00 &  0.00\\
instance n=20 120.alb & 1 & 0 & Optimal &  0.08 & 6 &  6.00 &  0.00\\
instance n=20 121.alb & 1 & 0 & Optimal &  0.11 & 5 &  5.00 &  0.00\\
instance n=20 122.alb & 1 & 0 & Optimal &  0.09 & 6 &  6.00 &  0.00\\
instance n=20 123.alb & 1 & 0 & Optimal &  0.10 & 5 &  5.00 &  0.00\\
instance n=20 124.alb & 1 & 0 & Optimal &  0.06 & 5 &  5.00 &  0.00\\
instance n=20 125.alb & 1 & 0 & Optimal &  0.08 & 5 &  5.00 &  0.00\\
instance n=20 126.alb & 1 & 0 & Optimal &  0.08 & 5 &  5.00 &  0.00\\
instance n=20 127.alb & 1 & 0 & Optimal &  0.09 & 4 &  4.00 &  0.00\\
instance n=20 128.alb & 1 & 0 & Optimal &  0.08 & 5 &  5.00 &  0.00\\
instance n=20 129.alb & 1 & 0 & Optimal &  0.09 & 5 &  5.00 &  0.00\\
instance n=20 13.alb & 1 & 0 & Optimal &  0.03 & 3 &  3.00 &  0.00\\
instance n=20 130.alb & 1 & 0 & Optimal &  0.09 & 6 &  6.00 &  0.00\\
instance n=20 131.alb & 1 & 0 & Optimal &  0.09 & 7 &  7.00 &  0.00\\
instance n=20 132.alb & 1 & 0 & Optimal &  0.06 & 4 &  4.00 &  0.00\\
instance n=20 133.alb & 1 & 0 & Optimal &  0.08 & 5 &  5.00 &  0.00\\
instance n=20 134.alb & 1 & 0 & Optimal &  0.11 & 6 &  6.00 &  0.00\\
instance n=20 135.alb & 1 & 0 & Optimal &  0.08 & 6 &  6.00 &  0.00\\
instance n=20 136.alb & 1 & 0 & Optimal &  0.39 & 6 &  6.00 &  0.00\\
instance n=20 137.alb & 1 & 0 & Optimal &  0.08 & 5 &  5.00 &  0.00\\
instance n=20 138.alb & 1 & 0 & Optimal &  0.16 & 5 &  5.00 &  0.00\\
instance n=20 139.alb & 1 & 0 & Optimal &  0.11 & 5 &  5.00 &  0.00\\
instance n=20 14.alb & 1 & 0 & Optimal &  0.03 & 3 &  3.00 &  0.00\\
instance n=20 140.alb & 1 & 0 & Optimal &  0.09 & 5 &  5.00 &  0.00\\
instance n=20 141.alb & 1 & 0 & Optimal &  0.07 & 3 &  3.00 &  0.00\\
instance n=20 142.alb & 1 & 0 & Optimal &  0.09 & 3 &  3.00 &  0.00\\
instance n=20 143.alb & 1 & 0 & Optimal &  0.11 & 3 &  3.00 &  0.00\\
instance n=20 144.alb & 1 & 0 & Optimal &  0.09 & 4 &  4.00 &  0.00\\
instance n=20 145.alb & 1 & 0 & Optimal &  0.12 & 3 &  3.00 &  0.00\\
instance n=20 146.alb & 1 & 0 & Optimal &  0.08 & 3 &  3.00 &  0.00\\
instance n=20 147.alb & 1 & 0 & Optimal &  0.10 & 3 &  3.00 &  0.00\\
instance n=20 148.alb & 1 & 0 & Optimal &  0.08 & 3 &  3.00 &  0.00\\
instance n=20 149.alb & 1 & 0 & Optimal &  0.11 & 3 &  3.00 &  0.00\\
instance n=20 15.alb & 1 & 0 & Optimal &  0.03 & 3 &  3.00 &  0.00\\
instance n=20 150.alb & 1 & 0 & Optimal &  0.08 & 3 &  3.00 &  0.00\\
instance n=20 151.alb & 1 & 0 & Optimal &  0.08 & 3 &  3.00 &  0.00\\
instance n=20 152.alb & 1 & 0 & Optimal &  0.10 & 3 &  3.00 &  0.00\\
instance n=20 153.alb & 1 & 0 & Optimal &  0.11 & 3 &  3.00 &  0.00\\
instance n=20 154.alb & 1 & 0 & Optimal &  0.09 & 3 &  3.00 &  0.00\\
instance n=20 155.alb & 1 & 0 & Optimal &  0.08 & 3 &  3.00 &  0.00\\
instance n=20 156.alb & 1 & 0 & Optimal &  0.10 & 3 &  3.00 &  0.00\\
instance n=20 157.alb & 1 & 0 & Optimal &  0.10 & 3 &  3.00 &  0.00\\
instance n=20 158.alb & 1 & 0 & Optimal &  0.08 & 3 &  3.00 &  0.00\\
instance n=20 159.alb & 1 & 0 & Optimal &  0.09 & 3 &  3.00 &  0.00\\
instance n=20 16.alb & 1 & 0 & Optimal &  0.36 & 12 & 12.00 &  0.00\\
instance n=20 160.alb & 1 & 0 & Optimal &  0.09 & 3 &  3.00 &  0.00\\
instance n=20 161.alb & 1 & 0 & Optimal &  0.07 & 3 &  3.00 &  0.00\\
instance n=20 162.alb & 1 & 0 & Optimal &  0.10 & 3 &  3.00 &  0.00\\
instance n=20 163.alb & 1 & 0 & Optimal &  0.09 & 3 &  3.00 &  0.00\\
instance n=20 164.alb & 1 & 0 & Optimal &  0.08 & 4 &  4.00 &  0.00\\
instance n=20 165.alb & 1 & 0 & Optimal &  0.09 & 3 &  3.00 &  0.00\\
instance n=20 166.alb & 1 & 0 & Optimal &  5.90 & 12 & 12.00 &  0.00\\
instance n=20 167.alb & 1 & 0 & Optimal &  2.21 & 11 & 11.00 &  0.00\\
instance n=20 168.alb & 1 & 0 & Optimal &  0.36 & 10 & 10.00 &  0.00\\
instance n=20 169.alb & 1 & 0 & Optimal &  0.91 & 11 & 11.00 &  0.00\\
instance n=20 17.alb & 1 & 0 & Optimal &  0.04 & 10 & 10.00 &  0.00\\
instance n=20 170.alb & 1 & 0 & Optimal &  0.26 & 11 & 11.00 &  0.00\\
instance n=20 171.alb & 1 & 0 & Optimal & 71.31 & 13 & 13.00 &  0.00\\
instance n=20 172.alb & 1 & 0 & Optimal &  0.35 & 11 & 11.00 &  0.00\\
instance n=20 173.alb & 1 & 0 & Optimal &  0.10 & 11 & 11.00 &  0.00\\
instance n=20 174.alb & 1 & 0 & Optimal &  1.84 & 12 & 12.00 &  0.00\\
instance n=20 175.alb & 1 & 0 & Optimal &  0.10 & 10 & 10.00 &  0.00\\
instance n=20 176.alb & 1 & 0 & Optimal &  1.66 & 11 & 11.00 &  0.00\\
instance n=20 177.alb & 1 & 0 & Optimal &  2.55 & 10 & 10.00 &  0.00\\
instance n=20 178.alb & 1 & 0 & Optimal &  0.30 & 11 & 11.00 &  0.00\\
instance n=20 179.alb & 1 & 0 & Optimal &  0.17 & 11 & 11.00 &  0.00\\
instance n=20 18.alb & 1 & 0 & Optimal &  0.35 & 11 & 11.00 &  0.00\\
instance n=20 180.alb & 1 & 0 & Optimal & 20.93 & 13 & 13.00 &  0.00\\
instance n=20 181.alb & 1 & 0 & Optimal &  0.31 & 11 & 11.00 &  0.00\\
instance n=20 182.alb & 1 & 0 & Optimal &  3.31 & 11 & 11.00 &  0.00\\
instance n=20 183.alb & 1 & 0 & Optimal & 15.45 & 13 & 13.00 &  0.00\\
instance n=20 184.alb & 1 & 0 & Optimal &  2.65 & 12 & 12.00 &  0.00\\
instance n=20 185.alb & 1 & 0 & Optimal & 25.00 & 15 & 15.00 &  0.00\\
instance n=20 186.alb & 1 & 0 & Optimal & 17.88 & 14 & 14.00 &  0.00\\
instance n=20 187.alb & 1 & 0 & Optimal &  0.12 & 10 & 10.00 &  0.00\\
instance n=20 188.alb & 1 & 0 & Optimal &  0.73 & 11 & 11.00 &  0.00\\
instance n=20 189.alb & 1 & 0 & Optimal &  3.72 & 13 & 13.00 &  0.00\\
instance n=20 19.alb & 1 & 0 & Optimal &  3.65 & 14 & 14.00 &  0.00\\
instance n=20 190.alb & 1 & 0 & Optimal & 69.35 & 15 & 15.00 &  0.00\\
instance n=20 191.alb & 1 & 0 & Optimal &  0.14 & 4 &  4.00 &  0.00\\
instance n=20 192.alb & 1 & 0 & Optimal &  0.13 & 5 &  5.00 &  0.00\\
instance n=20 193.alb & 1 & 0 & Optimal &  0.08 & 5 &  5.00 &  0.00\\
instance n=20 194.alb & 1 & 0 & Optimal &  0.09 & 6 &  6.00 &  0.00\\
instance n=20 195.alb & 1 & 0 & Optimal &  0.12 & 6 &  6.00 &  0.00\\
instance n=20 196.alb & 1 & 0 & Optimal &  0.14 & 5 &  5.00 &  0.00\\
instance n=20 197.alb & 1 & 0 & Optimal &  0.16 & 4 &  4.00 &  0.00\\
instance n=20 198.alb & 1 & 0 & Optimal &  0.15 & 6 &  6.00 &  0.00\\
instance n=20 199.alb & 1 & 0 & Optimal &  0.15 & 5 &  5.00 &  0.00\\
instance n=20 2.alb & 1 & 0 & Optimal &  0.02 & 3 &  3.00 &  0.00\\
instance n=20 20.alb & 1 & 0 & Optimal &  0.25 & 11 & 11.00 &  0.00\\
instance n=20 200.alb & 1 & 0 & Optimal &  0.13 & 6 &  6.00 &  0.00\\
instance n=20 201.alb & 1 & 0 & Optimal &  0.11 & 6 &  6.00 &  0.00\\
instance n=20 202.alb & 1 & 0 & Optimal &  0.58 & 4 &  4.00 &  0.00\\
instance n=20 203.alb & 1 & 0 & Optimal &  0.14 & 4 &  4.00 &  0.00\\
instance n=20 204.alb & 1 & 0 & Optimal &  0.11 & 5 &  5.00 &  0.00\\
instance n=20 205.alb & 1 & 0 & Optimal &  0.11 & 6 &  6.00 &  0.00\\
instance n=20 206.alb & 1 & 0 & Optimal &  0.11 & 5 &  5.00 &  0.00\\
instance n=20 207.alb & 1 & 0 & Optimal &  0.16 & 6 &  6.00 &  0.00\\
instance n=20 208.alb & 1 & 0 & Optimal &  0.16 & 5 &  5.00 &  0.00\\
instance n=20 209.alb & 1 & 0 & Optimal &  0.13 & 4 &  4.00 &  0.00\\
instance n=20 21.alb & 1 & 0 & Optimal &  1.87 & 14 & 14.00 &  0.00\\
instance n=20 210.alb & 1 & 0 & Optimal &  0.13 & 5 &  5.00 &  0.00\\
instance n=20 211.alb & 1 & 0 & Optimal &  0.13 & 5 &  5.00 &  0.00\\
instance n=20 212.alb & 1 & 0 & Optimal &  0.13 & 5 &  5.00 &  0.00\\
instance n=20 213.alb & 1 & 0 & Optimal &  0.11 & 5 &  5.00 &  0.00\\
instance n=20 214.alb & 1 & 0 & Optimal &  0.10 & 5 &  5.00 &  0.00\\
instance n=20 215.alb & 1 & 0 & Optimal &  0.10 & 5 &  5.00 &  0.00\\
instance n=20 216.alb & 1 & 0 & Optimal &  0.10 & 3 &  3.00 &  0.00\\
instance n=20 217.alb & 1 & 0 & Optimal &  0.20 & 4 &  4.00 &  0.00\\
instance n=20 218.alb & 1 & 0 & Optimal &  0.10 & 3 &  3.00 &  0.00\\
instance n=20 219.alb & 1 & 0 & Optimal &  0.09 & 3 &  3.00 &  0.00\\
instance n=20 22.alb & 1 & 0 & Optimal &  0.53 & 12 & 12.00 &  0.00\\
instance n=20 220.alb & 1 & 0 & Optimal &  0.14 & 3 &  3.00 &  0.00\\
instance n=20 221.alb & 1 & 0 & Optimal &  0.10 & 3 &  3.00 &  0.00\\
instance n=20 222.alb & 1 & 0 & Optimal &  0.08 & 3 &  3.00 &  0.00\\
instance n=20 223.alb & 1 & 0 & Optimal &  0.08 & 3 &  3.00 &  0.00\\
instance n=20 224.alb & 1 & 0 & Optimal &  0.10 & 3 &  3.00 &  0.00\\
instance n=20 225.alb & 1 & 0 & Optimal &  0.10 & 3 &  3.00 &  0.00\\
instance n=20 226.alb & 1 & 0 & Optimal &  0.08 & 3 &  3.00 &  0.00\\
instance n=20 227.alb & 1 & 0 & Optimal &  0.11 & 3 &  3.00 &  0.00\\
instance n=20 228.alb & 1 & 0 & Optimal &  0.06 & 2 &  2.00 &  0.00\\
instance n=20 229.alb & 1 & 0 & Optimal &  0.12 & 3 &  3.00 &  0.00\\
instance n=20 23.alb & 1 & 0 & Optimal & 12.65 & 13 & 13.00 &  0.00\\
instance n=20 230.alb & 1 & 0 & Optimal &  0.10 & 3 &  3.00 &  0.00\\
instance n=20 231.alb & 1 & 0 & Optimal &  0.08 & 3 &  3.00 &  0.00\\
instance n=20 232.alb & 1 & 0 & Optimal &  0.12 & 3 &  3.00 &  0.00\\
instance n=20 233.alb & 1 & 0 & Optimal &  0.10 & 3 &  3.00 &  0.00\\
instance n=20 234.alb & 1 & 0 & Optimal &  0.10 & 3 &  3.00 &  0.00\\
instance n=20 235.alb & 1 & 0 & Optimal &  0.08 & 3 &  3.00 &  0.00\\
instance n=20 236.alb & 1 & 0 & Optimal &  0.08 & 3 &  3.00 &  0.00\\
instance n=20 237.alb & 1 & 0 & Optimal &  0.10 & 3 &  3.00 &  0.00\\
instance n=20 238.alb & 1 & 0 & Optimal &  0.10 & 3 &  3.00 &  0.00\\
instance n=20 239.alb & 1 & 0 & Optimal &  0.12 & 3 &  3.00 &  0.00\\
instance n=20 24.alb & 1 & 0 & Optimal &  0.10 & 11 & 11.00 &  0.00\\
instance n=20 240.alb & 1 & 0 & Optimal &  0.10 & 3 &  3.00 &  0.00\\
instance n=20 241.alb & 1 & 0 & Optimal &  0.86 & 13 & 13.00 &  0.00\\
instance n=20 242.alb & 1 & 0 & Optimal &  0.52 & 12 & 12.00 &  0.00\\
instance n=20 243.alb & 1 & 0 & Optimal &  0.53 & 10 & 10.00 &  0.00\\
instance n=20 244.alb & 1 & 0 & Optimal &  0.47 & 11 & 11.00 &  0.00\\
instance n=20 245.alb & 1 & 0 & Optimal &  0.47 & 13 & 13.00 &  0.00\\
instance n=20 246.alb & 1 & 0 & Optimal &  1.38 & 13 & 13.00 &  0.00\\
instance n=20 247.alb & 1 & 0 & Optimal &  0.33 & 11 & 11.00 &  0.00\\
instance n=20 248.alb & 1 & 0 & Optimal &  0.47 & 11 & 11.00 &  0.00\\
instance n=20 249.alb & 1 & 0 & Optimal &  1.32 & 13 & 13.00 &  0.00\\
instance n=20 25.alb & 1 & 0 & Optimal &  0.19 & 11 & 11.00 &  0.00\\
instance n=20 250.alb & 1 & 0 & Optimal &  0.15 & 10 & 10.00 &  0.00\\
instance n=20 251.alb & 1 & 0 & Optimal &  0.49 & 12 & 12.00 &  0.00\\
instance n=20 252.alb & 1 & 0 & Optimal &  1.19 & 11 & 11.00 &  0.00\\
instance n=20 253.alb & 1 & 0 & Optimal &  1.37 & 13 & 13.00 &  0.00\\
instance n=20 254.alb & 1 & 0 & Optimal &  0.50 & 12 & 12.00 &  0.00\\
instance n=20 255.alb & 1 & 0 & Optimal &  2.09 & 13 & 13.00 &  0.00\\
instance n=20 256.alb & 1 & 0 & Optimal &  0.97 & 14 & 14.00 &  0.00\\
instance n=20 257.alb & 1 & 0 & Optimal &  0.11 & 10 & 10.00 &  0.00\\
instance n=20 258.alb & 1 & 0 & Optimal &  0.88 & 13 & 13.00 &  0.00\\
instance n=20 259.alb & 1 & 0 & Optimal &  0.50 & 13 & 13.00 &  0.00\\
instance n=20 26.alb & 1 & 0 & Optimal &  0.99 & 12 & 12.00 &  0.00\\
instance n=20 260.alb & 1 & 0 & Optimal &  1.71 & 12 & 12.00 &  0.00\\
instance n=20 261.alb & 1 & 0 & Optimal &  0.99 & 12 & 12.00 &  0.00\\
instance n=20 262.alb & 1 & 0 & Optimal &  0.57 & 11 & 11.00 &  0.00\\
instance n=20 263.alb & 1 & 0 & Optimal &  0.93 & 12 & 12.00 &  0.00\\
instance n=20 264.alb & 1 & 0 & Optimal &  0.88 & 12 & 12.00 &  0.00\\
instance n=20 265.alb & 1 & 0 & Optimal &  0.52 & 12 & 12.00 &  0.00\\
instance n=20 266.alb & 1 & 0 & Optimal &  0.66 & 5 &  5.00 &  0.00\\
instance n=20 267.alb & 1 & 0 & Optimal &  0.11 & 6 &  6.00 &  0.00\\
instance n=20 268.alb & 1 & 0 & Optimal &  0.16 & 6 &  6.00 &  0.00\\
instance n=20 269.alb & 1 & 0 & Optimal &  0.71 & 7 &  7.00 &  0.00\\
instance n=20 27.alb & 1 & 0 & Optimal &  3.25 & 13 & 13.00 &  0.00\\
instance n=20 270.alb & 1 & 0 & Optimal &  0.69 & 7 &  7.00 &  0.00\\
instance n=20 271.alb & 1 & 0 & Optimal &  0.66 & 6 &  6.00 &  0.00\\
instance n=20 272.alb & 1 & 0 & Optimal &  0.15 & 5 &  5.00 &  0.00\\
instance n=20 273.alb & 1 & 0 & Optimal &  0.11 & 5 &  5.00 &  0.00\\
instance n=20 274.alb & 1 & 0 & Optimal &  0.69 & 6 &  6.00 &  0.00\\
instance n=20 275.alb & 1 & 0 & Optimal &  0.14 & 5 &  5.00 &  0.00\\
instance n=20 276.alb & 1 & 0 & Optimal &  0.17 & 4 &  4.00 &  0.00\\
instance n=20 277.alb & 1 & 0 & Optimal &  0.13 & 4 &  4.00 &  0.00\\
instance n=20 278.alb & 1 & 0 & Optimal &  0.76 & 6 &  6.00 &  0.00\\
instance n=20 279.alb & 1 & 0 & Optimal &  0.17 & 6 &  6.00 &  0.00\\
instance n=20 28.alb & 1 & 0 & Optimal &  2.05 & 12 & 12.00 &  0.00\\
instance n=20 280.alb & 1 & 0 & Optimal &  0.17 & 5 &  5.00 &  0.00\\
instance n=20 281.alb & 1 & 0 & Optimal &  0.13 & 4 &  4.00 &  0.00\\
instance n=20 282.alb & 1 & 0 & Optimal &  0.21 & 4 &  4.00 &  0.00\\
instance n=20 283.alb & 1 & 0 & Optimal &  0.17 & 5 &  5.00 &  0.00\\
instance n=20 284.alb & 1 & 0 & Optimal &  0.11 & 5 &  5.00 &  0.00\\
instance n=20 285.alb & 1 & 0 & Optimal &  0.14 & 5 &  5.00 &  0.00\\
instance n=20 286.alb & 1 & 0 & Optimal &  0.14 & 5 &  5.00 &  0.00\\
instance n=20 287.alb & 1 & 0 & Optimal &  0.16 & 5 &  5.00 &  0.00\\
instance n=20 288.alb & 1 & 0 & Optimal &  0.16 & 6 &  6.00 &  0.00\\
instance n=20 289.alb & 1 & 0 & Optimal &  0.14 & 5 &  5.00 &  0.00\\
instance n=20 29.alb & 1 & 0 & Optimal &  0.03 & 10 & 10.00 &  0.00\\
instance n=20 290.alb & 1 & 0 & Optimal &  0.17 & 5 &  5.00 &  0.00\\
instance n=20 291.alb & 1 & 0 & Optimal &  0.24 & 3 &  3.00 &  0.00\\
instance n=20 292.alb & 1 & 0 & Optimal &  0.17 & 3 &  3.00 &  0.00\\
instance n=20 293.alb & 1 & 0 & Optimal &  0.14 & 3 &  3.00 &  0.00\\
instance n=20 294.alb & 1 & 0 & Optimal &  0.24 & 3 &  3.00 &  0.00\\
instance n=20 295.alb & 1 & 0 & Optimal &  0.14 & 3 &  3.00 &  0.00\\
instance n=20 296.alb & 1 & 0 & Optimal &  0.13 & 3 &  3.00 &  0.00\\
instance n=20 297.alb & 1 & 0 & Optimal &  0.19 & 3 &  3.00 &  0.00\\
instance n=20 298.alb & 1 & 0 & Optimal &  0.19 & 3 &  3.00 &  0.00\\
instance n=20 299.alb & 1 & 0 & Optimal &  0.16 & 3 &  3.00 &  0.00\\
instance n=20 3.alb & 1 & 0 & Optimal &  0.04 & 3 &  3.00 &  0.00\\
instance n=20 30.alb & 1 & 0 & Optimal & 14.70 & 16 & 16.00 &  0.00\\
instance n=20 300.alb & 1 & 0 & Optimal &  0.16 & 4 &  4.00 &  0.00\\
instance n=20 301.alb & 1 & 0 & Optimal &  0.16 & 3 &  3.00 &  0.00\\
instance n=20 302.alb & 1 & 0 & Optimal &  0.21 & 3 &  3.00 &  0.00\\
instance n=20 303.alb & 1 & 0 & Optimal &  0.19 & 3 &  3.00 &  0.00\\
instance n=20 304.alb & 1 & 0 & Optimal &  0.19 & 3 &  3.00 &  0.00\\
instance n=20 305.alb & 1 & 0 & Optimal &  0.22 & 3 &  3.00 &  0.00\\
instance n=20 306.alb & 1 & 0 & Optimal &  0.13 & 3 &  3.00 &  0.00\\
instance n=20 307.alb & 1 & 0 & Optimal &  0.22 & 3 &  3.00 &  0.00\\
instance n=20 308.alb & 1 & 0 & Optimal &  0.22 & 3 &  3.00 &  0.00\\
instance n=20 309.alb & 1 & 0 & Optimal &  0.14 & 3 &  3.00 &  0.00\\
instance n=20 31.alb & 1 & 0 & Optimal &  0.56 & 12 & 12.00 &  0.00\\
instance n=20 310.alb & 1 & 0 & Optimal &  0.17 & 3 &  3.00 &  0.00\\
instance n=20 311.alb & 1 & 0 & Optimal &  0.16 & 3 &  3.00 &  0.00\\
instance n=20 312.alb & 1 & 0 & Optimal &  0.19 & 4 &  4.00 &  0.00\\
instance n=20 313.alb & 1 & 0 & Optimal &  0.18 & 3 &  3.00 &  0.00\\
instance n=20 314.alb & 1 & 0 & Optimal &  0.16 & 3 &  3.00 &  0.00\\
instance n=20 315.alb & 1 & 0 & Optimal &  0.22 & 3 &  3.00 &  0.00\\
instance n=20 316.alb & 1 & 0 & Optimal &  0.17 & 10 & 10.00 &  0.00\\
instance n=20 317.alb & 1 & 0 & Optimal &  1.81 & 10 & 10.00 &  0.00\\
instance n=20 318.alb & 1 & 0 & Optimal &  0.30 & 10 & 10.00 &  0.00\\
instance n=20 319.alb & 1 & 0 & Optimal & 19.16 & 14 & 14.00 &  0.00\\
instance n=20 32.alb & 1 & 0 & Optimal & 15.54 & 13 & 13.00 &  0.00\\
instance n=20 320.alb & 1 & 0 & Optimal &  2.91 & 12 & 12.00 &  0.00\\
instance n=20 321.alb & 1 & 0 & Solution & 120.04 & 14 & 11.00 & 21.43\\
instance n=20 322.alb & 1 & 0 & Optimal & 21.71 & 12 & 12.00 &  0.00\\
instance n=20 323.alb & 1 & 0 & Optimal & 15.25 & 13 & 13.00 &  0.00\\
instance n=20 324.alb & 1 & 0 & Optimal &  0.57 & 9 &  9.00 &  0.00\\
instance n=20 325.alb & 1 & 0 & Solution & 120.04 & 14 & 12.00 & 14.29\\
instance n=20 326.alb & 1 & 0 & Optimal & 40.65 & 14 & 14.00 &  0.00\\
instance n=20 327.alb & 1 & 0 & Optimal & 42.57 & 13 & 13.00 &  0.00\\
instance n=20 328.alb & 1 & 0 & Optimal & 28.06 & 13 & 13.00 &  0.00\\
instance n=20 329.alb & 1 & 0 & Optimal &  0.33 & 10 & 10.00 &  0.00\\
instance n=20 33.alb & 1 & 0 & Optimal &  0.11 & 11 & 11.00 &  0.00\\
instance n=20 330.alb & 1 & 0 & Optimal & 21.95 & 12 & 12.00 &  0.00\\
instance n=20 331.alb & 1 & 0 & Optimal & 40.32 & 13 & 13.00 &  0.00\\
instance n=20 332.alb & 1 & 0 & Optimal &  6.12 & 13 & 13.00 &  0.00\\
instance n=20 333.alb & 1 & 0 & Optimal &  1.48 & 11 & 11.00 &  0.00\\
instance n=20 334.alb & 1 & 0 & Optimal &  0.21 & 10 & 10.00 &  0.00\\
instance n=20 335.alb & 1 & 0 & Solution & 120.05 & 14 & 11.00 & 21.43\\
instance n=20 336.alb & 1 & 0 & Optimal &  1.06 & 11 & 11.00 &  0.00\\
instance n=20 337.alb & 1 & 0 & Optimal &  0.18 & 10 & 10.00 &  0.00\\
instance n=20 338.alb & 1 & 0 & Optimal & 27.74 & 14 & 14.00 &  0.00\\
instance n=20 339.alb & 1 & 0 & Optimal & 35.97 & 13 & 13.00 &  0.00\\
instance n=20 34.alb & 1 & 0 & Optimal &  1.17 & 12 & 12.00 &  0.00\\
instance n=20 340.alb & 1 & 0 & Optimal &  2.52 & 11 & 11.00 &  0.00\\
instance n=20 341.alb & 1 & 0 & Optimal &  0.17 & 6 &  6.00 &  0.00\\
instance n=20 342.alb & 1 & 0 & Optimal &  0.18 & 6 &  6.00 &  0.00\\
instance n=20 343.alb & 1 & 0 & Optimal &  0.19 & 6 &  6.00 &  0.00\\
instance n=20 344.alb & 1 & 0 & Optimal &  0.17 & 6 &  6.00 &  0.00\\
instance n=20 345.alb & 1 & 0 & Optimal &  0.26 & 4 &  4.00 &  0.00\\
instance n=20 346.alb & 1 & 0 & Optimal &  0.32 & 5 &  5.00 &  0.00\\
instance n=20 347.alb & 1 & 0 & Optimal &  0.27 & 6 &  6.00 &  0.00\\
instance n=20 348.alb & 1 & 0 & Optimal &  0.22 & 5 &  5.00 &  0.00\\
instance n=20 349.alb & 1 & 0 & Optimal &  0.20 & 5 &  5.00 &  0.00\\
instance n=20 35.alb & 1 & 0 & Optimal &  0.43 & 12 & 12.00 &  0.00\\
instance n=20 350.alb & 1 & 0 & Optimal &  0.22 & 5 &  5.00 &  0.00\\
instance n=20 351.alb & 1 & 0 & Optimal &  0.21 & 5 &  5.00 &  0.00\\
instance n=20 352.alb & 1 & 0 & Optimal &  0.24 & 4 &  4.00 &  0.00\\
instance n=20 353.alb & 1 & 0 & Optimal &  0.19 & 6 &  6.00 &  0.00\\
instance n=20 354.alb & 1 & 0 & Optimal &  0.27 & 6 &  6.00 &  0.00\\
instance n=20 355.alb & 1 & 0 & Optimal &  0.17 & 5 &  5.00 &  0.00\\
instance n=20 356.alb & 1 & 0 & Optimal &  0.24 & 5 &  5.00 &  0.00\\
instance n=20 357.alb & 1 & 0 & Optimal &  0.27 & 5 &  5.00 &  0.00\\
instance n=20 358.alb & 1 & 0 & Optimal &  0.19 & 4 &  4.00 &  0.00\\
instance n=20 359.alb & 1 & 0 & Optimal &  0.21 & 4 &  4.00 &  0.00\\
instance n=20 36.alb & 1 & 0 & Optimal &  0.92 & 13 & 13.00 &  0.00\\
instance n=20 360.alb & 1 & 0 & Optimal &  0.27 & 6 &  6.00 &  0.00\\
instance n=20 361.alb & 1 & 0 & Optimal &  0.20 & 5 &  5.00 &  0.00\\
instance n=20 362.alb & 1 & 0 & Optimal &  0.17 & 5 &  5.00 &  0.00\\
instance n=20 363.alb & 1 & 0 & Optimal &  0.27 & 7 &  7.00 &  0.00\\
instance n=20 364.alb & 1 & 0 & Optimal &  0.20 & 4 &  4.00 &  0.00\\
instance n=20 365.alb & 1 & 0 & Optimal &  0.20 & 5 &  5.00 &  0.00\\
instance n=20 366.alb & 1 & 0 & Optimal &  0.15 & 3 &  3.00 &  0.00\\
instance n=20 367.alb & 1 & 0 & Optimal &  0.14 & 3 &  3.00 &  0.00\\
instance n=20 368.alb & 1 & 0 & Optimal &  0.14 & 3 &  3.00 &  0.00\\
instance n=20 369.alb & 1 & 0 & Optimal &  0.16 & 3 &  3.00 &  0.00\\
instance n=20 37.alb & 1 & 0 & Optimal &  0.67 & 12 & 12.00 &  0.00\\
instance n=20 370.alb & 1 & 0 & Optimal &  0.15 & 3 &  3.00 &  0.00\\
instance n=20 371.alb & 1 & 0 & Optimal &  0.16 & 3 &  3.00 &  0.00\\
instance n=20 372.alb & 1 & 0 & Optimal &  0.16 & 3 &  3.00 &  0.00\\
instance n=20 373.alb & 1 & 0 & Optimal &  0.24 & 3 &  3.00 &  0.00\\
instance n=20 374.alb & 1 & 0 & Optimal &  0.16 & 3 &  3.00 &  0.00\\
instance n=20 375.alb & 1 & 0 & Optimal &  0.21 & 3 &  3.00 &  0.00\\
instance n=20 376.alb & 1 & 0 & Optimal &  0.14 & 3 &  3.00 &  0.00\\
instance n=20 377.alb & 1 & 0 & Optimal &  0.17 & 3 &  3.00 &  0.00\\
instance n=20 378.alb & 1 & 0 & Optimal &  0.14 & 3 &  3.00 &  0.00\\
instance n=20 379.alb & 1 & 0 & Optimal &  0.24 & 4 &  4.00 &  0.00\\
instance n=20 38.alb & 1 & 0 & Optimal &  0.19 & 12 & 12.00 &  0.00\\
instance n=20 380.alb & 1 & 0 & Optimal &  0.15 & 3 &  3.00 &  0.00\\
instance n=20 381.alb & 1 & 0 & Optimal &  0.14 & 3 &  3.00 &  0.00\\
instance n=20 382.alb & 1 & 0 & Optimal &  0.24 & 4 &  4.00 &  0.00\\
instance n=20 383.alb & 1 & 0 & Optimal &  0.17 & 3 &  3.00 &  0.00\\
instance n=20 384.alb & 1 & 0 & Optimal &  0.16 & 3 &  3.00 &  0.00\\
instance n=20 385.alb & 1 & 0 & Optimal &  0.14 & 3 &  3.00 &  0.00\\
instance n=20 386.alb & 1 & 0 & Optimal &  0.14 & 3 &  3.00 &  0.00\\
instance n=20 387.alb & 1 & 0 & Optimal &  0.16 & 3 &  3.00 &  0.00\\
instance n=20 388.alb & 1 & 0 & Optimal &  0.18 & 3 &  3.00 &  0.00\\
instance n=20 389.alb & 1 & 0 & Optimal &  0.17 & 3 &  3.00 &  0.00\\
instance n=20 39.alb & 1 & 0 & Optimal &  0.36 & 13 & 13.00 &  0.00\\
instance n=20 390.alb & 1 & 0 & Optimal &  0.16 & 3 &  3.00 &  0.00\\
instance n=20 391.alb & 1 & 0 & Optimal &  0.74 & 11 & 10.00 &  9.09\\
instance n=20 392.alb & 1 & 0 & Optimal &  2.22 & 14 & 14.00 &  0.00\\
instance n=20 393.alb & 1 & 0 & Optimal &  2.23 & 11 & 10.00 &  9.09\\
instance n=20 394.alb & 1 & 0 & Optimal &  1.68 & 12 & 12.00 &  0.00\\
instance n=20 395.alb & 1 & 0 & Optimal &  0.72 & 12 & 12.00 &  0.00\\
instance n=20 396.alb & 1 & 0 & Optimal &  2.73 & 13 & 13.00 &  0.00\\
instance n=20 397.alb & 1 & 0 & Optimal &  0.94 & 10 & 10.00 &  0.00\\
instance n=20 398.alb & 1 & 0 & Optimal &  0.68 & 11 & 11.00 &  0.00\\
instance n=20 399.alb & 1 & 0 & Optimal &  1.61 & 13 & 13.00 &  0.00\\
instance n=20 4.alb & 1 & 0 & Optimal &  0.02 & 3 &  3.00 &  0.00\\
instance n=20 40.alb & 1 & 0 & Optimal &  1.37 & 12 & 12.00 &  0.00\\
instance n=20 400.alb & 1 & 0 & Optimal &  1.55 & 12 & 12.00 &  0.00\\
instance n=20 401.alb & 1 & 0 & Optimal &  1.60 & 12 & 12.00 &  0.00\\
instance n=20 402.alb & 1 & 0 & Optimal &  0.82 & 12 & 12.00 &  0.00\\
instance n=20 403.alb & 1 & 0 & Optimal &  1.63 & 12 & 12.00 &  0.00\\
instance n=20 404.alb & 1 & 0 & Optimal &  1.73 & 10 & 10.00 &  0.00\\
instance n=20 405.alb & 1 & 0 & Optimal &  1.63 & 12 & 12.00 &  0.00\\
instance n=20 406.alb & 1 & 0 & Optimal &  5.09 & 14 & 14.00 &  0.00\\
instance n=20 407.alb & 1 & 0 & Optimal &  0.24 & 10 & 10.00 &  0.00\\
instance n=20 408.alb & 1 & 0 & Optimal &  3.75 & 14 & 14.00 &  0.00\\
instance n=20 409.alb & 1 & 0 & Optimal &  1.19 & 12 & 12.00 &  0.00\\
instance n=20 41.alb & 1 & 0 & Optimal &  0.01 & 6 &  6.00 &  0.00\\
instance n=20 410.alb & 1 & 0 & Optimal &  1.26 & 11 & 11.00 &  0.00\\
instance n=20 411.alb & 1 & 0 & Optimal &  7.00 & 15 & 15.00 &  0.00\\
instance n=20 412.alb & 1 & 0 & Optimal &  0.99 & 11 & 11.00 &  0.00\\
instance n=20 413.alb & 1 & 0 & Optimal &  0.25 & 10 & 10.00 &  0.00\\
instance n=20 414.alb & 1 & 0 & Optimal &  3.08 & 12 & 12.00 &  0.00\\
instance n=20 415.alb & 1 & 0 & Optimal &  0.19 & 10 & 10.00 &  0.00\\
instance n=20 416.alb & 1 & 0 & Optimal &  0.24 & 6 &  6.00 &  0.00\\
instance n=20 417.alb & 1 & 0 & Optimal &  0.24 & 5 &  5.00 &  0.00\\
instance n=20 418.alb & 1 & 0 & Optimal &  0.19 & 6 &  6.00 &  0.00\\
instance n=20 419.alb & 1 & 0 & Optimal &  0.19 & 4 &  4.00 &  0.00\\
instance n=20 42.alb & 1 & 0 & Optimal &  0.03 & 5 &  5.00 &  0.00\\
instance n=20 420.alb & 1 & 0 & Optimal &  0.27 & 5 &  5.00 &  0.00\\
instance n=20 421.alb & 1 & 0 & Optimal &  0.25 & 6 &  6.00 &  0.00\\
instance n=20 422.alb & 1 & 0 & Optimal &  0.17 & 4 &  4.00 &  0.00\\
instance n=20 423.alb & 1 & 0 & Optimal &  0.20 & 6 &  6.00 &  0.00\\
instance n=20 424.alb & 1 & 0 & Optimal &  0.31 & 5 &  5.00 &  0.00\\
instance n=20 425.alb & 1 & 0 & Optimal &  0.27 & 6 &  6.00 &  0.00\\
instance n=20 426.alb & 1 & 0 & Optimal &  0.24 & 5 &  5.00 &  0.00\\
instance n=20 427.alb & 1 & 0 & Optimal &  0.20 & 6 &  6.00 &  0.00\\
instance n=20 428.alb & 1 & 0 & Optimal &  0.22 & 5 &  5.00 &  0.00\\
instance n=20 429.alb & 1 & 0 & Optimal &  0.22 & 4 &  4.00 &  0.00\\
instance n=20 43.alb & 1 & 0 & Optimal &  0.03 & 5 &  5.00 &  0.00\\
instance n=20 430.alb & 1 & 0 & Optimal &  0.28 & 5 &  5.00 &  0.00\\
instance n=20 431.alb & 1 & 0 & Optimal &  0.28 & 6 &  6.00 &  0.00\\
instance n=20 432.alb & 1 & 0 & Optimal &  0.17 & 5 &  5.00 &  0.00\\
instance n=20 433.alb & 1 & 0 & Optimal &  0.22 & 5 &  5.00 &  0.00\\
instance n=20 434.alb & 1 & 0 & Optimal &  0.21 & 5 &  5.00 &  0.00\\
instance n=20 435.alb & 1 & 0 & Optimal &  1.18 & 7 &  7.00 &  0.00\\
instance n=20 436.alb & 1 & 0 & Optimal &  0.20 & 5 &  5.00 &  0.00\\
instance n=20 437.alb & 1 & 0 & Optimal &  0.27 & 5 &  5.00 &  0.00\\
instance n=20 438.alb & 1 & 0 & Optimal &  0.22 & 6 &  6.00 &  0.00\\
instance n=20 439.alb & 1 & 0 & Optimal &  0.20 & 5 &  5.00 &  0.00\\
instance n=20 44.alb & 1 & 0 & Optimal &  0.04 & 5 &  5.00 &  0.00\\
instance n=20 440.alb & 1 & 0 & Optimal &  0.22 & 5 &  5.00 &  0.00\\
instance n=20 441.alb & 1 & 0 & Optimal &  0.17 & 3 &  3.00 &  0.00\\
instance n=20 442.alb & 1 & 0 & Optimal &  0.23 & 3 &  3.00 &  0.00\\
instance n=20 443.alb & 1 & 0 & Optimal &  0.20 & 3 &  3.00 &  0.00\\
instance n=20 444.alb & 1 & 0 & Optimal &  0.24 & 3 &  3.00 &  0.00\\
instance n=20 445.alb & 1 & 0 & Optimal &  0.18 & 3 &  3.00 &  0.00\\
instance n=20 446.alb & 1 & 0 & Optimal &  0.17 & 3 &  3.00 &  0.00\\
instance n=20 447.alb & 1 & 0 & Optimal &  0.17 & 3 &  3.00 &  0.00\\
instance n=20 448.alb & 1 & 0 & Optimal &  0.17 & 3 &  3.00 &  0.00\\
instance n=20 449.alb & 1 & 0 & Optimal &  0.25 & 3 &  3.00 &  0.00\\
instance n=20 45.alb & 1 & 0 & Optimal &  0.03 & 6 &  6.00 &  0.00\\
instance n=20 450.alb & 1 & 0 & Optimal &  0.24 & 3 &  3.00 &  0.00\\
instance n=20 451.alb & 1 & 0 & Optimal &  0.22 & 3 &  3.00 &  0.00\\
instance n=20 452.alb & 1 & 0 & Optimal &  0.20 & 3 &  3.00 &  0.00\\
instance n=20 453.alb & 1 & 0 & Optimal &  0.23 & 3 &  3.00 &  0.00\\
instance n=20 454.alb & 1 & 0 & Optimal &  0.20 & 3 &  3.00 &  0.00\\
instance n=20 455.alb & 1 & 0 & Optimal &  0.17 & 3 &  3.00 &  0.00\\
instance n=20 456.alb & 1 & 0 & Optimal &  0.17 & 4 &  4.00 &  0.00\\
instance n=20 457.alb & 1 & 0 & Optimal &  0.22 & 3 &  3.00 &  0.00\\
instance n=20 458.alb & 1 & 0 & Optimal &  0.17 & 3 &  3.00 &  0.00\\
instance n=20 459.alb & 1 & 0 & Optimal &  0.19 & 3 &  3.00 &  0.00\\
instance n=20 46.alb & 1 & 0 & Optimal &  0.02 & 4 &  4.00 &  0.00\\
instance n=20 460.alb & 1 & 0 & Optimal &  0.25 & 3 &  3.00 &  0.00\\
instance n=20 461.alb & 1 & 0 & Optimal &  0.22 & 3 &  3.00 &  0.00\\
instance n=20 462.alb & 1 & 0 & Optimal &  0.22 & 3 &  3.00 &  0.00\\
instance n=20 463.alb & 1 & 0 & Optimal &  0.19 & 3 &  3.00 &  0.00\\
instance n=20 464.alb & 1 & 0 & Optimal &  0.19 & 3 &  3.00 &  0.00\\
instance n=20 465.alb & 1 & 0 & Optimal &  0.16 & 3 &  3.00 &  0.00\\
instance n=20 466.alb & 1 & 0 & Optimal &  1.11 & 13 & 13.00 &  0.00\\
instance n=20 467.alb & 1 & 0 & Optimal &  1.08 & 14 & 14.00 &  0.00\\
instance n=20 468.alb & 1 & 0 & Optimal &  1.15 & 13 & 13.00 &  0.00\\
instance n=20 469.alb & 1 & 0 & Optimal &  0.86 & 14 & 14.00 &  0.00\\
instance n=20 47.alb & 1 & 0 & Optimal &  0.04 & 4 &  4.00 &  0.00\\
instance n=20 470.alb & 1 & 0 & Optimal &  1.19 & 12 & 12.00 &  0.00\\
instance n=20 471.alb & 1 & 0 & Optimal &  1.01 & 12 & 12.00 &  0.00\\
instance n=20 472.alb & 1 & 0 & Optimal &  0.97 & 13 & 13.00 &  0.00\\
instance n=20 473.alb & 1 & 0 & Optimal &  1.08 & 10 & 10.00 &  0.00\\
instance n=20 474.alb & 1 & 0 & Optimal &  1.07 & 14 & 14.00 &  0.00\\
instance n=20 475.alb & 1 & 0 & Optimal &  1.13 & 11 & 11.00 &  0.00\\
instance n=20 476.alb & 1 & 0 & Optimal &  1.00 & 11 & 11.00 &  0.00\\
instance n=20 477.alb & 1 & 0 & Optimal &  1.13 & 11 & 11.00 &  0.00\\
instance n=20 478.alb & 1 & 0 & Optimal &  1.29 & 12 & 12.00 &  0.00\\
instance n=20 479.alb & 1 & 0 & Optimal &  1.15 & 13 & 13.00 &  0.00\\
instance n=20 48.alb & 1 & 0 & Optimal &  0.05 & 5 &  5.00 &  0.00\\
instance n=20 480.alb & 1 & 0 & Optimal &  1.15 & 13 & 13.00 &  0.00\\
instance n=20 481.alb & 1 & 0 & Optimal &  0.94 & 13 & 13.00 &  0.00\\
instance n=20 482.alb & 1 & 0 & Optimal &  1.30 & 13 & 13.00 &  0.00\\
instance n=20 483.alb & 1 & 0 & Optimal &  1.07 & 12 & 12.00 &  0.00\\
instance n=20 484.alb & 1 & 0 & Optimal &  1.15 & 13 & 13.00 &  0.00\\
instance n=20 485.alb & 1 & 0 & Optimal &  0.99 & 15 & 15.00 &  0.00\\
instance n=20 486.alb & 1 & 0 & Optimal &  1.14 & 11 & 11.00 &  0.00\\
instance n=20 487.alb & 1 & 0 & Optimal &  1.23 & 12 & 12.00 &  0.00\\
instance n=20 488.alb & 1 & 0 & Optimal &  1.10 & 15 & 15.00 &  0.00\\
instance n=20 489.alb & 1 & 0 & Optimal &  1.04 & 12 & 12.00 &  0.00\\
instance n=20 49.alb & 1 & 0 & Optimal &  0.03 & 4 &  4.00 &  0.00\\
instance n=20 490.alb & 1 & 0 & Optimal &  1.22 & 12 & 12.00 &  0.00\\
instance n=20 491.alb & 1 & 0 & Optimal &  0.23 & 6 &  6.00 &  0.00\\
instance n=20 492.alb & 1 & 0 & Optimal &  0.24 & 5 &  5.00 &  0.00\\
instance n=20 493.alb & 1 & 0 & Optimal &  0.27 & 5 &  5.00 &  0.00\\
instance n=20 494.alb & 1 & 0 & Optimal &  0.22 & 6 &  6.00 &  0.00\\
instance n=20 495.alb & 1 & 0 & Optimal &  0.27 & 6 &  6.00 &  0.00\\
instance n=20 496.alb & 1 & 0 & Optimal &  0.27 & 5 &  5.00 &  0.00\\
instance n=20 497.alb & 1 & 0 & Optimal &  0.25 & 6 &  6.00 &  0.00\\
instance n=20 498.alb & 1 & 0 & Optimal &  0.39 & 6 &  6.00 &  0.00\\
instance n=20 499.alb & 1 & 0 & Optimal &  0.25 & 5 &  5.00 &  0.00\\
instance n=20 5.alb & 1 & 0 & Optimal &  0.03 & 3 &  3.00 &  0.00\\
instance n=20 50.alb & 1 & 0 & Optimal &  0.05 & 4 &  4.00 &  0.00\\
instance n=20 500.alb & 1 & 0 & Optimal &  1.33 & 8 &  8.00 &  0.00\\
instance n=20 501.alb & 1 & 0 & Optimal &  0.28 & 5 &  5.00 &  0.00\\
instance n=20 502.alb & 1 & 0 & Optimal &  0.19 & 4 &  4.00 &  0.00\\
instance n=20 503.alb & 1 & 0 & Optimal &  0.24 & 6 &  6.00 &  0.00\\
instance n=20 504.alb & 1 & 0 & Optimal &  0.28 & 6 &  6.00 &  0.00\\
instance n=20 505.alb & 1 & 0 & Optimal &  0.36 & 6 &  6.00 &  0.00\\
instance n=20 506.alb & 1 & 0 & Optimal &  0.31 & 5 &  5.00 &  0.00\\
instance n=20 507.alb & 1 & 0 & Optimal &  0.24 & 5 &  5.00 &  0.00\\
instance n=20 508.alb & 1 & 0 & Optimal &  0.28 & 5 &  5.00 &  0.00\\
instance n=20 509.alb & 1 & 0 & Optimal &  0.20 & 4 &  4.00 &  0.00\\
instance n=20 51.alb & 1 & 0 & Optimal &  0.03 & 4 &  4.00 &  0.00\\
instance n=20 510.alb & 1 & 0 & Optimal &  0.25 & 5 &  5.00 &  0.00\\
instance n=20 511.alb & 1 & 0 & Optimal &  0.44 & 5 &  5.00 &  0.00\\
instance n=20 512.alb & 1 & 0 & Optimal &  0.31 & 5 &  5.00 &  0.00\\
instance n=20 513.alb & 1 & 0 & Optimal &  0.27 & 5 &  5.00 &  0.00\\
instance n=20 514.alb & 1 & 0 & Optimal &  0.22 & 5 &  5.00 &  0.00\\
instance n=20 515.alb & 1 & 0 & Optimal &  1.67 & 6 &  6.00 &  0.00\\
instance n=20 516.alb & 1 & 0 & Optimal &  0.36 & 3 &  3.00 &  0.00\\
instance n=20 517.alb & 1 & 0 & Optimal &  0.31 & 3 &  3.00 &  0.00\\
instance n=20 518.alb & 1 & 0 & Optimal &  0.35 & 3 &  3.00 &  0.00\\
instance n=20 519.alb & 1 & 0 & Optimal &  0.35 & 3 &  3.00 &  0.00\\
instance n=20 52.alb & 1 & 0 & Optimal &  0.05 & 4 &  4.00 &  0.00\\
instance n=20 520.alb & 1 & 0 & Optimal &  0.39 & 3 &  3.00 &  0.00\\
instance n=20 521.alb & 1 & 0 & Optimal &  0.36 & 3 &  3.00 &  0.00\\
instance n=20 522.alb & 1 & 0 & Optimal &  0.30 & 3 &  3.00 &  0.00\\
instance n=20 523.alb & 1 & 0 & Optimal &  0.28 & 3 &  3.00 &  0.00\\
instance n=20 524.alb & 1 & 0 & Optimal &  0.33 & 3 &  3.00 &  0.00\\
instance n=20 525.alb & 1 & 0 & Optimal &  0.35 & 3 &  3.00 &  0.00\\
instance n=20 53.alb & 1 & 0 & Optimal &  0.03 & 5 &  5.00 &  0.00\\
instance n=20 54.alb & 1 & 0 & Optimal &  0.03 & 5 &  5.00 &  0.00\\
instance n=20 55.alb & 1 & 0 & Optimal &  0.04 & 5 &  5.00 &  0.00\\
instance n=20 56.alb & 1 & 0 & Optimal &  0.03 & 4 &  4.00 &  0.00\\
instance n=20 57.alb & 1 & 0 & Optimal &  0.03 & 4 &  4.00 &  0.00\\
instance n=20 58.alb & 1 & 0 & Optimal &  0.03 & 5 &  5.00 &  0.00\\
instance n=20 59.alb & 1 & 0 & Optimal &  0.04 & 4 &  4.00 &  0.00\\
instance n=20 6.alb & 1 & 0 & Optimal &  0.02 & 3 &  3.00 &  0.00\\
instance n=20 60.alb & 1 & 0 & Optimal &  0.05 & 6 &  6.00 &  0.00\\
instance n=20 61.alb & 1 & 0 & Optimal &  0.04 & 7 &  7.00 &  0.00\\
instance n=20 62.alb & 1 & 0 & Optimal &  0.03 & 5 &  5.00 &  0.00\\
instance n=20 63.alb & 1 & 0 & Optimal &  0.05 & 5 &  5.00 &  0.00\\
instance n=20 64.alb & 1 & 0 & Optimal &  0.04 & 5 &  5.00 &  0.00\\
instance n=20 65.alb & 1 & 0 & Optimal &  0.04 & 5 &  5.00 &  0.00\\
instance n=20 66.alb & 1 & 0 & Optimal &  0.02 & 3 &  3.00 &  0.00\\
instance n=20 67.alb & 1 & 0 & Optimal &  0.03 & 3 &  3.00 &  0.00\\
instance n=20 68.alb & 1 & 0 & Optimal &  0.05 & 3 &  3.00 &  0.00\\
instance n=20 69.alb & 1 & 0 & Optimal &  0.01 & 2 &  2.00 &  0.00\\
instance n=20 7.alb & 1 & 0 & Optimal &  0.02 & 3 &  3.00 &  0.00\\
instance n=20 70.alb & 1 & 0 & Optimal &  0.06 & 3 &  3.00 &  0.00\\
instance n=20 71.alb & 1 & 0 & Optimal &  0.03 & 3 &  3.00 &  0.00\\
instance n=20 72.alb & 1 & 0 & Optimal &  0.03 & 3 &  3.00 &  0.00\\
instance n=20 73.alb & 1 & 0 & Optimal &  0.01 & 2 &  2.00 &  0.00\\
instance n=20 74.alb & 1 & 0 & Optimal &  0.05 & 3 &  3.00 &  0.00\\
instance n=20 75.alb & 1 & 0 & Optimal &  0.03 & 3 &  3.00 &  0.00\\
instance n=20 76.alb & 1 & 0 & Optimal &  0.05 & 3 &  3.00 &  0.00\\
instance n=20 77.alb & 1 & 0 & Optimal &  0.05 & 3 &  3.00 &  0.00\\
instance n=20 78.alb & 1 & 0 & Optimal &  0.03 & 3 &  3.00 &  0.00\\
instance n=20 79.alb & 1 & 0 & Optimal &  0.03 & 3 &  3.00 &  0.00\\
instance n=20 8.alb & 1 & 0 & Optimal &  0.03 & 3 &  3.00 &  0.00\\
instance n=20 80.alb & 1 & 0 & Optimal &  0.05 & 3 &  3.00 &  0.00\\
instance n=20 81.alb & 1 & 0 & Optimal &  0.05 & 3 &  3.00 &  0.00\\
instance n=20 82.alb & 1 & 0 & Optimal &  0.05 & 4 &  4.00 &  0.00\\
instance n=20 83.alb & 1 & 0 & Optimal &  0.03 & 3 &  3.00 &  0.00\\
instance n=20 84.alb & 1 & 0 & Optimal &  0.05 & 3 &  3.00 &  0.00\\
instance n=20 85.alb & 1 & 0 & Optimal &  0.05 & 3 &  3.00 &  0.00\\
instance n=20 86.alb & 1 & 0 & Optimal &  0.04 & 3 &  3.00 &  0.00\\
instance n=20 87.alb & 1 & 0 & Optimal &  0.04 & 3 &  3.00 &  0.00\\
instance n=20 88.alb & 1 & 0 & Optimal &  0.05 & 3 &  3.00 &  0.00\\
instance n=20 89.alb & 1 & 0 & Optimal &  0.06 & 3 &  3.00 &  0.00\\
instance n=20 9.alb & 1 & 0 & Optimal &  0.02 & 3 &  3.00 &  0.00\\
instance n=20 90.alb & 1 & 0 & Optimal &  0.05 & 3 &  3.00 &  0.00\\
instance n=20 91.alb & 1 & 0 & Optimal &  0.20 & 11 & 11.00 &  0.00\\
instance n=20 92.alb & 1 & 0 & Optimal &  0.26 & 11 & 11.00 &  0.00\\
instance n=20 93.alb & 1 & 0 & Optimal &  0.36 & 13 & 13.00 &  0.00\\
instance n=20 94.alb & 1 & 0 & Optimal &  0.06 & 10 & 10.00 &  0.00\\
instance n=20 95.alb & 1 & 0 & Optimal &  0.23 & 12 & 12.00 &  0.00\\
instance n=20 96.alb & 1 & 0 & Optimal &  0.22 & 10 & 10.00 &  0.00\\
instance n=20 97.alb & 1 & 0 & Optimal &  2.52 & 15 & 15.00 &  0.00\\
instance n=20 98.alb & 1 & 0 & Optimal &  0.54 & 13 & 13.00 &  0.00\\
instance n=20 99.alb & 1 & 0 & Optimal &  0.59 & 12 & 12.00 &  0.00\\
instance n=50 1.alb & 1 & 0 & Optimal &  0.02 & 8 &  8.00 &  0.00\\
instance n=50 10.alb & 1 & 0 & Optimal &  0.03 & 7 &  7.00 &  0.00\\
instance n=50 100.alb & 1 & 0 & Optimal &  0.07 & 7 &  7.00 &  0.00\\
instance n=50 101.alb & 1 & 0 & Solution & 120.02 & 30 & 27.00 & 10.00\\
instance n=50 102.alb & 1 & 0 & Solution & 120.02 & 32 & 28.00 & 12.50\\
instance n=50 103.alb & 1 & 0 & Solution & 120.04 & 29 & 26.00 & 10.34\\
instance n=50 104.alb & 1 & 0 & Solution & 120.03 & 27 & 25.00 &  7.41\\
instance n=50 105.alb & 1 & 0 & Optimal & 93.42 & 24 & 24.00 &  0.00\\
instance n=50 106.alb & 1 & 0 & Solution & 120.04 & 28 & 26.00 &  7.14\\
instance n=50 107.alb & 1 & 0 & Solution & 120.02 & 28 & 27.00 &  3.57\\
instance n=50 108.alb & 1 & 0 & Solution & 120.02 & 30 & 27.00 & 10.00\\
instance n=50 109.alb & 1 & 0 & Solution & 120.04 & 30 & 26.00 & 13.33\\
instance n=50 11.alb & 1 & 0 & Optimal &  0.02 & 7 &  7.00 &  0.00\\
instance n=50 110.alb & 1 & 0 & Solution & 120.02 & 26 & 25.00 &  3.85\\
instance n=50 111.alb & 1 & 0 & Solution & 120.01 & 28 & 26.00 &  7.14\\
instance n=50 112.alb & 1 & 0 & Solution & 120.03 & 27 & 25.00 &  7.41\\
instance n=50 113.alb & 1 & 0 & Solution & 120.03 & 28 & 26.00 &  7.14\\
instance n=50 114.alb & 1 & 0 & Solution & 120.03 & 27 & 25.00 &  7.41\\
instance n=50 115.alb & 1 & 0 & Solution & 120.03 & 28 & 26.00 &  7.14\\
instance n=50 116.alb & 1 & 0 & Solution & 120.03 & 32 & 29.00 &  9.38\\
instance n=50 117.alb & 1 & 0 & Solution & 120.01 & 27 & 25.00 &  7.41\\
instance n=50 118.alb & 1 & 0 & Solution & 120.03 & 29 & 26.00 & 10.34\\
instance n=50 119.alb & 1 & 0 & Optimal &  5.79 & 25 & 25.00 &  0.00\\
instance n=50 12.alb & 1 & 0 & Optimal &  0.05 & 6 &  6.00 &  0.00\\
instance n=50 120.alb & 1 & 0 & Solution & 120.01 & 27 & 26.00 &  3.70\\
instance n=50 121.alb & 1 & 0 & Solution & 120.02 & 32 & 27.00 & 15.63\\
instance n=50 122.alb & 1 & 0 & Solution & 120.02 & 29 & 28.00 &  3.45\\
instance n=50 123.alb & 1 & 0 & Solution & 120.02 & 32 & 27.00 & 15.63\\
instance n=50 124.alb & 1 & 0 & Solution & 120.02 & 29 & 27.00 &  6.90\\
instance n=50 125.alb & 1 & 0 & Solution & 120.02 & 33 & 28.00 & 15.15\\
instance n=50 126.alb & 1 & 0 & Optimal &  0.10 & 12 & 12.00 &  0.00\\
instance n=50 127.alb & 1 & 0 & Optimal &  0.09 & 14 & 14.00 &  0.00\\
instance n=50 128.alb & 1 & 0 & Optimal &  0.42 & 12 & 12.00 &  0.00\\
instance n=50 129.alb & 1 & 0 & Optimal &  0.11 & 13 & 13.00 &  0.00\\
instance n=50 13.alb & 1 & 0 & Optimal &  0.02 & 6 &  6.00 &  0.00\\
instance n=50 130.alb & 1 & 0 & Optimal &  0.11 & 13 & 13.00 &  0.00\\
instance n=50 131.alb & 1 & 0 & Optimal &  0.11 & 12 & 12.00 &  0.00\\
instance n=50 132.alb & 1 & 0 & Optimal &  1.44 & 12 & 12.00 &  0.00\\
instance n=50 133.alb & 1 & 0 & Optimal &  0.07 & 12 & 12.00 &  0.00\\
instance n=50 134.alb & 1 & 0 & Optimal &  1.10 & 14 & 14.00 &  0.00\\
instance n=50 135.alb & 1 & 0 & Optimal &  0.47 & 13 & 13.00 &  0.00\\
instance n=50 136.alb & 1 & 0 & Optimal &  0.11 & 11 & 11.00 &  0.00\\
instance n=50 137.alb & 1 & 0 & Optimal &  0.11 & 11 & 11.00 &  0.00\\
instance n=50 138.alb & 1 & 0 & Optimal &  0.10 & 12 & 12.00 &  0.00\\
instance n=50 139.alb & 1 & 0 & Optimal &  3.67 & 11 & 11.00 &  0.00\\
instance n=50 14.alb & 1 & 0 & Optimal &  0.03 & 7 &  7.00 &  0.00\\
instance n=50 140.alb & 1 & 0 & Optimal &  0.20 & 12 & 12.00 &  0.00\\
instance n=50 141.alb & 1 & 0 & Optimal &  0.17 & 13 & 13.00 &  0.00\\
instance n=50 142.alb & 1 & 0 & Optimal &  0.11 & 11 & 11.00 &  0.00\\
instance n=50 143.alb & 1 & 0 & Optimal &  0.28 & 12 & 12.00 &  0.00\\
instance n=50 144.alb & 1 & 0 & Optimal &  0.23 & 13 & 13.00 &  0.00\\
instance n=50 145.alb & 1 & 0 & Optimal &  0.25 & 10 & 10.00 &  0.00\\
instance n=50 146.alb & 1 & 0 & Optimal &  0.20 & 13 & 13.00 &  0.00\\
instance n=50 147.alb & 1 & 0 & Optimal &  0.29 & 13 & 13.00 &  0.00\\
instance n=50 148.alb & 1 & 0 & Optimal &  0.17 & 10 & 10.00 &  0.00\\
instance n=50 149.alb & 1 & 0 & Optimal &  0.18 & 12 & 12.00 &  0.00\\
instance n=50 15.alb & 1 & 0 & Optimal &  0.02 & 8 &  8.00 &  0.00\\
instance n=50 150.alb & 1 & 0 & Optimal &  0.14 & 11 & 11.00 &  0.00\\
instance n=50 151.alb & 1 & 0 & Optimal &  0.14 & 7 &  7.00 &  0.00\\
instance n=50 152.alb & 1 & 0 & Optimal &  0.10 & 7 &  7.00 &  0.00\\
instance n=50 153.alb & 1 & 0 & Optimal &  0.56 & 7 &  7.00 &  0.00\\
instance n=50 154.alb & 1 & 0 & Optimal &  0.12 & 8 &  8.00 &  0.00\\
instance n=50 155.alb & 1 & 0 & Optimal &  0.09 & 7 &  7.00 &  0.00\\
instance n=50 156.alb & 1 & 0 & Optimal &  0.09 & 7 &  7.00 &  0.00\\
instance n=50 157.alb & 1 & 0 & Optimal &  0.10 & 8 &  8.00 &  0.00\\
instance n=50 158.alb & 1 & 0 & Optimal &  0.09 & 7 &  7.00 &  0.00\\
instance n=50 159.alb & 1 & 0 & Optimal &  0.12 & 7 &  7.00 &  0.00\\
instance n=50 16.alb & 1 & 0 & Optimal &  0.04 & 8 &  8.00 &  0.00\\
instance n=50 160.alb & 1 & 0 & Optimal &  0.11 & 8 &  8.00 &  0.00\\
instance n=50 161.alb & 1 & 0 & Optimal &  0.13 & 7 &  7.00 &  0.00\\
instance n=50 162.alb & 1 & 0 & Optimal &  0.11 & 8 &  8.00 &  0.00\\
instance n=50 163.alb & 1 & 0 & Optimal &  0.11 & 7 &  7.00 &  0.00\\
instance n=50 164.alb & 1 & 0 & Optimal &  0.12 & 7 &  7.00 &  0.00\\
instance n=50 165.alb & 1 & 0 & Optimal &  0.13 & 8 &  8.00 &  0.00\\
instance n=50 166.alb & 1 & 0 & Optimal &  0.12 & 8 &  8.00 &  0.00\\
instance n=50 167.alb & 1 & 0 & Optimal &  0.60 & 7 &  7.00 &  0.00\\
instance n=50 168.alb & 1 & 0 & Optimal &  0.69 & 8 &  8.00 &  0.00\\
instance n=50 169.alb & 1 & 0 & Optimal &  0.10 & 8 &  8.00 &  0.00\\
instance n=50 17.alb & 1 & 0 & Optimal &  0.03 & 7 &  7.00 &  0.00\\
instance n=50 170.alb & 1 & 0 & Optimal &  0.38 & 7 &  7.00 &  0.00\\
instance n=50 171.alb & 1 & 0 & Optimal &  0.12 & 8 &  8.00 &  0.00\\
instance n=50 172.alb & 1 & 0 & Optimal &  0.12 & 7 &  7.00 &  0.00\\
instance n=50 173.alb & 1 & 0 & Optimal &  0.36 & 7 &  7.00 &  0.00\\
instance n=50 174.alb & 1 & 0 & Optimal &  0.13 & 7 &  7.00 &  0.00\\
instance n=50 175.alb & 1 & 0 & Optimal &  0.09 & 7 &  7.00 &  0.00\\
instance n=50 176.alb & 1 & 0 & Solution & 120.03 & 27 & 25.00 &  7.41\\
instance n=50 177.alb & 1 & 0 & Solution & 120.03 & 28 & 26.00 &  7.14\\
instance n=50 178.alb & 1 & 0 & Solution & 120.05 & 28 & 26.00 &  7.14\\
instance n=50 179.alb & 1 & 0 & Solution & 120.03 & 27 & 25.00 &  7.41\\
instance n=50 18.alb & 1 & 0 & Optimal &  0.04 & 7 &  7.00 &  0.00\\
instance n=50 180.alb & 1 & 0 & Solution & 120.02 & 26 & 25.00 &  3.85\\
instance n=50 181.alb & 1 & 0 & Solution & 120.04 & 29 & 27.00 &  6.90\\
instance n=50 182.alb & 1 & 0 & Solution & 120.01 & 27 & 25.00 &  7.41\\
instance n=50 183.alb & 1 & 0 & Solution & 120.03 & 29 & 26.00 & 10.34\\
instance n=50 184.alb & 1 & 0 & Solution & 120.02 & 38 & 29.00 & 23.68\\
instance n=50 185.alb & 1 & 0 & Solution & 120.03 & 27 & 25.00 &  7.41\\
instance n=50 186.alb & 1 & 0 & Solution & 120.03 & 26 & 25.00 &  3.85\\
instance n=50 187.alb & 1 & 0 & Solution & 120.03 & 26 & 25.00 &  3.85\\
instance n=50 188.alb & 1 & 0 & Solution & 120.05 & 25 & 24.00 &  4.00\\
instance n=50 189.alb & 1 & 0 & Solution & 120.03 & 26 & 25.00 &  3.85\\
instance n=50 19.alb & 1 & 0 & Optimal &  0.03 & 8 &  8.00 &  0.00\\
instance n=50 190.alb & 1 & 0 & Solution & 120.03 & 30 & 26.00 & 13.33\\
instance n=50 191.alb & 1 & 0 & Solution & 120.02 & 28 & 26.00 &  7.14\\
instance n=50 192.alb & 1 & 0 & Solution & 120.02 & 27 & 26.00 &  3.70\\
instance n=50 193.alb & 1 & 0 & Solution & 120.02 & 28 & 27.00 &  3.57\\
instance n=50 194.alb & 1 & 0 & Solution & 120.03 & 28 & 26.00 &  7.14\\
instance n=50 195.alb & 1 & 0 & Solution & 120.05 & 28 & 26.00 &  7.14\\
instance n=50 196.alb & 1 & 0 & Solution & 120.03 & 27 & 26.00 &  3.70\\
instance n=50 197.alb & 1 & 0 & Solution & 120.03 & 28 & 27.00 &  3.57\\
instance n=50 198.alb & 1 & 0 & Solution & 120.03 & 28 & 26.00 &  7.14\\
instance n=50 199.alb & 1 & 0 & Solution & 120.04 & 29 & 27.00 &  6.90\\
instance n=50 2.alb & 1 & 0 & Optimal &  0.02 & 6 &  6.00 &  0.00\\
instance n=50 20.alb & 1 & 0 & Optimal &  0.03 & 8 &  8.00 &  0.00\\
instance n=50 200.alb & 1 & 0 & Solution & 120.03 & 25 & 24.00 &  4.00\\
instance n=50 201.alb & 1 & 0 & Optimal &  0.20 & 13 & 13.00 &  0.00\\
instance n=50 202.alb & 1 & 0 & Optimal &  0.47 & 9 &  9.00 &  0.00\\
instance n=50 203.alb & 1 & 0 & Optimal &  0.37 & 11 & 11.00 &  0.00\\
instance n=50 204.alb & 1 & 0 & Optimal &  1.01 & 10 & 10.00 &  0.00\\
instance n=50 205.alb & 1 & 0 & Optimal &  0.20 & 13 & 13.00 &  0.00\\
instance n=50 206.alb & 1 & 0 & Optimal & 13.25 & 11 & 11.00 &  0.00\\
instance n=50 207.alb & 1 & 0 & Optimal &  0.13 & 10 & 10.00 &  0.00\\
instance n=50 208.alb & 1 & 0 & Optimal &  0.32 & 13 & 13.00 &  0.00\\
instance n=50 209.alb & 1 & 0 & Optimal &  0.22 & 11 & 11.00 &  0.00\\
instance n=50 21.alb & 1 & 0 & Optimal &  0.03 & 6 &  6.00 &  0.00\\
instance n=50 210.alb & 1 & 0 & Optimal &  0.25 & 13 & 13.00 &  0.00\\
instance n=50 211.alb & 1 & 0 & Optimal &  0.14 & 12 & 12.00 &  0.00\\
instance n=50 212.alb & 1 & 0 & Optimal &  0.18 & 10 & 10.00 &  0.00\\
instance n=50 213.alb & 1 & 0 & Optimal &  0.15 & 13 & 13.00 &  0.00\\
instance n=50 214.alb & 1 & 0 & Optimal &  0.14 & 11 & 11.00 &  0.00\\
instance n=50 215.alb & 1 & 0 & Optimal &  0.23 & 11 & 11.00 &  0.00\\
instance n=50 216.alb & 1 & 0 & Optimal &  0.41 & 12 & 12.00 &  0.00\\
instance n=50 217.alb & 1 & 0 & Optimal &  1.16 & 13 & 13.00 &  0.00\\
instance n=50 218.alb & 1 & 0 & Optimal &  0.12 & 12 & 12.00 &  0.00\\
instance n=50 219.alb & 1 & 0 & Optimal &  0.20 & 11 & 11.00 &  0.00\\
instance n=50 22.alb & 1 & 0 & Optimal &  0.03 & 7 &  7.00 &  0.00\\
instance n=50 220.alb & 1 & 0 & Optimal &  0.14 & 11 & 11.00 &  0.00\\
instance n=50 221.alb & 1 & 0 & Optimal &  1.02 & 11 & 11.00 &  0.00\\
instance n=50 222.alb & 1 & 0 & Optimal &  0.16 & 14 & 14.00 &  0.00\\
instance n=50 223.alb & 1 & 0 & Optimal &  1.67 & 11 & 11.00 &  0.00\\
instance n=50 224.alb & 1 & 0 & Optimal &  0.12 & 11 & 11.00 &  0.00\\
instance n=50 225.alb & 1 & 0 & Optimal &  0.20 & 12 & 12.00 &  0.00\\
instance n=50 226.alb & 1 & 0 & Optimal &  0.14 & 7 &  7.00 &  0.00\\
instance n=50 227.alb & 1 & 0 & Optimal &  0.22 & 6 &  6.00 &  0.00\\
instance n=50 228.alb & 1 & 0 & Optimal &  0.21 & 6 &  6.00 &  0.00\\
instance n=50 229.alb & 1 & 0 & Optimal &  0.16 & 6 &  6.00 &  0.00\\
instance n=50 23.alb & 1 & 0 & Optimal &  0.03 & 7 &  7.00 &  0.00\\
instance n=50 230.alb & 1 & 0 & Optimal &  0.24 & 7 &  7.00 &  0.00\\
instance n=50 231.alb & 1 & 0 & Optimal &  0.28 & 7 &  7.00 &  0.00\\
instance n=50 232.alb & 1 & 0 & Optimal &  0.98 & 7 &  7.00 &  0.00\\
instance n=50 233.alb & 1 & 0 & Optimal &  0.20 & 6 &  6.00 &  0.00\\
instance n=50 234.alb & 1 & 0 & Optimal &  0.22 & 8 &  8.00 &  0.00\\
instance n=50 235.alb & 1 & 0 & Optimal &  0.26 & 7 &  7.00 &  0.00\\
instance n=50 236.alb & 1 & 0 & Optimal &  0.48 & 7 &  7.00 &  0.00\\
instance n=50 237.alb & 1 & 0 & Optimal &  0.20 & 8 &  8.00 &  0.00\\
instance n=50 238.alb & 1 & 0 & Optimal &  0.26 & 7 &  7.00 &  0.00\\
instance n=50 239.alb & 1 & 0 & Optimal &  0.36 & 7 &  7.00 &  0.00\\
instance n=50 24.alb & 1 & 0 & Optimal &  0.03 & 7 &  7.00 &  0.00\\
instance n=50 240.alb & 1 & 0 & Optimal &  0.14 & 7 &  7.00 &  0.00\\
instance n=50 241.alb & 1 & 0 & Optimal &  0.19 & 7 &  7.00 &  0.00\\
instance n=50 242.alb & 1 & 0 & Optimal &  0.24 & 8 &  8.00 &  0.00\\
instance n=50 243.alb & 1 & 0 & Optimal &  0.17 & 7 &  7.00 &  0.00\\
instance n=50 244.alb & 1 & 0 & Optimal &  0.55 & 7 &  7.00 &  0.00\\
instance n=50 245.alb & 1 & 0 & Optimal &  0.30 & 7 &  7.00 &  0.00\\
instance n=50 246.alb & 1 & 0 & Optimal &  0.21 & 8 &  8.00 &  0.00\\
instance n=50 247.alb & 1 & 0 & Optimal &  0.24 & 7 &  7.00 &  0.00\\
instance n=50 248.alb & 1 & 0 & Optimal &  0.13 & 7 &  7.00 &  0.00\\
instance n=50 249.alb & 1 & 0 & Optimal &  0.54 & 7 &  7.00 &  0.00\\
instance n=50 25.alb & 1 & 0 & Optimal &  0.03 & 6 &  6.00 &  0.00\\
instance n=50 250.alb & 1 & 0 & Optimal &  0.18 & 7 &  7.00 &  0.00\\
instance n=50 251.alb & 1 & 0 & Solution & 120.03 & 27 & 26.00 &  3.70\\
instance n=50 252.alb & 1 & 0 & Solution & 120.05 & 32 & 28.00 & 12.50\\
instance n=50 253.alb & 1 & 0 & Solution & 120.03 & 28 & 26.00 &  7.14\\
instance n=50 254.alb & 1 & 0 & Solution & 120.05 & 30 & 28.00 &  6.67\\
instance n=50 255.alb & 1 & 0 & Solution & 120.05 & 29 & 27.00 &  6.90\\
instance n=50 256.alb & 1 & 0 & Solution & 120.04 & 30 & 28.00 &  6.67\\
instance n=50 257.alb & 1 & 0 & Solution & 120.05 & 33 & 29.00 & 12.12\\
instance n=50 258.alb & 1 & 0 & Solution & 120.03 & 28 & 27.00 &  3.57\\
instance n=50 259.alb & 1 & 0 & Solution & 120.04 & 31 & 26.00 & 16.13\\
instance n=50 26.alb & 1 & 0 & Solution & 120.01 & 27 & 25.00 &  7.41\\
instance n=50 260.alb & 1 & 0 & Solution & 120.05 & 29 & 27.00 &  6.90\\
instance n=50 261.alb & 1 & 0 & Solution & 120.05 & 28 & 27.00 &  3.57\\
instance n=50 262.alb & 1 & 0 & Solution & 120.04 & 31 & 26.00 & 16.13\\
instance n=50 263.alb & 1 & 0 & Optimal & 118.45 & 29 & 29.00 &  0.00\\
instance n=50 264.alb & 1 & 0 & Solution & 120.04 & 27 & 26.00 &  3.70\\
instance n=50 265.alb & 1 & 0 & Solution & 120.05 & 27 & 26.00 &  3.70\\
instance n=50 266.alb & 1 & 0 & Optimal & 89.34 & 29 & 29.00 &  0.00\\
instance n=50 267.alb & 1 & 0 & Solution & 120.04 & 28 & 27.00 &  3.57\\
instance n=50 268.alb & 1 & 0 & Solution & 120.04 & 29 & 27.00 &  6.90\\
instance n=50 269.alb & 1 & 0 & Optimal & 37.05 & 26 & 26.00 &  0.00\\
instance n=50 27.alb & 1 & 0 & Solution & 120.01 & 30 & 27.00 & 10.00\\
instance n=50 270.alb & 1 & 0 & Solution & 120.04 & 28 & 27.00 &  3.57\\
instance n=50 271.alb & 1 & 0 & Solution & 120.04 & 31 & 29.00 &  6.45\\
instance n=50 272.alb & 1 & 0 & Solution & 120.03 & 27 & 26.00 &  3.70\\
instance n=50 273.alb & 1 & 0 & Solution & 120.04 & 27 & 26.00 &  3.70\\
instance n=50 274.alb & 1 & 0 & Solution & 120.04 & 29 & 27.00 &  6.90\\
instance n=50 275.alb & 1 & 0 & Optimal &  7.83 & 27 & 27.00 &  0.00\\
instance n=50 276.alb & 1 & 0 & Optimal &  0.88 & 12 & 12.00 &  0.00\\
instance n=50 277.alb & 1 & 0 & Optimal &  0.16 & 13 & 13.00 &  0.00\\
instance n=50 278.alb & 1 & 0 & Optimal &  0.63 & 12 & 12.00 &  0.00\\
instance n=50 279.alb & 1 & 0 & Optimal &  0.16 & 11 & 11.00 &  0.00\\
instance n=50 28.alb & 1 & 0 & Solution & 120.01 & 28 & 26.00 &  7.14\\
instance n=50 280.alb & 1 & 0 & Optimal &  0.19 & 13 & 13.00 &  0.00\\
instance n=50 281.alb & 1 & 0 & Optimal &  0.36 & 11 & 11.00 &  0.00\\
instance n=50 282.alb & 1 & 0 & Optimal &  4.74 & 12 & 12.00 &  0.00\\
instance n=50 283.alb & 1 & 0 & Optimal &  0.50 & 12 & 12.00 &  0.00\\
instance n=50 284.alb & 1 & 0 & Optimal &  0.20 & 11 & 11.00 &  0.00\\
instance n=50 285.alb & 1 & 0 & Optimal &  0.76 & 13 & 13.00 &  0.00\\
instance n=50 286.alb & 1 & 0 & Optimal &  0.94 & 11 & 11.00 &  0.00\\
instance n=50 287.alb & 1 & 0 & Optimal &  0.91 & 12 & 12.00 &  0.00\\
instance n=50 288.alb & 1 & 0 & Optimal &  0.49 & 10 & 10.00 &  0.00\\
instance n=50 289.alb & 1 & 0 & Optimal &  0.79 & 11 & 11.00 &  0.00\\
instance n=50 29.alb & 1 & 0 & Solution & 120.01 & 29 & 25.00 & 13.79\\
instance n=50 290.alb & 1 & 0 & Optimal &  0.52 & 14 & 14.00 &  0.00\\
instance n=50 291.alb & 1 & 0 & Optimal &  0.20 & 12 & 12.00 &  0.00\\
instance n=50 292.alb & 1 & 0 & Optimal &  0.17 & 13 & 13.00 &  0.00\\
instance n=50 293.alb & 1 & 0 & Optimal &  0.20 & 12 & 12.00 &  0.00\\
instance n=50 294.alb & 1 & 0 & Optimal &  0.20 & 13 & 13.00 &  0.00\\
instance n=50 295.alb & 1 & 0 & Optimal &  1.49 & 16 & 16.00 &  0.00\\
instance n=50 296.alb & 1 & 0 & Optimal &  0.24 & 13 & 13.00 &  0.00\\
instance n=50 297.alb & 1 & 0 & Optimal &  0.24 & 13 & 13.00 &  0.00\\
instance n=50 298.alb & 1 & 0 & Optimal &  0.55 & 11 & 11.00 &  0.00\\
instance n=50 299.alb & 1 & 0 & Optimal &  2.84 & 12 & 12.00 &  0.00\\
instance n=50 3.alb & 1 & 0 & Optimal &  0.04 & 8 &  8.00 &  0.00\\
instance n=50 30.alb & 1 & 0 & Solution & 120.01 & 27 & 25.00 &  7.41\\
instance n=50 300.alb & 1 & 0 & Optimal &  0.25 & 12 & 12.00 &  0.00\\
instance n=50 301.alb & 1 & 0 & Optimal &  0.33 & 6 &  6.00 &  0.00\\
instance n=50 302.alb & 1 & 0 & Optimal &  0.23 & 7 &  7.00 &  0.00\\
instance n=50 303.alb & 1 & 0 & Optimal &  0.20 & 8 &  8.00 &  0.00\\
instance n=50 304.alb & 1 & 0 & Optimal &  0.20 & 7 &  7.00 &  0.00\\
instance n=50 305.alb & 1 & 0 & Optimal &  0.20 & 8 &  8.00 &  0.00\\
instance n=50 306.alb & 1 & 0 & Optimal &  0.28 & 7 &  7.00 &  0.00\\
instance n=50 307.alb & 1 & 0 & Optimal &  0.27 & 7 &  7.00 &  0.00\\
instance n=50 308.alb & 1 & 0 & Optimal &  0.40 & 8 &  8.00 &  0.00\\
instance n=50 309.alb & 1 & 0 & Optimal &  0.49 & 7 &  7.00 &  0.00\\
instance n=50 31.alb & 1 & 0 & Solution & 120.01 & 28 & 25.00 & 10.71\\
instance n=50 310.alb & 1 & 0 & Optimal &  0.20 & 8 &  8.00 &  0.00\\
instance n=50 311.alb & 1 & 0 & Optimal &  0.20 & 8 &  8.00 &  0.00\\
instance n=50 312.alb & 1 & 0 & Optimal &  0.22 & 6 &  6.00 &  0.00\\
instance n=50 313.alb & 1 & 0 & Optimal &  0.20 & 8 &  8.00 &  0.00\\
instance n=50 314.alb & 1 & 0 & Optimal &  0.22 & 7 &  7.00 &  0.00\\
instance n=50 315.alb & 1 & 0 & Optimal &  0.30 & 8 &  8.00 &  0.00\\
instance n=50 316.alb & 1 & 0 & Optimal &  0.17 & 8 &  8.00 &  0.00\\
instance n=50 317.alb & 1 & 0 & Optimal &  0.17 & 6 &  6.00 &  0.00\\
instance n=50 318.alb & 1 & 0 & Optimal &  0.33 & 8 &  8.00 &  0.00\\
instance n=50 319.alb & 1 & 0 & Optimal &  0.20 & 7 &  7.00 &  0.00\\
instance n=50 32.alb & 1 & 0 & Optimal &  2.07 & 25 & 25.00 &  0.00\\
instance n=50 320.alb & 1 & 0 & Optimal &  0.22 & 8 &  8.00 &  0.00\\
instance n=50 321.alb & 1 & 0 & Optimal &  0.28 & 6 &  6.00 &  0.00\\
instance n=50 322.alb & 1 & 0 & Optimal &  0.20 & 7 &  7.00 &  0.00\\
instance n=50 323.alb & 1 & 0 & Optimal &  0.27 & 7 &  7.00 &  0.00\\
instance n=50 324.alb & 1 & 0 & Optimal &  0.30 & 7 &  7.00 &  0.00\\
instance n=50 325.alb & 1 & 0 & Optimal &  0.19 & 7 &  7.00 &  0.00\\
instance n=50 326.alb & 1 & 0 & Solution & 120.04 & 33 & 28.00 & 15.15\\
instance n=50 327.alb & 1 & 0 & Solution & 120.06 & 28 & 25.00 & 10.71\\
instance n=50 328.alb & 1 & 0 & Solution & 120.04 & 32 & 28.00 & 12.50\\
instance n=50 329.alb & 1 & 0 & Solution & 120.03 & 25 & 24.00 &  4.00\\
instance n=50 33.alb & 1 & 0 & Solution & 120.00 & 25 & 24.00 &  4.00\\
instance n=50 330.alb & 1 & 0 & Solution & 120.04 & 29 & 25.00 & 13.79\\
instance n=50 331.alb & 1 & 0 & Solution & 120.05 & 29 & 27.00 &  6.90\\
instance n=50 332.alb & 1 & 0 & Solution & 120.03 & 25 & 24.00 &  4.00\\
instance n=50 333.alb & 1 & 0 & Solution & 120.06 & 28 & 26.00 &  7.14\\
instance n=50 334.alb & 1 & 0 & Solution & 120.06 & 29 & 25.00 & 13.79\\
instance n=50 335.alb & 1 & 0 & Solution & 120.05 & 27 & 26.00 &  3.70\\
instance n=50 336.alb & 1 & 0 & Solution & 120.03 & 26 & 25.00 &  3.85\\
instance n=50 337.alb & 1 & 0 & Solution & 120.04 & 26 & 25.00 &  3.85\\
instance n=50 338.alb & 1 & 0 & Solution & 120.05 & 27 & 26.00 &  3.70\\
instance n=50 339.alb & 1 & 0 & Solution & 120.05 & 27 & 26.00 &  3.70\\
instance n=50 34.alb & 1 & 0 & Solution & 120.01 & 30 & 27.00 & 10.00\\
instance n=50 340.alb & 1 & 0 & Solution & 120.03 & 28 & 26.00 &  7.14\\
instance n=50 341.alb & 1 & 0 & Solution & 120.05 & 27 & 25.00 &  7.41\\
instance n=50 342.alb & 1 & 0 & Solution & 120.05 & 28 & 26.00 &  7.14\\
instance n=50 343.alb & 1 & 0 & Solution & 120.04 & 27 & 25.00 &  7.41\\
instance n=50 344.alb & 1 & 0 & Solution & 120.06 & 30 & 27.00 & 10.00\\
instance n=50 345.alb & 1 & 0 & Solution & 120.03 & 29 & 27.00 &  6.90\\
instance n=50 346.alb & 1 & 0 & Solution & 120.06 & 27 & 25.00 &  7.41\\
instance n=50 347.alb & 1 & 0 & Solution & 120.06 & 26 & 25.00 &  3.85\\
instance n=50 348.alb & 1 & 0 & Solution & 120.06 & 30 & 25.00 & 16.67\\
instance n=50 349.alb & 1 & 0 & Solution & 120.06 & 28 & 26.00 &  7.14\\
instance n=50 35.alb & 1 & 0 & Solution & 120.00 & 32 & 27.00 & 15.63\\
instance n=50 350.alb & 1 & 0 & Solution & 120.03 & 24 & 23.00 &  4.17\\
instance n=50 351.alb & 1 & 0 & Optimal &  0.19 & 12 & 12.00 &  0.00\\
instance n=50 352.alb & 1 & 0 & Optimal &  3.50 & 10 & 10.00 &  0.00\\
instance n=50 353.alb & 1 & 0 & Optimal &  0.42 & 13 & 13.00 &  0.00\\
instance n=50 354.alb & 1 & 0 & Solution & 120.04 & 14 & 13.00 &  7.14\\
instance n=50 355.alb & 1 & 0 & Optimal &  0.27 & 11 & 11.00 &  0.00\\
instance n=50 356.alb & 1 & 0 & Optimal &  0.20 & 15 & 15.00 &  0.00\\
instance n=50 357.alb & 1 & 0 & Optimal &  0.24 & 12 & 12.00 &  0.00\\
instance n=50 358.alb & 1 & 0 & Optimal &  0.25 & 11 & 11.00 &  0.00\\
instance n=50 359.alb & 1 & 0 & Optimal &  0.24 & 10 & 10.00 &  0.00\\
instance n=50 36.alb & 1 & 0 & Solution & 120.01 & 31 & 27.00 & 12.90\\
instance n=50 360.alb & 1 & 0 & Optimal &  0.61 & 12 & 12.00 &  0.00\\
instance n=50 361.alb & 1 & 0 & Optimal &  0.25 & 11 & 11.00 &  0.00\\
instance n=50 362.alb & 1 & 0 & Optimal &  0.32 & 10 & 10.00 &  0.00\\
instance n=50 363.alb & 1 & 0 & Solution & 120.05 & 12 & 11.00 &  8.33\\
instance n=50 364.alb & 1 & 0 & Optimal &  0.19 & 13 & 13.00 &  0.00\\
instance n=50 365.alb & 1 & 0 & Optimal &  0.28 & 11 & 11.00 &  0.00\\
instance n=50 366.alb & 1 & 0 & Optimal &  0.27 & 13 & 13.00 &  0.00\\
instance n=50 367.alb & 1 & 0 & Optimal &  0.25 & 12 & 12.00 &  0.00\\
instance n=50 368.alb & 1 & 0 & Optimal &  0.27 & 12 & 12.00 &  0.00\\
instance n=50 369.alb & 1 & 0 & Optimal &  0.58 & 12 & 12.00 &  0.00\\
instance n=50 37.alb & 1 & 0 & Solution & 120.01 & 32 & 27.00 & 15.63\\
instance n=50 370.alb & 1 & 0 & Optimal &  0.33 & 12 & 12.00 &  0.00\\
instance n=50 371.alb & 1 & 0 & Optimal &  2.56 & 11 & 11.00 &  0.00\\
instance n=50 372.alb & 1 & 0 & Optimal &  1.71 & 10 & 10.00 &  0.00\\
instance n=50 373.alb & 1 & 0 & Optimal &  0.22 & 12 & 12.00 &  0.00\\
instance n=50 374.alb & 1 & 0 & Optimal &  0.27 & 11 & 11.00 &  0.00\\
instance n=50 375.alb & 1 & 0 & Optimal &  1.08 & 13 & 13.00 &  0.00\\
instance n=50 376.alb & 1 & 0 & Optimal &  0.28 & 7 &  7.00 &  0.00\\
instance n=50 377.alb & 1 & 0 & Optimal &  0.21 & 7 &  7.00 &  0.00\\
instance n=50 378.alb & 1 & 0 & Optimal &  0.21 & 8 &  8.00 &  0.00\\
instance n=50 379.alb & 1 & 0 & Optimal &  0.25 & 7 &  7.00 &  0.00\\
instance n=50 38.alb & 1 & 0 & Solution & 120.00 & 31 & 28.00 &  9.68\\
instance n=50 380.alb & 1 & 0 & Optimal &  0.30 & 7 &  7.00 &  0.00\\
instance n=50 381.alb & 1 & 0 & Optimal &  0.33 & 8 &  8.00 &  0.00\\
instance n=50 382.alb & 1 & 0 & Optimal &  0.25 & 6 &  6.00 &  0.00\\
instance n=50 383.alb & 1 & 0 & Optimal &  0.25 & 7 &  7.00 &  0.00\\
instance n=50 384.alb & 1 & 0 & Optimal &  1.32 & 8 &  8.00 &  0.00\\
instance n=50 385.alb & 1 & 0 & Optimal &  0.22 & 7 &  7.00 &  0.00\\
instance n=50 386.alb & 1 & 0 & Optimal &  0.33 & 7 &  7.00 &  0.00\\
instance n=50 387.alb & 1 & 0 & Optimal &  0.27 & 8 &  8.00 &  0.00\\
instance n=50 388.alb & 1 & 0 & Optimal &  0.30 & 7 &  7.00 &  0.00\\
instance n=50 389.alb & 1 & 0 & Optimal &  0.24 & 8 &  8.00 &  0.00\\
instance n=50 39.alb & 1 & 0 & Solution & 120.00 & 29 & 26.00 & 10.34\\
instance n=50 390.alb & 1 & 0 & Optimal &  1.56 & 7 &  7.00 &  0.00\\
instance n=50 391.alb & 1 & 0 & Optimal &  0.30 & 7 &  7.00 &  0.00\\
instance n=50 392.alb & 1 & 0 & Optimal &  0.22 & 8 &  8.00 &  0.00\\
instance n=50 393.alb & 1 & 0 & Optimal &  0.30 & 7 &  7.00 &  0.00\\
instance n=50 394.alb & 1 & 0 & Optimal &  0.29 & 8 &  8.00 &  0.00\\
instance n=50 395.alb & 1 & 0 & Optimal &  0.27 & 7 &  7.00 &  0.00\\
instance n=50 396.alb & 1 & 0 & Optimal &  0.39 & 8 &  8.00 &  0.00\\
instance n=50 397.alb & 1 & 0 & Optimal &  0.22 & 7 &  7.00 &  0.00\\
instance n=50 398.alb & 1 & 0 & Optimal &  0.96 & 6 &  6.00 &  0.00\\
instance n=50 399.alb & 1 & 0 & Optimal &  2.02 & 7 &  7.00 &  0.00\\
instance n=50 4.alb & 1 & 0 & Optimal &  0.04 & 7 &  7.00 &  0.00\\
instance n=50 40.alb & 1 & 0 & Solution & 120.00 & 26 & 25.00 &  3.85\\
instance n=50 400.alb & 1 & 0 & Optimal &  0.24 & 8 &  8.00 &  0.00\\
instance n=50 401.alb & 1 & 0 & Solution & 120.04 & 28 & 26.00 &  7.14\\
instance n=50 402.alb & 1 & 0 & Solution & 120.04 & 27 & 26.00 &  3.70\\
instance n=50 403.alb & 1 & 0 & Solution & 120.06 & 34 & 30.00 & 11.76\\
instance n=50 404.alb & 1 & 0 & Solution & 120.07 & 31 & 26.00 & 16.13\\
instance n=50 405.alb & 1 & 0 & Solution & 120.05 & 27 & 26.00 &  3.70\\
instance n=50 406.alb & 1 & 0 & Solution & 120.06 & 32 & 30.00 &  6.25\\
instance n=50 407.alb & 1 & 0 & Solution & 120.06 & 29 & 26.00 & 10.34\\
instance n=50 408.alb & 1 & 0 & Optimal & 37.74 & 26 & 26.00 &  0.00\\
instance n=50 409.alb & 1 & 0 & Solution & 120.07 & 33 & 27.00 & 18.18\\
instance n=50 41.alb & 1 & 0 & Solution & 120.01 & 26 & 25.00 &  3.85\\
instance n=50 410.alb & 1 & 0 & Solution & 120.05 & 28 & 26.00 &  7.14\\
instance n=50 411.alb & 1 & 0 & Solution & 120.06 & 29 & 27.00 &  6.90\\
instance n=50 412.alb & 1 & 0 & Optimal & 109.80 & 26 & 26.00 &  0.00\\
instance n=50 413.alb & 1 & 0 & Solution & 120.07 & 30 & 26.00 & 13.33\\
instance n=50 414.alb & 1 & 0 & Solution & 120.05 & 27 & 25.00 &  7.41\\
instance n=50 415.alb & 1 & 0 & Solution & 120.07 & 28 & 26.00 &  7.14\\
instance n=50 416.alb & 1 & 0 & Solution & 120.10 & 27 & 26.00 &  3.70\\
instance n=50 417.alb & 1 & 0 & Solution & 120.06 & 30 & 27.00 & 10.00\\
instance n=50 418.alb & 1 & 0 & Solution & 120.07 & 27 & 25.00 &  7.41\\
instance n=50 419.alb & 1 & 0 & Solution & 120.08 & 33 & 28.00 & 15.15\\
instance n=50 42.alb & 1 & 0 & Solution & 120.00 & 24 & 23.00 &  4.17\\
instance n=50 420.alb & 1 & 0 & Solution & 120.05 & 28 & 26.00 &  7.14\\
instance n=50 421.alb & 1 & 0 & Solution & 120.06 & 34 & 29.00 & 14.71\\
instance n=50 422.alb & 1 & 0 & Solution & 120.05 & 29 & 26.00 & 10.34\\
instance n=50 423.alb & 1 & 0 & Solution & 120.03 & 29 & 26.00 & 10.34\\
instance n=50 424.alb & 1 & 0 & Solution & 120.05 & 27 & 26.00 &  3.70\\
instance n=50 425.alb & 1 & 0 & Solution & 120.07 & 34 & 30.00 & 11.76\\
instance n=50 426.alb & 1 & 0 & Optimal &  1.30 & 11 & 11.00 &  0.00\\
instance n=50 427.alb & 1 & 0 & Optimal &  0.37 & 12 & 12.00 &  0.00\\
instance n=50 428.alb & 1 & 0 & Optimal &  0.31 & 13 & 13.00 &  0.00\\
instance n=50 429.alb & 1 & 0 & Optimal &  0.31 & 11 & 11.00 &  0.00\\
instance n=50 43.alb & 1 & 0 & Optimal &  1.60 & 25 & 25.00 &  0.00\\
instance n=50 430.alb & 1 & 0 & Optimal &  1.26 & 14 & 14.00 &  0.00\\
instance n=50 431.alb & 1 & 0 & Optimal &  0.36 & 11 & 11.00 &  0.00\\
instance n=50 432.alb & 1 & 0 & Optimal &  1.30 & 12 & 12.00 &  0.00\\
instance n=50 433.alb & 1 & 0 & Optimal &  0.35 & 12 & 12.00 &  0.00\\
instance n=50 434.alb & 1 & 0 & Optimal &  0.57 & 11 & 11.00 &  0.00\\
instance n=50 435.alb & 1 & 0 & Optimal &  0.32 & 11 & 11.00 &  0.00\\
instance n=50 436.alb & 1 & 0 & Optimal &  0.24 & 11 & 11.00 &  0.00\\
instance n=50 437.alb & 1 & 0 & Optimal &  6.40 & 12 & 12.00 &  0.00\\
instance n=50 438.alb & 1 & 0 & Optimal &  4.98 & 10 & 10.00 &  0.00\\
instance n=50 439.alb & 1 & 0 & Optimal &  2.31 & 12 & 12.00 &  0.00\\
instance n=50 44.alb & 1 & 0 & Solution & 120.00 & 25 & 24.00 &  4.00\\
instance n=50 440.alb & 1 & 0 & Optimal &  8.07 & 13 & 13.00 &  0.00\\
instance n=50 441.alb & 1 & 0 & Optimal &  0.28 & 11 & 11.00 &  0.00\\
instance n=50 442.alb & 1 & 0 & Optimal &  0.64 & 12 & 12.00 &  0.00\\
instance n=50 443.alb & 1 & 0 & Optimal &  1.42 & 11 & 11.00 &  0.00\\
instance n=50 444.alb & 1 & 0 & Optimal &  0.36 & 12 & 12.00 &  0.00\\
instance n=50 445.alb & 1 & 0 & Optimal &  0.40 & 12 & 12.00 &  0.00\\
instance n=50 446.alb & 1 & 0 & Optimal &  0.71 & 12 & 12.00 &  0.00\\
instance n=50 447.alb & 1 & 0 & Optimal &  0.61 & 13 & 13.00 &  0.00\\
instance n=50 448.alb & 1 & 0 & Optimal &  7.05 & 12 & 12.00 &  0.00\\
instance n=50 449.alb & 1 & 0 & Optimal &  0.42 & 11 & 11.00 &  0.00\\
instance n=50 45.alb & 1 & 0 & Solution & 120.01 & 25 & 24.00 &  4.00\\
instance n=50 450.alb & 1 & 0 & Optimal &  0.33 & 11 & 11.00 &  0.00\\
instance n=50 451.alb & 1 & 0 & Optimal &  0.53 & 8 &  8.00 &  0.00\\
instance n=50 452.alb & 1 & 0 & Optimal &  0.28 & 8 &  8.00 &  0.00\\
instance n=50 453.alb & 1 & 0 & Optimal &  0.37 & 7 &  7.00 &  0.00\\
instance n=50 454.alb & 1 & 0 & Optimal &  1.08 & 8 &  8.00 &  0.00\\
instance n=50 455.alb & 1 & 0 & Optimal &  0.37 & 6 &  6.00 &  0.00\\
instance n=50 456.alb & 1 & 0 & Optimal &  0.46 & 8 &  8.00 &  0.00\\
instance n=50 457.alb & 1 & 0 & Optimal &  0.49 & 8 &  8.00 &  0.00\\
instance n=50 458.alb & 1 & 0 & Optimal &  0.60 & 7 &  7.00 &  0.00\\
instance n=50 459.alb & 1 & 0 & Optimal &  0.50 & 7 &  7.00 &  0.00\\
instance n=50 46.alb & 1 & 0 & Solution & 120.02 & 28 & 26.00 &  7.14\\
instance n=50 460.alb & 1 & 0 & Optimal &  0.53 & 7 &  7.00 &  0.00\\
instance n=50 461.alb & 1 & 0 & Optimal &  0.60 & 6 &  6.00 &  0.00\\
instance n=50 462.alb & 1 & 0 & Optimal &  0.33 & 7 &  7.00 &  0.00\\
instance n=50 463.alb & 1 & 0 & Optimal &  0.44 & 8 &  8.00 &  0.00\\
instance n=50 464.alb & 1 & 0 & Optimal &  0.50 & 6 &  6.00 &  0.00\\
instance n=50 465.alb & 1 & 0 & Optimal &  0.41 & 8 &  8.00 &  0.00\\
instance n=50 466.alb & 1 & 0 & Optimal &  0.63 & 7 &  7.00 &  0.00\\
instance n=50 467.alb & 1 & 0 & Optimal &  1.20 & 9 &  9.00 &  0.00\\
instance n=50 468.alb & 1 & 0 & Optimal &  0.41 & 7 &  7.00 &  0.00\\
instance n=50 469.alb & 1 & 0 & Optimal &  0.48 & 8 &  8.00 &  0.00\\
instance n=50 47.alb & 1 & 0 & Solution & 119.99 & 28 & 26.00 &  7.14\\
instance n=50 470.alb & 1 & 0 & Optimal &  0.36 & 8 &  8.00 &  0.00\\
instance n=50 471.alb & 1 & 0 & Optimal &  0.46 & 7 &  7.00 &  0.00\\
instance n=50 472.alb & 1 & 0 & Optimal &  0.38 & 8 &  8.00 &  0.00\\
instance n=50 473.alb & 1 & 0 & Optimal &  0.47 & 7 &  7.00 &  0.00\\
instance n=50 474.alb & 1 & 0 & Optimal &  0.47 & 7 &  7.00 &  0.00\\
instance n=50 475.alb & 1 & 0 & Optimal &  1.04 & 6 &  6.00 &  0.00\\
instance n=50 476.alb & 1 & 0 & Optimal &  1.13 & 28 & 28.00 &  0.00\\
instance n=50 477.alb & 1 & 0 & Optimal &  7.35 & 29 & 29.00 &  0.00\\
instance n=50 478.alb & 1 & 0 & Optimal & 10.16 & 32 & 32.00 &  0.00\\
instance n=50 479.alb & 1 & 0 & Optimal &  0.75 & 28 & 28.00 &  0.00\\
instance n=50 48.alb & 1 & 0 & Solution & 120.02 & 27 & 26.00 &  3.70\\
instance n=50 480.alb & 1 & 0 & Optimal &  1.32 & 34 & 34.00 &  0.00\\
instance n=50 481.alb & 1 & 0 & Optimal &  2.48 & 28 & 28.00 &  0.00\\
instance n=50 482.alb & 1 & 0 & Optimal &  1.79 & 27 & 27.00 &  0.00\\
instance n=50 483.alb & 1 & 0 & Optimal &  6.74 & 30 & 30.00 &  0.00\\
instance n=50 484.alb & 1 & 0 & Optimal &  1.87 & 32 & 32.00 &  0.00\\
instance n=50 485.alb & 1 & 0 & Optimal &  2.69 & 31 & 31.00 &  0.00\\
instance n=50 486.alb & 1 & 0 & Optimal &  1.51 & 32 & 31.00 &  3.13\\
instance n=50 487.alb & 1 & 0 & Optimal &  2.58 & 31 & 31.00 &  0.00\\
instance n=50 488.alb & 1 & 0 & Optimal &  6.63 & 31 & 31.00 &  0.00\\
instance n=50 489.alb & 1 & 0 & Optimal &  5.94 & 35 & 35.00 &  0.00\\
instance n=50 49.alb & 1 & 0 & Solution & 120.02 & 25 & 24.00 &  4.00\\
instance n=50 490.alb & 1 & 0 & Optimal &  2.56 & 29 & 29.00 &  0.00\\
instance n=50 491.alb & 1 & 0 & Optimal & 70.18 & 35 & 35.00 &  0.00\\
instance n=50 492.alb & 1 & 0 & Optimal &  7.01 & 29 & 29.00 &  0.00\\
instance n=50 493.alb & 1 & 0 & Optimal &  6.23 & 30 & 30.00 &  0.00\\
instance n=50 494.alb & 1 & 0 & Optimal &  3.77 & 32 & 32.00 &  0.00\\
instance n=50 495.alb & 1 & 0 & Optimal &  3.83 & 34 & 34.00 &  0.00\\
instance n=50 496.alb & 1 & 0 & Optimal &  3.96 & 29 & 29.00 &  0.00\\
instance n=50 497.alb & 1 & 0 & Optimal &  5.59 & 30 & 30.00 &  0.00\\
instance n=50 498.alb & 1 & 0 & Optimal &  1.51 & 30 & 30.00 &  0.00\\
instance n=50 499.alb & 1 & 0 & Optimal &  1.72 & 33 & 33.00 &  0.00\\
instance n=50 5.alb & 1 & 0 & Optimal &  0.03 & 7 &  7.00 &  0.00\\
instance n=50 50.alb & 1 & 0 & Solution & 120.00 & 27 & 25.00 &  7.41\\
instance n=50 500.alb & 1 & 0 & Optimal &  3.03 & 34 & 34.00 &  0.00\\
instance n=50 501.alb & 1 & 0 & Optimal &  1.21 & 12 & 12.00 &  0.00\\
instance n=50 502.alb & 1 & 0 & Optimal &  0.85 & 10 & 10.00 &  0.00\\
instance n=50 503.alb & 1 & 0 & Optimal &  1.21 & 13 & 13.00 &  0.00\\
instance n=50 504.alb & 1 & 0 & Optimal &  0.97 & 11 & 11.00 &  0.00\\
instance n=50 505.alb & 1 & 0 & Optimal &  0.94 & 12 & 12.00 &  0.00\\
instance n=50 506.alb & 1 & 0 & Optimal &  0.39 & 11 & 11.00 &  0.00\\
instance n=50 507.alb & 1 & 0 & Optimal &  0.64 & 13 & 13.00 &  0.00\\
instance n=50 508.alb & 1 & 0 & Optimal &  1.02 & 14 & 14.00 &  0.00\\
instance n=50 509.alb & 1 & 0 & Optimal &  0.36 & 13 & 13.00 &  0.00\\
instance n=50 51.alb & 1 & 0 & Optimal &  0.05 & 12 & 12.00 &  0.00\\
instance n=50 510.alb & 1 & 0 & Optimal &  1.22 & 11 & 11.00 &  0.00\\
instance n=50 511.alb & 1 & 0 & Optimal &  1.35 & 13 & 13.00 &  0.00\\
instance n=50 512.alb & 1 & 0 & Optimal &  1.16 & 13 & 13.00 &  0.00\\
instance n=50 513.alb & 1 & 0 & Optimal &  0.60 & 12 & 12.00 &  0.00\\
instance n=50 514.alb & 1 & 0 & Optimal &  1.32 & 12 & 12.00 &  0.00\\
instance n=50 515.alb & 1 & 0 & Optimal &  1.21 & 11 & 11.00 &  0.00\\
instance n=50 516.alb & 1 & 0 & Optimal &  0.88 & 13 & 13.00 &  0.00\\
instance n=50 517.alb & 1 & 0 & Optimal &  0.88 & 14 & 14.00 &  0.00\\
instance n=50 518.alb & 1 & 0 & Optimal &  1.19 & 11 & 11.00 &  0.00\\
instance n=50 519.alb & 1 & 0 & Optimal &  0.42 & 12 & 12.00 &  0.00\\
instance n=50 52.alb & 1 & 0 & Optimal &  0.05 & 11 & 11.00 &  0.00\\
instance n=50 520.alb & 1 & 0 & Optimal &  0.57 & 11 & 11.00 &  0.00\\
instance n=50 521.alb & 1 & 0 & Optimal &  0.33 & 10 & 10.00 &  0.00\\
instance n=50 522.alb & 1 & 0 & Optimal &  0.47 & 11 & 11.00 &  0.00\\
instance n=50 523.alb & 1 & 0 & Optimal &  1.01 & 11 & 11.00 &  0.00\\
instance n=50 524.alb & 1 & 0 & Optimal &  1.04 & 14 & 14.00 &  0.00\\
instance n=50 525.alb & 1 & 0 & Optimal &  1.29 & 11 & 11.00 &  0.00\\
instance n=50 53.alb & 1 & 0 & Solution & 120.01 & 13 & 12.00 &  7.69\\
instance n=50 54.alb & 1 & 0 & Optimal &  0.05 & 11 & 11.00 &  0.00\\
instance n=50 55.alb & 1 & 0 & Optimal &  0.07 & 13 & 13.00 &  0.00\\
instance n=50 56.alb & 1 & 0 & Optimal &  0.06 & 11 & 11.00 &  0.00\\
instance n=50 57.alb & 1 & 0 & Optimal &  0.06 & 13 & 13.00 &  0.00\\
instance n=50 58.alb & 1 & 0 & Optimal &  0.06 & 11 & 11.00 &  0.00\\
instance n=50 59.alb & 1 & 0 & Optimal &  0.06 & 11 & 11.00 &  0.00\\
instance n=50 6.alb & 1 & 0 & Optimal &  0.05 & 6 &  6.00 &  0.00\\
instance n=50 60.alb & 1 & 0 & Optimal &  0.23 & 12 & 12.00 &  0.00\\
instance n=50 61.alb & 1 & 0 & Optimal &  0.05 & 13 & 13.00 &  0.00\\
instance n=50 62.alb & 1 & 0 & Optimal &  0.06 & 13 & 13.00 &  0.00\\
instance n=50 63.alb & 1 & 0 & Optimal &  0.05 & 12 & 12.00 &  0.00\\
instance n=50 64.alb & 1 & 0 & Optimal &  0.05 & 13 & 13.00 &  0.00\\
instance n=50 65.alb & 1 & 0 & Optimal &  0.04 & 12 & 12.00 &  0.00\\
instance n=50 66.alb & 1 & 0 & Optimal &  0.25 & 12 & 12.00 &  0.00\\
instance n=50 67.alb & 1 & 0 & Optimal &  0.37 & 12 & 12.00 &  0.00\\
instance n=50 68.alb & 1 & 0 & Optimal &  0.08 & 12 & 12.00 &  0.00\\
instance n=50 69.alb & 1 & 0 & Optimal &  0.29 & 12 & 12.00 &  0.00\\
instance n=50 7.alb & 1 & 0 & Optimal &  0.03 & 7 &  7.00 &  0.00\\
instance n=50 70.alb & 1 & 0 & Optimal &  0.06 & 10 & 10.00 &  0.00\\
instance n=50 71.alb & 1 & 0 & Optimal &  0.09 & 13 & 13.00 &  0.00\\
instance n=50 72.alb & 1 & 0 & Optimal &  0.07 & 11 & 11.00 &  0.00\\
instance n=50 73.alb & 1 & 0 & Optimal &  0.07 & 11 & 11.00 &  0.00\\
instance n=50 74.alb & 1 & 0 & Optimal &  0.06 & 12 & 12.00 &  0.00\\
instance n=50 75.alb & 1 & 0 & Optimal &  0.74 & 11 & 11.00 &  0.00\\
instance n=50 76.alb & 1 & 0 & Optimal &  0.09 & 7 &  7.00 &  0.00\\
instance n=50 77.alb & 1 & 0 & Optimal &  0.06 & 7 &  7.00 &  0.00\\
instance n=50 78.alb & 1 & 0 & Optimal &  0.09 & 7 &  7.00 &  0.00\\
instance n=50 79.alb & 1 & 0 & Optimal &  0.20 & 8 &  8.00 &  0.00\\
instance n=50 8.alb & 1 & 0 & Optimal &  0.05 & 7 &  7.00 &  0.00\\
instance n=50 80.alb & 1 & 0 & Optimal &  0.08 & 7 &  7.00 &  0.00\\
instance n=50 81.alb & 1 & 0 & Optimal &  0.09 & 7 &  7.00 &  0.00\\
instance n=50 82.alb & 1 & 0 & Optimal &  0.08 & 6 &  6.00 &  0.00\\
instance n=50 83.alb & 1 & 0 & Optimal &  0.08 & 8 &  8.00 &  0.00\\
instance n=50 84.alb & 1 & 0 & Optimal &  0.08 & 7 &  7.00 &  0.00\\
instance n=50 85.alb & 1 & 0 & Optimal &  0.08 & 8 &  8.00 &  0.00\\
instance n=50 86.alb & 1 & 0 & Optimal &  0.08 & 7 &  7.00 &  0.00\\
instance n=50 87.alb & 1 & 0 & Optimal &  0.08 & 8 &  8.00 &  0.00\\
instance n=50 88.alb & 1 & 0 & Optimal &  0.08 & 8 &  8.00 &  0.00\\
instance n=50 89.alb & 1 & 0 & Optimal &  0.09 & 7 &  7.00 &  0.00\\
instance n=50 9.alb & 1 & 0 & Optimal &  0.03 & 9 &  9.00 &  0.00\\
instance n=50 90.alb & 1 & 0 & Optimal &  0.42 & 7 &  7.00 &  0.00\\
instance n=50 91.alb & 1 & 0 & Optimal &  0.08 & 7 &  7.00 &  0.00\\
instance n=50 92.alb & 1 & 0 & Optimal &  0.08 & 7 &  7.00 &  0.00\\
instance n=50 93.alb & 1 & 0 & Optimal &  0.06 & 7 &  7.00 &  0.00\\
instance n=50 94.alb & 1 & 0 & Optimal &  0.11 & 7 &  7.00 &  0.00\\
instance n=50 95.alb & 1 & 0 & Optimal &  0.08 & 7 &  7.00 &  0.00\\
instance n=50 96.alb & 1 & 0 & Optimal &  0.10 & 7 &  7.00 &  0.00\\
instance n=50 97.alb & 1 & 0 & Optimal &  0.22 & 7 &  7.00 &  0.00\\
instance n=50 98.alb & 1 & 0 & Optimal &  0.11 & 8 &  8.00 &  0.00\\
instance n=50 99.alb & 1 & 0 & Optimal &  0.11 & 7 &  7.00 &  0.00\\
\end{longtable}



\section{Results for CPSat}

\begin{longtable}{lrrlrrrr}
\caption{Results for SALBP-1 Problems (CPSat) (2100 Instances)}\\\toprule
Name & \shortstack{Nr\\Jobs} & \shortstack{Nr\\Machines} & Status & Time & Makespan & Bound & \shortstack{Gap\\Percent}\\ \midrule
\endhead
\bottomrule
\endfoot
instance n=1000 1.alb & 1 & 0 & Solution & 120.07 & 136 & 135.00 &  0.74\\
instance n=1000 10.alb & 1 & 0 & Solution & 120.08 & 141 & 140.00 &  0.71\\
instance n=1000 100.alb & 1 & 0 & Solution & 120.08 & 139 & 137.00 &  1.44\\
instance n=1000 101.alb & 1 & 0 & Solution & 120.17 & 554 & 430.00 & 22.38\\
instance n=1000 102.alb & 1 & 0 & Solution & 120.18 & 556 & 446.00 & 19.78\\
instance n=1000 103.alb & 1 & 0 & Solution & 120.23 & 560 & 469.00 & 16.25\\
instance n=1000 104.alb & 1 & 0 & Solution & 120.16 & 550 & 439.00 & 20.18\\
instance n=1000 105.alb & 1 & 0 & Solution & 120.16 & 545 & 439.00 & 19.45\\
instance n=1000 106.alb & 1 & 0 & Solution & 120.19 & 552 & 432.00 & 21.74\\
instance n=1000 107.alb & 1 & 0 & Solution & 120.14 & 540 & 444.00 & 17.78\\
instance n=1000 108.alb & 1 & 0 & Solution & 120.13 & 543 & 461.00 & 15.10\\
instance n=1000 109.alb & 1 & 0 & Solution & 120.19 & 546 & 427.00 & 21.79\\
instance n=1000 11.alb & 1 & 0 & Solution & 120.08 & 135 & 134.00 &  0.74\\
instance n=1000 110.alb & 1 & 0 & Solution & 120.18 & 557 & 430.00 & 22.80\\
instance n=1000 111.alb & 1 & 0 & Solution & 120.13 & 544 & 449.00 & 17.46\\
instance n=1000 112.alb & 1 & 0 & Solution & 120.18 & 549 & 449.00 & 18.21\\
instance n=1000 113.alb & 1 & 0 & Solution & 120.16 & 537 & 459.00 & 14.53\\
instance n=1000 114.alb & 1 & 0 & Solution & 120.13 & 548 & 425.00 & 22.45\\
instance n=1000 115.alb & 1 & 0 & Solution & 120.16 & 541 & 430.00 & 20.52\\
instance n=1000 116.alb & 1 & 0 & Solution & 120.17 & 543 & 449.00 & 17.31\\
instance n=1000 117.alb & 1 & 0 & Solution & 120.17 & 548 & 443.00 & 19.16\\
instance n=1000 118.alb & 1 & 0 & Solution & 120.13 & 564 & 452.00 & 19.86\\
instance n=1000 119.alb & 1 & 0 & Solution & 120.14 & 534 & 424.00 & 20.60\\
instance n=1000 12.alb & 1 & 0 & Solution & 120.07 & 135 & 134.00 &  0.74\\
instance n=1000 120.alb & 1 & 0 & Solution & 120.15 & 549 & 466.00 & 15.12\\
instance n=1000 121.alb & 1 & 0 & Solution & 120.15 & 543 & 470.00 & 13.44\\
instance n=1000 122.alb & 1 & 0 & Solution & 120.16 & 533 & 465.00 & 12.76\\
instance n=1000 123.alb & 1 & 0 & Solution & 120.15 & 556 & 441.00 & 20.68\\
instance n=1000 124.alb & 1 & 0 & Solution & 120.20 & 543 & 452.00 & 16.76\\
instance n=1000 125.alb & 1 & 0 & Solution & 120.16 & 545 & 440.00 & 19.27\\
instance n=1000 126.alb & 1 & 0 & Solution & 120.15 & 232 & 228.00 &  1.72\\
instance n=1000 127.alb & 1 & 0 & Solution & 120.09 & 224 & 221.00 &  1.34\\
instance n=1000 128.alb & 1 & 0 & Solution & 120.11 & 225 & 222.00 &  1.33\\
instance n=1000 129.alb & 1 & 0 & Solution & 120.12 & 226 & 223.00 &  1.33\\
instance n=1000 13.alb & 1 & 0 & Solution & 120.08 & 132 & 131.00 &  0.76\\
instance n=1000 130.alb & 1 & 0 & Solution & 120.09 & 225 & 221.00 &  1.78\\
instance n=1000 131.alb & 1 & 0 & Solution & 120.10 & 224 & 220.00 &  1.79\\
instance n=1000 132.alb & 1 & 0 & Solution & 120.09 & 218 & 214.00 &  1.83\\
instance n=1000 133.alb & 1 & 0 & Solution & 120.11 & 229 & 226.00 &  1.31\\
instance n=1000 134.alb & 1 & 0 & Solution & 120.12 & 219 & 215.00 &  1.83\\
instance n=1000 135.alb & 1 & 0 & Solution & 120.13 & 229 & 225.00 &  1.75\\
instance n=1000 136.alb & 1 & 0 & Solution & 120.11 & 232 & 228.00 &  1.72\\
instance n=1000 137.alb & 1 & 0 & Solution & 120.09 & 216 & 213.00 &  1.39\\
instance n=1000 138.alb & 1 & 0 & Solution & 120.12 & 225 & 221.00 &  1.78\\
instance n=1000 139.alb & 1 & 0 & Solution & 120.09 & 228 & 224.00 &  1.75\\
instance n=1000 14.alb & 1 & 0 & Solution & 120.08 & 138 & 136.00 &  1.45\\
instance n=1000 140.alb & 1 & 0 & Solution & 120.11 & 230 & 226.00 &  1.74\\
instance n=1000 141.alb & 1 & 0 & Solution & 120.10 & 219 & 215.00 &  1.83\\
instance n=1000 142.alb & 1 & 0 & Solution & 120.09 & 224 & 220.00 &  1.79\\
instance n=1000 143.alb & 1 & 0 & Solution & 120.10 & 217 & 213.00 &  1.84\\
instance n=1000 144.alb & 1 & 0 & Solution & 120.11 & 220 & 217.00 &  1.36\\
instance n=1000 145.alb & 1 & 0 & Solution & 120.10 & 223 & 220.00 &  1.35\\
instance n=1000 146.alb & 1 & 0 & Solution & 120.11 & 223 & 219.00 &  1.79\\
instance n=1000 147.alb & 1 & 0 & Solution & 120.13 & 233 & 229.00 &  1.72\\
instance n=1000 148.alb & 1 & 0 & Solution & 120.37 & 223 & 219.00 &  1.79\\
instance n=1000 149.alb & 1 & 0 & Solution & 120.10 & 241 & 237.00 &  1.66\\
instance n=1000 15.alb & 1 & 0 & Solution & 120.08 & 137 & 136.00 &  0.73\\
instance n=1000 150.alb & 1 & 0 & Solution & 120.09 & 225 & 222.00 &  1.33\\
instance n=1000 151.alb & 1 & 0 & Solution & 120.08 & 140 & 138.00 &  1.43\\
instance n=1000 152.alb & 1 & 0 & Solution & 120.07 & 138 & 136.00 &  1.45\\
instance n=1000 153.alb & 1 & 0 & Solution & 120.08 & 139 & 137.00 &  1.44\\
instance n=1000 154.alb & 1 & 0 & Solution & 120.10 & 142 & 140.00 &  1.41\\
instance n=1000 155.alb & 1 & 0 & Solution & 120.08 & 141 & 139.00 &  1.42\\
instance n=1000 156.alb & 1 & 0 & Solution & 120.10 & 143 & 141.00 &  1.40\\
instance n=1000 157.alb & 1 & 0 & Solution & 120.09 & 142 & 140.00 &  1.41\\
instance n=1000 158.alb & 1 & 0 & Solution & 120.10 & 137 & 136.00 &  0.73\\
instance n=1000 159.alb & 1 & 0 & Solution & 120.09 & 139 & 138.00 &  0.72\\
instance n=1000 16.alb & 1 & 0 & Solution & 120.09 & 138 & 137.00 &  0.72\\
instance n=1000 160.alb & 1 & 0 & Solution & 120.08 & 140 & 138.00 &  1.43\\
instance n=1000 161.alb & 1 & 0 & Solution & 120.08 & 134 & 133.00 &  0.75\\
instance n=1000 162.alb & 1 & 0 & Solution & 120.09 & 137 & 136.00 &  0.73\\
instance n=1000 163.alb & 1 & 0 & Solution & 120.09 & 141 & 139.00 &  1.42\\
instance n=1000 164.alb & 1 & 0 & Solution & 120.09 & 143 & 141.00 &  1.40\\
instance n=1000 165.alb & 1 & 0 & Solution & 120.09 & 137 & 135.00 &  1.46\\
instance n=1000 166.alb & 1 & 0 & Solution & 120.08 & 141 & 139.00 &  1.42\\
instance n=1000 167.alb & 1 & 0 & Solution & 120.11 & 140 & 139.00 &  0.71\\
instance n=1000 168.alb & 1 & 0 & Solution & 120.09 & 140 & 138.00 &  1.43\\
instance n=1000 169.alb & 1 & 0 & Solution & 120.10 & 136 & 134.00 &  1.47\\
instance n=1000 17.alb & 1 & 0 & Solution & 120.09 & 136 & 135.00 &  0.74\\
instance n=1000 170.alb & 1 & 0 & Solution & 120.08 & 136 & 134.00 &  1.47\\
instance n=1000 171.alb & 1 & 0 & Solution & 120.09 & 138 & 137.00 &  0.72\\
instance n=1000 172.alb & 1 & 0 & Solution & 120.09 & 136 & 135.00 &  0.74\\
instance n=1000 173.alb & 1 & 0 & Solution & 120.10 & 136 & 135.00 &  0.74\\
instance n=1000 174.alb & 1 & 0 & Solution & 120.08 & 137 & 136.00 &  0.73\\
instance n=1000 175.alb & 1 & 0 & Solution & 120.09 & 140 & 138.00 &  1.43\\
instance n=1000 176.alb & 1 & 0 & Solution & 120.11 & 562 & 322.00 & 42.70\\
instance n=1000 177.alb & 1 & 0 & Solution & 120.09 & 563 & 326.00 & 42.10\\
instance n=1000 178.alb & 1 & 0 & Solution & 120.11 & 567 & 325.00 & 42.68\\
instance n=1000 179.alb & 1 & 0 & Solution & 120.11 & 571 & 315.00 & 44.83\\
instance n=1000 18.alb & 1 & 0 & Solution & 120.08 & 135 & 134.00 &  0.74\\
instance n=1000 180.alb & 1 & 0 & Solution & 120.11 & 567 & 317.00 & 44.09\\
instance n=1000 181.alb & 1 & 0 & Solution & 120.16 & 567 & 320.00 & 43.56\\
instance n=1000 182.alb & 1 & 0 & Solution & 120.11 & 565 & 315.00 & 44.25\\
instance n=1000 183.alb & 1 & 0 & Solution & 120.11 & 554 & 317.00 & 42.78\\
instance n=1000 184.alb & 1 & 0 & Solution & 120.14 & 561 & 317.00 & 43.49\\
instance n=1000 185.alb & 1 & 0 & Solution & 120.13 & 557 & 319.00 & 42.73\\
instance n=1000 186.alb & 1 & 0 & Solution & 120.12 & 562 & 325.00 & 42.17\\
instance n=1000 187.alb & 1 & 0 & Solution & 120.13 & 565 & 331.00 & 41.42\\
instance n=1000 188.alb & 1 & 0 & Solution & 120.12 & 555 & 332.00 & 40.18\\
instance n=1000 189.alb & 1 & 0 & Solution & 120.12 & 555 & 317.00 & 42.88\\
instance n=1000 19.alb & 1 & 0 & Solution & 120.08 & 138 & 137.00 &  0.72\\
instance n=1000 190.alb & 1 & 0 & Solution & 120.11 & 563 & 313.00 & 44.40\\
instance n=1000 191.alb & 1 & 0 & Solution & 120.11 & 560 & 328.00 & 41.43\\
instance n=1000 192.alb & 1 & 0 & Solution & 120.13 & 563 & 326.00 & 42.10\\
instance n=1000 193.alb & 1 & 0 & Solution & 120.08 & 568 & 327.00 & 42.43\\
instance n=1000 194.alb & 1 & 0 & Solution & 120.13 & 568 & 319.00 & 43.84\\
instance n=1000 195.alb & 1 & 0 & Solution & 120.08 & 568 & 315.00 & 44.54\\
instance n=1000 196.alb & 1 & 0 & Solution & 120.11 & 560 & 320.00 & 42.86\\
instance n=1000 197.alb & 1 & 0 & Solution & 120.12 & 546 & 336.00 & 38.46\\
instance n=1000 198.alb & 1 & 0 & Solution & 120.13 & 567 & 318.00 & 43.92\\
instance n=1000 199.alb & 1 & 0 & Solution & 120.12 & 547 & 328.00 & 40.04\\
instance n=1000 2.alb & 1 & 0 & Solution & 120.07 & 138 & 137.00 &  0.72\\
instance n=1000 20.alb & 1 & 0 & Solution & 120.09 & 139 & 138.00 &  0.72\\
instance n=1000 200.alb & 1 & 0 & Solution & 120.09 & 556 & 322.00 & 42.09\\
instance n=1000 201.alb & 1 & 0 & Solution & 120.09 & 233 & 215.00 &  7.73\\
instance n=1000 202.alb & 1 & 0 & Solution & 120.11 & 230 & 188.00 & 18.26\\
instance n=1000 203.alb & 1 & 0 & Solution & 120.09 & 234 & 210.00 & 10.26\\
instance n=1000 204.alb & 1 & 0 & Solution & 120.10 & 232 & 218.00 &  6.03\\
instance n=1000 205.alb & 1 & 0 & Solution & 120.10 & 234 & 188.00 & 19.66\\
instance n=1000 206.alb & 1 & 0 & Solution & 120.10 & 233 & 192.00 & 17.60\\
instance n=1000 207.alb & 1 & 0 & Solution & 120.10 & 235 & 197.00 & 16.17\\
instance n=1000 208.alb & 1 & 0 & Solution & 120.12 & 234 & 226.00 &  3.42\\
instance n=1000 209.alb & 1 & 0 & Solution & 120.11 & 232 & 194.00 & 16.38\\
instance n=1000 21.alb & 1 & 0 & Solution & 120.10 & 139 & 138.00 &  0.72\\
instance n=1000 210.alb & 1 & 0 & Solution & 120.11 & 229 & 205.00 & 10.48\\
instance n=1000 211.alb & 1 & 0 & Solution & 120.10 & 224 & 189.00 & 15.63\\
instance n=1000 212.alb & 1 & 0 & Solution & 120.10 & 221 & 182.00 & 17.65\\
instance n=1000 213.alb & 1 & 0 & Solution & 120.12 & 238 & 211.00 & 11.34\\
instance n=1000 214.alb & 1 & 0 & Solution & 120.10 & 230 & 206.00 & 10.43\\
instance n=1000 215.alb & 1 & 0 & Solution & 120.07 & 227 & 218.00 &  3.96\\
instance n=1000 216.alb & 1 & 0 & Solution & 120.27 & 225 & 190.00 & 15.56\\
instance n=1000 217.alb & 1 & 0 & Solution & 120.10 & 229 & 191.00 & 16.59\\
instance n=1000 218.alb & 1 & 0 & Solution & 120.09 & 223 & 212.00 &  4.93\\
instance n=1000 219.alb & 1 & 0 & Solution & 120.10 & 237 & 223.00 &  5.91\\
instance n=1000 22.alb & 1 & 0 & Solution & 120.09 & 139 & 137.00 &  1.44\\
instance n=1000 220.alb & 1 & 0 & Solution & 120.09 & 229 & 200.00 & 12.66\\
instance n=1000 221.alb & 1 & 0 & Solution & 120.10 & 236 & 211.00 & 10.59\\
instance n=1000 222.alb & 1 & 0 & Solution & 120.12 & 226 & 212.00 &  6.19\\
instance n=1000 223.alb & 1 & 0 & Solution & 120.10 & 226 & 205.00 &  9.29\\
instance n=1000 224.alb & 1 & 0 & Solution & 120.11 & 231 & 219.00 &  5.19\\
instance n=1000 225.alb & 1 & 0 & Solution & 120.11 & 234 & 210.00 & 10.26\\
instance n=1000 226.alb & 1 & 0 & Solution & 120.10 & 138 & 136.00 &  1.45\\
instance n=1000 227.alb & 1 & 0 & Solution & 120.10 & 140 & 138.00 &  1.43\\
instance n=1000 228.alb & 1 & 0 & Solution & 120.09 & 135 & 133.00 &  1.48\\
instance n=1000 229.alb & 1 & 0 & Solution & 120.11 & 136 & 134.00 &  1.47\\
instance n=1000 23.alb & 1 & 0 & Solution & 120.35 & 137 & 136.00 &  0.73\\
instance n=1000 230.alb & 1 & 0 & Solution & 120.09 & 133 & 131.00 &  1.50\\
instance n=1000 231.alb & 1 & 0 & Solution & 120.11 & 140 & 138.00 &  1.43\\
instance n=1000 232.alb & 1 & 0 & Solution & 120.10 & 135 & 133.00 &  1.48\\
instance n=1000 233.alb & 1 & 0 & Solution & 120.13 & 137 & 135.00 &  1.46\\
instance n=1000 234.alb & 1 & 0 & Solution & 120.10 & 139 & 137.00 &  1.44\\
instance n=1000 235.alb & 1 & 0 & Solution & 120.09 & 134 & 133.00 &  0.75\\
instance n=1000 236.alb & 1 & 0 & Solution & 120.10 & 138 & 136.00 &  1.45\\
instance n=1000 237.alb & 1 & 0 & Solution & 120.08 & 140 & 138.00 &  1.43\\
instance n=1000 238.alb & 1 & 0 & Solution & 120.12 & 139 & 138.00 &  0.72\\
instance n=1000 239.alb & 1 & 0 & Solution & 120.09 & 136 & 135.00 &  0.74\\
instance n=1000 24.alb & 1 & 0 & Solution & 120.07 & 141 & 140.00 &  0.71\\
instance n=1000 240.alb & 1 & 0 & Solution & 120.09 & 137 & 135.00 &  1.46\\
instance n=1000 241.alb & 1 & 0 & Solution & 120.10 & 140 & 138.00 &  1.43\\
instance n=1000 242.alb & 1 & 0 & Solution & 120.21 & 137 & 135.00 &  1.46\\
instance n=1000 243.alb & 1 & 0 & Solution & 120.10 & 139 & 137.00 &  1.44\\
instance n=1000 244.alb & 1 & 0 & Solution & 120.12 & 138 & 137.00 &  0.72\\
instance n=1000 245.alb & 1 & 0 & Solution & 120.09 & 137 & 135.00 &  1.46\\
instance n=1000 246.alb & 1 & 0 & Solution & 120.10 & 137 & 135.00 &  1.46\\
instance n=1000 247.alb & 1 & 0 & Solution & 120.10 & 140 & 138.00 &  1.43\\
instance n=1000 248.alb & 1 & 0 & Solution & 120.13 & 141 & 138.00 &  2.13\\
instance n=1000 249.alb & 1 & 0 & Solution & 120.09 & 140 & 138.00 &  1.43\\
instance n=1000 25.alb & 1 & 0 & Solution & 120.08 & 137 & 136.00 &  0.73\\
instance n=1000 250.alb & 1 & 0 & Solution & 120.09 & 142 & 140.00 &  1.41\\
instance n=1000 251.alb & 1 & 0 & Solution & 120.18 & 577 & 445.00 & 22.88\\
instance n=1000 252.alb & 1 & 0 & Solution & 120.17 & 569 & 453.00 & 20.39\\
instance n=1000 253.alb & 1 & 0 & Solution & 120.18 & 577 & 403.00 & 30.16\\
instance n=1000 254.alb & 1 & 0 & Solution & 120.17 & 568 & 409.00 & 27.99\\
instance n=1000 255.alb & 1 & 0 & Solution & 120.16 & 556 & 440.00 & 20.86\\
instance n=1000 256.alb & 1 & 0 & Solution & 120.16 & 561 & 427.00 & 23.89\\
instance n=1000 257.alb & 1 & 0 & Solution & 120.17 & 571 & 407.00 & 28.72\\
instance n=1000 258.alb & 1 & 0 & Solution & 120.15 & 562 & 423.00 & 24.73\\
instance n=1000 259.alb & 1 & 0 & Solution & 120.20 & 561 & 444.00 & 20.86\\
instance n=1000 26.alb & 1 & 0 & Solution & 120.14 & 554 & 316.00 & 42.96\\
instance n=1000 260.alb & 1 & 0 & Solution & 120.16 & 556 & 437.00 & 21.40\\
instance n=1000 261.alb & 1 & 0 & Solution & 120.18 & 566 & 420.00 & 25.80\\
instance n=1000 262.alb & 1 & 0 & Solution & 120.19 & 554 & 441.00 & 20.40\\
instance n=1000 263.alb & 1 & 0 & Solution & 120.16 & 564 & 443.00 & 21.45\\
instance n=1000 264.alb & 1 & 0 & Solution & 120.18 & 560 & 434.00 & 22.50\\
instance n=1000 265.alb & 1 & 0 & Solution & 120.17 & 580 & 429.00 & 26.03\\
instance n=1000 266.alb & 1 & 0 & Solution & 120.12 & 560 & 416.00 & 25.71\\
instance n=1000 267.alb & 1 & 0 & Solution & 120.15 & 584 & 413.00 & 29.28\\
instance n=1000 268.alb & 1 & 0 & Solution & 120.14 & 558 & 444.00 & 20.43\\
instance n=1000 269.alb & 1 & 0 & Solution & 120.20 & 564 & 434.00 & 23.05\\
instance n=1000 27.alb & 1 & 0 & Solution & 120.13 & 554 & 314.00 & 43.32\\
instance n=1000 270.alb & 1 & 0 & Solution & 120.16 & 590 & 440.00 & 25.42\\
instance n=1000 271.alb & 1 & 0 & Solution & 120.16 & 555 & 410.00 & 26.13\\
instance n=1000 272.alb & 1 & 0 & Solution & 120.16 & 577 & 430.00 & 25.48\\
instance n=1000 273.alb & 1 & 0 & Solution & 120.15 & 566 & 428.00 & 24.38\\
instance n=1000 274.alb & 1 & 0 & Solution & 120.15 & 566 & 429.00 & 24.20\\
instance n=1000 275.alb & 1 & 0 & Solution & 120.16 & 571 & 445.00 & 22.07\\
instance n=1000 276.alb & 1 & 0 & Solution & 120.12 & 222 & 217.00 &  2.25\\
instance n=1000 277.alb & 1 & 0 & Solution & 120.12 & 230 & 225.00 &  2.17\\
instance n=1000 278.alb & 1 & 0 & Solution & 120.11 & 225 & 220.00 &  2.22\\
instance n=1000 279.alb & 1 & 0 & Solution & 120.11 & 220 & 215.00 &  2.27\\
instance n=1000 28.alb & 1 & 0 & Solution & 120.10 & 541 & 300.00 & 44.55\\
instance n=1000 280.alb & 1 & 0 & Solution & 120.09 & 230 & 226.00 &  1.74\\
instance n=1000 281.alb & 1 & 0 & Solution & 120.09 & 224 & 219.00 &  2.23\\
instance n=1000 282.alb & 1 & 0 & Solution & 120.11 & 219 & 214.00 &  2.28\\
instance n=1000 283.alb & 1 & 0 & Solution & 120.12 & 229 & 224.00 &  2.18\\
instance n=1000 284.alb & 1 & 0 & Solution & 120.11 & 222 & 217.00 &  2.25\\
instance n=1000 285.alb & 1 & 0 & Solution & 120.10 & 225 & 221.00 &  1.78\\
instance n=1000 286.alb & 1 & 0 & Solution & 120.21 & 226 & 221.00 &  2.21\\
instance n=1000 287.alb & 1 & 0 & Solution & 120.11 & 229 & 224.00 &  2.18\\
instance n=1000 288.alb & 1 & 0 & Solution & 120.14 & 224 & 219.00 &  2.23\\
instance n=1000 289.alb & 1 & 0 & Solution & 120.15 & 225 & 220.00 &  2.22\\
instance n=1000 29.alb & 1 & 0 & Solution & 120.09 & 544 & 317.00 & 41.73\\
instance n=1000 290.alb & 1 & 0 & Solution & 120.10 & 227 & 222.00 &  2.20\\
instance n=1000 291.alb & 1 & 0 & Solution & 120.11 & 230 & 225.00 &  2.17\\
instance n=1000 292.alb & 1 & 0 & Solution & 120.13 & 231 & 226.00 &  2.16\\
instance n=1000 293.alb & 1 & 0 & Solution & 120.10 & 231 & 225.00 &  2.60\\
instance n=1000 294.alb & 1 & 0 & Solution & 120.09 & 235 & 230.00 &  2.13\\
instance n=1000 295.alb & 1 & 0 & Solution & 120.14 & 233 & 227.00 &  2.58\\
instance n=1000 296.alb & 1 & 0 & Solution & 120.12 & 212 & 208.00 &  1.89\\
instance n=1000 297.alb & 1 & 0 & Solution & 120.13 & 221 & 217.00 &  1.81\\
instance n=1000 298.alb & 1 & 0 & Solution & 120.10 & 219 & 214.00 &  2.28\\
instance n=1000 299.alb & 1 & 0 & Solution & 120.12 & 231 & 226.00 &  2.16\\
instance n=1000 3.alb & 1 & 0 & Solution & 120.09 & 138 & 136.00 &  1.45\\
instance n=1000 30.alb & 1 & 0 & Solution & 120.11 & 570 & 314.00 & 44.91\\
instance n=1000 300.alb & 1 & 0 & Solution & 120.11 & 234 & 228.00 &  2.56\\
instance n=1000 301.alb & 1 & 0 & Solution & 120.11 & 138 & 137.00 &  0.72\\
instance n=1000 302.alb & 1 & 0 & Solution & 120.09 & 140 & 139.00 &  0.71\\
instance n=1000 303.alb & 1 & 0 & Solution & 120.09 & 140 & 138.00 &  1.43\\
instance n=1000 304.alb & 1 & 0 & Solution & 120.08 & 138 & 136.00 &  1.45\\
instance n=1000 305.alb & 1 & 0 & Solution & 120.11 & 141 & 140.00 &  0.71\\
instance n=1000 306.alb & 1 & 0 & Solution & 120.09 & 136 & 135.00 &  0.74\\
instance n=1000 307.alb & 1 & 0 & Solution & 120.10 & 137 & 136.00 &  0.73\\
instance n=1000 308.alb & 1 & 0 & Solution & 120.12 & 139 & 137.00 &  1.44\\
instance n=1000 309.alb & 1 & 0 & Solution & 120.10 & 136 & 135.00 &  0.74\\
instance n=1000 31.alb & 1 & 0 & Solution & 120.10 & 556 & 317.00 & 42.99\\
instance n=1000 310.alb & 1 & 0 & Solution & 120.10 & 143 & 141.00 &  1.40\\
instance n=1000 311.alb & 1 & 0 & Solution & 120.09 & 140 & 139.00 &  0.71\\
instance n=1000 312.alb & 1 & 0 & Solution & 120.09 & 136 & 135.00 &  0.74\\
instance n=1000 313.alb & 1 & 0 & Solution & 120.08 & 139 & 138.00 &  0.72\\
instance n=1000 314.alb & 1 & 0 & Solution & 120.09 & 143 & 142.00 &  0.70\\
instance n=1000 315.alb & 1 & 0 & Solution & 120.09 & 138 & 136.00 &  1.45\\
instance n=1000 316.alb & 1 & 0 & Solution & 120.10 & 138 & 137.00 &  0.72\\
instance n=1000 317.alb & 1 & 0 & Solution & 120.10 & 137 & 136.00 &  0.73\\
instance n=1000 318.alb & 1 & 0 & Solution & 120.09 & 139 & 138.00 &  0.72\\
instance n=1000 319.alb & 1 & 0 & Solution & 120.09 & 142 & 140.00 &  1.41\\
instance n=1000 32.alb & 1 & 0 & Solution & 120.11 & 554 & 318.00 & 42.60\\
instance n=1000 320.alb & 1 & 0 & Solution & 120.09 & 143 & 141.00 &  1.40\\
instance n=1000 321.alb & 1 & 0 & Solution & 120.07 & 141 & 140.00 &  0.71\\
instance n=1000 322.alb & 1 & 0 & Solution & 120.09 & 140 & 139.00 &  0.71\\
instance n=1000 323.alb & 1 & 0 & Solution & 120.09 & 139 & 138.00 &  0.72\\
instance n=1000 324.alb & 1 & 0 & Solution & 120.11 & 141 & 140.00 &  0.71\\
instance n=1000 325.alb & 1 & 0 & Solution & 120.12 & 139 & 138.00 &  0.72\\
instance n=1000 326.alb & 1 & 0 & Solution & 120.11 & 551 & 304.00 & 44.83\\
instance n=1000 327.alb & 1 & 0 & Solution & 120.12 & 564 & 325.00 & 42.38\\
instance n=1000 328.alb & 1 & 0 & Solution & 120.14 & 553 & 322.00 & 41.77\\
instance n=1000 329.alb & 1 & 0 & Solution & 120.11 & 565 & 324.00 & 42.65\\
instance n=1000 33.alb & 1 & 0 & Solution & 120.13 & 551 & 322.00 & 41.56\\
instance n=1000 330.alb & 1 & 0 & Solution & 120.12 & 545 & 319.00 & 41.47\\
instance n=1000 331.alb & 1 & 0 & Solution & 120.11 & 550 & 318.00 & 42.18\\
instance n=1000 332.alb & 1 & 0 & Solution & 120.10 & 543 & 323.00 & 40.52\\
instance n=1000 333.alb & 1 & 0 & Solution & 120.10 & 555 & 318.00 & 42.70\\
instance n=1000 334.alb & 1 & 0 & Solution & 120.11 & 544 & 332.00 & 38.97\\
instance n=1000 335.alb & 1 & 0 & Solution & 120.12 & 548 & 307.00 & 43.98\\
instance n=1000 336.alb & 1 & 0 & Solution & 120.16 & 544 & 328.00 & 39.71\\
instance n=1000 337.alb & 1 & 0 & Solution & 120.18 & 554 & 312.00 & 43.68\\
instance n=1000 338.alb & 1 & 0 & Solution & 120.14 & 554 & 320.00 & 42.24\\
instance n=1000 339.alb & 1 & 0 & Solution & 120.12 & 557 & 309.00 & 44.52\\
instance n=1000 34.alb & 1 & 0 & Solution & 120.11 & 575 & 325.00 & 43.48\\
instance n=1000 340.alb & 1 & 0 & Solution & 120.10 & 567 & 318.00 & 43.92\\
instance n=1000 341.alb & 1 & 0 & Solution & 120.17 & 555 & 313.00 & 43.60\\
instance n=1000 342.alb & 1 & 0 & Solution & 120.15 & 552 & 312.00 & 43.48\\
instance n=1000 343.alb & 1 & 0 & Solution & 120.13 & 552 & 328.00 & 40.58\\
instance n=1000 344.alb & 1 & 0 & Solution & 120.15 & 552 & 318.00 & 42.39\\
instance n=1000 345.alb & 1 & 0 & Solution & 120.11 & 560 & 315.00 & 43.75\\
instance n=1000 346.alb & 1 & 0 & Solution & 120.12 & 550 & 316.00 & 42.55\\
instance n=1000 347.alb & 1 & 0 & Solution & 120.18 & 549 & 316.00 & 42.44\\
instance n=1000 348.alb & 1 & 0 & Solution & 120.14 & 570 & 321.00 & 43.68\\
instance n=1000 349.alb & 1 & 0 & Solution & 120.13 & 559 & 335.00 & 40.07\\
instance n=1000 35.alb & 1 & 0 & Solution & 120.13 & 544 & 321.00 & 40.99\\
instance n=1000 350.alb & 1 & 0 & Solution & 120.14 & 539 & 307.00 & 43.04\\
instance n=1000 351.alb & 1 & 0 & Solution & 120.12 & 232 & 216.00 &  6.90\\
instance n=1000 352.alb & 1 & 0 & Solution & 120.10 & 231 & 208.00 &  9.96\\
instance n=1000 353.alb & 1 & 0 & Solution & 120.09 & 220 & 210.00 &  4.55\\
instance n=1000 354.alb & 1 & 0 & Solution & 120.11 & 226 & 212.00 &  6.19\\
instance n=1000 355.alb & 1 & 0 & Solution & 120.08 & 224 & 220.00 &  1.79\\
instance n=1000 356.alb & 1 & 0 & Solution & 120.10 & 230 & 221.00 &  3.91\\
instance n=1000 357.alb & 1 & 0 & Solution & 120.10 & 216 & 121.00 & 43.98\\
instance n=1000 358.alb & 1 & 0 & Solution & 120.09 & 223 & 218.00 &  2.24\\
instance n=1000 359.alb & 1 & 0 & Solution & 120.10 & 226 & 215.00 &  4.87\\
instance n=1000 36.alb & 1 & 0 & Solution & 120.12 & 547 & 316.00 & 42.23\\
instance n=1000 360.alb & 1 & 0 & Solution & 120.11 & 233 & 215.00 &  7.73\\
instance n=1000 361.alb & 1 & 0 & Solution & 120.11 & 219 & 215.00 &  1.83\\
instance n=1000 362.alb & 1 & 0 & Solution & 120.11 & 227 & 204.00 & 10.13\\
instance n=1000 363.alb & 1 & 0 & Solution & 120.09 & 219 & 206.00 &  5.94\\
instance n=1000 364.alb & 1 & 0 & Solution & 120.12 & 224 & 220.00 &  1.79\\
instance n=1000 365.alb & 1 & 0 & Solution & 120.11 & 231 & 217.00 &  6.06\\
instance n=1000 366.alb & 1 & 0 & Solution & 120.10 & 231 & 214.00 &  7.36\\
instance n=1000 367.alb & 1 & 0 & Solution & 120.13 & 231 & 217.00 &  6.06\\
instance n=1000 368.alb & 1 & 0 & Solution & 120.11 & 230 & 219.00 &  4.78\\
instance n=1000 369.alb & 1 & 0 & Solution & 120.11 & 224 & 181.00 & 19.20\\
instance n=1000 37.alb & 1 & 0 & Solution & 120.14 & 566 & 314.00 & 44.52\\
instance n=1000 370.alb & 1 & 0 & Solution & 120.10 & 227 & 210.00 &  7.49\\
instance n=1000 371.alb & 1 & 0 & Solution & 120.15 & 223 & 215.00 &  3.59\\
instance n=1000 372.alb & 1 & 0 & Solution & 120.10 & 234 & 208.00 & 11.11\\
instance n=1000 373.alb & 1 & 0 & Solution & 120.09 & 222 & 215.00 &  3.15\\
instance n=1000 374.alb & 1 & 0 & Solution & 120.10 & 222 & 219.00 &  1.35\\
instance n=1000 375.alb & 1 & 0 & Solution & 120.11 & 230 & 214.00 &  6.96\\
instance n=1000 376.alb & 1 & 0 & Solution & 120.09 & 134 & 132.00 &  1.49\\
instance n=1000 377.alb & 1 & 0 & Solution & 120.09 & 138 & 137.00 &  0.72\\
instance n=1000 378.alb & 1 & 0 & Solution & 120.10 & 135 & 134.00 &  0.74\\
instance n=1000 379.alb & 1 & 0 & Solution & 120.11 & 139 & 137.00 &  1.44\\
instance n=1000 38.alb & 1 & 0 & Solution & 120.12 & 564 & 309.00 & 45.21\\
instance n=1000 380.alb & 1 & 0 & Solution & 120.11 & 136 & 134.00 &  1.47\\
instance n=1000 381.alb & 1 & 0 & Solution & 120.10 & 140 & 138.00 &  1.43\\
instance n=1000 382.alb & 1 & 0 & Solution & 120.09 & 132 & 131.00 &  0.76\\
instance n=1000 383.alb & 1 & 0 & Solution & 120.09 & 140 & 138.00 &  1.43\\
instance n=1000 384.alb & 1 & 0 & Solution & 120.10 & 141 & 139.00 &  1.42\\
instance n=1000 385.alb & 1 & 0 & Solution & 120.10 & 137 & 135.00 &  1.46\\
instance n=1000 386.alb & 1 & 0 & Solution & 120.12 & 141 & 139.00 &  1.42\\
instance n=1000 387.alb & 1 & 0 & Solution & 120.11 & 139 & 137.00 &  1.44\\
instance n=1000 388.alb & 1 & 0 & Solution & 120.10 & 138 & 137.00 &  0.72\\
instance n=1000 389.alb & 1 & 0 & Solution & 120.11 & 137 & 136.00 &  0.73\\
instance n=1000 39.alb & 1 & 0 & Solution & 120.12 & 565 & 313.00 & 44.60\\
instance n=1000 390.alb & 1 & 0 & Solution & 120.11 & 137 & 136.00 &  0.73\\
instance n=1000 391.alb & 1 & 0 & Solution & 120.11 & 137 & 135.00 &  1.46\\
instance n=1000 392.alb & 1 & 0 & Solution & 120.11 & 137 & 136.00 &  0.73\\
instance n=1000 393.alb & 1 & 0 & Solution & 120.10 & 137 & 136.00 &  0.73\\
instance n=1000 394.alb & 1 & 0 & Solution & 120.10 & 140 & 138.00 &  1.43\\
instance n=1000 395.alb & 1 & 0 & Solution & 120.10 & 141 & 139.00 &  1.42\\
instance n=1000 396.alb & 1 & 0 & Solution & 120.10 & 138 & 136.00 &  1.45\\
instance n=1000 397.alb & 1 & 0 & Solution & 120.13 & 142 & 140.00 &  1.41\\
instance n=1000 398.alb & 1 & 0 & Solution & 120.11 & 136 & 134.00 &  1.47\\
instance n=1000 399.alb & 1 & 0 & Solution & 120.11 & 141 & 139.00 &  1.42\\
instance n=1000 4.alb & 1 & 0 & Solution & 120.10 & 139 & 138.00 &  0.72\\
instance n=1000 40.alb & 1 & 0 & Solution & 120.11 & 529 & 318.00 & 39.89\\
instance n=1000 400.alb & 1 & 0 & Solution & 120.11 & 142 & 140.00 &  1.41\\
instance n=1000 401.alb & 1 & 0 & Solution & 120.11 & 553 & 413.00 & 25.32\\
instance n=1000 402.alb & 1 & 0 & Solution & 120.19 & 556 & 424.00 & 23.74\\
instance n=1000 403.alb & 1 & 0 & Solution & 120.18 & 557 & 396.00 & 28.90\\
instance n=1000 404.alb & 1 & 0 & Solution & 120.18 & 554 & 430.00 & 22.38\\
instance n=1000 405.alb & 1 & 0 & Solution & 120.17 & 562 & 458.00 & 18.51\\
instance n=1000 406.alb & 1 & 0 & Solution & 120.16 & 547 & 403.00 & 26.33\\
instance n=1000 407.alb & 1 & 0 & Solution & 120.19 & 555 & 399.00 & 28.11\\
instance n=1000 408.alb & 1 & 0 & Solution & 120.18 & 563 & 412.00 & 26.82\\
instance n=1000 409.alb & 1 & 0 & Solution & 120.16 & 566 & 413.00 & 27.03\\
instance n=1000 41.alb & 1 & 0 & Solution & 120.13 & 555 & 336.00 & 39.46\\
instance n=1000 410.alb & 1 & 0 & Solution & 120.18 & 575 & 431.00 & 25.04\\
instance n=1000 411.alb & 1 & 0 & Solution & 120.22 & 558 & 422.00 & 24.37\\
instance n=1000 412.alb & 1 & 0 & Solution & 120.15 & 558 & 393.00 & 29.57\\
instance n=1000 413.alb & 1 & 0 & Solution & 120.19 & 558 & 411.00 & 26.34\\
instance n=1000 414.alb & 1 & 0 & Solution & 120.18 & 562 & 406.00 & 27.76\\
instance n=1000 415.alb & 1 & 0 & Solution & 120.17 & 561 & 413.00 & 26.38\\
instance n=1000 416.alb & 1 & 0 & Solution & 120.19 & 562 & 398.00 & 29.18\\
instance n=1000 417.alb & 1 & 0 & Solution & 120.12 & 594 & 406.00 & 31.65\\
instance n=1000 418.alb & 1 & 0 & Solution & 120.16 & 552 & 438.00 & 20.65\\
instance n=1000 419.alb & 1 & 0 & Solution & 120.19 & 577 & 423.00 & 26.69\\
instance n=1000 42.alb & 1 & 0 & Solution & 120.12 & 534 & 306.00 & 42.70\\
instance n=1000 420.alb & 1 & 0 & Solution & 120.16 & 556 & 429.00 & 22.84\\
instance n=1000 421.alb & 1 & 0 & Solution & 120.18 & 556 & 402.00 & 27.70\\
instance n=1000 422.alb & 1 & 0 & Solution & 120.11 & 552 & 420.00 & 23.91\\
instance n=1000 423.alb & 1 & 0 & Solution & 120.11 & 561 & 396.00 & 29.41\\
instance n=1000 424.alb & 1 & 0 & Solution & 120.10 & 548 & 431.00 & 21.35\\
instance n=1000 425.alb & 1 & 0 & Solution & 120.19 & 567 & 395.00 & 30.34\\
instance n=1000 426.alb & 1 & 0 & Solution & 120.10 & 229 & 224.00 &  2.18\\
instance n=1000 427.alb & 1 & 0 & Solution & 120.12 & 234 & 229.00 &  2.14\\
instance n=1000 428.alb & 1 & 0 & Solution & 120.12 & 228 & 224.00 &  1.75\\
instance n=1000 429.alb & 1 & 0 & Solution & 120.11 & 239 & 235.00 &  1.67\\
instance n=1000 43.alb & 1 & 0 & Solution & 120.13 & 541 & 325.00 & 39.93\\
instance n=1000 430.alb & 1 & 0 & Solution & 120.10 & 224 & 220.00 &  1.79\\
instance n=1000 431.alb & 1 & 0 & Solution & 120.13 & 234 & 230.00 &  1.71\\
instance n=1000 432.alb & 1 & 0 & Solution & 120.13 & 232 & 227.00 &  2.16\\
instance n=1000 433.alb & 1 & 0 & Solution & 120.10 & 234 & 229.00 &  2.14\\
instance n=1000 434.alb & 1 & 0 & Solution & 120.10 & 215 & 212.00 &  1.40\\
instance n=1000 435.alb & 1 & 0 & Solution & 120.10 & 231 & 227.00 &  1.73\\
instance n=1000 436.alb & 1 & 0 & Solution & 120.12 & 231 & 226.00 &  2.16\\
instance n=1000 437.alb & 1 & 0 & Solution & 120.12 & 226 & 222.00 &  1.77\\
instance n=1000 438.alb & 1 & 0 & Solution & 120.14 & 225 & 221.00 &  1.78\\
instance n=1000 439.alb & 1 & 0 & Solution & 120.13 & 230 & 225.00 &  2.17\\
instance n=1000 44.alb & 1 & 0 & Solution & 120.12 & 554 & 313.00 & 43.50\\
instance n=1000 440.alb & 1 & 0 & Solution & 120.12 & 230 & 225.00 &  2.17\\
instance n=1000 441.alb & 1 & 0 & Solution & 120.11 & 226 & 221.00 &  2.21\\
instance n=1000 442.alb & 1 & 0 & Solution & 120.12 & 235 & 230.00 &  2.13\\
instance n=1000 443.alb & 1 & 0 & Solution & 120.11 & 221 & 217.00 &  1.81\\
instance n=1000 444.alb & 1 & 0 & Solution & 120.12 & 227 & 222.00 &  2.20\\
instance n=1000 445.alb & 1 & 0 & Solution & 120.10 & 235 & 229.00 &  2.55\\
instance n=1000 446.alb & 1 & 0 & Solution & 120.12 & 232 & 228.00 &  1.72\\
instance n=1000 447.alb & 1 & 0 & Solution & 120.11 & 226 & 221.00 &  2.21\\
instance n=1000 448.alb & 1 & 0 & Solution & 120.11 & 226 & 222.00 &  1.77\\
instance n=1000 449.alb & 1 & 0 & Solution & 120.11 & 238 & 232.00 &  2.52\\
instance n=1000 45.alb & 1 & 0 & Solution & 120.14 & 534 & 318.00 & 40.45\\
instance n=1000 450.alb & 1 & 0 & Solution & 120.10 & 224 & 220.00 &  1.79\\
instance n=1000 451.alb & 1 & 0 & Solution & 120.12 & 139 & 136.00 &  2.16\\
instance n=1000 452.alb & 1 & 0 & Solution & 120.09 & 134 & 132.00 &  1.49\\
instance n=1000 453.alb & 1 & 0 & Solution & 120.10 & 140 & 138.00 &  1.43\\
instance n=1000 454.alb & 1 & 0 & Solution & 120.11 & 142 & 139.00 &  2.11\\
instance n=1000 455.alb & 1 & 0 & Solution & 120.09 & 139 & 136.00 &  2.16\\
instance n=1000 456.alb & 1 & 0 & Solution & 120.10 & 137 & 135.00 &  1.46\\
instance n=1000 457.alb & 1 & 0 & Solution & 120.10 & 139 & 137.00 &  1.44\\
instance n=1000 458.alb & 1 & 0 & Solution & 120.10 & 137 & 135.00 &  1.46\\
instance n=1000 459.alb & 1 & 0 & Solution & 120.10 & 140 & 137.00 &  2.14\\
instance n=1000 46.alb & 1 & 0 & Solution & 120.09 & 545 & 314.00 & 42.39\\
instance n=1000 460.alb & 1 & 0 & Solution & 120.12 & 140 & 138.00 &  1.43\\
instance n=1000 461.alb & 1 & 0 & Solution & 120.11 & 139 & 137.00 &  1.44\\
instance n=1000 462.alb & 1 & 0 & Solution & 120.10 & 138 & 136.00 &  1.45\\
instance n=1000 463.alb & 1 & 0 & Solution & 120.10 & 138 & 136.00 &  1.45\\
instance n=1000 464.alb & 1 & 0 & Solution & 120.10 & 141 & 138.00 &  2.13\\
instance n=1000 465.alb & 1 & 0 & Solution & 120.10 & 141 & 138.00 &  2.13\\
instance n=1000 466.alb & 1 & 0 & Solution & 120.09 & 136 & 133.00 &  2.21\\
instance n=1000 467.alb & 1 & 0 & Solution & 120.12 & 140 & 138.00 &  1.43\\
instance n=1000 468.alb & 1 & 0 & Solution & 120.10 & 139 & 137.00 &  1.44\\
instance n=1000 469.alb & 1 & 0 & Solution & 120.11 & 139 & 137.00 &  1.44\\
instance n=1000 47.alb & 1 & 0 & Solution & 120.08 & 547 & 303.00 & 44.61\\
instance n=1000 470.alb & 1 & 0 & Solution & 120.11 & 137 & 135.00 &  1.46\\
instance n=1000 471.alb & 1 & 0 & Solution & 120.10 & 138 & 135.00 &  2.17\\
instance n=1000 472.alb & 1 & 0 & Solution & 120.10 & 142 & 140.00 &  1.41\\
instance n=1000 473.alb & 1 & 0 & Solution & 120.10 & 138 & 135.00 &  2.17\\
instance n=1000 474.alb & 1 & 0 & Solution & 120.08 & 139 & 136.00 &  2.16\\
instance n=1000 475.alb & 1 & 0 & Solution & 120.10 & 138 & 136.00 &  1.45\\
instance n=1000 476.alb & 1 & 0 & Solution & 120.13 & 575 & 494.00 & 14.09\\
instance n=1000 477.alb & 1 & 0 & Solution & 120.13 & 585 & 524.00 & 10.43\\
instance n=1000 478.alb & 1 & 0 & Solution & 120.14 & 594 & 545.00 &  8.25\\
instance n=1000 479.alb & 1 & 0 & Solution & 120.16 & 577 & 490.00 & 15.08\\
instance n=1000 48.alb & 1 & 0 & Solution & 120.13 & 573 & 329.00 & 42.58\\
instance n=1000 480.alb & 1 & 0 & Solution & 120.12 & 566 & 507.00 & 10.42\\
instance n=1000 481.alb & 1 & 0 & Solution & 120.22 & 580 & 519.00 & 10.52\\
instance n=1000 482.alb & 1 & 0 & Solution & 120.20 & 603 & 498.00 & 17.41\\
instance n=1000 483.alb & 1 & 0 & Solution & 120.19 & 571 & 502.00 & 12.08\\
instance n=1000 484.alb & 1 & 0 & Solution & 120.26 & 588 & 512.00 & 12.93\\
instance n=1000 485.alb & 1 & 0 & Solution & 120.12 & 584 & 518.00 & 11.30\\
instance n=1000 486.alb & 1 & 0 & Solution & 120.12 & 575 & 504.00 & 12.35\\
instance n=1000 487.alb & 1 & 0 & Solution & 120.20 & 582 & 492.00 & 15.46\\
instance n=1000 488.alb & 1 & 0 & Solution & 120.13 & 575 & 511.00 & 11.13\\
instance n=1000 489.alb & 1 & 0 & Solution & 120.13 & 568 & 487.00 & 14.26\\
instance n=1000 49.alb & 1 & 0 & Solution & 120.11 & 546 & 323.00 & 40.84\\
instance n=1000 490.alb & 1 & 0 & Solution & 120.20 & 576 & 499.00 & 13.37\\
instance n=1000 491.alb & 1 & 0 & Solution & 120.23 & 571 & 495.00 & 13.31\\
instance n=1000 492.alb & 1 & 0 & Solution & 120.21 & 592 & 515.00 & 13.01\\
instance n=1000 493.alb & 1 & 0 & Solution & 120.12 & 564 & 498.00 & 11.70\\
instance n=1000 494.alb & 1 & 0 & Solution & 120.29 & 579 & 515.00 & 11.05\\
instance n=1000 495.alb & 1 & 0 & Solution & 120.24 & 595 & 508.00 & 14.62\\
instance n=1000 496.alb & 1 & 0 & Solution & 120.11 & 563 & 505.00 & 10.30\\
instance n=1000 497.alb & 1 & 0 & Solution & 120.22 & 569 & 499.00 & 12.30\\
instance n=1000 498.alb & 1 & 0 & Solution & 120.13 & 585 & 523.00 & 10.60\\
instance n=1000 499.alb & 1 & 0 & Solution & 120.12 & 567 & 505.00 & 10.93\\
instance n=1000 5.alb & 1 & 0 & Solution & 120.08 & 136 & 135.00 &  0.74\\
instance n=1000 50.alb & 1 & 0 & Solution & 120.10 & 535 & 303.00 & 43.36\\
instance n=1000 500.alb & 1 & 0 & Solution & 120.22 & 584 & 507.00 & 13.18\\
instance n=1000 501.alb & 1 & 0 & Solution & 120.10 & 233 & 227.00 &  2.58\\
instance n=1000 502.alb & 1 & 0 & Solution & 120.11 & 229 & 224.00 &  2.18\\
instance n=1000 503.alb & 1 & 0 & Solution & 120.16 & 232 & 225.00 &  3.02\\
instance n=1000 504.alb & 1 & 0 & Solution & 120.11 & 233 & 227.00 &  2.58\\
instance n=1000 505.alb & 1 & 0 & Solution & 120.11 & 219 & 213.00 &  2.74\\
instance n=1000 506.alb & 1 & 0 & Solution & 120.10 & 229 & 223.00 &  2.62\\
instance n=1000 507.alb & 1 & 0 & Solution & 120.11 & 226 & 220.00 &  2.65\\
instance n=1000 508.alb & 1 & 0 & Solution & 120.11 & 224 & 219.00 &  2.23\\
instance n=1000 509.alb & 1 & 0 & Solution & 120.15 & 231 & 225.00 &  2.60\\
instance n=1000 51.alb & 1 & 0 & Solution & 120.09 & 230 & 226.00 &  1.74\\
instance n=1000 510.alb & 1 & 0 & Solution & 120.11 & 233 & 226.00 &  3.00\\
instance n=1000 511.alb & 1 & 0 & Solution & 120.11 & 237 & 230.00 &  2.95\\
instance n=1000 512.alb & 1 & 0 & Solution & 120.11 & 224 & 219.00 &  2.23\\
instance n=1000 513.alb & 1 & 0 & Solution & 120.16 & 226 & 219.00 &  3.10\\
instance n=1000 514.alb & 1 & 0 & Solution & 120.13 & 233 & 226.00 &  3.00\\
instance n=1000 515.alb & 1 & 0 & Solution & 120.11 & 228 & 221.00 &  3.07\\
instance n=1000 516.alb & 1 & 0 & Solution & 120.11 & 235 & 229.00 &  2.55\\
instance n=1000 517.alb & 1 & 0 & Solution & 120.11 & 227 & 221.00 &  2.64\\
instance n=1000 518.alb & 1 & 0 & Solution & 120.12 & 226 & 220.00 &  2.65\\
instance n=1000 519.alb & 1 & 0 & Solution & 120.11 & 228 & 221.00 &  3.07\\
instance n=1000 52.alb & 1 & 0 & Solution & 120.15 & 232 & 197.00 & 15.09\\
instance n=1000 520.alb & 1 & 0 & Solution & 120.12 & 232 & 226.00 &  2.59\\
instance n=1000 521.alb & 1 & 0 & Solution & 120.14 & 236 & 229.00 &  2.97\\
instance n=1000 522.alb & 1 & 0 & Solution & 120.11 & 221 & 215.00 &  2.71\\
instance n=1000 523.alb & 1 & 0 & Solution & 120.12 & 226 & 220.00 &  2.65\\
instance n=1000 524.alb & 1 & 0 & Solution & 120.11 & 233 & 226.00 &  3.00\\
instance n=1000 525.alb & 1 & 0 & Solution & 120.15 & 227 & 221.00 &  2.64\\
instance n=1000 53.alb & 1 & 0 & Solution & 120.10 & 231 & 209.00 &  9.52\\
instance n=1000 54.alb & 1 & 0 & Solution & 120.14 & 223 & 203.00 &  8.97\\
instance n=1000 55.alb & 1 & 0 & Solution & 120.08 & 221 & 212.00 &  4.07\\
instance n=1000 56.alb & 1 & 0 & Solution & 120.09 & 231 & 198.00 & 14.29\\
instance n=1000 57.alb & 1 & 0 & Solution & 120.09 & 227 & 196.00 & 13.66\\
instance n=1000 58.alb & 1 & 0 & Solution & 120.08 & 227 & 200.00 & 11.89\\
instance n=1000 59.alb & 1 & 0 & Solution & 120.09 & 226 & 204.00 &  9.73\\
instance n=1000 6.alb & 1 & 0 & Solution & 120.08 & 143 & 141.00 &  1.40\\
instance n=1000 60.alb & 1 & 0 & Solution & 120.10 & 234 & 215.00 &  8.12\\
instance n=1000 61.alb & 1 & 0 & Solution & 120.09 & 233 & 196.00 & 15.88\\
instance n=1000 62.alb & 1 & 0 & Solution & 120.08 & 226 & 197.00 & 12.83\\
instance n=1000 63.alb & 1 & 0 & Solution & 120.08 & 230 & 197.00 & 14.35\\
instance n=1000 64.alb & 1 & 0 & Solution & 120.09 & 233 & 209.00 & 10.30\\
instance n=1000 65.alb & 1 & 0 & Solution & 120.11 & 228 & 210.00 &  7.89\\
instance n=1000 66.alb & 1 & 0 & Solution & 120.10 & 230 & 224.00 &  2.61\\
instance n=1000 67.alb & 1 & 0 & Solution & 120.10 & 227 & 192.00 & 15.42\\
instance n=1000 68.alb & 1 & 0 & Solution & 120.08 & 231 & 201.00 & 12.99\\
instance n=1000 69.alb & 1 & 0 & Solution & 120.10 & 227 & 206.00 &  9.25\\
instance n=1000 7.alb & 1 & 0 & Solution & 120.07 & 138 & 136.00 &  1.45\\
instance n=1000 70.alb & 1 & 0 & Solution & 120.11 & 232 & 203.00 & 12.50\\
instance n=1000 71.alb & 1 & 0 & Solution & 120.10 & 233 & 188.00 & 19.31\\
instance n=1000 72.alb & 1 & 0 & Solution & 120.08 & 226 & 206.00 &  8.85\\
instance n=1000 73.alb & 1 & 0 & Solution & 120.10 & 225 & 212.00 &  5.78\\
instance n=1000 74.alb & 1 & 0 & Solution & 120.08 & 231 & 218.00 &  5.63\\
instance n=1000 75.alb & 1 & 0 & Solution & 120.09 & 231 & 222.00 &  3.90\\
instance n=1000 76.alb & 1 & 0 & Solution & 120.07 & 138 & 136.00 &  1.45\\
instance n=1000 77.alb & 1 & 0 & Solution & 120.08 & 137 & 136.00 &  0.73\\
instance n=1000 78.alb & 1 & 0 & Solution & 120.07 & 140 & 138.00 &  1.43\\
instance n=1000 79.alb & 1 & 0 & Solution & 120.09 & 143 & 142.00 &  0.70\\
instance n=1000 8.alb & 1 & 0 & Solution & 120.08 & 140 & 138.00 &  1.43\\
instance n=1000 80.alb & 1 & 0 & Solution & 120.09 & 141 & 140.00 &  0.71\\
instance n=1000 81.alb & 1 & 0 & Solution & 120.08 & 137 & 136.00 &  0.73\\
instance n=1000 82.alb & 1 & 0 & Solution & 120.11 & 137 & 136.00 &  0.73\\
instance n=1000 83.alb & 1 & 0 & Solution & 120.92 & 141 & 140.00 &  0.71\\
instance n=1000 84.alb & 1 & 0 & Solution & 120.14 & 136 & 135.00 &  0.74\\
instance n=1000 85.alb & 1 & 0 & Solution & 120.10 & 137 & 136.00 &  0.73\\
instance n=1000 86.alb & 1 & 0 & Solution & 120.08 & 139 & 138.00 &  0.72\\
instance n=1000 87.alb & 1 & 0 & Solution & 120.09 & 142 & 140.00 &  1.41\\
instance n=1000 88.alb & 1 & 0 & Solution & 120.09 & 142 & 140.00 &  1.41\\
instance n=1000 89.alb & 1 & 0 & Solution & 120.10 & 141 & 140.00 &  0.71\\
instance n=1000 9.alb & 1 & 0 & Solution & 120.07 & 136 & 134.00 &  1.47\\
instance n=1000 90.alb & 1 & 0 & Solution & 120.10 & 139 & 138.00 &  0.72\\
instance n=1000 91.alb & 1 & 0 & Solution & 120.09 & 142 & 141.00 &  0.70\\
instance n=1000 92.alb & 1 & 0 & Solution & 120.10 & 137 & 136.00 &  0.73\\
instance n=1000 93.alb & 1 & 0 & Solution & 120.08 & 138 & 137.00 &  0.72\\
instance n=1000 94.alb & 1 & 0 & Solution & 120.09 & 139 & 137.00 &  1.44\\
instance n=1000 95.alb & 1 & 0 & Solution & 120.10 & 137 & 136.00 &  0.73\\
instance n=1000 96.alb & 1 & 0 & Solution & 120.07 & 139 & 137.00 &  1.44\\
instance n=1000 97.alb & 1 & 0 & Solution & 120.10 & 140 & 138.00 &  1.43\\
instance n=1000 98.alb & 1 & 0 & Solution & 120.12 & 137 & 136.00 &  0.73\\
instance n=1000 99.alb & 1 & 0 & Solution & 120.09 & 137 & 136.00 &  0.73\\
instance n=100 1.alb & 1 & 0 & Optimal & 17.05 & 23 & 23.00 &  0.00\\
instance n=100 10.alb & 1 & 0 & Optimal &  1.16 & 22 & 22.00 &  0.00\\
instance n=100 100.alb & 1 & 0 & Optimal & 120.03 & 25 & 25.00 &  0.00\\
instance n=100 101.alb & 1 & 0 & Optimal & 120.03 & 15 & 15.00 &  0.00\\
instance n=100 102.alb & 1 & 0 & Optimal &  0.39 & 14 & 14.00 &  0.00\\
instance n=100 103.alb & 1 & 0 & Optimal &  0.14 & 14 & 14.00 &  0.00\\
instance n=100 104.alb & 1 & 0 & Optimal &  0.09 & 14 & 14.00 &  0.00\\
instance n=100 105.alb & 1 & 0 & Optimal &  0.52 & 13 & 13.00 &  0.00\\
instance n=100 106.alb & 1 & 0 & Optimal &  0.15 & 14 & 14.00 &  0.00\\
instance n=100 107.alb & 1 & 0 & Optimal &  0.09 & 14 & 14.00 &  0.00\\
instance n=100 108.alb & 1 & 0 & Optimal & 120.02 & 14 & 14.00 &  0.00\\
instance n=100 109.alb & 1 & 0 & Optimal &  0.13 & 15 & 15.00 &  0.00\\
instance n=100 11.alb & 1 & 0 & Optimal & 12.16 & 24 & 24.00 &  0.00\\
instance n=100 110.alb & 1 & 0 & Optimal &  0.13 & 13 & 13.00 &  0.00\\
instance n=100 111.alb & 1 & 0 & Optimal &  0.11 & 16 & 16.00 &  0.00\\
instance n=100 112.alb & 1 & 0 & Optimal & 20.93 & 13 & 13.00 &  0.00\\
instance n=100 113.alb & 1 & 0 & Optimal &  0.39 & 14 & 14.00 &  0.00\\
instance n=100 114.alb & 1 & 0 & Optimal &  0.12 & 13 & 13.00 &  0.00\\
instance n=100 115.alb & 1 & 0 & Optimal & 120.01 & 14 & 14.00 &  0.00\\
instance n=100 116.alb & 1 & 0 & Optimal &  0.13 & 16 & 16.00 &  0.00\\
instance n=100 117.alb & 1 & 0 & Optimal & 120.03 & 15 & 15.00 &  0.00\\
instance n=100 118.alb & 1 & 0 & Optimal &  0.30 & 15 & 15.00 &  0.00\\
instance n=100 119.alb & 1 & 0 & Optimal &  0.11 & 14 & 14.00 &  0.00\\
instance n=100 12.alb & 1 & 0 & Optimal & 66.54 & 25 & 25.00 &  0.00\\
instance n=100 120.alb & 1 & 0 & Optimal &  0.12 & 14 & 14.00 &  0.00\\
instance n=100 121.alb & 1 & 0 & Optimal &  0.14 & 15 & 15.00 &  0.00\\
instance n=100 122.alb & 1 & 0 & Optimal &  0.26 & 13 & 13.00 &  0.00\\
instance n=100 123.alb & 1 & 0 & Optimal &  0.13 & 15 & 15.00 &  0.00\\
instance n=100 124.alb & 1 & 0 & Optimal & 120.01 & 15 & 15.00 &  0.00\\
instance n=100 125.alb & 1 & 0 & Optimal &  0.12 & 14 & 14.00 &  0.00\\
instance n=100 126.alb & 1 & 0 & Solution & 120.12 & 51 & 50.00 &  1.96\\
instance n=100 127.alb & 1 & 0 & Solution & 120.33 & 52 & 50.00 &  3.85\\
instance n=100 128.alb & 1 & 0 & Solution & 120.11 & 57 & 56.00 &  1.75\\
instance n=100 129.alb & 1 & 0 & Optimal &  1.96 & 54 & 54.00 &  0.00\\
instance n=100 13.alb & 1 & 0 & Optimal &  0.33 & 24 & 24.00 &  0.00\\
instance n=100 130.alb & 1 & 0 & Solution & 120.12 & 55 & 52.00 &  5.45\\
instance n=100 131.alb & 1 & 0 & Solution & 120.13 & 53 & 51.00 &  3.77\\
instance n=100 132.alb & 1 & 0 & Solution & 120.27 & 58 & 56.00 &  3.45\\
instance n=100 133.alb & 1 & 0 & Solution & 120.20 & 55 & 53.00 &  3.64\\
instance n=100 134.alb & 1 & 0 & Solution & 120.13 & 54 & 52.00 &  3.70\\
instance n=100 135.alb & 1 & 0 & Solution & 120.13 & 55 & 53.00 &  3.64\\
instance n=100 136.alb & 1 & 0 & Solution & 120.09 & 52 & 50.00 &  3.85\\
instance n=100 137.alb & 1 & 0 & Solution & 120.22 & 54 & 51.00 &  5.56\\
instance n=100 138.alb & 1 & 0 & Optimal &  9.65 & 56 & 56.00 &  0.00\\
instance n=100 139.alb & 1 & 0 & Optimal & 120.02 & 51 & 51.00 &  0.00\\
instance n=100 14.alb & 1 & 0 & Optimal & 120.03 & 20 & 20.00 &  0.00\\
instance n=100 140.alb & 1 & 0 & Solution & 120.45 & 55 & 54.00 &  1.82\\
instance n=100 141.alb & 1 & 0 & Solution & 120.24 & 51 & 49.00 &  3.92\\
instance n=100 142.alb & 1 & 0 & Solution & 120.15 & 55 & 52.00 &  5.45\\
instance n=100 143.alb & 1 & 0 & Solution & 120.10 & 53 & 51.00 &  3.77\\
instance n=100 144.alb & 1 & 0 & Solution & 120.09 & 49 & 47.00 &  4.08\\
instance n=100 145.alb & 1 & 0 & Solution & 120.26 & 56 & 53.00 &  5.36\\
instance n=100 146.alb & 1 & 0 & Optimal &  3.64 & 53 & 53.00 &  0.00\\
instance n=100 147.alb & 1 & 0 & Solution & 120.11 & 59 & 58.00 &  1.69\\
instance n=100 148.alb & 1 & 0 & Solution & 120.12 & 52 & 50.00 &  3.85\\
instance n=100 149.alb & 1 & 0 & Solution & 120.12 & 55 & 54.00 &  1.82\\
instance n=100 15.alb & 1 & 0 & Optimal &  0.08 & 24 & 24.00 &  0.00\\
instance n=100 150.alb & 1 & 0 & Solution & 120.13 & 57 & 54.00 &  5.26\\
instance n=100 151.alb & 1 & 0 & Solution & 120.10 & 22 & 21.00 &  4.55\\
instance n=100 152.alb & 1 & 0 & Optimal &  0.58 & 22 & 22.00 &  0.00\\
instance n=100 153.alb & 1 & 0 & Optimal & 120.02 & 21 & 21.00 &  0.00\\
instance n=100 154.alb & 1 & 0 & Optimal &  0.15 & 25 & 25.00 &  0.00\\
instance n=100 155.alb & 1 & 0 & Optimal &  0.78 & 22 & 22.00 &  0.00\\
instance n=100 156.alb & 1 & 0 & Optimal &  0.63 & 23 & 23.00 &  0.00\\
instance n=100 157.alb & 1 & 0 & Optimal &  0.79 & 26 & 26.00 &  0.00\\
instance n=100 158.alb & 1 & 0 & Optimal &  0.39 & 23 & 23.00 &  0.00\\
instance n=100 159.alb & 1 & 0 & Optimal &  0.14 & 19 & 19.00 &  0.00\\
instance n=100 16.alb & 1 & 0 & Optimal & 120.03 & 23 & 23.00 &  0.00\\
instance n=100 160.alb & 1 & 0 & Optimal &  0.68 & 22 & 22.00 &  0.00\\
instance n=100 161.alb & 1 & 0 & Solution & 120.11 & 23 & 22.00 &  4.35\\
instance n=100 162.alb & 1 & 0 & Optimal & 120.04 & 22 & 22.00 &  0.00\\
instance n=100 163.alb & 1 & 0 & Optimal &  0.15 & 25 & 25.00 &  0.00\\
instance n=100 164.alb & 1 & 0 & Optimal &  0.10 & 23 & 23.00 &  0.00\\
instance n=100 165.alb & 1 & 0 & Solution & 120.10 & 25 & 24.00 &  4.00\\
instance n=100 166.alb & 1 & 0 & Optimal &  0.59 & 24 & 24.00 &  0.00\\
instance n=100 167.alb & 1 & 0 & Optimal &  0.13 & 22 & 22.00 &  0.00\\
instance n=100 168.alb & 1 & 0 & Optimal & 120.03 & 21 & 21.00 &  0.00\\
instance n=100 169.alb & 1 & 0 & Optimal &  0.61 & 21 & 21.00 &  0.00\\
instance n=100 17.alb & 1 & 0 & Solution & 120.08 & 22 & 21.00 &  4.55\\
instance n=100 170.alb & 1 & 0 & Optimal & 24.40 & 24 & 24.00 &  0.00\\
instance n=100 171.alb & 1 & 0 & Solution & 120.12 & 25 & 24.00 &  4.00\\
instance n=100 172.alb & 1 & 0 & Optimal &  0.66 & 24 & 24.00 &  0.00\\
instance n=100 173.alb & 1 & 0 & Solution & 120.12 & 25 & 24.00 &  4.00\\
instance n=100 174.alb & 1 & 0 & Optimal & 120.03 & 22 & 22.00 &  0.00\\
instance n=100 175.alb & 1 & 0 & Solution & 120.11 & 27 & 26.00 &  3.70\\
instance n=100 176.alb & 1 & 0 & Optimal &  0.11 & 13 & 13.00 &  0.00\\
instance n=100 177.alb & 1 & 0 & Optimal & 120.03 & 14 & 14.00 &  0.00\\
instance n=100 178.alb & 1 & 0 & Optimal & 120.02 & 15 & 15.00 &  0.00\\
instance n=100 179.alb & 1 & 0 & Optimal &  0.12 & 15 & 15.00 &  0.00\\
instance n=100 18.alb & 1 & 0 & Solution & 120.09 & 20 & 19.00 &  5.00\\
instance n=100 180.alb & 1 & 0 & Optimal & 120.02 & 15 & 15.00 &  0.00\\
instance n=100 181.alb & 1 & 0 & Optimal & 120.03 & 13 & 13.00 &  0.00\\
instance n=100 182.alb & 1 & 0 & Optimal &  0.12 & 15 & 15.00 &  0.00\\
instance n=100 183.alb & 1 & 0 & Optimal & 120.01 & 14 & 14.00 &  0.00\\
instance n=100 184.alb & 1 & 0 & Optimal & 120.02 & 14 & 14.00 &  0.00\\
instance n=100 185.alb & 1 & 0 & Optimal & 26.08 & 15 & 15.00 &  0.00\\
instance n=100 186.alb & 1 & 0 & Optimal & 120.02 & 14 & 14.00 &  0.00\\
instance n=100 187.alb & 1 & 0 & Optimal & 66.69 & 13 & 13.00 &  0.00\\
instance n=100 188.alb & 1 & 0 & Optimal &  0.11 & 16 & 16.00 &  0.00\\
instance n=100 189.alb & 1 & 0 & Optimal &  6.38 & 14 & 14.00 &  0.00\\
instance n=100 19.alb & 1 & 0 & Optimal & 120.03 & 23 & 23.00 &  0.00\\
instance n=100 190.alb & 1 & 0 & Optimal & 120.02 & 13 & 13.00 &  0.00\\
instance n=100 191.alb & 1 & 0 & Optimal & 120.02 & 14 & 14.00 &  0.00\\
instance n=100 192.alb & 1 & 0 & Optimal & 120.02 & 13 & 13.00 &  0.00\\
instance n=100 193.alb & 1 & 0 & Optimal & 40.76 & 15 & 15.00 &  0.00\\
instance n=100 194.alb & 1 & 0 & Optimal &  0.23 & 15 & 15.00 &  0.00\\
instance n=100 195.alb & 1 & 0 & Optimal &  0.27 & 15 & 15.00 &  0.00\\
instance n=100 196.alb & 1 & 0 & Optimal & 120.02 & 15 & 15.00 &  0.00\\
instance n=100 197.alb & 1 & 0 & Optimal &  3.96 & 15 & 15.00 &  0.00\\
instance n=100 198.alb & 1 & 0 & Optimal & 120.03 & 13 & 13.00 &  0.00\\
instance n=100 199.alb & 1 & 0 & Optimal &  0.11 & 14 & 14.00 &  0.00\\
instance n=100 2.alb & 1 & 0 & Optimal & 120.04 & 21 & 21.00 &  0.00\\
instance n=100 20.alb & 1 & 0 & Optimal & 120.02 & 21 & 21.00 &  0.00\\
instance n=100 200.alb & 1 & 0 & Optimal & 43.68 & 15 & 15.00 &  0.00\\
instance n=100 201.alb & 1 & 0 & Solution & 120.11 & 53 & 52.00 &  1.89\\
instance n=100 202.alb & 1 & 0 & Optimal & 120.06 & 61 & 61.00 &  0.00\\
instance n=100 203.alb & 1 & 0 & Optimal & 120.04 & 52 & 52.00 &  0.00\\
instance n=100 204.alb & 1 & 0 & Solution & 120.33 & 51 & 49.00 &  3.92\\
instance n=100 205.alb & 1 & 0 & Solution & 120.11 & 57 & 56.00 &  1.75\\
instance n=100 206.alb & 1 & 0 & Solution & 120.17 & 52 & 50.00 &  3.85\\
instance n=100 207.alb & 1 & 0 & Solution & 120.12 & 51 & 50.00 &  1.96\\
instance n=100 208.alb & 1 & 0 & Solution & 120.10 & 57 & 56.00 &  1.75\\
instance n=100 209.alb & 1 & 0 & Solution & 120.16 & 55 & 54.00 &  1.82\\
instance n=100 21.alb & 1 & 0 & Optimal &  0.59 & 21 & 21.00 &  0.00\\
instance n=100 210.alb & 1 & 0 & Solution & 120.12 & 52 & 51.00 &  1.92\\
instance n=100 211.alb & 1 & 0 & Optimal & 46.45 & 51 & 51.00 &  0.00\\
instance n=100 212.alb & 1 & 0 & Solution & 120.12 & 52 & 51.00 &  1.92\\
instance n=100 213.alb & 1 & 0 & Solution & 120.13 & 52 & 51.00 &  1.92\\
instance n=100 214.alb & 1 & 0 & Solution & 120.48 & 55 & 53.00 &  3.64\\
instance n=100 215.alb & 1 & 0 & Solution & 120.43 & 50 & 48.00 &  4.00\\
instance n=100 216.alb & 1 & 0 & Solution & 120.12 & 52 & 51.00 &  1.92\\
instance n=100 217.alb & 1 & 0 & Solution & 120.27 & 52 & 51.00 &  1.92\\
instance n=100 218.alb & 1 & 0 & Solution & 120.11 & 53 & 52.00 &  1.89\\
instance n=100 219.alb & 1 & 0 & Solution & 120.14 & 52 & 51.00 &  1.92\\
instance n=100 22.alb & 1 & 0 & Solution & 120.09 & 25 & 24.00 &  4.00\\
instance n=100 220.alb & 1 & 0 & Solution & 120.11 & 53 & 52.00 &  1.89\\
instance n=100 221.alb & 1 & 0 & Solution & 120.12 & 57 & 56.00 &  1.75\\
instance n=100 222.alb & 1 & 0 & Solution & 120.29 & 53 & 51.00 &  3.77\\
instance n=100 223.alb & 1 & 0 & Solution & 120.13 & 51 & 50.00 &  1.96\\
instance n=100 224.alb & 1 & 0 & Optimal & 120.06 & 55 & 55.00 &  0.00\\
instance n=100 225.alb & 1 & 0 & Solution & 120.39 & 53 & 52.00 &  1.89\\
instance n=100 226.alb & 1 & 0 & Solution & 120.11 & 25 & 24.00 &  4.00\\
instance n=100 227.alb & 1 & 0 & Solution & 120.10 & 27 & 26.00 &  3.70\\
instance n=100 228.alb & 1 & 0 & Optimal &  4.63 & 22 & 22.00 &  0.00\\
instance n=100 229.alb & 1 & 0 & Optimal &  0.46 & 24 & 24.00 &  0.00\\
instance n=100 23.alb & 1 & 0 & Optimal &  0.11 & 24 & 24.00 &  0.00\\
instance n=100 230.alb & 1 & 0 & Optimal & 120.05 & 23 & 23.00 &  0.00\\
instance n=100 231.alb & 1 & 0 & Optimal &  0.83 & 22 & 22.00 &  0.00\\
instance n=100 232.alb & 1 & 0 & Optimal &  0.47 & 22 & 22.00 &  0.00\\
instance n=100 233.alb & 1 & 0 & Solution & 120.10 & 23 & 22.00 &  4.35\\
instance n=100 234.alb & 1 & 0 & Optimal &  0.48 & 23 & 23.00 &  0.00\\
instance n=100 235.alb & 1 & 0 & Optimal &  0.64 & 26 & 26.00 &  0.00\\
instance n=100 236.alb & 1 & 0 & Solution & 120.10 & 23 & 22.00 &  4.35\\
instance n=100 237.alb & 1 & 0 & Optimal &  0.47 & 23 & 23.00 &  0.00\\
instance n=100 238.alb & 1 & 0 & Optimal &  4.08 & 23 & 23.00 &  0.00\\
instance n=100 239.alb & 1 & 0 & Optimal &  0.24 & 21 & 21.00 &  0.00\\
instance n=100 24.alb & 1 & 0 & Optimal &  4.05 & 24 & 24.00 &  0.00\\
instance n=100 240.alb & 1 & 0 & Optimal &  2.53 & 22 & 22.00 &  0.00\\
instance n=100 241.alb & 1 & 0 & Optimal & 38.36 & 22 & 22.00 &  0.00\\
instance n=100 242.alb & 1 & 0 & Optimal &  2.22 & 23 & 23.00 &  0.00\\
instance n=100 243.alb & 1 & 0 & Solution & 120.09 & 24 & 23.00 &  4.17\\
instance n=100 244.alb & 1 & 0 & Optimal &  3.28 & 21 & 21.00 &  0.00\\
instance n=100 245.alb & 1 & 0 & Solution & 120.53 & 24 & 23.00 &  4.17\\
instance n=100 246.alb & 1 & 0 & Optimal &  1.04 & 26 & 26.00 &  0.00\\
instance n=100 247.alb & 1 & 0 & Optimal & 12.65 & 22 & 22.00 &  0.00\\
instance n=100 248.alb & 1 & 0 & Optimal & 34.41 & 19 & 19.00 &  0.00\\
instance n=100 249.alb & 1 & 0 & Optimal &  1.41 & 21 & 21.00 &  0.00\\
instance n=100 25.alb & 1 & 0 & Optimal & 120.03 & 22 & 22.00 &  0.00\\
instance n=100 250.alb & 1 & 0 & Optimal &  0.56 & 24 & 24.00 &  0.00\\
instance n=100 251.alb & 1 & 0 & Optimal &  0.11 & 15 & 15.00 &  0.00\\
instance n=100 252.alb & 1 & 0 & Optimal &  0.44 & 14 & 14.00 &  0.00\\
instance n=100 253.alb & 1 & 0 & Optimal &  0.10 & 14 & 14.00 &  0.00\\
instance n=100 254.alb & 1 & 0 & Optimal &  0.34 & 14 & 14.00 &  0.00\\
instance n=100 255.alb & 1 & 0 & Optimal &  0.14 & 14 & 14.00 &  0.00\\
instance n=100 256.alb & 1 & 0 & Optimal & 34.15 & 15 & 15.00 &  0.00\\
instance n=100 257.alb & 1 & 0 & Optimal & 120.03 & 12 & 12.00 &  0.00\\
instance n=100 258.alb & 1 & 0 & Optimal &  3.00 & 14 & 14.00 &  0.00\\
instance n=100 259.alb & 1 & 0 & Optimal &  0.57 & 15 & 15.00 &  0.00\\
instance n=100 26.alb & 1 & 0 & Optimal & 120.03 & 14 & 14.00 &  0.00\\
instance n=100 260.alb & 1 & 0 & Optimal &  5.78 & 15 & 15.00 &  0.00\\
instance n=100 261.alb & 1 & 0 & Optimal &  0.09 & 14 & 14.00 &  0.00\\
instance n=100 262.alb & 1 & 0 & Optimal &  0.03 & 14 & 14.00 &  0.00\\
instance n=100 263.alb & 1 & 0 & Optimal &  0.12 & 14 & 14.00 &  0.00\\
instance n=100 264.alb & 1 & 0 & Optimal &  2.65 & 15 & 15.00 &  0.00\\
instance n=100 265.alb & 1 & 0 & Optimal &  0.10 & 14 & 14.00 &  0.00\\
instance n=100 266.alb & 1 & 0 & Optimal &  0.44 & 13 & 13.00 &  0.00\\
instance n=100 267.alb & 1 & 0 & Optimal &  0.58 & 13 & 13.00 &  0.00\\
instance n=100 268.alb & 1 & 0 & Optimal &  0.09 & 15 & 15.00 &  0.00\\
instance n=100 269.alb & 1 & 0 & Optimal &  0.10 & 15 & 15.00 &  0.00\\
instance n=100 27.alb & 1 & 0 & Optimal & 120.02 & 13 & 13.00 &  0.00\\
instance n=100 270.alb & 1 & 0 & Optimal &  0.10 & 13 & 13.00 &  0.00\\
instance n=100 271.alb & 1 & 0 & Optimal & 120.02 & 13 & 13.00 &  0.00\\
instance n=100 272.alb & 1 & 0 & Optimal &  0.11 & 14 & 14.00 &  0.00\\
instance n=100 273.alb & 1 & 0 & Optimal & 11.72 & 13 & 13.00 &  0.00\\
instance n=100 274.alb & 1 & 0 & Optimal &  2.00 & 13 & 13.00 &  0.00\\
instance n=100 275.alb & 1 & 0 & Optimal &  0.09 & 13 & 13.00 &  0.00\\
instance n=100 276.alb & 1 & 0 & Solution & 120.12 & 60 & 58.00 &  3.33\\
instance n=100 277.alb & 1 & 0 & Solution & 120.14 & 57 & 54.00 &  5.26\\
instance n=100 278.alb & 1 & 0 & Solution & 120.13 & 57 & 55.00 &  3.51\\
instance n=100 279.alb & 1 & 0 & Solution & 120.11 & 53 & 52.00 &  1.89\\
instance n=100 28.alb & 1 & 0 & Optimal &  0.44 & 14 & 14.00 &  0.00\\
instance n=100 280.alb & 1 & 0 & Solution & 120.11 & 55 & 52.00 &  5.45\\
instance n=100 281.alb & 1 & 0 & Solution & 121.37 & 62 & 60.00 &  3.23\\
instance n=100 282.alb & 1 & 0 & Solution & 120.14 & 60 & 57.00 &  5.00\\
instance n=100 283.alb & 1 & 0 & Solution & 120.14 & 55 & 53.00 &  3.64\\
instance n=100 284.alb & 1 & 0 & Solution & 120.12 & 55 & 54.00 &  1.82\\
instance n=100 285.alb & 1 & 0 & Solution & 120.13 & 55 & 52.00 &  5.45\\
instance n=100 286.alb & 1 & 0 & Solution & 120.16 & 56 & 55.00 &  1.79\\
instance n=100 287.alb & 1 & 0 & Optimal &  9.49 & 54 & 54.00 &  0.00\\
instance n=100 288.alb & 1 & 0 & Solution & 120.35 & 56 & 53.00 &  5.36\\
instance n=100 289.alb & 1 & 0 & Optimal & 45.20 & 62 & 62.00 &  0.00\\
instance n=100 29.alb & 1 & 0 & Optimal & 74.27 & 14 & 14.00 &  0.00\\
instance n=100 290.alb & 1 & 0 & Solution & 120.12 & 54 & 52.00 &  3.70\\
instance n=100 291.alb & 1 & 0 & Solution & 120.11 & 52 & 49.00 &  5.77\\
instance n=100 292.alb & 1 & 0 & Solution & 120.12 & 57 & 55.00 &  3.51\\
instance n=100 293.alb & 1 & 0 & Solution & 120.14 & 52 & 50.00 &  3.85\\
instance n=100 294.alb & 1 & 0 & Solution & 120.12 & 57 & 54.00 &  5.26\\
instance n=100 295.alb & 1 & 0 & Solution & 120.12 & 56 & 55.00 &  1.79\\
instance n=100 296.alb & 1 & 0 & Solution & 120.11 & 55 & 53.00 &  3.64\\
instance n=100 297.alb & 1 & 0 & Optimal & 93.01 & 58 & 58.00 &  0.00\\
instance n=100 298.alb & 1 & 0 & Solution & 120.22 & 58 & 57.00 &  1.72\\
instance n=100 299.alb & 1 & 0 & Solution & 120.13 & 55 & 54.00 &  1.82\\
instance n=100 3.alb & 1 & 0 & Optimal &  0.33 & 20 & 20.00 &  0.00\\
instance n=100 30.alb & 1 & 0 & Optimal & 120.03 & 15 & 15.00 &  0.00\\
instance n=100 300.alb & 1 & 0 & Solution & 120.14 & 54 & 51.00 &  5.56\\
instance n=100 301.alb & 1 & 0 & Optimal & 120.03 & 23 & 23.00 &  0.00\\
instance n=100 302.alb & 1 & 0 & Optimal &  0.56 & 24 & 24.00 &  0.00\\
instance n=100 303.alb & 1 & 0 & Optimal & 120.04 & 24 & 24.00 &  0.00\\
instance n=100 304.alb & 1 & 0 & Optimal &  0.48 & 21 & 21.00 &  0.00\\
instance n=100 305.alb & 1 & 0 & Optimal & 63.57 & 22 & 22.00 &  0.00\\
instance n=100 306.alb & 1 & 0 & Optimal &  1.61 & 24 & 24.00 &  0.00\\
instance n=100 307.alb & 1 & 0 & Solution & 120.09 & 24 & 23.00 &  4.17\\
instance n=100 308.alb & 1 & 0 & Optimal & 120.07 & 20 & 20.00 &  0.00\\
instance n=100 309.alb & 1 & 0 & Solution & 120.10 & 22 & 21.00 &  4.55\\
instance n=100 31.alb & 1 & 0 & Optimal &  0.10 & 14 & 14.00 &  0.00\\
instance n=100 310.alb & 1 & 0 & Optimal &  0.15 & 23 & 23.00 &  0.00\\
instance n=100 311.alb & 1 & 0 & Optimal &  2.57 & 21 & 21.00 &  0.00\\
instance n=100 312.alb & 1 & 0 & Optimal & 120.03 & 22 & 22.00 &  0.00\\
instance n=100 313.alb & 1 & 0 & Optimal & 27.59 & 23 & 23.00 &  0.00\\
instance n=100 314.alb & 1 & 0 & Optimal &  0.59 & 19 & 19.00 &  0.00\\
instance n=100 315.alb & 1 & 0 & Optimal & 120.03 & 22 & 22.00 &  0.00\\
instance n=100 316.alb & 1 & 0 & Optimal & 120.04 & 24 & 24.00 &  0.00\\
instance n=100 317.alb & 1 & 0 & Optimal &  0.26 & 26 & 26.00 &  0.00\\
instance n=100 318.alb & 1 & 0 & Optimal &  0.24 & 21 & 21.00 &  0.00\\
instance n=100 319.alb & 1 & 0 & Optimal &  0.43 & 23 & 23.00 &  0.00\\
instance n=100 32.alb & 1 & 0 & Optimal & 120.01 & 14 & 14.00 &  0.00\\
instance n=100 320.alb & 1 & 0 & Optimal &  0.11 & 22 & 22.00 &  0.00\\
instance n=100 321.alb & 1 & 0 & Optimal &  3.31 & 26 & 26.00 &  0.00\\
instance n=100 322.alb & 1 & 0 & Solution & 120.11 & 24 & 23.00 &  4.17\\
instance n=100 323.alb & 1 & 0 & Optimal & 13.16 & 24 & 24.00 &  0.00\\
instance n=100 324.alb & 1 & 0 & Optimal &  0.11 & 23 & 23.00 &  0.00\\
instance n=100 325.alb & 1 & 0 & Optimal & 120.08 & 25 & 25.00 &  0.00\\
instance n=100 326.alb & 1 & 0 & Optimal & 120.01 & 13 & 13.00 &  0.00\\
instance n=100 327.alb & 1 & 0 & Optimal & 120.03 & 14 & 14.00 &  0.00\\
instance n=100 328.alb & 1 & 0 & Solution & 120.18 & 15 & 14.00 &  6.67\\
instance n=100 329.alb & 1 & 0 & Optimal & 120.02 & 14 & 14.00 &  0.00\\
instance n=100 33.alb & 1 & 0 & Optimal &  1.81 & 15 & 15.00 &  0.00\\
instance n=100 330.alb & 1 & 0 & Optimal & 39.16 & 14 & 14.00 &  0.00\\
instance n=100 331.alb & 1 & 0 & Optimal & 120.03 & 14 & 14.00 &  0.00\\
instance n=100 332.alb & 1 & 0 & Optimal & 120.01 & 14 & 14.00 &  0.00\\
instance n=100 333.alb & 1 & 0 & Optimal & 120.02 & 15 & 15.00 &  0.00\\
instance n=100 334.alb & 1 & 0 & Optimal & 120.03 & 14 & 14.00 &  0.00\\
instance n=100 335.alb & 1 & 0 & Optimal & 120.01 & 13 & 13.00 &  0.00\\
instance n=100 336.alb & 1 & 0 & Optimal & 120.03 & 15 & 15.00 &  0.00\\
instance n=100 337.alb & 1 & 0 & Optimal & 120.02 & 13 & 13.00 &  0.00\\
instance n=100 338.alb & 1 & 0 & Optimal & 120.03 & 14 & 14.00 &  0.00\\
instance n=100 339.alb & 1 & 0 & Optimal &  6.68 & 14 & 14.00 &  0.00\\
instance n=100 34.alb & 1 & 0 & Optimal & 29.63 & 15 & 15.00 &  0.00\\
instance n=100 340.alb & 1 & 0 & Optimal & 120.01 & 14 & 14.00 &  0.00\\
instance n=100 341.alb & 1 & 0 & Optimal & 120.01 & 16 & 16.00 &  0.00\\
instance n=100 342.alb & 1 & 0 & Optimal & 120.01 & 14 & 14.00 &  0.00\\
instance n=100 343.alb & 1 & 0 & Optimal & 120.03 & 16 & 16.00 &  0.00\\
instance n=100 344.alb & 1 & 0 & Optimal & 57.97 & 15 & 15.00 &  0.00\\
instance n=100 345.alb & 1 & 0 & Optimal & 120.02 & 14 & 14.00 &  0.00\\
instance n=100 346.alb & 1 & 0 & Optimal & 120.03 & 14 & 14.00 &  0.00\\
instance n=100 347.alb & 1 & 0 & Optimal & 120.02 & 14 & 14.00 &  0.00\\
instance n=100 348.alb & 1 & 0 & Optimal & 120.01 & 14 & 14.00 &  0.00\\
instance n=100 349.alb & 1 & 0 & Optimal & 120.02 & 13 & 13.00 &  0.00\\
instance n=100 35.alb & 1 & 0 & Optimal & 120.02 & 15 & 15.00 &  0.00\\
instance n=100 350.alb & 1 & 0 & Optimal & 27.41 & 14 & 14.00 &  0.00\\
instance n=100 351.alb & 1 & 0 & Solution & 120.11 & 59 & 58.00 &  1.69\\
instance n=100 352.alb & 1 & 0 & Optimal &  0.14 & 63 & 63.00 &  0.00\\
instance n=100 353.alb & 1 & 0 & Solution & 120.25 & 52 & 50.00 &  3.85\\
instance n=100 354.alb & 1 & 0 & Solution & 120.12 & 52 & 51.00 &  1.92\\
instance n=100 355.alb & 1 & 0 & Solution & 120.56 & 55 & 53.00 &  3.64\\
instance n=100 356.alb & 1 & 0 & Solution & 120.15 & 60 & 59.00 &  1.67\\
instance n=100 357.alb & 1 & 0 & Optimal & 120.07 & 53 & 53.00 &  0.00\\
instance n=100 358.alb & 1 & 0 & Solution & 120.12 & 52 & 51.00 &  1.92\\
instance n=100 359.alb & 1 & 0 & Solution & 120.11 & 53 & 52.00 &  1.89\\
instance n=100 36.alb & 1 & 0 & Optimal & 120.03 & 14 & 14.00 &  0.00\\
instance n=100 360.alb & 1 & 0 & Optimal & 46.00 & 54 & 54.00 &  0.00\\
instance n=100 361.alb & 1 & 0 & Solution & 120.12 & 52 & 50.00 &  3.85\\
instance n=100 362.alb & 1 & 0 & Optimal & 120.07 & 57 & 57.00 &  0.00\\
instance n=100 363.alb & 1 & 0 & Solution & 120.11 & 53 & 51.00 &  3.77\\
instance n=100 364.alb & 1 & 0 & Solution & 120.14 & 52 & 51.00 &  1.92\\
instance n=100 365.alb & 1 & 0 & Solution & 120.13 & 53 & 52.00 &  1.89\\
instance n=100 366.alb & 1 & 0 & Optimal & 120.03 & 61 & 61.00 &  0.00\\
instance n=100 367.alb & 1 & 0 & Optimal & 120.06 & 55 & 55.00 &  0.00\\
instance n=100 368.alb & 1 & 0 & Optimal & 120.05 & 58 & 58.00 &  0.00\\
instance n=100 369.alb & 1 & 0 & Solution & 120.12 & 51 & 50.00 &  1.96\\
instance n=100 37.alb & 1 & 0 & Optimal & 120.02 & 14 & 14.00 &  0.00\\
instance n=100 370.alb & 1 & 0 & Solution & 120.12 & 57 & 56.00 &  1.75\\
instance n=100 371.alb & 1 & 0 & Solution & 120.13 & 53 & 51.00 &  3.77\\
instance n=100 372.alb & 1 & 0 & Solution & 120.13 & 49 & 48.00 &  2.04\\
instance n=100 373.alb & 1 & 0 & Solution & 120.53 & 51 & 50.00 &  1.96\\
instance n=100 374.alb & 1 & 0 & Solution & 120.12 & 52 & 51.00 &  1.92\\
instance n=100 375.alb & 1 & 0 & Optimal & 120.03 & 57 & 57.00 &  0.00\\
instance n=100 376.alb & 1 & 0 & Optimal &  0.21 & 23 & 23.00 &  0.00\\
instance n=100 377.alb & 1 & 0 & Solution & 120.39 & 21 & 20.00 &  4.76\\
instance n=100 378.alb & 1 & 0 & Optimal & 13.14 & 22 & 22.00 &  0.00\\
instance n=100 379.alb & 1 & 0 & Optimal & 120.03 & 23 & 23.00 &  0.00\\
instance n=100 38.alb & 1 & 0 & Optimal & 120.02 & 14 & 14.00 &  0.00\\
instance n=100 380.alb & 1 & 0 & Solution & 120.10 & 23 & 22.00 &  4.35\\
instance n=100 381.alb & 1 & 0 & Optimal &  0.59 & 24 & 24.00 &  0.00\\
instance n=100 382.alb & 1 & 0 & Optimal & 120.03 & 25 & 25.00 &  0.00\\
instance n=100 383.alb & 1 & 0 & Optimal &  0.23 & 25 & 25.00 &  0.00\\
instance n=100 384.alb & 1 & 0 & Optimal &  1.10 & 25 & 25.00 &  0.00\\
instance n=100 385.alb & 1 & 0 & Optimal &  0.10 & 22 & 22.00 &  0.00\\
instance n=100 386.alb & 1 & 0 & Solution & 120.10 & 24 & 23.00 &  4.17\\
instance n=100 387.alb & 1 & 0 & Optimal &  0.11 & 22 & 22.00 &  0.00\\
instance n=100 388.alb & 1 & 0 & Optimal & 94.23 & 25 & 25.00 &  0.00\\
instance n=100 389.alb & 1 & 0 & Optimal &  1.34 & 23 & 23.00 &  0.00\\
instance n=100 39.alb & 1 & 0 & Optimal & 120.02 & 14 & 14.00 &  0.00\\
instance n=100 390.alb & 1 & 0 & Optimal & 26.72 & 22 & 22.00 &  0.00\\
instance n=100 391.alb & 1 & 0 & Optimal &  0.45 & 20 & 20.00 &  0.00\\
instance n=100 392.alb & 1 & 0 & Optimal &  0.12 & 22 & 22.00 &  0.00\\
instance n=100 393.alb & 1 & 0 & Solution & 120.10 & 24 & 23.00 &  4.17\\
instance n=100 394.alb & 1 & 0 & Optimal &  0.58 & 22 & 22.00 &  0.00\\
instance n=100 395.alb & 1 & 0 & Optimal &  2.06 & 24 & 24.00 &  0.00\\
instance n=100 396.alb & 1 & 0 & Optimal & 75.42 & 20 & 20.00 &  0.00\\
instance n=100 397.alb & 1 & 0 & Solution & 120.14 & 26 & 25.00 &  3.85\\
instance n=100 398.alb & 1 & 0 & Optimal & 120.04 & 25 & 25.00 &  0.00\\
instance n=100 399.alb & 1 & 0 & Optimal &  1.48 & 23 & 23.00 &  0.00\\
instance n=100 4.alb & 1 & 0 & Optimal &  1.52 & 24 & 24.00 &  0.00\\
instance n=100 40.alb & 1 & 0 & Optimal &  1.30 & 14 & 14.00 &  0.00\\
instance n=100 400.alb & 1 & 0 & Optimal &  0.56 & 24 & 24.00 &  0.00\\
instance n=100 401.alb & 1 & 0 & Optimal &  0.09 & 15 & 15.00 &  0.00\\
instance n=100 402.alb & 1 & 0 & Optimal &  0.57 & 15 & 15.00 &  0.00\\
instance n=100 403.alb & 1 & 0 & Optimal &  0.97 & 14 & 14.00 &  0.00\\
instance n=100 404.alb & 1 & 0 & Optimal &  0.09 & 15 & 15.00 &  0.00\\
instance n=100 405.alb & 1 & 0 & Optimal &  1.16 & 13 & 13.00 &  0.00\\
instance n=100 406.alb & 1 & 0 & Optimal &  0.10 & 14 & 14.00 &  0.00\\
instance n=100 407.alb & 1 & 0 & Optimal &  0.16 & 15 & 15.00 &  0.00\\
instance n=100 408.alb & 1 & 0 & Optimal & 120.03 & 14 & 14.00 &  0.00\\
instance n=100 409.alb & 1 & 0 & Optimal &  0.10 & 15 & 15.00 &  0.00\\
instance n=100 41.alb & 1 & 0 & Optimal & 120.03 & 13 & 13.00 &  0.00\\
instance n=100 410.alb & 1 & 0 & Optimal &  0.10 & 14 & 14.00 &  0.00\\
instance n=100 411.alb & 1 & 0 & Optimal &  5.46 & 14 & 14.00 &  0.00\\
instance n=100 412.alb & 1 & 0 & Optimal &  0.11 & 14 & 14.00 &  0.00\\
instance n=100 413.alb & 1 & 0 & Optimal &  0.29 & 14 & 14.00 &  0.00\\
instance n=100 414.alb & 1 & 0 & Optimal & 120.04 & 14 & 14.00 &  0.00\\
instance n=100 415.alb & 1 & 0 & Optimal & 15.44 & 13 & 13.00 &  0.00\\
instance n=100 416.alb & 1 & 0 & Optimal &  0.23 & 14 & 14.00 &  0.00\\
instance n=100 417.alb & 1 & 0 & Optimal &  0.11 & 15 & 15.00 &  0.00\\
instance n=100 418.alb & 1 & 0 & Optimal &  0.11 & 16 & 16.00 &  0.00\\
instance n=100 419.alb & 1 & 0 & Optimal &  1.55 & 14 & 14.00 &  0.00\\
instance n=100 42.alb & 1 & 0 & Optimal & 120.02 & 14 & 14.00 &  0.00\\
instance n=100 420.alb & 1 & 0 & Optimal &  0.11 & 14 & 14.00 &  0.00\\
instance n=100 421.alb & 1 & 0 & Optimal &  1.01 & 14 & 14.00 &  0.00\\
instance n=100 422.alb & 1 & 0 & Optimal &  0.10 & 15 & 15.00 &  0.00\\
instance n=100 423.alb & 1 & 0 & Optimal &  1.00 & 14 & 14.00 &  0.00\\
instance n=100 424.alb & 1 & 0 & Optimal &  0.12 & 14 & 14.00 &  0.00\\
instance n=100 425.alb & 1 & 0 & Optimal & 39.57 & 15 & 15.00 &  0.00\\
instance n=100 426.alb & 1 & 0 & Solution & 120.12 & 60 & 58.00 &  3.33\\
instance n=100 427.alb & 1 & 0 & Solution & 120.13 & 55 & 54.00 &  1.82\\
instance n=100 428.alb & 1 & 0 & Solution & 120.13 & 55 & 54.00 &  1.82\\
instance n=100 429.alb & 1 & 0 & Solution & 120.12 & 58 & 57.00 &  1.72\\
instance n=100 43.alb & 1 & 0 & Optimal & 120.03 & 14 & 14.00 &  0.00\\
instance n=100 430.alb & 1 & 0 & Solution & 120.13 & 53 & 52.00 &  1.89\\
instance n=100 431.alb & 1 & 0 & Solution & 120.13 & 54 & 52.00 &  3.70\\
instance n=100 432.alb & 1 & 0 & Solution & 120.11 & 56 & 54.00 &  3.57\\
instance n=100 433.alb & 1 & 0 & Optimal & 62.53 & 52 & 52.00 &  0.00\\
instance n=100 434.alb & 1 & 0 & Solution & 120.14 & 56 & 55.00 &  1.79\\
instance n=100 435.alb & 1 & 0 & Solution & 121.23 & 56 & 52.00 &  7.14\\
instance n=100 436.alb & 1 & 0 & Solution & 120.11 & 52 & 49.00 &  5.77\\
instance n=100 437.alb & 1 & 0 & Solution & 120.15 & 53 & 51.00 &  3.77\\
instance n=100 438.alb & 1 & 0 & Solution & 120.36 & 55 & 52.00 &  5.45\\
instance n=100 439.alb & 1 & 0 & Solution & 120.38 & 55 & 54.00 &  1.82\\
instance n=100 44.alb & 1 & 0 & Optimal &  0.09 & 14 & 14.00 &  0.00\\
instance n=100 440.alb & 1 & 0 & Solution & 120.24 & 53 & 51.00 &  3.77\\
instance n=100 441.alb & 1 & 0 & Solution & 120.13 & 52 & 51.00 &  1.92\\
instance n=100 442.alb & 1 & 0 & Solution & 120.11 & 52 & 49.00 &  5.77\\
instance n=100 443.alb & 1 & 0 & Solution & 120.14 & 55 & 53.00 &  3.64\\
instance n=100 444.alb & 1 & 0 & Solution & 120.14 & 54 & 50.00 &  7.41\\
instance n=100 445.alb & 1 & 0 & Solution & 120.13 & 55 & 54.00 &  1.82\\
instance n=100 446.alb & 1 & 0 & Solution & 120.15 & 57 & 54.00 &  5.26\\
instance n=100 447.alb & 1 & 0 & Solution & 120.39 & 54 & 52.00 &  3.70\\
instance n=100 448.alb & 1 & 0 & Solution & 120.19 & 55 & 54.00 &  1.82\\
instance n=100 449.alb & 1 & 0 & Solution & 120.15 & 55 & 52.00 &  5.45\\
instance n=100 45.alb & 1 & 0 & Optimal & 120.03 & 14 & 14.00 &  0.00\\
instance n=100 450.alb & 1 & 0 & Solution & 121.22 & 53 & 52.00 &  1.89\\
instance n=100 451.alb & 1 & 0 & Optimal &  0.19 & 26 & 26.00 &  0.00\\
instance n=100 452.alb & 1 & 0 & Optimal &  0.45 & 22 & 22.00 &  0.00\\
instance n=100 453.alb & 1 & 0 & Optimal &  0.40 & 24 & 24.00 &  0.00\\
instance n=100 454.alb & 1 & 0 & Optimal &  0.15 & 23 & 23.00 &  0.00\\
instance n=100 455.alb & 1 & 0 & Optimal &  0.51 & 23 & 23.00 &  0.00\\
instance n=100 456.alb & 1 & 0 & Optimal &  0.39 & 26 & 26.00 &  0.00\\
instance n=100 457.alb & 1 & 0 & Optimal &  0.38 & 23 & 23.00 &  0.00\\
instance n=100 458.alb & 1 & 0 & Optimal &  0.18 & 24 & 24.00 &  0.00\\
instance n=100 459.alb & 1 & 0 & Optimal &  0.31 & 23 & 23.00 &  0.00\\
instance n=100 46.alb & 1 & 0 & Optimal & 120.03 & 14 & 14.00 &  0.00\\
instance n=100 460.alb & 1 & 0 & Optimal &  0.16 & 23 & 23.00 &  0.00\\
instance n=100 461.alb & 1 & 0 & Optimal &  1.55 & 23 & 23.00 &  0.00\\
instance n=100 462.alb & 1 & 0 & Optimal &  0.32 & 23 & 23.00 &  0.00\\
instance n=100 463.alb & 1 & 0 & Optimal &  0.68 & 26 & 26.00 &  0.00\\
instance n=100 464.alb & 1 & 0 & Optimal &  0.14 & 25 & 25.00 &  0.00\\
instance n=100 465.alb & 1 & 0 & Optimal &  0.65 & 22 & 22.00 &  0.00\\
instance n=100 466.alb & 1 & 0 & Optimal &  0.40 & 26 & 26.00 &  0.00\\
instance n=100 467.alb & 1 & 0 & Optimal &  1.69 & 21 & 21.00 &  0.00\\
instance n=100 468.alb & 1 & 0 & Optimal &  0.58 & 25 & 25.00 &  0.00\\
instance n=100 469.alb & 1 & 0 & Optimal &  0.14 & 22 & 22.00 &  0.00\\
instance n=100 47.alb & 1 & 0 & Optimal & 120.03 & 14 & 14.00 &  0.00\\
instance n=100 470.alb & 1 & 0 & Optimal &  1.43 & 26 & 26.00 &  0.00\\
instance n=100 471.alb & 1 & 0 & Optimal &  0.55 & 26 & 26.00 &  0.00\\
instance n=100 472.alb & 1 & 0 & Optimal &  0.24 & 23 & 23.00 &  0.00\\
instance n=100 473.alb & 1 & 0 & Optimal &  0.48 & 28 & 28.00 &  0.00\\
instance n=100 474.alb & 1 & 0 & Optimal &  0.44 & 23 & 23.00 &  0.00\\
instance n=100 475.alb & 1 & 0 & Optimal &  1.21 & 24 & 24.00 &  0.00\\
instance n=100 476.alb & 1 & 0 & Optimal &  0.12 & 14 & 14.00 &  0.00\\
instance n=100 477.alb & 1 & 0 & Optimal &  0.11 & 14 & 14.00 &  0.00\\
instance n=100 478.alb & 1 & 0 & Optimal &  0.12 & 14 & 14.00 &  0.00\\
instance n=100 479.alb & 1 & 0 & Optimal &  0.30 & 16 & 16.00 &  0.00\\
instance n=100 48.alb & 1 & 0 & Optimal & 120.03 & 15 & 15.00 &  0.00\\
instance n=100 480.alb & 1 & 0 & Optimal &  0.09 & 15 & 15.00 &  0.00\\
instance n=100 481.alb & 1 & 0 & Optimal &  0.13 & 15 & 15.00 &  0.00\\
instance n=100 482.alb & 1 & 0 & Optimal &  0.46 & 15 & 15.00 &  0.00\\
instance n=100 483.alb & 1 & 0 & Optimal &  0.18 & 14 & 14.00 &  0.00\\
instance n=100 484.alb & 1 & 0 & Optimal &  0.12 & 14 & 14.00 &  0.00\\
instance n=100 485.alb & 1 & 0 & Optimal &  1.44 & 16 & 16.00 &  0.00\\
instance n=100 486.alb & 1 & 0 & Optimal &  0.09 & 15 & 15.00 &  0.00\\
instance n=100 487.alb & 1 & 0 & Optimal &  0.19 & 15 & 15.00 &  0.00\\
instance n=100 488.alb & 1 & 0 & Optimal &  0.41 & 16 & 16.00 &  0.00\\
instance n=100 489.alb & 1 & 0 & Optimal &  0.67 & 13 & 13.00 &  0.00\\
instance n=100 49.alb & 1 & 0 & Optimal & 120.01 & 14 & 14.00 &  0.00\\
instance n=100 490.alb & 1 & 0 & Optimal &  0.10 & 15 & 15.00 &  0.00\\
instance n=100 491.alb & 1 & 0 & Optimal &  1.63 & 16 & 16.00 &  0.00\\
instance n=100 492.alb & 1 & 0 & Optimal &  0.48 & 14 & 14.00 &  0.00\\
instance n=100 493.alb & 1 & 0 & Optimal &  0.34 & 14 & 14.00 &  0.00\\
instance n=100 494.alb & 1 & 0 & Optimal &  0.09 & 14 & 14.00 &  0.00\\
instance n=100 495.alb & 1 & 0 & Optimal &  0.09 & 15 & 15.00 &  0.00\\
instance n=100 496.alb & 1 & 0 & Optimal &  0.24 & 14 & 14.00 &  0.00\\
instance n=100 497.alb & 1 & 0 & Optimal &  0.11 & 13 & 13.00 &  0.00\\
instance n=100 498.alb & 1 & 0 & Optimal &  0.10 & 14 & 14.00 &  0.00\\
instance n=100 499.alb & 1 & 0 & Optimal &  0.13 & 14 & 14.00 &  0.00\\
instance n=100 5.alb & 1 & 0 & Optimal &  0.10 & 22 & 22.00 &  0.00\\
instance n=100 50.alb & 1 & 0 & Optimal & 120.02 & 14 & 14.00 &  0.00\\
instance n=100 500.alb & 1 & 0 & Optimal &  0.12 & 14 & 14.00 &  0.00\\
instance n=100 501.alb & 1 & 0 & Optimal &  1.75 & 62 & 62.00 &  0.00\\
instance n=100 502.alb & 1 & 0 & Optimal &  0.37 & 64 & 64.00 &  0.00\\
instance n=100 503.alb & 1 & 0 & Optimal &  1.27 & 60 & 60.00 &  0.00\\
instance n=100 504.alb & 1 & 0 & Optimal &  9.11 & 60 & 60.00 &  0.00\\
instance n=100 505.alb & 1 & 0 & Optimal &  0.40 & 61 & 61.00 &  0.00\\
instance n=100 506.alb & 1 & 0 & Optimal &  0.67 & 57 & 57.00 &  0.00\\
instance n=100 507.alb & 1 & 0 & Optimal &  4.43 & 59 & 59.00 &  0.00\\
instance n=100 508.alb & 1 & 0 & Optimal &  2.32 & 56 & 56.00 &  0.00\\
instance n=100 509.alb & 1 & 0 & Optimal &  0.98 & 57 & 57.00 &  0.00\\
instance n=100 51.alb & 1 & 0 & Solution & 120.12 & 50 & 49.00 &  2.00\\
instance n=100 510.alb & 1 & 0 & Optimal &  5.09 & 58 & 58.00 &  0.00\\
instance n=100 511.alb & 1 & 0 & Optimal &  3.50 & 59 & 59.00 &  0.00\\
instance n=100 512.alb & 1 & 0 & Optimal &  0.33 & 60 & 60.00 &  0.00\\
instance n=100 513.alb & 1 & 0 & Optimal &  6.71 & 62 & 62.00 &  0.00\\
instance n=100 514.alb & 1 & 0 & Optimal &  4.16 & 58 & 58.00 &  0.00\\
instance n=100 515.alb & 1 & 0 & Optimal &  5.58 & 61 & 61.00 &  0.00\\
instance n=100 516.alb & 1 & 0 & Optimal &  0.13 & 70 & 70.00 &  0.00\\
instance n=100 517.alb & 1 & 0 & Optimal &  1.97 & 62 & 62.00 &  0.00\\
instance n=100 518.alb & 1 & 0 & Optimal &  1.34 & 57 & 57.00 &  0.00\\
instance n=100 519.alb & 1 & 0 & Optimal &  0.83 & 61 & 61.00 &  0.00\\
instance n=100 52.alb & 1 & 0 & Solution & 120.29 & 53 & 52.00 &  1.89\\
instance n=100 520.alb & 1 & 0 & Optimal &  5.41 & 60 & 60.00 &  0.00\\
instance n=100 521.alb & 1 & 0 & Optimal &  0.94 & 70 & 70.00 &  0.00\\
instance n=100 522.alb & 1 & 0 & Optimal & 10.23 & 59 & 59.00 &  0.00\\
instance n=100 523.alb & 1 & 0 & Optimal &  4.62 & 55 & 55.00 &  0.00\\
instance n=100 524.alb & 1 & 0 & Optimal &  3.83 & 59 & 59.00 &  0.00\\
instance n=100 525.alb & 1 & 0 & Optimal &  5.42 & 62 & 62.00 &  0.00\\
instance n=100 53.alb & 1 & 0 & Optimal & 25.98 & 52 & 52.00 &  0.00\\
instance n=100 54.alb & 1 & 0 & Optimal & 120.05 & 51 & 51.00 &  0.00\\
instance n=100 55.alb & 1 & 0 & Solution & 120.10 & 53 & 52.00 &  1.89\\
instance n=100 56.alb & 1 & 0 & Solution & 120.12 & 52 & 51.00 &  1.92\\
instance n=100 57.alb & 1 & 0 & Solution & 120.11 & 54 & 53.00 &  1.85\\
instance n=100 58.alb & 1 & 0 & Solution & 120.11 & 57 & 56.00 &  1.75\\
instance n=100 59.alb & 1 & 0 & Optimal & 120.05 & 57 & 57.00 &  0.00\\
instance n=100 6.alb & 1 & 0 & Optimal & 120.04 & 22 & 22.00 &  0.00\\
instance n=100 60.alb & 1 & 0 & Solution & 120.18 & 54 & 53.00 &  1.85\\
instance n=100 61.alb & 1 & 0 & Solution & 120.09 & 55 & 54.00 &  1.82\\
instance n=100 62.alb & 1 & 0 & Solution & 120.17 & 52 & 50.00 &  3.85\\
instance n=100 63.alb & 1 & 0 & Optimal & 120.05 & 61 & 61.00 &  0.00\\
instance n=100 64.alb & 1 & 0 & Solution & 120.20 & 56 & 55.00 &  1.79\\
instance n=100 65.alb & 1 & 0 & Solution & 120.11 & 62 & 61.00 &  1.61\\
instance n=100 66.alb & 1 & 0 & Solution & 120.66 & 51 & 50.00 &  1.96\\
instance n=100 67.alb & 1 & 0 & Solution & 120.14 & 55 & 54.00 &  1.82\\
instance n=100 68.alb & 1 & 0 & Optimal &  0.23 & 57 & 57.00 &  0.00\\
instance n=100 69.alb & 1 & 0 & Optimal & 120.03 & 53 & 53.00 &  0.00\\
instance n=100 7.alb & 1 & 0 & Optimal &  6.66 & 26 & 26.00 &  0.00\\
instance n=100 70.alb & 1 & 0 & Solution & 120.11 & 53 & 51.00 &  3.77\\
instance n=100 71.alb & 1 & 0 & Solution & 120.11 & 53 & 52.00 &  1.89\\
instance n=100 72.alb & 1 & 0 & Solution & 120.16 & 53 & 52.00 &  1.89\\
instance n=100 73.alb & 1 & 0 & Solution & 120.12 & 56 & 55.00 &  1.79\\
instance n=100 74.alb & 1 & 0 & Solution & 120.12 & 51 & 50.00 &  1.96\\
instance n=100 75.alb & 1 & 0 & Optimal & 120.05 & 54 & 54.00 &  0.00\\
instance n=100 76.alb & 1 & 0 & Optimal &  0.10 & 23 & 23.00 &  0.00\\
instance n=100 77.alb & 1 & 0 & Optimal &  0.67 & 20 & 20.00 &  0.00\\
instance n=100 78.alb & 1 & 0 & Optimal &  3.46 & 21 & 21.00 &  0.00\\
instance n=100 79.alb & 1 & 0 & Optimal &  0.47 & 21 & 21.00 &  0.00\\
instance n=100 8.alb & 1 & 0 & Optimal &  0.40 & 24 & 24.00 &  0.00\\
instance n=100 80.alb & 1 & 0 & Optimal & 120.05 & 22 & 22.00 &  0.00\\
instance n=100 81.alb & 1 & 0 & Optimal & 46.05 & 20 & 20.00 &  0.00\\
instance n=100 82.alb & 1 & 0 & Optimal &  0.12 & 21 & 21.00 &  0.00\\
instance n=100 83.alb & 1 & 0 & Optimal & 35.51 & 22 & 22.00 &  0.00\\
instance n=100 84.alb & 1 & 0 & Solution & 120.07 & 27 & 26.00 &  3.70\\
instance n=100 85.alb & 1 & 0 & Solution & 120.06 & 25 & 24.00 &  4.00\\
instance n=100 86.alb & 1 & 0 & Optimal &  0.71 & 23 & 23.00 &  0.00\\
instance n=100 87.alb & 1 & 0 & Optimal &  0.54 & 22 & 22.00 &  0.00\\
instance n=100 88.alb & 1 & 0 & Solution & 120.08 & 24 & 23.00 &  4.17\\
instance n=100 89.alb & 1 & 0 & Optimal &  9.69 & 24 & 24.00 &  0.00\\
instance n=100 9.alb & 1 & 0 & Optimal & 23.46 & 23 & 23.00 &  0.00\\
instance n=100 90.alb & 1 & 0 & Solution & 120.06 & 21 & 20.00 &  4.76\\
instance n=100 91.alb & 1 & 0 & Optimal &  0.50 & 25 & 25.00 &  0.00\\
instance n=100 92.alb & 1 & 0 & Optimal &  0.12 & 24 & 24.00 &  0.00\\
instance n=100 93.alb & 1 & 0 & Optimal & 120.03 & 27 & 27.00 &  0.00\\
instance n=100 94.alb & 1 & 0 & Optimal & 120.04 & 22 & 22.00 &  0.00\\
instance n=100 95.alb & 1 & 0 & Optimal &  2.14 & 21 & 21.00 &  0.00\\
instance n=100 96.alb & 1 & 0 & Optimal & 120.02 & 21 & 21.00 &  0.00\\
instance n=100 97.alb & 1 & 0 & Optimal &  0.55 & 22 & 22.00 &  0.00\\
instance n=100 98.alb & 1 & 0 & Optimal & 15.98 & 22 & 22.00 &  0.00\\
instance n=100 99.alb & 1 & 0 & Optimal &  0.51 & 22 & 22.00 &  0.00\\
instance n=20 1.alb & 1 & 0 & Optimal &  0.02 & 3 &  3.00 &  0.00\\
instance n=20 10.alb & 1 & 0 & Optimal &  0.02 & 3 &  3.00 &  0.00\\
instance n=20 100.alb & 1 & 0 & Optimal &  0.03 & 11 & 11.00 &  0.00\\
instance n=20 101.alb & 1 & 0 & Optimal &  0.29 & 13 & 13.00 &  0.00\\
instance n=20 102.alb & 1 & 0 & Optimal &  0.12 & 13 & 13.00 &  0.00\\
instance n=20 103.alb & 1 & 0 & Optimal &  0.12 & 12 & 12.00 &  0.00\\
instance n=20 104.alb & 1 & 0 & Optimal &  0.01 & 11 & 11.00 &  0.00\\
instance n=20 105.alb & 1 & 0 & Optimal &  0.02 & 12 & 12.00 &  0.00\\
instance n=20 106.alb & 1 & 0 & Optimal &  0.13 & 10 & 10.00 &  0.00\\
instance n=20 107.alb & 1 & 0 & Optimal &  0.06 & 14 & 14.00 &  0.00\\
instance n=20 108.alb & 1 & 0 & Optimal &  0.02 & 15 & 15.00 &  0.00\\
instance n=20 109.alb & 1 & 0 & Optimal &  0.03 & 12 & 12.00 &  0.00\\
instance n=20 11.alb & 1 & 0 & Optimal &  0.02 & 3 &  3.00 &  0.00\\
instance n=20 110.alb & 1 & 0 & Optimal &  0.02 & 11 & 11.00 &  0.00\\
instance n=20 111.alb & 1 & 0 & Optimal &  0.04 & 13 & 13.00 &  0.00\\
instance n=20 112.alb & 1 & 0 & Optimal &  0.02 & 11 & 11.00 &  0.00\\
instance n=20 113.alb & 1 & 0 & Optimal &  0.02 & 12 & 12.00 &  0.00\\
instance n=20 114.alb & 1 & 0 & Optimal &  0.02 & 13 & 13.00 &  0.00\\
instance n=20 115.alb & 1 & 0 & Optimal &  0.02 & 11 & 11.00 &  0.00\\
instance n=20 116.alb & 1 & 0 & Optimal &  0.03 & 5 &  5.00 &  0.00\\
instance n=20 117.alb & 1 & 0 & Optimal &  0.02 & 5 &  5.00 &  0.00\\
instance n=20 118.alb & 1 & 0 & Optimal &  0.01 & 5 &  5.00 &  0.00\\
instance n=20 119.alb & 1 & 0 & Optimal &  0.01 & 6 &  6.00 &  0.00\\
instance n=20 12.alb & 1 & 0 & Optimal &  0.01 & 3 &  3.00 &  0.00\\
instance n=20 120.alb & 1 & 0 & Optimal &  0.03 & 6 &  6.00 &  0.00\\
instance n=20 121.alb & 1 & 0 & Optimal &  0.02 & 5 &  5.00 &  0.00\\
instance n=20 122.alb & 1 & 0 & Optimal &  0.01 & 6 &  6.00 &  0.00\\
instance n=20 123.alb & 1 & 0 & Optimal &  0.02 & 5 &  5.00 &  0.00\\
instance n=20 124.alb & 1 & 0 & Optimal &  0.02 & 5 &  5.00 &  0.00\\
instance n=20 125.alb & 1 & 0 & Optimal &  0.01 & 5 &  5.00 &  0.00\\
instance n=20 126.alb & 1 & 0 & Optimal &  0.02 & 5 &  5.00 &  0.00\\
instance n=20 127.alb & 1 & 0 & Optimal &  0.02 & 4 &  4.00 &  0.00\\
instance n=20 128.alb & 1 & 0 & Optimal &  0.02 & 5 &  5.00 &  0.00\\
instance n=20 129.alb & 1 & 0 & Optimal &  0.02 & 5 &  5.00 &  0.00\\
instance n=20 13.alb & 1 & 0 & Optimal &  0.01 & 3 &  3.00 &  0.00\\
instance n=20 130.alb & 1 & 0 & Optimal &  0.02 & 6 &  6.00 &  0.00\\
instance n=20 131.alb & 1 & 0 & Optimal &  0.02 & 7 &  7.00 &  0.00\\
instance n=20 132.alb & 1 & 0 & Optimal &  0.02 & 4 &  4.00 &  0.00\\
instance n=20 133.alb & 1 & 0 & Optimal &  0.01 & 5 &  5.00 &  0.00\\
instance n=20 134.alb & 1 & 0 & Optimal &  0.11 & 6 &  6.00 &  0.00\\
instance n=20 135.alb & 1 & 0 & Optimal &  0.11 & 6 &  6.00 &  0.00\\
instance n=20 136.alb & 1 & 0 & Optimal &  0.01 & 6 &  6.00 &  0.00\\
instance n=20 137.alb & 1 & 0 & Optimal &  0.02 & 5 &  5.00 &  0.00\\
instance n=20 138.alb & 1 & 0 & Optimal &  0.01 & 5 &  5.00 &  0.00\\
instance n=20 139.alb & 1 & 0 & Optimal &  0.02 & 5 &  5.00 &  0.00\\
instance n=20 14.alb & 1 & 0 & Optimal &  0.01 & 3 &  3.00 &  0.00\\
instance n=20 140.alb & 1 & 0 & Optimal &  0.01 & 5 &  5.00 &  0.00\\
instance n=20 141.alb & 1 & 0 & Optimal &  0.02 & 3 &  3.00 &  0.00\\
instance n=20 142.alb & 1 & 0 & Optimal &  0.01 & 3 &  3.00 &  0.00\\
instance n=20 143.alb & 1 & 0 & Optimal &  0.01 & 3 &  3.00 &  0.00\\
instance n=20 144.alb & 1 & 0 & Optimal &  0.01 & 4 &  4.00 &  0.00\\
instance n=20 145.alb & 1 & 0 & Optimal &  0.02 & 3 &  3.00 &  0.00\\
instance n=20 146.alb & 1 & 0 & Optimal &  0.02 & 3 &  3.00 &  0.00\\
instance n=20 147.alb & 1 & 0 & Optimal &  0.01 & 3 &  3.00 &  0.00\\
instance n=20 148.alb & 1 & 0 & Optimal &  0.02 & 3 &  3.00 &  0.00\\
instance n=20 149.alb & 1 & 0 & Optimal &  0.01 & 3 &  3.00 &  0.00\\
instance n=20 15.alb & 1 & 0 & Optimal &  0.02 & 3 &  3.00 &  0.00\\
instance n=20 150.alb & 1 & 0 & Optimal &  0.02 & 3 &  3.00 &  0.00\\
instance n=20 151.alb & 1 & 0 & Optimal &  0.02 & 3 &  3.00 &  0.00\\
instance n=20 152.alb & 1 & 0 & Optimal &  0.01 & 3 &  3.00 &  0.00\\
instance n=20 153.alb & 1 & 0 & Optimal &  0.01 & 3 &  3.00 &  0.00\\
instance n=20 154.alb & 1 & 0 & Optimal &  0.02 & 3 &  3.00 &  0.00\\
instance n=20 155.alb & 1 & 0 & Optimal &  0.02 & 3 &  3.00 &  0.00\\
instance n=20 156.alb & 1 & 0 & Optimal &  0.01 & 3 &  3.00 &  0.00\\
instance n=20 157.alb & 1 & 0 & Optimal &  0.02 & 3 &  3.00 &  0.00\\
instance n=20 158.alb & 1 & 0 & Optimal &  0.01 & 3 &  3.00 &  0.00\\
instance n=20 159.alb & 1 & 0 & Optimal &  0.01 & 3 &  3.00 &  0.00\\
instance n=20 16.alb & 1 & 0 & Optimal &  0.02 & 12 & 12.00 &  0.00\\
instance n=20 160.alb & 1 & 0 & Optimal &  0.02 & 3 &  3.00 &  0.00\\
instance n=20 161.alb & 1 & 0 & Optimal &  0.02 & 3 &  3.00 &  0.00\\
instance n=20 162.alb & 1 & 0 & Optimal &  0.01 & 3 &  3.00 &  0.00\\
instance n=20 163.alb & 1 & 0 & Optimal &  0.02 & 3 &  3.00 &  0.00\\
instance n=20 164.alb & 1 & 0 & Optimal &  0.11 & 4 &  4.00 &  0.00\\
instance n=20 165.alb & 1 & 0 & Optimal &  0.01 & 3 &  3.00 &  0.00\\
instance n=20 166.alb & 1 & 0 & Optimal &  0.13 & 12 & 12.00 &  0.00\\
instance n=20 167.alb & 1 & 0 & Optimal &  0.03 & 11 & 11.00 &  0.00\\
instance n=20 168.alb & 1 & 0 & Optimal &  0.02 & 10 & 10.00 &  0.00\\
instance n=20 169.alb & 1 & 0 & Optimal &  0.03 & 11 & 11.00 &  0.00\\
instance n=20 17.alb & 1 & 0 & Optimal &  0.03 & 10 & 10.00 &  0.00\\
instance n=20 170.alb & 1 & 0 & Optimal &  0.02 & 11 & 11.00 &  0.00\\
instance n=20 171.alb & 1 & 0 & Optimal &  0.16 & 13 & 13.00 &  0.00\\
instance n=20 172.alb & 1 & 0 & Optimal &  0.01 & 11 & 11.00 &  0.00\\
instance n=20 173.alb & 1 & 0 & Optimal &  0.05 & 11 & 11.00 &  0.00\\
instance n=20 174.alb & 1 & 0 & Optimal &  0.04 & 12 & 12.00 &  0.00\\
instance n=20 175.alb & 1 & 0 & Optimal &  0.11 & 10 & 10.00 &  0.00\\
instance n=20 176.alb & 1 & 0 & Optimal &  0.02 & 11 & 11.00 &  0.00\\
instance n=20 177.alb & 1 & 0 & Optimal &  0.36 & 10 & 10.00 &  0.00\\
instance n=20 178.alb & 1 & 0 & Optimal &  0.02 & 11 & 11.00 &  0.00\\
instance n=20 179.alb & 1 & 0 & Optimal &  0.01 & 11 & 11.00 &  0.00\\
instance n=20 18.alb & 1 & 0 & Optimal &  0.02 & 11 & 11.00 &  0.00\\
instance n=20 180.alb & 1 & 0 & Optimal &  0.02 & 13 & 13.00 &  0.00\\
instance n=20 181.alb & 1 & 0 & Optimal &  0.02 & 11 & 11.00 &  0.00\\
instance n=20 182.alb & 1 & 0 & Optimal &  0.02 & 11 & 11.00 &  0.00\\
instance n=20 183.alb & 1 & 0 & Optimal &  0.12 & 13 & 13.00 &  0.00\\
instance n=20 184.alb & 1 & 0 & Optimal &  0.01 & 12 & 12.00 &  0.00\\
instance n=20 185.alb & 1 & 0 & Optimal &  0.02 & 15 & 15.00 &  0.00\\
instance n=20 186.alb & 1 & 0 & Optimal &  0.82 & 14 & 14.00 &  0.00\\
instance n=20 187.alb & 1 & 0 & Optimal &  0.03 & 10 & 10.00 &  0.00\\
instance n=20 188.alb & 1 & 0 & Optimal &  0.04 & 11 & 11.00 &  0.00\\
instance n=20 189.alb & 1 & 0 & Optimal &  0.01 & 13 & 13.00 &  0.00\\
instance n=20 19.alb & 1 & 0 & Optimal &  0.05 & 14 & 14.00 &  0.00\\
instance n=20 190.alb & 1 & 0 & Optimal &  0.05 & 15 & 15.00 &  0.00\\
instance n=20 191.alb & 1 & 0 & Optimal &  0.01 & 4 &  4.00 &  0.00\\
instance n=20 192.alb & 1 & 0 & Optimal &  0.01 & 5 &  5.00 &  0.00\\
instance n=20 193.alb & 1 & 0 & Optimal &  0.01 & 5 &  5.00 &  0.00\\
instance n=20 194.alb & 1 & 0 & Optimal &  0.04 & 6 &  6.00 &  0.00\\
instance n=20 195.alb & 1 & 0 & Optimal &  0.02 & 6 &  6.00 &  0.00\\
instance n=20 196.alb & 1 & 0 & Optimal &  0.03 & 5 &  5.00 &  0.00\\
instance n=20 197.alb & 1 & 0 & Optimal &  0.02 & 4 &  4.00 &  0.00\\
instance n=20 198.alb & 1 & 0 & Optimal &  0.02 & 6 &  6.00 &  0.00\\
instance n=20 199.alb & 1 & 0 & Optimal &  0.10 & 5 &  5.00 &  0.00\\
instance n=20 2.alb & 1 & 0 & Optimal &  0.01 & 3 &  3.00 &  0.00\\
instance n=20 20.alb & 1 & 0 & Optimal &  0.03 & 11 & 11.00 &  0.00\\
instance n=20 200.alb & 1 & 0 & Optimal &  0.01 & 6 &  6.00 &  0.00\\
instance n=20 201.alb & 1 & 0 & Optimal &  0.02 & 6 &  6.00 &  0.00\\
instance n=20 202.alb & 1 & 0 & Optimal &  0.10 & 4 &  4.00 &  0.00\\
instance n=20 203.alb & 1 & 0 & Optimal &  0.02 & 4 &  4.00 &  0.00\\
instance n=20 204.alb & 1 & 0 & Optimal &  0.11 & 5 &  5.00 &  0.00\\
instance n=20 205.alb & 1 & 0 & Optimal &  0.02 & 6 &  6.00 &  0.00\\
instance n=20 206.alb & 1 & 0 & Optimal &  0.02 & 5 &  5.00 &  0.00\\
instance n=20 207.alb & 1 & 0 & Optimal &  0.06 & 6 &  6.00 &  0.00\\
instance n=20 208.alb & 1 & 0 & Optimal &  0.02 & 5 &  5.00 &  0.00\\
instance n=20 209.alb & 1 & 0 & Optimal &  0.03 & 4 &  4.00 &  0.00\\
instance n=20 21.alb & 1 & 0 & Optimal &  0.02 & 14 & 14.00 &  0.00\\
instance n=20 210.alb & 1 & 0 & Optimal &  0.01 & 5 &  5.00 &  0.00\\
instance n=20 211.alb & 1 & 0 & Optimal &  0.01 & 5 &  5.00 &  0.00\\
instance n=20 212.alb & 1 & 0 & Optimal &  0.01 & 5 &  5.00 &  0.00\\
instance n=20 213.alb & 1 & 0 & Optimal &  0.01 & 5 &  5.00 &  0.00\\
instance n=20 214.alb & 1 & 0 & Optimal &  0.02 & 5 &  5.00 &  0.00\\
instance n=20 215.alb & 1 & 0 & Optimal &  0.02 & 5 &  5.00 &  0.00\\
instance n=20 216.alb & 1 & 0 & Optimal &  0.01 & 3 &  3.00 &  0.00\\
instance n=20 217.alb & 1 & 0 & Optimal &  0.01 & 4 &  4.00 &  0.00\\
instance n=20 218.alb & 1 & 0 & Optimal &  0.01 & 3 &  3.00 &  0.00\\
instance n=20 219.alb & 1 & 0 & Optimal &  0.02 & 3 &  3.00 &  0.00\\
instance n=20 22.alb & 1 & 0 & Optimal &  0.02 & 12 & 12.00 &  0.00\\
instance n=20 220.alb & 1 & 0 & Optimal &  0.03 & 3 &  3.00 &  0.00\\
instance n=20 221.alb & 1 & 0 & Optimal &  0.01 & 3 &  3.00 &  0.00\\
instance n=20 222.alb & 1 & 0 & Optimal &  0.02 & 3 &  3.00 &  0.00\\
instance n=20 223.alb & 1 & 0 & Optimal &  0.01 & 3 &  3.00 &  0.00\\
instance n=20 224.alb & 1 & 0 & Optimal &  0.01 & 3 &  3.00 &  0.00\\
instance n=20 225.alb & 1 & 0 & Optimal &  0.02 & 3 &  3.00 &  0.00\\
instance n=20 226.alb & 1 & 0 & Optimal &  0.01 & 3 &  3.00 &  0.00\\
instance n=20 227.alb & 1 & 0 & Optimal &  0.01 & 3 &  3.00 &  0.00\\
instance n=20 228.alb & 1 & 0 & Optimal &  0.01 & 2 &  2.00 &  0.00\\
instance n=20 229.alb & 1 & 0 & Optimal &  0.02 & 3 &  3.00 &  0.00\\
instance n=20 23.alb & 1 & 0 & Optimal &  0.09 & 13 & 13.00 &  0.00\\
instance n=20 230.alb & 1 & 0 & Optimal &  0.02 & 3 &  3.00 &  0.00\\
instance n=20 231.alb & 1 & 0 & Optimal &  0.02 & 3 &  3.00 &  0.00\\
instance n=20 232.alb & 1 & 0 & Optimal &  0.01 & 3 &  3.00 &  0.00\\
instance n=20 233.alb & 1 & 0 & Optimal &  0.02 & 3 &  3.00 &  0.00\\
instance n=20 234.alb & 1 & 0 & Optimal &  0.01 & 3 &  3.00 &  0.00\\
instance n=20 235.alb & 1 & 0 & Optimal &  0.01 & 3 &  3.00 &  0.00\\
instance n=20 236.alb & 1 & 0 & Optimal &  0.02 & 3 &  3.00 &  0.00\\
instance n=20 237.alb & 1 & 0 & Optimal &  0.01 & 3 &  3.00 &  0.00\\
instance n=20 238.alb & 1 & 0 & Optimal &  0.01 & 3 &  3.00 &  0.00\\
instance n=20 239.alb & 1 & 0 & Optimal &  0.01 & 3 &  3.00 &  0.00\\
instance n=20 24.alb & 1 & 0 & Optimal &  0.02 & 11 & 11.00 &  0.00\\
instance n=20 240.alb & 1 & 0 & Optimal &  0.02 & 3 &  3.00 &  0.00\\
instance n=20 241.alb & 1 & 0 & Optimal &  0.10 & 13 & 13.00 &  0.00\\
instance n=20 242.alb & 1 & 0 & Optimal &  0.02 & 12 & 12.00 &  0.00\\
instance n=20 243.alb & 1 & 0 & Optimal &  0.11 & 10 & 10.00 &  0.00\\
instance n=20 244.alb & 1 & 0 & Optimal &  0.02 & 11 & 11.00 &  0.00\\
instance n=20 245.alb & 1 & 0 & Optimal &  0.02 & 13 & 13.00 &  0.00\\
instance n=20 246.alb & 1 & 0 & Optimal &  0.03 & 13 & 13.00 &  0.00\\
instance n=20 247.alb & 1 & 0 & Optimal &  0.12 & 11 & 11.00 &  0.00\\
instance n=20 248.alb & 1 & 0 & Optimal &  0.02 & 11 & 11.00 &  0.00\\
instance n=20 249.alb & 1 & 0 & Optimal &  0.02 & 13 & 13.00 &  0.00\\
instance n=20 25.alb & 1 & 0 & Optimal &  0.11 & 11 & 11.00 &  0.00\\
instance n=20 250.alb & 1 & 0 & Optimal &  0.02 & 10 & 10.00 &  0.00\\
instance n=20 251.alb & 1 & 0 & Optimal &  0.01 & 12 & 12.00 &  0.00\\
instance n=20 252.alb & 1 & 0 & Optimal &  0.03 & 11 & 11.00 &  0.00\\
instance n=20 253.alb & 1 & 0 & Optimal &  0.01 & 13 & 13.00 &  0.00\\
instance n=20 254.alb & 1 & 0 & Optimal &  0.03 & 12 & 12.00 &  0.00\\
instance n=20 255.alb & 1 & 0 & Optimal &  0.03 & 13 & 13.00 &  0.00\\
instance n=20 256.alb & 1 & 0 & Optimal &  0.02 & 14 & 14.00 &  0.00\\
instance n=20 257.alb & 1 & 0 & Optimal &  0.10 & 10 & 10.00 &  0.00\\
instance n=20 258.alb & 1 & 0 & Optimal &  0.02 & 13 & 13.00 &  0.00\\
instance n=20 259.alb & 1 & 0 & Optimal &  0.02 & 13 & 13.00 &  0.00\\
instance n=20 26.alb & 1 & 0 & Optimal &  0.02 & 12 & 12.00 &  0.00\\
instance n=20 260.alb & 1 & 0 & Optimal &  0.01 & 12 & 12.00 &  0.00\\
instance n=20 261.alb & 1 & 0 & Optimal &  0.03 & 12 & 12.00 &  0.00\\
instance n=20 262.alb & 1 & 0 & Optimal &  0.02 & 11 & 11.00 &  0.00\\
instance n=20 263.alb & 1 & 0 & Optimal &  0.03 & 12 & 12.00 &  0.00\\
instance n=20 264.alb & 1 & 0 & Optimal &  0.10 & 12 & 12.00 &  0.00\\
instance n=20 265.alb & 1 & 0 & Optimal &  0.02 & 12 & 12.00 &  0.00\\
instance n=20 266.alb & 1 & 0 & Optimal &  0.03 & 5 &  5.00 &  0.00\\
instance n=20 267.alb & 1 & 0 & Optimal &  0.01 & 6 &  6.00 &  0.00\\
instance n=20 268.alb & 1 & 0 & Optimal &  0.01 & 6 &  6.00 &  0.00\\
instance n=20 269.alb & 1 & 0 & Optimal &  0.10 & 7 &  7.00 &  0.00\\
instance n=20 27.alb & 1 & 0 & Optimal &  0.03 & 13 & 13.00 &  0.00\\
instance n=20 270.alb & 1 & 0 & Optimal &  0.10 & 7 &  7.00 &  0.00\\
instance n=20 271.alb & 1 & 0 & Optimal &  0.01 & 6 &  6.00 &  0.00\\
instance n=20 272.alb & 1 & 0 & Optimal &  0.01 & 5 &  5.00 &  0.00\\
instance n=20 273.alb & 1 & 0 & Optimal &  0.02 & 5 &  5.00 &  0.00\\
instance n=20 274.alb & 1 & 0 & Optimal &  0.09 & 6 &  6.00 &  0.00\\
instance n=20 275.alb & 1 & 0 & Optimal &  0.03 & 5 &  5.00 &  0.00\\
instance n=20 276.alb & 1 & 0 & Optimal &  0.01 & 4 &  4.00 &  0.00\\
instance n=20 277.alb & 1 & 0 & Optimal &  0.01 & 4 &  4.00 &  0.00\\
instance n=20 278.alb & 1 & 0 & Optimal &  0.09 & 6 &  6.00 &  0.00\\
instance n=20 279.alb & 1 & 0 & Optimal &  0.02 & 6 &  6.00 &  0.00\\
instance n=20 28.alb & 1 & 0 & Optimal &  0.02 & 12 & 12.00 &  0.00\\
instance n=20 280.alb & 1 & 0 & Optimal &  0.01 & 5 &  5.00 &  0.00\\
instance n=20 281.alb & 1 & 0 & Optimal &  0.02 & 4 &  4.00 &  0.00\\
instance n=20 282.alb & 1 & 0 & Optimal &  0.02 & 4 &  4.00 &  0.00\\
instance n=20 283.alb & 1 & 0 & Optimal &  0.01 & 5 &  5.00 &  0.00\\
instance n=20 284.alb & 1 & 0 & Optimal &  0.02 & 5 &  5.00 &  0.00\\
instance n=20 285.alb & 1 & 0 & Optimal &  0.01 & 5 &  5.00 &  0.00\\
instance n=20 286.alb & 1 & 0 & Optimal &  0.02 & 5 &  5.00 &  0.00\\
instance n=20 287.alb & 1 & 0 & Optimal &  0.02 & 5 &  5.00 &  0.00\\
instance n=20 288.alb & 1 & 0 & Optimal &  0.02 & 6 &  6.00 &  0.00\\
instance n=20 289.alb & 1 & 0 & Optimal &  0.01 & 5 &  5.00 &  0.00\\
instance n=20 29.alb & 1 & 0 & Optimal &  0.05 & 10 & 10.00 &  0.00\\
instance n=20 290.alb & 1 & 0 & Optimal &  0.01 & 5 &  5.00 &  0.00\\
instance n=20 291.alb & 1 & 0 & Optimal &  0.02 & 3 &  3.00 &  0.00\\
instance n=20 292.alb & 1 & 0 & Optimal &  0.03 & 3 &  3.00 &  0.00\\
instance n=20 293.alb & 1 & 0 & Optimal &  0.01 & 3 &  3.00 &  0.00\\
instance n=20 294.alb & 1 & 0 & Optimal &  0.03 & 3 &  3.00 &  0.00\\
instance n=20 295.alb & 1 & 0 & Optimal &  0.02 & 3 &  3.00 &  0.00\\
instance n=20 296.alb & 1 & 0 & Optimal &  0.01 & 3 &  3.00 &  0.00\\
instance n=20 297.alb & 1 & 0 & Optimal &  0.01 & 3 &  3.00 &  0.00\\
instance n=20 298.alb & 1 & 0 & Optimal &  0.01 & 3 &  3.00 &  0.00\\
instance n=20 299.alb & 1 & 0 & Optimal &  0.02 & 3 &  3.00 &  0.00\\
instance n=20 3.alb & 1 & 0 & Optimal &  0.01 & 3 &  3.00 &  0.00\\
instance n=20 30.alb & 1 & 0 & Optimal &  0.04 & 16 & 16.00 &  0.00\\
instance n=20 300.alb & 1 & 0 & Optimal &  0.02 & 4 &  4.00 &  0.00\\
instance n=20 301.alb & 1 & 0 & Optimal &  0.01 & 3 &  3.00 &  0.00\\
instance n=20 302.alb & 1 & 0 & Optimal &  0.02 & 3 &  3.00 &  0.00\\
instance n=20 303.alb & 1 & 0 & Optimal &  0.01 & 3 &  3.00 &  0.00\\
instance n=20 304.alb & 1 & 0 & Optimal &  0.01 & 3 &  3.00 &  0.00\\
instance n=20 305.alb & 1 & 0 & Optimal &  0.01 & 3 &  3.00 &  0.00\\
instance n=20 306.alb & 1 & 0 & Optimal &  0.01 & 3 &  3.00 &  0.00\\
instance n=20 307.alb & 1 & 0 & Optimal &  0.02 & 3 &  3.00 &  0.00\\
instance n=20 308.alb & 1 & 0 & Optimal &  0.01 & 3 &  3.00 &  0.00\\
instance n=20 309.alb & 1 & 0 & Optimal &  0.02 & 3 &  3.00 &  0.00\\
instance n=20 31.alb & 1 & 0 & Optimal &  0.07 & 12 & 12.00 &  0.00\\
instance n=20 310.alb & 1 & 0 & Optimal &  0.02 & 3 &  3.00 &  0.00\\
instance n=20 311.alb & 1 & 0 & Optimal &  0.01 & 3 &  3.00 &  0.00\\
instance n=20 312.alb & 1 & 0 & Optimal &  0.02 & 4 &  4.00 &  0.00\\
instance n=20 313.alb & 1 & 0 & Optimal &  0.01 & 3 &  3.00 &  0.00\\
instance n=20 314.alb & 1 & 0 & Optimal &  0.02 & 3 &  3.00 &  0.00\\
instance n=20 315.alb & 1 & 0 & Optimal &  0.02 & 3 &  3.00 &  0.00\\
instance n=20 316.alb & 1 & 0 & Optimal &  0.06 & 10 & 10.00 &  0.00\\
instance n=20 317.alb & 1 & 0 & Optimal &  0.06 & 10 & 10.00 &  0.00\\
instance n=20 318.alb & 1 & 0 & Optimal &  0.02 & 10 & 10.00 &  0.00\\
instance n=20 319.alb & 1 & 0 & Optimal &  0.18 & 14 & 14.00 &  0.00\\
instance n=20 32.alb & 1 & 0 & Optimal &  0.07 & 13 & 13.00 &  0.00\\
instance n=20 320.alb & 1 & 0 & Optimal &  0.01 & 12 & 12.00 &  0.00\\
instance n=20 321.alb & 1 & 0 & Optimal &  1.08 & 14 & 14.00 &  0.00\\
instance n=20 322.alb & 1 & 0 & Optimal &  0.31 & 12 & 12.00 &  0.00\\
instance n=20 323.alb & 1 & 0 & Optimal &  0.02 & 13 & 13.00 &  0.00\\
instance n=20 324.alb & 1 & 0 & Optimal &  0.07 & 9 &  9.00 &  0.00\\
instance n=20 325.alb & 1 & 0 & Optimal &  0.02 & 14 & 14.00 &  0.00\\
instance n=20 326.alb & 1 & 0 & Optimal &  0.53 & 14 & 14.00 &  0.00\\
instance n=20 327.alb & 1 & 0 & Optimal &  1.12 & 13 & 13.00 &  0.00\\
instance n=20 328.alb & 1 & 0 & Optimal &  0.01 & 13 & 13.00 &  0.00\\
instance n=20 329.alb & 1 & 0 & Optimal &  0.06 & 10 & 10.00 &  0.00\\
instance n=20 33.alb & 1 & 0 & Optimal &  0.03 & 11 & 11.00 &  0.00\\
instance n=20 330.alb & 1 & 0 & Optimal &  0.09 & 12 & 12.00 &  0.00\\
instance n=20 331.alb & 1 & 0 & Optimal &  0.08 & 13 & 13.00 &  0.00\\
instance n=20 332.alb & 1 & 0 & Optimal &  0.05 & 13 & 13.00 &  0.00\\
instance n=20 333.alb & 1 & 0 & Optimal &  0.03 & 11 & 11.00 &  0.00\\
instance n=20 334.alb & 1 & 0 & Optimal &  0.07 & 10 & 10.00 &  0.00\\
instance n=20 335.alb & 1 & 0 & Optimal &  0.02 & 14 & 14.00 &  0.00\\
instance n=20 336.alb & 1 & 0 & Optimal &  0.01 & 11 & 11.00 &  0.00\\
instance n=20 337.alb & 1 & 0 & Optimal &  0.05 & 10 & 10.00 &  0.00\\
instance n=20 338.alb & 1 & 0 & Optimal &  0.10 & 14 & 14.00 &  0.00\\
instance n=20 339.alb & 1 & 0 & Optimal &  0.01 & 13 & 13.00 &  0.00\\
instance n=20 34.alb & 1 & 0 & Optimal &  0.11 & 12 & 12.00 &  0.00\\
instance n=20 340.alb & 1 & 0 & Optimal &  0.11 & 11 & 11.00 &  0.00\\
instance n=20 341.alb & 1 & 0 & Optimal &  0.02 & 6 &  6.00 &  0.00\\
instance n=20 342.alb & 1 & 0 & Optimal &  0.02 & 6 &  6.00 &  0.00\\
instance n=20 343.alb & 1 & 0 & Optimal &  0.06 & 6 &  6.00 &  0.00\\
instance n=20 344.alb & 1 & 0 & Optimal &  0.02 & 6 &  6.00 &  0.00\\
instance n=20 345.alb & 1 & 0 & Optimal &  0.02 & 4 &  4.00 &  0.00\\
instance n=20 346.alb & 1 & 0 & Optimal &  0.02 & 5 &  5.00 &  0.00\\
instance n=20 347.alb & 1 & 0 & Optimal &  0.01 & 6 &  6.00 &  0.00\\
instance n=20 348.alb & 1 & 0 & Optimal &  0.02 & 5 &  5.00 &  0.00\\
instance n=20 349.alb & 1 & 0 & Optimal &  0.02 & 5 &  5.00 &  0.00\\
instance n=20 35.alb & 1 & 0 & Optimal &  0.04 & 12 & 12.00 &  0.00\\
instance n=20 350.alb & 1 & 0 & Optimal &  0.01 & 5 &  5.00 &  0.00\\
instance n=20 351.alb & 1 & 0 & Optimal &  0.02 & 5 &  5.00 &  0.00\\
instance n=20 352.alb & 1 & 0 & Optimal &  0.01 & 4 &  4.00 &  0.00\\
instance n=20 353.alb & 1 & 0 & Optimal &  0.02 & 6 &  6.00 &  0.00\\
instance n=20 354.alb & 1 & 0 & Optimal &  0.02 & 6 &  6.00 &  0.00\\
instance n=20 355.alb & 1 & 0 & Optimal &  0.02 & 5 &  5.00 &  0.00\\
instance n=20 356.alb & 1 & 0 & Optimal &  0.02 & 5 &  5.00 &  0.00\\
instance n=20 357.alb & 1 & 0 & Optimal &  0.02 & 5 &  5.00 &  0.00\\
instance n=20 358.alb & 1 & 0 & Optimal &  0.03 & 4 &  4.00 &  0.00\\
instance n=20 359.alb & 1 & 0 & Optimal &  0.02 & 4 &  4.00 &  0.00\\
instance n=20 36.alb & 1 & 0 & Optimal &  0.02 & 13 & 13.00 &  0.00\\
instance n=20 360.alb & 1 & 0 & Optimal &  0.03 & 6 &  6.00 &  0.00\\
instance n=20 361.alb & 1 & 0 & Optimal &  0.03 & 5 &  5.00 &  0.00\\
instance n=20 362.alb & 1 & 0 & Optimal &  0.02 & 5 &  5.00 &  0.00\\
instance n=20 363.alb & 1 & 0 & Optimal &  0.92 & 7 &  7.00 &  0.00\\
instance n=20 364.alb & 1 & 0 & Optimal &  0.01 & 4 &  4.00 &  0.00\\
instance n=20 365.alb & 1 & 0 & Optimal &  0.01 & 5 &  5.00 &  0.00\\
instance n=20 366.alb & 1 & 0 & Optimal &  0.01 & 3 &  3.00 &  0.00\\
instance n=20 367.alb & 1 & 0 & Optimal &  0.02 & 3 &  3.00 &  0.00\\
instance n=20 368.alb & 1 & 0 & Optimal &  0.02 & 3 &  3.00 &  0.00\\
instance n=20 369.alb & 1 & 0 & Optimal &  0.02 & 3 &  3.00 &  0.00\\
instance n=20 37.alb & 1 & 0 & Optimal &  0.01 & 12 & 12.00 &  0.00\\
instance n=20 370.alb & 1 & 0 & Optimal &  0.02 & 3 &  3.00 &  0.00\\
instance n=20 371.alb & 1 & 0 & Optimal &  0.02 & 3 &  3.00 &  0.00\\
instance n=20 372.alb & 1 & 0 & Optimal &  0.02 & 3 &  3.00 &  0.00\\
instance n=20 373.alb & 1 & 0 & Optimal &  0.02 & 3 &  3.00 &  0.00\\
instance n=20 374.alb & 1 & 0 & Optimal &  0.01 & 3 &  3.00 &  0.00\\
instance n=20 375.alb & 1 & 0 & Optimal &  0.01 & 3 &  3.00 &  0.00\\
instance n=20 376.alb & 1 & 0 & Optimal &  0.01 & 3 &  3.00 &  0.00\\
instance n=20 377.alb & 1 & 0 & Optimal &  0.02 & 3 &  3.00 &  0.00\\
instance n=20 378.alb & 1 & 0 & Optimal &  0.02 & 3 &  3.00 &  0.00\\
instance n=20 379.alb & 1 & 0 & Optimal &  0.02 & 4 &  4.00 &  0.00\\
instance n=20 38.alb & 1 & 0 & Optimal &  0.02 & 12 & 12.00 &  0.00\\
instance n=20 380.alb & 1 & 0 & Optimal &  0.02 & 3 &  3.00 &  0.00\\
instance n=20 381.alb & 1 & 0 & Optimal &  0.01 & 3 &  3.00 &  0.00\\
instance n=20 382.alb & 1 & 0 & Optimal &  0.02 & 4 &  4.00 &  0.00\\
instance n=20 383.alb & 1 & 0 & Optimal &  0.01 & 3 &  3.00 &  0.00\\
instance n=20 384.alb & 1 & 0 & Optimal &  0.02 & 3 &  3.00 &  0.00\\
instance n=20 385.alb & 1 & 0 & Optimal &  0.01 & 3 &  3.00 &  0.00\\
instance n=20 386.alb & 1 & 0 & Optimal &  0.01 & 3 &  3.00 &  0.00\\
instance n=20 387.alb & 1 & 0 & Optimal &  0.02 & 3 &  3.00 &  0.00\\
instance n=20 388.alb & 1 & 0 & Optimal &  0.01 & 3 &  3.00 &  0.00\\
instance n=20 389.alb & 1 & 0 & Optimal &  0.01 & 3 &  3.00 &  0.00\\
instance n=20 39.alb & 1 & 0 & Optimal &  0.05 & 13 & 13.00 &  0.00\\
instance n=20 390.alb & 1 & 0 & Optimal &  0.02 & 3 &  3.00 &  0.00\\
instance n=20 391.alb & 1 & 0 & Optimal &  0.03 & 11 & 11.00 &  0.00\\
instance n=20 392.alb & 1 & 0 & Optimal &  0.12 & 14 & 14.00 &  0.00\\
instance n=20 393.alb & 1 & 0 & Optimal &  0.11 & 11 & 11.00 &  0.00\\
instance n=20 394.alb & 1 & 0 & Optimal &  0.12 & 12 & 12.00 &  0.00\\
instance n=20 395.alb & 1 & 0 & Optimal &  0.02 & 12 & 12.00 &  0.00\\
instance n=20 396.alb & 1 & 0 & Optimal &  0.10 & 13 & 13.00 &  0.00\\
instance n=20 397.alb & 1 & 0 & Optimal &  0.11 & 10 & 10.00 &  0.00\\
instance n=20 398.alb & 1 & 0 & Optimal &  0.02 & 11 & 11.00 &  0.00\\
instance n=20 399.alb & 1 & 0 & Optimal &  0.01 & 13 & 13.00 &  0.00\\
instance n=20 4.alb & 1 & 0 & Optimal &  0.02 & 3 &  3.00 &  0.00\\
instance n=20 40.alb & 1 & 0 & Optimal &  0.03 & 12 & 12.00 &  0.00\\
instance n=20 400.alb & 1 & 0 & Optimal &  0.03 & 12 & 12.00 &  0.00\\
instance n=20 401.alb & 1 & 0 & Optimal &  0.12 & 12 & 12.00 &  0.00\\
instance n=20 402.alb & 1 & 0 & Optimal &  0.02 & 12 & 12.00 &  0.00\\
instance n=20 403.alb & 1 & 0 & Optimal &  0.01 & 12 & 12.00 &  0.00\\
instance n=20 404.alb & 1 & 0 & Optimal &  0.10 & 10 & 10.00 &  0.00\\
instance n=20 405.alb & 1 & 0 & Optimal &  0.02 & 12 & 12.00 &  0.00\\
instance n=20 406.alb & 1 & 0 & Optimal &  0.01 & 14 & 14.00 &  0.00\\
instance n=20 407.alb & 1 & 0 & Optimal &  0.03 & 10 & 10.00 &  0.00\\
instance n=20 408.alb & 1 & 0 & Optimal &  0.11 & 14 & 14.00 &  0.00\\
instance n=20 409.alb & 1 & 0 & Optimal &  0.10 & 12 & 12.00 &  0.00\\
instance n=20 41.alb & 1 & 0 & Optimal &  0.01 & 6 &  6.00 &  0.00\\
instance n=20 410.alb & 1 & 0 & Optimal &  0.03 & 11 & 11.00 &  0.00\\
instance n=20 411.alb & 1 & 0 & Optimal &  0.11 & 15 & 15.00 &  0.00\\
instance n=20 412.alb & 1 & 0 & Optimal &  0.12 & 11 & 11.00 &  0.00\\
instance n=20 413.alb & 1 & 0 & Optimal &  0.03 & 10 & 10.00 &  0.00\\
instance n=20 414.alb & 1 & 0 & Optimal &  0.11 & 12 & 12.00 &  0.00\\
instance n=20 415.alb & 1 & 0 & Optimal &  0.02 & 10 & 10.00 &  0.00\\
instance n=20 416.alb & 1 & 0 & Optimal &  0.02 & 6 &  6.00 &  0.00\\
instance n=20 417.alb & 1 & 0 & Optimal &  0.01 & 5 &  5.00 &  0.00\\
instance n=20 418.alb & 1 & 0 & Optimal &  0.01 & 6 &  6.00 &  0.00\\
instance n=20 419.alb & 1 & 0 & Optimal &  0.02 & 4 &  4.00 &  0.00\\
instance n=20 42.alb & 1 & 0 & Optimal &  0.02 & 5 &  5.00 &  0.00\\
instance n=20 420.alb & 1 & 0 & Optimal &  0.02 & 5 &  5.00 &  0.00\\
instance n=20 421.alb & 1 & 0 & Optimal &  0.03 & 6 &  6.00 &  0.00\\
instance n=20 422.alb & 1 & 0 & Optimal &  0.01 & 4 &  4.00 &  0.00\\
instance n=20 423.alb & 1 & 0 & Optimal &  0.02 & 6 &  6.00 &  0.00\\
instance n=20 424.alb & 1 & 0 & Optimal &  0.01 & 5 &  5.00 &  0.00\\
instance n=20 425.alb & 1 & 0 & Optimal &  0.02 & 6 &  6.00 &  0.00\\
instance n=20 426.alb & 1 & 0 & Optimal &  0.01 & 5 &  5.00 &  0.00\\
instance n=20 427.alb & 1 & 0 & Optimal &  0.02 & 6 &  6.00 &  0.00\\
instance n=20 428.alb & 1 & 0 & Optimal &  0.02 & 5 &  5.00 &  0.00\\
instance n=20 429.alb & 1 & 0 & Optimal &  0.02 & 4 &  4.00 &  0.00\\
instance n=20 43.alb & 1 & 0 & Optimal &  0.02 & 5 &  5.00 &  0.00\\
instance n=20 430.alb & 1 & 0 & Optimal &  0.02 & 5 &  5.00 &  0.00\\
instance n=20 431.alb & 1 & 0 & Optimal &  0.02 & 6 &  6.00 &  0.00\\
instance n=20 432.alb & 1 & 0 & Optimal &  0.01 & 5 &  5.00 &  0.00\\
instance n=20 433.alb & 1 & 0 & Optimal &  0.02 & 5 &  5.00 &  0.00\\
instance n=20 434.alb & 1 & 0 & Optimal &  0.01 & 5 &  5.00 &  0.00\\
instance n=20 435.alb & 1 & 0 & Optimal &  0.01 & 7 &  7.00 &  0.00\\
instance n=20 436.alb & 1 & 0 & Optimal &  0.02 & 5 &  5.00 &  0.00\\
instance n=20 437.alb & 1 & 0 & Optimal &  0.01 & 5 &  5.00 &  0.00\\
instance n=20 438.alb & 1 & 0 & Optimal &  0.02 & 6 &  6.00 &  0.00\\
instance n=20 439.alb & 1 & 0 & Optimal &  0.02 & 5 &  5.00 &  0.00\\
instance n=20 44.alb & 1 & 0 & Optimal &  0.01 & 5 &  5.00 &  0.00\\
instance n=20 440.alb & 1 & 0 & Optimal &  0.01 & 5 &  5.00 &  0.00\\
instance n=20 441.alb & 1 & 0 & Optimal &  0.01 & 3 &  3.00 &  0.00\\
instance n=20 442.alb & 1 & 0 & Optimal &  0.02 & 3 &  3.00 &  0.00\\
instance n=20 443.alb & 1 & 0 & Optimal &  0.02 & 3 &  3.00 &  0.00\\
instance n=20 444.alb & 1 & 0 & Optimal &  0.03 & 3 &  3.00 &  0.00\\
instance n=20 445.alb & 1 & 0 & Optimal &  0.01 & 3 &  3.00 &  0.00\\
instance n=20 446.alb & 1 & 0 & Optimal &  0.02 & 3 &  3.00 &  0.00\\
instance n=20 447.alb & 1 & 0 & Optimal &  0.02 & 3 &  3.00 &  0.00\\
instance n=20 448.alb & 1 & 0 & Optimal &  0.02 & 3 &  3.00 &  0.00\\
instance n=20 449.alb & 1 & 0 & Optimal &  0.02 & 3 &  3.00 &  0.00\\
instance n=20 45.alb & 1 & 0 & Optimal &  0.03 & 6 &  6.00 &  0.00\\
instance n=20 450.alb & 1 & 0 & Optimal &  0.01 & 3 &  3.00 &  0.00\\
instance n=20 451.alb & 1 & 0 & Optimal &  0.02 & 3 &  3.00 &  0.00\\
instance n=20 452.alb & 1 & 0 & Optimal &  0.01 & 3 &  3.00 &  0.00\\
instance n=20 453.alb & 1 & 0 & Optimal &  0.01 & 3 &  3.00 &  0.00\\
instance n=20 454.alb & 1 & 0 & Optimal &  0.02 & 3 &  3.00 &  0.00\\
instance n=20 455.alb & 1 & 0 & Optimal &  0.01 & 3 &  3.00 &  0.00\\
instance n=20 456.alb & 1 & 0 & Optimal &  0.01 & 4 &  4.00 &  0.00\\
instance n=20 457.alb & 1 & 0 & Optimal &  0.01 & 3 &  3.00 &  0.00\\
instance n=20 458.alb & 1 & 0 & Optimal &  0.02 & 3 &  3.00 &  0.00\\
instance n=20 459.alb & 1 & 0 & Optimal &  0.01 & 3 &  3.00 &  0.00\\
instance n=20 46.alb & 1 & 0 & Optimal &  0.01 & 4 &  4.00 &  0.00\\
instance n=20 460.alb & 1 & 0 & Optimal &  0.03 & 3 &  3.00 &  0.00\\
instance n=20 461.alb & 1 & 0 & Optimal &  0.02 & 3 &  3.00 &  0.00\\
instance n=20 462.alb & 1 & 0 & Optimal &  0.01 & 3 &  3.00 &  0.00\\
instance n=20 463.alb & 1 & 0 & Optimal &  0.01 & 3 &  3.00 &  0.00\\
instance n=20 464.alb & 1 & 0 & Optimal &  0.02 & 3 &  3.00 &  0.00\\
instance n=20 465.alb & 1 & 0 & Optimal &  0.01 & 3 &  3.00 &  0.00\\
instance n=20 466.alb & 1 & 0 & Optimal &  0.01 & 13 & 13.00 &  0.00\\
instance n=20 467.alb & 1 & 0 & Optimal &  0.01 & 14 & 14.00 &  0.00\\
instance n=20 468.alb & 1 & 0 & Optimal &  0.02 & 13 & 13.00 &  0.00\\
instance n=20 469.alb & 1 & 0 & Optimal &  0.02 & 14 & 14.00 &  0.00\\
instance n=20 47.alb & 1 & 0 & Optimal &  0.02 & 4 &  4.00 &  0.00\\
instance n=20 470.alb & 1 & 0 & Optimal &  0.02 & 12 & 12.00 &  0.00\\
instance n=20 471.alb & 1 & 0 & Optimal &  0.02 & 12 & 12.00 &  0.00\\
instance n=20 472.alb & 1 & 0 & Optimal &  0.01 & 13 & 13.00 &  0.00\\
instance n=20 473.alb & 1 & 0 & Optimal &  0.01 & 10 & 10.00 &  0.00\\
instance n=20 474.alb & 1 & 0 & Optimal &  0.02 & 14 & 14.00 &  0.00\\
instance n=20 475.alb & 1 & 0 & Optimal &  0.02 & 11 & 11.00 &  0.00\\
instance n=20 476.alb & 1 & 0 & Optimal &  0.02 & 11 & 11.00 &  0.00\\
instance n=20 477.alb & 1 & 0 & Optimal &  0.02 & 11 & 11.00 &  0.00\\
instance n=20 478.alb & 1 & 0 & Optimal &  0.01 & 12 & 12.00 &  0.00\\
instance n=20 479.alb & 1 & 0 & Optimal &  0.02 & 13 & 13.00 &  0.00\\
instance n=20 48.alb & 1 & 0 & Optimal &  0.02 & 5 &  5.00 &  0.00\\
instance n=20 480.alb & 1 & 0 & Optimal &  0.02 & 13 & 13.00 &  0.00\\
instance n=20 481.alb & 1 & 0 & Optimal &  0.02 & 13 & 13.00 &  0.00\\
instance n=20 482.alb & 1 & 0 & Optimal &  0.01 & 13 & 13.00 &  0.00\\
instance n=20 483.alb & 1 & 0 & Optimal &  0.02 & 12 & 12.00 &  0.00\\
instance n=20 484.alb & 1 & 0 & Optimal &  0.01 & 13 & 13.00 &  0.00\\
instance n=20 485.alb & 1 & 0 & Optimal &  0.02 & 15 & 15.00 &  0.00\\
instance n=20 486.alb & 1 & 0 & Optimal &  0.02 & 11 & 11.00 &  0.00\\
instance n=20 487.alb & 1 & 0 & Optimal &  0.02 & 12 & 12.00 &  0.00\\
instance n=20 488.alb & 1 & 0 & Optimal &  0.01 & 15 & 15.00 &  0.00\\
instance n=20 489.alb & 1 & 0 & Optimal &  0.02 & 12 & 12.00 &  0.00\\
instance n=20 49.alb & 1 & 0 & Optimal &  0.02 & 4 &  4.00 &  0.00\\
instance n=20 490.alb & 1 & 0 & Optimal &  0.02 & 12 & 12.00 &  0.00\\
instance n=20 491.alb & 1 & 0 & Optimal &  0.02 & 6 &  6.00 &  0.00\\
instance n=20 492.alb & 1 & 0 & Optimal &  0.02 & 5 &  5.00 &  0.00\\
instance n=20 493.alb & 1 & 0 & Optimal &  0.01 & 5 &  5.00 &  0.00\\
instance n=20 494.alb & 1 & 0 & Optimal &  0.01 & 6 &  6.00 &  0.00\\
instance n=20 495.alb & 1 & 0 & Optimal &  0.01 & 6 &  6.00 &  0.00\\
instance n=20 496.alb & 1 & 0 & Optimal &  0.02 & 5 &  5.00 &  0.00\\
instance n=20 497.alb & 1 & 0 & Optimal &  0.02 & 6 &  6.00 &  0.00\\
instance n=20 498.alb & 1 & 0 & Optimal &  0.01 & 6 &  6.00 &  0.00\\
instance n=20 499.alb & 1 & 0 & Optimal &  0.02 & 5 &  5.00 &  0.00\\
instance n=20 5.alb & 1 & 0 & Optimal &  0.01 & 3 &  3.00 &  0.00\\
instance n=20 50.alb & 1 & 0 & Optimal &  0.01 & 4 &  4.00 &  0.00\\
instance n=20 500.alb & 1 & 0 & Optimal &  0.02 & 8 &  8.00 &  0.00\\
instance n=20 501.alb & 1 & 0 & Optimal &  0.03 & 5 &  5.00 &  0.00\\
instance n=20 502.alb & 1 & 0 & Optimal &  0.02 & 4 &  4.00 &  0.00\\
instance n=20 503.alb & 1 & 0 & Optimal &  0.02 & 6 &  6.00 &  0.00\\
instance n=20 504.alb & 1 & 0 & Optimal &  0.02 & 6 &  6.00 &  0.00\\
instance n=20 505.alb & 1 & 0 & Optimal &  0.02 & 6 &  6.00 &  0.00\\
instance n=20 506.alb & 1 & 0 & Optimal &  0.01 & 5 &  5.00 &  0.00\\
instance n=20 507.alb & 1 & 0 & Optimal &  0.02 & 5 &  5.00 &  0.00\\
instance n=20 508.alb & 1 & 0 & Optimal &  0.02 & 5 &  5.00 &  0.00\\
instance n=20 509.alb & 1 & 0 & Optimal &  0.01 & 4 &  4.00 &  0.00\\
instance n=20 51.alb & 1 & 0 & Optimal &  0.02 & 4 &  4.00 &  0.00\\
instance n=20 510.alb & 1 & 0 & Optimal &  0.02 & 5 &  5.00 &  0.00\\
instance n=20 511.alb & 1 & 0 & Optimal &  0.01 & 5 &  5.00 &  0.00\\
instance n=20 512.alb & 1 & 0 & Optimal &  0.02 & 5 &  5.00 &  0.00\\
instance n=20 513.alb & 1 & 0 & Optimal &  0.02 & 5 &  5.00 &  0.00\\
instance n=20 514.alb & 1 & 0 & Optimal &  0.02 & 5 &  5.00 &  0.00\\
instance n=20 515.alb & 1 & 0 & Optimal &  0.02 & 6 &  6.00 &  0.00\\
instance n=20 516.alb & 1 & 0 & Optimal &  0.02 & 3 &  3.00 &  0.00\\
instance n=20 517.alb & 1 & 0 & Optimal &  0.01 & 3 &  3.00 &  0.00\\
instance n=20 518.alb & 1 & 0 & Optimal &  0.01 & 3 &  3.00 &  0.00\\
instance n=20 519.alb & 1 & 0 & Optimal &  0.01 & 3 &  3.00 &  0.00\\
instance n=20 52.alb & 1 & 0 & Optimal &  0.02 & 4 &  4.00 &  0.00\\
instance n=20 520.alb & 1 & 0 & Optimal &  0.01 & 3 &  3.00 &  0.00\\
instance n=20 521.alb & 1 & 0 & Optimal &  0.01 & 3 &  3.00 &  0.00\\
instance n=20 522.alb & 1 & 0 & Optimal &  0.02 & 3 &  3.00 &  0.00\\
instance n=20 523.alb & 1 & 0 & Optimal &  0.02 & 3 &  3.00 &  0.00\\
instance n=20 524.alb & 1 & 0 & Optimal &  0.02 & 3 &  3.00 &  0.00\\
instance n=20 525.alb & 1 & 0 & Optimal &  0.01 & 3 &  3.00 &  0.00\\
instance n=20 53.alb & 1 & 0 & Optimal &  0.01 & 5 &  5.00 &  0.00\\
instance n=20 54.alb & 1 & 0 & Optimal &  0.02 & 5 &  5.00 &  0.00\\
instance n=20 55.alb & 1 & 0 & Optimal &  0.02 & 5 &  5.00 &  0.00\\
instance n=20 56.alb & 1 & 0 & Optimal &  0.03 & 4 &  4.00 &  0.00\\
instance n=20 57.alb & 1 & 0 & Optimal &  0.01 & 4 &  4.00 &  0.00\\
instance n=20 58.alb & 1 & 0 & Optimal &  0.10 & 5 &  5.00 &  0.00\\
instance n=20 59.alb & 1 & 0 & Optimal &  0.11 & 4 &  4.00 &  0.00\\
instance n=20 6.alb & 1 & 0 & Optimal &  0.01 & 3 &  3.00 &  0.00\\
instance n=20 60.alb & 1 & 0 & Optimal &  0.11 & 6 &  6.00 &  0.00\\
instance n=20 61.alb & 1 & 0 & Optimal &  0.03 & 7 &  7.00 &  0.00\\
instance n=20 62.alb & 1 & 0 & Optimal &  0.02 & 5 &  5.00 &  0.00\\
instance n=20 63.alb & 1 & 0 & Optimal &  0.03 & 5 &  5.00 &  0.00\\
instance n=20 64.alb & 1 & 0 & Optimal &  0.02 & 5 &  5.00 &  0.00\\
instance n=20 65.alb & 1 & 0 & Optimal &  0.02 & 5 &  5.00 &  0.00\\
instance n=20 66.alb & 1 & 0 & Optimal &  0.01 & 3 &  3.00 &  0.00\\
instance n=20 67.alb & 1 & 0 & Optimal &  0.01 & 3 &  3.00 &  0.00\\
instance n=20 68.alb & 1 & 0 & Optimal &  0.01 & 3 &  3.00 &  0.00\\
instance n=20 69.alb & 1 & 0 & Optimal &  0.01 & 2 &  2.00 &  0.00\\
instance n=20 7.alb & 1 & 0 & Optimal &  0.02 & 3 &  3.00 &  0.00\\
instance n=20 70.alb & 1 & 0 & Optimal &  0.10 & 3 &  3.00 &  0.00\\
instance n=20 71.alb & 1 & 0 & Optimal &  0.02 & 3 &  3.00 &  0.00\\
instance n=20 72.alb & 1 & 0 & Optimal &  0.02 & 3 &  3.00 &  0.00\\
instance n=20 73.alb & 1 & 0 & Optimal &  0.01 & 2 &  2.00 &  0.00\\
instance n=20 74.alb & 1 & 0 & Optimal &  0.01 & 3 &  3.00 &  0.00\\
instance n=20 75.alb & 1 & 0 & Optimal &  0.01 & 3 &  3.00 &  0.00\\
instance n=20 76.alb & 1 & 0 & Optimal &  0.01 & 3 &  3.00 &  0.00\\
instance n=20 77.alb & 1 & 0 & Optimal &  0.01 & 3 &  3.00 &  0.00\\
instance n=20 78.alb & 1 & 0 & Optimal &  0.02 & 3 &  3.00 &  0.00\\
instance n=20 79.alb & 1 & 0 & Optimal &  0.01 & 3 &  3.00 &  0.00\\
instance n=20 8.alb & 1 & 0 & Optimal &  0.01 & 3 &  3.00 &  0.00\\
instance n=20 80.alb & 1 & 0 & Optimal &  0.02 & 3 &  3.00 &  0.00\\
instance n=20 81.alb & 1 & 0 & Optimal &  0.01 & 3 &  3.00 &  0.00\\
instance n=20 82.alb & 1 & 0 & Optimal &  0.03 & 4 &  4.00 &  0.00\\
instance n=20 83.alb & 1 & 0 & Optimal &  0.02 & 3 &  3.00 &  0.00\\
instance n=20 84.alb & 1 & 0 & Optimal &  0.02 & 3 &  3.00 &  0.00\\
instance n=20 85.alb & 1 & 0 & Optimal &  0.01 & 3 &  3.00 &  0.00\\
instance n=20 86.alb & 1 & 0 & Optimal &  0.02 & 3 &  3.00 &  0.00\\
instance n=20 87.alb & 1 & 0 & Optimal &  0.01 & 3 &  3.00 &  0.00\\
instance n=20 88.alb & 1 & 0 & Optimal &  0.02 & 3 &  3.00 &  0.00\\
instance n=20 89.alb & 1 & 0 & Optimal &  0.02 & 3 &  3.00 &  0.00\\
instance n=20 9.alb & 1 & 0 & Optimal &  0.01 & 3 &  3.00 &  0.00\\
instance n=20 90.alb & 1 & 0 & Optimal &  0.01 & 3 &  3.00 &  0.00\\
instance n=20 91.alb & 1 & 0 & Optimal &  0.03 & 11 & 11.00 &  0.00\\
instance n=20 92.alb & 1 & 0 & Optimal &  0.01 & 11 & 11.00 &  0.00\\
instance n=20 93.alb & 1 & 0 & Optimal &  0.04 & 13 & 13.00 &  0.00\\
instance n=20 94.alb & 1 & 0 & Optimal &  0.03 & 10 & 10.00 &  0.00\\
instance n=20 95.alb & 1 & 0 & Optimal &  0.10 & 12 & 12.00 &  0.00\\
instance n=20 96.alb & 1 & 0 & Optimal &  0.02 & 10 & 10.00 &  0.00\\
instance n=20 97.alb & 1 & 0 & Optimal &  0.10 & 15 & 15.00 &  0.00\\
instance n=20 98.alb & 1 & 0 & Optimal &  0.02 & 13 & 13.00 &  0.00\\
instance n=20 99.alb & 1 & 0 & Optimal &  0.12 & 12 & 12.00 &  0.00\\
instance n=50 1.alb & 1 & 0 & Optimal &  0.15 & 8 &  8.00 &  0.00\\
instance n=50 10.alb & 1 & 0 & Optimal & 120.02 & 7 &  7.00 &  0.00\\
instance n=50 100.alb & 1 & 0 & Optimal &  0.06 & 7 &  7.00 &  0.00\\
instance n=50 101.alb & 1 & 0 & Optimal & 17.99 & 30 & 30.00 &  0.00\\
instance n=50 102.alb & 1 & 0 & Optimal & 78.08 & 32 & 32.00 &  0.00\\
instance n=50 103.alb & 1 & 0 & Optimal &  0.17 & 29 & 29.00 &  0.00\\
instance n=50 104.alb & 1 & 0 & Optimal &  1.29 & 27 & 27.00 &  0.00\\
instance n=50 105.alb & 1 & 0 & Optimal & 23.50 & 24 & 24.00 &  0.00\\
instance n=50 106.alb & 1 & 0 & Optimal & 18.60 & 28 & 28.00 &  0.00\\
instance n=50 107.alb & 1 & 0 & Optimal &  4.01 & 28 & 28.00 &  0.00\\
instance n=50 108.alb & 1 & 0 & Optimal &  0.72 & 30 & 30.00 &  0.00\\
instance n=50 109.alb & 1 & 0 & Optimal &  0.12 & 30 & 30.00 &  0.00\\
instance n=50 11.alb & 1 & 0 & Optimal &  0.06 & 7 &  7.00 &  0.00\\
instance n=50 110.alb & 1 & 0 & Optimal &  0.44 & 26 & 26.00 &  0.00\\
instance n=50 111.alb & 1 & 0 & Optimal &  0.31 & 28 & 28.00 &  0.00\\
instance n=50 112.alb & 1 & 0 & Optimal &  1.07 & 27 & 27.00 &  0.00\\
instance n=50 113.alb & 1 & 0 & Optimal & 11.14 & 28 & 28.00 &  0.00\\
instance n=50 114.alb & 1 & 0 & Optimal &  0.68 & 27 & 27.00 &  0.00\\
instance n=50 115.alb & 1 & 0 & Optimal & 109.51 & 28 & 28.00 &  0.00\\
instance n=50 116.alb & 1 & 0 & Optimal &  0.31 & 32 & 32.00 &  0.00\\
instance n=50 117.alb & 1 & 0 & Optimal & 19.30 & 27 & 27.00 &  0.00\\
instance n=50 118.alb & 1 & 0 & Optimal &  0.74 & 29 & 29.00 &  0.00\\
instance n=50 119.alb & 1 & 0 & Optimal &  0.33 & 25 & 25.00 &  0.00\\
instance n=50 12.alb & 1 & 0 & Optimal & 120.02 & 6 &  6.00 &  0.00\\
instance n=50 120.alb & 1 & 0 & Optimal &  1.71 & 27 & 27.00 &  0.00\\
instance n=50 121.alb & 1 & 0 & Optimal &  8.70 & 32 & 32.00 &  0.00\\
instance n=50 122.alb & 1 & 0 & Optimal & 20.50 & 29 & 29.00 &  0.00\\
instance n=50 123.alb & 1 & 0 & Optimal &  2.65 & 32 & 32.00 &  0.00\\
instance n=50 124.alb & 1 & 0 & Optimal &  1.25 & 29 & 29.00 &  0.00\\
instance n=50 125.alb & 1 & 0 & Optimal &  0.08 & 33 & 33.00 &  0.00\\
instance n=50 126.alb & 1 & 0 & Optimal &  0.07 & 12 & 12.00 &  0.00\\
instance n=50 127.alb & 1 & 0 & Optimal &  0.27 & 14 & 14.00 &  0.00\\
instance n=50 128.alb & 1 & 0 & Optimal &  0.37 & 12 & 12.00 &  0.00\\
instance n=50 129.alb & 1 & 0 & Optimal &  0.09 & 13 & 13.00 &  0.00\\
instance n=50 13.alb & 1 & 0 & Optimal &  0.63 & 6 &  6.00 &  0.00\\
instance n=50 130.alb & 1 & 0 & Optimal &  0.17 & 13 & 13.00 &  0.00\\
instance n=50 131.alb & 1 & 0 & Optimal &  0.07 & 12 & 12.00 &  0.00\\
instance n=50 132.alb & 1 & 0 & Optimal &  0.80 & 12 & 12.00 &  0.00\\
instance n=50 133.alb & 1 & 0 & Optimal &  0.04 & 12 & 12.00 &  0.00\\
instance n=50 134.alb & 1 & 0 & Optimal &  0.13 & 14 & 14.00 &  0.00\\
instance n=50 135.alb & 1 & 0 & Optimal &  0.22 & 13 & 13.00 &  0.00\\
instance n=50 136.alb & 1 & 0 & Optimal &  0.05 & 11 & 11.00 &  0.00\\
instance n=50 137.alb & 1 & 0 & Optimal &  0.05 & 11 & 11.00 &  0.00\\
instance n=50 138.alb & 1 & 0 & Optimal &  0.05 & 12 & 12.00 &  0.00\\
instance n=50 139.alb & 1 & 0 & Optimal &  3.70 & 11 & 11.00 &  0.00\\
instance n=50 14.alb & 1 & 0 & Optimal &  0.04 & 7 &  7.00 &  0.00\\
instance n=50 140.alb & 1 & 0 & Optimal &  0.03 & 12 & 12.00 &  0.00\\
instance n=50 141.alb & 1 & 0 & Optimal &  0.61 & 13 & 13.00 &  0.00\\
instance n=50 142.alb & 1 & 0 & Optimal &  0.12 & 11 & 11.00 &  0.00\\
instance n=50 143.alb & 1 & 0 & Optimal &  0.08 & 12 & 12.00 &  0.00\\
instance n=50 144.alb & 1 & 0 & Optimal &  0.07 & 13 & 13.00 &  0.00\\
instance n=50 145.alb & 1 & 0 & Optimal &  0.24 & 10 & 10.00 &  0.00\\
instance n=50 146.alb & 1 & 0 & Optimal &  0.12 & 13 & 13.00 &  0.00\\
instance n=50 147.alb & 1 & 0 & Optimal &  0.26 & 13 & 13.00 &  0.00\\
instance n=50 148.alb & 1 & 0 & Optimal &  0.04 & 10 & 10.00 &  0.00\\
instance n=50 149.alb & 1 & 0 & Optimal &  0.08 & 12 & 12.00 &  0.00\\
instance n=50 15.alb & 1 & 0 & Optimal &  0.04 & 8 &  8.00 &  0.00\\
instance n=50 150.alb & 1 & 0 & Optimal &  0.07 & 11 & 11.00 &  0.00\\
instance n=50 151.alb & 1 & 0 & Optimal &  0.12 & 7 &  7.00 &  0.00\\
instance n=50 152.alb & 1 & 0 & Optimal &  0.71 & 7 &  7.00 &  0.00\\
instance n=50 153.alb & 1 & 0 & Optimal &  1.28 & 7 &  7.00 &  0.00\\
instance n=50 154.alb & 1 & 0 & Optimal &  0.06 & 8 &  8.00 &  0.00\\
instance n=50 155.alb & 1 & 0 & Optimal &  0.02 & 7 &  7.00 &  0.00\\
instance n=50 156.alb & 1 & 0 & Optimal &  0.04 & 7 &  7.00 &  0.00\\
instance n=50 157.alb & 1 & 0 & Optimal &  0.72 & 8 &  8.00 &  0.00\\
instance n=50 158.alb & 1 & 0 & Optimal &  5.39 & 7 &  7.00 &  0.00\\
instance n=50 159.alb & 1 & 0 & Optimal &  0.04 & 7 &  7.00 &  0.00\\
instance n=50 16.alb & 1 & 0 & Optimal &  0.03 & 8 &  8.00 &  0.00\\
instance n=50 160.alb & 1 & 0 & Optimal &  0.11 & 8 &  8.00 &  0.00\\
instance n=50 161.alb & 1 & 0 & Optimal & 120.01 & 7 &  7.00 &  0.00\\
instance n=50 162.alb & 1 & 0 & Optimal & 57.69 & 8 &  8.00 &  0.00\\
instance n=50 163.alb & 1 & 0 & Optimal &  5.10 & 7 &  7.00 &  0.00\\
instance n=50 164.alb & 1 & 0 & Optimal &  0.28 & 7 &  7.00 &  0.00\\
instance n=50 165.alb & 1 & 0 & Optimal &  0.22 & 8 &  8.00 &  0.00\\
instance n=50 166.alb & 1 & 0 & Optimal &  0.03 & 8 &  8.00 &  0.00\\
instance n=50 167.alb & 1 & 0 & Optimal &  2.69 & 7 &  7.00 &  0.00\\
instance n=50 168.alb & 1 & 0 & Optimal &  0.59 & 8 &  8.00 &  0.00\\
instance n=50 169.alb & 1 & 0 & Optimal &  6.11 & 8 &  8.00 &  0.00\\
instance n=50 17.alb & 1 & 0 & Optimal &  0.03 & 7 &  7.00 &  0.00\\
instance n=50 170.alb & 1 & 0 & Optimal &  2.96 & 7 &  7.00 &  0.00\\
instance n=50 171.alb & 1 & 0 & Optimal &  1.73 & 8 &  8.00 &  0.00\\
instance n=50 172.alb & 1 & 0 & Optimal &  0.23 & 7 &  7.00 &  0.00\\
instance n=50 173.alb & 1 & 0 & Optimal &  0.59 & 7 &  7.00 &  0.00\\
instance n=50 174.alb & 1 & 0 & Optimal &  4.45 & 7 &  7.00 &  0.00\\
instance n=50 175.alb & 1 & 0 & Optimal &  0.93 & 7 &  7.00 &  0.00\\
instance n=50 176.alb & 1 & 0 & Optimal & 21.25 & 27 & 27.00 &  0.00\\
instance n=50 177.alb & 1 & 0 & Solution & 120.13 & 28 & 27.00 &  3.57\\
instance n=50 178.alb & 1 & 0 & Solution & 120.12 & 28 & 27.00 &  3.57\\
instance n=50 179.alb & 1 & 0 & Optimal &  9.31 & 26 & 26.00 &  0.00\\
instance n=50 18.alb & 1 & 0 & Optimal &  0.04 & 7 &  7.00 &  0.00\\
instance n=50 180.alb & 1 & 0 & Optimal &  0.44 & 26 & 26.00 &  0.00\\
instance n=50 181.alb & 1 & 0 & Optimal &  3.32 & 29 & 29.00 &  0.00\\
instance n=50 182.alb & 1 & 0 & Optimal & 120.05 & 26 & 26.00 &  0.00\\
instance n=50 183.alb & 1 & 0 & Optimal & 29.62 & 28 & 28.00 &  0.00\\
instance n=50 184.alb & 1 & 0 & Optimal &  0.06 & 38 & 38.00 &  0.00\\
instance n=50 185.alb & 1 & 0 & Optimal & 41.90 & 26 & 26.00 &  0.00\\
instance n=50 186.alb & 1 & 0 & Optimal &  0.94 & 26 & 26.00 &  0.00\\
instance n=50 187.alb & 1 & 0 & Solution & 120.79 & 26 & 25.00 &  3.85\\
instance n=50 188.alb & 1 & 0 & Solution & 121.18 & 25 & 24.00 &  4.00\\
instance n=50 189.alb & 1 & 0 & Solution & 120.15 & 26 & 25.00 &  3.85\\
instance n=50 19.alb & 1 & 0 & Optimal &  0.19 & 8 &  8.00 &  0.00\\
instance n=50 190.alb & 1 & 0 & Optimal &  1.72 & 30 & 30.00 &  0.00\\
instance n=50 191.alb & 1 & 0 & Solution & 121.36 & 28 & 27.00 &  3.57\\
instance n=50 192.alb & 1 & 0 & Optimal &  2.66 & 27 & 27.00 &  0.00\\
instance n=50 193.alb & 1 & 0 & Optimal & 38.33 & 28 & 28.00 &  0.00\\
instance n=50 194.alb & 1 & 0 & Optimal & 23.55 & 28 & 28.00 &  0.00\\
instance n=50 195.alb & 1 & 0 & Optimal &  2.55 & 28 & 28.00 &  0.00\\
instance n=50 196.alb & 1 & 0 & Optimal & 31.41 & 27 & 27.00 &  0.00\\
instance n=50 197.alb & 1 & 0 & Optimal & 120.03 & 28 & 28.00 &  0.00\\
instance n=50 198.alb & 1 & 0 & Optimal &  0.08 & 28 & 28.00 &  0.00\\
instance n=50 199.alb & 1 & 0 & Optimal &  0.09 & 29 & 29.00 &  0.00\\
instance n=50 2.alb & 1 & 0 & Optimal & 67.63 & 6 &  6.00 &  0.00\\
instance n=50 20.alb & 1 & 0 & Optimal &  0.04 & 8 &  8.00 &  0.00\\
instance n=50 200.alb & 1 & 0 & Solution & 121.04 & 25 & 24.00 &  4.00\\
instance n=50 201.alb & 1 & 0 & Optimal &  0.04 & 13 & 13.00 &  0.00\\
instance n=50 202.alb & 1 & 0 & Optimal &  1.56 & 9 &  9.00 &  0.00\\
instance n=50 203.alb & 1 & 0 & Optimal &  0.04 & 11 & 11.00 &  0.00\\
instance n=50 204.alb & 1 & 0 & Optimal &  0.99 & 10 & 10.00 &  0.00\\
instance n=50 205.alb & 1 & 0 & Optimal &  0.04 & 13 & 13.00 &  0.00\\
instance n=50 206.alb & 1 & 0 & Optimal & 120.06 & 11 & 11.00 &  0.00\\
instance n=50 207.alb & 1 & 0 & Optimal &  0.65 & 10 & 10.00 &  0.00\\
instance n=50 208.alb & 1 & 0 & Optimal &  0.23 & 13 & 13.00 &  0.00\\
instance n=50 209.alb & 1 & 0 & Optimal &  1.16 & 11 & 11.00 &  0.00\\
instance n=50 21.alb & 1 & 0 & Optimal & 120.03 & 6 &  6.00 &  0.00\\
instance n=50 210.alb & 1 & 0 & Optimal &  0.04 & 13 & 13.00 &  0.00\\
instance n=50 211.alb & 1 & 0 & Optimal &  0.03 & 12 & 12.00 &  0.00\\
instance n=50 212.alb & 1 & 0 & Optimal &  0.08 & 10 & 10.00 &  0.00\\
instance n=50 213.alb & 1 & 0 & Optimal &  0.04 & 13 & 13.00 &  0.00\\
instance n=50 214.alb & 1 & 0 & Optimal &  4.68 & 11 & 11.00 &  0.00\\
instance n=50 215.alb & 1 & 0 & Optimal &  0.06 & 11 & 11.00 &  0.00\\
instance n=50 216.alb & 1 & 0 & Optimal &  0.30 & 12 & 12.00 &  0.00\\
instance n=50 217.alb & 1 & 0 & Optimal &  0.84 & 13 & 13.00 &  0.00\\
instance n=50 218.alb & 1 & 0 & Optimal &  0.04 & 12 & 12.00 &  0.00\\
instance n=50 219.alb & 1 & 0 & Optimal &  0.28 & 11 & 11.00 &  0.00\\
instance n=50 22.alb & 1 & 0 & Optimal & 120.02 & 7 &  7.00 &  0.00\\
instance n=50 220.alb & 1 & 0 & Optimal &  0.04 & 11 & 11.00 &  0.00\\
instance n=50 221.alb & 1 & 0 & Optimal &  2.64 & 11 & 11.00 &  0.00\\
instance n=50 222.alb & 1 & 0 & Optimal &  0.29 & 14 & 14.00 &  0.00\\
instance n=50 223.alb & 1 & 0 & Optimal &  0.28 & 11 & 11.00 &  0.00\\
instance n=50 224.alb & 1 & 0 & Optimal &  0.09 & 11 & 11.00 &  0.00\\
instance n=50 225.alb & 1 & 0 & Optimal &  0.03 & 12 & 12.00 &  0.00\\
instance n=50 226.alb & 1 & 0 & Optimal &  0.05 & 7 &  7.00 &  0.00\\
instance n=50 227.alb & 1 & 0 & Optimal &  0.09 & 6 &  6.00 &  0.00\\
instance n=50 228.alb & 1 & 0 & Optimal &  0.04 & 6 &  6.00 &  0.00\\
instance n=50 229.alb & 1 & 0 & Optimal &  0.03 & 6 &  6.00 &  0.00\\
instance n=50 23.alb & 1 & 0 & Optimal &  0.04 & 7 &  7.00 &  0.00\\
instance n=50 230.alb & 1 & 0 & Optimal &  0.06 & 7 &  7.00 &  0.00\\
instance n=50 231.alb & 1 & 0 & Optimal &  0.03 & 7 &  7.00 &  0.00\\
instance n=50 232.alb & 1 & 0 & Optimal &  0.05 & 7 &  7.00 &  0.00\\
instance n=50 233.alb & 1 & 0 & Optimal &  0.03 & 6 &  6.00 &  0.00\\
instance n=50 234.alb & 1 & 0 & Optimal &  0.09 & 8 &  8.00 &  0.00\\
instance n=50 235.alb & 1 & 0 & Optimal &  0.05 & 7 &  7.00 &  0.00\\
instance n=50 236.alb & 1 & 0 & Optimal &  0.40 & 7 &  7.00 &  0.00\\
instance n=50 237.alb & 1 & 0 & Optimal &  0.03 & 8 &  8.00 &  0.00\\
instance n=50 238.alb & 1 & 0 & Optimal &  0.06 & 7 &  7.00 &  0.00\\
instance n=50 239.alb & 1 & 0 & Optimal &  0.05 & 7 &  7.00 &  0.00\\
instance n=50 24.alb & 1 & 0 & Optimal & 120.02 & 7 &  7.00 &  0.00\\
instance n=50 240.alb & 1 & 0 & Optimal &  0.04 & 7 &  7.00 &  0.00\\
instance n=50 241.alb & 1 & 0 & Optimal &  0.08 & 7 &  7.00 &  0.00\\
instance n=50 242.alb & 1 & 0 & Optimal &  0.07 & 8 &  8.00 &  0.00\\
instance n=50 243.alb & 1 & 0 & Optimal &  0.12 & 7 &  7.00 &  0.00\\
instance n=50 244.alb & 1 & 0 & Optimal &  0.05 & 7 &  7.00 &  0.00\\
instance n=50 245.alb & 1 & 0 & Optimal &  0.04 & 7 &  7.00 &  0.00\\
instance n=50 246.alb & 1 & 0 & Optimal &  0.22 & 8 &  8.00 &  0.00\\
instance n=50 247.alb & 1 & 0 & Optimal &  0.05 & 7 &  7.00 &  0.00\\
instance n=50 248.alb & 1 & 0 & Optimal &  0.06 & 7 &  7.00 &  0.00\\
instance n=50 249.alb & 1 & 0 & Optimal &  0.18 & 7 &  7.00 &  0.00\\
instance n=50 25.alb & 1 & 0 & Optimal &  0.06 & 6 &  6.00 &  0.00\\
instance n=50 250.alb & 1 & 0 & Optimal &  0.04 & 7 &  7.00 &  0.00\\
instance n=50 251.alb & 1 & 0 & Optimal &  1.08 & 27 & 27.00 &  0.00\\
instance n=50 252.alb & 1 & 0 & Optimal &  4.59 & 32 & 32.00 &  0.00\\
instance n=50 253.alb & 1 & 0 & Optimal &  4.81 & 28 & 28.00 &  0.00\\
instance n=50 254.alb & 1 & 0 & Optimal &  0.06 & 30 & 30.00 &  0.00\\
instance n=50 255.alb & 1 & 0 & Optimal &  0.59 & 29 & 29.00 &  0.00\\
instance n=50 256.alb & 1 & 0 & Optimal &  0.39 & 30 & 30.00 &  0.00\\
instance n=50 257.alb & 1 & 0 & Optimal &  3.78 & 33 & 33.00 &  0.00\\
instance n=50 258.alb & 1 & 0 & Optimal &  4.75 & 28 & 28.00 &  0.00\\
instance n=50 259.alb & 1 & 0 & Optimal &  3.69 & 31 & 31.00 &  0.00\\
instance n=50 26.alb & 1 & 0 & Optimal & 83.72 & 27 & 27.00 &  0.00\\
instance n=50 260.alb & 1 & 0 & Optimal &  0.73 & 29 & 29.00 &  0.00\\
instance n=50 261.alb & 1 & 0 & Optimal &  2.79 & 28 & 28.00 &  0.00\\
instance n=50 262.alb & 1 & 0 & Optimal &  0.92 & 31 & 31.00 &  0.00\\
instance n=50 263.alb & 1 & 0 & Optimal &  0.92 & 29 & 29.00 &  0.00\\
instance n=50 264.alb & 1 & 0 & Optimal &  2.51 & 27 & 27.00 &  0.00\\
instance n=50 265.alb & 1 & 0 & Optimal &  0.81 & 27 & 27.00 &  0.00\\
instance n=50 266.alb & 1 & 0 & Optimal &  4.79 & 29 & 29.00 &  0.00\\
instance n=50 267.alb & 1 & 0 & Optimal &  5.15 & 28 & 28.00 &  0.00\\
instance n=50 268.alb & 1 & 0 & Optimal &  6.11 & 29 & 29.00 &  0.00\\
instance n=50 269.alb & 1 & 0 & Optimal &  0.56 & 26 & 26.00 &  0.00\\
instance n=50 27.alb & 1 & 0 & Optimal & 13.18 & 30 & 30.00 &  0.00\\
instance n=50 270.alb & 1 & 0 & Optimal &  0.31 & 28 & 28.00 &  0.00\\
instance n=50 271.alb & 1 & 0 & Optimal &  2.56 & 31 & 31.00 &  0.00\\
instance n=50 272.alb & 1 & 0 & Optimal &  2.09 & 27 & 27.00 &  0.00\\
instance n=50 273.alb & 1 & 0 & Optimal &  5.52 & 27 & 27.00 &  0.00\\
instance n=50 274.alb & 1 & 0 & Optimal &  0.07 & 29 & 29.00 &  0.00\\
instance n=50 275.alb & 1 & 0 & Optimal &  0.87 & 27 & 27.00 &  0.00\\
instance n=50 276.alb & 1 & 0 & Optimal &  0.06 & 12 & 12.00 &  0.00\\
instance n=50 277.alb & 1 & 0 & Optimal &  0.08 & 13 & 13.00 &  0.00\\
instance n=50 278.alb & 1 & 0 & Optimal &  0.10 & 12 & 12.00 &  0.00\\
instance n=50 279.alb & 1 & 0 & Optimal &  0.05 & 11 & 11.00 &  0.00\\
instance n=50 28.alb & 1 & 0 & Optimal &  0.08 & 28 & 28.00 &  0.00\\
instance n=50 280.alb & 1 & 0 & Optimal &  0.09 & 13 & 13.00 &  0.00\\
instance n=50 281.alb & 1 & 0 & Optimal &  0.08 & 11 & 11.00 &  0.00\\
instance n=50 282.alb & 1 & 0 & Optimal &  3.49 & 12 & 12.00 &  0.00\\
instance n=50 283.alb & 1 & 0 & Optimal &  0.27 & 12 & 12.00 &  0.00\\
instance n=50 284.alb & 1 & 0 & Optimal &  0.05 & 11 & 11.00 &  0.00\\
instance n=50 285.alb & 1 & 0 & Optimal &  0.20 & 13 & 13.00 &  0.00\\
instance n=50 286.alb & 1 & 0 & Optimal &  0.32 & 11 & 11.00 &  0.00\\
instance n=50 287.alb & 1 & 0 & Optimal &  0.96 & 12 & 12.00 &  0.00\\
instance n=50 288.alb & 1 & 0 & Optimal &  0.06 & 10 & 10.00 &  0.00\\
instance n=50 289.alb & 1 & 0 & Optimal &  0.24 & 11 & 11.00 &  0.00\\
instance n=50 29.alb & 1 & 0 & Optimal &  0.04 & 29 & 29.00 &  0.00\\
instance n=50 290.alb & 1 & 0 & Optimal &  0.09 & 14 & 14.00 &  0.00\\
instance n=50 291.alb & 1 & 0 & Optimal &  0.09 & 12 & 12.00 &  0.00\\
instance n=50 292.alb & 1 & 0 & Optimal &  0.07 & 13 & 13.00 &  0.00\\
instance n=50 293.alb & 1 & 0 & Optimal &  0.04 & 12 & 12.00 &  0.00\\
instance n=50 294.alb & 1 & 0 & Optimal &  0.07 & 13 & 13.00 &  0.00\\
instance n=50 295.alb & 1 & 0 & Optimal &  0.09 & 16 & 16.00 &  0.00\\
instance n=50 296.alb & 1 & 0 & Optimal &  0.15 & 13 & 13.00 &  0.00\\
instance n=50 297.alb & 1 & 0 & Optimal &  0.07 & 13 & 13.00 &  0.00\\
instance n=50 298.alb & 1 & 0 & Optimal &  0.07 & 11 & 11.00 &  0.00\\
instance n=50 299.alb & 1 & 0 & Optimal &  2.00 & 12 & 12.00 &  0.00\\
instance n=50 3.alb & 1 & 0 & Optimal &  0.31 & 8 &  8.00 &  0.00\\
instance n=50 30.alb & 1 & 0 & Optimal & 120.06 & 26 & 26.00 &  0.00\\
instance n=50 300.alb & 1 & 0 & Optimal &  0.04 & 12 & 12.00 &  0.00\\
instance n=50 301.alb & 1 & 0 & Optimal & 120.01 & 6 &  6.00 &  0.00\\
instance n=50 302.alb & 1 & 0 & Optimal & 120.03 & 7 &  7.00 &  0.00\\
instance n=50 303.alb & 1 & 0 & Optimal & 120.02 & 8 &  8.00 &  0.00\\
instance n=50 304.alb & 1 & 0 & Optimal &  0.47 & 7 &  7.00 &  0.00\\
instance n=50 305.alb & 1 & 0 & Optimal & 120.03 & 8 &  8.00 &  0.00\\
instance n=50 306.alb & 1 & 0 & Optimal & 36.00 & 7 &  7.00 &  0.00\\
instance n=50 307.alb & 1 & 0 & Optimal & 120.02 & 7 &  7.00 &  0.00\\
instance n=50 308.alb & 1 & 0 & Optimal &  2.72 & 8 &  8.00 &  0.00\\
instance n=50 309.alb & 1 & 0 & Optimal &  1.61 & 7 &  7.00 &  0.00\\
instance n=50 31.alb & 1 & 0 & Solution & 120.15 & 28 & 27.00 &  3.57\\
instance n=50 310.alb & 1 & 0 & Optimal &  0.04 & 8 &  8.00 &  0.00\\
instance n=50 311.alb & 1 & 0 & Optimal &  9.54 & 8 &  8.00 &  0.00\\
instance n=50 312.alb & 1 & 0 & Optimal &  0.26 & 6 &  6.00 &  0.00\\
instance n=50 313.alb & 1 & 0 & Optimal & 120.03 & 8 &  8.00 &  0.00\\
instance n=50 314.alb & 1 & 0 & Optimal & 17.64 & 7 &  7.00 &  0.00\\
instance n=50 315.alb & 1 & 0 & Optimal & 120.02 & 8 &  8.00 &  0.00\\
instance n=50 316.alb & 1 & 0 & Optimal &  0.70 & 8 &  8.00 &  0.00\\
instance n=50 317.alb & 1 & 0 & Optimal &  0.03 & 6 &  6.00 &  0.00\\
instance n=50 318.alb & 1 & 0 & Optimal &  0.16 & 8 &  8.00 &  0.00\\
instance n=50 319.alb & 1 & 0 & Optimal &  0.22 & 7 &  7.00 &  0.00\\
instance n=50 32.alb & 1 & 0 & Optimal & 26.75 & 25 & 25.00 &  0.00\\
instance n=50 320.alb & 1 & 0 & Optimal & 120.02 & 8 &  8.00 &  0.00\\
instance n=50 321.alb & 1 & 0 & Optimal &  0.03 & 6 &  6.00 &  0.00\\
instance n=50 322.alb & 1 & 0 & Optimal & 120.02 & 7 &  7.00 &  0.00\\
instance n=50 323.alb & 1 & 0 & Optimal & 120.02 & 7 &  7.00 &  0.00\\
instance n=50 324.alb & 1 & 0 & Optimal & 120.02 & 7 &  7.00 &  0.00\\
instance n=50 325.alb & 1 & 0 & Optimal &  0.24 & 7 &  7.00 &  0.00\\
instance n=50 326.alb & 1 & 0 & Optimal &  0.65 & 33 & 33.00 &  0.00\\
instance n=50 327.alb & 1 & 0 & Optimal & 113.77 & 28 & 28.00 &  0.00\\
instance n=50 328.alb & 1 & 0 & Optimal &  0.47 & 32 & 32.00 &  0.00\\
instance n=50 329.alb & 1 & 0 & Solution & 120.13 & 25 & 24.00 &  4.00\\
instance n=50 33.alb & 1 & 0 & Solution & 120.15 & 25 & 24.00 &  4.00\\
instance n=50 330.alb & 1 & 0 & Optimal &  0.07 & 29 & 29.00 &  0.00\\
instance n=50 331.alb & 1 & 0 & Optimal & 120.05 & 29 & 29.00 &  0.00\\
instance n=50 332.alb & 1 & 0 & Solution & 120.91 & 25 & 24.00 &  4.00\\
instance n=50 333.alb & 1 & 0 & Optimal &  5.15 & 28 & 28.00 &  0.00\\
instance n=50 334.alb & 1 & 0 & Optimal &  0.03 & 29 & 29.00 &  0.00\\
instance n=50 335.alb & 1 & 0 & Optimal & 120.06 & 27 & 27.00 &  0.00\\
instance n=50 336.alb & 1 & 0 & Solution & 120.11 & 26 & 25.00 &  3.85\\
instance n=50 337.alb & 1 & 0 & Optimal &  0.42 & 26 & 26.00 &  0.00\\
instance n=50 338.alb & 1 & 0 & Optimal & 82.25 & 26 & 26.00 &  0.00\\
instance n=50 339.alb & 1 & 0 & Optimal &  0.08 & 27 & 27.00 &  0.00\\
instance n=50 34.alb & 1 & 0 & Optimal &  0.11 & 30 & 30.00 &  0.00\\
instance n=50 340.alb & 1 & 0 & Solution & 120.13 & 28 & 27.00 &  3.57\\
instance n=50 341.alb & 1 & 0 & Optimal & 120.04 & 27 & 27.00 &  0.00\\
instance n=50 342.alb & 1 & 0 & Solution & 121.05 & 28 & 27.00 &  3.57\\
instance n=50 343.alb & 1 & 0 & Optimal & 120.05 & 27 & 27.00 &  0.00\\
instance n=50 344.alb & 1 & 0 & Optimal &  2.17 & 30 & 30.00 &  0.00\\
instance n=50 345.alb & 1 & 0 & Optimal & 120.04 & 29 & 29.00 &  0.00\\
instance n=50 346.alb & 1 & 0 & Optimal &  5.19 & 27 & 27.00 &  0.00\\
instance n=50 347.alb & 1 & 0 & Optimal & 110.19 & 25 & 25.00 &  0.00\\
instance n=50 348.alb & 1 & 0 & Optimal &  0.03 & 30 & 30.00 &  0.00\\
instance n=50 349.alb & 1 & 0 & Optimal &  0.95 & 28 & 28.00 &  0.00\\
instance n=50 35.alb & 1 & 0 & Optimal &  9.75 & 31 & 31.00 &  0.00\\
instance n=50 350.alb & 1 & 0 & Solution & 120.14 & 24 & 23.00 &  4.17\\
instance n=50 351.alb & 1 & 0 & Optimal &  0.03 & 12 & 12.00 &  0.00\\
instance n=50 352.alb & 1 & 0 & Optimal & 120.04 & 10 & 10.00 &  0.00\\
instance n=50 353.alb & 1 & 0 & Optimal &  0.06 & 13 & 13.00 &  0.00\\
instance n=50 354.alb & 1 & 0 & Solution & 120.11 & 14 & 13.00 &  7.14\\
instance n=50 355.alb & 1 & 0 & Optimal &  0.03 & 11 & 11.00 &  0.00\\
instance n=50 356.alb & 1 & 0 & Optimal &  0.05 & 15 & 15.00 &  0.00\\
instance n=50 357.alb & 1 & 0 & Optimal &  0.04 & 12 & 12.00 &  0.00\\
instance n=50 358.alb & 1 & 0 & Optimal &  0.17 & 11 & 11.00 &  0.00\\
instance n=50 359.alb & 1 & 0 & Optimal & 120.04 & 10 & 10.00 &  0.00\\
instance n=50 36.alb & 1 & 0 & Optimal &  0.26 & 31 & 31.00 &  0.00\\
instance n=50 360.alb & 1 & 0 & Optimal &  0.05 & 12 & 12.00 &  0.00\\
instance n=50 361.alb & 1 & 0 & Optimal &  0.27 & 11 & 11.00 &  0.00\\
instance n=50 362.alb & 1 & 0 & Optimal &  0.04 & 10 & 10.00 &  0.00\\
instance n=50 363.alb & 1 & 0 & Solution & 120.11 & 12 & 11.00 &  8.33\\
instance n=50 364.alb & 1 & 0 & Optimal &  0.55 & 13 & 13.00 &  0.00\\
instance n=50 365.alb & 1 & 0 & Optimal &  0.38 & 11 & 11.00 &  0.00\\
instance n=50 366.alb & 1 & 0 & Optimal &  0.03 & 13 & 13.00 &  0.00\\
instance n=50 367.alb & 1 & 0 & Optimal &  0.17 & 12 & 12.00 &  0.00\\
instance n=50 368.alb & 1 & 0 & Optimal &  0.05 & 12 & 12.00 &  0.00\\
instance n=50 369.alb & 1 & 0 & Optimal &  0.41 & 12 & 12.00 &  0.00\\
instance n=50 37.alb & 1 & 0 & Solution & 120.73 & 32 & 31.00 &  3.13\\
instance n=50 370.alb & 1 & 0 & Optimal &  0.05 & 12 & 12.00 &  0.00\\
instance n=50 371.alb & 1 & 0 & Optimal & 82.83 & 11 & 11.00 &  0.00\\
instance n=50 372.alb & 1 & 0 & Optimal & 120.04 & 10 & 10.00 &  0.00\\
instance n=50 373.alb & 1 & 0 & Optimal &  1.20 & 12 & 12.00 &  0.00\\
instance n=50 374.alb & 1 & 0 & Optimal &  0.04 & 11 & 11.00 &  0.00\\
instance n=50 375.alb & 1 & 0 & Optimal & 120.04 & 13 & 13.00 &  0.00\\
instance n=50 376.alb & 1 & 0 & Optimal &  0.05 & 7 &  7.00 &  0.00\\
instance n=50 377.alb & 1 & 0 & Optimal &  0.04 & 7 &  7.00 &  0.00\\
instance n=50 378.alb & 1 & 0 & Optimal &  0.06 & 8 &  8.00 &  0.00\\
instance n=50 379.alb & 1 & 0 & Optimal &  0.05 & 7 &  7.00 &  0.00\\
instance n=50 38.alb & 1 & 0 & Optimal &  0.36 & 31 & 31.00 &  0.00\\
instance n=50 380.alb & 1 & 0 & Optimal &  0.03 & 7 &  7.00 &  0.00\\
instance n=50 381.alb & 1 & 0 & Optimal &  0.33 & 8 &  8.00 &  0.00\\
instance n=50 382.alb & 1 & 0 & Optimal &  0.04 & 6 &  6.00 &  0.00\\
instance n=50 383.alb & 1 & 0 & Optimal &  0.08 & 7 &  7.00 &  0.00\\
instance n=50 384.alb & 1 & 0 & Optimal &  0.15 & 8 &  8.00 &  0.00\\
instance n=50 385.alb & 1 & 0 & Optimal &  0.04 & 7 &  7.00 &  0.00\\
instance n=50 386.alb & 1 & 0 & Optimal &  0.04 & 7 &  7.00 &  0.00\\
instance n=50 387.alb & 1 & 0 & Optimal &  0.06 & 8 &  8.00 &  0.00\\
instance n=50 388.alb & 1 & 0 & Optimal &  0.04 & 7 &  7.00 &  0.00\\
instance n=50 389.alb & 1 & 0 & Optimal &  0.04 & 8 &  8.00 &  0.00\\
instance n=50 39.alb & 1 & 0 & Solution & 120.14 & 29 & 28.00 &  3.45\\
instance n=50 390.alb & 1 & 0 & Optimal &  0.07 & 7 &  7.00 &  0.00\\
instance n=50 391.alb & 1 & 0 & Optimal &  0.03 & 7 &  7.00 &  0.00\\
instance n=50 392.alb & 1 & 0 & Optimal &  0.04 & 8 &  8.00 &  0.00\\
instance n=50 393.alb & 1 & 0 & Optimal &  0.06 & 7 &  7.00 &  0.00\\
instance n=50 394.alb & 1 & 0 & Optimal &  0.03 & 8 &  8.00 &  0.00\\
instance n=50 395.alb & 1 & 0 & Optimal &  0.04 & 7 &  7.00 &  0.00\\
instance n=50 396.alb & 1 & 0 & Optimal &  0.03 & 8 &  8.00 &  0.00\\
instance n=50 397.alb & 1 & 0 & Optimal &  0.04 & 7 &  7.00 &  0.00\\
instance n=50 398.alb & 1 & 0 & Optimal &  0.03 & 6 &  6.00 &  0.00\\
instance n=50 399.alb & 1 & 0 & Optimal &  4.46 & 7 &  7.00 &  0.00\\
instance n=50 4.alb & 1 & 0 & Optimal &  0.09 & 7 &  7.00 &  0.00\\
instance n=50 40.alb & 1 & 0 & Optimal & 120.04 & 26 & 26.00 &  0.00\\
instance n=50 400.alb & 1 & 0 & Optimal &  0.04 & 8 &  8.00 &  0.00\\
instance n=50 401.alb & 1 & 0 & Optimal & 59.04 & 28 & 28.00 &  0.00\\
instance n=50 402.alb & 1 & 0 & Optimal &  2.01 & 27 & 27.00 &  0.00\\
instance n=50 403.alb & 1 & 0 & Optimal &  2.30 & 34 & 34.00 &  0.00\\
instance n=50 404.alb & 1 & 0 & Optimal &  4.18 & 31 & 31.00 &  0.00\\
instance n=50 405.alb & 1 & 0 & Optimal &  3.45 & 27 & 27.00 &  0.00\\
instance n=50 406.alb & 1 & 0 & Optimal &  2.78 & 32 & 32.00 &  0.00\\
instance n=50 407.alb & 1 & 0 & Optimal &  6.19 & 29 & 29.00 &  0.00\\
instance n=50 408.alb & 1 & 0 & Optimal &  0.44 & 26 & 26.00 &  0.00\\
instance n=50 409.alb & 1 & 0 & Optimal &  5.51 & 33 & 33.00 &  0.00\\
instance n=50 41.alb & 1 & 0 & Optimal & 120.05 & 25 & 25.00 &  0.00\\
instance n=50 410.alb & 1 & 0 & Optimal &  0.30 & 28 & 28.00 &  0.00\\
instance n=50 411.alb & 1 & 0 & Optimal &  0.08 & 29 & 29.00 &  0.00\\
instance n=50 412.alb & 1 & 0 & Optimal &  0.12 & 26 & 26.00 &  0.00\\
instance n=50 413.alb & 1 & 0 & Optimal &  0.14 & 30 & 30.00 &  0.00\\
instance n=50 414.alb & 1 & 0 & Optimal & 33.49 & 27 & 27.00 &  0.00\\
instance n=50 415.alb & 1 & 0 & Optimal &  0.31 & 28 & 28.00 &  0.00\\
instance n=50 416.alb & 1 & 0 & Optimal &  0.19 & 27 & 27.00 &  0.00\\
instance n=50 417.alb & 1 & 0 & Optimal & 59.79 & 30 & 30.00 &  0.00\\
instance n=50 418.alb & 1 & 0 & Optimal &  1.04 & 27 & 27.00 &  0.00\\
instance n=50 419.alb & 1 & 0 & Optimal & 11.42 & 33 & 33.00 &  0.00\\
instance n=50 42.alb & 1 & 0 & Solution & 120.98 & 24 & 23.00 &  4.17\\
instance n=50 420.alb & 1 & 0 & Optimal & 13.02 & 28 & 28.00 &  0.00\\
instance n=50 421.alb & 1 & 0 & Optimal &  3.69 & 34 & 34.00 &  0.00\\
instance n=50 422.alb & 1 & 0 & Optimal &  3.03 & 29 & 29.00 &  0.00\\
instance n=50 423.alb & 1 & 0 & Optimal &  0.24 & 29 & 29.00 &  0.00\\
instance n=50 424.alb & 1 & 0 & Optimal &  0.80 & 27 & 27.00 &  0.00\\
instance n=50 425.alb & 1 & 0 & Optimal &  6.30 & 34 & 34.00 &  0.00\\
instance n=50 426.alb & 1 & 0 & Optimal &  0.30 & 11 & 11.00 &  0.00\\
instance n=50 427.alb & 1 & 0 & Optimal &  0.03 & 12 & 12.00 &  0.00\\
instance n=50 428.alb & 1 & 0 & Optimal &  0.14 & 13 & 13.00 &  0.00\\
instance n=50 429.alb & 1 & 0 & Optimal &  0.04 & 11 & 11.00 &  0.00\\
instance n=50 43.alb & 1 & 0 & Optimal &  1.58 & 25 & 25.00 &  0.00\\
instance n=50 430.alb & 1 & 0 & Optimal &  1.01 & 14 & 14.00 &  0.00\\
instance n=50 431.alb & 1 & 0 & Optimal &  0.05 & 11 & 11.00 &  0.00\\
instance n=50 432.alb & 1 & 0 & Optimal &  0.37 & 12 & 12.00 &  0.00\\
instance n=50 433.alb & 1 & 0 & Optimal &  0.04 & 12 & 12.00 &  0.00\\
instance n=50 434.alb & 1 & 0 & Optimal &  0.07 & 11 & 11.00 &  0.00\\
instance n=50 435.alb & 1 & 0 & Optimal &  0.54 & 11 & 11.00 &  0.00\\
instance n=50 436.alb & 1 & 0 & Optimal &  0.23 & 11 & 11.00 &  0.00\\
instance n=50 437.alb & 1 & 0 & Optimal &  1.92 & 12 & 12.00 &  0.00\\
instance n=50 438.alb & 1 & 0 & Optimal &  1.43 & 10 & 10.00 &  0.00\\
instance n=50 439.alb & 1 & 0 & Optimal &  0.45 & 12 & 12.00 &  0.00\\
instance n=50 44.alb & 1 & 0 & Solution & 120.15 & 25 & 24.00 &  4.00\\
instance n=50 440.alb & 1 & 0 & Optimal &  0.84 & 13 & 13.00 &  0.00\\
instance n=50 441.alb & 1 & 0 & Optimal &  0.06 & 11 & 11.00 &  0.00\\
instance n=50 442.alb & 1 & 0 & Optimal &  0.11 & 12 & 12.00 &  0.00\\
instance n=50 443.alb & 1 & 0 & Optimal &  0.06 & 11 & 11.00 &  0.00\\
instance n=50 444.alb & 1 & 0 & Optimal &  0.09 & 12 & 12.00 &  0.00\\
instance n=50 445.alb & 1 & 0 & Optimal &  0.24 & 12 & 12.00 &  0.00\\
instance n=50 446.alb & 1 & 0 & Optimal &  0.08 & 12 & 12.00 &  0.00\\
instance n=50 447.alb & 1 & 0 & Optimal &  0.08 & 13 & 13.00 &  0.00\\
instance n=50 448.alb & 1 & 0 & Optimal &  0.80 & 12 & 12.00 &  0.00\\
instance n=50 449.alb & 1 & 0 & Optimal &  0.07 & 11 & 11.00 &  0.00\\
instance n=50 45.alb & 1 & 0 & Solution & 120.12 & 25 & 24.00 &  4.00\\
instance n=50 450.alb & 1 & 0 & Optimal &  0.05 & 11 & 11.00 &  0.00\\
instance n=50 451.alb & 1 & 0 & Optimal &  0.06 & 8 &  8.00 &  0.00\\
instance n=50 452.alb & 1 & 0 & Optimal &  0.03 & 8 &  8.00 &  0.00\\
instance n=50 453.alb & 1 & 0 & Optimal &  0.02 & 7 &  7.00 &  0.00\\
instance n=50 454.alb & 1 & 0 & Optimal &  0.08 & 8 &  8.00 &  0.00\\
instance n=50 455.alb & 1 & 0 & Optimal &  0.02 & 6 &  6.00 &  0.00\\
instance n=50 456.alb & 1 & 0 & Optimal &  0.04 & 8 &  8.00 &  0.00\\
instance n=50 457.alb & 1 & 0 & Optimal &  0.03 & 8 &  8.00 &  0.00\\
instance n=50 458.alb & 1 & 0 & Optimal &  0.03 & 7 &  7.00 &  0.00\\
instance n=50 459.alb & 1 & 0 & Optimal &  0.01 & 7 &  7.00 &  0.00\\
instance n=50 46.alb & 1 & 0 & Optimal &  0.31 & 28 & 28.00 &  0.00\\
instance n=50 460.alb & 1 & 0 & Optimal &  0.04 & 7 &  7.00 &  0.00\\
instance n=50 461.alb & 1 & 0 & Optimal &  0.01 & 6 &  6.00 &  0.00\\
instance n=50 462.alb & 1 & 0 & Optimal &  0.04 & 7 &  7.00 &  0.00\\
instance n=50 463.alb & 1 & 0 & Optimal &  0.04 & 8 &  8.00 &  0.00\\
instance n=50 464.alb & 1 & 0 & Optimal &  0.03 & 6 &  6.00 &  0.00\\
instance n=50 465.alb & 1 & 0 & Optimal &  0.03 & 8 &  8.00 &  0.00\\
instance n=50 466.alb & 1 & 0 & Optimal &  0.03 & 7 &  7.00 &  0.00\\
instance n=50 467.alb & 1 & 0 & Optimal &  0.08 & 9 &  9.00 &  0.00\\
instance n=50 468.alb & 1 & 0 & Optimal &  0.05 & 7 &  7.00 &  0.00\\
instance n=50 469.alb & 1 & 0 & Optimal &  0.07 & 8 &  8.00 &  0.00\\
instance n=50 47.alb & 1 & 0 & Optimal &  3.94 & 28 & 28.00 &  0.00\\
instance n=50 470.alb & 1 & 0 & Optimal &  0.03 & 8 &  8.00 &  0.00\\
instance n=50 471.alb & 1 & 0 & Optimal &  0.06 & 7 &  7.00 &  0.00\\
instance n=50 472.alb & 1 & 0 & Optimal &  0.03 & 8 &  8.00 &  0.00\\
instance n=50 473.alb & 1 & 0 & Optimal &  0.02 & 7 &  7.00 &  0.00\\
instance n=50 474.alb & 1 & 0 & Optimal &  0.01 & 7 &  7.00 &  0.00\\
instance n=50 475.alb & 1 & 0 & Optimal &  0.05 & 6 &  6.00 &  0.00\\
instance n=50 476.alb & 1 & 0 & Optimal &  0.08 & 28 & 28.00 &  0.00\\
instance n=50 477.alb & 1 & 0 & Optimal &  0.06 & 29 & 29.00 &  0.00\\
instance n=50 478.alb & 1 & 0 & Optimal &  0.12 & 32 & 32.00 &  0.00\\
instance n=50 479.alb & 1 & 0 & Optimal &  0.06 & 28 & 28.00 &  0.00\\
instance n=50 48.alb & 1 & 0 & Optimal &  0.56 & 27 & 27.00 &  0.00\\
instance n=50 480.alb & 1 & 0 & Optimal &  0.04 & 34 & 34.00 &  0.00\\
instance n=50 481.alb & 1 & 0 & Optimal &  0.06 & 28 & 28.00 &  0.00\\
instance n=50 482.alb & 1 & 0 & Optimal &  0.03 & 27 & 27.00 &  0.00\\
instance n=50 483.alb & 1 & 0 & Optimal &  0.11 & 30 & 30.00 &  0.00\\
instance n=50 484.alb & 1 & 0 & Optimal &  0.03 & 32 & 32.00 &  0.00\\
instance n=50 485.alb & 1 & 0 & Optimal &  0.08 & 31 & 31.00 &  0.00\\
instance n=50 486.alb & 1 & 0 & Optimal &  0.04 & 32 & 32.00 &  0.00\\
instance n=50 487.alb & 1 & 0 & Optimal &  0.12 & 31 & 31.00 &  0.00\\
instance n=50 488.alb & 1 & 0 & Optimal &  0.06 & 31 & 31.00 &  0.00\\
instance n=50 489.alb & 1 & 0 & Optimal &  0.05 & 35 & 35.00 &  0.00\\
instance n=50 49.alb & 1 & 0 & Optimal &  0.58 & 25 & 25.00 &  0.00\\
instance n=50 490.alb & 1 & 0 & Optimal &  0.03 & 29 & 29.00 &  0.00\\
instance n=50 491.alb & 1 & 0 & Optimal &  0.05 & 35 & 35.00 &  0.00\\
instance n=50 492.alb & 1 & 0 & Optimal &  0.04 & 29 & 29.00 &  0.00\\
instance n=50 493.alb & 1 & 0 & Optimal &  0.11 & 30 & 30.00 &  0.00\\
instance n=50 494.alb & 1 & 0 & Optimal &  0.04 & 32 & 32.00 &  0.00\\
instance n=50 495.alb & 1 & 0 & Optimal &  0.06 & 34 & 34.00 &  0.00\\
instance n=50 496.alb & 1 & 0 & Optimal &  0.09 & 29 & 29.00 &  0.00\\
instance n=50 497.alb & 1 & 0 & Optimal &  0.12 & 30 & 30.00 &  0.00\\
instance n=50 498.alb & 1 & 0 & Optimal &  0.08 & 30 & 30.00 &  0.00\\
instance n=50 499.alb & 1 & 0 & Optimal &  0.06 & 33 & 33.00 &  0.00\\
instance n=50 5.alb & 1 & 0 & Optimal &  0.05 & 7 &  7.00 &  0.00\\
instance n=50 50.alb & 1 & 0 & Solution & 120.15 & 27 & 26.00 &  3.70\\
instance n=50 500.alb & 1 & 0 & Optimal &  0.04 & 34 & 34.00 &  0.00\\
instance n=50 501.alb & 1 & 0 & Optimal &  0.04 & 12 & 12.00 &  0.00\\
instance n=50 502.alb & 1 & 0 & Optimal &  0.03 & 10 & 10.00 &  0.00\\
instance n=50 503.alb & 1 & 0 & Optimal &  0.06 & 13 & 13.00 &  0.00\\
instance n=50 504.alb & 1 & 0 & Optimal &  0.05 & 11 & 11.00 &  0.00\\
instance n=50 505.alb & 1 & 0 & Optimal &  0.03 & 12 & 12.00 &  0.00\\
instance n=50 506.alb & 1 & 0 & Optimal &  0.04 & 11 & 11.00 &  0.00\\
instance n=50 507.alb & 1 & 0 & Optimal &  0.05 & 13 & 13.00 &  0.00\\
instance n=50 508.alb & 1 & 0 & Optimal &  0.04 & 14 & 14.00 &  0.00\\
instance n=50 509.alb & 1 & 0 & Optimal &  0.04 & 13 & 13.00 &  0.00\\
instance n=50 51.alb & 1 & 0 & Optimal &  0.53 & 12 & 12.00 &  0.00\\
instance n=50 510.alb & 1 & 0 & Optimal &  0.08 & 11 & 11.00 &  0.00\\
instance n=50 511.alb & 1 & 0 & Optimal &  0.03 & 13 & 13.00 &  0.00\\
instance n=50 512.alb & 1 & 0 & Optimal &  0.05 & 13 & 13.00 &  0.00\\
instance n=50 513.alb & 1 & 0 & Optimal &  0.07 & 12 & 12.00 &  0.00\\
instance n=50 514.alb & 1 & 0 & Optimal &  0.05 & 12 & 12.00 &  0.00\\
instance n=50 515.alb & 1 & 0 & Optimal &  0.05 & 11 & 11.00 &  0.00\\
instance n=50 516.alb & 1 & 0 & Optimal &  0.04 & 13 & 13.00 &  0.00\\
instance n=50 517.alb & 1 & 0 & Optimal &  0.04 & 14 & 14.00 &  0.00\\
instance n=50 518.alb & 1 & 0 & Optimal &  0.04 & 11 & 11.00 &  0.00\\
instance n=50 519.alb & 1 & 0 & Optimal &  0.04 & 12 & 12.00 &  0.00\\
instance n=50 52.alb & 1 & 0 & Optimal &  0.03 & 11 & 11.00 &  0.00\\
instance n=50 520.alb & 1 & 0 & Optimal &  0.04 & 11 & 11.00 &  0.00\\
instance n=50 521.alb & 1 & 0 & Optimal &  0.05 & 10 & 10.00 &  0.00\\
instance n=50 522.alb & 1 & 0 & Optimal &  0.04 & 11 & 11.00 &  0.00\\
instance n=50 523.alb & 1 & 0 & Optimal &  0.05 & 11 & 11.00 &  0.00\\
instance n=50 524.alb & 1 & 0 & Optimal &  0.04 & 14 & 14.00 &  0.00\\
instance n=50 525.alb & 1 & 0 & Optimal &  0.08 & 11 & 11.00 &  0.00\\
instance n=50 53.alb & 1 & 0 & Solution & 120.12 & 13 & 12.00 &  7.69\\
instance n=50 54.alb & 1 & 0 & Optimal &  0.04 & 11 & 11.00 &  0.00\\
instance n=50 55.alb & 1 & 0 & Optimal &  0.05 & 13 & 13.00 &  0.00\\
instance n=50 56.alb & 1 & 0 & Optimal &  0.04 & 11 & 11.00 &  0.00\\
instance n=50 57.alb & 1 & 0 & Optimal &  0.07 & 13 & 13.00 &  0.00\\
instance n=50 58.alb & 1 & 0 & Optimal &  0.12 & 11 & 11.00 &  0.00\\
instance n=50 59.alb & 1 & 0 & Optimal &  7.70 & 11 & 11.00 &  0.00\\
instance n=50 6.alb & 1 & 0 & Optimal &  0.06 & 6 &  6.00 &  0.00\\
instance n=50 60.alb & 1 & 0 & Optimal &  0.30 & 12 & 12.00 &  0.00\\
instance n=50 61.alb & 1 & 0 & Optimal &  0.06 & 13 & 13.00 &  0.00\\
instance n=50 62.alb & 1 & 0 & Optimal &  0.03 & 13 & 13.00 &  0.00\\
instance n=50 63.alb & 1 & 0 & Optimal & 120.04 & 12 & 12.00 &  0.00\\
instance n=50 64.alb & 1 & 0 & Optimal &  0.05 & 13 & 13.00 &  0.00\\
instance n=50 65.alb & 1 & 0 & Optimal &  1.96 & 12 & 12.00 &  0.00\\
instance n=50 66.alb & 1 & 0 & Optimal &  0.76 & 12 & 12.00 &  0.00\\
instance n=50 67.alb & 1 & 0 & Optimal &  0.33 & 12 & 12.00 &  0.00\\
instance n=50 68.alb & 1 & 0 & Optimal &  0.06 & 12 & 12.00 &  0.00\\
instance n=50 69.alb & 1 & 0 & Optimal &  0.14 & 12 & 12.00 &  0.00\\
instance n=50 7.alb & 1 & 0 & Optimal &  0.03 & 7 &  7.00 &  0.00\\
instance n=50 70.alb & 1 & 0 & Optimal &  0.04 & 10 & 10.00 &  0.00\\
instance n=50 71.alb & 1 & 0 & Optimal &  0.15 & 13 & 13.00 &  0.00\\
instance n=50 72.alb & 1 & 0 & Optimal & 37.28 & 11 & 11.00 &  0.00\\
instance n=50 73.alb & 1 & 0 & Optimal &  0.09 & 11 & 11.00 &  0.00\\
instance n=50 74.alb & 1 & 0 & Optimal & 32.81 & 12 & 12.00 &  0.00\\
instance n=50 75.alb & 1 & 0 & Optimal &  0.32 & 11 & 11.00 &  0.00\\
instance n=50 76.alb & 1 & 0 & Optimal &  0.03 & 7 &  7.00 &  0.00\\
instance n=50 77.alb & 1 & 0 & Optimal &  0.04 & 7 &  7.00 &  0.00\\
instance n=50 78.alb & 1 & 0 & Optimal &  1.81 & 7 &  7.00 &  0.00\\
instance n=50 79.alb & 1 & 0 & Optimal &  0.28 & 8 &  8.00 &  0.00\\
instance n=50 8.alb & 1 & 0 & Optimal &  4.95 & 7 &  7.00 &  0.00\\
instance n=50 80.alb & 1 & 0 & Optimal &  1.83 & 7 &  7.00 &  0.00\\
instance n=50 81.alb & 1 & 0 & Optimal &  0.03 & 7 &  7.00 &  0.00\\
instance n=50 82.alb & 1 & 0 & Optimal &  0.04 & 6 &  6.00 &  0.00\\
instance n=50 83.alb & 1 & 0 & Optimal &  0.07 & 8 &  8.00 &  0.00\\
instance n=50 84.alb & 1 & 0 & Optimal &  0.03 & 7 &  7.00 &  0.00\\
instance n=50 85.alb & 1 & 0 & Optimal &  0.05 & 8 &  8.00 &  0.00\\
instance n=50 86.alb & 1 & 0 & Optimal &  0.05 & 7 &  7.00 &  0.00\\
instance n=50 87.alb & 1 & 0 & Optimal &  0.08 & 8 &  8.00 &  0.00\\
instance n=50 88.alb & 1 & 0 & Optimal &  1.30 & 8 &  8.00 &  0.00\\
instance n=50 89.alb & 1 & 0 & Optimal &  0.07 & 7 &  7.00 &  0.00\\
instance n=50 9.alb & 1 & 0 & Optimal &  1.13 & 9 &  9.00 &  0.00\\
instance n=50 90.alb & 1 & 0 & Optimal &  0.05 & 7 &  7.00 &  0.00\\
instance n=50 91.alb & 1 & 0 & Optimal &  3.30 & 7 &  7.00 &  0.00\\
instance n=50 92.alb & 1 & 0 & Optimal &  0.06 & 7 &  7.00 &  0.00\\
instance n=50 93.alb & 1 & 0 & Optimal &  0.03 & 7 &  7.00 &  0.00\\
instance n=50 94.alb & 1 & 0 & Optimal &  0.92 & 7 &  7.00 &  0.00\\
instance n=50 95.alb & 1 & 0 & Optimal &  0.06 & 7 &  7.00 &  0.00\\
instance n=50 96.alb & 1 & 0 & Optimal &  0.06 & 7 &  7.00 &  0.00\\
instance n=50 97.alb & 1 & 0 & Optimal &  0.04 & 7 &  7.00 &  0.00\\
instance n=50 98.alb & 1 & 0 & Optimal &  0.12 & 8 &  8.00 &  0.00\\
instance n=50 99.alb & 1 & 0 & Optimal &  0.09 & 7 &  7.00 &  0.00\\
\end{longtable}



\clearpage
\chapter{Test Scheduling Problems}

Due to the number of instances given, we only run problems for 30 seconds, some results are still missing. The original instance data was given in Prolog format, we generate a JSON equivalent, which is used as input to create the problems.

\section{Results for CPOptimizer}

\begin{longtable}{lrrlrrrr}
\caption{Results for Test Scheduling Problems (544 Instances)}\\\toprule
Name & \shortstack{Nr\\Jobs} & \shortstack{Nr\\Machines} & Status & Time & Makespan & Bound & \shortstack{Gap\\Percent}\\ \midrule
\endhead
\bottomrule
\endfoot
t100m10r10-1.pl.json & 100 & 10 & Solution & 30.24 & 10491 & 9055.00 & 13.69\\
t100m10r10-10.pl.json & 100 & 10 & Solution & 30.05 & 9593 & 8369.00 & 12.76\\
t100m10r10-11.pl.json & 100 & 10 & Solution & 30.06 & 5317 & 5100.00 &  4.08\\
t100m10r10-12.pl.json & 100 & 10 & Solution & 30.07 & 6539 & 5613.00 & 14.16\\
t100m10r10-13.pl.json & 100 & 10 & Solution & 30.05 & 6831 & 6786.00 &  0.66\\
t100m10r10-14.pl.json & 100 & 10 & Solution & 30.04 & 5775 & 5257.00 &  8.97\\
t100m10r10-15.pl.json & 100 & 10 & Solution & 30.04 & 6105 & 5012.00 & 17.90\\
t100m10r10-16.pl.json & 100 & 10 & Solution & 30.08 & 12563 & 11589.00 &  7.75\\
t100m10r10-17.pl.json & 100 & 10 & Solution & 30.09 & 8954 & 8114.00 &  9.38\\
t100m10r10-18.pl.json & 100 & 10 & Solution & 30.04 & 10180 & 9304.00 &  8.61\\
t100m10r10-19.pl.json & 100 & 10 & Solution & 30.09 & 9812 & 8514.00 & 13.23\\
t100m10r10-2.pl.json & 100 & 10 & Solution & 30.07 & 11593 & 9807.00 & 15.41\\
t100m10r10-20.pl.json & 100 & 10 & Solution & 30.15 & 12287 & 10686.00 & 13.03\\
t100m10r10-3.pl.json & 100 & 10 & Solution & 30.06 & 6878 & 6379.00 &  7.26\\
t100m10r10-4.pl.json & 100 & 10 & Solution & 30.11 & 11041 & 9111.00 & 17.48\\
t100m10r10-5.pl.json & 100 & 10 & Solution & 30.09 & 12157 & 11823.00 &  2.75\\
t100m10r10-6.pl.json & 100 & 10 & Solution & 30.06 & 11688 & 10914.00 &  6.62\\
t100m10r10-7.pl.json & 100 & 10 & Solution & 30.05 & 6435 & 5732.00 & 10.92\\
t100m10r10-8.pl.json & 100 & 10 & Solution & 30.10 & 11056 & 10010.00 &  9.46\\
t100m10r10-9.pl.json & 100 & 10 & Solution & 30.11 & 9878 & 7991.00 & 19.10\\
t100m10r3-1.pl.json & 100 & 10 & Optimal &  0.62 & 8711 & 8711.00 &  0.00\\
t100m10r3-10.pl.json & 100 & 10 & Optimal &  0.43 & 8958 & 8958.00 &  0.00\\
t100m10r3-11.pl.json & 100 & 10 & Optimal &  0.15 & 9560 & 9560.00 &  0.00\\
t100m10r3-12.pl.json & 100 & 10 & Optimal &  0.38 & 7892 & 7892.00 &  0.00\\
t100m10r3-13.pl.json & 100 & 10 & Optimal &  0.09 & 10078 & 10077.00 &  0.01\\
t100m10r3-14.pl.json & 100 & 10 & Optimal &  0.36 & 8681 & 8681.00 &  0.00\\
t100m10r3-15.pl.json & 100 & 10 & Optimal &  0.17 & 8810 & 8810.00 &  0.00\\
t100m10r3-16.pl.json & 100 & 10 & Optimal &  0.47 & 11182 & 11182.00 &  0.00\\
t100m10r3-17.pl.json & 100 & 10 & Optimal &  0.74 & 7534 & 7534.00 &  0.00\\
t100m10r3-18.pl.json & 100 & 10 & Solution & 30.10 & 10376 & 9934.00 &  4.26\\
t100m10r3-19.pl.json & 100 & 10 & Solution & 30.03 & 7706 & 6970.00 &  9.55\\
t100m10r3-2.pl.json & 100 & 10 & Optimal &  0.29 & 7082 & 7082.00 &  0.00\\
t100m10r3-20.pl.json & 100 & 10 & Optimal &  0.17 & 9025 & 9025.00 &  0.00\\
t100m10r3-3.pl.json & 100 & 10 & Optimal &  0.42 & 10054 & 10053.00 &  0.01\\
t100m10r3-4.pl.json & 100 & 10 & Optimal &  0.10 & 13122 & 13121.00 &  0.01\\
t100m10r3-5.pl.json & 100 & 10 & Optimal &  1.50 & 7545 & 7545.00 &  0.00\\
t100m10r3-6.pl.json & 100 & 10 & Optimal &  0.93 & 7840 & 7840.00 &  0.00\\
t100m10r3-7.pl.json & 100 & 10 & Optimal &  0.16 & 11010 & 11009.00 &  0.01\\
t100m10r3-8.pl.json & 100 & 10 & Optimal &  0.16 & 9112 & 9112.00 &  0.00\\
t100m10r3-9.pl.json & 100 & 10 & Optimal &  0.34 & 8532 & 8532.00 &  0.00\\
t100m10r5-1.pl.json & 100 & 10 & Solution & 30.04 & 7304 & 7300.00 &  0.05\\
t100m10r5-10.pl.json & 100 & 10 & Optimal &  1.42 & 6972 & 6972.00 &  0.00\\
t100m10r5-11.pl.json & 100 & 10 & Solution & 30.08 & 9091 & 8568.00 &  5.75\\
t100m10r5-12.pl.json & 100 & 10 & Optimal &  0.66 & 6538 & 6538.00 &  0.00\\
t100m10r5-13.pl.json & 100 & 10 & Optimal &  0.67 & 8972 & 8972.00 &  0.00\\
t100m10r5-14.pl.json & 100 & 10 & Solution & 30.07 & 10478 & 10347.00 &  1.25\\
t100m10r5-15.pl.json & 100 & 10 & Solution & 30.05 & 5762 & 5647.00 &  2.00\\
t100m10r5-16.pl.json & 100 & 10 & Solution & 30.04 & 7019 & 6207.00 & 11.57\\
t100m10r5-17.pl.json & 100 & 10 & Optimal &  0.23 & 6728 & 6728.00 &  0.00\\
t100m10r5-18.pl.json & 100 & 10 & Solution & 30.12 & 8987 & 8811.00 &  1.96\\
t100m10r5-19.pl.json & 100 & 10 & Optimal &  0.98 & 8885 & 8885.00 &  0.00\\
t100m10r5-2.pl.json & 100 & 10 & Optimal &  2.05 & 9010 & 9010.00 &  0.00\\
t100m10r5-20.pl.json & 100 & 10 & Optimal &  0.91 & 7022 & 7022.00 &  0.00\\
t100m10r5-3.pl.json & 100 & 10 & Optimal &  0.99 & 8820 & 8820.00 &  0.00\\
t100m10r5-4.pl.json & 100 & 10 & Optimal &  1.02 & 10753 & 10753.00 &  0.00\\
t100m10r5-5.pl.json & 100 & 10 & Optimal &  2.03 & 6608 & 6608.00 &  0.00\\
t100m10r5-6.pl.json & 100 & 10 & Solution & 30.06 & 9452 & 8456.00 & 10.54\\
t100m10r5-7.pl.json & 100 & 10 & Solution & 30.05 & 8186 & 7664.00 &  6.38\\
t100m10r5-8.pl.json & 100 & 10 & Solution & 30.12 & 11383 & 10079.00 & 11.46\\
t100m10r5-9.pl.json & 100 & 10 & Solution & 30.05 & 11649 & 10683.00 &  8.29\\
t100m20r10-1.pl.json & 100 & 20 & Solution & 30.19 & 12412 & 12180.00 &  1.87\\
t100m20r10-10.pl.json & 100 & 20 & Solution & 30.05 & 12646 & 10953.00 & 13.39\\
t100m20r10-11.pl.json & 100 & 20 & Solution & 30.09 & 8687 & 7289.00 & 16.09\\
t100m20r10-12.pl.json & 100 & 20 & Solution & 30.20 & 7391 & 6774.00 &  8.35\\
t100m20r10-13.pl.json & 100 & 20 & Solution & 30.08 & 9695 & 9229.00 &  4.81\\
t100m20r10-14.pl.json & 100 & 20 & Solution & 30.16 & 10027 & 8652.00 & 13.71\\
t100m20r10-15.pl.json & 100 & 20 & Solution & 30.04 & 6544 & 5362.00 & 18.06\\
t100m20r10-16.pl.json & 100 & 20 & Solution & 30.10 & 9264 & 8343.00 &  9.94\\
t100m20r10-17.pl.json & 100 & 20 & Solution & 30.15 & 8611 & 7381.00 & 14.28\\
t100m20r10-18.pl.json & 100 & 20 & Optimal &  1.74 & 4843 & 4843.00 &  0.00\\
t100m20r10-19.pl.json & 100 & 20 & Solution & 30.16 & 12320 & 11752.00 &  4.61\\
t100m20r10-2.pl.json & 100 & 20 & Solution & 30.14 & 7740 & 6890.00 & 10.98\\
t100m20r10-20.pl.json & 100 & 20 & Solution & 30.11 & 9873 & 8562.00 & 13.28\\
t100m20r10-3.pl.json & 100 & 20 & Solution & 30.07 & 7133 & 6295.00 & 11.75\\
t100m20r10-4.pl.json & 100 & 20 & Solution & 30.21 & 9510 & 9052.00 &  4.82\\
t100m20r10-5.pl.json & 100 & 20 & Solution & 30.13 & 9230 & 8459.00 &  8.35\\
t100m20r10-6.pl.json & 100 & 20 & Solution & 30.10 & 8781 & 7619.00 & 13.23\\
t100m20r10-7.pl.json & 100 & 20 & Solution & 30.18 & 11313 & 9767.00 & 13.67\\
t100m20r10-8.pl.json & 100 & 20 & Solution & 30.12 & 7096 & 7041.00 &  0.78\\
t100m20r10-9.pl.json & 100 & 20 & Solution & 30.19 & 10835 & 10019.00 &  7.53\\
t100m20r3-1.pl.json & 100 & 20 & Optimal &  0.59 & 6585 & 6585.00 &  0.00\\
t100m20r3-10.pl.json & 100 & 20 & Optimal &  0.28 & 8535 & 8535.00 &  0.00\\
t100m20r3-11.pl.json & 100 & 20 & Optimal &  0.60 & 9084 & 9084.00 &  0.00\\
t100m20r3-12.pl.json & 100 & 20 & Optimal &  0.28 & 9066 & 9066.00 &  0.00\\
t100m20r3-13.pl.json & 100 & 20 & Solution & 30.09 & 11412 & 9974.00 & 12.60\\
t100m20r3-14.pl.json & 100 & 20 & Optimal &  0.54 & 8786 & 8786.00 &  0.00\\
t100m20r3-15.pl.json & 100 & 20 & Optimal &  0.27 & 10205 & 10204.00 &  0.01\\
t100m20r3-16.pl.json & 100 & 20 & Optimal &  0.28 & 8856 & 8856.00 &  0.00\\
t100m20r3-17.pl.json & 100 & 20 & Optimal &  1.30 & 5451 & 5451.00 &  0.00\\
t100m20r3-18.pl.json & 100 & 20 & Optimal &  0.51 & 8752 & 8752.00 &  0.00\\
t100m20r3-19.pl.json & 100 & 20 & Solution & 30.13 & 8909 & 8860.00 &  0.55\\
t100m20r3-2.pl.json & 100 & 20 & Optimal &  0.26 & 8498 & 8498.00 &  0.00\\
t100m20r3-20.pl.json & 100 & 20 & Optimal &  0.87 & 7880 & 7880.00 &  0.00\\
t100m20r3-3.pl.json & 100 & 20 & Solution & 30.21 & 12170 & 11987.00 &  1.50\\
t100m20r3-4.pl.json & 100 & 20 & Optimal &  0.53 & 12258 & 12257.00 &  0.01\\
t100m20r3-5.pl.json & 100 & 20 & Optimal &  0.25 & 11932 & 11931.00 &  0.01\\
t100m20r3-6.pl.json & 100 & 20 & Optimal &  0.28 & 8531 & 8531.00 &  0.00\\
t100m20r3-7.pl.json & 100 & 20 & Optimal &  0.28 & 6512 & 6512.00 &  0.00\\
t100m20r3-8.pl.json & 100 & 20 & Optimal &  3.31 & 10690 & 10689.00 &  0.01\\
t100m20r3-9.pl.json & 100 & 20 & Optimal &  0.30 & 8255 & 8255.00 &  0.00\\
t100m20r5-1.pl.json & 100 & 20 & Optimal &  0.34 & 9098 & 9098.00 &  0.00\\
t100m20r5-10.pl.json & 100 & 20 & Solution & 30.04 & 8340 & 7964.00 &  4.51\\
t100m20r5-11.pl.json & 100 & 20 & Solution & 30.11 & 6828 & 5564.00 & 18.51\\
t100m20r5-12.pl.json & 100 & 20 & Optimal &  3.25 & 8704 & 8704.00 &  0.00\\
t100m20r5-13.pl.json & 100 & 20 & Optimal &  0.70 & 8880 & 8880.00 &  0.00\\
t100m20r5-14.pl.json & 100 & 20 & Solution & 30.26 & 10590 & 9727.00 &  8.15\\
t100m20r5-15.pl.json & 100 & 20 & Optimal &  0.59 & 8953 & 8953.00 &  0.00\\
t100m20r5-16.pl.json & 100 & 20 & Solution & 30.15 & 7864 & 7594.00 &  3.43\\
t100m20r5-17.pl.json & 100 & 20 & Solution & 30.15 & 5685 & 5524.00 &  2.83\\
t100m20r5-18.pl.json & 100 & 20 & Optimal &  1.06 & 6617 & 6617.00 &  0.00\\
t100m20r5-19.pl.json & 100 & 20 & Optimal &  0.42 & 9461 & 9461.00 &  0.00\\
t100m20r5-2.pl.json & 100 & 20 & Optimal &  0.38 & 9566 & 9566.00 &  0.00\\
t100m20r5-20.pl.json & 100 & 20 & Solution & 30.06 & 11569 & 10228.00 & 11.59\\
t100m20r5-3.pl.json & 100 & 20 & Optimal &  1.74 & 9366 & 9366.00 &  0.00\\
t100m20r5-4.pl.json & 100 & 20 & Solution & 30.07 & 14108 & 12456.00 & 11.71\\
t100m20r5-5.pl.json & 100 & 20 & Optimal &  0.35 & 8585 & 8585.00 &  0.00\\
t100m20r5-6.pl.json & 100 & 20 & Solution & 30.12 & 7528 & 6539.00 & 13.14\\
t100m20r5-7.pl.json & 100 & 20 & Solution & 30.13 & 11254 & 10099.00 & 10.26\\
t100m20r5-8.pl.json & 100 & 20 & Optimal &  2.49 & 5812 & 5812.00 &  0.00\\
t100m20r5-9.pl.json & 100 & 20 & Solution & 30.16 & 6634 & 6496.00 &  2.08\\
t100m50r10-1.pl.json & 100 & 50 & Solution & 30.17 & 7299 & 6941.00 &  4.90\\
t100m50r10-10.pl.json & 100 & 50 & Solution & 30.23 & 5201 & 5108.00 &  1.79\\
t100m50r10-11.pl.json & 100 & 50 & Solution & 30.09 & 4970 & 4782.00 &  3.78\\
t100m50r10-12.pl.json & 100 & 50 & Solution & 30.06 & 9335 & 9122.00 &  2.28\\
t100m50r10-13.pl.json & 100 & 50 & Solution & 30.26 & 9759 & 8828.00 &  9.54\\
t100m50r10-14.pl.json & 100 & 50 & Solution & 30.10 & 10704 & 8290.00 & 22.55\\
t100m50r10-15.pl.json & 100 & 50 & Solution & 30.08 & 8637 & 7804.00 &  9.64\\
t100m50r10-16.pl.json & 100 & 50 & Solution & 30.14 & 14087 & 12381.00 & 12.11\\
t100m50r10-17.pl.json & 100 & 50 & Solution & 30.18 & 9600 & 9151.00 &  4.68\\
t100m50r10-18.pl.json & 100 & 50 & Solution & 30.34 & 7214 & 7120.00 &  1.30\\
t100m50r10-19.pl.json & 100 & 50 & Solution & 30.18 & 8559 & 8059.00 &  5.84\\
t100m50r10-2.pl.json & 100 & 50 & Solution & 30.25 & 7968 & 7568.00 &  5.02\\
t100m50r10-20.pl.json & 100 & 50 & Solution & 30.09 & 8421 & 7939.00 &  5.72\\
t100m50r10-3.pl.json & 100 & 50 & Optimal &  0.33 & 6937 & 6937.00 &  0.00\\
t100m50r10-4.pl.json & 100 & 50 & Solution & 30.16 & 9952 & 8525.00 & 14.34\\
t100m50r10-5.pl.json & 100 & 50 & Optimal &  1.35 & 9859 & 9859.00 &  0.00\\
t100m50r10-6.pl.json & 100 & 50 & Solution & 30.31 & 7696 & 6837.00 & 11.16\\
t100m50r10-7.pl.json & 100 & 50 & Optimal &  1.17 & 9542 & 9542.00 &  0.00\\
t100m50r10-8.pl.json & 100 & 50 & Solution & 30.07 & 10719 & 9176.00 & 14.39\\
t100m50r10-9.pl.json & 100 & 50 & Solution & 30.07 & 10411 & 9375.00 &  9.95\\
t100m50r3-1.pl.json & 100 & 50 & Optimal &  0.46 & 9937 & 9937.00 &  0.00\\
t100m50r3-10.pl.json & 100 & 50 & Solution & 30.06 & 8946 & 8877.00 &  0.77\\
t100m50r3-11.pl.json & 100 & 50 & Optimal &  1.01 & 6141 & 6141.00 &  0.00\\
t100m50r3-12.pl.json & 100 & 50 & Optimal &  0.87 & 6473 & 6473.00 &  0.00\\
t100m50r3-13.pl.json & 100 & 50 & Optimal &  0.47 & 8653 & 8653.00 &  0.00\\
t100m50r3-14.pl.json & 100 & 50 & Solution & 30.09 & 13018 & 12796.00 &  1.71\\
t100m50r3-15.pl.json & 100 & 50 & Optimal &  3.29 & 9056 & 9056.00 &  0.00\\
t100m50r3-16.pl.json & 100 & 50 & Optimal &  0.41 & 8680 & 8680.00 &  0.00\\
t100m50r3-17.pl.json & 100 & 50 & Optimal &  0.55 & 8197 & 8197.00 &  0.00\\
t100m50r3-18.pl.json & 100 & 50 & Optimal &  0.38 & 9318 & 9318.00 &  0.00\\
t100m50r3-19.pl.json & 100 & 50 & Optimal &  0.35 & 12265 & 12264.00 &  0.01\\
t100m50r3-2.pl.json & 100 & 50 & Optimal &  0.79 & 11030 & 11029.00 &  0.01\\
t100m50r3-20.pl.json & 100 & 50 & Optimal &  0.38 & 7662 & 7662.00 &  0.00\\
t100m50r3-3.pl.json & 100 & 50 & Optimal &  0.46 & 5348 & 5348.00 &  0.00\\
t100m50r3-4.pl.json & 100 & 50 & Optimal &  2.02 & 7800 & 7800.00 &  0.00\\
t100m50r3-5.pl.json & 100 & 50 & Optimal &  0.83 & 4207 & 4207.00 &  0.00\\
t100m50r3-6.pl.json & 100 & 50 & Optimal &  6.31 & 10596 & 10596.00 &  0.00\\
t100m50r3-7.pl.json & 100 & 50 & Optimal &  0.43 & 7826 & 7826.00 &  0.00\\
t100m50r3-8.pl.json & 100 & 50 & Optimal &  0.81 & 7865 & 7865.00 &  0.00\\
t100m50r3-9.pl.json & 100 & 50 & Optimal &  0.48 & 7891 & 7891.00 &  0.00\\
t100m50r5-1.pl.json & 100 & 50 & Optimal &  0.78 & 7926 & 7926.00 &  0.00\\
t100m50r5-10.pl.json & 100 & 50 & Solution & 30.23 & 7299 & 6521.00 & 10.66\\
t100m50r5-11.pl.json & 100 & 50 & Optimal &  1.56 & 9417 & 9417.00 &  0.00\\
t100m50r5-12.pl.json & 100 & 50 & Optimal &  3.81 & 8824 & 8824.00 &  0.00\\
t100m50r5-13.pl.json & 100 & 50 & Solution & 30.05 & 10473 & 9115.00 & 12.97\\
t100m50r5-14.pl.json & 100 & 50 & Solution & 30.33 & 7503 & 7134.00 &  4.92\\
t100m50r5-15.pl.json & 100 & 50 & Solution & 30.06 & 10141 & 9853.00 &  2.84\\
t100m50r5-16.pl.json & 100 & 50 & Optimal &  0.47 & 6481 & 6481.00 &  0.00\\
t100m50r5-17.pl.json & 100 & 50 & Optimal &  0.50 & 6129 & 6129.00 &  0.00\\
t100m50r5-18.pl.json & 100 & 50 & Solution & 30.06 & 9100 & 8337.00 &  8.38\\
t100m50r5-19.pl.json & 100 & 50 & Solution & 30.20 & 6762 & 6356.00 &  6.00\\
t100m50r5-2.pl.json & 100 & 50 & Optimal &  1.00 & 6651 & 6651.00 &  0.00\\
t100m50r5-20.pl.json & 100 & 50 & Solution & 30.05 & 6894 & 6667.00 &  3.29\\
t100m50r5-3.pl.json & 100 & 50 & Solution & 30.19 & 7944 & 7857.00 &  1.10\\
t100m50r5-4.pl.json & 100 & 50 & Optimal &  1.39 & 8296 & 8296.00 &  0.00\\
t100m50r5-5.pl.json & 100 & 50 & Optimal &  1.26 & 9977 & 9977.00 &  0.00\\
t100m50r5-6.pl.json & 100 & 50 & Optimal &  0.91 & 8240 & 8240.00 &  0.00\\
t100m50r5-7.pl.json & 100 & 50 & Optimal &  1.34 & 10904 & 10903.00 &  0.01\\
t100m50r5-8.pl.json & 100 & 50 & Optimal &  0.90 & 8293 & 8293.00 &  0.00\\
t100m50r5-9.pl.json & 100 & 50 & Solution & 30.06 & 7879 & 7622.00 &  3.26\\
t20m10r10-1.pl.json & 20 & 10 & Optimal &  0.07 & 1337 & 1337.00 &  0.00\\
t20m10r10-10.pl.json & 20 & 10 & Optimal &  0.05 & 3882 & 3882.00 &  0.00\\
t20m10r10-11.pl.json & 20 & 10 & Optimal &  0.06 & 2002 & 2002.00 &  0.00\\
t20m10r10-12.pl.json & 20 & 10 & Optimal &  0.31 & 1257 & 1257.00 &  0.00\\
t20m10r10-13.pl.json & 20 & 10 & Optimal &  0.06 & 2110 & 2110.00 &  0.00\\
t20m10r10-14.pl.json & 20 & 10 & Optimal &  2.43 & 2546 & 2546.00 &  0.00\\
t20m10r10-15.pl.json & 20 & 10 & Optimal &  0.05 & 3344 & 3344.00 &  0.00\\
t20m10r10-16.pl.json & 20 & 10 & Optimal &  3.87 & 1643 & 1643.00 &  0.00\\
t20m10r10-17.pl.json & 20 & 10 & Optimal &  0.43 & 1069 & 1069.00 &  0.00\\
t20m10r10-18.pl.json & 20 & 10 & Optimal &  0.04 & 3041 & 3041.00 &  0.00\\
t20m10r10-19.pl.json & 20 & 10 & Optimal &  0.04 & 2422 & 2422.00 &  0.00\\
t20m10r10-2.pl.json & 20 & 10 & Optimal &  0.05 & 1819 & 1819.00 &  0.00\\
t20m10r10-20.pl.json & 20 & 10 & Optimal &  0.05 & 1595 & 1595.00 &  0.00\\
t20m10r10-3.pl.json & 20 & 10 & Solution & 30.02 & 843 & 771.00 &  8.54\\
t20m10r10-4.pl.json & 20 & 10 & Optimal &  0.04 & 1396 & 1396.00 &  0.00\\
t20m10r10-5.pl.json & 20 & 10 & Optimal &  0.05 & 1710 & 1710.00 &  0.00\\
t20m10r10-6.pl.json & 20 & 10 & Optimal &  0.03 & 2434 & 2434.00 &  0.00\\
t20m10r10-7.pl.json & 20 & 10 & Optimal &  0.41 & 2696 & 2696.00 &  0.00\\
t20m10r10-8.pl.json & 20 & 10 & Optimal &  0.03 & 1329 & 1329.00 &  0.00\\
t20m10r10-9.pl.json & 20 & 10 & Optimal &  4.48 & 2933 & 2933.00 &  0.00\\
t20m10r3-1.pl.json & 20 & 10 & Optimal &  0.05 & 1876 & 1876.00 &  0.00\\
t20m10r3-10.pl.json & 20 & 10 & Optimal &  0.05 & 1652 & 1652.00 &  0.00\\
t20m10r3-11.pl.json & 20 & 10 & Optimal &  0.04 & 1640 & 1640.00 &  0.00\\
t20m10r3-12.pl.json & 20 & 10 & Optimal &  0.03 & 1758 & 1758.00 &  0.00\\
t20m10r3-13.pl.json & 20 & 10 & Optimal &  0.03 & 3099 & 3099.00 &  0.00\\
t20m10r3-14.pl.json & 20 & 10 & Solution & 30.01 & 3891 & 3520.00 &  9.53\\
t20m10r3-15.pl.json & 20 & 10 & Optimal &  0.05 & 1433 & 1433.00 &  0.00\\
t20m10r3-16.pl.json & 20 & 10 & Optimal &  0.04 & 1564 & 1564.00 &  0.00\\
t20m10r3-17.pl.json & 20 & 10 & Optimal &  0.04 & 2321 & 2321.00 &  0.00\\
t20m10r3-18.pl.json & 20 & 10 & Solution & 30.01 & 821 & 746.00 &  9.14\\
t20m10r3-19.pl.json & 20 & 10 & Optimal &  0.09 & 1236 & 1236.00 &  0.00\\
t20m10r3-2.pl.json & 20 & 10 & Optimal &  0.05 & 3258 & 3258.00 &  0.00\\
t20m10r3-20.pl.json & 20 & 10 & Optimal &  0.04 & 2168 & 2168.00 &  0.00\\
t20m10r3-3.pl.json & 20 & 10 & Optimal &  0.03 & 2255 & 2255.00 &  0.00\\
t20m10r3-4.pl.json & 20 & 10 & Optimal &  0.03 & 2707 & 2707.00 &  0.00\\
t20m10r3-5.pl.json & 20 & 10 & Optimal &  0.05 & 2381 & 2381.00 &  0.00\\
t20m10r3-6.pl.json & 20 & 10 & Optimal &  0.03 & 3043 & 3043.00 &  0.00\\
t20m10r3-7.pl.json & 20 & 10 & Optimal &  0.05 & 1738 & 1738.00 &  0.00\\
t20m10r3-8.pl.json & 20 & 10 & Optimal &  2.74 & 1278 & 1278.00 &  0.00\\
t20m10r3-9.pl.json & 20 & 10 & Optimal &  0.04 & 2874 & 2874.00 &  0.00\\
t20m10r5-1.pl.json & 20 & 10 & Optimal &  0.04 & 2586 & 2586.00 &  0.00\\
t20m10r5-10.pl.json & 20 & 10 & Optimal &  0.05 & 2260 & 2260.00 &  0.00\\
t20m10r5-11.pl.json & 20 & 10 & Optimal &  0.03 & 3487 & 3487.00 &  0.00\\
t20m10r5-12.pl.json & 20 & 10 & Optimal &  0.03 & 1559 & 1559.00 &  0.00\\
t20m10r5-13.pl.json & 20 & 10 & Optimal &  0.22 & 1457 & 1457.00 &  0.00\\
t20m10r5-14.pl.json & 20 & 10 & Optimal &  0.06 & 1141 & 1141.00 &  0.00\\
t20m10r5-15.pl.json & 20 & 10 & Optimal &  0.18 & 821 & 821.00 &  0.00\\
t20m10r5-16.pl.json & 20 & 10 & Optimal &  0.03 & 2910 & 2910.00 &  0.00\\
t20m10r5-17.pl.json & 20 & 10 & Optimal &  0.05 & 2337 & 2337.00 &  0.00\\
t20m10r5-18.pl.json & 20 & 10 & Optimal &  3.96 & 2920 & 2920.00 &  0.00\\
t20m10r5-19.pl.json & 20 & 10 & Optimal &  0.03 & 1952 & 1952.00 &  0.00\\
t20m10r5-2.pl.json & 20 & 10 & Optimal &  0.03 & 1639 & 1639.00 &  0.00\\
t20m10r5-20.pl.json & 20 & 10 & Optimal &  0.03 & 2660 & 2660.00 &  0.00\\
t20m10r5-3.pl.json & 20 & 10 & Optimal &  0.05 & 1406 & 1406.00 &  0.00\\
t20m10r5-4.pl.json & 20 & 10 & Optimal &  0.05 & 2658 & 2658.00 &  0.00\\
t20m10r5-5.pl.json & 20 & 10 & Optimal &  0.08 & 794 & 794.00 &  0.00\\
t20m10r5-6.pl.json & 20 & 10 & Optimal &  0.03 & 2398 & 2398.00 &  0.00\\
t20m10r5-7.pl.json & 20 & 10 & Optimal &  0.04 & 1430 & 1430.00 &  0.00\\
t20m10r5-8.pl.json & 20 & 10 & Optimal &  0.06 & 976 & 976.00 &  0.00\\
t20m10r5-9.pl.json & 20 & 10 & Optimal &  0.04 & 2953 & 2953.00 &  0.00\\
t30m10r10-1.pl.json & 30 & 10 & Optimal &  6.81 & 3344 & 3344.00 &  0.00\\
t30m10r10-10.pl.json & 30 & 10 & Solution & 30.03 & 4692 & 4146.00 & 11.64\\
t30m10r10-11.pl.json & 30 & 10 & Optimal &  0.06 & 2905 & 2905.00 &  0.00\\
t30m10r10-12.pl.json & 30 & 10 & Optimal &  0.06 & 3672 & 3672.00 &  0.00\\
t30m10r10-13.pl.json & 30 & 10 & Optimal &  0.36 & 2778 & 2778.00 &  0.00\\
t30m10r10-14.pl.json & 30 & 10 & Optimal &  2.31 & 2741 & 2741.00 &  0.00\\
t30m10r10-15.pl.json & 30 & 10 & Optimal &  0.05 & 2388 & 2388.00 &  0.00\\
t30m10r10-16.pl.json & 30 & 10 & Solution & 30.03 & 4225 & 3900.00 &  7.69\\
t30m10r10-17.pl.json & 30 & 10 & Optimal &  0.08 & 1504 & 1504.00 &  0.00\\
t30m10r10-18.pl.json & 30 & 10 & Solution & 30.03 & 3287 & 2730.00 & 16.95\\
t30m10r10-19.pl.json & 30 & 10 & Optimal &  0.05 & 3874 & 3874.00 &  0.00\\
t30m10r10-2.pl.json & 30 & 10 & Optimal &  0.03 & 3169 & 3169.00 &  0.00\\
t30m10r10-20.pl.json & 30 & 10 & Optimal &  0.05 & 2691 & 2691.00 &  0.00\\
t30m10r10-3.pl.json & 30 & 10 & Solution & 30.01 & 3360 & 2851.00 & 15.15\\
t30m10r10-4.pl.json & 30 & 10 & Optimal &  0.06 & 3452 & 3452.00 &  0.00\\
t30m10r10-5.pl.json & 30 & 10 & Optimal &  0.05 & 2785 & 2785.00 &  0.00\\
t30m10r10-6.pl.json & 30 & 10 & Solution & 30.03 & 1013 & 775.00 & 23.49\\
t30m10r10-7.pl.json & 30 & 10 & Optimal & 27.69 & 3755 & 3755.00 &  0.00\\
t30m10r10-8.pl.json & 30 & 10 & Solution & 30.02 & 4613 & 4160.00 &  9.82\\
t30m10r10-9.pl.json & 30 & 10 & Optimal &  0.03 & 2770 & 2770.00 &  0.00\\
t30m10r3-1.pl.json & 30 & 10 & Optimal &  0.05 & 2901 & 2901.00 &  0.00\\
t30m10r3-10.pl.json & 30 & 10 & Optimal &  0.04 & 4829 & 4829.00 &  0.00\\
t30m10r3-11.pl.json & 30 & 10 & Optimal &  0.04 & 2584 & 2584.00 &  0.00\\
t30m10r3-12.pl.json & 30 & 10 & Optimal &  0.03 & 2130 & 2130.00 &  0.00\\
t30m10r3-13.pl.json & 30 & 10 & Optimal &  0.03 & 4253 & 4253.00 &  0.00\\
t30m10r3-14.pl.json & 30 & 10 & Optimal &  0.17 & 1393 & 1393.00 &  0.00\\
t30m10r3-15.pl.json & 30 & 10 & Optimal &  0.03 & 4149 & 4149.00 &  0.00\\
t30m10r3-16.pl.json & 30 & 10 & Optimal &  0.05 & 2027 & 2027.00 &  0.00\\
t30m10r3-17.pl.json & 30 & 10 & Optimal &  0.05 & 2975 & 2975.00 &  0.00\\
t30m10r3-18.pl.json & 30 & 10 & Optimal &  0.05 & 5477 & 5477.00 &  0.00\\
t30m10r3-19.pl.json & 30 & 10 & Solution & 30.01 & 1289 & 1042.00 & 19.16\\
t30m10r3-2.pl.json & 30 & 10 & Optimal &  0.14 & 2523 & 2523.00 &  0.00\\
t30m10r3-20.pl.json & 30 & 10 & Optimal &  0.05 & 4754 & 4754.00 &  0.00\\
t30m10r3-3.pl.json & 30 & 10 & Optimal &  0.04 & 2793 & 2793.00 &  0.00\\
t30m10r3-4.pl.json & 30 & 10 & Optimal &  0.69 & 2809 & 2809.00 &  0.00\\
t30m10r3-5.pl.json & 30 & 10 & Optimal &  0.04 & 3758 & 3758.00 &  0.00\\
t30m10r3-6.pl.json & 30 & 10 & Optimal &  0.05 & 2870 & 2870.00 &  0.00\\
t30m10r3-7.pl.json & 30 & 10 & Optimal &  0.05 & 2122 & 2122.00 &  0.00\\
t30m10r3-8.pl.json & 30 & 10 & Optimal &  0.03 & 2862 & 2862.00 &  0.00\\
t30m10r3-9.pl.json & 30 & 10 & Optimal &  0.08 & 2754 & 2754.00 &  0.00\\
t30m10r5-1.pl.json & 30 & 10 & Optimal &  0.04 & 1998 & 1998.00 &  0.00\\
t30m10r5-10.pl.json & 30 & 10 & Optimal &  0.04 & 3743 & 3743.00 &  0.00\\
t30m10r5-11.pl.json & 30 & 10 & Optimal &  0.05 & 2138 & 2138.00 &  0.00\\
t30m10r5-12.pl.json & 30 & 10 & Optimal &  0.05 & 2251 & 2251.00 &  0.00\\
t30m10r5-13.pl.json & 30 & 10 & Optimal &  0.05 & 2632 & 2632.00 &  0.00\\
t30m10r5-14.pl.json & 30 & 10 & Optimal &  0.06 & 2201 & 2201.00 &  0.00\\
t30m10r5-15.pl.json & 30 & 10 & Optimal &  0.09 & 2339 & 2339.00 &  0.00\\
t30m10r5-16.pl.json & 30 & 10 & Optimal &  0.05 & 4293 & 4293.00 &  0.00\\
t30m10r5-17.pl.json & 30 & 10 & Optimal &  0.11 & 1314 & 1314.00 &  0.00\\
t30m10r5-18.pl.json & 30 & 10 & Optimal &  0.07 & 2169 & 2169.00 &  0.00\\
t30m10r5-19.pl.json & 30 & 10 & Solution & 30.01 & 1346 & 1279.00 &  4.98\\
t30m10r5-2.pl.json & 30 & 10 & Optimal &  0.05 & 2399 & 2399.00 &  0.00\\
t30m10r5-20.pl.json & 30 & 10 & Optimal &  0.05 & 1486 & 1486.00 &  0.00\\
t30m10r5-3.pl.json & 30 & 10 & Optimal &  0.05 & 2494 & 2494.00 &  0.00\\
t30m10r5-4.pl.json & 30 & 10 & Optimal &  0.03 & 3405 & 3405.00 &  0.00\\
t30m10r5-5.pl.json & 30 & 10 & Solution & 30.02 & 5243 & 4550.00 & 13.22\\
t30m10r5-6.pl.json & 30 & 10 & Optimal &  0.05 & 2382 & 2382.00 &  0.00\\
t30m10r5-7.pl.json & 30 & 10 & Optimal &  0.06 & 2018 & 2018.00 &  0.00\\
t30m10r5-8.pl.json & 30 & 10 & Optimal &  0.04 & 3089 & 3089.00 &  0.00\\
t30m10r5-9.pl.json & 30 & 10 & Optimal &  0.05 & 3704 & 3704.00 &  0.00\\
t30m20r10-1.pl.json & 30 & 20 & Solution & 30.03 & 3702 & 2850.00 & 23.01\\
t30m20r10-10.pl.json & 30 & 20 & Optimal &  4.79 & 2508 & 2508.00 &  0.00\\
t30m20r10-11.pl.json & 30 & 20 & Solution & 30.02 & 3648 & 3482.00 &  4.55\\
t30m20r10-12.pl.json & 30 & 20 & Optimal &  0.09 & 4214 & 4214.00 &  0.00\\
t30m20r10-13.pl.json & 30 & 20 & Optimal & 15.77 & 3980 & 3980.00 &  0.00\\
t30m20r10-14.pl.json & 30 & 20 & Optimal & 13.92 & 3141 & 3141.00 &  0.00\\
t30m20r10-15.pl.json & 30 & 20 & Solution & 30.02 & 4322 & 3457.00 & 20.01\\
t30m20r10-16.pl.json & 30 & 20 & Optimal &  0.11 & 4002 & 4002.00 &  0.00\\
t30m20r10-17.pl.json & 30 & 20 & Solution & 30.02 & 4161 & 3363.00 & 19.18\\
t30m20r10-18.pl.json & 30 & 20 & Optimal &  6.32 & 1992 & 1992.00 &  0.00\\
t30m20r10-19.pl.json & 30 & 20 & Solution & 30.04 & 2789 & 2250.00 & 19.33\\
t30m20r10-2.pl.json & 30 & 20 & Solution & 30.02 & 3982 & 3447.00 & 13.44\\
t30m20r10-20.pl.json & 30 & 20 & Optimal &  5.60 & 2314 & 2314.00 &  0.00\\
t30m20r10-3.pl.json & 30 & 20 & Optimal &  0.09 & 2158 & 2158.00 &  0.00\\
t30m20r10-4.pl.json & 30 & 20 & Solution & 30.03 & 4040 & 3217.00 & 20.37\\
t30m20r10-5.pl.json & 30 & 20 & Optimal &  0.09 & 1237 & 1237.00 &  0.00\\
t30m20r10-6.pl.json & 30 & 20 & Solution & 30.04 & 3770 & 3600.00 &  4.51\\
t30m20r10-7.pl.json & 30 & 20 & Optimal &  0.08 & 2266 & 2266.00 &  0.00\\
t30m20r10-8.pl.json & 30 & 20 & Optimal &  2.08 & 1855 & 1855.00 &  0.00\\
t30m20r10-9.pl.json & 30 & 20 & Optimal &  3.60 & 2028 & 2028.00 &  0.00\\
t30m20r3-1.pl.json & 30 & 20 & Optimal &  0.08 & 2200 & 2200.00 &  0.00\\
t30m20r3-10.pl.json & 30 & 20 & Optimal &  0.07 & 3291 & 3291.00 &  0.00\\
t30m20r3-11.pl.json & 30 & 20 & Optimal &  0.08 & 4473 & 4473.00 &  0.00\\
t30m20r3-12.pl.json & 30 & 20 & Solution & 30.02 & 5060 & 4931.00 &  2.55\\
t30m20r3-13.pl.json & 30 & 20 & Optimal &  0.07 & 3536 & 3536.00 &  0.00\\
t30m20r3-14.pl.json & 30 & 20 & Optimal &  0.08 & 3432 & 3432.00 &  0.00\\
t30m20r3-15.pl.json & 30 & 20 & Optimal &  0.08 & 3463 & 3463.00 &  0.00\\
t30m20r3-16.pl.json & 30 & 20 & Optimal &  0.07 & 3893 & 3893.00 &  0.00\\
t30m20r3-17.pl.json & 30 & 20 & Optimal &  0.07 & 1892 & 1892.00 &  0.00\\
t30m20r3-18.pl.json & 30 & 20 & Optimal &  0.08 & 2653 & 2653.00 &  0.00\\
t30m20r3-19.pl.json & 30 & 20 & Optimal &  0.08 & 3141 & 3141.00 &  0.00\\
t30m20r3-2.pl.json & 30 & 20 & Optimal &  0.08 & 1251 & 1251.00 &  0.00\\
t30m20r3-20.pl.json & 30 & 20 & Optimal &  5.77 & 2745 & 2745.00 &  0.00\\
t30m20r3-3.pl.json & 30 & 20 & Optimal &  0.08 & 3434 & 3434.00 &  0.00\\
t30m20r3-4.pl.json & 30 & 20 & Optimal &  0.10 & 2394 & 2394.00 &  0.00\\
t30m20r3-5.pl.json & 30 & 20 & Optimal &  0.06 & 3776 & 3776.00 &  0.00\\
t30m20r3-6.pl.json & 30 & 20 & Optimal &  0.08 & 2250 & 2250.00 &  0.00\\
t30m20r3-7.pl.json & 30 & 20 & Optimal &  0.12 & 1693 & 1693.00 &  0.00\\
t30m20r3-8.pl.json & 30 & 20 & Optimal &  0.08 & 4997 & 4997.00 &  0.00\\
t30m20r3-9.pl.json & 30 & 20 & Optimal &  0.08 & 4898 & 4898.00 &  0.00\\
t30m20r5-1.pl.json & 30 & 20 & Solution & 30.02 & 3195 & 2787.00 & 12.77\\
t30m20r5-10.pl.json & 30 & 20 & Optimal &  5.14 & 2133 & 2133.00 &  0.00\\
t30m20r5-11.pl.json & 30 & 20 & Optimal &  0.08 & 3974 & 3974.00 &  0.00\\
t30m20r5-12.pl.json & 30 & 20 & Optimal &  0.08 & 2197 & 2197.00 &  0.00\\
t30m20r5-13.pl.json & 30 & 20 & Optimal &  0.09 & 2296 & 2296.00 &  0.00\\
t30m20r5-14.pl.json & 30 & 20 & Optimal &  0.07 & 3861 & 3861.00 &  0.00\\
t30m20r5-15.pl.json & 30 & 20 & Optimal &  0.08 & 2353 & 2353.00 &  0.00\\
t30m20r5-16.pl.json & 30 & 20 & Optimal &  4.27 & 2751 & 2751.00 &  0.00\\
t30m20r5-17.pl.json & 30 & 20 & Optimal &  0.08 & 3555 & 3555.00 &  0.00\\
t30m20r5-18.pl.json & 30 & 20 & Optimal &  0.06 & 2384 & 2384.00 &  0.00\\
t30m20r5-19.pl.json & 30 & 20 & Optimal &  0.11 & 2080 & 2080.00 &  0.00\\
t30m20r5-2.pl.json & 30 & 20 & Optimal &  0.10 & 1715 & 1715.00 &  0.00\\
t30m20r5-20.pl.json & 30 & 20 & Optimal &  0.10 & 4176 & 4176.00 &  0.00\\
t30m20r5-3.pl.json & 30 & 20 & Solution & 30.05 & 4528 & 4037.00 & 10.84\\
t30m20r5-4.pl.json & 30 & 20 & Optimal &  0.09 & 3083 & 3083.00 &  0.00\\
t30m20r5-5.pl.json & 30 & 20 & Optimal &  0.08 & 1969 & 1969.00 &  0.00\\
t30m20r5-6.pl.json & 30 & 20 & Optimal &  0.08 & 4250 & 4250.00 &  0.00\\
t30m20r5-7.pl.json & 30 & 20 & Optimal &  0.08 & 3036 & 3036.00 &  0.00\\
t30m20r5-8.pl.json & 30 & 20 & Optimal &  1.55 & 2834 & 2834.00 &  0.00\\
t30m20r5-9.pl.json & 30 & 20 & Optimal &  0.10 & 2343 & 2343.00 &  0.00\\
t40m10r10-1.pl.json & 40 & 10 & Optimal &  0.11 & 2514 & 2514.00 &  0.00\\
t40m10r10-10.pl.json & 40 & 10 & Optimal &  0.08 & 3557 & 3557.00 &  0.00\\
t40m10r10-11.pl.json & 40 & 10 & Solution & 30.03 & 4556 & 4262.00 &  6.45\\
t40m10r10-12.pl.json & 40 & 10 & Solution & 30.01 & 5225 & 4355.00 & 16.65\\
t40m10r10-13.pl.json & 40 & 10 & Optimal & 16.47 & 2789 & 2789.00 &  0.00\\
t40m10r10-14.pl.json & 40 & 10 & Optimal &  0.47 & 1648 & 1648.00 &  0.00\\
t40m10r10-15.pl.json & 40 & 10 & Optimal &  2.03 & 1844 & 1844.00 &  0.00\\
t40m10r10-16.pl.json & 40 & 10 & Solution & 30.02 & 3749 & 3380.00 &  9.84\\
t40m10r10-17.pl.json & 40 & 10 & Optimal &  0.14 & 2363 & 2363.00 &  0.00\\
t40m10r10-18.pl.json & 40 & 10 & Optimal &  0.06 & 4973 & 4973.00 &  0.00\\
t40m10r10-19.pl.json & 40 & 10 & Optimal &  0.06 & 3181 & 3181.00 &  0.00\\
t40m10r10-2.pl.json & 40 & 10 & Optimal &  0.20 & 2350 & 2350.00 &  0.00\\
t40m10r10-20.pl.json & 40 & 10 & Solution & 30.04 & 2730 & 2470.00 &  9.52\\
t40m10r10-3.pl.json & 40 & 10 & Optimal &  0.06 & 3717 & 3717.00 &  0.00\\
t40m10r10-4.pl.json & 40 & 10 & Optimal &  0.08 & 3414 & 3414.00 &  0.00\\
t40m10r10-5.pl.json & 40 & 10 & Optimal &  5.68 & 2852 & 2852.00 &  0.00\\
t40m10r10-6.pl.json & 40 & 10 & Solution & 30.02 & 3262 & 2910.00 & 10.79\\
t40m10r10-7.pl.json & 40 & 10 & Optimal &  0.08 & 4572 & 4572.00 &  0.00\\
t40m10r10-8.pl.json & 40 & 10 & Solution & 30.03 & 3776 & 3385.00 & 10.35\\
t40m10r10-9.pl.json & 40 & 10 & Optimal &  0.11 & 2524 & 2524.00 &  0.00\\
t40m10r3-1.pl.json & 40 & 10 & Optimal &  0.09 & 4832 & 4832.00 &  0.00\\
t40m10r3-10.pl.json & 40 & 10 & Optimal &  0.15 & 2442 & 2442.00 &  0.00\\
t40m10r3-11.pl.json & 40 & 10 & Optimal &  0.06 & 3218 & 3218.00 &  0.00\\
t40m10r3-12.pl.json & 40 & 10 & Optimal &  0.06 & 3863 & 3863.00 &  0.00\\
t40m10r3-13.pl.json & 40 & 10 & Optimal &  0.07 & 3564 & 3564.00 &  0.00\\
t40m10r3-14.pl.json & 40 & 10 & Optimal &  0.08 & 4913 & 4913.00 &  0.00\\
t40m10r3-15.pl.json & 40 & 10 & Optimal &  0.26 & 3785 & 3785.00 &  0.00\\
t40m10r3-16.pl.json & 40 & 10 & Optimal &  0.11 & 2840 & 2840.00 &  0.00\\
t40m10r3-17.pl.json & 40 & 10 & Optimal &  0.06 & 5506 & 5506.00 &  0.00\\
t40m10r3-18.pl.json & 40 & 10 & Optimal &  0.08 & 3848 & 3848.00 &  0.00\\
t40m10r3-19.pl.json & 40 & 10 & Optimal &  0.11 & 2259 & 2259.00 &  0.00\\
t40m10r3-2.pl.json & 40 & 10 & Solution & 30.04 & 1727 & 1589.00 &  7.99\\
t40m10r3-20.pl.json & 40 & 10 & Optimal &  0.09 & 4157 & 4157.00 &  0.00\\
t40m10r3-3.pl.json & 40 & 10 & Optimal &  0.08 & 4903 & 4903.00 &  0.00\\
t40m10r3-4.pl.json & 40 & 10 & Solution & 30.03 & 1635 & 1341.00 & 17.98\\
t40m10r3-5.pl.json & 40 & 10 & Optimal &  0.16 & 1984 & 1984.00 &  0.00\\
t40m10r3-6.pl.json & 40 & 10 & Optimal &  0.06 & 5005 & 5005.00 &  0.00\\
t40m10r3-7.pl.json & 40 & 10 & Solution & 30.03 & 5545 & 5188.00 &  6.44\\
t40m10r3-8.pl.json & 40 & 10 & Optimal &  0.08 & 3658 & 3658.00 &  0.00\\
t40m10r3-9.pl.json & 40 & 10 & Optimal &  0.19 & 3830 & 3830.00 &  0.00\\
t40m10r5-1.pl.json & 40 & 10 & Optimal &  0.08 & 4857 & 4857.00 &  0.00\\
t40m10r5-10.pl.json & 40 & 10 & Optimal &  0.08 & 3989 & 3989.00 &  0.00\\
t40m10r5-11.pl.json & 40 & 10 & Optimal &  0.08 & 5238 & 5238.00 &  0.00\\
t40m10r5-12.pl.json & 40 & 10 & Optimal &  0.08 & 4584 & 4584.00 &  0.00\\
t40m10r5-13.pl.json & 40 & 10 & Optimal &  0.09 & 2307 & 2307.00 &  0.00\\
t40m10r5-14.pl.json & 40 & 10 & Optimal &  0.30 & 1826 & 1826.00 &  0.00\\
t40m10r5-15.pl.json & 40 & 10 & Optimal &  0.11 & 1926 & 1926.00 &  0.00\\
t40m10r5-16.pl.json & 40 & 10 & Optimal &  0.11 & 5216 & 5216.00 &  0.00\\
t40m10r5-17.pl.json & 40 & 10 & Optimal &  0.08 & 7162 & 7162.00 &  0.00\\
t40m10r5-18.pl.json & 40 & 10 & Optimal &  0.11 & 4892 & 4892.00 &  0.00\\
t40m10r5-19.pl.json & 40 & 10 & Optimal &  0.08 & 4027 & 4027.00 &  0.00\\
t40m10r5-2.pl.json & 40 & 10 & Optimal &  8.38 & 4099 & 4099.00 &  0.00\\
t40m10r5-20.pl.json & 40 & 10 & Solution & 30.02 & 4899 & 4755.00 &  2.94\\
t40m10r5-3.pl.json & 40 & 10 & Optimal &  0.08 & 3113 & 3113.00 &  0.00\\
t40m10r5-4.pl.json & 40 & 10 & Optimal &  0.10 & 6626 & 6626.00 &  0.00\\
t40m10r5-5.pl.json & 40 & 10 & Optimal &  0.08 & 3828 & 3828.00 &  0.00\\
t40m10r5-6.pl.json & 40 & 10 & Optimal &  0.09 & 4213 & 4213.00 &  0.00\\
t40m10r5-7.pl.json & 40 & 10 & Optimal &  0.28 & 4303 & 4303.00 &  0.00\\
t40m10r5-8.pl.json & 40 & 10 & Solution & 30.03 & 3559 & 3189.00 & 10.40\\
t40m10r5-9.pl.json & 40 & 10 & Optimal &  0.41 & 1953 & 1953.00 &  0.00\\
t40m20r10-1.pl.json & 40 & 20 & Solution & 30.09 & 4518 & 3972.00 & 12.08\\
t40m20r10-10.pl.json & 40 & 20 & Optimal & 12.43 & 3862 & 3862.00 &  0.00\\
t40m20r10-11.pl.json & 40 & 20 & Optimal &  0.14 & 1952 & 1952.00 &  0.00\\
t40m20r10-12.pl.json & 40 & 20 & Optimal &  0.14 & 4129 & 4129.00 &  0.00\\
t40m20r10-13.pl.json & 40 & 20 & Optimal &  0.28 & 2927 & 2927.00 &  0.00\\
t40m20r10-14.pl.json & 40 & 20 & Solution & 30.05 & 2701 & 2381.00 & 11.85\\
t40m20r10-15.pl.json & 40 & 20 & Optimal & 11.77 & 3168 & 3168.00 &  0.00\\
t40m20r10-16.pl.json & 40 & 20 & Optimal &  0.14 & 2812 & 2812.00 &  0.00\\
t40m20r10-17.pl.json & 40 & 20 & Solution & 30.07 & 4288 & 3718.00 & 13.29\\
t40m20r10-18.pl.json & 40 & 20 & Solution & 30.05 & 3611 & 3194.00 & 11.55\\
t40m20r10-19.pl.json & 40 & 20 & Optimal & 12.23 & 2891 & 2891.00 &  0.00\\
t40m20r10-2.pl.json & 40 & 20 & Optimal &  8.74 & 3284 & 3284.00 &  0.00\\
t40m20r10-20.pl.json & 40 & 20 & Solution & 30.04 & 5506 & 4945.00 & 10.19\\
t40m20r10-3.pl.json & 40 & 20 & Solution & 30.08 & 5981 & 5478.00 &  8.41\\
t40m20r10-4.pl.json & 40 & 20 & Optimal &  0.14 & 3409 & 3409.00 &  0.00\\
t40m20r10-5.pl.json & 40 & 20 & Solution & 30.06 & 5113 & 4278.00 & 16.33\\
t40m20r10-6.pl.json & 40 & 20 & Solution & 30.03 & 2376 & 2333.00 &  1.81\\
t40m20r10-7.pl.json & 40 & 20 & Solution & 30.06 & 4799 & 4243.00 & 11.59\\
t40m20r10-8.pl.json & 40 & 20 & Solution & 30.02 & 3924 & 3327.00 & 15.21\\
t40m20r10-9.pl.json & 40 & 20 & Optimal &  3.86 & 2043 & 2043.00 &  0.00\\
t40m20r3-1.pl.json & 40 & 20 & Optimal &  0.16 & 3524 & 3524.00 &  0.00\\
t40m20r3-10.pl.json & 40 & 20 & Optimal &  0.19 & 3110 & 3110.00 &  0.00\\
t40m20r3-11.pl.json & 40 & 20 & Optimal &  0.15 & 3695 & 3695.00 &  0.00\\
t40m20r3-12.pl.json & 40 & 20 & Optimal &  0.24 & 4828 & 4828.00 &  0.00\\
t40m20r3-13.pl.json & 40 & 20 & Optimal &  0.25 & 4010 & 4010.00 &  0.00\\
t40m20r3-14.pl.json & 40 & 20 & Optimal &  0.14 & 2752 & 2752.00 &  0.00\\
t40m20r3-15.pl.json & 40 & 20 & Optimal &  0.16 & 3312 & 3312.00 &  0.00\\
t40m20r3-16.pl.json & 40 & 20 & Optimal &  0.16 & 4304 & 4304.00 &  0.00\\
t40m20r3-17.pl.json & 40 & 20 & Optimal &  0.17 & 3991 & 3991.00 &  0.00\\
t40m20r3-18.pl.json & 40 & 20 & Optimal &  0.17 & 5733 & 5733.00 &  0.00\\
t40m20r3-19.pl.json & 40 & 20 & Optimal &  0.17 & 3581 & 3581.00 &  0.00\\
t40m20r3-2.pl.json & 40 & 20 & Optimal &  0.17 & 4869 & 4869.00 &  0.00\\
t40m20r3-20.pl.json & 40 & 20 & Optimal &  0.17 & 3514 & 3514.00 &  0.00\\
t40m20r3-3.pl.json & 40 & 20 & Optimal &  0.24 & 2503 & 2503.00 &  0.00\\
t40m20r3-4.pl.json & 40 & 20 & Optimal &  0.13 & 4323 & 4323.00 &  0.00\\
t40m20r3-5.pl.json & 40 & 20 & Optimal &  0.17 & 3626 & 3626.00 &  0.00\\
t40m20r3-6.pl.json & 40 & 20 & Optimal &  0.17 & 2488 & 2488.00 &  0.00\\
t40m20r3-7.pl.json & 40 & 20 & Optimal &  0.17 & 3470 & 3470.00 &  0.00\\
t40m20r3-8.pl.json & 40 & 20 & Optimal &  0.24 & 6730 & 6730.00 &  0.00\\
t40m20r3-9.pl.json & 40 & 20 & Optimal &  0.20 & 4656 & 4656.00 &  0.00\\
t40m20r5-1.pl.json & 40 & 20 & Optimal &  0.28 & 1318 & 1318.00 &  0.00\\
t40m20r5-10.pl.json & 40 & 20 & Optimal &  0.25 & 2216 & 2216.00 &  0.00\\
t40m20r5-11.pl.json & 40 & 20 & Optimal &  0.25 & 3538 & 3538.00 &  0.00\\
t40m20r5-12.pl.json & 40 & 20 & Optimal &  0.23 & 5346 & 5346.00 &  0.00\\
t40m20r5-13.pl.json & 40 & 20 & Solution & 30.03 & 4589 & 4393.00 &  4.27\\
t40m20r5-14.pl.json & 40 & 20 & Optimal &  0.17 & 2243 & 2243.00 &  0.00\\
t40m20r5-15.pl.json & 40 & 20 & Solution & 30.08 & 3869 & 3590.00 &  7.21\\
t40m20r5-16.pl.json & 40 & 20 & Optimal &  0.17 & 4319 & 4319.00 &  0.00\\
t40m20r5-17.pl.json & 40 & 20 & Optimal &  0.18 & 4866 & 4866.00 &  0.00\\
t40m20r5-18.pl.json & 40 & 20 & Optimal &  0.39 & 5802 & 5802.00 &  0.00\\
t40m20r5-19.pl.json & 40 & 20 & Solution & 30.06 & 4197 & 4072.00 &  2.98\\
t40m20r5-2.pl.json & 40 & 20 & Optimal &  0.16 & 2634 & 2634.00 &  0.00\\
t40m20r5-20.pl.json & 40 & 20 & Solution & 30.03 & 6482 & 6232.00 &  3.86\\
t40m20r5-3.pl.json & 40 & 20 & Optimal &  0.19 & 4391 & 4391.00 &  0.00\\
t40m20r5-4.pl.json & 40 & 20 & Optimal &  9.64 & 4610 & 4610.00 &  0.00\\
t40m20r5-5.pl.json & 40 & 20 & Optimal &  0.17 & 3105 & 3105.00 &  0.00\\
t40m20r5-6.pl.json & 40 & 20 & Optimal &  0.16 & 4760 & 4760.00 &  0.00\\
t40m20r5-7.pl.json & 40 & 20 & Optimal &  0.31 & 1218 & 1218.00 &  0.00\\
t40m20r5-8.pl.json & 40 & 20 & Solution & 30.05 & 2601 & 2190.00 & 15.80\\
t40m20r5-9.pl.json & 40 & 20 & Optimal &  0.19 & 3141 & 3141.00 &  0.00\\
t500m100r10-10.pl.json & 500 & 100 & Solution & 30.54 & 43793 & 795.00 & 98.18\\
t500m100r10-11.pl.json & 500 & 100 & Solution & 30.92 & 36367 & 801.00 & 97.80\\
t500m100r10-13.pl.json & 500 & 100 & Solution & 30.63 & 45030 & 801.00 & 98.22\\
t500m100r10-14.pl.json & 500 & 100 & Solution & 30.54 & 40089 & 800.00 & 98.00\\
t500m100r10-15.pl.json & 500 & 100 & Solution & 30.45 & 41425 & 801.00 & 98.07\\
t500m100r10-16.pl.json & 500 & 100 & Solution & 30.65 & 40463 & 801.00 & 98.02\\
t500m100r10-17.pl.json & 500 & 100 & Solution & 30.43 & 33209 & 798.00 & 97.60\\
t500m100r10-18.pl.json & 500 & 100 & Solution & 30.44 & 41028 & 801.00 & 98.05\\
t500m100r10-19.pl.json & 500 & 100 & Solution & 30.94 & 49137 & 801.00 & 98.37\\
t500m100r10-2.pl.json & 500 & 100 & Solution & 30.54 & 42142 & 796.00 & 98.11\\
t500m100r10-20.pl.json & 500 & 100 & Solution & 30.35 & 38167 & 801.00 & 97.90\\
t500m100r10-3.pl.json & 500 & 100 & Solution & 30.39 & 37653 & 801.00 & 97.87\\
t500m100r10-4.pl.json & 500 & 100 & Solution & 30.67 & 39921 & 798.00 & 98.00\\
t500m100r10-5.pl.json & 500 & 100 & Solution & 30.47 & 35252 & 800.00 & 97.73\\
t500m100r10-6.pl.json & 500 & 100 & Solution & 30.65 & 41172 & 801.00 & 98.05\\
t500m100r10-7.pl.json & 500 & 100 & Solution & 30.97 & 41044 & 800.00 & 98.05\\
t500m100r10-8.pl.json & 500 & 100 & Solution & 30.52 & 46351 & 800.00 & 98.27\\
t500m100r10-9.pl.json & 500 & 100 & Solution & 30.51 & 40539 & 800.00 & 98.03\\
t500m100r3-1.pl.json & 500 & 100 & Solution & 30.55 & 39303 & 801.00 & 97.96\\
t500m100r3-10.pl.json & 500 & 100 & Solution & 30.65 & 42052 & 801.00 & 98.10\\
t500m100r3-11.pl.json & 500 & 100 & Solution & 30.62 & 38084 & 794.00 & 97.92\\
t500m100r3-12.pl.json & 500 & 100 & Solution & 30.70 & 38483 & 800.00 & 97.92\\
t500m100r3-13.pl.json & 500 & 100 & Solution & 30.57 & 35447 & 801.00 & 97.74\\
t500m100r3-14.pl.json & 500 & 100 & Solution & 30.42 & 40571 & 798.00 & 98.03\\
t500m100r3-15.pl.json & 500 & 100 & Solution & 30.45 & 38987 & 801.00 & 97.95\\
t500m100r3-16.pl.json & 500 & 100 & Solution & 30.59 & 41984 & 798.00 & 98.10\\
t500m100r3-18.pl.json & 500 & 100 & Solution & 30.89 & 39919 & 801.00 & 97.99\\
t500m100r3-19.pl.json & 500 & 100 & Optimal & 10.63 & 41896 & 41892.00 &  0.01\\
t500m100r3-2.pl.json & 500 & 100 & Optimal & 10.86 & 41211 & 41207.00 &  0.01\\
t500m100r3-20.pl.json & 500 & 100 & Solution & 30.78 & 38551 & 800.00 & 97.92\\
t500m100r3-3.pl.json & 500 & 100 & Solution & 30.79 & 35516 & 798.00 & 97.75\\
t500m100r3-4.pl.json & 500 & 100 & Solution & 30.36 & 32084 & 798.00 & 97.51\\
t500m100r3-5.pl.json & 500 & 100 & Solution & 30.66 & 38761 & 801.00 & 97.93\\
t500m100r3-6.pl.json & 500 & 100 & Solution & 30.52 & 46048 & 800.00 & 98.26\\
t500m100r3-7.pl.json & 500 & 100 & Solution & 30.45 & 37680 & 800.00 & 97.88\\
t500m100r3-8.pl.json & 500 & 100 & Solution & 30.69 & 40838 & 799.00 & 98.04\\
t500m100r3-9.pl.json & 500 & 100 & Solution & 30.85 & 44803 & 801.00 & 98.21\\
t500m100r5-1.pl.json & 500 & 100 & Solution & 30.49 & 36936 & 797.00 & 97.84\\
t500m100r5-10.pl.json & 500 & 100 & Solution & 31.15 & 30332 & 800.00 & 97.36\\
t500m100r5-11.pl.json & 500 & 100 & Solution & 30.80 & 37660 & 801.00 & 97.87\\
t500m100r5-12.pl.json & 500 & 100 & Solution & 30.42 & 39090 & 799.00 & 97.96\\
t500m100r5-13.pl.json & 500 & 100 & Solution & 30.39 & 44171 & 801.00 & 98.19\\
t500m100r5-14.pl.json & 500 & 100 & Solution & 30.45 & 39568 & 800.00 & 97.98\\
t500m100r5-15.pl.json & 500 & 100 & Solution & 30.57 & 38257 & 800.00 & 97.91\\
t500m100r5-16.pl.json & 500 & 100 & Solution & 30.61 & 35151 & 798.00 & 97.73\\
t500m100r5-17.pl.json & 500 & 100 & Solution & 30.72 & 39749 & 797.00 & 97.99\\
t500m100r5-18.pl.json & 500 & 100 & Solution & 30.54 & 45868 & 801.00 & 98.25\\
t500m100r5-19.pl.json & 500 & 100 & Solution & 30.40 & 46018 & 801.00 & 98.26\\
t500m100r5-2.pl.json & 500 & 100 & Solution & 30.58 & 43708 & 800.00 & 98.17\\
t500m100r5-20.pl.json & 500 & 100 & Solution & 30.82 & 39466 & 800.00 & 97.97\\
t500m100r5-3.pl.json & 500 & 100 & Solution & 30.77 & 42468 & 801.00 & 98.11\\
t500m100r5-4.pl.json & 500 & 100 & Solution & 30.57 & 33936 & 801.00 & 97.64\\
t500m100r5-5.pl.json & 500 & 100 & Solution & 30.69 & 38103 & 795.00 & 97.91\\
t500m100r5-6.pl.json & 500 & 100 & Solution & 30.62 & 45271 & 801.00 & 98.23\\
t500m100r5-7.pl.json & 500 & 100 & Solution & 30.68 & 43542 & 800.00 & 98.16\\
t500m100r5-8.pl.json & 500 & 100 & Solution & 30.74 & 38116 & 796.00 & 97.91\\
t500m100r5-9.pl.json & 500 & 100 & Solution & 30.43 & 39282 & 801.00 & 97.96\\
t500m10r10-1.pl.json & 500 & 10 & Solution & 30.06 & 48213 & 42756.00 & 11.32\\
t500m10r10-10.pl.json & 500 & 10 & Solution & 30.08 & 35490 & 30745.00 & 13.37\\
t500m10r10-11.pl.json & 500 & 10 & Solution & 30.10 & 47651 & 42832.00 & 10.11\\
t500m10r10-12.pl.json & 500 & 10 & Solution & 30.09 & 43253 & 35908.00 & 16.98\\
\end{longtable}



\clearpage
\chapter{J\&J Hybrid Flexible Flowshop with Transportation Times}

\section{Without Transportation Times}

\end{document}